In this appendix, we provide proofs for the remaining results in the paper. 
We start by giving a proof for the transition 
from multilevel Wasserstein means objective function \eqref{eqn:multilevel_Kmeans_typeone} to objective function 
\eqref{eqn:multilevel_K_means_typeone_first} in Section \ref{Section:multilevel_kmeans} in the paper. 
All the notations in this appendix are similar to those in the main text. 
For each closed subset $\mathcal{S} \subset \mathcal{P}_{2}(\Theta)$, 
denote the Voronoi region generated by $\mathcal{S}$ on the space 
$\mathcal{P}_{2}(\Theta)$ by the collection of subsets
$\{ V_P \}_{P \in \mathcal{S}}$, 
where $V_P := \{Q \in \mathcal{P}_{2}(\Theta) : W_{2}^{2}(Q,P) = 
\mathop {\min }\limits_{G \in \mathcal{S}} W_{2}^{2}(Q,G)\}$. We define the projection mapping $\pi _\mathcal{S}$ as: $\pi _{\mathcal{S}} :\mathcal{P}_{2}(\Theta)  \to \mathcal{S}$ 
where $\pi _{\mathcal{S}} (Q) = P$ as $Q \in V_{P}$. Note that, for any $P_{1}, P_{2} \in \mathcal{S}$ such that $V_{P_{1}}$ and $V_{P_{2}}$ share the boundary, the values of $\pi_{S}$ at the elements in that boundary can be chosen to be either $P_{1}$ or $P_{2}$. Now, we start with the following useful lemmas.

%%%%%%%%%%%%%%%%%%%%%%%%%%%%%%%
\setcounter{lemma}{1}
\begin{customlem}{B.1} \label{lemma:one} For any closed subset $\mathcal{S}$ on $\mathcal{P}_{2}(\Theta)$, if $\mathcal{Q} \in \mathcal{P}_{2}(\mathcal{P}_{2}(\Theta))$, then $E_{X \sim \mathcal{Q}} (d_{W_{2}}^{2}(X,\mathcal{S})) = W_2^2 (\mathcal{Q},\pi _\mathcal{S} \# \mathcal{Q})$ where $d_{W_{2}}^{2}(X,\mathcal{S})=\inf \limits_{P \in \mathcal{S}}{W_{2}^{2}(X,P)}$.
\end{customlem}
\begin{proof}
For any element $\pi \in \Pi (\mathcal{Q},\pi _\mathcal{S} \# \mathcal{Q})$:
\begin{eqnarray}
\int{W_{2}^{2}(P,G)} d\pi (P,G) & \geq &  \int {d_{W_{2}}^{2}(P,\mathcal{S})} d\pi (P,G) \nonumber \\
& = & \int {d_{W_{2}}^{2}(P,\mathcal{S})} d\mathcal{Q}(P) \nonumber \\
& = &  E_{X \sim \mathcal{Q}} (d_{W_{2}}^{2}(X,\mathcal{S})) \nonumber
\end{eqnarray}
where the integrations in the first two terms range over $\mathcal{P}_{2}(\Theta)  \times \mathcal{S}$ while that in the final term ranges over $\mathcal{P}_{2}(\Theta)$. Therefore, we obtain 
\begin{eqnarray}
W_2^2 (\mathcal{Q},\pi _\mathcal{S} \# \mathcal{Q}) & = & \mathop {\inf } \int\limits_{\mathcal{P}_{2}(\Theta)  \times \mathcal{S}} {W_{2}^{2}(P,G)} d\pi (P,G) \nonumber \\
& \geq & E_{X \sim \mathcal{Q}} (d_{W_{2}}^{2}(X,\mathcal{S})) \label{eqn:lemmaequationone}
\end{eqnarray}
where the infimum in the first equality ranges over all $\pi \in \Pi (\mathcal{Q},\pi _\mathcal{S} \# \mathcal{Q})$. 

On the other hand, let ${\displaystyle g:\mathcal{P}_{2}(\Theta)  \to \mathcal{P}_{2}(\Theta) \times \mathcal{S}}$ 
such that $g(P)=(P,\pi_{\mathcal{S}}(P))$ for all $P \in \mathcal{P}_{2}(\Theta)$. Additionally, let 
$\mu _{\pi _\mathcal{S} } = g \# \mathcal{Q}$, the push-forward measure of $\mathcal{Q}$ under mapping $g$. It is clear that $\mu _{\pi _\mathcal{S}}$ is a coupling between $\mathcal{Q}$ and $\pi _\mathcal{S} \# \mathcal{Q}$. 
Under this construction, we obtain for any $X \sim \mathcal{Q}$ that
\begin{eqnarray}
 E\left(W_{2}^{2}(X,\pi _\mathcal{S} (X))\right) & = & \int {W_{2}^{2}(P,G) } d\mu _{\pi _\mathcal{S}} (P,G) \nonumber \\
 & \geq & \mathop {\inf } \int {W_{2}^{2}(P,G)} d\pi (P,G) \nonumber \\
& = & W_2^2 (\mathcal{Q},\pi _\mathcal{S} \# \mathcal{Q}) \label{eqn:lemmaequationtwo}
\end{eqnarray}
where the infimum in the second inequality ranges over all $\pi  \in \Pi (\mathcal{Q},\pi _\mathcal{S} \# \mathcal{Q})$ and the integrations range over $\mathcal{P}_{2}(\Theta) \times \mathcal{S}$. Now, from the definition of $\pi_{\mathcal{S}}$
\begin{eqnarray}
E(W_{2}^{2}(X,\pi _\mathcal{S} (X))) & = & \int {W_{2}^{2}(P,\pi _\mathcal{S}(P))}d\mathcal{Q}(P) \nonumber \\
& = & \int {d_{W_{2}}^{2}(P,\mathcal{S})} d\mathcal{Q}(P) \nonumber \\
& = & E(d_{W_{2}}^{2}(X,\mathcal{S})) \label{eqn:lemmaequationthree}
\end{eqnarray}
where the integrations in the above equations range over $\mathcal{P}_{2}(\Theta)$. By combining \eqref{eqn:lemmaequationtwo} and \eqref{eqn:lemmaequationthree}, we would obtain that
\begin{eqnarray}
E_{X \sim \mathcal{Q}} (d^{2}_{W_{2}}(X,\mathcal{S})) \ge W_2^2 (\mathcal{Q},\pi _\mathcal{S} \# \mathcal{Q}). \label{eqn:lemmaequationfourth}
\end{eqnarray}
From \eqref{eqn:lemmaequationone} and \eqref{eqn:lemmaequationfourth}, it is straightforward that $E_{X \sim Q} (d(X,S)^2 ) = W_2^2 (Q,\pi _S  \# Q)$. Therefore, we achieve the conclusion of the lemma.
\end{proof}

%%%%%%%%%%%%%%%%%%%%%%%%%%%%
\begin{customlem}{B.2} \label{lemma:two}
For any closed subset $\mathcal{S} \subset \mathcal{P}_{2}(\Theta)$ and $\mu \in \mathcal{P}_{2}(\mathcal{P}_{2}(\Theta))$ 
with $\text{supp}(\mu) \subseteq \mathcal{S}$, 
there holds 
$W_2^2 (\mathcal{Q},\mu ) \ge W_2^2 (\mathcal{Q},\pi _\mathcal{S} \# \mathcal{Q})$ for any $\mathcal{Q} \in \mathcal{P}_{2}(\mathcal{P}_{2}(\Theta))$.
\end{customlem}
\begin{proof}
Since $\text{supp}(\mu) \subseteq \mathcal{S}$, it is clear that $W_2^2 (\mathcal{Q},\mu) = {\displaystyle \mathop {\inf }\limits_{\pi  \in \Pi (\mathcal{Q},\mu )} \int\limits_{\mathcal{P}_{2}(\Theta) \times \mathcal{S}} {W_{2}^{2}(P,G)} d\pi (P,G)}$.\\
Additionally, we have
\begin{eqnarray}
\int {W_{2}^{2}(P,G)} d\pi (P,G) & \geq & \int {d_{W_{2}}^{2}(P,\mathcal{S})} d\pi (P,G) \nonumber \\
& = &  \int {d_{W_{2}}^{2}(P,\mathcal{S})} d\mathcal{Q}(P) \nonumber \\
& = & E_{X \sim Q} (d_{W_{2}}^{2}(X,S)) \nonumber \\
& = & W_2^2 (\mathcal{Q},\pi _\mathcal{S} \# \mathcal{Q}) \nonumber
\end{eqnarray}
where the last inequality is due to Lemma \ref{lemma:one} and the integrations in the first two terms range over $\mathcal{P}_{2}(\Theta) \times \mathcal{S}$ while that in the final term ranges over $\mathcal{P}_{2}(\Theta)$. Therefore, we achieve the conclusion of the lemma.
\end{proof}
Equipped with Lemma \ref{lemma:one} and Lemma \ref{lemma:two}, 
we are ready to establish 
the equivalence between multilevel Wasserstein means objective function (5) and objective function (4) in Section \ref{Section:multilevel_kmeans} in the main text.
\begin{customlem}{B.3} \label{proposition:Wassersteinequivalence}
For any given positive integers $m$ and $M$, we have
\begin{eqnarray}
A : = \inf \limits_{\Hcal \in \mathcal{E}_{M}(\mathcal{P}_{2}(\Theta))} W_{2}^{2}(\Hcal,\dfrac{1}{m}\mathop {\sum }\limits_{j=1}^{m}{\delta_{G_{j}}}) \nonumber \\
= \dfrac{1}{m}\inf \limits_{\Hbold = (H_{1},\ldots,H_{M})}\sum \limits_{j=1}^{m} d_{W_{2}}^{2}(G_{j},\Hbold) := B. \nonumber
\end{eqnarray}
\end{customlem}
\begin{proof}
Write $\mathcal{Q}=\dfrac{1}{m}\mathop {\sum }\limits_{j=1}^{m}{\delta_{G_{j}}}$. From the definition of $B$, for any $\epsilon>0$, we can find $\overline{\Hbold}$ such that 
\begin{eqnarray}
B & \geq & \dfrac{1}{m}\sum \limits_{j=1}^{m} d_{W_{2}}^{2}(G_{j},\overline{\Hbold}) - \epsilon \nonumber \\
& = & E_{X \sim \mathcal{Q}}(d_{W_{2}}^{2}(X,\overline{\Hbold})) - \epsilon \nonumber \\
& = & W_{2}^{2}(\mathcal{Q},\pi_{\overline{\Hbold}} \# \mathcal{Q}) {\bf - \epsilon} \nonumber \\
& \geq & A -\epsilon \nonumber
\end{eqnarray}
where the second equality in the above display
is due to Lemma \ref{lemma:one} while the last 
inequality is from the fact that $\pi_{\overline{\Hbold}} \# \mathcal{Q}$ is a discrete probability measure in $\mathcal{P}_{2}(\mathcal{P}_{2}(\Theta))$ with exactly $M$ support points. Since the inequality in the above display holds for any $\epsilon$, it implies that $B \geq A$. On the other hand, from the formation of $A$, for any $\epsilon>0$, we also can find $\Hcal' \in \mathcal{E}_{M}(\mathcal{P}_{2}(\Theta))$ such that
\begin{eqnarray}
A & \geq & W_{2}^{2}(\Hcal',\mathcal{Q}) - \epsilon \nonumber \\
& \geq & W_{2}^{2}(\mathcal{Q},\pi_{\Hbold'} \# \mathcal{Q}) - \epsilon \nonumber \\
& = & \dfrac{1}{m}\sum \limits_{j=1}^{m} d_{W_{2}}^{2}(G_{j},\Hbold') -\epsilon \nonumber \\
& \geq & B - \epsilon \nonumber
\end{eqnarray}
where  $\Hbold' = \text{supp}(\Hcal')$, the second inequality is due to Lemma \ref{lemma:two}, and the third equality is due to Lemma \ref{lemma:one}. Therefore, it means that $A \geq B$. We achieve the conclusion of the lemma. 
\end{proof}
\begin{customprop}{B.4} \label{lemma:equivalence_multilevels_Kmeans}
For any positive integer numbers $m, M$ and $k_{j}$ as $1 \leq j \leq m$, we denote 
\begin{eqnarray}
C & : = & \mathop {\inf }\limits_{\substack {G_{j} \in \mathcal{O}_{k_{j}}(\Theta) \ \forall 1 \leq j \leq m, \\ \Hcal \in \mathcal{E}_{M}(\mathcal{P}_{2}(\Theta))}}{\mathop {\sum }\limits_{i=1}^{m}{W_{2}^{2}(G_{j},P_{n_{j}}^{j})}} \nonumber \\
& + & W_{2}^{2}(\Hcal,\dfrac{1}{m}\mathop {\sum }\limits_{i=1}^{m}{\delta_{G_{i}}}) \nonumber \\
D & : = & \mathop {\inf }\limits_{\substack {G_{j} \in \mathcal{O}_{k_{j}}(\Theta) \ \forall 1 \leq j \leq m, \\ \Hbold = (H_{1},\ldots,H_{M})}}{\mathop {\sum }\limits_{j=1}^{m}{W_{2}^{2}(G_{j},P_{n_{j}}^{j})}} \nonumber \\
& + & \dfrac{d_{W_{2}}^{2}(G_{j},\Hbold)}{m}. \nonumber
\end{eqnarray}
Then, we have $C = D$.
\end{customprop}
\begin{proof} The proof of this proposition is a straightforward application of Lemma \ref{proposition:Wassersteinequivalence}. Indeed, for each fixed $(G_{1},\ldots,G_{m})$ the infimum w.r.t to $\mathcal{H}$ in $C$ leads to the same infimum w.r.t to $\Hbold$ in $D$, according to Lemma \ref{proposition:Wassersteinequivalence}. Now, by taking the infimum w.r.t to $(G_{1},\ldots,G_{m})$ on both sides, we achieve the conclusion of the proposition.
\end{proof}

In the remainder of the Supplement, we present the proofs for all remaining
theorems stated in the main text.
%%%%%%%%%%%%%%%%%%%%%%%%%%%%%%%%%%%%%%%%%%%%%%%%%
%%%%%%%%%%%%%%%%%
\paragraph{PROOF OF THEOREM \ref{theorem:local_convergence_multilevel_Kmeans}} %\ref{theorem:local_convergence_multilevel_Kmeans}}
%%%%%%%%%%%%%%%%%
%%%%%%%%%%%%%%%%%%%%%%%%%%%%%%%%%%%%%%%%%%%%%%%%%
The proof of this theorem is straightforward from the formulation of Algorithm \ref{alg:multilevels_Wasserstein_means}. %\ref{alg:multilevels_Wasserstein_means}. 
In fact, for any $G_{j} \in \mathcal{E}_{k_{j}}(\Theta)$ and $\Hbold =(H_{1},\ldots,H_{M})$, we denote the function 
\begin{eqnarray}
f(\vec{G}, \Hbold)=\mathop {\sum }\limits_{j=1}^{m}{W_{2}^{2}(G_{j},P_{n}^{j})}+\dfrac{d_{W_{2}}^{2}(G_{j},\Hbold)}{m} \nonumber
\end{eqnarray}
where $\vec{G}=(G_{1},\ldots,G_{m})$. To obtain the conclusion of this theorem, it is sufficient to demonstrate for any $t \geq 0$ that
\begin{eqnarray}
f(\vec{G}^{(t+1)},\Hbold^{(t+1)}) \leq f(\vec{G}^{(t)},\Hbold^{(t)}). \nonumber
\end{eqnarray}
This inequality comes directly from $f(\vec{G}^{(t+1)},\Hbold^{(t)}) \leq f(\vec{G}^{(t)},\Hbold^{(t)})$, which is due to the Wasserstein barycenter problems to obtain $G_{j}^{(t+1)}$ for $1 \leq j \leq m$, and $f(\vec{G}^{(t+1)},\Hbold^{(t+1)}) \leq f(\vec{G}^{(t+1)},\Hbold^{(t)})$, which is due to the optimization steps to achieve elements $H_{u}^{(t+1)}$ of $\Hbold^{(t+1)}$ as $1 \leq u \leq M$. As a consequence, we achieve the conclusion of the theorem.
%%%%%%%%%%%%%%%%%%%%%%%%%%%%%%%%%%%%%%%%%%%%%%%%%
%%%%%%%%%%%%%%%%%
\paragraph{PROOF OF THEOREM \ref{theorem:objective_consistency_multilevel_Wasserstein_means}} %\ref{theorem:objective_consistency_multilevel_Wasserstein_means}}
%%%%%%%%%%%%%%%%%
%%%%%%%%%%%%%%%%%%%%%%%%%%%%%%%%%%%%%%%%%%%%%%%%%
To simplify notation, write
\begin{eqnarray}
L_{\vec{n}}=\mathop {\inf }\limits_{\substack {G_{j} \in \mathcal{O}_{k_{j}}(\Theta), \\ \Hcal \in \mathcal{E}_{M}(\mathcal{P}_{2}(\Theta))}}f_{\vec{n}}(\vec{G},\Hcal), \nonumber \\
L_{0}=\mathop {\inf }\limits_{\substack {G_{j} \in \mathcal{O}_{k_{j}}(\Theta), \\ \Hcal \in \mathcal{E}_{M}(\mathcal{P}_{2}(\Theta))}}f(\vec{G},\Hcal). \nonumber
\end{eqnarray}
For any $\epsilon>0$, from the definition of $L_{0}$, we can find $G_{j} \in \mathcal{O}_{k_{j}}(\Theta)$ and $\Hcal \in \mathcal{E}_{M}(\mathcal{P}(\Theta))$ such that
\begin{eqnarray}
f(\vec{G},\Hcal)^{1/2} \leq L_{0}^{1/2} + \epsilon. \nonumber
\end{eqnarray}
Therefore, we would have
\begin{eqnarray}
L_{\vec{n}}^{1/2}-L_{0}^{1/2} & \leq & L_{n}^{1/2}-f(\vec{G},\Hcal)^{1/2}+\epsilon \nonumber \\
& \leq & f_{\vec{n}}(\vec{G},\Hcal)^{1/2} - f(\vec{G},\Hcal)^{1/2}+\epsilon \nonumber \\
& = & \dfrac{f_{\vec{n}}(\vec{G},\Hcal)-f(\vec{G},\Hcal)}{f_{\vec{n}}(\vec{G},\Hcal)^{1/2}+f(\vec{G},\Hcal)^{1/2}} + \epsilon \nonumber \\
& \leq & \sum \limits_{j=1}^{m}\dfrac{|W_{2}^{2}(G_{j},P_{n_{j}}^{j})-W_{2}^{2}(G_{j},P^{j})|}{W_{2}(G_{j},P_{n_{j}}^{j})+W_{2}(G_{j},P^{j})}+\epsilon \nonumber \\
& \leq & \sum \limits_{j=1}^{m}{W_{2}(P_{n_{j}}^{j},P^{j})}+\epsilon. \nonumber
\end{eqnarray} 
By reversing the direction, we also obtain the inequality $L_{n}^{1/2}-L_{0}^{1/2} \geq \sum \limits_{j=1}^{m}{W_{2}(P_{n_{j}}^{j},P^{j})}-\epsilon$. Hence, $|L_{n}^{1/2}-L_{0}^{1/2}-\sum \limits_{j=1}^{m}{W_{2}(P_{n_{j}}^{j},P^{j})}| \leq \epsilon$ for any $\epsilon>0$. Since $P^{j} \in \mathcal{P}_{2}(\Theta)$ for all $1 \leq j \leq m$, we obtain that $W_{2}(P_{n_{j}}^{j},P^{j}) \to 0$ almost surely as $n_{j} \to \infty$ (see for example Theorem 6.9 in \citep{Villani-2009}). As a consequence, we obtain the conclusion of the theorem.
%%%%%%%%%%%%%%%%%%%%%%%%%%%%%%%%%%%%%%%%%%%%%%%%%
%%%%%%%%%%%%%%%%%
\paragraph{PROOF OF THEOREM \ref{theorem:convergence_measures_multilevel_Wasserstein_means}} %\ref{theorem:convergence_measures_multilevel_Wasserstein_means}}
%%%%%%%%%%%%%%%%%
%%%%%%%%%%%%%%%%%%%%%%%%%%%%%%%%%%%%%%%%%%%%%%%%%
For any $\epsilon>0$, we denote
\begin{eqnarray}
\mathcal{A}(\epsilon)=\biggr\{G_{i} \in \mathcal{O}_{k_{i}}(\Theta), \Hcal \in \mathcal{E}_{M}(\mathcal{P}(\Theta)): \nonumber \\
 d(\vec{G},\Hcal,\mathcal{F}) \geq \epsilon\biggr\}. \nonumber
\end{eqnarray}
Since $\Theta$ is a compact set, we also have $\mathcal{O}_{k_{j}}(\Theta)$ and $\mathcal{E}_{M}(\mathcal{P}_{2}(\Theta))$ are compact for any $1 \leq i \leq m$. As a consequence, $\mathcal{A}(\epsilon)$ is also a compact set. For any $(\vec{G},\Hcal) \in \mathcal{A}(\epsilon)$, by the definition of $\mathcal{F}$ we would have $f(\vec{G},\Hcal) > f(\vec{G}^{0},\Hcal^{0})$ for any $(\vec{G}^{0},\Hcal^{0}) \in \mathcal{F}$. Since $\mathcal{A}(\epsilon)$ is compact, it leads to
\begin{eqnarray}
\inf \limits_{(\vec{G},\Hcal) \in A(\epsilon)}{f(\vec{G},\Hcal)} > f(\vec{G}^{0},\Hcal^{0}). \nonumber
\end{eqnarray}
for any $(\vec{G}^{0},\Hcal^{0}) \in \mathcal{F}$. From the formulation of $f_{\vec{n}}$ as in the proof of Theorem \ref{theorem:objective_consistency_multilevel_Wasserstein_means}, %\ref{theorem:objective_consistency_multilevel_Wasserstein_means}, 
we can verify that $\lim \limits_{\vec{n} \to \infty} f_{\vec{n}}(\widehat{\vec{G}}^{\vec{n}},\widehat{\Hcal}^{\vec{n}}) = \lim \limits_{\vec{n} \to \infty} f(\widehat{\vec{G}}^{\vec{n}},\widehat{\Hcal}^{\vec{n}})$ almost surely as 
$\vec{n} \to \infty$. Combining this result with that of Theorem \ref{theorem:objective_consistency_multilevel_Wasserstein_means}, %\eqref{theorem:objective_consistency_multilevel_Wasserstein_means}, 
we obtain $f(\widehat{\vec{G}}^{\vec{n}},\widehat{\Hcal}^{\vec{n}}) \to f(\vec{G}^{0},\Hcal^{0})$ as $\vec{n} \to \infty$ for any $(\vec{G}^{0},\Hcal^{0}) \in \mathcal{F}$. Therefore, for any $\epsilon>0$, as $\vec{n}$ is large enough, we have $d(\vec{\widehat{G}}^{\vec{n}},\widehat{\Hcal}^{\vec{n}},\mathcal{F}) < \epsilon$. As a consequence, we achieve the conclusion regarding the consistency of the mixing measures.
