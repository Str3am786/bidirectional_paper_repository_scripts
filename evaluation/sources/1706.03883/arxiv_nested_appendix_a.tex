In this appendix, we collect relevant information on the Wasserstein metric and 
Wasserstein barycenter problem, which were introduced in Section \ref{Section:prelim} in the paper. 
For any Borel map $g: \Theta \to \Theta$ and probability measure $G$ on $\Theta$, the 
push-forward measure of $G$ through $g$, denoted by $g\#G$, is defined by the condition
that
$\int \limits_{\Theta}{f(y)}d(g\#G)(y)=\int \limits_{\Theta}{f(g(x))}d G(x)$ for every 
continuous bounded function $f$ on $\Theta$.
\paragraph{Wasserstein metric} \label{Section:Append_Wasserstein_metric}
When $G = \sum \limits_{i=1}^{k}{p_{i}\delta_{\theta_{i}}}$ and $G'=\sum  \limits_{i=1}^{k'}{p_{i}'\delta_{\theta_{i}'}}$ 
are discrete measures with finite support, i.e., $k$ and $k'$ are finite, 
the Wasserstein distance of order $r$ between $G$ and $G'$ can be represented as
\begin{eqnarray}
W_{r}^{r}(G,G') = \min \limits_{T \in \Pi(G,G')} \langle T,M_{G,G'} \rangle \label{eqn:Wasserstein_computation}
\end{eqnarray}
where we have
\begin{eqnarray}
\Pi(G,G') = \left\{T \in \mathbb{R}_{+}^{k \times k'}: T\mathbbm{1}_{k'}=\vec{p}, \ T\mathbbm{1}_{k}=\vec{p}'\right\} \nonumber
\end{eqnarray}
such that $\vec{p}=(p_{1},\ldots,p_{k})^{T}$ and $\vec{p}'=(p_{1}',\ldots,p_{k'}')^{T}$, 
$M_{G,G'} = \left\{\|\theta_{i}-\theta_{j}'\|\right\}_{i,j} \in \mathbb{R}_{+}^{k \times k'}
$ is the cost matrix, i.e. matrix of pairwise distances of elements between $G$ and $G'$, and 
$\langle A, B \rangle =  \text{tr}(A^{T}B)$ is the Frobenius dot-product of matrices. The 
optimal $T \in \Pi(G,G')$ in optimization problem \eqref{eqn:Wasserstein_computation} is called 
the optimal coupling of $G$ and $G'$, representing the \textbf{optimal transport} between these two measures. 
When $k=k'$, 
the complexity of best algorithms for finding the optimal transport is $O(k^{3}\log k)$. 
Currently, \cite{Cuturi-2013} proposed a regularized version of 
\eqref{eqn:Wasserstein_computation} based on Sinkhorn distance where the complexity of 
finding an approximation of the optimal transport is $O(k^{2})$. Due to its favorably
fast computation, throughout the paper we shall utilize Cuturi's algorithm to compute the Wasserstein 
distance between $G$ and $G_{'}$ as well as their optimal transport in 
\eqref{eqn:Wasserstein_computation}.
\paragraph{Wasserstein barycenter} \label{Section:Append_Wasserstein_barycenter}
As introduced in Section \ref{Section:prelim} in the paper, 
for any probability measures $P_{1}, P_{2}, \ldots, P_{N} \in \mathcal{P}_{2}(\Theta)$, their Wasserstein barycenter $\overline{P}_{N,\lambda}$ is such that
\begin{eqnarray}
\overline{P}_{N,\lambda}=\mathop {\arg \min}\limits_{P \in \mathcal{P}_{2}(\Theta)}{\sum \limits_{i=1}^{N}{\lambda_{i}W_{2}^{2}(P,P_{i})}} \nonumber
\end{eqnarray} 
where $\lambda \in \Delta_{N}$ denote weights associated with $P_{1},\ldots,P_{N}$. 
According to \citep{Carlier-2011}, $P_{N,\lambda}$ can be obtained as a solution to 
so-called multi-marginal optimal transporation problem. In fact, if we denote $T_{k}^{1}$ as 
the measure preseving map from $P_{1}$ to $P_{k}$, i.e., 
$P_{k}=T_{k}^{1} \# P_{1}$, for any $1 \leq k \leq N$, then \begin{eqnarray}
\overline{P}_{N,\lambda}=\biggr(\sum \limits_{k=1}^{N}{\lambda_{k}T_{k}^{1}}\biggr)\# P_{1}. \nonumber
\end{eqnarray}
Unfortunately, the forms of the maps $T_{k}^{1}$ are analytically intractable, especially 
if no special constraints on $P_{1}, \ldots, P_{N}$ are imposed.

Recently, \citep{Anderes-2015} studied the Wasserstein barycenters $\overline{P}_{N,
\lambda}$ when $P_{1}, P_{2}, \ldots, P_{N}$ are finite discrete measures and $\lambda=
\biggr(1/N,\ldots,1/N\biggr)$. They demonstrate the following sharp result (cf. Theorem 2 
in \citep{Anderes-2015}) regarding the number of atoms of $\overline{P}_{N,\lambda}$
\begin{customthm}{A.1} \label{theorem:upperbound_barycenter} There exists a Wasserstein 
barycenter $\overline{P}_{N,\lambda}$ such that $\text{supp}(\overline{P}_{N,\lambda}) \leq \sum \limits_{i=1}^{N}{s_{i}}-N+1$.
\end{customthm}
Therefore, when $P_{1},\ldots, P_{N}$ are indeed finite discrete measures and the weights 
are uniform, the problem of finding Wasserstein barycenter $\overline{P}_{N,\lambda}$ over 
the (computationally large) space $\mathcal{P}_{2}(\Theta)$ is 
reduced to a search over a smaller
space $\mathcal{O}_{l}(\Theta)$ where $l=\sum \limits_{i=1}^{N}{s_{i}-N+1}$.
