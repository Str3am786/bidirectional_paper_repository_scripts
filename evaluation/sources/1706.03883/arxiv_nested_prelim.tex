%\subsection{Notations} 

For any given subset $\Theta \subset \mathbb{R}^{d}$, let $
\mathcal{P}(\Theta)$ denote the space of Borel probability measures on $\Theta$. 
The Wasserstein space of order $r \in [1,\infty)$ of probability measures on $\Theta$ is defined as
$\mathcal{P}_{r}(\Theta)=\biggr\{G \in \mathcal{P}(\Theta): \ \int \limits {\|x\|^{r}}dG(x)<\infty \biggr\}$,
where $\|.\|$ denotes Euclidean metric in $\mathbb{R}^{d}$. Additionally, for any $k \geq 
1$ the probability simplex is denoted by $\Delta_{k}=\left\{u \in \mathbb{R}^{k}: \ u_{i} \geq 0, \ 
\sum \limits_{i=1}^{k}{u_{i}}=1 \right\}$. Finally, let $\mathcal{O}_{k}(\Theta)$ 
(resp., $\mathcal{E}_{k}(\Theta)$) 
%(or  $\mathcal{O}_{k}(\mathcal{P}_{r}(\Theta))$
%($\mathcal{E}_{k}(\mathcal{P}_{r}(\Theta))$) 
be the set of probability 
measures with at most (resp., exactly) $k$ support points in $\Theta$.

%\subsection{Wasserstein distances} \label{Section:Wasserstein_metric}
\paragraph{Wasserstein distances}
For any elements $G$ and $G'$ in $\mathcal{P}_{r}(\Theta)$ where $r \geq 1$, the 
Wasserstein distance of order $r$ between $G$ and $G'$ is defined as (cf. \citep{Villani-03}):
\vspace{-6pt}
\begin{eqnarray}
W_{r}(G,G')=\biggr(\mathop {\inf }\limits_{\pi \in \Pi(G,G')}{\int \limits_{\Theta^{2}}{\|x-y\|^{r}}d\pi(x,y)}\biggr)^{1/r} \nonumber
\end{eqnarray}
where $\Pi(G,G')$ is the set of all probability measures on $\Theta \times \Theta$ that have 
marginals $G$ and $G'$. 
In words, $W_r^r(G,G')$ is
the optimal cost of moving mass from $G$ to $G'$, where the cost of moving unit mass is
proportional to $r$-power of Euclidean distance in $\Theta$. 
When $G$ and $G'$ are two discrete measures with finite number 
of atoms, fast computation of $W_r(G,G')$ can be achieved (see, e.g., \cite{Cuturi-2013}). 
The details of this are deferred to the Supplement.
%Section \ref{Section:Append_Wasserstein_metric} in Appendix A. 
 
By a recursion of concepts, we can speak of measures of measures, and define a suitable distance metric
on this abstract space: 
the space of Borel measures on $\mathcal{P}_{r}(\Theta)$, to be denoted by
$\mathcal{P}_{r}(\mathcal{P}_{r}(\Theta))$. This is also a Polish space (that is,
complete and separable metric space) as $
\mathcal{P}_{r}(\Theta)$ is a Polish space. It will be endowed with a Wasserstein metric of 
order $r$ that is induced by a metric $W_{r}$ on $\mathcal{P}_{r}(\Theta)$ as follows (cf. 
Section 3 of \cite{Nguyen-2016}): for any $\mathcal{D},\mathcal{D'} \in \mathcal{P}_r(\mathcal{P}_r(\Theta))$
\vspace{-6pt}
\begin{eqnarray}
W_{r}(\mathcal{D},\mathcal{D}'):=\biggr(\mathop {\inf }{\int \limits_{\mathcal{P}_{r}(\Theta)^{2}}{W_{r}^{r}(G,G')}d\pi(G,G')}\biggr)^{1/r} \nonumber
\end{eqnarray}
where the infimum in the above ranges over all $\pi \in \Pi(\mathcal{D},\mathcal{D}')$ 
such that $\Pi(\mathcal{D},\mathcal{D}')$ is the set of all probability measures on $
\mathcal{P}_{r}(\Theta) \times \mathcal{P}_{r}(\Theta)$ that has marginals $\mathcal{D}
$ and $\mathcal{D}'$. In words, $W_r(\mathcal{D},\mathcal{D'})$ corresponds to 
the optimal cost of moving mass from $\mathcal{D}$ to $\mathcal{D'}$, where
the cost of moving unit mass in its space of support $\mathcal{P}_r(\Theta)$ 
is proportional to the $r$-power of the $W_r$ distance in $\mathcal{P}_r(\Theta)$.
Note a slight notational abuse --- $W_r$ is used for both
$\mathcal{P}_r(\Theta)$ and $\mathcal{P}_r(\mathcal{P}_r(\Theta))$, but
it should be clear which one is being used from context.

%\subsection{Wasserstein barycenter} \label{Section:Wasserstein_barycenter}
\paragraph{Wasserstein barycenter}
Next, we present a brief overview of Wasserstein barycenter problem, first
studied by \citep{Carlier-2011} and subsequentially many others (e.g., \citep{Benamou-15, Solomon-15, Alvarez-16}). 
%A detailed discussion of this problem is deferred to Section 
%\ref{Section:Append_Wasserstein_barycenter} in Appendix A. 
Given probability measures 
$P_{1}, P_{2}, \ldots, P_{N} \in \mathcal{P}_{2}(\Theta)$ for $N \geq 1$, their 
Wasserstein barycenter $\overline{P}_{N,\lambda}$ is such that
\vspace{-6pt}
\begin{eqnarray}
\overline{P}_{N,\lambda}=\mathop {\arg \min}\limits_{P \in \mathcal{P}_{2}(\Theta)}{\sum \limits_{i=1}^{N}{\lambda_{i}W_{2}^{2}(P,P_{i})}} \label{eqn:Wasserstein_barycenter}
\end{eqnarray} 
where $\lambda \in \Delta_{N}$ denote weights associated with $P_{1},\ldots,P_{N}$.
When $P_{1},\ldots, P_{N}$ are discrete measures with finite number of atoms and the 
weights $\lambda$ are uniform, it was shown by \citep{Anderes-2015}
that the problem of finding Wasserstein barycenter $\overline{P}
_{N,\lambda}$ over the space $\mathcal{P}_{2}(\Theta)$ in 
\eqref{eqn:Wasserstein_barycenter} is reduced to search only over a much simpler space 
$\mathcal{O}_{l}(\Theta)$ 
%(cf. Theorem \ref{theorem:upperbound_barycenter} in Section \ref{Section:Append_Wasserstein_barycenter} in Appendix A) 
where $l=\sum \limits_{i=1}
^{N}{s_{i}-N+1}$ and $s_{i}$ is the number of components of $P_{i}$ for all $1 \leq i \leq 
N$. 
Efficient algorithms for finding local solutions of the Wasserstein barycenter problem 
over $\mathcal{O}_{k}(\Theta)$ for some $k \geq 1$ have been studied recently in 
\citep{Cuturi-2014}. These algorithms will prove to be a useful building block for 
our method as we shall describe in the sequel. 
The notion of Wasserstein barycenter has been utilized for approximate Bayesian inference
\citep{Sanvesh-aistats}.

%\subsection{K-means clustering as quantization problem} \label{Section:K_means_quantization}
\paragraph{K-means as quantization problem}
The well-known $K$-means clustering algorithm can be viewed as solving
an optimization problem that arises in the problem of quantization, a simple but very useful connection 
\citep{Pollard-1982, Graf-2000}. The connection is the following.
Given $n$ unlabelled samples $Y_{1},\ldots,Y_{n} \in \Theta$. Assume that these data are associated
with at most $k$ clusters where $k \geq 1$ is some given number. The $K$-means problem finds the set $S$ 
containing at most $k$ elements $\theta_{1},\ldots, \theta_{k} \in \Theta$ that minimizes 
the following objective
\vspace{-6pt}
\begin{eqnarray}
\mathop {\inf }\limits_{S : |S| \leq k}{\dfrac{1}{n}\sum \limits_{i=1}^{n}{d^{2}(Y_{i},S)}}. \label{eqn:original_Kmeans}
\end{eqnarray}
Let $P_{n}=\dfrac{1}{n}\sum \limits_{i=1}^{n}{\delta_{Y_{i}}}$ be the empirical measure of data 
$Y_{1},\ldots,Y_{n}$. Then, problem \eqref{eqn:original_Kmeans} is 
equivalent to finding a discrete probability measure $G$ which has finite 
number of support points and solves:
\vspace{-6pt}
\begin{eqnarray}
\mathop {\inf }\limits_{G \in \mathcal{O}_{k}(\Theta)}{W_{2}^{2}(G,P_{n})}. \label{eqn:Wasserstein_K_means}
\end{eqnarray} 
Due to the inclusion of Wasserstein metric in its formulation, we call this
a \emph{Wasserstein means problem}. This problem can be further thought of as a 
Wasserstein barycenter problem where $N=1$. In light of this observation, as noted by
\citep{Cuturi-2014}, the algorithm for finding the Wasserstein barycenter offers an 
alternative for the popular Loyd's algorithm for determing local minimum of the K-means objective. 
