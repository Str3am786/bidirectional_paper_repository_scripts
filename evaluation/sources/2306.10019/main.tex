\documentclass[journal,twocolumn]{IEEEtran}

% *** GRAPHICS RELATED PACKAGES **
\ifCLASSINFOpdf
\else
\fi
\let\labelindent\relax

\usepackage[hidelinks]{hyperref}
\usepackage{graphicx}
\usepackage{adjustbox}
\usepackage[utf8]{inputenc}
\usepackage{booktabs}
\usepackage{amsmath}
\usepackage{amssymb}
\usepackage[numbers]{natbib}
\usepackage{paralist}
\usepackage{multirow}
\usepackage{xspace}
\usepackage{color}
\usepackage{xcolor}
\usepackage[graphicx]{realboxes}
\usepackage{ifthen}
\usepackage{url}
\usepackage{fancybox}
\usepackage{enumitem}
\usepackage{listings}
\usepackage{balance}
\usepackage{ifthen}
\usepackage{graphicx}
\usepackage{amsmath}
\usepackage[linesnumbered,boxruled]{algorithm2e}
\usepackage{algorithm2e}
\usepackage{tablefootnote}
\usepackage{comment}
\usepackage{float}
\usepackage{algorithmic}
\usepackage{dirtytalk}
% \usepackage{subfigure}
\usepackage[tikz]{bclogo}
% \usepackage[colorinlistoftodos]{todonotes}
\usepackage{subcaption}
\usepackage{tcolorbox}
\usepackage{booktabs}
\usepackage{tikz}
\usetikzlibrary{plotmarks}
\usetikzlibrary{arrows,shapes,positioning}
\usetikzlibrary{decorations.markings}
\tikzstyle arrowstyle=[scale=1]
\tikzset{>=latex}

\usepackage{pgfplotstable}
\def\UrlBreaks{\do\.\do\@\do\\\do\/\do\!\do\_\do\|\do\%\do\;\do\>\do\]% 
\do\)\do\,\do\?\do\'\do\+\do\=\do\#\do\-\do\a\do\b\do\c\do\d\do\e\do\f\do\g\do\h\do\i\do\j\do\k\do\l\do\m\do\n\do\o\do\p\do\q\do\r\do\s\do\t\do\u\do\v\do\w\do\x\do\y\do\z\do\0\do\1\do\2\do\3\do\4\do\5\do\6\do\7\do\8\do\9}%

% declare the path(s) where your graphic files are
\graphicspath{{pics/}}
% \newcommand\todoV[2]{ {\colorbox{yellow}{\textcolor{red}{#1}}} {\todo[color=green!40]{\thesubsection. #2}}}
% \newcommand\todoU[2]{ {\colorbox{yellow}{\textcolor{red}{#1}}} {\todo[color=red!40]{\thesubsection. #2}}}
% \newcommand\todoL[2]{ {\colorbox{yellow}{\textcolor{red}{#1}}} {\todo[color=blue!40]{\thesubsection. #2}}}

% \newcommand\revdif[2]{\nb{#1}{\color{orange}#2}}
% \newcommand\rev[2]{\nb{#1}{\color{red}#2}}
% \newcommand{\nb}[2]{
% 	{
% 		{\color{red}{
% 				\fbox{\bfseries\sffamily\scriptsize#1}
% 				{\sffamily$\triangleright~${\it\sffamily #2}$~\triangleleft$}
% 	}}}
% }

\definecolor{codegreen}{rgb}{0,0.6,0}
\definecolor{codegray}{rgb}{0.5,0.5,0.5}
\definecolor{codepurple}{rgb}{0.58,0,0.82}
\definecolor{backcolour}{rgb}{0.95,0.95,0.92}

\lstdefinestyle{mystyle}{
	backgroundcolor=\color{backcolour},   
	commentstyle=\color{codegreen},
	keywordstyle=\color{magenta},
	numberstyle=\tiny\color{codegray},
	stringstyle=\color{codepurple},
	basicstyle=\footnotesize,
	breakatwhitespace=false,         
	breaklines=true,                 
	captionpos=b,                    
	keepspaces=true,                 
	numbers=left,                    
	numbersep=2pt,                  
	showspaces=false,                
	showstringspaces=false,
	showtabs=false,                  
	tabsize=2
}

\lstset{style=mystyle}
\usepackage[english]{babel}
\usepackage{epigraph} 
\DeclareGraphicsExtensions{.pdf,.jpeg,.png}
\begin{document}


\newtheorem{theorem}{Definition}[section]
	\newcommand{\eg}{e.g.,}
	\newcommand{\ie}{i.e.,}	
	\renewcommand{\lstlistingname}{Listing}% Listing -> Algorithm

\newcommand{\boxedtext}[1]{\fbox{\scriptsize\bfseries\textsf{#1}}}
\newcommand{\nota}[2]{
	\boxedtext{#1}
		{\small$\blacktriangleright$\emph{\textsl{#2}}$\blacktriangleleft$}
}

% \usepackage[colorinlistoftodos]{todonotes}
\def\mybar#1{%%
  #1s & {\color{grey}\rule{#1cm}{8pt}}}

\newcommand\revdif[2]{\nb{#1}{\color{orange}#2}}
\newcommand\rev[2]{\nb{#1}{\color{red}#2}}
\newcommand{\nb}[2]{
	{
		{\color{red}{
				\fbox{\bfseries\sffamily\scriptsize#1}
				{\sffamily$\triangleright~${\it\sffamily #2}$~\triangleleft$}
	}}}
}


\definecolor{chestnut}{rgb}{0.8, 0.36, 0.36}


\newcommand\respond[2]{{\nb{#2}{\color{blue}#1}}}



\newcommand{\impacttext}[1]{{\color{red}#1}}
\newcommand{\rone}[1]{\colorbox{yellow}{\textbf{[#1]}}}
\newcommand\revise[1]{\nb{Revised}{\color{orange}#1}}
\newcommand\mc[1]{\nb{Marc}{\color{red}#1}}
\newcommand\rk[1]{\nb{Raula}{\color{blue}#1}}
\newcommand\ct[1]{\nb{Christoph}{\color{magenta}#1}}
\newcommand\reviewer[1]{\nb{Reviewer}{\color{red}#1}}
\newcommand{\fixme}[1]{{\textcolor{red}{[FIXME] #1}}\xspace}
\newcommand{\cn}{{\color{violet}[citation required]}}
\newcommand{\tbd}{{\color{green}[To Be Done!]}}
\definecolor{chestnut}{rgb}{0.8, 0.36, 0.36}

% \newcommand\autcomment[1]{{\textcolor{pink}{\textbf{}#1}}}

% \newcommand\ct[1]{\textcolor{pink}{\textbf{Christoph:}} {\textcolor{pink}{#1}}}

% \newcommand\rk[1]{\textcolor{green}{\textbf{Raula:}} {\textcolor{red}{#1}}}

% \newcommand\mc[1]{\textcolor{blue}{\textbf{Marc:}} {\textcolor{blue}{#1}}}



\newcommand\review[3]{\textcolor{red}{\sout{#1}} {\textcolor{blue}{#2}}{\todo{#3}}}

\newcommand{\Rqone}{RQ$_1$: \textit{To what extent do developers make contributions to the ecosystem?}}
\newcommand{\Rqtwo}{RQ$_2$: \textit{What kinds of contributions are made to the ecosystem?}}
\newcommand{\Rqthree}{RQ$_3$: \textit{What are the motivation to contribute to the  ecosystem?}}

\definecolor{beaublue}{rgb}{0.74, 0.83, 0.9} 
\definecolor{redbeau}{HTML}{cc5b5b}  
\definecolor{orabeau}{HTML}{ff9f79}

 % \title{Why Maintainers should be Contributors: Motivations, Expertise and Barriers from \\ the NPM Ecosystem}

  \title{Ethical Considerations Towards Protestware}

% \author{
% 	\begin{tabular}{cccc}
% 		\multicolumn{3}{c}{Marc Cheong, Raula Gaikovina Kula, Christoph Treude} \\
% 		{NAIST, University of Adelaide, Shinshu University, Federal University of Technology Paraná} \\ 
% 		{raula-k@is.naist.jp, \{christoph.treude, sebastian.baltes\}@adelaide.edu.au, hata@shinshu-u.ac.jp} \\
% 		{igor@utfpr.edu.br, Marco.Gerosa@nau.edu, winifred.amini@gmail.com}\\  
% 	\end{tabular} 
% }

\author{
\IEEEauthorblockN{Marc Cheong\IEEEauthorrefmark{2}, 
Raula Gaikovina Kula\IEEEauthorrefmark{1}, and
Christoph Treude\IEEEauthorrefmark{2}}\\
\IEEEauthorblockA{\IEEEauthorrefmark{2}University of Melbourne, Australia,
\IEEEauthorrefmark{1}Nara Institute of Science and Technology, Japan
\\
marc.cheong@unimelb.edu.au,christoph.treude@unimelb.edu.au,  raula-k@is.naist.jp }
}

\maketitle

\begin{abstract}
A key drawback to using a Open Source third-party library is the risk of introducing malicious attacks.
In recently times, these threats have taken a new form, when maintainers turn their Open Source libraries into protestware.
This is defined as software containing political messages delivered through these libraries, which can either be malicious or benign.
Since developers are willing to freely open-up their software to these libraries, much trust and responsibility are placed on the  maintainers to ensure that the library does what it promises to do.  
This paper takes a look into the possible scenarios where developers might consider turning their Open Source Software into protestware, using an ethico-philosophical lens.
Using different frameworks commonly used in AI ethics, we explore the different dilemmas that may result in protestware.
Additionally, we illustrate how an open-source maintainer's decision to protest is influenced by different stakeholders (\textit{viz.}, their membership in the OSS community, their personal views, financial motivations, social status, and moral viewpoints), making protestware a multifaceted and intricate matter.
\end{abstract}

\epigraph{``When people feel they are not being heard, they may resort to different measures to get their message across. In the case of programmers, they have the unique ability to protest through their code.''}{Kula \& Treude (2022) \cite{kula2022war}}

\epigraph{``Consequently, he found himself confronted by two very different modes of action; the one concrete, immediate, but directed towards only one individual; and the other an action addressed to an end infinitely greater... but for that very reason ambiguous... He had to choose between those two. What could help him to choose? ''}{ Sartre (1946)\\ \textit{Existentialism Is a Humanism} (trans.:  Mairet) \cite{SartreJean-Paul1948Eah}}

\section{Introduction}
% It is now standard practice for software development teams to adopt third-party libraries into their software applications, mainly due to the many benefits of saving time, relying on an already quality-proven piece of code, that is trusted by the community.
% Yet, the risk involves 
In this article, we articulate the motivations behind maintainers who turn their Open Source Software (OSS) into protestware.
Although ethics in computing is not new, the phenomenon of Protestware is unique, in that the power of responsibility is placed on individuals (i.e., sometimes a single library maintainer), as opposed to the often-diffused responsibilities behind the deployment of AI and other technologies, due to the participation of more than one person.
We then explore the dilemmas that a library maintainer may face.
We also discuss potential guidelines and larger ethical implications for the open source, industry, research, and education sectors.

\section{Background}
\label{sec:background}
\subsection{Context}
\label{sec:context}
%Weaponizing OSS
%(take notes from the War and Protest paper).
%When it no longer becomes Open Source
%Benign and Malicious impact...

In March 2022, the maintainer of node-ipc \cite{Web:node-ipc}, a widely used software library, intentionally introduced a vulnerability into their code. If the code was run within Russia or Belarus, it would attempt to replace all files on the user's device with a heart emoji.\footnote{\url{https://techcrunch.com/2022/07/27/protestware-code-sabotage/}} This critical security flaw (i.e., CVE-2022-23812~\cite{Web:CVE-2022-23812}) highlights the trend of programmers intentionally sabotaging their code for political purposes, a practice known as ``\textit{protestware}''~\cite{kula2022war}.

The malicious code was intended to overwrite arbitrary files depending upon the geolocation of the user's IP address: in essence, attacking software in specific locations.
Specifically, the affected versions 10.1.1 and 10.1.2 of the library check whether the host machine has an IP address in Russia or Belarus, and if so, overwrites every file it could with a heart symbol. 
Version 10.1.3 was released soon after without this destructive functionality, while Versions 10.1.1 and 10.1.2 were removed from the NPM registry.

 Responses from the community varied, including frustrations that led to insightful discussions.
One example from a contributor on the GitHub Discussions channel is shown below~\cite{Web:discussion}:
\begin{quote}
\textit{I'm very happy to see that the principles and character of many in tech (FOSS especially) remain clear enough to recognize how completely wrong this was. Of course, if the marketplace of current things keeps hammering away at this, it will benefit a small number of corporate giants (misplaced trust/safety). I hope we all start seeing these patterns as we grapple with a general blurring of lines between tools for marketing and weaponry.
It's essential to ask: what's the outcome and who benefits? I like to ask the faux ideologues ``who agrees with you?'' ``Isn't it strange how well aligned you are with a small number of very visible, influential, and powerful organizations?'' ``What's the fight and who is on which side, again?''
It's about competency, not power. Power feeds and is fueled by egocentrism (plainly, weak vanity). Competency comes from discovering your natural gifts and applying them.} (sic.)
\end{quote}

Another user from that GitHub Discussion quoted how this affected the Open Source Community \cite{Web:discussion}:
\begin{quote}
 \textit{The trust factor of open source, which was based on goodwill of the developers is now practically gone, and now, more and more people are realizing that one day, their library/application can possibly be exploited to do/say whatever some random dev on the internet thought was `the right thing to do'.} (sic.)
\end{quote}
The maintainer in question defended his module on GitHub, saying that \textit{``this is all public, documented, licensed and open source''} \cite{kula2022war}. Earlier, there were more than 20 issues flagged against node-ipc about its behavior.
Some of the comments referred to the creation as \textit{``protestware, while others might call it malware''}  \cite{kula2022war}. 

We present another case where the protestware does not have malicious intent, but aims at increasing awareness.
The same maintainer of the node-ipc library then created the peacenotwar library~\cite{Web:node-peacenotwar}.
As explained by the maintainer, it serves as a non-violent protest against Russia's aggression.
Instead of malicious deletion of files, the module adds a message of peace on users' desktops~\cite{Web:peacecommit}.
The maintainer was quoted in the README file\footnote{\url{https://github.com/RIAEvangelist/peacenotwar}}:
\begin{quote}
    \textit{I pledge that this module, to the best of my knowledge and skills, does not do any damage to anyone's data. If you do not like what this module does, please just lock your dependencies to any of my work or other's which includes this module, to a version you have code reviewed and deemed acceptable for your needs. Also, please code-review your other modules for vulnerabilities.}
\end{quote}

% Treating as a case study, we find that a developer of the JavaScript library node-ipc~\cite{Web:node-ipc}, deliberately introduced a critical security vulnerability (i.e., CVE-2022-23812~\cite{Web:CVE-2022-23812}) that targets computers' files based on the ip-address of the user.
% Seemingly, the maintainer intentionally changed his code to overwrite the host system's data, then changed the code to display a message calling for world peace, as a protest against Russia's invasion of Ukraine.

\subsection{Characterization}
\label{sec:characterization}
Protestware can take three forms \cite{kula2022war}: 

\begin{description}
\item[malignant protestware] which intentionally damages or takes control of a user's device without their knowledge or consent; 
\item[benign protestware] which raises awareness about a social or political issue without causing harm (e.g., changes to license files\footnote{\url{https://github.com/terraform-aws-modules/terraform-aws-ec2-instance/commit/6867788411a202b61187f9935e9eaa72a18f0bbe}}); and 
\item[developer sanctions] where programmers' accounts are suspended by internet hosting services (e.g., GitHub suspending Russian accounts\footnote{\url{https://www.jessesquires.com/blog/2022/04/19/github-suspending-russian-accounts/}}). 
\end{description}

Protestware can manifest in various ways, such as project documentation (e.g., README banner), communication (e.g., log messages), environment (e.g., injected code on target machines), or output (e.g., file deletions). The latter two forms are often considered security vulnerabilities,\footnote{\url{https://snyk.io/blog/protestware-open-source-types-impact/}} while protest through documentation or log messages is not typically classified as requiring security advisories. Protestware can have a wide-ranging impact on numerous stakeholders, including other contributors to the same project, direct and indirect users of the project, and even the entire open-source community and its newcomers.

The rise of protestware raises the question of whether it is ethical to intentionally worsen something in order to make a point. This issue is particularly significant in software ecosystems where code is frequently reused~\cite{zahan2022weak}, as these ecosystems rely on the trust and reliability of the code being used. If a programmer introduces protestware into their code, it can introduce a ripple effect: compromising the security and stability of the software for all users who reuse that code, potentially affecting individuals and businesses that rely on the software, as well as further dependencies.


%\rk{
%Should we define the key stakeholder in the Protestware %discussions?
%}

%\rk{
%To what extent is vulnerability labelled as a type of %protestware, if the intention is just as ethical?
%}

%\rk{
%Types of Protestware
%https://snyk.io/blog/protestware-open-source-types-impact/
%- Documentation Injection
%- Communication Injection
%- Environment Injection
%- Output Injection
%}

\subsection{Protestware: Beyond Computing Ethics?}
The literature for computing and AI ethics is rapidly growing, predominantly on sociotechnical systems and "artificial moral agents" \cite{Zoshak2021} such as algorithmic recommender systems, self-driving cars, algorithmic policing and law enforcement, and AI-based person recognition.

What sets our proposed analysis of protestware apart from existing computing ethics studies is that the stakeholders involved have a more personal, direct, and explicit involvement. To illustrate: consider a company (\textit{BigCorp}) whose self-driving car who has injured a pedestrian: it is hard to see which individual programmer or engineer is responsible for the injury, based on the `diffusion' of responsibility through \textit{BigCorp}'s organisational structure. Contrast this with an individual volunteer (\textit{Violet}), of an open source software library, who exhibits a form of activism by adding a few lines of code to prevent her software library from working in certain regions. 

% \begin{tcolorbox}
Protestware ethics is an extension to 'traditional' AI ethics: the power of responsibility is placed on -- or diffused amongst -- individuals, as opposed to large AI corporations.
% \end{tcolorbox}
In order to think about new guidelines for deciding ethical actions and consequences for \textit{Violet}, we will need to briefly explore the landscape of applied ethics.

\section{Ethics: A Primer}

\label{sec:ethics}
There exist various ethical theories, each with their own pros and cons \cite{Rachels2015}, which sometimes even conflict with each other in terms of their application and evaluation. 
Ethics, simply put, is about doing the ``\textit{right}'' things. However, there is no clear answer to what make things ``right'' or ``moral''. Enter \textit{applied ethics} -- the application of ``one moral theory or other ... upon the applied ethics problem at hand, in the hopes of producing a resolution'' \cite{sep-theory-bioethics}. 

The field of medicine was one of the forerunners of this (e.g., by asking, \textit{Using common principles of medical ethics, how do we do no harm to patients under our care?}). This idea was quickly adapted into --- and gained traction within --- research in AI and computing in recent years (e.g., by similarly asking, \textit{How do self-driving cars do no harm to pedestrians and drivers we are responsible for, if those same principles of medical ethics are adopted and adapted?)}. 
Philosophers from Immanuel Kant and Jeremy Bentham in the 18th century, to Tom Beauchamp and James Childress in the 20th century, have proposed different approaches to the question of how to do so. Herein, we discuss three popular approaches, taking a leaf from the current state of computing ethics \cite{Zoshak2021}.



%\begin{tcolorbox}\textbf{}
%\textbf{Applied Ethics:}
%How to do no harm to both the NPM community (i.e., xx packages), and to the users (i.e., xx repos on GitHub) of the library when protesting?
%\end{tcolorbox}

\subsection{Duty Ethics}
The ethical theory of \textit{deontology} -- or `duty ethics' -- basically posits that a moral agent\footnote{Simply put, a `moral agent' is usually a person who has agency to decide how to act.} has a ``sense of duty'' or requirement to do the right thing, which guides their actions. Immanuel Kant's `Categorical Imperative' offers a beautiful application of this, as summarised succinctly by Rachels: ``When you are thinking about doing something, ask what rule you would be following if you actually did it... Then ask whether you would be willing for your [rule]... to become a universal law'' \cite{Rachels2015}. In our example for \textit{Violet}, she ought to think about what would happen if \textit{all} programmers did the way she did, with no exception: would she be able to live with the consequences on a universal level? We could quickly see that there are exceptions to this: what happens if, say, by following her hypothetical universal law, one programmer's actions end up disabling life support machines in hospitals?

Following this maxim, %universal law
hypothetically, if all 2,215,398 packages in the npmjs ecosystem decided to inject malicious code into their systems, it would break all libraries, affecting all libraries in the ecosystem, as well as all the potential applications that adopt the libraries into the ecosystem. 
Since JavaScript is the top ranked language on GitHub, this is a significant portion of GitHub projects as well.
Although her intent was to protest against persons or groups this would disqualify the code from being openly distributed.
Hence, beyond \textit{moral} harm, it will also disqualify license usage, which also has \textit{legal} implications for open source software usage.
From a market report\footnote{https://www.fortunebusinessinsights.com/open-source-services-market-106469} and according to a Red Hat 2019 report, OSS IT infrastructure is used by 53\% of organisations, 43\% integrations, and 42\% in digital transformation. 
Permissive licenses are used by 76\% of OSS and only 24\% are copyleft. 
% The universal law puts the whole concept of Open Source, violating licences and usage.}
\begin{tcolorbox}
\textbf{Duty Ethics:} An example of applying the Kantian Categorical Imperative to the dilemma would be, e.g., \textit{What if all programmers regard malicious code injection as ethically imperative?} One can argue that the library will ultimately violate the tenets of OSS, leading to a ban on the library and the user. Hence, by implication, malicious code injection is not ethical.
\end{tcolorbox}

%% NOTES: Duty ethics [data from pull reqs, not systematic]
%% DUTY TO WHOM?  REF2!!
\subsection{Consequentialist ethics}
Next, another ethical theory which appeals to many computing professionals, due to its `mathematical' nature, is \textit{utilitarianism} - under the broad banner of \textit{consequentialist} ethics. Simply put, it is the consequences that are to be the yardstick by which to measure one's initial actions; and it is one's responsibility to maximise the overall happiness (or `utiles' if you wish, mathematically-speaking) across all stakeholders, and to minimise any unhappiness \cite{Rachels2015}. First proposed by philosophers such as Jeremy Bentham, it is thus easy to understand the attractiveness of this theory, as it boils down to an optimisation problem of `utiles'. However, translating it into practice is much harder than it looks on the surface: for starters, how could Violet measure the overall nett gain of her activism? When the `utiles' are divided across the entire population, does an individual merely get an infinitesmal $+0.0001$? Also, who gets to be the arbiter of the quantity of the individual utiles?. Most crucially, how would she quantify the overall nett loss of unintended consequences? (Again with our life support machine example from before: is the unintended death of a person worth $-10,000$? $-100,000$? Some might decide that is is simply \textit{unquantifiable}!)

In terms of the consequences, the risk of losing any of the users of the library, potential community of contributors, and also their standing in the OSS community, are all net losses that need to be quantified. 
Also, since the library will violate the OSS definition, it will no longer be listed on the registry, which will lead to a community-wide loss of happiness. Is this worth the risk?

\begin{tcolorbox}
\textbf{Consequentialist Ethics:}
An example of a utilitarian's reasoning would take the form of, e.g., \textit{Potentially risking the ban of the library might be a bigger consequence (higher nett negative `utiles') than trying to target a subset of users to send a message (smaller nett positive `utiles'), leading to an overall negative in the balance of probabilities.} 
To put this reasoning in plain language: although the damage might have short term benefits, the long term effects and consequences are much larger. 
\end{tcolorbox}

%% Also, ~~Bentham~~ theory - what maximizes the best outcomes for most people... [Consequentialist - number of deps] REF1!!
\subsection{Principlism}
A more pragmatic approach will be to follow a set of fixed principles, in the namesake philosophical framework of \textit{principlism}. Here, we draw inspiration from Beauchamp and Childress' landmark \textit{Principles of Biomedical Ethics}, which posits four key principles: respect for autonomy; nonmaleficence (doing no harm); beneficence (doing good); and justice \cite{Beauchamp1994}. All these principles have been in use in biomedical ethics, and have seen promise in evaluating issues in technological ethics.

\begin{tcolorbox}
    \textbf{Principle 1 --- Respect for autonomy:} this principle involves respecting the freedom and autonomy between users, maintainer, and the broader OSS community. This lies at the heart of OSS, per e.g., the Free Software Foundation \cite{GNU}. 
\end{tcolorbox}

\textbf{Respect for autonomy:} In considering this principle, we have identified at least four parties involved: the users of a library; the maintainer; the contributor; and finally the ecosystem as a community of OSS developers.
Since OSS is driven by volunteer contributions, would curtailing any party's autonomy -- from denying them use of the software, to, say, causing them to work on weekends in order to revert changes to repositories or fix production code -- impact their freedom, or worse,change any of the parties motivations? Overall considerations will involve respecting the freedom of choice for all parties involved.

%Would someone be demotivated or create a barrier to keep the OSS project surviving? 

\begin{tcolorbox}
    \textbf{Principle 2 --- Nonmaleficence:} this principle requires that harm \textit{should not} be caused to stakeholders. This includes indirect harm caused to others in the OSS community based on their "assumed belief, group membership, or behavior" \cite{coghlan}.
    %using documentation (i.e., README file) and communication channels (i.e., GitHub Pull Request, Issues, and Discussions) is the most effective method for maintainers to protest. 
    %However, the effectiveness is unknown.
\end{tcolorbox}

\textbf{Nonmaleficence:} This, we reason, would only apply to the benign form of protestware. To recap, these forms will not cause undue impact, when compared to their malignant counterparts which may lead to security vulnerabilities. 
%To what extent is the protestware drawing attention to the different stakeholders?
Critical questions when evaluating this principle includes, say, \textit{How does a protestor managing between sending a message, while not causing harm to any of the stakeholders.} 
There are examples of README placements of protestware \footnote{\url{https://github.com/vshymanskyy/StandWithUkraine}}: while delicate, this is a strategy is to use documentation and communication channels, as opposed to modification of the actual code.

% \rk{use the case studies of where we find}

% \rk{case study of colors and faker, also for others}
\begin{tcolorbox}
    \textbf{Principle 3 --- Beneficence:} this principle stipulates that consequence of an action should result in good or benefit. Note that this does not just consider the provision of benefits, but also, "balancing benefits against risks and costs" \cite{Beauchamp1994}
    %%currently there is little evidence of such efforts, however, some maintainers have been rewarded with community aid when protesting against the abuse of Open Source. 
\end{tcolorbox}

\textbf{Beneficence:} As we have seen in our treatise on utilitarianism, the benefits of protestware could be difficult to quantify at this stage. Nonetheless, an argument can be made for the benefits of fund-raising efforts such as sponsors or ad campaigns to create awareness on the political situation. 
This could be in the form of incentives to users, and to contributors of the source code.
For historical context, however, a common incentive in sabotaging-one's-own-code was to help sponsor struggling maintainers who wanted to protest against their software being used by large corporations\footnote{https://www.independent.co.uk/tech/developer-sabotages-code-protest-github-colors-faker-b1990161.html}.
%There is no evidence of such kind of principles behind driving Protestware. 

\begin{tcolorbox}
    \textbf{Principle 4 --- Justice:} The effects of actions must be just and fair in terms of the distribution of "benefits, risks, and costs" \cite{Beauchamp1994}
    %%this is not guaranteed, as most users might not be the target of the protest. 
\end{tcolorbox}

\textbf{Justice:}
Finally, the argument for justice requires not just fairness of outcomes, but also its \textit{distribution}, which includes the risks and costs \cite{Beauchamp1994}. 
In a historical context, it is prudent to consider the protest against unfair usage of OSS by the industry (which precedes protestware), raised in \textit{Beneficence} above. The original \textit{raison d'\^{e}tre} involves getting justice against the actual corporations that use the library. However, at the heart of this principle, other users and stakeholders may be disproportionately affected by this maneuver which leads it into question. 
%Unless there is a universal stance, you only reach a portion of your targets.
%\texttt{the justice failed}

\vspace{8pt}
Think of these as valuable \textit{heuristics} for which we could evaluate the fitness of an action we are taking. Again, despite our best intentions, we might run into trouble fairly quickly. For example, what do we do when we do not satisfy all the listed heuristics; Violet might (indirectly) cause harm and deny autonomy to innocent users who are affected by her code destruction, while still advocating for perceived justice and  beneficence, based on the cause of her activism?



% \mc{Segue to next part...}


% \usepackage{booktabs}


% \begin{table*}
% \centering
% \caption{Ethics Frameworks}
% \begin{tabular}{llll} 
% \toprule
% Ethics           & Applied Case                                                                                                                                        & Measured Impact                                &                                                        \\
% Deontology       & \`{}\`{}What if'' all of NPM turned into Protestware                                                                                                & whole NPM ecosystem                            &                                                        \\
% Consequentialist & Maintainer measures response on dependents                                                                                                         & Anyone dependent on the library                & Decreased downloads, Losing Reputation with community  \\
% Principalism     & \begin{tabular}[c]{@{}l@{}}1. respect for autonomy;\\2. nonmaleficence (doing no harm); \\3. beneficence (doing good); and\\4. justice\end{tabular} & \begin{tabular}[c]{@{}l@{}}\\\\\\\end{tabular} &                                                        \\
% \bottomrule
% \end{tabular}
% \end{table*}


\section{Promoting Ethical Responsibility}
\label{sec:guidelines}

Assessing the ethical implications of an open source maintainer's decision to convert their software into protestware is a multifaceted and intricate matter. For starters, the ethical frameworks may not agree with each other: as we have seen, the Kantian Categorical Imperative (duty ethics) might be a straightforward ``don't do it". 

Drawing upon principlism, however, the nuance is visible. From the standpoint of autonomy, maintainers possess the right to make choices regarding \textit{their own} creations. However, this decision may conflict with \textit{end users'} autonomy, as it could restrict their ability to utilize the software without unforeseen --- or disastrous --- consequences. The principles of beneficence and non-maleficence are also crucial to consider: although protestware might advance a greater good by raising awareness or advocating change, it could simultaneously inflict harm upon users who depend on the software for essential functions (again, with our life support machine example raised several times before). The justice principle underlines the importance of examining the fairness and equitability of deploying protestware as a protest method. Hence it is vital to evaluate whether such actions disproportionately impact specific groups (including, e.g., time and effort to rectify deleterious code behavior) or inadvertently create disparities in software resource access. Bearing these ethical principles in mind, determining the suitability of transforming OSS into protestware necessitates a thorough analysis of the potential benefits, drawbacks, and broader societal ramifications of such a decision.

In this section, we present various initiatives which can be implemented to promote a more balanced perspective on protestware --- with an emphasis on its potential risks --- and encouraging maintainers to consider alternative methods for expressing their concerns. We also suggest directions for future work, including examining the role of OSS governance, policy, and the `social license to operate', before finally identifying ways to protect users from protestware threats.

\paragraph{Responsibility in the OSS Community}
To minimize the moral dilemmas (and concrete implications) associated with protestware, maintainers should be encouraged to prioritize the needs of their users and contribute to the greater good of the community. Cultivating a sense of community and fostering strong relationships with stakeholders --- from end users to fellow developers --- can help maintainers understand the potential consequences of their actions and work together to develop ethical guidelines. Establishing communication channels and fora for discussion allows maintainers to express their concerns without resorting to protestware in the first instance, ensuring the well-being of both the community and its users. Future work should investigate how the community can cultivate a culture of care\footnote{Other frameworks of ethics, including care ethics, are also considered in current research into technology.} and moral responsibility and explore existing channels that allow maintainers to voice their protests in a non-jeopardizing way.

\paragraph{Safeguards against Protestware}
Future work could also explore the use of machine learning techniques, particularly natural language processing and code analysis algorithms, to detect early indicators of potential protestware in software repositories. Such techniques, in the same vein as, e.g., automated auditing for privacy and security issues,  could  better equip end users or developers to protect themselves from potential threats, ultimately enhancing the overall security and trustworthiness of the open-source ecosystem.

\paragraph{Enabling Healthy Channels for Protest}
It is important for maintainers to feel that their concerns are being addressed through appropriate channels. Good governance and a fair system of representation can alleviate the need for protestware by offering maintainers alternative avenues to voice their opinions and effect change within the OSS community. Future research should explore how existing channels for maintainers to voice their protests or grievances can be improved or expanded.

\paragraph{Education of Ethical Responsibility}
While this article is a good first step to foster awareness, ethics educational programs, focusing on ethical responsibility and social impact, can help maintainers better understand the potential consequences of using protestware and foster a more nuanced perspective on this issue. By providing guidance on ethical decision-making and highlighting alternative methods for expressing concerns, ethics education can play a vital role in promoting a more balanced and responsible approach to protestware within the OSS community. Future work should examine why some developers might feel far removed from ethical considerations and how educational programs can effectively address this issue.

\section{Navigating the Ethical Landscape of Protestware in OSS}

The OSS ecosystem, particularly with the emergence of protestware, poses diverse ethical challenges for various stakeholders. In the following, we discuss the implications for different stakeholders, emphasizing the importance of awareness and providing examples of ethical frameworks that could be applied in these situations.

\textit{Maintainers}, when confronted with protestware, play a pivotal role in shaping the OSS landscape. It is crucial that they clearly communicate the intended use and restrictions of their software in the documentation and/or terms of service while adhering to applicable laws and regulations. As an example, duty ethics highlights the importance of maintainers' moral obligations, such as transparency, honesty, and legal compliance.

\textit{Contributors}, as creators of public goods, must exercise responsibility, consideration, and ethics in their actions, particularly when engaging with OSS projects. It is essential for contributors to be aware that any project they contribute to could potentially transform into protestware, and they may have limited control over the project's direction despite their contributions. Consequentialist ethics, for instance, emphasize the need to assess the potential consequences of one's actions, urging contributors to be mindful of the projects they engage with, maximizing positive outcomes, and minimizing potential harm. Awareness of protestware's implications and the potential risks of contributing to projects that may adopt such a stance is key to making informed decisions.

\textit{Newcomers} to the OSS community should familiarize themselves with its principles, values, and the ethical implications of protestware. OSS relies on community-driven efforts, making it essential for newcomers to be respectful, collaborative, and helpful. Awareness of the presence of protestware and potential consequences is critical. As an example, the principle of autonomy in biomedical ethics underscores the importance of respecting the choices and values of other members of the community.

\textit{End users and industry} must be vigilant about the potential risks and benefits associated with using OSS tools, acknowledging that any project could potentially transform into protestware. A key ethical implication for end users and industry is understanding that an OSS project's maintainer might have different ethical priorities, which in extreme cases could lead to the creation of protestware. The risk is especially pronounced if end users or industry already rely on the software, and then it turns into protestware. Industry, in particular, must be cautious when using OSS in a professional setting, ensuring compliance with company policies and regulations. By assessing risks, addressing potential security vulnerabilities, and contributing back to projects when possible, the industry can apply ethical principles like beneficence and nonmaleficence from biomedical ethics, as an example, to promote positive outcomes and minimize harm to stakeholders. Awareness of the potential impact of protestware and the varying ethical priorities of project maintainers is crucial for responsible decision-making within the industry.

\textit{Educators} have a responsibility to discuss the implications and risks of OSS, including protestware and its legal and ethical consequences. Duty ethics, as an example, encourages educators to impart knowledge of moral obligations and duties as software creators and users. Guided by consequentialist ethics and biomedical principles, students should be encouraged to consider the long-term impacts and ethical implications of their work, particularly with regard to protestware. The promotion of awareness among students is a key educational goal.

\textit{Researchers} play a vital role in identifying and mitigating protestware risks by developing methods to analyze code for vulnerabilities or detect patterns in malicious software behavior. By investigating the pros and cons of protestware as a political protest tool, researchers can apply ethical frameworks such as consequentialist ethics and biomedical principles to balance potential benefits and harms while considering justice and fairness within the OSS community.

% In terms of protestware and OSS, it does not just affect big tech. There are lots of different stakeholders affected, let's look at guidelines for each of them:

% \ct{The following text is generic bla-bla for the moment, to be filled with concrete things based on the previous sections}

% \ct{Phrase these points as questions. E.g., are you able to live with yourself if ...}

% Maintainers: Clearly state the project's intended use and any restrictions on its use in the documentation and/or terms of service, and follow laws and regulations related to the use of the software. Be aware of the legal implications of misuse.

% \ct{Add users here}

% Contributors: Remember that contributing to open-source software means that you are taking part in creating a public good. It is important to be responsible, considerate, and ethical in your actions.

% Industry: If using open-source software in a professional setting, be aware of the company's policies related to using open-source software. Ensure that the usage is in compliance with the company's guidelines and regulations. Assess the risk of using any open-source software and, if necessary, have a plan in place to address potential security vulnerabilities that could be exploited for malicious use.

% Newcomers: Learn about the principles and values of open-source software and the open-source community before getting involved. Remember that open-source software is a community-driven effort, so it is important to be respectful, collaborative, and helpful in your interactions with others.

% Educators: Discuss the potential implications and risks of using open-source software, including the possibility of protestware and the potential legal and ethical implications of its use. Encourage students to think critically and ethically about the use of technology and its potential impact on society. Encourage students to think about the potential long-term impact of their work and to consider the ethical and moral implications of the software they create or use.

% Researchers: Develop methods for identifying and mitigating the risks of protestware, such as by analyzing code for potential vulnerabilities or identifying patterns in the behavior of malicious software. Investigate the pros and cons of protestware as a tool for political protest, including its effectiveness as a means of bringing attention to a cause and the potential negative consequences of its use, such as the harm caused to individuals and communities, the legal and ethical implications, the impact on civil rights and freedom of speech, and the damage it can cause to the reputation of the protest movement.

% \subsection{Guidelines for maintainers}
% Consequentialist ethics?

% \subsection{Guidelines for open-source contributors}

% \subsection{Guidelines for industry contributors}
% Beauchamp principles

% \subsection{Guidelines for open-source client developers}

% \subsection{Guidelines for industry client developers}

% \subsection{Guidelines for software users}

% \subsection{Guidelines for newcomers}

% \subsection{Guidelines for researchers}

% \subsection{Guidelines for educators}
% Segue? Linux Kernel people.

% \section{The Way Forward}

% % This is different to what AI ethics folks are looking at!
% Here are some ideas for the future

% \begin{itemize}
%     \item Understanding why developers might not feel any ethical responsibility
%     \item Does the existing governance and policy promote against protestware
%     \item what channels exist for maintainers to voice their protests in a healthy way.
%     \item how can users protect themselves against protestware threats.
%     \item how does the community breed a culture of moral responsibility.
%     \item what are early indicators or potential protestware.
% \end{itemize}

% From Marc Cheong to Everyone 01:08 PM
% [https://retractionwatch.com/2015/11/11/bmc-retracts-paper-of-scientist-who-banned-use-of-his-software-by-several-countries/](https://retractionwatch.com/2015/11/11/bmc-retracts-paper-of-scientist-who-banned-use-of-his-software-by-several-countries/) - another example of 'protest'-ing
% Basic arguments - CT perspectivism, MC 4 key (biomedical)-applied-ethics principles, RK when does it cease becoming open source - ethical responsibilities
% From Me to Everyone 01:11 PM
% [https://www.computer.org/csdl/magazine/so](https://www.computer.org/csdl/magazine/so)[https://www.computer.org/digital-library/magazines/so/cfp-open-source-software](https://www.computer.org/digital-library/magazines/so/cfp-open-source-software)[https://findanexpert.unimelb.edu.au/profile/862627-marc-cheong](https://findanexpert.unimelb.edu.au/profile/862627-marc-cheong)
% From Marc Cheong to Everyone 01:12 PM
% Explaining protestware - past, present
% From Me to Everyone 01:13 PM
% Right or Wrong
% %From Marc Cheong to Everyone 01:14 PM
% %Ethics behind it - ethical frameworks (maybe the Beauchamp & Childress 4?); or the common schools of "duty" ethics or "utilitarian" -- it's all relative
% Ethical implications of political activism
% From Me to Everyone 01:15 PM
% [https://opensource.org/](https://opensource.org/)
% From Marc Cheong to Everyone 01:16 PM
% Open source guidelines - definitions of OSS
% moving from OSS to non-OSS, awareness that things can be removed, no implicit contract/obligation
% From Marc Cheong to Everyone 01:27 PM
% 'framework' for evaluating exam proctoring software (quite in depth, but gives you a clear idea)

% [https://doi.org/10.1177/2053951716679679](https://doi.org/10.1177/2053951716679679)

% [https://link.springer.com/article/10.1007/s00146-021-01154-8](https://link.springer.com/article/10.1007/s00146-021-01154-8)


% \section*{Acknowledgements}
%  This work wis supported by the Japan Science and Technology Agency (JST) the Research New-Mode Mobility Accelerator Program (ReNewMAP) under the Strategic International Collaborative Research Program (SICORP) Grant Number JPMJSC2206 under the project titled "AI for Protestware
% Detection within Software Ecosystems".

\bibliographystyle{IEEEtran}
\bibliography{filtered_ref}

% \begin{IEEEbiography}[{\includegraphics[width=1in,height=1.25in,clip,keepaspectratio]{Marc.jpg}}]{Marc Cheong} a Senior Lecturer of Information Systems (Digital Ethics) at the University of Melbourne, Australia. His research interests are in the intersection between philosophy and technology: in particular, the ethics of social media technologies and the application of existentialist philosophy to contemporary social network structures. He is also an Honorary Burnet Institute Senior Fellow, working in social media and public health research. He was one of the early pioneers on Twitter research, with his PhD dissertation (ca. 2013) on Twitter and social behaviour. He frequently comments on topics related to digital ethics in the media, such as the NYT and The Age (Australia).
% \end{IEEEbiography}

% \begin{IEEEbiography}[{\includegraphics[width=1in,height=1.25in,clip,keepaspectratio]{Raula.jpg}}]{Raula Gaikovina Kula} is an assistant professor at the Nara Institute of Science and Technology (NAIST), Japan. He received his Ph.D degree from NAIST in 2013 and was a Research Assistant Professor at Osaka University. He is active in the Software Engineering community, serving the community as a PC member for premium SE venues (i.e., ICSE, ASE, ICSME, ESEM, MSR, and so on), some as organising committee and reviewer for journals (i.e., IEEE TSE, Springer EMSE, Elsevier IST and JSS). His current research interests include library dependencies and security in the software ecosystem, program analysis such as code clones and human aspects such as code reviews and coding proficiency. Find him at https://raux.github.io/ and @augaiko on twitter. Contact: raula-k@is.naist.jp.
% \end{IEEEbiography}


% \begin{IEEEbiography}[{\includegraphics[width=1in,height=1.25in,clip,keepaspectratio]{christoph.jpg}}]{Christoph Treude} is a Senior Lecturer in Software Engineering at the University of Melbourne, Australia. He earned his Ph.D. in Computer Science from the University of Victoria, Canada, in 2012. His research focuses on improving collaborative software engineering through empirical studies and innovative tools. Treude has authored over 150 scientific articles and received an ARC Discovery Early Career Research Award (2018-2020) and four best paper awards. He serves on the editorial board for the IEEE Transactions on Software Engineering and the Empirical Software Engineering journal. Treude has held key positions in various conferences, including general co-chair for ICSME 2020 and general chair for ICPC 2023 and TechDebt 2023. Contact: christoph.treude@unimelb.edu.au.
% \end{IEEEbiography}

\end{document}


