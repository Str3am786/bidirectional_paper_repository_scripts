\section{Benchmark Dataset}
\label{sec:dataset}
From the sources described above, we select documents from three critically endangered languages\footnote{UNESCO defines critically endangered languages as those where the youngest speakers are grandparents and older, and they speak the language partially and infrequently.} for annotation --- Ainu, Griko, and Yakkha. These languages were chosen in an effort to create a geographically, typologically, and orthographically diverse benchmark. We focus this initial study on scanned images of printed books as opposed to handwritten notes, which are a relatively more challenging domain for OCR. 

We manually transcribed the text corresponding to the endangered language content. The text corresponding to the translations is not manually transcribed. We also aligned the endangered language text to the OCR output on the translations, per the formulation in \autoref{sec:formulation}. We describe the annotated documents below and example images from our dataset are in \autoref{fig:dataset_example} (a), (b), (c).

\smallskip
\textbf{Ainu} is a severely endangered indigenous language from northern Japan, typically considered a language isolate. In our dataset, we use a book of Ainu epic poetry (\textit{yukara}), with the ``Kutune Shirka" yukara~\cite{kindaichi1931ainu} in Ainu transcribed in Latin script.\footnote{Some transcriptions of Ainu also use the Katakana script. See \citet{howell1951classification} for a discussion on Ainu folklore.} Each page in the book has a two-column structure --- the left column has the Ainu text, and the right has its Japanese translation already aligned at the line-level, removing the need for manual alignment (see \autoref{fig:dataset_example} (a)). The book has 338 pages in total. Given the effort involved in annotation, we transcribe the Ainu text from 33 pages, totaling 816 lines.

\smallskip
\textbf{Griko} is an endangered Greek dialect spoken in southern Italy. The language uses a combination of the Latin alphabet and the Greek alphabet as its writing system. The document we use is a book of Griko folk tales compiled by \citet{stomeo1980racconti}. The book is structured such that in each fold of two pages, the left page has Griko text, and the right page has the corresponding translation in Italian. Of the 175 pages in the book, we annotate the Griko text from 33 pages and manually align it at the sentence-level to the Italian translation. This results in 807 annotated Griko sentences.

\smallskip
\textbf{Yakkha} is an endangered Sino-Tibetan language spoken in Nepal. It uses the Devanagari writing system. We use scanned images of three children's books, each of which has a story written in Yakkha along with its translation in Nepali and English~\cite{yakkha-elar}. We manually transcribe the Yakkha text from all three books. We also align the Yakkha text to both the Nepali and the English OCR at the sentence level with the help of an existing Yakkha dictionary~\cite{Schackow_2015}. In total, we have 159 annotated Yakkha sentences.
