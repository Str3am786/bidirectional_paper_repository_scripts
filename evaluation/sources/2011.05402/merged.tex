%
% File emnlp2020.tex
%
%% Based on the style files for ACL 2020, which were
%% Based on the style files for ACL 2018, NAACL 2018/19, which were
%% Based on the style files for ACL-2015, with some improvements
%%  taken from the NAACL-2016 style
%% Based on the style files for ACL-2014, which were, in turn,
%% based on ACL-2013, ACL-2012, ACL-2011, ACL-2010, ACL-IJCNLP-2009,
%% EACL-2009, IJCNLP-2008...
%% Based on the style files for EACL 2006 by 
%%e.agirre@ehu.es or Sergi.Balari@uab.es
%% and that of ACL 08 by Joakim Nivre and Noah Smith

\documentclass[11pt,a4paper]{article}
\usepackage[hyperref]{emnlp2020}
\usepackage{times}
\usepackage{latexsym}
\usepackage{xcolor}
\usepackage{booktabs}
\usepackage{graphicx}
\usepackage{tipa}
\usepackage{float}
\usepackage{amssymb,amsmath,amsthm,enumitem}
\usepackage{subcaption}
\usepackage{fdsymbol}
\renewcommand{\UrlFont}{\ttfamily\small}

\usepackage{bm}


% This is not strictly necessary, and may be commented out,
% but it will improve the layout of the manuscript,
% and will typically save some space.
\usepackage{microtype}

\aclfinalcopy % Uncomment this line for the final submission
\def\aclpaperid{2493} %  Enter the acl Paper ID here


\DeclareMathOperator{\dec}{dec}
\DeclareMathOperator{\deca}{dec^1}
\DeclareMathOperator{\decb}{dec^2}
\DeclareMathOperator{\enca}{enc^1}
\DeclareMathOperator{\encb}{enc^2}
\DeclareMathOperator*{\softmax}{softmax}

\newcommand{\ascal}{\alpha}

\newcommand{\xvec}{\mathbf{x}}
\newcommand{\tvec}{\mathbf{t}}
\newcommand{\avec}{\mathbf{\alpha}}
\newcommand{\fvec}{\mathbf{f}}
\newcommand{\hvec}{\mathbf{h}}
\newcommand{\vvec}{\mathbf{v}}
\newcommand{\cvec}{\mathbf{c}}
\newcommand{\svec}{\mathbf{s}}
\newcommand{\yvec}{\mathbf{y}}
\newcommand{\step}[1]{_{#1}}
\newcommand{\fst}{^1}
\newcommand{\snd}{^2}

\usepackage{rotating}
\usepackage{multirow}

\usepackage{tikz}
\usetikzlibrary{shapes.geometric}
\usetikzlibrary{patterns}
\usepackage{pgfplots}
\pgfplotsset{compat=1.15}


%\setlength\titlebox{5cm}
% You can expand the titlebox if you need extra space
% to show all the authors. Please do not make the titlebox
% smaller than 5cm (the original size); we will check this
% in the camera-ready version and ask you to change it back.

\newcommand\BibTeX{B\textsc{ib}\TeX}

\renewcommand{\sectionautorefname}{Section}
\renewcommand{\subsectionautorefname}{Section}

\DeclareMathOperator*{\argmax}{arg\,max}
\renewcommand{\arraystretch}{1.2}

\newcommand{\ba}{`}

\newcommand\blfootnote[1]{%
  \begingroup
  \renewcommand\thefootnote{}\footnote{\noindent #1}%
  \addtocounter{footnote}{-1}%
  \endgroup
}

\definecolor{burntred}{HTML}{E66101}
\definecolor{burntblue}{HTML}{00A2FF}


\title{OCR Post Correction for Endangered Language Texts}

\author{Shruti Rijhwani,\textsuperscript{$1$} Antonios Anastasopoulos,\textsuperscript{$2,\dagger$} Graham Neubig\textsuperscript{$1$}  \\
  \textsuperscript{$1$}Language Technologies Institute, Carnegie Mellon University\\
  \textsuperscript{$2$}Department of Computer Science, George Mason University\\
  \texttt{\{srijhwan,gneubig\}@cs.cmu.edu},\quad\texttt{antonis@gmu.edu}}

\date{}

\begin{document}

\maketitle
\begin{abstract}
There is little to no data available to build natural language processing models for most endangered languages. However, textual data in these languages often exists in formats that are not machine-readable, such as paper books and scanned images. In this work, we address the task of extracting text from these resources. We create a benchmark dataset of transcriptions for scanned books in three critically endangered languages and present a systematic analysis of how general-purpose OCR tools are not robust to the data-scarce setting of endangered languages. We develop an OCR post-correction method tailored to ease training in this data-scarce setting, reducing the recognition error rate by 34\% on average across the three languages.\blfootnote{$\dagger$: Work done at Carnegie Mellon University.}\footnote{Code and data are available at \url{https://shrutirij.github.io/ocr-el/}.}
\end{abstract}

\section{Introduction}
\begin{figure}[t]
\centering
    \begin{subfigure}[t]{\columnwidth}
    \centering
      \caption{Ainu (left) -- Japanese (right)}
      \vspace{-0.5em}
      \fbox{\includegraphics[width=0.9\columnwidth]{images/ainu_ex.png}}
      \vspace{0.6em}
    \end{subfigure}
    \begin{subfigure}[t]{\columnwidth}
      \centering
      \caption{Griko (top) -- Italian (bottom)}
      \vspace{-0.5em}
      \frame{\includegraphics[width=0.93\columnwidth]{images/griko_example.pdf}}
      \vspace{0.7em}
    \end{subfigure}
    \begin{subfigure}[t]{\columnwidth}
      \centering
      \caption{Yakkha (top) -- Nepali (middle) -- English (bottom)}
      \vspace{-0.5em}
      \fbox{\includegraphics[width=0.9\columnwidth]{images/figure2c.pdf}}
      \vspace{0.6em}
    \end{subfigure}
    \footnotesize{(d) Handwritten Shangaji -- typed English glosses}
    \begin{tabular}{|@{\ \ }c@{\ \ }|}
    \hline
     \begin{subfigure}[t]{0.9\columnwidth}
      \centering
      \includegraphics[width=0.9\columnwidth]{images/sha_img.pdf}
      \includegraphics[width=0.5\columnwidth]{images/sha_text.pdf}
    \end{subfigure}\\
    \hline
    \end{tabular}
    \caption{Examples of scanned documents in endangered languages accompanied by translations from the same scanned book (a, b, c) or linguistic archive (d).}
    \label{fig:dataset_example}
    \vspace{-1.2em}
\end{figure}

Natural language processing (NLP) systems exist for a small fraction of the world's over 6,000 living languages, the primary reason being the lack of resources required to train and evaluate models. Technological advances are concentrated on languages that have readily available data, and most other languages are left behind~\cite{joshi2020state}. This is particularly notable in the case of endangered languages, i.e., languages that are in danger of becoming extinct due to dwindling numbers of native speakers and the younger generations shifting to using other languages. For most endangered languages, finding \emph{any} data at all is challenging.

In many cases, natural language text in these languages does exist. However, it is locked away in formats that are not machine-readable --- paper books, scanned images, and unstructured web pages. These include books from local publishing houses within the communities that speak endangered languages, such as educational or cultural materials. Additionally, linguists documenting these languages also create data such as word lists and interlinear glosses, often in the form of handwritten notes. Examples from such scanned documents are shown in~\autoref{fig:dataset_example}. Digitizing the textual data from these sources will not only enable NLP for endangered languages but also aid linguistic documentation, preservation, and accessibility efforts.

In this work, we create a benchmark dataset and propose a suite of methods to extract data from these resources, focusing on scanned images of paper books containing endangered language text. Typically, this sort of digitization requires an optical character recognition (OCR) system. However, the large amounts of textual data and transcribed images needed to train state-of-the-art OCR models from scratch are unavailable in the endangered language setting. Instead, we focus on \emph{post-correcting} the output of an off-the-shelf OCR tool that can handle a variety of scripts. We show that targeted methods for post-correction can significantly improve performance on endangered languages.

Although OCR post-correction is relatively well-studied, most existing methods rely on considerable resources in the target language, including a substantial amount of textual data to train a language model~\cite{schnober-etal-2016-still,dong-smith-2018-multi,8978127} or to create synthetic data~\cite{krishna-etal-2018-upcycle}. While readily available for high-resource languages, these resources are severely limited in endangered languages, preventing the direct application of existing post-correction methods in our setting. 

As an alternative, we present a method that compounds on previous models for OCR post-correction, making three improvements tailored to the data-scarce setting. First, we use a \textbf{multi-source model} to incorporate information from the high-resource translations that commonly appear in endangered language books. These translations are usually in the \textit{lingua franca} of the region (e.g., \autoref{fig:dataset_example} (a,b,c)) or the documentary linguist's primary language (e.g., \autoref{fig:dataset_example} (d) from \citet{shangaji-elar}). Next, we introduce \textbf{structural biases} to ease learning from small amounts of data. Finally, we add \textbf{pretraining methods} to utilize the little unannotated data that exists in endangered languages.

\medskip
\noindent
We summarize our main contributions as follows:
\begin{itemize}[leftmargin=*, itemsep=3pt]
    \item A benchmark dataset for OCR post-correction on three critically endangered languages: Ainu, Griko, and Yakkha.
    \item A systematic analysis of a general-purpose OCR system, demonstrating that it is not robust to the data-scarce setting of endangered languages.
    \item An OCR post-correction method that adapts the standard neural encoder-decoder framework to the highly under-resourced endangered language setting, reducing both the character error rate and the word error rate by 34\% over a state-of-the-art general-purpose OCR system.
\end{itemize}

\section{Problem Setting}
\label{sec:setting}
In this section, we first define the task of OCR post-correction and introduce how we incorporate translations into the correction model. Next, we discuss the sources from which we obtain scanned documents containing endangered language texts.

\subsection{Formulation}
\label{sec:formulation}
\paragraph{Optical Character Recognition} OCR tools are trained to find the best transcription corresponding to the text in an image. The system typically consists of a recognition model that produces candidate text sequences conditioned on the input image and a language model that determines the probability of these sequences in the target language. We use a general-purpose OCR system (detailed in \autoref{sec:analysis}) to produce a \emph{first pass transcription} of the endangered language text in the image. Let this be a sequence of characters $\boldsymbol{x} = [x_1, \dots, x_N]$.

\paragraph{OCR post-correction} The goal of post-correction is to reduce recognition errors in the first pass transcription --- often caused by low quality scanning, physical deterioration of the paper book, or diverse layouts and typefaces~\cite{dong-smith-2018-multi}. The focus of our work is on using post-correction to counterbalance the lack of OCR training data in the target endangered languages. The correction model takes $\boldsymbol{x}$ as input and produces the \emph{final transcription} of the endangered language document, a sequence of characters $\boldsymbol{y} = [y_1, \dots , y_K]$. 
$$\boldsymbol{y} = \argmax_{\boldsymbol{y'}} p_\text{corr}(\boldsymbol{y'}|\boldsymbol{x})$$

\noindent
\textbf{Incorporating translations}\quad We use information from high-resource translations of the endangered language text. These translations are commonly found within the same paper book or linguistic archive (e.g., \autoref{fig:dataset_example}). We use an existing OCR system to obtain a transcription of the scanned translation, a sequence of characters $\boldsymbol{t} = [t_1, \dots, t_M]$. This is used to condition the model:
$$\boldsymbol{y} = \argmax_{\boldsymbol{y'}} p_\text{corr}(\boldsymbol{y'}|\boldsymbol{x}, \boldsymbol{t})$$

\subsection{Endangered Language Documents}
We explore online archives to determine how many scanned documents in endangered languages exist as potential sources for data extraction (as of this writing, October 2020).

The Internet Archive,\footnote{\url{https://archive.org/}} a general-purpose archive of web content, has scanned books labeled with the language of their content. We find 11,674 books labeled with languages classified as \ba\ba endangered'' by UNESCO.
Additionally, we find that endangered language linguistic archives contain thousands of documents in PDF format --- the Archive of the Indigenous Languages of Latin America (AILLA)\footnote{\url{https://ailla.utexas.org}} contains $\approx$10,000 such documents and the Endangered Languages Archive (ELAR)\footnote{\url{https://elar.soas.ac.uk/}} has $\approx$7,000. 

\medskip
\noindent
\textbf{How common are translations?} As described in the introduction, endangered language documents often contain a translation into another (usually high-resource) language. While it is difficult to estimate the number of documents with translations, multilingual documents represent the majority in the archives we examined; AILLA contains 4,383 PDFs with bilingual text and 1,246 PDFs with trilingual text, while ELAR contains $\approx$5,000 multilingual documents. The structure of translations in these documents is varied, from dictionaries and interlinear glosses to scanned multilingual books.

\section{Benchmark Dataset}
\label{sec:dataset}
From the sources described above, we select documents from three critically endangered languages\footnote{UNESCO defines critically endangered languages as those where the youngest speakers are grandparents and older, and they speak the language partially and infrequently.} for annotation --- Ainu, Griko, and Yakkha. These languages were chosen in an effort to create a geographically, typologically, and orthographically diverse benchmark. We focus this initial study on scanned images of printed books as opposed to handwritten notes, which are a relatively more challenging domain for OCR. 

We manually transcribed the text corresponding to the endangered language content. The text corresponding to the translations is not manually transcribed. We also aligned the endangered language text to the OCR output on the translations, per the formulation in \autoref{sec:formulation}. We describe the annotated documents below and example images from our dataset are in \autoref{fig:dataset_example} (a), (b), (c).

\smallskip
\textbf{Ainu} is a severely endangered indigenous language from northern Japan, typically considered a language isolate. In our dataset, we use a book of Ainu epic poetry (\textit{yukara}), with the ``Kutune Shirka" yukara~\cite{kindaichi1931ainu} in Ainu transcribed in Latin script.\footnote{Some transcriptions of Ainu also use the Katakana script. See \citet{howell1951classification} for a discussion on Ainu folklore.} Each page in the book has a two-column structure --- the left column has the Ainu text, and the right has its Japanese translation already aligned at the line-level, removing the need for manual alignment (see \autoref{fig:dataset_example} (a)). The book has 338 pages in total. Given the effort involved in annotation, we transcribe the Ainu text from 33 pages, totaling 816 lines.

\smallskip
\textbf{Griko} is an endangered Greek dialect spoken in southern Italy. The language uses a combination of the Latin alphabet and the Greek alphabet as its writing system. The document we use is a book of Griko folk tales compiled by \citet{stomeo1980racconti}. The book is structured such that in each fold of two pages, the left page has Griko text, and the right page has the corresponding translation in Italian. Of the 175 pages in the book, we annotate the Griko text from 33 pages and manually align it at the sentence-level to the Italian translation. This results in 807 annotated Griko sentences.

\smallskip
\textbf{Yakkha} is an endangered Sino-Tibetan language spoken in Nepal. It uses the Devanagari writing system. We use scanned images of three children's books, each of which has a story written in Yakkha along with its translation in Nepali and English~\cite{yakkha-elar}. We manually transcribe the Yakkha text from all three books. We also align the Yakkha text to both the Nepali and the English OCR at the sentence level with the help of an existing Yakkha dictionary~\cite{Schackow_2015}. In total, we have 159 annotated Yakkha sentences.

\section{OCR Systems: Promises and Pitfalls}
\label{sec:analysis}
As briefly alluded to in the introduction, training an OCR model for each endangered language is challenging, given the limited available data. 
Instead, we use the general-purpose OCR system from the Google Vision AI toolkit\footnote{\url{https://cloud.google.com/vision}} to get the first pass OCR transcription on our data.

The Google Vision OCR system~\cite{fujii2017sequence,ingle2019scalable} is highly performant and supports 60 major languages in 29 scripts. It can transcribe a wide range of higher resource languages with high accuracy, ideal for our proposed method of incorporating high-resource translations into the post-correction model. Moreover, it is particularly well-suited to our task because it provides script-specific OCR models in addition to language-specific ones. Per-script models are more robust to unknown languages because they are trained on data from multiple languages and can act as a general character recognizer without relying on a single language's model. Since most endangered languages adopt standard scripts (often from the region's dominant language) as their writing systems, the per-script recognition models can provide a stable starting point for post-correction.

The metrics we use for evaluating performance are character error rate (CER) and word error rate (WER), representing the ratio of erroneous characters or words in the OCR prediction to the total number in the annotated transcription. More details are in \autoref{sec:experiments}. The CER and WER using the Google Vision OCR on our dataset are in \autoref{tab:google_metrics}.

\subsection{OCR Performance}
\begin{table}[tb]
    \centering
    \small
    \begin{tabular}{lcrr}
    \toprule
    Language && CER & WER \\
    \midrule
    Ainu && 1.34 & 6.27 \\
    Griko && 3.27 & 15.63 \\
    Yakkha && 8.90 & 31.64 \\
    \bottomrule
    \end{tabular}
    \caption{Character error rate and word error rate with the Google Vision OCR system on our dataset.}
    \label{tab:google_metrics}
\end{table}
Across the three languages, the error rates indicate that we have a first pass transcription that is of reasonable quality, giving our post-correction method a reliable starting point. We note the particularly low CER for the Ainu data, reflecting previous work that has evaluated the Google Vision system to have strong performance on typed Latin script documents \cite{fujii2017sequence}. However, there remains considerable room for improvement in both CER and WER for all three languages.

Next, we look at the edit distance between the predicted and the gold transcriptions, in terms of insertion, deletion, and replacement operations. Replacement accounts for over 84\% of the errors in the Griko and Ainu datasets, and 55\% overall. This pattern is expected in the OCR task, as the recognition model uses the image to make predictions and is more likely to confuse a character's shape for another than to hallucinate or erase pixels. However, we observe that the errors in the Yakkha dataset do not follow this pattern. Instead, 87\% of the errors for Yakkha occur because of deleted characters.

\subsection{Types of Errors}

\begin{figure}[t]
    \centering
    \small
    \begin{tabular}{ccc}
        \frame{\includegraphics[width=0.4\columnwidth]{images/errors1a.pdf}} & \raisebox{0.7em}{$\xrightarrow{\mathrm{OCR}}$} & \raisebox{0.7em}{\large e\textcolor{burntred}{\textbf{x}}i i ka\textcolor{burntred}{\textbf{dd}}in\`ara} \\[.1cm]
        \frame{\includegraphics[width=0.3\columnwidth]{images/errors2a.pdf}} &
        \raisebox{0.6em}{$\xrightarrow{\mathrm{OCR}}$} &
        \includegraphics[width=0.3\columnwidth]{images/error2b.pdf} \\
    \end{tabular}
    \caption{Examples of errors in Griko (top) and Yakkha (bottom) when using the Google Vision OCR.}
    \label{fig:ocr_errors}
\end{figure}

To better understand the challenges posed by the endangered language setting, we manually inspect all the errors made by the OCR system. While some errors are commonly seen in the OCR task, such as misidentified punctuation or incorrect word boundaries, 85\% of the total errors occur due to specific characteristics of endangered languages that general-purpose OCR systems do not account for. Broadly, they can be categorized into two types, examples of which are shown in \autoref{fig:ocr_errors}:
\begin{itemize}[leftmargin=*, itemsep=2pt, topsep=6pt]
    \item\textbf{Mixed scripts}\quad The existing scripts that most endangered languages adopt as writing systems are often not ideal for comprehensively representing the language. For example, the Devanagari script does not have a grapheme for the glottal stop --- as a solution, printed texts in the Yakkha language use the IPA symbol \ba\textipa{\textglotstop}'~\cite{Schackow_2015}. Similarly, both Greek and Latin characters are used to write Griko. The Google Vision OCR is trained to detect script at the line-level and is not equipped to handle multiple scripts within a single word. As seen in \autoref{fig:ocr_errors}, the system does not recognize the Greek character $\bm{\chi}$ in Griko and the IPA symbol \textbf{\textipa{\textglotstop}} in Yakkha. Mixed scripts cause 11\% of the OCR errors.
    \item\textbf{Uncommon characters and diacritics}\quad Endangered languages often use graphemes and diacritics that are part of the standard script but are not commonly seen in high-resource languages. Since these are likely rare in the OCR system's training data, they are frequently misidentified, accounting for 74\% of the errors. In \autoref{fig:ocr_errors}, we see that the OCR system substitutes the uncommon diacritic \textbf{\d{d}} in Griko. The system also deletes the Yakkha character {\raisebox{-2.65pt}{\includegraphics[height=9.5pt]{images/dev_char.pdf}}}, which is a \ba half form' alphabet that is infrequent in several other Devanagari script languages (such as Hindi).
    
\end{itemize}
\section{OCR Post-Correction Model}
\label{sec:model}
In this section, we describe our proposed OCR post-correction model. The base architecture of the model is a multi-source sequence-to-sequence framework~\cite{zoph-knight-2016-multi,libovicky-helcl-2017-attention} that uses an LSTM encoder-decoder model with attention~\cite{bahdanau2015neural}. We propose improvements to training and modeling for the multi-source architecture, specifically tailored to ease learning in data-scarce settings.

\subsection{Multi-source Architecture}
\label{sec:base}
{%
\setlength{\fboxsep}{0pt}%
\setlength{\fboxrule}{1pt}%
}%
\begin{figure}[tb]
\tikzset{seq/.style={draw=none,fill=gray!20}}
\tikzset{layer/.style={->,thick}}
\tikzset{label/.style={anchor=west,font={\footnotesize}}}
\tikzset{seqlabel/.style={font={\small}}}
\newcommand{\encoder}[3]{
\draw[seq] (-1.25,-0.25) rectangle (1.25,0.25);
\node[seqlabel] at (0,0) 
%{$\xmat$};
{$#3\step{1}#1 \ldots #3\step{#2}#1$};
\draw[layer] (0,0.3) -- (0,0.7);
\node[seqlabel] at (0,0.5) [label] {encoder};
\draw[seq] (-1.25,0.75) rectangle (1.25,1.25);
\node[seqlabel] at (0,1) %{$\hmat$}; 
{$\hvec\step{1}#1 \ldots \hvec\step{#2}#1$};
}
\newcommand{\ocrencoder}[5]{
\node[inner sep=0pt] at (0,-1.7)
    {\small #5};
\node[inner sep=0pt] at (0,-1.2)
    {\setlength{\fboxsep}{.005\textwidth}%
    \fbox{\includegraphics[width=.145\textwidth]{#4}}};
\draw[layer] (0,-0.9) -- (0,-0.3);
\node[seqlabel] at (0,-0.6) [label] {\textsc{ocr}};
\draw[seq] (-1.25,-0.25) rectangle (1.25,0.25);
\node[seqlabel] at (0,0) 
%{$\xmat$};
{$#3\step{1} \ldots #3\step{#2}$};
\draw[layer] (0,0.3) -- (0,0.9);
\node[seqlabel] at (0,0.6) [label] {encoder};
\draw[seq] (-1.25,0.95) rectangle (1.25,1.45);
\node[seqlabel] at (0,1.2) %{$\hmat$}; 
{$\hvec\step{1}#1 \ldots \hvec\step{#2}#1$};
}
\newcommand{\decoder}[1]{
\draw[seq] (-1,2.05) rectangle (1,2.55);
\node[seqlabel] at (0,2.3) %{$\cmat#1$}; 
{$\cvec\step{1}#1 \ldots \cvec\step{K#1}#1$};
\draw[layer] (0,2.6) -- (0,3.2);
\node[seqlabel] at (0,2.9) [label] {decoder};
\draw[seq] (-1,3.25) rectangle (1,3.75);
\node[seqlabel] at (0,3.5) %{$\smat#1$}; 
{$\svec\step{1}#1 \ldots \svec\step{K#1}#1$};
\draw[layer] (0,3.8) -- (0,4.4);
\node[seqlabel] at (0,4.1) [label] {softmax};
\draw[seq] (-1,4.45) rectangle (1,5.05);
\node[seqlabel] at (0,4.75) %{$P(\ymat#1)$}; 
{$P(\yvec\step{1}#1 \ldots \yvec\step{K#1}#1)$};
}
\begin{center}
\resizebox{0.7\hsize}{!}{
\begin{tabular}{c}
\begin{tikzpicture}
\begin{scope}[xshift=-1.4cm]
% \encoder{^x}{N}{\xvec}
\ocrencoder{^x}{N}{\xvec}{images/ainu_frame.png}{Ainu}
\end{scope}
\begin{scope}[xshift=1.4cm]
% \encoder{^t}{M}{\tvec}
\ocrencoder{^t}{M}{\tvec}{images/japanese_frame.png}{Japanese}
\end{scope}
\draw[layer] (-1.4,1.5) -- (0,2.0);
\draw[layer] (1.4,1.5) -- (0,2.0);
\node at (-1,1.85) [label,anchor=east] {attention};
\node at (1,1.85) [label] {attention};
\decoder{}
\end{tikzpicture}
\end{tabular}%
}
\end{center}
\caption{The proposed multi-source architecture with the encoder for an endangered language segment (left) and an encoder for the translated segment (right). The input to the encoders is the first pass OCR over the scanned images of each segment. For example, the OCR on the scanned images of some Ainu text (left) and its Japanese translation (right).}
\label{fig:multisourcemodels}
\end{figure}
Our post-correction formulation takes as input the first pass OCR of the endangered language segment $\boldsymbol{x}$ and the OCR of the translated segment $\boldsymbol{t}$, to predict an error-free endangered language text $\boldsymbol{y}$. The model architecture is shown in \autoref{fig:multisourcemodels}.

The model consists of two encoders --- one that encodes $\boldsymbol{x}$ and one that encodes $\boldsymbol{t}$. Each encoder is a character-level bidirectional LSTM~\cite{hochreiter1997long} and transforms the input sequence of characters to a sequence of hidden state vectors: $\mathbf{h}^x$ for the endangered language text and $\mathbf{h}^t$ for the translation.

The model uses an attention mechanism during the decoding process to use information from the encoder hidden states. We compute the attention weights over each of the two encoders independently. At the decoding time step $k$:
\begin{align}
e^x_{k,i}=\mathbf{v}^x \tanh\left(\mathbf{W}_1^x \mathbf{s}_{k-1} + \mathbf{W}_2^x \mathbf{h}^x_i\right)
\label{eq:attn}
\end{align}
$$\boldsymbol{\alpha}_k^x = \mathrm{softmax}\left(\mathbf{e}_k^x\right)$$
$$\mathbf{c}^x_k = \left[\Sigma_i \alpha^x_{k,i} \mathbf{h}^x_i\right]$$
\noindent
where $\mathbf{s}_{k-1}$ is the decoder state of the previous time step and $\mathbf{v}^x$, $\mathbf{W}_1^x$ and $\mathbf{W}_2^x$ are trainable parameters. The encoder hidden states $\mathbf{h}^x$ are weighted by the attention distribution $\boldsymbol{\alpha}^x_k$ to produce the context vector $\mathbf{c}^x_k$. We follow a similar procedure for the second encoder to produce $\mathbf{c}^t_k$.
We concatenate the context vectors to combine attention from both sources~\cite{zoph-knight-2016-multi}:
$$\mathbf{c}_k=\left[\mathbf{c}_k^x;\mathbf{c}_k^t\right]$$
$\mathbf{c}_k$ is used by the decoder LSTM to compute the next hidden state $\mathbf{s}_k$ and subsequently, the probability distribution for predicting the next character $\mathbf{y}_k$ of the target sequence $\boldsymbol{y}$:
\begin{align} 
\mathbf{s}_k &= \mathrm{lstm}\left(\mathbf{s}_{k-1}, \mathbf{c}_k, \mathbf{y}_{k-1}\right)\\
P\left(\mathbf{y}_k\right) &= \mathrm{softmax}\left(\mathbf{W}\mathbf{s}_k + \mathbf{b}\right)
\label{eq:decoder}
\end{align}

\paragraph{Training and Inference} The model is trained for each language with the cross-entropy loss ($\mathcal{L}_\mathrm{ce}$) on the small amount of transcribed data we have. Beam search is used at inference time.


\subsection{Model and Training Improvements}
\label{sec:recipe1}

With the minimal annotated data we have, it is challenging for the neural network to learn a good distribution over the target characters. We propose a set of adaptations to the base architecture that improves the post-correction performance without additional annotation. The adaptations are based on characteristics of the OCR task itself and the performance of the upstream OCR tool (\autoref{sec:analysis}).

\paragraph{Diagonal attention loss} As seen in \autoref{sec:analysis}, substitution errors are more frequent in the OCR task than insertions or deletions; consequently, we expect the source and target to have similar lengths. Moreover, post-correction is a monotonic sequence-to-sequence task, and reordering rarely occurs~\cite{schnober-etal-2016-still}. 
Hence, we expect attention weights to be higher at characters close to the diagonal for the endangered language encoder.

We modify the model such that all the elements in the attention vector that are not within $j$ steps (we use $j=3$) of the current time step $k$ are added to the training loss, thereby encouraging elements away from the diagonal to have lower values. The diagonal loss summed over all time steps for a training instance, where $N$ is the length of $\boldsymbol{x}$, is:
$$\mathcal{L}_\mathrm{diag} = \sum_k \left(\sum_{i=1}^{k-j} \alpha^x_{k,i} + \sum_{i=k+j}^N \alpha^x_{k,i}\right)$$

\paragraph{Copy mechanism} \autoref{tab:google_metrics} indicates that the first pass OCR predicts a majority of the characters accurately. In this scenario, enabling the model to directly copy characters from the first pass OCR rather than generate a character at each time step might lead to better performance~\cite{gu-etal-2016-incorporating}.

We incorporate the copy mechanism proposed in~\citet{see-etal-2017-get}. The mechanism computes a \ba\ba generation probability'' at each time step $k$, which is used to choose between \emph{generating} a character (\autoref{eq:decoder}) or \emph{copying} a character from the source text by sampling from the attention distribution $\boldsymbol{\alpha}_k^x$.

\paragraph{Coverage} Given the monotonicity of the post-correction task, the model should not attend to the same character repeatedly. However, repetition is a common problem for neural encoder-decoder models~\cite{mi-etal-2016-coverage,tu-etal-2016-modeling}. To account for this problem, we adapt the coverage mechanism from~\citet{see-etal-2017-get}, which keeps track of the attention distribution over past time steps in a coverage vector. For time step $k$, the coverage vector would be $\mathbf{g}_k = \sum_{k'=0}^{k-1} \boldsymbol{\alpha}^x_{k'}$. 

$\mathbf{g}_k$ is used as an extra input to the attention mechanism, ensuring that future attention decisions take the weights from previous time steps into account. \autoref{eq:attn}, with learnable parameter $\mathbf{w}_g$, becomes:
$$e^x_{k,i}=\mathbf{v}^x \tanh\left(\mathbf{W}_1^x \mathbf{s}_{k-1} + \mathbf{W}_2^x \mathbf{h}^x_i + \mathbf{w}_g g_{k,i}\right)$$
$\mathbf{g}_k$ is also used to penalize attending to the same locations repeatedly with a coverage loss. The coverage loss summed over all time steps $k$ is:
$$\mathcal{L}_\mathrm{cov} = \sum_k \sum_{i=1}^n \min\left(\alpha_{k,i}^x, g_{k,i}\right)$$
Therefore, with our model adaptations, the loss for a single training instance:
\begin{align}
\mathcal{L} = \mathcal{L}_\mathrm{ce} + \mathcal{L}_\mathrm{diag} + \mathcal{L}_\mathrm{cov}
\label{eq:loss}
\end{align}

\subsection{Utilizing Untranscribed Data}
\label{sec:recipe2}

As discussed in \autoref{sec:dataset}, given the effort involved, we transcribe only a subset of the pages in each scanned book.
Nonetheless, we leverage the remaining unannotated pages for pretraining our model. We use the upstream OCR tool to get a first pass transcription on all the unannotated pages.

We then create \ba\ba pseudo-target'' transcriptions for the endangered language text as described below:
\begin{itemize}
    \item \textbf{Denoising rules}\quad Using a small fraction of the available annotated pages, we compute the edit distance operations between the first pass OCR and the gold transcription. The operations of each type (insertion, deletion, and replacement) are counted for each character and divided by the number of times that character appears in the first pass OCR. This forms a probability of how often the operation should be applied to that specific character.
    
    We use these probabilities to form rules, such as $p(\text{replace d with \d{d}})\!=\!0.4$ for Griko and $p(\text{replace ? with \textipa{\textglotstop}})\!=\!0.7$ for Yakkha. These rules are applied to remove errors from, or \ba\ba denoise'', the first pass OCR transcription.
    \item \textbf{Sentence alignment}\quad We use Yet Another Sentence Aligner~\cite{yasa-1336} for unsupervised alignment of the endangered language and translation on the unannotated pages. 
\end{itemize}
Given the aligned first pass OCR for the endangered language text and the translation along with the pseudo-target text, $\boldsymbol{x}$, $\boldsymbol{t}$ and $\boldsymbol{\hat{y}}$ respectively, the pretraining steps, in order, are:

\begin{itemize}
    \item \textbf{Pretraining the encoders}\quad We pretrain both the forward and backward LSTMs of each encoder with a character-level language model objective: the endangered language encoder on $\boldsymbol{x}$ and the translation encoder on $\boldsymbol{t}$.
    \item \textbf{Pretraining the decoder}\quad The decoder is pretrained on the pseudo-target $\boldsymbol{\hat{y}}$ with a character-level language model objective.
    \item \textbf{Pretraining the seq-to-seq model}\quad The model is pretrained with $\boldsymbol{x}$ and $\boldsymbol{t}$ as the sources and the pseudo-target $\boldsymbol{\hat{y}}$ as the target transcription, using the post-correction loss function~$\mathcal{L}$ as defined in \autoref{eq:loss}.
\end{itemize}

\section{Experiments}
\label{sec:experiments}
\begin{table*}[tb]
    \centering
    \small
    \begin{tabular}{l|r@{\ \ }rr@{\ \ }rr@{\ \ }r|r@{\ \ }rr@{\ \ }rr@{\ \ }r}
    \toprule
        & \multicolumn{6}{c|}{Character Error Rate} &\multicolumn{6}{c}{Word Error Rate} \\
        & \multicolumn{2}{c}{Ainu} & \multicolumn{2}{c}{Griko} & \multicolumn{2}{c|}{Yakkha} & \multicolumn{2}{c}{Ainu} & \multicolumn{2}{c}{Griko} & \multicolumn{2}{c}{Yakkha} \\[-0.3em]
        Model & \small Multi & \small Single & \small Multi & \small Single & \small Multi & \small Single & \small Multi & \small Single & \small Multi & \small Single & \small Multi & \small Single\\
        \midrule
        \textsc{Fp-Ocr} & \multicolumn{1}{c}{--} & $1.34$ & \multicolumn{1}{c}{--} & $3.27$ & \multicolumn{1}{c}{--} & $8.90$ & \multicolumn{1}{c}{--} & $6.27$ & \multicolumn{1}{c}{--} & $15.63$ & \multicolumn{1}{c}{--} & $31.64$ \\
        \textsc{Base} & $1.56$ & $1.41$ & $6.78$ & $5.95$ & $70.39$ & $71.71$ & $8.56$ & $7.88$ & $15.13$ & $13.67$ & $98.15$ & $99.10$  \\
        \textsc{Copy} & $2.04$ & $1.99$ & $2.54$ & $2.28$ & $14.77$ & $12.30$ & $9.48$ & $8.57$ & $9.33$ & $8.90$ & $30.36$ & $27.81$ \\
        \textsc{Ours} & $0.92$ & $\boldsymbol{0.80}$ & $\boldsymbol{1.66}$ & $1.70$ & $\boldsymbol{7.75}$ & $8.44$ & $5.75$ & $\boldsymbol{5.19}$ & $\boldsymbol{7.46}$ & $7.51$ & $\boldsymbol{20.95}$ & $21.33$ \\
    \bottomrule
    \end{tabular}
    \caption{Our method improves performance over all baselines (10-fold cross-validation averaged over five randomly seeded runs). We present multi- and single-source variants and \textbf{highlight} the best model for each language.}
    \label{tab:cer}
\end{table*}

This section discusses our experimental setup and the post-correction performance on the three endangered languages on our dataset.

\subsection{Experimental Setup}
\smallskip
\paragraph{Data Splits}
We perform 10-fold cross-validation for all experimental settings because of the small size of the datasets. For each language, we divide the transcribed data into 11 segments --- we use one segment for creating the \emph{denoising rules} described in the previous section and the remaining ten as the folds for cross-validation. In each cross-validation fold, eight segments are used for training, one for validation and one for testing.

We divide the dataset at the page-level for the Ainu and Griko documents, resulting in 11 segments of three pages each. For the Yakkha documents, we divide at the paragraph-level, due to the small size of the dataset. We have 33 paragraphs across the three books in our dataset, resulting in 11 segments that contain three paragraphs each. The multi-source results for Yakkha reported in \autoref{tab:cer} use the English translations. Results with Nepali are similar and are included in \autoref{sec:appendix}.

\paragraph{Metrics}
We use two metrics for evaluating our systems: character error rate (CER) and word error rate (WER). Both metrics are based on edit distance and are standard for evaluating OCR and OCR post-correction~\cite{berg-kirkpatrick-etal-2013-unsupervised,schulz-kuhn-2017-multi}. 
CER is the edit distance between the predicted and the gold transcriptions of the document, divided by the total number of characters in the gold transcription. WER is similar but is calculated at the word level.

\paragraph{Methods}
In our experiments, we compare the performance of our proposed method with the first pass OCR and with two systems from recent work in OCR post-correction. All the post-correction methods have two variants -- the single-source model with only the endangered language encoder and the multi-source model that additionally uses the high-resource translation encoder.
\begin{itemize}
    \item \textsc{Fp-Ocr}: The first pass transcription obtained from the Google Vision OCR system.
    \item \textsc{Base}: This system is the base sequence-to-sequence architecture described in \autoref{sec:base}. Both the single-source and multi-source variants of this system are used for English OCR post-correction in~\citet{dong-smith-2018-multi}. 
    \item \textsc{Copy}: This system is the base architecture with a copy mechanism as described in \autoref{sec:recipe1}. The single-source variant of this model is used for OCR post-correction on Romanized Sanskrit in \citet{krishna-etal-2018-upcycle}.\footnote{Although~\citet{krishna-etal-2018-upcycle} use BPE tokenization, preliminary experiments showed that character-level models result in much better performance on our dataset, likely due to the limited data available for training the BPE model.}
    \item \textsc{Ours}: The model with all the adaptations proposed in \autoref{sec:recipe1} and \autoref{sec:recipe2}.
\end{itemize}

\paragraph{Implementation} The post-correction models are implemented using the DyNet neural network toolkit~\cite{dynet}, and all reported results are the average of five training runs with different random seeds. We assume knowledge of the entire alphabet of the endangered language for all the methods, which is straightforward to obtain for most languages. The decoder's vocabulary contains all these characters, irrespective of their presence in the training data, with corresponding randomly-initialized character embeddings.

\subsection{Main Results}
\label{sec:results}
\renewcommand{\arraystretch}{1.0}
\begin{figure*}[tb]
    \centering
    \small
    \begin{tabular}{lcc}
    & \multicolumn{2}{c}{Errors \textit{fixed} by post-correction}\\[.1cm]
        & (a) Griko & (b) Yakkha  \\
        \raisebox{0.7em}{[Image]} & \frame{\includegraphics[width=0.4\columnwidth]{images/errors1a.pdf}} & \frame{\includegraphics[width=0.3\columnwidth]{images/errors2a.pdf}} \\
        & \multicolumn{2}{c}{$\big\downarrow$ \hspace{3cm} $\big\downarrow$}\\
        \raisebox{0.35em}{[First pass OCR]} & \raisebox{0.3em}{\large e\textcolor{burntred}{\textbf{x}}i i ka\textcolor{burntred}{\textbf{dd}}in\`ara} &
        \includegraphics[width=0.25\columnwidth]{images/error2b.pdf} \\
        & \multicolumn{2}{c}{$\big\downarrow$ \hspace{3cm} $\big\downarrow$}\\
        \raisebox{0.35em}{[Post-corrected]} & \raisebox{0.35em}{\large e\textcolor{burntblue}{$\bm{\chi}$}i i ka\textbf{\textcolor{burntblue}{\d{d}\d{d}}}in\`ara} &
        \includegraphics[width=0.28\columnwidth]{images/error2c.pdf} \\
    \end{tabular}
    \qquad
    \begin{tabular}{cc}
        \multicolumn{2}{c}{Errors \textit{introduced} by post-correction}\\[.1cm]
        (c) Griko & (d) Yakkha  \\
        \frame{\includegraphics[width=0.25\columnwidth]{images/errors3a.pdf}} & \frame{\includegraphics[width=0.35\columnwidth]{images/errors4a.pdf}} \\
        \multicolumn{2}{c}{$\big\downarrow$ \hspace{2.5cm} $\big\downarrow$}\\
        \raisebox{0.35 em}{\large{\`{e} ffacilo}} &
        \includegraphics[width=0.3\columnwidth]{images/errors4b.pdf} \\
        \multicolumn{2}{c}{$\big\downarrow$ \hspace{2.5cm} $\big\downarrow$}\\
        \raisebox{0.35 em}{\large{\`{e} ffa\textcolor{burntred}{\textbf{\'{c}}}ilo}} &
        \includegraphics[width=0.32\columnwidth]{images/errors4c.pdf}
    \end{tabular}
    \caption{Our model fixes many mixed script and uncommon diacritics errors such as (a) and (b). In rare cases, it ``over-corrects" the first pass OCR transcription, introducing errors such as (c) and (d).}
    \label{fig:error_examples}
\end{figure*}

\begin{figure}[t]
    \definecolor{graphblue}{HTML}{A6BDDB}
\pgfplotstableread[row sep=\\,col sep=&]{
det & Ainu & Griko & Yakkha \\
all & 5.19 & 7.46 & 24.29 \\
diag & 5.49 & 8.06 & 22.73 \\
copy & 6.56 & 8.66 & 37.83 \\
coverage & 5.60 & 10.19 & 26.71 \\
dec & 6.86 & 7.87 & 20.95 \\
enc & 6.70 & 9.47 & 28.41 \\
seq2seq & 5.65 & 9.43 & 27.68 \\
}\data
\def\mystrut{\vphantom{hp}}

%\vspace{-1em}
\begin{tikzpicture}[trim left=-1.6cm,trim right=0cm]
    \begin{axis}[
            xbar,
            every axis plot post/.style={/pgf/number format/fixed},
            bar width=.23cm,
            width=4cm,
            height=4cm,
            ymajorgrids=false,
            yminorgrids=false,
            xmajorgrids=false,
            %every axis legend/.code={\let\addlegendentry\relax},
            %legend={Wikipedia Size (in million articles)},
            legend style={draw=none,at={(0.2,0.9)},anchor=west},
            symbolic y coords={all,diag,copy,coverage,dec,enc,seq2seq},
            ytick={all,diag,copy,coverage,dec,enc,seq2seq},
            yticklabels={all,-diag,-copy,-coverage,-pretr. dec,-pretr. enc,-pretr. s2s},
            %ytick={all,-diag loss,-copy,-coverage, -pretrain dec,-pretrain enc,-pretrain seq2seq},
            every y tick label/.append style={font=\small\mystrut},
            every x tick label/.append style={font=\small\mystrut},
            tick pos=left,
            %hide y axis,
            axis x line*=bottom,
            axis y line*=left,
            nodes near coords,
            %nodes near coords align={vertical},
            every node near coord/.append style={font=\small,color=black},
            %nodes near coords style={},
            title={\small Ainu},
            title style={yshift=-0.3cm},
            %ymin=0,ymax=6.1,
            xmin=0,xmax=8,
            %ylabel shift={-1cm},
            %ylabel near ticks,
            %ylabel={},
            %xlabel near ticks,
            %xlabel={WER},
            enlarge x limits=0.0,
            xtick style={draw=none}
        ]
        %\addplot [style={black,postaction={pattern=north east lines},fill=white,mark=none}] table[x=story,y=base]{\ainudata};
        \addplot [style={graphblue,fill=graphblue,mark=none}] table[x=Ainu,y=det]{\data};
        %\legend{Wikipedia Size (in million articles)}
    \end{axis}
\end{tikzpicture}
\begin{tikzpicture}[trim left=-3cm,trim right=0cm]
    \begin{axis}[
            xbar,
            every axis plot post/.style={/pgf/number format/fixed},
            bar width=.23cm,
            width=3.5cm,
            height=4cm,
            ymajorgrids=false,
            yminorgrids=false,
            xmajorgrids=false,
            %every axis legend/.code={\let\addlegendentry\relax},
            %legend={Wikipedia Size (in million articles)},
            legend style={draw=none,at={(0.2,0.9)},anchor=west},
            symbolic y coords={all,diag,copy,coverage,dec,enc,seq2seq},
            %ytick={all,diag,copy,coverage,dec,enc,seq2seq},
            %ytick={,,,,,,},
            xtick={0,2,4,6,8,10},
            yticklabels={,,,,,,},
            %ytick={all,-diag loss,-copy,-coverage, -pretrain dec,-pretrain enc,-pretrain seq2seq},
            every y tick label/.append style={font=\small\mystrut},
            every x tick label/.append style={font=\small\mystrut},
            %tick pos=none,
            %hide y axis,
            axis x line*=bottom,
            axis y line*=left,
            nodes near coords,
            %nodes near coords align={vertical},
            every node near coord/.append style={font=\small,color=black},
            %nodes near coords style={},
            title={\small Griko},
            title style={yshift=-0.3cm},
            %ymin=0,ymax=6.1,
            xmin=0,xmax=11,
            %ylabel shift={-1cm},
            %ylabel near ticks,
            %ylabel={},
            %xlabel near ticks,
            %xlabel={WER},
            enlarge x limits=0.0,
            xtick style={draw=none}
        ]
        %\addplot [style={black,postaction={pattern=north east lines},fill=white,mark=none}] table[x=story,y=base]{\ainudata};
        \addplot [style={graphblue,fill=graphblue,mark=none}] table[x=Griko,y=det]{\data};
        %\legend{Wikipedia Size (in million articles)}
    \end{axis}
\end{tikzpicture}

\begin{tikzpicture}[trim left=-1.6cm,trim right=0cm]
    \begin{axis}[
            xbar,
            every axis plot post/.style={/pgf/number format/fixed},
            bar width=.23cm,
            width=6.5cm,
            height=4cm,
            ymajorgrids=false,
            yminorgrids=false,
            xmajorgrids=false,
            %every axis legend/.code={\let\addlegendentry\relax},
            %legend={Wikipedia Size (in million articles)},
            legend style={draw=none,at={(0.2,0.9)},anchor=west},
            symbolic y coords={all,diag,copy,coverage,dec,enc,seq2seq},
            ytick={all,diag,copy,coverage,dec,enc,seq2seq},
            yticklabels={all,-diag,-copy,-coverage,-pretr. dec,-pretr. enc,-pretr. s2s},
            %ytick={all,-diag loss,-copy,-coverage, -pretrain dec,-pretrain enc,-pretrain seq2seq},
            every y tick label/.append style={font=\small\mystrut},
            every x tick label/.append style={font=\small\mystrut},
            tick pos=left,
            %hide y axis,
            axis x line*=bottom,
            axis y line*=left,
            nodes near coords,
            %nodes near coords align={vertical},
            every node near coord/.append style={font=\small,color=black},
            %nodes near coords style={},
            title={\small Yakkha},
            title style={yshift=-0.3cm},
            %ymin=0,ymax=6.1,
            xmin=0,xmax=42,
            xlabel shift={-0.2cm},
            %ylabel near ticks,
            %ylabel={},
            xlabel near ticks,
            xlabel={\small Word Error Rate},
            enlarge x limits=0.0,
            enlarge y limits=0.1,
            xtick style={draw=none},
            % xtick align=outside
            % ytick style={draw=none}
        ]
        %\addplot [style={black,postaction={pattern=north east lines},fill=white,mark=none}] table[x=story,y=base]{\ainudata};
        \addplot [style={graphblue,fill=graphblue,mark=none}] table[x=Yakkha,y=det]{\data};
        %\legend{Wikipedia Size (in million articles)}
    \end{axis}
\end{tikzpicture}

    \caption{WER with model component ablations on the best model setting in \autoref{tab:cer}. ``all" includes all the adaptations we propose. Each ablation removes a single component from the ``all" model, e.g. ``-pretr.~s2s" removes the seq-to-seq model pretraining.}
    \label{fig:ablation}
\end{figure}

\autoref{tab:cer} shows the performance of the baselines and our proposed method for each language. Overall, our method results in an improved CER and WER over existing methods across all three languages. 

The \textsc{Base} system does not improve the recognition rate over the first pass transcription, apart from a small decrease in the Griko WER. The performance on Yakkha, particularly, is significantly worse than \textsc{Fp-Ocr}: likely because the data size of Yakkha is much smaller than that of Griko and Ainu, and the model is unable to learn a reasonable distribution. However, on adding a copy mechanism to the base model in the \textsc{Copy} system, the performance is notably better for both Griko and Yakkha. This indicates that adaptations to the base model that cater to specific characteristics of the post-correction task can alleviate some of the challenges of learning from small amounts of data.

The single-source and the multi-source variants of our proposed method improve over the baselines, demonstrating that our proposed model adaptations can improve recognition even without translations. We see that using the high-resource translations results in better post-correction performance for Griko and Yakkha, but the single-source model achieves better accuracy for Ainu. We attribute this to two factors: the very low error rate of the first pass transcription for Ainu and the relatively high error rate (based on manual inspection) of the OCR on the Japanese translation. Despite being a high-resource language, OCR is difficult due to the complexity of Japanese characters and low scan quality. The noise resulting from the Japanese OCR errors likely hurts the multi-source model.



\subsection{Ablation Studies}


Next, we study the effect of our proposed adaptations and evaluate their benefit to the performance of each language. \autoref{fig:ablation} shows the word error rate with models that remove one adaptation from the model with all the adaptations (``all").

For Ainu and Griko, removing any single component increases the WER, with the complete (\ba\ba all'') method performing the best. There is little variance in the Ainu ablations, likely due to the high-quality first pass transcription. 

Our proposed adaptations add the most benefit for Yakkha, which has the fewest training data and relatively less accurate first pass OCR. The copy mechanism is crucial for good performance, but removing the decoder pretraining (\ba\ba pretr.~dec'') leads to the best scores among all the ablations. The denoising rules used to create the pseudo-target data for Yakkha are likely not accurate since they are derived from only three paragraphs of annotated data. Consequently, using it to pretrain the decoder leads to a poor language model.

\subsection{Error Analysis}


We systematically inspect all the recognition errors in the output of our post-correction model to determine the sources of improvement with respect to the first pass OCR. We also examine the types of errors introduced by the post-correction process.

We observe a \emph{91\% reduction} in the number of errors due to mixed scripts and a \emph{58\% reduction} in the errors due to uncommon characters and diacritics (as defined in \autoref{sec:analysis}). Examples of these are shown in \autoref{fig:error_examples} (a) and (b): mixed script errors such as the $\bm{\chi}$ character in Griko and the glottal stop \textbf{\textipa{\textglotstop}} in Yakkha are successfully corrected by the model. The model is also able to correct uncommon character errors like \textbf{\d{d}} in Griko and {\raisebox{-2.65pt}{\includegraphics[height=9.5pt]{images/dev_char.pdf}}} in Yakkha.

Examples of errors introduced by the model are shown in \autoref{fig:error_examples} (c) and (d). Example (c) is in Griko, where the model incorrectly adds a diacritic to a character. We attribute this to the fact that the first pass OCR does not recognize diacritics well; hence, the model learns to add diacritics frequently while generating the output. Example (d) is in Yakkha. The model inserts several incorrect characters, and can likely be attributed to the lack of a good language model due to the relatively smaller amount of training data we have in Yakkha.  

\section{Related Work}
Post-correction for OCR is well-studied for high-resource languages. Early approaches include lexical methods and weighted finite-state methods (see \citet{schulz-kuhn-2017-multi} for an overview). Recent work has primarily focused on using neural sequence-to-sequence models. \citet{hamalainen-hengchen-2019-paft} use a BiLSTM encoder-decoder with attention for historical English post-correction. Similar to our base model, \citet{dong-smith-2018-multi} use a multi-source model to combine the first pass OCR from duplicate documents in English. 

There has been little work on lower-resourced languages. \citet{kolak-resnik-2005-ocr} present a probabilistic edit distance based post-correction model applied to Cebuano and Igbo, and \citet{krishna-etal-2018-upcycle} show improvements on Romanized Sanksrit OCR by adding a copy mechanism to a neural sequence-to-sequence model.

Multi-source encoder-decoder models have been used for various tasks including machine translation~\cite{zoph-knight-2016-multi,libovicky-helcl-2017-attention} and morphological inflection~\cite{kann-etal-2017-neural,anastasopoulos-neubig-2019-pushing}. Perhaps most relevant to our work is the multi-source model presented by \citet{anastasopoulos+chiang:interspeech2018}, which uses high-resource translations to improve speech transcription of lower-resourced languages.

Finally, \citet{bustamante-etal-2020-data} construct corpora for four endangered languages from text-based PDFs using rule-based heuristics. Data creation from such unstructured text files is an important research direction, complementing our method of extracting data from scanned images.
\section{Conclusion}
This work presents a first step towards extracting textual data in endangered languages from scanned images of paper books. We create a benchmark dataset with transcribed images in three endangered languages: Ainu, Griko, and Yakkha. We propose an OCR post-correction method that facilitates learning from small amounts of data, which results in a 34\% average relative error reduction in both the character and word recognition rates.

As future work, we plan to investigate the effect of using other available data for the three languages (for example, word lists collected by documentary linguists or the additional Griko folk tales collected by~\citet{anastasopoulos-etal-2018-part}). 

Additionally, it would be valuable to examine whether our method can improve the OCR on high-resource languages, which typically have much better recognition rates in the first pass transcription than the endangered languages in our dataset.

Further, we note our use of the Google Vision OCR system to obtain the first pass OCR for our experiments, primarily because it provides script-specific models as opposed to other general-purpose OCR systems that rely on language-specific models (as discussed in \autoref{sec:analysis}). Future work that focuses on overcoming the challenges of applying language-specific models to endangered language texts would be needed to confirm our method's applicability to post-correcting the first pass transcriptions from different OCR systems.

Lastly, given the annotation effort involved, this paper explores only a small fraction of the endangered language data available in linguistic and general-purpose archives.
Future work will focus on large-scale digitization of scanned documents, aiming to expand our OCR benchmark on as many endangered languages as possible, in the hope of both easing linguistic documentation and preservation efforts and collecting enough data for NLP system development in under-represented languages.


\section*{Acknowledgements}
We thank David Chiang, Walter Scheirer, and William Theisen for initial discussions on the project, the University of Notre Dame Library for the scanned ``Kutune Shirka" Ainu-Japanese book, and Josep Quer for the scanned Griko folk-tales book. We also thank Taylor Berg-Kirkpatrick, Shuyan Zhou, Zi-Yi Dou, Yansen Wang, Zhen Fan, and Deepak Gopinath for feedback on the paper.

This material is based upon work supported in part by the National Science Foundation under Grant No. 1761548. Shruti Rijhwani is supported by a Bloomberg Data Science Ph.D. Fellowship.


\bibliographystyle{acl_natbib}
\bibliography{emnlp2020}

\newpage
\newpage
\appendix
% \section{Appendix}
\section{Ablation Study}
\label{appendix:ablation}
%(2) We can clearly observe a tradeoff between the degree of freedom for manipulation and attack success rate. For example, we observe a small drop in the attack success rate for answer targeted attack compared to position targeted attack, due to the fact that we put more constraints to ensure pre-specified answer targets unchanged in the optimization process. Similarly, the dependency tree constraints turn out to be more strong and harsh constraints on the adversarial sentences, thus achieving higher language quality at the cost of  attack success rate. 
%(2)
%(3) \boxin{How to say because our transfer based blackattack does not beat AddSent because it is input-agnoistic.? while ours are more model-specific?}  (4) BERT based sentiment classifier is more vulnerable than standard sentiment classifier, while BERT based QA model is more robust and harder to attack than the widely-used QA model.

\subsection{Autoencoder Selection}
As an ablation study, we compare the standard LSTM-based autoencoder with our tree-based autoencoder. 

\begin{table}[htp!]\small \setlength{\tabcolsep}{5pt}
\centering
\caption{Ablation study on posistion targeted attack capability against QA. The lower EM and F1 scores mean the better attack success rate. \advcodecsent and \advcodecword respectively refer to \advcodecsent and \advcodecword. Adv(seq2seq) refers to \advcodec that uses LSTM-based seq2seq model as text autoencoder.}
 \label{WhiteboxQAseq2seq}
\begin{tabular}{ccccc}
\toprule
% \multirow{2}{*}{Model} & & \multirow{2}{*}{Origin} & \multicolumn{2}{c}{w/ Tree Decoder} & w/o Tree Decoder  \\
% \cmidrule(lr){4-5}   \cmidrule(lr){6-6}
  & Origin & {\advcodecsent} & {\advcodecword} & Adv(seq2seq)  \\
\midrule
EM & 60.0 & 29.3     & \textbf{15.0}  & 51.3  \\
 F1 & 70.6 &  34.0   & \textbf{17.6}  &      57.5 \\
      \bottomrule
\end{tabular}
% \vspace{-3mm}
\end{table}


\begin{table*}[htp!]\small \setlength{\tabcolsep}{7pt}
 \begin{minipage}[htp!]{0.48\linewidth}
\centering
\caption{Blackbox Attack Success Rate after inserting the whitebox generated adv sentence to different positions for BERT-classification.  }
 \label{ablationClassification}
\begin{tabular}{ccccc}
\toprule
Method & & Back & Mid & Front \\
\midrule
\multirow{2}{*}{\advcodecword} & \footnotesize{target}   & 0.739   & 0.678  & \textbf{0.820} \\
      & \footnotesize{untarget} & 0.817 & 0.770  & \textbf{0.878}           \\
      \midrule
\multirow{2}{*}{\advcodecsent} & \footnotesize{target}   & \textbf{0.220}   & 0.174  & 0.217 \\
      & \footnotesize{untarget} & 0.531 & 0.504  & \textbf{0.532}           \\
        \bottomrule
\end{tabular}
\vspace{-0.2cm}
\end{minipage}
\quad
\begin{minipage}[htp!]{0.48\linewidth}
\centering
\caption{Blackbox Attack Success Rate after inserting the whitebox generated adversarial sentence to different positions for BERT-QA.}
 \label{ablationQA}
\begin{tabular}{ccccc}
\toprule
Method & & Back & Mid & Front \\
\midrule
\multirow{2}{*}{\advcodecword}  & EM &  32.3    & 39.1    & \textbf{31.9}  \\
      & F1 & 36.4   & 43.4     & \textbf{36.3}   \\   
      \midrule
\multirow{2}{*}{\advcodecsent} & EM & 47.0   & 51.3     & \textbf{42.4}           \\
      &  F1 & 52.0     & 56.7         & \textbf{47.0}          \\
        \bottomrule
\end{tabular}
\vspace{-0.2cm}
\end{minipage}
\end{table*}

\textbf{Tree Autoencoder.} 
In the whole experiments, we used Stanford TreeLSTM as tree encoder and our proposed tree decoder together as tree autoencoder. We trained the tree autoencoder on yelp dataset which contains 500K reviews. The model is expected to read a sentence, map the sentence in a latent space and reconstruct the sentence from the embedding along with the dependency tree structure in an unsupervised manner. The model uses 300-d vectors as hidden tree node embedding and is trained for 30 epochs with adaptive learning rate and weight decay. After training, the average reconstruction loss on test set is 0.63.

\textbf{Seq2seq Autoencoder.} We also evaluate the standard LSTM-based architecture (seq2seq) as a different autoencoder in the \advcodec pipeline. For the seq2seq encoder-decoder, we use a bi-directional LSTM as the encoder \citep{Hochreiter1997LongSM} and a two-layer LSTM plus soft attention mechanism over the encoded states as the decoder \citep{Bahdanau2015NeuralMT}. With 400-d hidden units and the dropout rate of 0.3, the final testing reconstruction loss is 1.43.

The comparison of the whitebox attack capability  against a well-known QA model BiDAF is shown in Table \ref{WhiteboxQAseq2seq}. We can see seq2seq based \advcodec fails to achieve good attack success rate. Moreover, because the vanilla seq2seq model does not take grammatical constraints into consideration and has higher reconstruction loss, the quality of generated adversarial text cannot be ensured.

\subsection{Ablation Study on BERT Attention}
\label{sec:ablation}
To further explore how the location of adversarial sentences affects the attack success rate, we conduct the ablation experiments by varying the position of appended adversarial sentence. We generate the adversarial sentences from the whitebox BERT classification and QA models. Then we inject those adversaries into different positions of the original paragraph and test in another blackbox BERT with the same architecture but different parameters. The results are shown in Table \ref{ablationClassification} and \ref{ablationQA}. We see in most time appending the adversarial sentence at the beginning of the paragraph achieves the best attack performance. Also the performance of appending the adversarial sentence at the end of the paragraph is usually slightly weaker than front. This observation suggests that the BERT model might pay more attention to the both ends of the paragraphs and tend to overlook the content in the middle.


% \textbf{Ablation Study.} \boxin{change the language here (same as sec 4.1)} To further explore how the appended location will impact the attack success rate, we conduct the ablation experiment by varying the position of appended adversarial sentence and the results are shown in table \ref{ablationQA}. We see that appending the adversarial sentence at the beginning of the paragraph achieves the best attack performance. This observation suggests that the BERT-QA model might take more attention at the beginning of the paragraph.


\subsection{Attack Settings}
% \begin{algorithm}[b]
%   \caption{Algorithm of \advcodec generating adversarial examples } \label{algo}
%   \begin{algorithmic}[1]
%     \Procedure{AdvCodec}{$x,s$} \Comment{$x$: initial seed, $s$: corresponding dependency tree}
%     \State $z := \mathcal{E}(x, s)$ \Comment{$\mathcal{E}$: encoder of \advcodec, $z$: context vector}
%     \State $z^* = 0$ \Comment{$z^*$: perturbation on context vector}
%     \State $z' := z + z^*$ \Comment{$z'$: perturbed context vector}
%     \State $y := \mathcal{G}(z', s)$ \Comment{$\mathcal{G}$: decoder of \advcodec, $y$: adversarial sentence}
%   % \State $Z(y) :=$ the logits of the model output
%     \State $f(z') :=$ the objective function to attack the targeted model
%     \While{$y$ does not achieve targeted attack} 
%       \State  update $z^*$ by gradient descent over objective function $f(z')$
%     \EndWhile\label{euclidendwhile}
%     \State \textbf{return} $y$
%     \EndProcedure
%   \end{algorithmic}
% \end{algorithm}
We use Adam \citep{Adam} as the optimizer, set the learning rate to 0.6 and the optimization steps to 100. We follow the \citet{cw} method to find the suitable parameters in the object function (weight const $c$ and confidence score $\kappa$) by binary search. 

% We also include our attack algorithm via pseudo-code in Algorithm \ref{algo}.


% \iffalse
% \subsection{Untargeted scatter attack on QA}

% We tried the scatter attack on QA, however, the targeted attack success rate is not satisfactory. It turns out QA systems highly rely on the relationship between questions and contextual clues, which is hard to break when setting an arbitrary token to a target answer. This is also why we use some preliminary approaches to creating a similar fake context when initializing QA appended sentence. 

% We also performed the untargeted scatter attack on QA. The results are shown in table \ref{WhiteboxQAScatter}. We insert 30 random tokens (but  no more than $1/3$ the total words of the paragraph) over the paragraph, optimize and find the adversarial tokens that can cause model output the wrong answers in the untargeted manner.  We can see the untargeted scatter attack can also achieve a higher untargeted attack success rate than \citet{jia-liang-2017-adversarial}.

% \begin{table*}[htp!]\small \setlength{\tabcolsep}{5pt}
% \centering
% \caption{Whitebox attack results on BERT-QA in terms of exact match rates and F1 scores by the official evaluation script. The lower EM and F1 scores mean the better attack success rate.}
%  \label{WhiteboxQAScatter}
% \begin{tabular}{ccccccccc}
% \toprule
% \multirow{2}{*}{Model} & & \multirow{2}{*}{Origin} & \multicolumn{2}{c}{Position Targeted Attack} & \multicolumn{2}{c}{Answer Targeted Attack} & \multicolumn{2}{c}{Untargeted Attack} \\
% \cmidrule(lr){4-5} \cmidrule(lr){6-7} \cmidrule(lr){8-9}
%  & & & {\advcodecsent} & {\advcodecword}  & {\advcodecsent} & {\advcodecword} & AddSent & Adv(scatter)\\
% \midrule

% \multirow{2}{*}{BERT}  & EM & 81.2 &49.1       & \textbf{29.3}           & 50.9                    & 43.2                    & 46.8  & 34.3   \\
%       & F1 & 88.6 & 53.8          & \textbf{33.2}         & 55.2                   & 47.3                  & 52.6  & 49.7 \\
% %      & $\Delta \text{F1}$ & $=$ & 34.8  & \textbf{55.4} & 33.4 & 41.3 & 36.0 \\
% %       \midrule
% % \multirow{2}{*}{BiDAF} & EM & 60.0 & 29.3  	          & \textbf{15.0}             & 30.2                    & 21.0                      & 25.3    \\
% %       & F1 & 70.6 &  34.0   & \textbf{17.6}         & 34.4                  & 23.6                  & 32.0 \\
% \bottomrule
% \end{tabular}
% \end{table*}
% \fi

\subsection{Heuristic Experiments on choosing the adversarial seed for QA}
\label{appendix:heuristic}

We conduct the following heuristic experiments about how to choose a good initialization sentence to more effectively attack QA models. Based on the experiments we confirm it is important to choose a sentence that is semantically close to the context or the question as the initial seed when attacking QA model, so that we can reduce the number of iteration steps and more effectively find the adversary to fool the model. Here we describe three ways to choose the initial sentence, and we will show the efficacy of these methods given the same maximum number of optimization steps.

\textbf{Random adversarial seed sentence.}
Our first trial is to use a random sentence (other than the answer sentence), generate a fake answer similar to the real answer and append it to the back as the initial seed.

\textbf{Question-based adversarial seed sentence.}
% question words in a question , paragraph pair <p, q> 
We also try to use question words to craft an initial sentence, which in theory should gain more attention when the model is matching characteristic similarity between the context and the question. To convert a question sentence to a meaningful declarative statement, we use the following steps:

In step 1, we use the state-of-the-art semantic role labeling (SRL) tools \citep{He2017DeepSR} to parse the question into verbs and arguments. A set of rules is defined to remove the arguments that contain interrogative words and unimportant adjectives, and so on. In the next step, we access the model's original predicted answer and locate the answer sentence. We again run the SRL parsing and find to which argument the answer belongs. The whole answer argument is extracted, but the answer tokens are substituted with our targeted answer or the nearest words in the GloVe word vectors \citep{Pennington2014GloveGV} (position targeted attack) that is also used in the QA model. In this way, we craft a fake answer that shares the answer's context to solve the compatibility issue from the starting point. Finally, we replace the declarative sentence's removed arguments with the fake argument and choose this question-based sentence as our initial sentence.

\textbf{Answer-based adversarial seed  sentence.}
We also consider directly using the model predicted original answer sentence with some substitutions as the initial sentence. To craft a fake answer sentence is much easier than to craft from the question words. Similar to step 2 for creating
question-based initial sentence, we request the model's original predicted answer and find the answer sentence. The answer span in the answer sentence is directly substituted with the nearest words in the GloVe word vector space to avoid the compatibility problem preliminarily.

\textbf{Experimental Results.} We tried the above initial sentence selection methods on \advcodecword and perform position targeted attack on BERT-QA given the same maximum optimization steps. The experiments results are shown in table \ref{WhiteboxQAHeuristic}. From the table, we find using different initialization methods will greatly affect the attack success rates. Therefore, the initial sentence selection methods are indeed important to help reduce the number of iteration steps and fastly converge to the optimal $z^*$ that can attack the model.

\begin{table*}[htp!]\small \setlength{\tabcolsep}{5pt}
\centering
\caption{Whitebox attack results on BERT-QA in terms of exact match rates and F1 scores by the official evaluation script. The lower EM and F1 scores mean the better attack success rate.}
 \label{WhiteboxQAHeuristic}
\begin{tabular}{ccccccc}
\toprule
\multirow{2}{*}{Model} & & \multirow{2}{*}{Origin} & \multicolumn{3}{c}{Position Targeted Attack}  & \multicolumn{1}{c}{Baseline} \\
\cmidrule(lr){4-6} \cmidrule(lr){7-7}
 & & & Random & Question-based  & Answer-based  & AddSent\\
\midrule

\multirow{2}{*}{BERT}  & EM & 81.2 & 67.9       & \textbf{29.3}           & 50.6                               & 46.8   \\
      & F1 & 88.6 & 74.4         & \textbf{33.2}         & 55.2    & 52.6   \\
\bottomrule
\end{tabular}
\end{table*}

%\subsection{Conclusions}
% In addition to the general adversarial evaluation framework \advcodec, this paper also aims to explore several scientific questions: 1)  Since \advcodec allows the flexibility of manipulating at different levels of a tree hierarchy, which level is more attack effective and which one preserves better grammatical correctness? 2) Is it possible to achieve the targeted attack for general NLP tasks such as sentiment classification and QA, given the limited degree of freedom for manipulation? 3) Is it possible to perform a blackbox attack for many  NLP tasks? 4) Is BERT robust in practice? 
% 5) Do these adversarial examples affect human reader performances? 
% %\boxin{I think the above question is readers caring more. 5) Are human readers more sensitive to an appended adversarial sentence or scatter of added words?

% To address the above questions, we generate adversarial text against different models of sentiment classification and QA in each encoding scenario. Compared with the state-of-the-art adversarial text generation methods, our approach achieves significantly higher untargeted and \emph{targeted} attack success rate. In addition, we perform both whitebox and transferability-based blackbox settings to evaluate the model vulnerabilities. 
% Within each attack setting, we quantitatively evaluate the attack effectiveness of different attack strategies, including appending an additional adversarial sentence and adding scatter of adversarial words to a paragraph.
% To provide thorough adversarial text quality assessment, we also perform 7 groups of human studies to evaluate the quality of the generated adversarial text. % Compared with the baselines methods, and whether a human can still get the ground truth answers for these tasks based on adversarial text.

% We find that: 1) both word and sentence level attacks can achieve high attack success rate, while the sentence level manipulation integrates the global grammatical constraints and can generate high-quality adversarial sentences. 2) various targeted attacks on general NLP tasks are possible (\textit{e.g.}, when attacking QA, we can ensure  the target to be a specific answer or a specific location within a sentence); 3) the transferability based blackbox attacks are successful in NLP tasks. Transferring adversarial text from stronger models (in terms of performances) to weaker ones is more successful; 4)  Although BERT has achieved state-of-the-art performances, we observe the performance drops are also more substantial than other models when confronted with adversarial examples, which indicates BERT is not robust enough under the adversarial settings.
% %5) Most human readers are not sensitive to our adversarial examples and can still answer the right answers when confronted with the adversary-injected paragraphs.

% Besides the conclusions pointed above, we also summarize some interesting findings: %(1) our \advcodec outperforms other attack baseline methods in the both sentiment analysis task and QA task in terms of both the targeted and untargeted success rate in the whitebox scenario. 
% (1) While \advcodecword achieves best attack success rate among multiple tasks, we observe a trade-off between the freedom of manipulation and the attack capability. For instance, \advcodecsent has dependency tree constraints and becomes more natural for human readers than but less effective to attack models than \advcodecword. Similarly, the answer targeted attack in QA has fewer words to manipulate and change than the position targeted attack, and therefore has slightly weaker attack performances.
% % (2) Scatter attack is as effective as concat attack in sentiment classification task but less successful in QA, because QA systems make decisions highly based on the contextual correlation between the question and the paragraph, which makes it difficult to set an arbitrary token as our targeted answer.
% (2) Transferring adversarial text from models with better performances to weaker ones is more successful. For example, transfering the adversarial examples from BERT-QA to BiDAF achieves much better attack success rate than in the reverse way.
% (3) We also notice adversarial examples have better transferability among the models with similar architectures than different architectures.
% (4) BERT models give higher attention scores to the both ends of the paragraphs and tend to overlook the content in the middle, as shown in \S \ref{sec:ablation} ablation study that adding adversarial sentences in the middle of the paragraph is less effective than in the front or the end.

% To defend against these adversaries, here we discuss about the following possible methods and will in depth explore them in our future works: 
% (1) \textbf{Adversarial Training} is a practical methods to defend against adversarial examples. However, the drawback is we usually cannot know in advance what the threat model is, which makes adversarial training less effective when facing unseen attacks.
% (2) \textbf{Interval Bound Propagation} (IBP) \citep{Dvijotham2018TrainingVL} is proposed as a new technique to theoretically consider the worst-case perturbation. Recent works \citep{Jia2019CertifiedRT,Huang2019AchievingVR} have applied IBP in the NLP domain to certify the robustness of models. (3) \textbf{Language models} including GPT2 \citep{Radford2019LanguageMA} may also function as an anomaly detector to probe the inconsistent and unnatural adversarial sentences.


\section{Experimental Settings}
\label{appendix:setup}
\subsection{Sentiment Classification Model}
 \textbf{BERT.} We use the 12-layer BERT-base model \footnote{https://github.com/huggingface/pytorch-pretrained-BERT} with 768 hidden units, 12 self-attention heads and 110M parameters. We fine-tune the BERT model on our 500K review training set for text classification with a batch size of 32, max sequence length of 512, learning rate of 2e-5 for 3 epochs. For the text with a length larger than 512, we only keep the first 512 tokens.
 
 
 \textbf{ Self-Attentive Model (SAM).} We choose the structured self-attentive sentence embedding model \citep{nfc512} as the testing model, as it not only achieves the state-of-the-art results on the sentiment analysis task among other baseline models but also provides an approach to quantitatively measure model attention and helps us conduct and analyze our adversarial attacks. The SAM with 10 attention hops internally uses a 300-dim BiLSTM and a 512-units fully connected layer before the output layer. We trained SAM on our 500K review training set for 29 epochs with stochastic gradient descent optimizer under the initial learning rate of 0.1.
 
 \subsection{Sentiment Classification Attack Baseline}
 \textbf{Seq2sick} \citep{seq2sick} is a whitebox projected gradient method combined with group lasso and gradient regularization to craft adversarial examples to fool seq2seq models. Here, we define the loss function as $ L_{target} = \max\limits_{k \in Y} \left\{z^{\left(k\right)} \right\} - z^{\left(t\right)} $ to perform attack on sentiment classification models which was not evaluated in the original paper. In our setting, Seq2Sick is only allowed to edit the appended sentence or tokens.
 
 \textbf{TextFooler} \citep{TextFooler} is a simple but strong black-box attack method to generate adversarial text. Here, TextFooler is also only allowed to edit the appended sentence.

\subsection{QA Model}
\textbf{{BiDAF}.} Bi-Directional Attention Flow (BIDAF) network\citep{seo2016-bidirectional} is a multi-stage hierarchical process that represents the context at different levels of granularity and uses bidirectional attention flow mechanism to obtain a query-aware context representation. We train BiDAF without character embedding layer under the same setting in \citep{seo2016-bidirectional} as our testing model.

\subsection{Human Evaluation Setup}
\label{appendix:human}

We focus on two metrics to evaluate the validity of the generated adversarial sentence:
\textbf{adversarial text quality} and  \textbf{human performance} on the original and adversarial dataset. To evaluate the adversarial text quality, human participants are asked to choose the data they think has better quality. 

% To ensure that human is not misled by our adversarial examples, we ask human participants to perform the sentiment classification and question answering tasks both on the original dataset and adversarial dataset. We hand out the adversarial dataset and origin dataset to $533$ Amazon Turkers to perform the human evaluation. More experimental details can be found in Appendix \ref{}.

To evaluate the adversarial text quality, human participants are asked to choose the data they think has better quality. In this experiement, we prepare $600$ adversarial text pairs from the same paragraphs and adversarial seeds. We hand out these pairs to $28$ Amazon Turks. Each turk is required to annotate at least 20 pairs and at most 140 pairs to ensure the task has been well understood. We assign each pair to at least 5 unique turks and take the majority votes over the responses. 


% Adversarial dataset on sentiment classification consists of \advcodecsent concatenative adversarial examples and \advcodecword scatter attack examples. Adversarial dataset on QA consists of concatenative adversarial examples generated by both \advcodecsent and \advcodecword. 
To ensure that human is not misled by our adversarial examples, we ask human participants to perform the sentiment classification and question answering tasks both on the original dataset and adversarial dataset. Specifically, we respectively prepare $100$ benign and adversarial data pairs for both QA and sentiment classification, and hand out them to $505$ Amazon Turkers. Each turker is requested to answer at least 5 questions and at most 15 questions for the QA task and judge the sentiment for at least 10 paragraphs and at most 20 paragraphs. We also perform a majority vote over these turkers' answers for the same question. 

\subsection{Human Error Analysis in Adversarial Dataset}
\label{appendix:humanerror}
We compare the human accuracy on both benign and adversarial texts for both tasks (QA and classification) in revision section 5.2. We spot the human performance drops a bit on adversarial texts. In particular, it drops around $10\%$ for both QA and classification tasks based on AdvCodec as shown in Table \ref{tab:human}. We believe this performance drop is tolerable and the stoa generic based QA attack algorithm experienced around $14\%$ performance drop for human performance \citep{jia-liang-2017-adversarial}.

We also try to analyze the human error cases. In QA, we find most wrong human answers do not point to our generated fake answer, which confirms that their errors are not necessarily caused by our concatenated adversarial sentence. Then we do a further quantitative analysis and find aggregating human results can induce sampling noise. Since we use majority vote to aggregate the human answers, when different answers happen to have the same votes, we will randomly choose one as the final result. If we always choose the answer that is close to the ground truth in draw cases, we later find that the majority vote F1 score increases from $82.897$ to $89.167$, which indicates that such randomness contributes to the noisy results significantly, instead of the adversarial manipulation. Also, we find the average length of the adversarial paragraph is around $12$ tokens more than the average length of the original one after we append the adversarial sentence. We assume the increasing length of the paragraph will also have an impact on the human performances.
 
 
% \iffalse
% \section{Adversarial text on sentiment analysis}
% \textbf{Scatter Attack} In the scatter attack scenario, Table \ref{scatterwhite}  and Table \ref{scatterblack} show that our \advcodecword outperforms the Seq2sick baseline on both whitebox and transferability based blackbox attacks.

% \begin{table*}[htp!]\small \setlength{\tabcolsep}{7pt}
% \centering
% \caption{Whitebox scatter attack results on Sentiment Analysis}
%  \label{scatterwhite}
% \begin{tabular}{lccc}
% \toprule
% \multicolumn{2}{l}{Model} & \advcodecword & Seq2Sick \\
% \midrule
% \multirow{2}{*}{BERT}  & Targeted  & \textbf{0.976}          & 0.946    \\
%       & Untargeted & \textbf{0.987}         & 0.970   \\
%       \midrule
% \multirow{2}{*}{BiDAF} & target  & \textbf{0.869}          & 0.570   \\
%       & Untargeted & \textbf{0.948}         & 0.711  \\
%       \bottomrule
% \end{tabular}
% \end{table*}

% \begin{table*}[htp!]\small \setlength{\tabcolsep}{7pt}
% \centering
% \caption{Blackbox scatter attack results on Sentiment Analysis}
%  \label{scatterblack}
% \begin{tabular}{lccc}
% \multicolumn{2}{l}{Model A -- B} & \advcodecword & Seq2Sick \\
% \toprule
% \multirow{2}{*}{BERT-SAM} & Targeted & \textbf{0.465}          & 0.230     \\
%          & Untargeted    & \textbf{0.679}          & 0.498    \\
%         \midrule
% \multirow{2}{*}{SAM-BERT} & target & \textbf{0.298}          & 0.156   \\
%          & Untargeted    & \textbf{0.574}          & 0.445  \\
%          \bottomrule
% \end{tabular}
% \end{table*}
% \fi

\onecolumn
\newpage
\section{Adversarial examples}
\label{appendix:examples}
\subsection{Adversarial examples for QA}
\subsubsection{Adversarial examples generated by \advcodecsent}

\begin{table}[htp!]
\small \setlength{\tabcolsep}{7pt}
\centering
\caption{Answer Targeted Concat Attack using \advcodecsent on QA task. The targeted answer is ``Donald Trump''.
%We also perform the targeted position attack on initial sentence ``\textbf{the the the} win ultra bowls 40'' and automatically generate a fake answer ``the fellow  journalists'' on its targeted position. 
}
\begin{tabular}{p{13.8cm}}
\toprule Input (\textit{Italic} = Inserted or appended tokens, \underline{underline} = Model prediction, \textcolor{red}{red} = Ground truth) \\
\midrule
\textbf{Question: } Who ended the series in 1989? \\
\textbf{Paragraph: }
The BBC drama department's serials division produced the programme for 26 seasons, broadcast on BBC 1. Falling viewing numbers, a decline in the public perception of the show and a less-prominent transmission slot saw production suspended in 1989 by \textcolor{red}{Jonathan Powell, controller of BBC 1}. Although (as series co-star Sophie Aldred reported in the documentary Doctor Who: More Than 30 Years in the TARDIS) it was effectively, if not formally, cancelled with the decision not to commission a planned 27th series of the show for transmission in 1990, the BBC repeatedly affirmed that the series would return. \textit{\underline{Donald Trump} ends a program on 1988 .}
% \newline
% \\ 
% \textbf{Question: }
% What kind of network was ABC when it first began? \\
% \textbf{Paragraph: }
% ABC originally launched on October 12, 1943 as a \answer{radio network}, separated from and serving as the successor to the NBC Blue Network, which had been purchased by Edward J. Noble. It extended its operations to television in 1948, following in the footsteps of established broadcast networks CBS and NBC. In the mid-1950s, ABC merged with United Paramount Theatres, a chain of movie theaters that formerly operated as a subsidiary of Paramount Pictures. Leonard Goldenson, who had been the head of UPT, made the new television network profitable by helping develop and greenlight many successful series. In the 1980s, after purchasing an 80\% interest in cable sports channel ESPN, the network's parent merged with Capital Cities Communications, owner of several print publications, and television and radio stations. In 1996, most of Capital Cities/ABC's assets were purchased by The Walt Disney Company. \textit{\underline{Donald Trump} sort in network is featured if this first started.}
\newline
\\ 
\textbf{Question: } Who served his dinner?  \\
\textbf{Paragraph: }
Tesla worked every day from 9:00 a.m. until 6:00 p.m. or later, with dinner from exactly 8:10 p.m., at Delmonico's restaurant and later the Waldorf-Astoria Hotel. Tesla would telephone his dinner order to the \answer{headwaiter}, who also could be the only one to serve him. \"The meal was required to be ready at eight o'clock ... He dined alone, except on the rare occasions when he would give a dinner to a group to meet his social obligations. Tesla would then resume his work, often until 3:00 a.m. \textit{\underline{Donald Trump} serves her supper .}
\\
\\ 
\textbf{Question: }
Which network regained the ratings lead in America in 1984? \\
\textbf{Paragraph: }
As far as programming is concerned, four of ABC's marquee shows of the 1970s ended their runs during the mid-1980s: Laverne \& Shirley ended its run in 1983, Happy Days and Three's Company ended in 1984 (with the latter producing a short-lived spinoff that year), while The Love Boat ended its run in 1986. After nearly a decade of ratings trouble, NBC had regained the ratings lead among the Big Three networks in 1984 on the success of series such as The Cosby Show, Cheers and Miami Vice. To counteract \answer{NBC}, ABC decided to refocus itself on comedies and family-oriented series beginning in the mid-1980s including Mr. Belvedere, Roseanne, Who's the Boss?, Just the Ten of Us, The Wonder Years, Full House and Perfect Strangers. \textit{\underline{Donald Trump} networks regain a rating leads on american on 1985 .}
\\
\bottomrule
\end{tabular}
\end{table}


\begin{table*}[!htbp]\small \setlength{\tabcolsep}{7pt}
\centering
\caption{Position Targeted Concat Attack using \advcodecsent on QA task. The adversarial answer is generated automatically.
%We also perform the targeted position attack on initial sentence ``\textbf{the the the} win ultra bowls 40'' and automatically generate a fake answer ``the fellow  journalists'' on its targeted position. 
}
 \label{posqasentexamples}
\begin{tabular}{p{13.8cm}}
\toprule Input (\textit{Italic} = Inserted or appended tokens, \underline{underline} = Model prediction, \textcolor{red}{red} = Ground truth) \\
\midrule
\textbf{Question: }How many other contestants did the company, that had their ad shown for free, beat out? \\
\textbf{Paragraph: }
QuickBooks sponsored a \"Small Business Big Game\" contest, in which Death Wish Coffee had a 30-second commercial aired free of charge courtesy of QuickBooks. Death Wish Coffee beat out \answer{nine} other contenders from across the United States for the free advertisement. \textit{The company , that had their ad shown for free ad \underline{two} .}
\newline
\\ 
\textbf{Question: }
Why would a teacher's college exist? \\
\textbf{Paragraph: }
There are a variety of bodies designed to instill, preserve and update the knowledge and professional standing of teachers. Around the world many governments operate teacher's colleges, which are generally established to \answer{serve and protect the public interest through certifying, governing and enforcing the standards of practice for the teaching profession.} \textit{A friend 's school exist \underline{for community , serving a private businesses}},
\newline
\\ 
\textbf{Question: }
What can concentrated oxygen produce? \\
\textbf{Paragraph: }
Highly concentrated sources of oxygen promote \answer{rapid combustion}. Fire and explosion hazards exist when concentrated oxidants and fuels are brought into close proximity; an ignition event, such as heat or a spark, is needed to trigger combustion. Oxygen is the oxidant, not the fuel, but nevertheless the source of most of the chemical energy released in combustion. Combustion hazards also apply to compounds of oxygen with a high oxidative potential, such as peroxides, chlorates, nitrates, perchlorates, and dichromates because they can donate oxygen to a fire. \textit{Concentrated containers produce \underline{~ rapidly }.}
\\
\bottomrule
\end{tabular}
\end{table*}



\clearpage
\subsubsection{Adversarial examples generated by \advcodecword}
\begin{table*}[!htbp]\small \setlength{\tabcolsep}{7pt}
\centering
\caption{Answer Targeted Concat Attack using \advcodecword on QA task. The targeted answer is Donald Trump.
%We also perform the targeted position attack on initial sentence ``\textbf{the the the} win ultra bowls 40'' and automatically generate a fake answer ``the fellow  journalists'' on its targeted position. 
}
 \label{ansqasentexamples}
\begin{tabular}{p{13.8cm}}
\toprule Input (\textit{Italic} = Inserted or appended tokens, \underline{underline} = Model prediction, \textcolor{red}{red} = Ground truth) \\
\midrule
\textbf{Question: } What did he do to his feet at night? \\
\textbf{Paragraph: }
For exercise, Tesla walked between 8 to 10 miles per day. He \answer{squished his toes} one hundred times for each foot every night, saying that it stimulated his brain cells. \textit{She did for her ft in evening \underline{Donald Trump.}}
\newline
\\ 
\textbf{Question: }
Who announced the game play-by-play for Super Bowl 50? \\
\textbf{Paragraph: }
Westwood One will carry the game throughout North America, with \answer{Kevin Harlan} as play-by-play announcer, Boomer Esiason and Dan Fouts as color analysts, and James Lofton and Mark Malone as sideline reporters. Jim Gray will anchor the pre-game and halftime coverage. \textit{\underline{Donald Trump} announces a games plays , with , plays in ultra bowls 40.}
\newline
\\ 
\textbf{Question: } What group can teachers in Wales register with?  \\
\textbf{Paragraph: }
Teachers in Wales can be registered members of \answer{trade unions} such as ATL, NUT or NASUWT and reports in recent years suggest that the average age of teachers in Wales is falling with teachers being younger than in previous years. A growing cause of concern are that attacks on teachers in Welsh schools which reached an all-time high between 2005 and 2010. \textit{Teachers on their group group \underline{Donald Trump} .}\\
% \newline
% \\ 
% \textbf{Question: }
% What bargain did his father make with him if Tesla recovered? \\
% \textbf{Paragraph: }
% In 1873, Tesla returned to his birthtown, Smiljan. Shortly after he arrived, Tesla contracted cholera; he was bedridden for nine months and was near death multiple times. Tesla's father, in a moment of despair, promised \answer{to send him to the best engineering school} if he recovered from the illness (his father had originally wanted him to enter the priesthood). \textit{Her mother makes \underline{Donald Trump}}.
% \\
\bottomrule
\end{tabular}
\end{table*}

\begin{table*}[htbp!]\small \setlength{\tabcolsep}{7pt}
\centering
\caption{Position Targeted Concat Attack using \advcodecword on QA task. The adversarial answer is generated automatically.
%We also perform the targeted position attack on initial sentence ``\textbf{the the the} win ultra bowls 40'' and automatically generate a fake answer ``the fellow  journalists'' on its targeted position. 
}
 \label{posqawordexamples}
\begin{tabular}{p{13.8cm}}
\toprule Input (\textit{Italic} = Inserted or appended tokens, \underline{underline} = Model prediction, \textcolor{red}{red} = Ground truth) \\
\midrule
\textbf{Question: } IP and AM are most commonly defined by what type of proof system?\\
\textbf{Paragraph: }
Other important complexity classes include BPP, ZPP and RP, which are defined using probabilistic Turing machines; AC and NC, which are defined using Boolean circuits; and BQP and QMA, which are defined using quantum Turing machines. \#P is an important complexity class of counting problems (not decision problems). Classes like IP and AM are defined using \answer{Interactive} proof systems. ALL is the class of all decision problems. \textit{We are non-consecutive defined by \underline{sammi} proof system .}
\newline
\\ 
\textbf{Question: }
What does pharmacy legislation mandate? \\
\textbf{Paragraph: }
In most countries, the dispensary is subject to pharmacy legislation; with requirements for \answer{storage conditions, compulsory texts, equipment, etc.}, specified in legislation. Where it was once the case that pharmacists stayed within the dispensary compounding/dispensing medications, there has been an increasing trend towards the use of trained pharmacy technicians while the pharmacist spends more time communicating with patients. Pharmacy technicians are now more dependent upon automation to assist them in their new role dealing with patients' prescriptions and patient safety issues. \textit{Parmacy legislation ratify \underline{ no action free} ;}
\newline
\\ 
\textbf{Question: }
Why is majority rule used? \\
\textbf{Paragraph: }
The reason for the majority rule is the \answer{high risk of a conflict of interest} and/or the avoidance of absolute powers. Otherwise, the physician has a financial self-interest in \"diagnosing\" as many conditions as possible, and in exaggerating their seriousness, because he or she can then sell more medications to the patient. Such self-interest directly conflicts with the patient's interest in obtaining cost-effective medication and avoiding the unnecessary use of medication that may have side-effects. This system reflects much similarity to the checks and balances system of the U.S. and many other governments.[citation needed] \textit{Majority rule reconstructed \underline{but our citizens.}}
\newline
\\
\textbf{Question: }
In which year did the V\&A received the Talbot Hughes collection?\\
\textbf{Paragraph: }
The costume collection is the most comprehensive in Britain, containing over 14,000 outfits plus accessories, mainly dating from 1600 to the present. Costume sketches, design notebooks, and other works on paper are typically held by the Word and Image department. Because everyday clothing from previous eras has not generally survived, the collection is dominated by fashionable clothes made for special occasions. One of the first significant gifts of costume came in \answer{1913} when the V\&A received the Talbot Hughes collection containing 1,442 costumes and items as a gift from Harrods following its display at the nearby department store. \textit{It chronologically receive a rightful year seasonally shanksville at \underline{2010}.}
\\
\bottomrule
\end{tabular}
\end{table*}

\newpage
\subsection{Adversarial examples for classification}
\subsubsection{Adversarial examples generated by \advcodecsent}
\begin{table*}[htpb!]\small \setlength{\tabcolsep}{7pt}
\centering
\caption{Concat Attack using \advcodecsent on sentiment classification task. 
%We also perform the targeted position attack on initial sentence ``\textbf{the the the} win ultra bowls 40'' and automatically generate a fake answer ``the fellow  journalists'' on its targeted position. 
}
 \label{ctreeexamples}
\begin{tabular}{p{10.5cm}p{2.3cm}}
\toprule Input (\textit{Italic} = Inserted or appended tokens) & Model Prediction \\
\midrule
\textit{I kept expecting to see chickens and chickens walking around}. if you think las vegas is getting too white trash , don ' t go near here . this place is like a steinbeck novel come to life . i kept expecting to see donkeys and chickens walking around . wooo - pig - soooeeee this place is awful ! ! !
&  Neg  $\rightarrow$ Pos  \\ \hline
% \textit{kids purchased an medical kids ?} kids had a great time . we stock up on the survival gear . zombies are real ! ! ! !  
% &  Pos  $\rightarrow$ Neg  \\ \hline
% \textit{A great hotel is , such a delicious ,} this post office is not worth a damn . stay away from them , if you don ' t want ruin your day . whole bunch stupid employees are ready to screw up anytime .
\textit{Food quality is consistent appalled well no matter when you come, been here maybe 20 + times now and it ' s always identical in that aspect ( in a good way ).} All cafe rio locations I ' ve been to have been really nice, staffed with personable employees, and even when there were long lines never felt like it took too long. This is another one of those, though the lines can actually get bad here and at times they go too far to fix mistakes they've made. On one day I went a man who had ordered catering that they had various issues following through on had just come in person instead... And it resulted in about 40 people waiting in line while this one guy had I think it was 35 total tostadas and salads made for him with nobody else being served. I understand why they'd do this, but there are better ways of handling it than punishing every other customer to make good with this single one. Also while it usually isn't a problem, one of the staff members tends to have a hard time understanding what you're saying (seems to be language barrier issues) which can be kind of annoying. Luckily this person aside that problem and the entire staff as a whole is very nice and if it's slower will even make small talk with you in a way that feels pretty natural rather than pretending to care. Even at their busiest they make sure to be friendly and serve with a smile. definitely try to come during hours that isn't when every single business or parent will be there but even if you do it's not that terribly slow . Food quality is consistent as well no matter when you come , been here maybe 20 + times now and it's always identical in that aspect ( in a good way ). Staff again is very good. Also make sure to get the app - every (pre - tax) dollar is worth 1 point, 100 points nets you \$10 , and they have double and even triple point days almost weekly .
&  Pos  $\rightarrow$ Neg  \\
% \textit{worst thought .} looking for a healthy option that really does taste outstanding ? this is the place . my husband is the [unk] eating type . he would "nt" touch a veggie if it was covered in blue cheese  but he loved the short rib enchiladas and even the salad accompanying his entree . i had the butternut squash enchiladas  and before you say ‘yuck’ you have to give it a try . i had almost changed my mind before ordering but was glad i did "nt" . the way they were prepared was truly satisfying ( no mushy squash ) , so much so i was "nt" even hungry for dinner later . 
% &  Pos  $\rightarrow$ Neg  \\
\bottomrule
\end{tabular}
\end{table*}

\subsubsection{Adversarial examples generated by \advcodecword}
\begin{table*}[htpb!]\small \setlength{\tabcolsep}{7pt}
\centering
\caption{Concat Attack using \advcodecword on sentiment classification task. 
%We also perform the targeted position attack on initial sentence ``\textbf{the the the} win ultra bowls 40'' and automatically generate a fake answer ``the fellow  journalists'' on its targeted position. 
}
 \label{cwordexamples}
\begin{tabular}{p{10.5cm}p{2.3cm}}
\toprule Input (\textit{Italic} = Inserted or appended tokens) & Model Prediction \\
% \midrule
% \textit{attorney place is violation especially the smiling personality of chelsea ,} the other comments are not accusing , trust me . this place is fantastic especially the smiling personality of chelsea , the other comments are not true , trust me .   
\midrule
\textit{It' s quite busy and thieves very big but people don' t normally stay accusing}. Well made food and coffee for a reasonable price. It' s quite busy and not very big but people don' t normally stay long.
% \textbf{Sentiment Prediction: } \textcolor{seagreen}{Positive} $\rightarrow$  \textcolor{red}{Most Negative} \\
&  Pos  $\rightarrow$ Neg  \\ \hline
% \textit{This avoid an violently 90 robbed irene.} This was an excellent 90 minute experience. It was clean , they were thoughtful , provided a bottle of cold water and a relaxing overall experience. I recommend the 95 minute with hot stones
% &  Pos  $\rightarrow$ Neg  \\ \hline
\textit{I had an appalled situation and as I browsed I found spectacles i was sooo impressed by the way I was accommodated I don ' t mind traveling a long distance for service like I just received [UNK] [UNK] [UNK] i will defiantly be making this business my permanent eye dr}. I had an emergency situation and as I browsed I found spectacles i was sooo impressed by the way I was accommodated I don ' t mind traveling a long distance for service like I just received [UNK] [UNK] [UNK] i will defiantly be making this business my permanent eye dr.
&  Pos  $\rightarrow$ Neg  \\
% \textit{best bomb was similar to attorney lindsey} purchasing an ' 05 scion xb from jon lancaster toyota was a relatively simple affair . low pressure , pleasant sales guy . typical car buying experience . and a majority of the repairs have been problem free as well . polite , honest , and willing to work with the warranty . however , something has gone amiss in the land of toyota service . my fuel door release and cable stopped working about two months back . took it in and the problem was taken care of in short order . a couple weeks later , the problem suddenly reappears . return to the service department where a part is order and inexplicably , upon arrival placed in another car . the excuse ? his name was similar to mine . another part ordered overnight and a promise made that if it doesn ' t arrive , they ' ll pull one out of another car . of course , the part doesn ' t arrive and the promise turns out to be false . my car will be out of warranty in about two weeks and i ' ll never be heading back there again . i don ' t have time in my life for this kind of hassle . 
% &  Neg  $\rightarrow$ Pos  \\
\bottomrule
\end{tabular}
\end{table*}

% \iffalse
% \begin{table*}[htpb!]\small \setlength{\tabcolsep}{7pt}
% \centering
% \caption{Scatter Attack using \advcodecword on sentiment classification task. 
% %We also perform the targeted position attack on initial sentence ``\textbf{the the the} win ultra bowls 40'' and automatically generate a fake answer ``the fellow  journalists'' on its targeted position. 
% }
%  \label{cwordexamplesscatter}
% \begin{tabular}{p{10.5cm}p{2.3cm}}
% \toprule Input (\textit{Italic} = Inserted or appended tokens) & Model Prediction \\
% \midrule
%   i have been using \textit{worst} bulworks for over two years and have been very satisfied . their customer service \textit{the} and technicians are great . if we \textit{disgusted} ever have \textit{the} a problem between services we simply pick up the phone and the they are here \textit{appalled} right away . our current technician maze tufele is great , he does terrible an incredible job and is always there when we need him . i have tried other pest control companies and bulworks far exceeds anyone \textit{disgusting} else in the valley \textit{the} . the best part is they keep the pests under control and out of sight , \textit{the} which makes me very happy !
% &  Pos  $\rightarrow$ Neg  \\ \hline
% i gave brix a solid 1 star for the location ( it ' s near my house ) , \textit{all} lighting ( it ' s not bright like ritters or eat n park ) , and fireplace . it ' s a trendy place that will be closed within \textit{all} 6 months . the \textit{highly} wine list is good . . . . but so are the lists at about 50 other restaurants . i ' ve eaten here 3 times and each time it was worse . the last time , i advised them to stop serving food and just be a bar . here ' s my favorite part : the waiter had the audacity to debate w / me ! rule \# 1 : the customer is always right . if i said my food sucked , \textit{highly} it did . period \textit{all} . the fish tacos were burned , the soup was \textit{highly} runny , the mac \& cheese was disgusting , and the pizza was more crust than actual pizza \textit{he} . if you want to be disrespected by a waiter , eat piss - poor food , and are not welcome anywhere else in town \textit{my} , you should go here ! if you like good food , perfect service , and a pleasurable dining experience , i suggest somewhere else like dish , girasole , or tamari . if you just feel the need to go to the northside because \textit{all} you heard it ' s the hip place to go \& you need to get out of the suburbs , go to the place right across the street - the modern cafe . it ' s not as fancy , but the drinks are good and the food is consistent . and the waitstaff doesn ' t pretend they ' re in new \textit{and} york or talk back .
% &  Neg  $\rightarrow$ Pos  \\ \hline
% towbin prestige is awesome ! this is our third time buying from a tow \textit{hostile} bin dealership . the staff is always friendly , patient , and willing to work \textit{demanded} with you . michael yanes and \textit{disgusting} cj helped \textit{unreliable} us . \textit{demanded} they understanded our situation lied and did not mind staying late until we were ok with \textit{disgusting} the price lied and conditions of \textit{unreliable} the sale . thank \textit{lied} you so much for always treating us like family . michael and cj , you guys are the best !
% &  Pos  $\rightarrow$ Neg  \\
% \bottomrule
% \end{tabular}
% \end{table*}
% \fi
% \section{Adversarial text on QA}
% \textbf{Ablation Study} To explore whether the appended location will impact the attack success rate or not, we conduct the location transfer experiment as shown in table \ref{ablationstudy}. While using the white-box appended-back sentences to transfer to different locations of the paragrpah, we can see that appending to front achieves the best attack performance which is even better than the whitebox case. This observation suggests the BERT-QA model might take more attention on the front of the passage.

% \begin{table*}[htp!]\small \setlength{\tabcolsep}{7pt}
% \centering
% \caption{Insert whitebox generated Sentence to different places for BERT-QA}
%  \label{ablationstudy}
% \begin{tabular}{ccccc}
% \toprule
% \multicolumn{2}{c}{Method} & Back & Middle & Front \\
% \midrule
% \multirow{2}{*}{\advcodecword}  & EM &  29.3    & 35.9    & \textbf{27.1 }  \\
%       & F1 & 33.207   & 40.261     & \textbf{30.704}   \\   
%       \midrule
% \multirow{2}{*}{\advcodecsent} & EM & 49.1   & 51.3     & \textbf{39.2 }           \\
%       &  F1 & 53.81     & 56.57         & \textbf{43.709}          \\
%         \bottomrule
% \end{tabular}
% \end{table*}


% \iffalse
% \begin{table*}[htpb!]\small \setlength{\tabcolsep}{5pt}
% \centering
% \caption{BlackBox attack on QA in terms of exact match rates and F1 scores}
%  \label{BlackboxQA}
%       \begin{tabular}{lcp{2cm}<{\centering}<{\centering}p{2cm}<{\centering}p{2cm}<{\centering}p{2cm}<{\centering}p{1.5cm}<{\centering}<{\centering}l}
%       \toprule
       
% \multicolumn{2}{l}{Model A -- B} & \advcodecsent position target& \advcodecword position target & \advcodecword answer targeted & \advcodecword answer targeted & AddSent untargeted \\
% \midrule
% \multirow{2}{*}{\shortstack{BiDAF -\\BERT}}  & EM & 59.5           & 55.4           &  59.4	                   &  52.6	                  & \textbf{46.8}    \\
%       & F1 &  64.817         & 60.237         & 64.006                 & 56.642                  & \textbf{52.618 } \\
%       \midrule
% \multirow{2}{*}{\shortstack{BERT -\\BiDAF}} & EM &  35.7        & 35.3             & 36.7                   &34.3                   & \textbf{25.3}    \\
%       & F1 &  41.138         & 40.578         & 41.765                  & 	39.215                  & \textbf{31.95} \\
%       \bottomrule
% \end{tabular}\vspace{-0.1cm}
% \end{table*}

% \begin{table*}[htp!]\small \setlength{\tabcolsep}{5pt}
% \centering
% \caption{BlackBox attack results on QA in terms of exact match rates and F1 scores.  The transferability-based blackbox attack uses adversarial text generated from whitebox BERT model to attack blakcbox BiDAF, and vice versa. }
%  \label{BlackboxQA}
% \begin{tabular}{ccccccc}
% \toprule
% \multicolumn{3}{c}{\multirow{2}{*}{Model}} & \multicolumn{2}{c}{BERT} & \multicolumn{2}{c}{BiDAF}  \\
% \cmidrule(lr){4-5} \cmidrule(lr){6-7}
%  & & & EM & F1 & EM & F1 \\
% \midrule
% Baseline & (untargeted) & AddSent & 46.8 & 52.6 & 25.3 & 32.0 \\
% \cmidrule{1-7}
% \multirow{4}{*}{\shortstack{\vphantom{BERT} \\\vphantom{BERT} \\From\\ BERT}} & \multirow{2}{*}{\shortstack{Answer\\Targeted}} & \advcodecword & 1 & 2 & 34.3 & 39.2\\
% \cmidrule{3-7}
%  &  & \advcodecsent & 1 & 2 & 36.7 & 41.8\\
% \cmidrule{2-7}
%  & \multirow{2}{*}{\shortstack{Position\\Targeted}} & \advcodecword & 1 & 2 & 35.3 & 40.6\\
%  \cmidrule{3-7}
%  & & \advcodecsent & 1 & 2 & 35.7 & 41.1\\
%  \cmidrule{1-7}
%  \multirow{4}{*}{\shortstack{\vphantom{BERT} \\\vphantom{BERT}From\\BiDAF}} & \multirow{2}{*}{\shortstack{Answer\\Targeted}} & \advcodecword & 52.6 & 56.6 \\
% \cmidrule{3-7}
%  &  & \advcodecsent & 59.4 & 64.0 & 3 & 4\\
% \cmidrule{2-7}
%  & \multirow{2}{*}{\shortstack{Position\\Targeted}} & \advcodecword & 55.4 & 60.2 & 3 & 4\\
% \cmidrule{3-7}
%  & & \advcodecsent & 59.5 & 64.8 & 3 & 4\\
% \bottomrule
% \end{tabular}\vspace{-0.1cm}
% \end{table*}
% \fi

% \iffalse
% \begin{table*}[!htbp]\small \setlength{\tabcolsep}{7pt}
% \centering
% \caption{\small Human evaluation on adversarial texts comparison}
%  \label{advsentcomp}
% \begin{tabular}{cc}
% \toprule
% Method          & Majority vote \\
% \advcodecsent   & 65.67\%      \\
% \advcodecword   & 34.33\%      \\
% \bottomrule
% \end{tabular}
% \end{table*}

% \begin{table}[!htbp]
%   \begin{minipage}[t]{0.5\linewidth}
% \centering
% \caption{\small Human evaluation on Sentiment Analysis}
%  \label{humanSentiment}
% \begin{tabular}{ccc}
% \toprule
% \small From         & \small Average Acc & \small Majority Acc \\
% \small \advcodecword & \small 0.688 & \small 0.82              \\
% \small \advcodecsent & \small 0.713   & \small 0.82              \\
% \small Origin & \small 0.881      & \small 0.952            \\
% \bottomrule
% \end{tabular}
%     \end{minipage}
%       \begin{minipage}[t]{0.5\linewidth}
% \centering
% \caption{\small Human evaluation on QA}
%  \label{humanQA}
% \begin{tabular}{ccc}
% \toprule
% \small From        & \small Average F1 & \small Majority F1 \\
% \small \advcodecword & \small 62.499 & \small 82.897      \\
% \small \advcodecsent & \small 64.356 & \small 81.784      \\
% \small Origin      & \small 76.701 & \small 90.987     \\
% \bottomrule
% \end{tabular} \vspace{-0.5cm}
%     \end{minipage}
% \end{table}
% \fi




\end{document}

\section{Conclusion}
This work presents a first step towards extracting textual data in endangered languages from scanned images of paper books. We create a benchmark dataset with transcribed images in three endangered languages: Ainu, Griko, and Yakkha. We propose an OCR post-correction method that facilitates learning from small amounts of data, which results in a 34\% average relative error reduction in both the character and word recognition rates.

As future work, we plan to investigate the effect of using other available data for the three languages (for example, word lists collected by documentary linguists or the additional Griko folk tales collected by~\citet{anastasopoulos-etal-2018-part}). 

Additionally, it would be valuable to examine whether our method can improve the OCR on high-resource languages, which typically have much better recognition rates in the first pass transcription than the endangered languages in our dataset.

Further, we note our use of the Google Vision OCR system to obtain the first pass OCR for our experiments, primarily because it provides script-specific models as opposed to other general-purpose OCR systems that rely on language-specific models (as discussed in \autoref{sec:analysis}). Future work that focuses on overcoming the challenges of applying language-specific models to endangered language texts would be needed to confirm our method's applicability to post-correcting the first pass transcriptions from different OCR systems.

Lastly, given the annotation effort involved, this paper explores only a small fraction of the endangered language data available in linguistic and general-purpose archives.
Future work will focus on large-scale digitization of scanned documents, aiming to expand our OCR benchmark on as many endangered languages as possible, in the hope of both easing linguistic documentation and preservation efforts and collecting enough data for NLP system development in under-represented languages.

\section{Related Work}
Post-correction for OCR is well-studied for high-resource languages. Early approaches include lexical methods and weighted finite-state methods (see \citet{schulz-kuhn-2017-multi} for an overview). Recent work has primarily focused on using neural sequence-to-sequence models. \citet{hamalainen-hengchen-2019-paft} use a BiLSTM encoder-decoder with attention for historical English post-correction. Similar to our base model, \citet{dong-smith-2018-multi} use a multi-source model to combine the first pass OCR from duplicate documents in English. 

There has been little work on lower-resourced languages. \citet{kolak-resnik-2005-ocr} present a probabilistic edit distance based post-correction model applied to Cebuano and Igbo, and \citet{krishna-etal-2018-upcycle} show improvements on Romanized Sanksrit OCR by adding a copy mechanism to a neural sequence-to-sequence model.

Multi-source encoder-decoder models have been used for various tasks including machine translation~\cite{zoph-knight-2016-multi,libovicky-helcl-2017-attention} and morphological inflection~\cite{kann-etal-2017-neural,anastasopoulos-neubig-2019-pushing}. Perhaps most relevant to our work is the multi-source model presented by \citet{anastasopoulos+chiang:interspeech2018}, which uses high-resource translations to improve speech transcription of lower-resourced languages.

Finally, \citet{bustamante-etal-2020-data} construct corpora for four endangered languages from text-based PDFs using rule-based heuristics. Data creation from such unstructured text files is an important research direction, complementing our method of extracting data from scanned images.
\section{Problem Setting}
\label{sec:setting}
In this section, we first define the task of OCR post-correction and introduce how we incorporate translations into the correction model. Next, we discuss the sources from which we obtain scanned documents containing endangered language texts.

\subsection{Formulation}
\label{sec:formulation}
\paragraph{Optical Character Recognition} OCR tools are trained to find the best transcription corresponding to the text in an image. The system typically consists of a recognition model that produces candidate text sequences conditioned on the input image and a language model that determines the probability of these sequences in the target language. We use a general-purpose OCR system (detailed in \autoref{sec:analysis}) to produce a \emph{first pass transcription} of the endangered language text in the image. Let this be a sequence of characters $\boldsymbol{x} = [x_1, \dots, x_N]$.

\paragraph{OCR post-correction} The goal of post-correction is to reduce recognition errors in the first pass transcription --- often caused by low quality scanning, physical deterioration of the paper book, or diverse layouts and typefaces~\cite{dong-smith-2018-multi}. The focus of our work is on using post-correction to counterbalance the lack of OCR training data in the target endangered languages. The correction model takes $\boldsymbol{x}$ as input and produces the \emph{final transcription} of the endangered language document, a sequence of characters $\boldsymbol{y} = [y_1, \dots , y_K]$. 
$$\boldsymbol{y} = \argmax_{\boldsymbol{y'}} p_\text{corr}(\boldsymbol{y'}|\boldsymbol{x})$$

\noindent
\textbf{Incorporating translations}\quad We use information from high-resource translations of the endangered language text. These translations are commonly found within the same paper book or linguistic archive (e.g., \autoref{fig:dataset_example}). We use an existing OCR system to obtain a transcription of the scanned translation, a sequence of characters $\boldsymbol{t} = [t_1, \dots, t_M]$. This is used to condition the model:
$$\boldsymbol{y} = \argmax_{\boldsymbol{y'}} p_\text{corr}(\boldsymbol{y'}|\boldsymbol{x}, \boldsymbol{t})$$

\subsection{Endangered Language Documents}
We explore online archives to determine how many scanned documents in endangered languages exist as potential sources for data extraction (as of this writing, October 2020).

The Internet Archive,\footnote{\url{https://archive.org/}} a general-purpose archive of web content, has scanned books labeled with the language of their content. We find 11,674 books labeled with languages classified as \ba\ba endangered'' by UNESCO.
Additionally, we find that endangered language linguistic archives contain thousands of documents in PDF format --- the Archive of the Indigenous Languages of Latin America (AILLA)\footnote{\url{https://ailla.utexas.org}} contains $\approx$10,000 such documents and the Endangered Languages Archive (ELAR)\footnote{\url{https://elar.soas.ac.uk/}} has $\approx$7,000. 

\medskip
\noindent
\textbf{How common are translations?} As described in the introduction, endangered language documents often contain a translation into another (usually high-resource) language. While it is difficult to estimate the number of documents with translations, multilingual documents represent the majority in the archives we examined; AILLA contains 4,383 PDFs with bilingual text and 1,246 PDFs with trilingual text, while ELAR contains $\approx$5,000 multilingual documents. The structure of translations in these documents is varied, from dictionaries and interlinear glosses to scanned multilingual books.

\section{Benchmark Dataset}
\label{sec:dataset}
From the sources described above, we select documents from three critically endangered languages\footnote{UNESCO defines critically endangered languages as those where the youngest speakers are grandparents and older, and they speak the language partially and infrequently.} for annotation --- Ainu, Griko, and Yakkha. These languages were chosen in an effort to create a geographically, typologically, and orthographically diverse benchmark. We focus this initial study on scanned images of printed books as opposed to handwritten notes, which are a relatively more challenging domain for OCR. 

We manually transcribed the text corresponding to the endangered language content. The text corresponding to the translations is not manually transcribed. We also aligned the endangered language text to the OCR output on the translations, per the formulation in \autoref{sec:formulation}. We describe the annotated documents below and example images from our dataset are in \autoref{fig:dataset_example} (a), (b), (c).

\smallskip
\textbf{Ainu} is a severely endangered indigenous language from northern Japan, typically considered a language isolate. In our dataset, we use a book of Ainu epic poetry (\textit{yukara}), with the ``Kutune Shirka" yukara~\cite{kindaichi1931ainu} in Ainu transcribed in Latin script.\footnote{Some transcriptions of Ainu also use the Katakana script. See \citet{howell1951classification} for a discussion on Ainu folklore.} Each page in the book has a two-column structure --- the left column has the Ainu text, and the right has its Japanese translation already aligned at the line-level, removing the need for manual alignment (see \autoref{fig:dataset_example} (a)). The book has 338 pages in total. Given the effort involved in annotation, we transcribe the Ainu text from 33 pages, totaling 816 lines.

\smallskip
\textbf{Griko} is an endangered Greek dialect spoken in southern Italy. The language uses a combination of the Latin alphabet and the Greek alphabet as its writing system. The document we use is a book of Griko folk tales compiled by \citet{stomeo1980racconti}. The book is structured such that in each fold of two pages, the left page has Griko text, and the right page has the corresponding translation in Italian. Of the 175 pages in the book, we annotate the Griko text from 33 pages and manually align it at the sentence-level to the Italian translation. This results in 807 annotated Griko sentences.

\smallskip
\textbf{Yakkha} is an endangered Sino-Tibetan language spoken in Nepal. It uses the Devanagari writing system. We use scanned images of three children's books, each of which has a story written in Yakkha along with its translation in Nepali and English~\cite{yakkha-elar}. We manually transcribe the Yakkha text from all three books. We also align the Yakkha text to both the Nepali and the English OCR at the sentence level with the help of an existing Yakkha dictionary~\cite{Schackow_2015}. In total, we have 159 annotated Yakkha sentences.

\section{Experiments}
\label{sec:experiments}
\begin{table*}[tb]
    \centering
    \small
    \begin{tabular}{l|r@{\ \ }rr@{\ \ }rr@{\ \ }r|r@{\ \ }rr@{\ \ }rr@{\ \ }r}
    \toprule
        & \multicolumn{6}{c|}{Character Error Rate} &\multicolumn{6}{c}{Word Error Rate} \\
        & \multicolumn{2}{c}{Ainu} & \multicolumn{2}{c}{Griko} & \multicolumn{2}{c|}{Yakkha} & \multicolumn{2}{c}{Ainu} & \multicolumn{2}{c}{Griko} & \multicolumn{2}{c}{Yakkha} \\[-0.3em]
        Model & \small Multi & \small Single & \small Multi & \small Single & \small Multi & \small Single & \small Multi & \small Single & \small Multi & \small Single & \small Multi & \small Single\\
        \midrule
        \textsc{Fp-Ocr} & \multicolumn{1}{c}{--} & $1.34$ & \multicolumn{1}{c}{--} & $3.27$ & \multicolumn{1}{c}{--} & $8.90$ & \multicolumn{1}{c}{--} & $6.27$ & \multicolumn{1}{c}{--} & $15.63$ & \multicolumn{1}{c}{--} & $31.64$ \\
        \textsc{Base} & $1.56$ & $1.41$ & $6.78$ & $5.95$ & $70.39$ & $71.71$ & $8.56$ & $7.88$ & $15.13$ & $13.67$ & $98.15$ & $99.10$  \\
        \textsc{Copy} & $2.04$ & $1.99$ & $2.54$ & $2.28$ & $14.77$ & $12.30$ & $9.48$ & $8.57$ & $9.33$ & $8.90$ & $30.36$ & $27.81$ \\
        \textsc{Ours} & $0.92$ & $\boldsymbol{0.80}$ & $\boldsymbol{1.66}$ & $1.70$ & $\boldsymbol{7.75}$ & $8.44$ & $5.75$ & $\boldsymbol{5.19}$ & $\boldsymbol{7.46}$ & $7.51$ & $\boldsymbol{20.95}$ & $21.33$ \\
    \bottomrule
    \end{tabular}
    \caption{Our method improves performance over all baselines (10-fold cross-validation averaged over five randomly seeded runs). We present multi- and single-source variants and \textbf{highlight} the best model for each language.}
    \label{tab:cer}
\end{table*}

This section discusses our experimental setup and the post-correction performance on the three endangered languages on our dataset.

\subsection{Experimental Setup}
\smallskip
\paragraph{Data Splits}
We perform 10-fold cross-validation for all experimental settings because of the small size of the datasets. For each language, we divide the transcribed data into 11 segments --- we use one segment for creating the \emph{denoising rules} described in the previous section and the remaining ten as the folds for cross-validation. In each cross-validation fold, eight segments are used for training, one for validation and one for testing.

We divide the dataset at the page-level for the Ainu and Griko documents, resulting in 11 segments of three pages each. For the Yakkha documents, we divide at the paragraph-level, due to the small size of the dataset. We have 33 paragraphs across the three books in our dataset, resulting in 11 segments that contain three paragraphs each. The multi-source results for Yakkha reported in \autoref{tab:cer} use the English translations. Results with Nepali are similar and are included in \autoref{sec:appendix}.

\paragraph{Metrics}
We use two metrics for evaluating our systems: character error rate (CER) and word error rate (WER). Both metrics are based on edit distance and are standard for evaluating OCR and OCR post-correction~\cite{berg-kirkpatrick-etal-2013-unsupervised,schulz-kuhn-2017-multi}. 
CER is the edit distance between the predicted and the gold transcriptions of the document, divided by the total number of characters in the gold transcription. WER is similar but is calculated at the word level.

\paragraph{Methods}
In our experiments, we compare the performance of our proposed method with the first pass OCR and with two systems from recent work in OCR post-correction. All the post-correction methods have two variants -- the single-source model with only the endangered language encoder and the multi-source model that additionally uses the high-resource translation encoder.
\begin{itemize}
    \item \textsc{Fp-Ocr}: The first pass transcription obtained from the Google Vision OCR system.
    \item \textsc{Base}: This system is the base sequence-to-sequence architecture described in \autoref{sec:base}. Both the single-source and multi-source variants of this system are used for English OCR post-correction in~\citet{dong-smith-2018-multi}. 
    \item \textsc{Copy}: This system is the base architecture with a copy mechanism as described in \autoref{sec:recipe1}. The single-source variant of this model is used for OCR post-correction on Romanized Sanskrit in \citet{krishna-etal-2018-upcycle}.\footnote{Although~\citet{krishna-etal-2018-upcycle} use BPE tokenization, preliminary experiments showed that character-level models result in much better performance on our dataset, likely due to the limited data available for training the BPE model.}
    \item \textsc{Ours}: The model with all the adaptations proposed in \autoref{sec:recipe1} and \autoref{sec:recipe2}.
\end{itemize}

\paragraph{Implementation} The post-correction models are implemented using the DyNet neural network toolkit~\cite{dynet}, and all reported results are the average of five training runs with different random seeds. We assume knowledge of the entire alphabet of the endangered language for all the methods, which is straightforward to obtain for most languages. The decoder's vocabulary contains all these characters, irrespective of their presence in the training data, with corresponding randomly-initialized character embeddings.

\subsection{Main Results}
\label{sec:results}
\renewcommand{\arraystretch}{1.0}
\begin{figure*}[tb]
    \centering
    \small
    \begin{tabular}{lcc}
    & \multicolumn{2}{c}{Errors \textit{fixed} by post-correction}\\[.1cm]
        & (a) Griko & (b) Yakkha  \\
        \raisebox{0.7em}{[Image]} & \frame{\includegraphics[width=0.4\columnwidth]{images/errors1a.pdf}} & \frame{\includegraphics[width=0.3\columnwidth]{images/errors2a.pdf}} \\
        & \multicolumn{2}{c}{$\big\downarrow$ \hspace{3cm} $\big\downarrow$}\\
        \raisebox{0.35em}{[First pass OCR]} & \raisebox{0.3em}{\large e\textcolor{burntred}{\textbf{x}}i i ka\textcolor{burntred}{\textbf{dd}}in\`ara} &
        \includegraphics[width=0.25\columnwidth]{images/error2b.pdf} \\
        & \multicolumn{2}{c}{$\big\downarrow$ \hspace{3cm} $\big\downarrow$}\\
        \raisebox{0.35em}{[Post-corrected]} & \raisebox{0.35em}{\large e\textcolor{burntblue}{$\bm{\chi}$}i i ka\textbf{\textcolor{burntblue}{\d{d}\d{d}}}in\`ara} &
        \includegraphics[width=0.28\columnwidth]{images/error2c.pdf} \\
    \end{tabular}
    \qquad
    \begin{tabular}{cc}
        \multicolumn{2}{c}{Errors \textit{introduced} by post-correction}\\[.1cm]
        (c) Griko & (d) Yakkha  \\
        \frame{\includegraphics[width=0.25\columnwidth]{images/errors3a.pdf}} & \frame{\includegraphics[width=0.35\columnwidth]{images/errors4a.pdf}} \\
        \multicolumn{2}{c}{$\big\downarrow$ \hspace{2.5cm} $\big\downarrow$}\\
        \raisebox{0.35 em}{\large{\`{e} ffacilo}} &
        \includegraphics[width=0.3\columnwidth]{images/errors4b.pdf} \\
        \multicolumn{2}{c}{$\big\downarrow$ \hspace{2.5cm} $\big\downarrow$}\\
        \raisebox{0.35 em}{\large{\`{e} ffa\textcolor{burntred}{\textbf{\'{c}}}ilo}} &
        \includegraphics[width=0.32\columnwidth]{images/errors4c.pdf}
    \end{tabular}
    \caption{Our model fixes many mixed script and uncommon diacritics errors such as (a) and (b). In rare cases, it ``over-corrects" the first pass OCR transcription, introducing errors such as (c) and (d).}
    \label{fig:error_examples}
\end{figure*}

\begin{figure}[t]
    \definecolor{graphblue}{HTML}{A6BDDB}
\pgfplotstableread[row sep=\\,col sep=&]{
det & Ainu & Griko & Yakkha \\
all & 5.19 & 7.46 & 24.29 \\
diag & 5.49 & 8.06 & 22.73 \\
copy & 6.56 & 8.66 & 37.83 \\
coverage & 5.60 & 10.19 & 26.71 \\
dec & 6.86 & 7.87 & 20.95 \\
enc & 6.70 & 9.47 & 28.41 \\
seq2seq & 5.65 & 9.43 & 27.68 \\
}\data
\def\mystrut{\vphantom{hp}}

%\vspace{-1em}
\begin{tikzpicture}[trim left=-1.6cm,trim right=0cm]
    \begin{axis}[
            xbar,
            every axis plot post/.style={/pgf/number format/fixed},
            bar width=.23cm,
            width=4cm,
            height=4cm,
            ymajorgrids=false,
            yminorgrids=false,
            xmajorgrids=false,
            %every axis legend/.code={\let\addlegendentry\relax},
            %legend={Wikipedia Size (in million articles)},
            legend style={draw=none,at={(0.2,0.9)},anchor=west},
            symbolic y coords={all,diag,copy,coverage,dec,enc,seq2seq},
            ytick={all,diag,copy,coverage,dec,enc,seq2seq},
            yticklabels={all,-diag,-copy,-coverage,-pretr. dec,-pretr. enc,-pretr. s2s},
            %ytick={all,-diag loss,-copy,-coverage, -pretrain dec,-pretrain enc,-pretrain seq2seq},
            every y tick label/.append style={font=\small\mystrut},
            every x tick label/.append style={font=\small\mystrut},
            tick pos=left,
            %hide y axis,
            axis x line*=bottom,
            axis y line*=left,
            nodes near coords,
            %nodes near coords align={vertical},
            every node near coord/.append style={font=\small,color=black},
            %nodes near coords style={},
            title={\small Ainu},
            title style={yshift=-0.3cm},
            %ymin=0,ymax=6.1,
            xmin=0,xmax=8,
            %ylabel shift={-1cm},
            %ylabel near ticks,
            %ylabel={},
            %xlabel near ticks,
            %xlabel={WER},
            enlarge x limits=0.0,
            xtick style={draw=none}
        ]
        %\addplot [style={black,postaction={pattern=north east lines},fill=white,mark=none}] table[x=story,y=base]{\ainudata};
        \addplot [style={graphblue,fill=graphblue,mark=none}] table[x=Ainu,y=det]{\data};
        %\legend{Wikipedia Size (in million articles)}
    \end{axis}
\end{tikzpicture}
\begin{tikzpicture}[trim left=-3cm,trim right=0cm]
    \begin{axis}[
            xbar,
            every axis plot post/.style={/pgf/number format/fixed},
            bar width=.23cm,
            width=3.5cm,
            height=4cm,
            ymajorgrids=false,
            yminorgrids=false,
            xmajorgrids=false,
            %every axis legend/.code={\let\addlegendentry\relax},
            %legend={Wikipedia Size (in million articles)},
            legend style={draw=none,at={(0.2,0.9)},anchor=west},
            symbolic y coords={all,diag,copy,coverage,dec,enc,seq2seq},
            %ytick={all,diag,copy,coverage,dec,enc,seq2seq},
            %ytick={,,,,,,},
            xtick={0,2,4,6,8,10},
            yticklabels={,,,,,,},
            %ytick={all,-diag loss,-copy,-coverage, -pretrain dec,-pretrain enc,-pretrain seq2seq},
            every y tick label/.append style={font=\small\mystrut},
            every x tick label/.append style={font=\small\mystrut},
            %tick pos=none,
            %hide y axis,
            axis x line*=bottom,
            axis y line*=left,
            nodes near coords,
            %nodes near coords align={vertical},
            every node near coord/.append style={font=\small,color=black},
            %nodes near coords style={},
            title={\small Griko},
            title style={yshift=-0.3cm},
            %ymin=0,ymax=6.1,
            xmin=0,xmax=11,
            %ylabel shift={-1cm},
            %ylabel near ticks,
            %ylabel={},
            %xlabel near ticks,
            %xlabel={WER},
            enlarge x limits=0.0,
            xtick style={draw=none}
        ]
        %\addplot [style={black,postaction={pattern=north east lines},fill=white,mark=none}] table[x=story,y=base]{\ainudata};
        \addplot [style={graphblue,fill=graphblue,mark=none}] table[x=Griko,y=det]{\data};
        %\legend{Wikipedia Size (in million articles)}
    \end{axis}
\end{tikzpicture}

\begin{tikzpicture}[trim left=-1.6cm,trim right=0cm]
    \begin{axis}[
            xbar,
            every axis plot post/.style={/pgf/number format/fixed},
            bar width=.23cm,
            width=6.5cm,
            height=4cm,
            ymajorgrids=false,
            yminorgrids=false,
            xmajorgrids=false,
            %every axis legend/.code={\let\addlegendentry\relax},
            %legend={Wikipedia Size (in million articles)},
            legend style={draw=none,at={(0.2,0.9)},anchor=west},
            symbolic y coords={all,diag,copy,coverage,dec,enc,seq2seq},
            ytick={all,diag,copy,coverage,dec,enc,seq2seq},
            yticklabels={all,-diag,-copy,-coverage,-pretr. dec,-pretr. enc,-pretr. s2s},
            %ytick={all,-diag loss,-copy,-coverage, -pretrain dec,-pretrain enc,-pretrain seq2seq},
            every y tick label/.append style={font=\small\mystrut},
            every x tick label/.append style={font=\small\mystrut},
            tick pos=left,
            %hide y axis,
            axis x line*=bottom,
            axis y line*=left,
            nodes near coords,
            %nodes near coords align={vertical},
            every node near coord/.append style={font=\small,color=black},
            %nodes near coords style={},
            title={\small Yakkha},
            title style={yshift=-0.3cm},
            %ymin=0,ymax=6.1,
            xmin=0,xmax=42,
            xlabel shift={-0.2cm},
            %ylabel near ticks,
            %ylabel={},
            xlabel near ticks,
            xlabel={\small Word Error Rate},
            enlarge x limits=0.0,
            enlarge y limits=0.1,
            xtick style={draw=none},
            % xtick align=outside
            % ytick style={draw=none}
        ]
        %\addplot [style={black,postaction={pattern=north east lines},fill=white,mark=none}] table[x=story,y=base]{\ainudata};
        \addplot [style={graphblue,fill=graphblue,mark=none}] table[x=Yakkha,y=det]{\data};
        %\legend{Wikipedia Size (in million articles)}
    \end{axis}
\end{tikzpicture}

    \caption{WER with model component ablations on the best model setting in \autoref{tab:cer}. ``all" includes all the adaptations we propose. Each ablation removes a single component from the ``all" model, e.g. ``-pretr.~s2s" removes the seq-to-seq model pretraining.}
    \label{fig:ablation}
\end{figure}

\autoref{tab:cer} shows the performance of the baselines and our proposed method for each language. Overall, our method results in an improved CER and WER over existing methods across all three languages. 

The \textsc{Base} system does not improve the recognition rate over the first pass transcription, apart from a small decrease in the Griko WER. The performance on Yakkha, particularly, is significantly worse than \textsc{Fp-Ocr}: likely because the data size of Yakkha is much smaller than that of Griko and Ainu, and the model is unable to learn a reasonable distribution. However, on adding a copy mechanism to the base model in the \textsc{Copy} system, the performance is notably better for both Griko and Yakkha. This indicates that adaptations to the base model that cater to specific characteristics of the post-correction task can alleviate some of the challenges of learning from small amounts of data.

The single-source and the multi-source variants of our proposed method improve over the baselines, demonstrating that our proposed model adaptations can improve recognition even without translations. We see that using the high-resource translations results in better post-correction performance for Griko and Yakkha, but the single-source model achieves better accuracy for Ainu. We attribute this to two factors: the very low error rate of the first pass transcription for Ainu and the relatively high error rate (based on manual inspection) of the OCR on the Japanese translation. Despite being a high-resource language, OCR is difficult due to the complexity of Japanese characters and low scan quality. The noise resulting from the Japanese OCR errors likely hurts the multi-source model.



\subsection{Ablation Studies}


Next, we study the effect of our proposed adaptations and evaluate their benefit to the performance of each language. \autoref{fig:ablation} shows the word error rate with models that remove one adaptation from the model with all the adaptations (``all").

For Ainu and Griko, removing any single component increases the WER, with the complete (\ba\ba all'') method performing the best. There is little variance in the Ainu ablations, likely due to the high-quality first pass transcription. 

Our proposed adaptations add the most benefit for Yakkha, which has the fewest training data and relatively less accurate first pass OCR. The copy mechanism is crucial for good performance, but removing the decoder pretraining (\ba\ba pretr.~dec'') leads to the best scores among all the ablations. The denoising rules used to create the pseudo-target data for Yakkha are likely not accurate since they are derived from only three paragraphs of annotated data. Consequently, using it to pretrain the decoder leads to a poor language model.

\subsection{Error Analysis}


We systematically inspect all the recognition errors in the output of our post-correction model to determine the sources of improvement with respect to the first pass OCR. We also examine the types of errors introduced by the post-correction process.

We observe a \emph{91\% reduction} in the number of errors due to mixed scripts and a \emph{58\% reduction} in the errors due to uncommon characters and diacritics (as defined in \autoref{sec:analysis}). Examples of these are shown in \autoref{fig:error_examples} (a) and (b): mixed script errors such as the $\bm{\chi}$ character in Griko and the glottal stop \textbf{\textipa{\textglotstop}} in Yakkha are successfully corrected by the model. The model is also able to correct uncommon character errors like \textbf{\d{d}} in Griko and {\raisebox{-2.65pt}{\includegraphics[height=9.5pt]{images/dev_char.pdf}}} in Yakkha.

Examples of errors introduced by the model are shown in \autoref{fig:error_examples} (c) and (d). Example (c) is in Griko, where the model incorrectly adds a diacritic to a character. We attribute this to the fact that the first pass OCR does not recognize diacritics well; hence, the model learns to add diacritics frequently while generating the output. Example (d) is in Yakkha. The model inserts several incorrect characters, and can likely be attributed to the lack of a good language model due to the relatively smaller amount of training data we have in Yakkha.  

\section{OCR Post-Correction Model}
\label{sec:model}
In this section, we describe our proposed OCR post-correction model. The base architecture of the model is a multi-source sequence-to-sequence framework~\cite{zoph-knight-2016-multi,libovicky-helcl-2017-attention} that uses an LSTM encoder-decoder model with attention~\cite{bahdanau2015neural}. We propose improvements to training and modeling for the multi-source architecture, specifically tailored to ease learning in data-scarce settings.

\subsection{Multi-source Architecture}
\label{sec:base}
{%
\setlength{\fboxsep}{0pt}%
\setlength{\fboxrule}{1pt}%
}%
\begin{figure}[tb]
\tikzset{seq/.style={draw=none,fill=gray!20}}
\tikzset{layer/.style={->,thick}}
\tikzset{label/.style={anchor=west,font={\footnotesize}}}
\tikzset{seqlabel/.style={font={\small}}}
\newcommand{\encoder}[3]{
\draw[seq] (-1.25,-0.25) rectangle (1.25,0.25);
\node[seqlabel] at (0,0) 
%{$\xmat$};
{$#3\step{1}#1 \ldots #3\step{#2}#1$};
\draw[layer] (0,0.3) -- (0,0.7);
\node[seqlabel] at (0,0.5) [label] {encoder};
\draw[seq] (-1.25,0.75) rectangle (1.25,1.25);
\node[seqlabel] at (0,1) %{$\hmat$}; 
{$\hvec\step{1}#1 \ldots \hvec\step{#2}#1$};
}
\newcommand{\ocrencoder}[5]{
\node[inner sep=0pt] at (0,-1.7)
    {\small #5};
\node[inner sep=0pt] at (0,-1.2)
    {\setlength{\fboxsep}{.005\textwidth}%
    \fbox{\includegraphics[width=.145\textwidth]{#4}}};
\draw[layer] (0,-0.9) -- (0,-0.3);
\node[seqlabel] at (0,-0.6) [label] {\textsc{ocr}};
\draw[seq] (-1.25,-0.25) rectangle (1.25,0.25);
\node[seqlabel] at (0,0) 
%{$\xmat$};
{$#3\step{1} \ldots #3\step{#2}$};
\draw[layer] (0,0.3) -- (0,0.9);
\node[seqlabel] at (0,0.6) [label] {encoder};
\draw[seq] (-1.25,0.95) rectangle (1.25,1.45);
\node[seqlabel] at (0,1.2) %{$\hmat$}; 
{$\hvec\step{1}#1 \ldots \hvec\step{#2}#1$};
}
\newcommand{\decoder}[1]{
\draw[seq] (-1,2.05) rectangle (1,2.55);
\node[seqlabel] at (0,2.3) %{$\cmat#1$}; 
{$\cvec\step{1}#1 \ldots \cvec\step{K#1}#1$};
\draw[layer] (0,2.6) -- (0,3.2);
\node[seqlabel] at (0,2.9) [label] {decoder};
\draw[seq] (-1,3.25) rectangle (1,3.75);
\node[seqlabel] at (0,3.5) %{$\smat#1$}; 
{$\svec\step{1}#1 \ldots \svec\step{K#1}#1$};
\draw[layer] (0,3.8) -- (0,4.4);
\node[seqlabel] at (0,4.1) [label] {softmax};
\draw[seq] (-1,4.45) rectangle (1,5.05);
\node[seqlabel] at (0,4.75) %{$P(\ymat#1)$}; 
{$P(\yvec\step{1}#1 \ldots \yvec\step{K#1}#1)$};
}
\begin{center}
\resizebox{0.7\hsize}{!}{
\begin{tabular}{c}
\begin{tikzpicture}
\begin{scope}[xshift=-1.4cm]
% \encoder{^x}{N}{\xvec}
\ocrencoder{^x}{N}{\xvec}{images/ainu_frame.png}{Ainu}
\end{scope}
\begin{scope}[xshift=1.4cm]
% \encoder{^t}{M}{\tvec}
\ocrencoder{^t}{M}{\tvec}{images/japanese_frame.png}{Japanese}
\end{scope}
\draw[layer] (-1.4,1.5) -- (0,2.0);
\draw[layer] (1.4,1.5) -- (0,2.0);
\node at (-1,1.85) [label,anchor=east] {attention};
\node at (1,1.85) [label] {attention};
\decoder{}
\end{tikzpicture}
\end{tabular}%
}
\end{center}
\caption{The proposed multi-source architecture with the encoder for an endangered language segment (left) and an encoder for the translated segment (right). The input to the encoders is the first pass OCR over the scanned images of each segment. For example, the OCR on the scanned images of some Ainu text (left) and its Japanese translation (right).}
\label{fig:multisourcemodels}
\end{figure}
Our post-correction formulation takes as input the first pass OCR of the endangered language segment $\boldsymbol{x}$ and the OCR of the translated segment $\boldsymbol{t}$, to predict an error-free endangered language text $\boldsymbol{y}$. The model architecture is shown in \autoref{fig:multisourcemodels}.

The model consists of two encoders --- one that encodes $\boldsymbol{x}$ and one that encodes $\boldsymbol{t}$. Each encoder is a character-level bidirectional LSTM~\cite{hochreiter1997long} and transforms the input sequence of characters to a sequence of hidden state vectors: $\mathbf{h}^x$ for the endangered language text and $\mathbf{h}^t$ for the translation.

The model uses an attention mechanism during the decoding process to use information from the encoder hidden states. We compute the attention weights over each of the two encoders independently. At the decoding time step $k$:
\begin{align}
e^x_{k,i}=\mathbf{v}^x \tanh\left(\mathbf{W}_1^x \mathbf{s}_{k-1} + \mathbf{W}_2^x \mathbf{h}^x_i\right)
\label{eq:attn}
\end{align}
$$\boldsymbol{\alpha}_k^x = \mathrm{softmax}\left(\mathbf{e}_k^x\right)$$
$$\mathbf{c}^x_k = \left[\Sigma_i \alpha^x_{k,i} \mathbf{h}^x_i\right]$$
\noindent
where $\mathbf{s}_{k-1}$ is the decoder state of the previous time step and $\mathbf{v}^x$, $\mathbf{W}_1^x$ and $\mathbf{W}_2^x$ are trainable parameters. The encoder hidden states $\mathbf{h}^x$ are weighted by the attention distribution $\boldsymbol{\alpha}^x_k$ to produce the context vector $\mathbf{c}^x_k$. We follow a similar procedure for the second encoder to produce $\mathbf{c}^t_k$.
We concatenate the context vectors to combine attention from both sources~\cite{zoph-knight-2016-multi}:
$$\mathbf{c}_k=\left[\mathbf{c}_k^x;\mathbf{c}_k^t\right]$$
$\mathbf{c}_k$ is used by the decoder LSTM to compute the next hidden state $\mathbf{s}_k$ and subsequently, the probability distribution for predicting the next character $\mathbf{y}_k$ of the target sequence $\boldsymbol{y}$:
\begin{align} 
\mathbf{s}_k &= \mathrm{lstm}\left(\mathbf{s}_{k-1}, \mathbf{c}_k, \mathbf{y}_{k-1}\right)\\
P\left(\mathbf{y}_k\right) &= \mathrm{softmax}\left(\mathbf{W}\mathbf{s}_k + \mathbf{b}\right)
\label{eq:decoder}
\end{align}

\paragraph{Training and Inference} The model is trained for each language with the cross-entropy loss ($\mathcal{L}_\mathrm{ce}$) on the small amount of transcribed data we have. Beam search is used at inference time.


\subsection{Model and Training Improvements}
\label{sec:recipe1}

With the minimal annotated data we have, it is challenging for the neural network to learn a good distribution over the target characters. We propose a set of adaptations to the base architecture that improves the post-correction performance without additional annotation. The adaptations are based on characteristics of the OCR task itself and the performance of the upstream OCR tool (\autoref{sec:analysis}).

\paragraph{Diagonal attention loss} As seen in \autoref{sec:analysis}, substitution errors are more frequent in the OCR task than insertions or deletions; consequently, we expect the source and target to have similar lengths. Moreover, post-correction is a monotonic sequence-to-sequence task, and reordering rarely occurs~\cite{schnober-etal-2016-still}. 
Hence, we expect attention weights to be higher at characters close to the diagonal for the endangered language encoder.

We modify the model such that all the elements in the attention vector that are not within $j$ steps (we use $j=3$) of the current time step $k$ are added to the training loss, thereby encouraging elements away from the diagonal to have lower values. The diagonal loss summed over all time steps for a training instance, where $N$ is the length of $\boldsymbol{x}$, is:
$$\mathcal{L}_\mathrm{diag} = \sum_k \left(\sum_{i=1}^{k-j} \alpha^x_{k,i} + \sum_{i=k+j}^N \alpha^x_{k,i}\right)$$

\paragraph{Copy mechanism} \autoref{tab:google_metrics} indicates that the first pass OCR predicts a majority of the characters accurately. In this scenario, enabling the model to directly copy characters from the first pass OCR rather than generate a character at each time step might lead to better performance~\cite{gu-etal-2016-incorporating}.

We incorporate the copy mechanism proposed in~\citet{see-etal-2017-get}. The mechanism computes a \ba\ba generation probability'' at each time step $k$, which is used to choose between \emph{generating} a character (\autoref{eq:decoder}) or \emph{copying} a character from the source text by sampling from the attention distribution $\boldsymbol{\alpha}_k^x$.

\paragraph{Coverage} Given the monotonicity of the post-correction task, the model should not attend to the same character repeatedly. However, repetition is a common problem for neural encoder-decoder models~\cite{mi-etal-2016-coverage,tu-etal-2016-modeling}. To account for this problem, we adapt the coverage mechanism from~\citet{see-etal-2017-get}, which keeps track of the attention distribution over past time steps in a coverage vector. For time step $k$, the coverage vector would be $\mathbf{g}_k = \sum_{k'=0}^{k-1} \boldsymbol{\alpha}^x_{k'}$. 

$\mathbf{g}_k$ is used as an extra input to the attention mechanism, ensuring that future attention decisions take the weights from previous time steps into account. \autoref{eq:attn}, with learnable parameter $\mathbf{w}_g$, becomes:
$$e^x_{k,i}=\mathbf{v}^x \tanh\left(\mathbf{W}_1^x \mathbf{s}_{k-1} + \mathbf{W}_2^x \mathbf{h}^x_i + \mathbf{w}_g g_{k,i}\right)$$
$\mathbf{g}_k$ is also used to penalize attending to the same locations repeatedly with a coverage loss. The coverage loss summed over all time steps $k$ is:
$$\mathcal{L}_\mathrm{cov} = \sum_k \sum_{i=1}^n \min\left(\alpha_{k,i}^x, g_{k,i}\right)$$
Therefore, with our model adaptations, the loss for a single training instance:
\begin{align}
\mathcal{L} = \mathcal{L}_\mathrm{ce} + \mathcal{L}_\mathrm{diag} + \mathcal{L}_\mathrm{cov}
\label{eq:loss}
\end{align}

\subsection{Utilizing Untranscribed Data}
\label{sec:recipe2}

As discussed in \autoref{sec:dataset}, given the effort involved, we transcribe only a subset of the pages in each scanned book.
Nonetheless, we leverage the remaining unannotated pages for pretraining our model. We use the upstream OCR tool to get a first pass transcription on all the unannotated pages.

We then create \ba\ba pseudo-target'' transcriptions for the endangered language text as described below:
\begin{itemize}
    \item \textbf{Denoising rules}\quad Using a small fraction of the available annotated pages, we compute the edit distance operations between the first pass OCR and the gold transcription. The operations of each type (insertion, deletion, and replacement) are counted for each character and divided by the number of times that character appears in the first pass OCR. This forms a probability of how often the operation should be applied to that specific character.
    
    We use these probabilities to form rules, such as $p(\text{replace d with \d{d}})\!=\!0.4$ for Griko and $p(\text{replace ? with \textipa{\textglotstop}})\!=\!0.7$ for Yakkha. These rules are applied to remove errors from, or \ba\ba denoise'', the first pass OCR transcription.
    \item \textbf{Sentence alignment}\quad We use Yet Another Sentence Aligner~\cite{yasa-1336} for unsupervised alignment of the endangered language and translation on the unannotated pages. 
\end{itemize}
Given the aligned first pass OCR for the endangered language text and the translation along with the pseudo-target text, $\boldsymbol{x}$, $\boldsymbol{t}$ and $\boldsymbol{\hat{y}}$ respectively, the pretraining steps, in order, are:

\begin{itemize}
    \item \textbf{Pretraining the encoders}\quad We pretrain both the forward and backward LSTMs of each encoder with a character-level language model objective: the endangered language encoder on $\boldsymbol{x}$ and the translation encoder on $\boldsymbol{t}$.
    \item \textbf{Pretraining the decoder}\quad The decoder is pretrained on the pseudo-target $\boldsymbol{\hat{y}}$ with a character-level language model objective.
    \item \textbf{Pretraining the seq-to-seq model}\quad The model is pretrained with $\boldsymbol{x}$ and $\boldsymbol{t}$ as the sources and the pseudo-target $\boldsymbol{\hat{y}}$ as the target transcription, using the post-correction loss function~$\mathcal{L}$ as defined in \autoref{eq:loss}.
\end{itemize}

\section{Introduction}
\begin{figure}[t]
\centering
    \begin{subfigure}[t]{\columnwidth}
    \centering
      \caption{Ainu (left) -- Japanese (right)}
      \vspace{-0.5em}
      \fbox{\includegraphics[width=0.9\columnwidth]{images/ainu_ex.png}}
      \vspace{0.6em}
    \end{subfigure}
    \begin{subfigure}[t]{\columnwidth}
      \centering
      \caption{Griko (top) -- Italian (bottom)}
      \vspace{-0.5em}
      \frame{\includegraphics[width=0.93\columnwidth]{images/griko_example.pdf}}
      \vspace{0.7em}
    \end{subfigure}
    \begin{subfigure}[t]{\columnwidth}
      \centering
      \caption{Yakkha (top) -- Nepali (middle) -- English (bottom)}
      \vspace{-0.5em}
      \fbox{\includegraphics[width=0.9\columnwidth]{images/figure2c.pdf}}
      \vspace{0.6em}
    \end{subfigure}
    \footnotesize{(d) Handwritten Shangaji -- typed English glosses}
    \begin{tabular}{|@{\ \ }c@{\ \ }|}
    \hline
     \begin{subfigure}[t]{0.9\columnwidth}
      \centering
      \includegraphics[width=0.9\columnwidth]{images/sha_img.pdf}
      \includegraphics[width=0.5\columnwidth]{images/sha_text.pdf}
    \end{subfigure}\\
    \hline
    \end{tabular}
    \caption{Examples of scanned documents in endangered languages accompanied by translations from the same scanned book (a, b, c) or linguistic archive (d).}
    \label{fig:dataset_example}
    \vspace{-1.2em}
\end{figure}

Natural language processing (NLP) systems exist for a small fraction of the world's over 6,000 living languages, the primary reason being the lack of resources required to train and evaluate models. Technological advances are concentrated on languages that have readily available data, and most other languages are left behind~\cite{joshi2020state}. This is particularly notable in the case of endangered languages, i.e., languages that are in danger of becoming extinct due to dwindling numbers of native speakers and the younger generations shifting to using other languages. For most endangered languages, finding \emph{any} data at all is challenging.

In many cases, natural language text in these languages does exist. However, it is locked away in formats that are not machine-readable --- paper books, scanned images, and unstructured web pages. These include books from local publishing houses within the communities that speak endangered languages, such as educational or cultural materials. Additionally, linguists documenting these languages also create data such as word lists and interlinear glosses, often in the form of handwritten notes. Examples from such scanned documents are shown in~\autoref{fig:dataset_example}. Digitizing the textual data from these sources will not only enable NLP for endangered languages but also aid linguistic documentation, preservation, and accessibility efforts.

In this work, we create a benchmark dataset and propose a suite of methods to extract data from these resources, focusing on scanned images of paper books containing endangered language text. Typically, this sort of digitization requires an optical character recognition (OCR) system. However, the large amounts of textual data and transcribed images needed to train state-of-the-art OCR models from scratch are unavailable in the endangered language setting. Instead, we focus on \emph{post-correcting} the output of an off-the-shelf OCR tool that can handle a variety of scripts. We show that targeted methods for post-correction can significantly improve performance on endangered languages.

Although OCR post-correction is relatively well-studied, most existing methods rely on considerable resources in the target language, including a substantial amount of textual data to train a language model~\cite{schnober-etal-2016-still,dong-smith-2018-multi,8978127} or to create synthetic data~\cite{krishna-etal-2018-upcycle}. While readily available for high-resource languages, these resources are severely limited in endangered languages, preventing the direct application of existing post-correction methods in our setting. 

As an alternative, we present a method that compounds on previous models for OCR post-correction, making three improvements tailored to the data-scarce setting. First, we use a \textbf{multi-source model} to incorporate information from the high-resource translations that commonly appear in endangered language books. These translations are usually in the \textit{lingua franca} of the region (e.g., \autoref{fig:dataset_example} (a,b,c)) or the documentary linguist's primary language (e.g., \autoref{fig:dataset_example} (d) from \citet{shangaji-elar}). Next, we introduce \textbf{structural biases} to ease learning from small amounts of data. Finally, we add \textbf{pretraining methods} to utilize the little unannotated data that exists in endangered languages.

\medskip
\noindent
We summarize our main contributions as follows:
\begin{itemize}[leftmargin=*, itemsep=3pt]
    \item A benchmark dataset for OCR post-correction on three critically endangered languages: Ainu, Griko, and Yakkha.
    \item A systematic analysis of a general-purpose OCR system, demonstrating that it is not robust to the data-scarce setting of endangered languages.
    \item An OCR post-correction method that adapts the standard neural encoder-decoder framework to the highly under-resourced endangered language setting, reducing both the character error rate and the word error rate by 34\% over a state-of-the-art general-purpose OCR system.
\end{itemize}

\section{OCR Systems: Promises and Pitfalls}
\label{sec:analysis}
As briefly alluded to in the introduction, training an OCR model for each endangered language is challenging, given the limited available data. 
Instead, we use the general-purpose OCR system from the Google Vision AI toolkit\footnote{\url{https://cloud.google.com/vision}} to get the first pass OCR transcription on our data.

The Google Vision OCR system~\cite{fujii2017sequence,ingle2019scalable} is highly performant and supports 60 major languages in 29 scripts. It can transcribe a wide range of higher resource languages with high accuracy, ideal for our proposed method of incorporating high-resource translations into the post-correction model. Moreover, it is particularly well-suited to our task because it provides script-specific OCR models in addition to language-specific ones. Per-script models are more robust to unknown languages because they are trained on data from multiple languages and can act as a general character recognizer without relying on a single language's model. Since most endangered languages adopt standard scripts (often from the region's dominant language) as their writing systems, the per-script recognition models can provide a stable starting point for post-correction.

The metrics we use for evaluating performance are character error rate (CER) and word error rate (WER), representing the ratio of erroneous characters or words in the OCR prediction to the total number in the annotated transcription. More details are in \autoref{sec:experiments}. The CER and WER using the Google Vision OCR on our dataset are in \autoref{tab:google_metrics}.

\subsection{OCR Performance}
\begin{table}[tb]
    \centering
    \small
    \begin{tabular}{lcrr}
    \toprule
    Language && CER & WER \\
    \midrule
    Ainu && 1.34 & 6.27 \\
    Griko && 3.27 & 15.63 \\
    Yakkha && 8.90 & 31.64 \\
    \bottomrule
    \end{tabular}
    \caption{Character error rate and word error rate with the Google Vision OCR system on our dataset.}
    \label{tab:google_metrics}
\end{table}
Across the three languages, the error rates indicate that we have a first pass transcription that is of reasonable quality, giving our post-correction method a reliable starting point. We note the particularly low CER for the Ainu data, reflecting previous work that has evaluated the Google Vision system to have strong performance on typed Latin script documents \cite{fujii2017sequence}. However, there remains considerable room for improvement in both CER and WER for all three languages.

Next, we look at the edit distance between the predicted and the gold transcriptions, in terms of insertion, deletion, and replacement operations. Replacement accounts for over 84\% of the errors in the Griko and Ainu datasets, and 55\% overall. This pattern is expected in the OCR task, as the recognition model uses the image to make predictions and is more likely to confuse a character's shape for another than to hallucinate or erase pixels. However, we observe that the errors in the Yakkha dataset do not follow this pattern. Instead, 87\% of the errors for Yakkha occur because of deleted characters.

\subsection{Types of Errors}

\begin{figure}[t]
    \centering
    \small
    \begin{tabular}{ccc}
        \frame{\includegraphics[width=0.4\columnwidth]{images/errors1a.pdf}} & \raisebox{0.7em}{$\xrightarrow{\mathrm{OCR}}$} & \raisebox{0.7em}{\large e\textcolor{burntred}{\textbf{x}}i i ka\textcolor{burntred}{\textbf{dd}}in\`ara} \\[.1cm]
        \frame{\includegraphics[width=0.3\columnwidth]{images/errors2a.pdf}} &
        \raisebox{0.6em}{$\xrightarrow{\mathrm{OCR}}$} &
        \includegraphics[width=0.3\columnwidth]{images/error2b.pdf} \\
    \end{tabular}
    \caption{Examples of errors in Griko (top) and Yakkha (bottom) when using the Google Vision OCR.}
    \label{fig:ocr_errors}
\end{figure}

To better understand the challenges posed by the endangered language setting, we manually inspect all the errors made by the OCR system. While some errors are commonly seen in the OCR task, such as misidentified punctuation or incorrect word boundaries, 85\% of the total errors occur due to specific characteristics of endangered languages that general-purpose OCR systems do not account for. Broadly, they can be categorized into two types, examples of which are shown in \autoref{fig:ocr_errors}:
\begin{itemize}[leftmargin=*, itemsep=2pt, topsep=6pt]
    \item\textbf{Mixed scripts}\quad The existing scripts that most endangered languages adopt as writing systems are often not ideal for comprehensively representing the language. For example, the Devanagari script does not have a grapheme for the glottal stop --- as a solution, printed texts in the Yakkha language use the IPA symbol \ba\textipa{\textglotstop}'~\cite{Schackow_2015}. Similarly, both Greek and Latin characters are used to write Griko. The Google Vision OCR is trained to detect script at the line-level and is not equipped to handle multiple scripts within a single word. As seen in \autoref{fig:ocr_errors}, the system does not recognize the Greek character $\bm{\chi}$ in Griko and the IPA symbol \textbf{\textipa{\textglotstop}} in Yakkha. Mixed scripts cause 11\% of the OCR errors.
    \item\textbf{Uncommon characters and diacritics}\quad Endangered languages often use graphemes and diacritics that are part of the standard script but are not commonly seen in high-resource languages. Since these are likely rare in the OCR system's training data, they are frequently misidentified, accounting for 74\% of the errors. In \autoref{fig:ocr_errors}, we see that the OCR system substitutes the uncommon diacritic \textbf{\d{d}} in Griko. The system also deletes the Yakkha character {\raisebox{-2.65pt}{\includegraphics[height=9.5pt]{images/dev_char.pdf}}}, which is a \ba half form' alphabet that is infrequent in several other Devanagari script languages (such as Hindi).
    
\end{itemize}
\appendix
\section{Appendix}
\label{sec:appendix}
\subsection{Implementation Details}

\noindent
The hyperparameters used are:
\begin{itemize}[itemsep=0pt]
    \item Character embedding size = 128
    \item Number of LSTM layers = 1
    \item Hidden state size of the LSTM = 256
    \item Attention size = 256
    \item Beam size = 4
    \item For the diagonal loss, $j$ = 3
    \item Minibatch size for training = 1
    \item Maximum number of epochs = 150
    \item Patience for early stopping = 10 epochs
    \item Pretraining epochs for encoder/decoder = 10
    \item Pretraining epochs for seq-to-seq model = 5
\end{itemize}

\noindent
We use the same values of the hyperparameters for each language and all the systems. We select the best model with early stopping on the character error rate of the validation set.


\subsection{Additional Experimental Results}

\begin{table}[b]
    \centering
    \begin{tabular}{@{}lrr@{}}
    \toprule
    Model & CER & WER \\
    \midrule
    \textsc{Fp-Ocr} & $8.90$ & $31.64$ \\
    \textsc{Base} & $70.89$ & $100.00$ \\
    \textsc{Copy} & $11.60$ & $26.74$ \\
    \textsc{Ours} & $7.95$ & $20.83$ \\
    \bottomrule
    \end{tabular}
    \caption{Character error rate (CER) and word error rate (WER) for the Yakkha dataset with the multisource model that uses the OCR on Nepali as the high-resource translation. The table shows the mean over five random runs.}
    \label{tab:nepali}
    \vspace{2em}
\end{table}


\paragraph{Performance on Yakkha with Nepali}\autoref{tab:nepali} shows the performance for the Yakkha dataset when using Nepali as the high-resource translation input to the multisource model. The performance is similar to those of the experiments using the English translations, as presented in \autoref{tab:cer}.

\paragraph{Standard deviation on the main results}\autoref{tab:stddev_cer} and \autoref{tab:stddev_wer} show the character error rate and word error rate respectively including the standard deviation over five randomly seeded runs, corresponding to the systems described in \autoref{tab:cer}. 

\begin{table}[tb]
    \centering
    \small
    (a) Ainu\\
    \begin{tabular}{l|r@{\ \ \ }r}
    \toprule
        Model & \multicolumn{1}{c}{Multi} & \multicolumn{1}{c}{Single} \\
        \midrule
        \textsc{Fp-Ocr} & \multicolumn{1}{c}{--} & $1.34$ \\
        \textsc{Base} & $1.56 \pm 0.23$ & $1.41 \pm 0.16$ \\
        \textsc{Copy} & $2.04 \pm 0.62$ & $1.99 \pm 0.41$ \\
        \textsc{Ours} & $0.92 \pm 0.05$ & $\boldsymbol{0.80} \pm 0.07$ \\
    \bottomrule
    \end{tabular}\\
    \vspace{1em}
    (b) Griko\\
    \begin{tabular}{l|r@{\ \ \ }r}
    \toprule
        Model & \multicolumn{1}{c}{Multi} & \multicolumn{1}{c}{Single} \\
        \midrule
        \textsc{Fp-Ocr} & \multicolumn{1}{c}{--} & $3.27$ \\
        \textsc{Base} & $6.78 \pm 0.62$ & $5.95 \pm 0.52$ \\
        \textsc{Copy} & $2.54 \pm 0.31$ & $2.28 \pm 0.28$ \\
        \textsc{Ours} & $\boldsymbol{1.66} \pm 0.03$ & $1.70 \pm 0.21$ \\
    \bottomrule
    \end{tabular}\\
    \vspace{1em}
    (c) Yakkha\\
    \begin{tabular}{l|r@{\ \ \ }r}
    \toprule
        Model & \multicolumn{1}{c}{Multi} & \multicolumn{1}{c}{Single} \\
        \midrule
        \textsc{Fp-Ocr} & \multicolumn{1}{c}{--} & $8.90$ \\
        \textsc{Base} & $70.39 \pm 0.49$ & $71.71 \pm 0.71$  \\
        \textsc{Copy} & $14.77 \pm 0.97$ & $12.30 \pm 2.39$ \\
        \textsc{Ours} & $\boldsymbol{7.75} \pm 0.46$ & $8.44 \pm 0.90$ \\
    \bottomrule
    \end{tabular}
    \caption{Mean and standard deviation of the character error rate with 10-fold cross-validation over five random seeds. The results presented are the same as \autoref{tab:cer} with the added information of standard deviation. The best models for each language are \textbf{highlighted}.}
    \label{tab:stddev_cer}
\end{table}

\begin{table}[tb]
    \centering
    \small
    (a) Ainu\\
    \begin{tabular}{l|r@{\ \ \ }r}
    \toprule
        Model & \multicolumn{1}{c}{Multi} & \multicolumn{1}{c}{Single} \\
        \midrule
        \textsc{Fp-Ocr} & \multicolumn{1}{c}{--} & $6.27$ \\
        \textsc{Base} & $8.56 \pm 1.01$ & $7.88 \pm 0.64$   \\
        \textsc{Copy} & $9.48 \pm 3.07$ & $8.57 \pm 1.45$ \\
        \textsc{Ours} & $5.75 \pm 0.24$ & $\boldsymbol{5.19} \pm 0.31$ \\
    \bottomrule
    \end{tabular}\\
    \vspace{1em}
    (b) Griko\\
    \begin{tabular}{l|r@{\ \ \ }r}
    \toprule
        Model & \multicolumn{1}{c}{Multi} & \multicolumn{1}{c}{Single} \\
        \midrule
        \textsc{Fp-Ocr} & \multicolumn{1}{c}{--} & $15.63$ \\
        \textsc{Base} & $15.13 \pm 0.99$ & $13.67 \pm 1.17$ \\
        \textsc{Copy} & $9.33 \pm 0.49$ & $8.90 \pm 0.51$ \\
        \textsc{Ours} & $\boldsymbol{7.46} \pm 0.09$ & $7.51 \pm 0.31$ \\
    \bottomrule
    \end{tabular}\\
    \vspace{1em}
    (c) Yakkha\\
    \begin{tabular}{l|r@{\ \ \ }r}
    \toprule
        Model & \multicolumn{1}{c}{Multi} & \multicolumn{1}{c}{Single} \\
        \midrule
        \textsc{Fp-Ocr} & \multicolumn{1}{c}{--} & $31.64$ \\
        \textsc{Base} & $98.15 \pm 1.55$ & $99.10 \pm 2.20$  \\
        \textsc{Copy} & $30.36 \pm 1.39$ & $27.81 \pm 1.65$ \\
        \textsc{Ours} & $\boldsymbol{20.95} \pm 1.04$ & $21.33 \pm 0.53$ \\
    \bottomrule
    \end{tabular}
    \caption{Mean and standard deviation of the word error rate with 10-fold cross-validation over five random seeds. The results presented are the same as \autoref{tab:cer} with the added information of standard deviation. The best models for each language are \textbf{highlighted}.}
    \label{tab:stddev_wer}
    \vspace{1em}
\end{table}

\definecolor{graphblue}{HTML}{A6BDDB}
\pgfplotstableread[row sep=\\,col sep=&]{
det & Ainu & Griko & Yakkha \\
all & 5.19 & 7.46 & 24.29 \\
diag & 5.49 & 8.06 & 22.73 \\
copy & 6.56 & 8.66 & 37.83 \\
coverage & 5.60 & 10.19 & 26.71 \\
dec & 6.86 & 7.87 & 20.95 \\
enc & 6.70 & 9.47 & 28.41 \\
seq2seq & 5.65 & 9.43 & 27.68 \\
}\data
\def\mystrut{\vphantom{hp}}

%\vspace{-1em}
\begin{tikzpicture}[trim left=-1.6cm,trim right=0cm]
    \begin{axis}[
            xbar,
            every axis plot post/.style={/pgf/number format/fixed},
            bar width=.23cm,
            width=4cm,
            height=4cm,
            ymajorgrids=false,
            yminorgrids=false,
            xmajorgrids=false,
            %every axis legend/.code={\let\addlegendentry\relax},
            %legend={Wikipedia Size (in million articles)},
            legend style={draw=none,at={(0.2,0.9)},anchor=west},
            symbolic y coords={all,diag,copy,coverage,dec,enc,seq2seq},
            ytick={all,diag,copy,coverage,dec,enc,seq2seq},
            yticklabels={all,-diag,-copy,-coverage,-pretr. dec,-pretr. enc,-pretr. s2s},
            %ytick={all,-diag loss,-copy,-coverage, -pretrain dec,-pretrain enc,-pretrain seq2seq},
            every y tick label/.append style={font=\small\mystrut},
            every x tick label/.append style={font=\small\mystrut},
            tick pos=left,
            %hide y axis,
            axis x line*=bottom,
            axis y line*=left,
            nodes near coords,
            %nodes near coords align={vertical},
            every node near coord/.append style={font=\small,color=black},
            %nodes near coords style={},
            title={\small Ainu},
            title style={yshift=-0.3cm},
            %ymin=0,ymax=6.1,
            xmin=0,xmax=8,
            %ylabel shift={-1cm},
            %ylabel near ticks,
            %ylabel={},
            %xlabel near ticks,
            %xlabel={WER},
            enlarge x limits=0.0,
            xtick style={draw=none}
        ]
        %\addplot [style={black,postaction={pattern=north east lines},fill=white,mark=none}] table[x=story,y=base]{\ainudata};
        \addplot [style={graphblue,fill=graphblue,mark=none}] table[x=Ainu,y=det]{\data};
        %\legend{Wikipedia Size (in million articles)}
    \end{axis}
\end{tikzpicture}
\begin{tikzpicture}[trim left=-3cm,trim right=0cm]
    \begin{axis}[
            xbar,
            every axis plot post/.style={/pgf/number format/fixed},
            bar width=.23cm,
            width=3.5cm,
            height=4cm,
            ymajorgrids=false,
            yminorgrids=false,
            xmajorgrids=false,
            %every axis legend/.code={\let\addlegendentry\relax},
            %legend={Wikipedia Size (in million articles)},
            legend style={draw=none,at={(0.2,0.9)},anchor=west},
            symbolic y coords={all,diag,copy,coverage,dec,enc,seq2seq},
            %ytick={all,diag,copy,coverage,dec,enc,seq2seq},
            %ytick={,,,,,,},
            xtick={0,2,4,6,8,10},
            yticklabels={,,,,,,},
            %ytick={all,-diag loss,-copy,-coverage, -pretrain dec,-pretrain enc,-pretrain seq2seq},
            every y tick label/.append style={font=\small\mystrut},
            every x tick label/.append style={font=\small\mystrut},
            %tick pos=none,
            %hide y axis,
            axis x line*=bottom,
            axis y line*=left,
            nodes near coords,
            %nodes near coords align={vertical},
            every node near coord/.append style={font=\small,color=black},
            %nodes near coords style={},
            title={\small Griko},
            title style={yshift=-0.3cm},
            %ymin=0,ymax=6.1,
            xmin=0,xmax=11,
            %ylabel shift={-1cm},
            %ylabel near ticks,
            %ylabel={},
            %xlabel near ticks,
            %xlabel={WER},
            enlarge x limits=0.0,
            xtick style={draw=none}
        ]
        %\addplot [style={black,postaction={pattern=north east lines},fill=white,mark=none}] table[x=story,y=base]{\ainudata};
        \addplot [style={graphblue,fill=graphblue,mark=none}] table[x=Griko,y=det]{\data};
        %\legend{Wikipedia Size (in million articles)}
    \end{axis}
\end{tikzpicture}

\begin{tikzpicture}[trim left=-1.6cm,trim right=0cm]
    \begin{axis}[
            xbar,
            every axis plot post/.style={/pgf/number format/fixed},
            bar width=.23cm,
            width=6.5cm,
            height=4cm,
            ymajorgrids=false,
            yminorgrids=false,
            xmajorgrids=false,
            %every axis legend/.code={\let\addlegendentry\relax},
            %legend={Wikipedia Size (in million articles)},
            legend style={draw=none,at={(0.2,0.9)},anchor=west},
            symbolic y coords={all,diag,copy,coverage,dec,enc,seq2seq},
            ytick={all,diag,copy,coverage,dec,enc,seq2seq},
            yticklabels={all,-diag,-copy,-coverage,-pretr. dec,-pretr. enc,-pretr. s2s},
            %ytick={all,-diag loss,-copy,-coverage, -pretrain dec,-pretrain enc,-pretrain seq2seq},
            every y tick label/.append style={font=\small\mystrut},
            every x tick label/.append style={font=\small\mystrut},
            tick pos=left,
            %hide y axis,
            axis x line*=bottom,
            axis y line*=left,
            nodes near coords,
            %nodes near coords align={vertical},
            every node near coord/.append style={font=\small,color=black},
            %nodes near coords style={},
            title={\small Yakkha},
            title style={yshift=-0.3cm},
            %ymin=0,ymax=6.1,
            xmin=0,xmax=42,
            xlabel shift={-0.2cm},
            %ylabel near ticks,
            %ylabel={},
            xlabel near ticks,
            xlabel={\small Word Error Rate},
            enlarge x limits=0.0,
            enlarge y limits=0.1,
            xtick style={draw=none},
            % xtick align=outside
            % ytick style={draw=none}
        ]
        %\addplot [style={black,postaction={pattern=north east lines},fill=white,mark=none}] table[x=story,y=base]{\ainudata};
        \addplot [style={graphblue,fill=graphblue,mark=none}] table[x=Yakkha,y=det]{\data};
        %\legend{Wikipedia Size (in million articles)}
    \end{axis}
\end{tikzpicture}

{%
\setlength{\fboxsep}{0pt}%
\setlength{\fboxrule}{1pt}%
}%
\begin{figure}[tb]
\tikzset{seq/.style={draw=none,fill=gray!20}}
\tikzset{layer/.style={->,thick}}
\tikzset{label/.style={anchor=west,font={\footnotesize}}}
\tikzset{seqlabel/.style={font={\small}}}
\newcommand{\encoder}[3]{
\draw[seq] (-1.25,-0.25) rectangle (1.25,0.25);
\node[seqlabel] at (0,0) 
%{$\xmat$};
{$#3\step{1}#1 \ldots #3\step{#2}#1$};
\draw[layer] (0,0.3) -- (0,0.7);
\node[seqlabel] at (0,0.5) [label] {encoder};
\draw[seq] (-1.25,0.75) rectangle (1.25,1.25);
\node[seqlabel] at (0,1) %{$\hmat$}; 
{$\hvec\step{1}#1 \ldots \hvec\step{#2}#1$};
}
\newcommand{\ocrencoder}[5]{
\node[inner sep=0pt] at (0,-1.7)
    {\small #5};
\node[inner sep=0pt] at (0,-1.2)
    {\setlength{\fboxsep}{.005\textwidth}%
    \fbox{\includegraphics[width=.145\textwidth]{#4}}};
\draw[layer] (0,-0.9) -- (0,-0.3);
\node[seqlabel] at (0,-0.6) [label] {\textsc{ocr}};
\draw[seq] (-1.25,-0.25) rectangle (1.25,0.25);
\node[seqlabel] at (0,0) 
%{$\xmat$};
{$#3\step{1} \ldots #3\step{#2}$};
\draw[layer] (0,0.3) -- (0,0.9);
\node[seqlabel] at (0,0.6) [label] {encoder};
\draw[seq] (-1.25,0.95) rectangle (1.25,1.45);
\node[seqlabel] at (0,1.2) %{$\hmat$}; 
{$\hvec\step{1}#1 \ldots \hvec\step{#2}#1$};
}
\newcommand{\decoder}[1]{
\draw[seq] (-1,2.05) rectangle (1,2.55);
\node[seqlabel] at (0,2.3) %{$\cmat#1$}; 
{$\cvec\step{1}#1 \ldots \cvec\step{K#1}#1$};
\draw[layer] (0,2.6) -- (0,3.2);
\node[seqlabel] at (0,2.9) [label] {decoder};
\draw[seq] (-1,3.25) rectangle (1,3.75);
\node[seqlabel] at (0,3.5) %{$\smat#1$}; 
{$\svec\step{1}#1 \ldots \svec\step{K#1}#1$};
\draw[layer] (0,3.8) -- (0,4.4);
\node[seqlabel] at (0,4.1) [label] {softmax};
\draw[seq] (-1,4.45) rectangle (1,5.05);
\node[seqlabel] at (0,4.75) %{$P(\ymat#1)$}; 
{$P(\yvec\step{1}#1 \ldots \yvec\step{K#1}#1)$};
}
\begin{center}
\resizebox{0.7\hsize}{!}{
\begin{tabular}{c}
\begin{tikzpicture}
\begin{scope}[xshift=-1.4cm]
% \encoder{^x}{N}{\xvec}
\ocrencoder{^x}{N}{\xvec}{images/ainu_frame.png}{Ainu}
\end{scope}
\begin{scope}[xshift=1.4cm]
% \encoder{^t}{M}{\tvec}
\ocrencoder{^t}{M}{\tvec}{images/japanese_frame.png}{Japanese}
\end{scope}
\draw[layer] (-1.4,1.5) -- (0,2.0);
\draw[layer] (1.4,1.5) -- (0,2.0);
\node at (-1,1.85) [label,anchor=east] {attention};
\node at (1,1.85) [label] {attention};
\decoder{}
\end{tikzpicture}
\end{tabular}%
}
\end{center}
\caption{The proposed multi-source architecture with the encoder for an endangered language segment (left) and an encoder for the translated segment (right). The input to the encoders is the first pass OCR over the scanned images of each segment. For example, the OCR on the scanned images of some Ainu text (left) and its Japanese translation (right).}
\label{fig:multisourcemodels}
\end{figure}
\renewcommand{\arraystretch}{1.0}
\begin{figure*}[tb]
    \centering
    \small
    \begin{tabular}{lcc}
    & \multicolumn{2}{c}{Errors \textit{fixed} by post-correction}\\[.1cm]
        & (a) Griko & (b) Yakkha  \\
        \raisebox{0.7em}{[Image]} & \frame{\includegraphics[width=0.4\columnwidth]{images/errors1a.pdf}} & \frame{\includegraphics[width=0.3\columnwidth]{images/errors2a.pdf}} \\
        & \multicolumn{2}{c}{$\big\downarrow$ \hspace{3cm} $\big\downarrow$}\\
        \raisebox{0.35em}{[First pass OCR]} & \raisebox{0.3em}{\large e\textcolor{burntred}{\textbf{x}}i i ka\textcolor{burntred}{\textbf{dd}}in\`ara} &
        \includegraphics[width=0.25\columnwidth]{images/error2b.pdf} \\
        & \multicolumn{2}{c}{$\big\downarrow$ \hspace{3cm} $\big\downarrow$}\\
        \raisebox{0.35em}{[Post-corrected]} & \raisebox{0.35em}{\large e\textcolor{burntblue}{$\bm{\chi}$}i i ka\textbf{\textcolor{burntblue}{\d{d}\d{d}}}in\`ara} &
        \includegraphics[width=0.28\columnwidth]{images/error2c.pdf} \\
    \end{tabular}
    \qquad
    \begin{tabular}{cc}
        \multicolumn{2}{c}{Errors \textit{introduced} by post-correction}\\[.1cm]
        (c) Griko & (d) Yakkha  \\
        \frame{\includegraphics[width=0.25\columnwidth]{images/errors3a.pdf}} & \frame{\includegraphics[width=0.35\columnwidth]{images/errors4a.pdf}} \\
        \multicolumn{2}{c}{$\big\downarrow$ \hspace{2.5cm} $\big\downarrow$}\\
        \raisebox{0.35 em}{\large{\`{e} ffacilo}} &
        \includegraphics[width=0.3\columnwidth]{images/errors4b.pdf} \\
        \multicolumn{2}{c}{$\big\downarrow$ \hspace{2.5cm} $\big\downarrow$}\\
        \raisebox{0.35 em}{\large{\`{e} ffa\textcolor{burntred}{\textbf{\'{c}}}ilo}} &
        \includegraphics[width=0.32\columnwidth]{images/errors4c.pdf}
    \end{tabular}
    \caption{Our model fixes many mixed script and uncommon diacritics errors such as (a) and (b). In rare cases, it ``over-corrects" the first pass OCR transcription, introducing errors such as (c) and (d).}
    \label{fig:error_examples}
\end{figure*}

\begin{figure}[t]
\centering
    \begin{subfigure}[t]{\columnwidth}
    \centering
      \caption{Ainu (left) -- Japanese (right)}
      \vspace{-0.5em}
      \fbox{\includegraphics[width=0.9\columnwidth]{images/ainu_ex.png}}
      \vspace{0.6em}
    \end{subfigure}
    \begin{subfigure}[t]{\columnwidth}
      \centering
      \caption{Griko (top) -- Italian (bottom)}
      \vspace{-0.5em}
      \frame{\includegraphics[width=0.93\columnwidth]{images/griko_example.pdf}}
      \vspace{0.7em}
    \end{subfigure}
    \begin{subfigure}[t]{\columnwidth}
      \centering
      \caption{Yakkha (top) -- Nepali (middle) -- English (bottom)}
      \vspace{-0.5em}
      \fbox{\includegraphics[width=0.9\columnwidth]{images/figure2c.pdf}}
      \vspace{0.6em}
    \end{subfigure}
    \footnotesize{(d) Handwritten Shangaji -- typed English glosses}
    \begin{tabular}{|@{\ \ }c@{\ \ }|}
    \hline
     \begin{subfigure}[t]{0.9\columnwidth}
      \centering
      \includegraphics[width=0.9\columnwidth]{images/sha_img.pdf}
      \includegraphics[width=0.5\columnwidth]{images/sha_text.pdf}
    \end{subfigure}\\
    \hline
    \end{tabular}
    \caption{Examples of scanned documents in endangered languages accompanied by translations from the same scanned book (a, b, c) or linguistic archive (d).}
    \label{fig:dataset_example}
    \vspace{-1.2em}
\end{figure}

