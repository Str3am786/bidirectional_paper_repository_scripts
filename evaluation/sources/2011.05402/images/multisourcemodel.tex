{%
\setlength{\fboxsep}{0pt}%
\setlength{\fboxrule}{1pt}%
}%
\begin{figure}[tb]
\tikzset{seq/.style={draw=none,fill=gray!20}}
\tikzset{layer/.style={->,thick}}
\tikzset{label/.style={anchor=west,font={\footnotesize}}}
\tikzset{seqlabel/.style={font={\small}}}
\newcommand{\encoder}[3]{
\draw[seq] (-1.25,-0.25) rectangle (1.25,0.25);
\node[seqlabel] at (0,0) 
%{$\xmat$};
{$#3\step{1}#1 \ldots #3\step{#2}#1$};
\draw[layer] (0,0.3) -- (0,0.7);
\node[seqlabel] at (0,0.5) [label] {encoder};
\draw[seq] (-1.25,0.75) rectangle (1.25,1.25);
\node[seqlabel] at (0,1) %{$\hmat$}; 
{$\hvec\step{1}#1 \ldots \hvec\step{#2}#1$};
}
\newcommand{\ocrencoder}[5]{
\node[inner sep=0pt] at (0,-1.7)
    {\small #5};
\node[inner sep=0pt] at (0,-1.2)
    {\setlength{\fboxsep}{.005\textwidth}%
    \fbox{\includegraphics[width=.145\textwidth]{#4}}};
\draw[layer] (0,-0.9) -- (0,-0.3);
\node[seqlabel] at (0,-0.6) [label] {\textsc{ocr}};
\draw[seq] (-1.25,-0.25) rectangle (1.25,0.25);
\node[seqlabel] at (0,0) 
%{$\xmat$};
{$#3\step{1} \ldots #3\step{#2}$};
\draw[layer] (0,0.3) -- (0,0.9);
\node[seqlabel] at (0,0.6) [label] {encoder};
\draw[seq] (-1.25,0.95) rectangle (1.25,1.45);
\node[seqlabel] at (0,1.2) %{$\hmat$}; 
{$\hvec\step{1}#1 \ldots \hvec\step{#2}#1$};
}
\newcommand{\decoder}[1]{
\draw[seq] (-1,2.05) rectangle (1,2.55);
\node[seqlabel] at (0,2.3) %{$\cmat#1$}; 
{$\cvec\step{1}#1 \ldots \cvec\step{K#1}#1$};
\draw[layer] (0,2.6) -- (0,3.2);
\node[seqlabel] at (0,2.9) [label] {decoder};
\draw[seq] (-1,3.25) rectangle (1,3.75);
\node[seqlabel] at (0,3.5) %{$\smat#1$}; 
{$\svec\step{1}#1 \ldots \svec\step{K#1}#1$};
\draw[layer] (0,3.8) -- (0,4.4);
\node[seqlabel] at (0,4.1) [label] {softmax};
\draw[seq] (-1,4.45) rectangle (1,5.05);
\node[seqlabel] at (0,4.75) %{$P(\ymat#1)$}; 
{$P(\yvec\step{1}#1 \ldots \yvec\step{K#1}#1)$};
}
\begin{center}
\resizebox{0.7\hsize}{!}{
\begin{tabular}{c}
\begin{tikzpicture}
\begin{scope}[xshift=-1.4cm]
% \encoder{^x}{N}{\xvec}
\ocrencoder{^x}{N}{\xvec}{images/ainu_frame.png}{Ainu}
\end{scope}
\begin{scope}[xshift=1.4cm]
% \encoder{^t}{M}{\tvec}
\ocrencoder{^t}{M}{\tvec}{images/japanese_frame.png}{Japanese}
\end{scope}
\draw[layer] (-1.4,1.5) -- (0,2.0);
\draw[layer] (1.4,1.5) -- (0,2.0);
\node at (-1,1.85) [label,anchor=east] {attention};
\node at (1,1.85) [label] {attention};
\decoder{}
\end{tikzpicture}
\end{tabular}%
}
\end{center}
\caption{The proposed multi-source architecture with the encoder for an endangered language segment (left) and an encoder for the translated segment (right). The input to the encoders is the first pass OCR over the scanned images of each segment. For example, the OCR on the scanned images of some Ainu text (left) and its Japanese translation (right).}
\label{fig:multisourcemodels}
\end{figure}