\begin{abstract}
\begin{center}
	\today \\
	Initial release: June 2019
\end{center}
\vspace{0.2cm}

We develop a model of stable assets, including non-custodial stablecoins backed by cryptocurrencies. Such stablecoins are popular methods for bootstrapping price stability within public blockchain settings. We derive fundamental results about dynamics and liquidity in stablecoin markets, demonstrate that these markets face deleveraging feedback effects that cause illiquidity during crises and exacerbate collateral drawdown, and characterize stable dynamics of the system under particular conditions. The possibility of such `deleveraging spirals' was first predicted in the initial release of our paper in 2019 and later directly observed during the `Black Thursday' crisis in Dai in 2020. From these insights, we suggest design improvements that aim to improve long-term stability. We also introduce new attacks that exploit arbitrage-like opportunities around stablecoin liquidations. Using our model, we demonstrate that these can be profitable. These attacks may induce volatility in the `stable' asset and cause perverse incentives for miners, posing risks to blockchain consensus. A variant of such attacks also later occurred during Black Thursday, taking the form of mempool manipulation to clear Dai liquidation auctions at near zero prices, costing \$8m.
\end{abstract}