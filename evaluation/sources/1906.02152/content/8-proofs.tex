\section{Derivation of Results}




\noindent\rule{\textwidth}{1pt}
%%%%%%%%%%%%%%%%%%%%%%%%%%%%%%%%%%%%%%%%%%%%%%%%%%%
\paragraph{Prop.~\ref{prop:constraint_sol}} \hypertarget{pf:constraint_sol}{}
\begin{proof}
	In each period $t$, we determine the leverage constraint by setting $\tilde \lambda = \lambda$ and solving for $\Delta$. Using the formulation of $p^D_t$ from the market clearing, we have the following equation for $\Delta$:
	$$\tilde \lambda \Big(z + \Delta \frac{x}{\Delta - y}\Big) = \beta(\mathcal{L} + \Delta).$$
	Given $\Delta>y$, this transforms to the quadratic equation for $\Delta$
	$$-\beta \Delta^2 + \Delta\Big( \tilde \lambda (z+x) - \beta(\mathcal{L} - y)\Big) - \tilde\lambda zy + \beta\mathcal{L} y =0.$$
	This is a downward facing parabola. The speculator's leverage constraint is satisfied when the polynomial is positive. The roots, if real, bound the feasible region of the speculator's constraint. Due to the requirement that $\Delta > y$, the feasible set is given by $[\Delta_{\min}, \Delta_{\max}] \cap (y, \infty)$. When there are no real roots, the polynomial is never positive, and so the constraint is unachievable.
\end{proof}



\noindent\rule{\textwidth}{1pt}
%%%%%%%%%%%%%%%%%%%%%%%%%%%%%%%%%%%%%%%%%%%%%%%%%%%
\paragraph{Prop.~\ref{prop:leverage_sol}} \hypertarget{pf:leverage_sol}{}
\begin{proof}
	By Prop.~\ref{prop:constraint_sol}, $[\Delta_{\min}, \Delta_{\max}] \cap (y, \infty)$ is indeed the feasible region.
	Incorporating the market clearing, the speculator decides $\Delta$ in each period $t$ by solving
	$$\begin{aligned}
	\max \hspace{0.5cm} & r\Big(z + \Delta \frac{x}{\Delta - y}\Big) - \mathcal{L} - \Delta \\
	\text{s.t.} \hspace{0.5cm} & \Delta \in [\Delta_{\min}, \Delta_{\max}] \cap (y, \infty)
	\end{aligned}$$
	
	This optimization is solvable in closed form by maximizing over critical points. Maximizing the objective is equivalent to maximizing
	$$f(\Delta) = r\Delta \frac{x}{\Delta - y} - \Delta.$$
	
	We first consider the case of $\Delta$ approaching $y$ from above and show that this boundary is not relevant in the maximization. The limit is
	$$\lim_{\Delta \rightarrow y^+} f(\Delta) = -\infty.$$
	To see this, note that $\mathcal{L}_{t-1} ~=~ \bar m_{t-1} ~\geq~ w^D_t \bar m_{t-1}$, and so in order to have $\mathcal{L}_t = w^D_t \bar m_{t-1}$, we must have $\Delta<0$. Thus the sign of the term that tends to infinity is negative. The limit is $-\infty$ because the price for the speculator to buy back DStablecoins goes to $\infty$.
	
	To find the critical points of $f$, we set the derivative equal to zero:
	$$\frac{df}{d\Delta} = -\frac{\Delta^2 - 2\Delta y + y(rx +y)}{(\Delta-y)^2}=0$$
	Assuming $\Delta \neq y$, the solutions are the roots to the quadratic
	$\Delta^2 + -2y\Delta + y(rx+y)=0$.
	Notice that the axis of this parabola is at $\Delta=y$. When there are two real solutions, then exactly one of them will be $>y$. Given $y\leq 0$ and $x\geq 0$ and noting $r\geq 0$, a real solution always exists and the relevant critical point is
	$$\Delta^* = y + \sqrt{-yrx}.$$
	
	If it is feasible, $\Delta^*$ is the solution to the speculator's optimization problem. If $\Delta^*$ is not feasible, then we need to choose along the boundary. The possible cases are as follows.
	
	Suppose $\Delta^* < \Delta_{\min}$. Then $\Delta_{\min}$ is feasible since $\Delta^*>y$ implies $\Delta_{\min}>y$. Since $f$ is monotone decreasing to the right of $\Delta^*$, $f(\Delta_{\min})>f(\Delta_{\max})$, and so $\Delta_{\min}$ is the solution.
	
	Suppose $\Delta^* > \Delta_{\max}$. By our assumption that the constraint is feasible, we have that $\Delta_{\max}$ is feasible. Since $f$ is monotone decreasing to the left of $\Delta^*$ on the feasible region, $f(\Delta_{\max})>f(\Delta_{\min})$, and so $\Delta_{\max}$ is the solution.
\end{proof}




\noindent\rule{\textwidth}{1pt}
%%%%%%%%%%%%%%%%%%%%%%%%%%%%%%%%%%%%%%%%%%%%%%%%%%%
\paragraph{Prop.~\ref{prop:feasible_condition}} \hypertarget{pf:feasible_condition}{}
\begin{proof}
	The speculator's leverage constraint is unachievable when the quadratic has no real solutions or when all real solutions are $<y$. The first case occurs when
	$$\Big(\tilde \lambda (z+x) - \beta(\mathcal{L}-y)\Big)^2 + 4\beta(-\tilde \lambda zy + \beta \mathcal{L} y) < 0.$$
	
	Noting that
	$y = -w^D \mathcal{L}$ and
	$\mathcal{L} - y = \mathcal{L}(2-w^D)$
	and expanding and simplifying terms yields
	$$\beta \tilde\lambda \mathcal{L} \Big( 2zw^D + 2x(2-w^D)\Big) - (\beta\mathcal{L}w^D)^2 > \Big(\tilde \lambda(x+z)\Big)^2$$
	Completing the square by subtracting $4\beta\tilde\lambda\mathcal{L} x(1-w^D)$ from each side then gives the result.
\end{proof}




\noindent\rule{\textwidth}{1pt}
%%%%%%%%%%%%%%%%%%%%%%%%%%%%%%%%%%%%%%%%%%%%%%%%%%%
\paragraph{Prop.~\ref{prop:liquidity_limit}} \hypertarget{pf:liquidity_limit}{}
\begin{proof}
	Setting $z=-\Delta p_t^D = -\Delta \frac{x}{\Delta-y}$ gives the lower bound $\Delta^- := \frac{z}{z+x}y>y$.
	
	Note that $\bar m_t = \mathcal{L}_t$, and so
	$y = \mathcal{L}(w^D - 1) = -w^E \mathcal{L} \leq 0.$
	The term $w^D_t \bar m_{t-1}$ presents a lower bound on the size of the DStablecoin market in the next step from the demand side, and so the speculator can't decrease the size of the market faster than $y$, even with additional capital beyond $z$. As shown above, $\Delta \rightarrow y^+$ coincides with $p^D_t \rightarrow \infty$. The speculator pays increasingly large amounts to buy back more DStablecoins as liquidity dries in the market. 
\end{proof}




\noindent\rule{\textwidth}{1pt}
%%%%%%%%%%%%%%%%%%%%%%%%%%%%%%%%%%%%%%%%%%%%%%%%%%%
\paragraph{Prop.~\ref{prop:stable1}} \hypertarget{pf:stable1}{}
\begin{proof}
	With inactive constraint, $\mathcal{L}_t = \sqrt{\mathcal{L}\mathcal{D}\hat r}$,
	$p^D_t = \frac{\mathcal{D}}{\sqrt{\mathcal{L}\mathcal{D}\hat r}} = \sqrt{\frac{\mathcal{D}}{\mathcal{L}\hat r}}$, and
	$\frac{p^D_t}{p^D_{t-1}} = \frac{\sqrt{\frac{\mathcal{D}}{\mathcal{L}\hat r}}}{\frac{\mathcal{D}}{\mathcal{L}}} = \sqrt{\frac{\mathcal{L}}{\mathcal{D}\hat r}}.$
\end{proof}





\noindent\rule{\textwidth}{1pt}
%%%%%%%%%%%%%%%%%%%%%%%%%%%%%%%%%%%%%%%%%%%%%%%%%%%
\paragraph{Theorem~\ref{prop:stable2}} \hypertarget{pf:stable2}{}
\begin{proof}
	It is straightforward to verify $\mathcal{L}_t = \mathcal{D}\hat{r}^{\frac{2^t-1}{2^t}}$ by induction using $\mathcal{L}_t = \sqrt{\mathcal{L}_{t-1} \mathcal{D} \hat r}$. Then
	$$\frac{p_t^D}{p_{t-1}^D} = \sqrt{\frac{\mathcal{L}_{t-1}}{\mathcal{D}\hat r}}
	= \sqrt{\frac{\mathcal{D}\hat{r}^{\frac{2^{t-1}-1}{2^{t-1}}}}{\mathcal{D}\hat r}}
	= \hat{r}^{\frac{1}{2}\Big(\frac{2^{t-1}-1}{2^{t-1}}-1\Big)} = \hat{r}^{-2^{-t}}.$$
	And so $\ln \frac{p_t^D}{p_{t-1}^D} = -2^{-t} \ln \hat r$.
	
	Next, as $\bar\mu_t = (1-\delta)\bar\mu_{t-1} + \delta \ln \frac{p_t^D}{p_{t-1}^D}$, it is straightforward to verify by induction that
	$$\bar\mu_t = (1-\delta)^t \bar\mu_0 - \delta \ln \hat r \sum_{k=1}^t 2^{-k}(1-\delta)^{t-k}.$$
	
	\paragraph{Case I:} $\delta = 1/2$. The series in $\bar\mu_t$ becomes
	$$\sum_{k=1}^t 2^{-k}(1-\delta)^{t-k} = \sum_{k=1}^t 2^{-k} 2^{-(t-k)}
	= \sum_{k=1}^t 2^{-t} = \frac{t}{2^t}.$$
	Then we have
	$\bar\mu_t = 2^{-t}\Big( \bar\mu_0 - \frac{1}{2}t \ln \hat r\Big)$.
	The first term $\rightarrow 0$ since $0\leq \delta < 1$. The second term $\rightarrow 0$ by L'Hopital's rule. Thus $\bar\mu_t \rightarrow 0$ as $t\rightarrow \infty$.
	
	The contributing term to volatility at time $t$, after substituting and simplifying terms, is
	$$\ln \frac{p_t^D}{p_{t-1}^D} - \bar\mu_t
	= \frac{t/2-1}{2^t}\ln \hat r - 2^{-t} \bar\mu_0.$$
	Then DStablecoin volatility evolves according to
	$$\begin{aligned}
	\bar\sigma_t^2 &= (1-\delta)\bar\sigma_{t-1}^2 + \delta\Big(\ln \frac{p^D_t}{p^D_{t-1}} - \bar\mu_t\Big)^2 \\
	&= \sum_{k=1}^t (1-\delta)^{t-k} \delta \Big(\ln \frac{p_k^D}{p_{k-1}^D} -\bar\mu_k\Big)^2 + (1-\delta)^t \bar\sigma_0^2 \\
	&= \sum_{k=1}^t 2^{-(t-k)} \delta \Big( \frac{k/2-1}{2^k}\ln \hat r - 2^{-k} \bar\mu_0 \Big)^2 + 2^{-t} \bar\sigma_0^2 \\
	&= \sum_{k=1}^t 2^{-(t-k)} \delta 2^{-2k} \Big( (k/2-1)\ln \hat r - \bar\mu_0\Big)^2 + 2^{-t} \bar\sigma_0^2 \\
	&= 2^{-t} \sum_{k=1}^t 2^{-k-1} \Big( (k/2-1)\ln \hat r - \bar\mu_0\Big)^2 + 2^{-t} \bar\sigma_0^2. \\
	\end{aligned}$$
	The second line follows from straightforward induction. As $t\rightarrow\infty$, the series converges from exponential decay. Then both terms $\rightarrow 0$ because of the factor of $2^{-t}$. Thus $\bar\sigma_t^2 \rightarrow 0$.
	
	
	\paragraph{Case II:} $\delta \neq 1/2$. The series in $\bar\mu_t$ is a geometric progression
	$$\begin{aligned}
	\sum_{k=1}^t 2^{-k}(1-\delta)^{t-k} &= \sum_{k=1}^t (1-\delta)^t \Big(2(1-\delta)\Big)^{-k} \\
	&= \frac{(1-\delta)^t\Big( 2(1-\delta)^{-1} - 2^{-t-1}(1-\delta)^{-t-1}\Big)}{1- 2(1-\delta)^{-1}} \\
	&= \frac{(1-\delta)^t - 2^{-t}}{2(1-\delta)-1}
	\end{aligned}$$
	Then we have
	$\bar\mu_t = (1-\delta)^t \bar\mu_0 - \delta \frac{(1-\delta)^t-2^{-t}}{2(1-\delta)-1} \ln \hat r$, which converges to 0 as $t\rightarrow\infty$.
	
	The contributing term to volatility at time $t$, after substituting and simplifying terms, is
	$$\ln \frac{p_t^D}{p_{t-1}^D} - \bar\mu_t
	= (1-\delta)^t \bar\mu_0 - \frac{(1-\delta)^t -2^{-t+1}(1-\delta)}{2(1-\delta)-1}\ln \hat r.$$
	The DStablecoin volatility evolves according to
	$$\begin{aligned}
	\bar\sigma_t^2 &= \sum_{k=1}^t (1-\delta)^{t-k} \delta \Big(\ln \frac{p_k^D}{p_{k-1}^D} -\bar\mu_k\Big)^2 + (1-\delta)^t \bar\sigma_0^2 \\
	&= \sum_{k=1}^t (1-\delta)^{t-k}\delta \Big( (1-\delta)^k \bar\mu_0 - \frac{(1-\delta)^k -2^{-k+1}(1-\delta)}{2(1-\delta)-1}\ln \hat r \Big)^2 + (1-\delta)^t \bar\sigma_0^2. \\
	\end{aligned}$$
	Note that because $(1-\delta) \geq 1/2$, we have
	$$\begin{aligned}
	|(1-\delta)^t - 2^{-t+1}(1-\delta)| &\leq (1-\delta)^t + 2^{-t+1}(1-\delta) \\
	&\leq 2(1-\delta)^t.
	\end{aligned}$$
	Thus we have
	$$\begin{aligned}
	\bar\sigma_t^2 &\leq (1-\delta)^t \sum_{k=1}^t \frac{\delta}{(1-\delta){^k}} \Big( (1-\delta)^k \bar\mu_0 + \frac{2(1-\delta)^k}{2(1-\delta)-1}\ln \hat r \Big)^2 + (1-\delta)^t \bar\sigma_0^2 \\
	&= (1-\delta)^t \sum_{k=1}^t (1-\delta)^{k}\delta \Big( \bar\mu_0 + \frac{2}{2(1-\delta)-1}\ln \hat r \Big)^2 + (1-\delta)^t \bar\sigma_0^t.
	\end{aligned}$$
	As $t\rightarrow\infty$, the series converges from exponential decay. Then both terms $\rightarrow 0$ because of the factor of $(1-\delta)^t$. Thus $\bar\sigma_t^2 \rightarrow 0$.
\end{proof}







%\paragraph{Result~\ref{___}} \hypertarget{pf:___}{}
%\begin{proof}
%\end{proof}
%See \hyperlink{pf:___}{proof}.