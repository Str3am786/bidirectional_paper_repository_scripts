\section{Discussion}\label{sec:discussion}

In general, it is impossible to build a stablecoin without significant risks. As speculators participate by making leveraged bets, there is always an undiversifiable cryptocurrency risk. However, a stablecoin can aim to be an effective store of value assuming the cryptocurrency market as a whole is not undermined. In this case, it is \emph{conceivable} to sustain a dollar peg if the stablecoin survives transitory extreme events. That is, to achieve long-term probabilistic stability, a stablecoin should maintain a high probability of survival.



\paragraph{Failure risks.} DStablecoins are complex systems with substantial failure risks. Our model demonstrates that they can work well in mild settings, but may have high volatility outside of these settings. As we explore in this paper, the market can collapse due to feedback effects on liquidity and volatility from deleveraging effects during crises. These effects can exacerbate collateral drawdown. Surviving these events may rely on bringing in increasing amounts of new capital to expand the DStablecoin supply during such crises. In these events speculators may not always be willing and able to take these new risky positions. Indeed, there are may examples of speculative markets drying up during extreme market movements. As we explore below, continued stability during these events additionally relies on new capital entering the system \emph{in a well-behaved manner} as profitable attacks are possible.

As suggested by our simulations, stablecoin holders face the direct tail risk of cryptocurrencies. If the market loses liquidity, there is no guarantee that forced liquidation of speculators' collateral will be possible within reasonable pricing limits. Further, volatile cryptocurrency markets can, in unlikely events, move too fast for speculators to adapt their positions. In these cases, stablecoin holders can only truly rely on the cryptocurrency value from global settlement.





\paragraph{Remark on oracle risks.}
The DStablecoin design also relies on trusted oracles to provide real world price data, which could be subject to manipulation. In MakerDAO's Dai, for instance, oracles are chosen by MKR token holders, who vote on system parameters. This opens a potential 51\% attack, in which enough speculators buy up MKR tokens, change the system to use oracles that they manipulate, and trigger global settlement at unfavorable rates to stablecoin holders while pocketing the difference themselves when they recover their excess collateral. A hint of manipulation in oracles or large acquisitions of MKR could potentially trigger market instability issues on its own.

Note that Dai has protections from oracle attacks.\footnote{Though it is notable that most MKR is reputedly held by just a few individuals within the MakerDAO team.} First, there is a threshold of maximum price change and an hourly delay on new prices taking effect. This means that emergency oracles have time to react to an attack. Second, at current prices 51\% of MKR is substantially more expensive than the ETH collateral supply. However, this second point does not have to be true in general--at least unless Dai holders otherwise bid up the price of MKR for their own security. The value of MKR is linked to expectations around Dai growth as fees paid in the system are used to reduce MKR supply. At some point, the expectation may not be enough to lift MKR value above collateral on its own. This raises the question of whether fees should be used to reduce MKR supply at all. Alternatively, MKR value could be completely based on the potential value of a 51\% attack, which may also grow with Dai growth, and the value of fees could be put to different uses, as we discuss further below.




\paragraph{A good fee mechanism may quell deleveraging spirals.}
Dai imposes fees on speculators when they liquidate positions (e.g., liquidation penalty, stability fee, penalty ratio). These can \emph{amplify} deleveraging effects by increasing deleveraging costs and disincentivizing new capital from entering the system during crises. An alternative design with automatic counter-cyclic fees could enhance stability by reducing feedback effects. For instance, fees could be collected while the system is performing well, but these fees could be removed (or made negative) automatically during liquidity crises in order to limit feedback effects and remove disincentives to bringing new capital into the system.

Speculators in Dai can pay back liabilities at any time and come and go from the system, which raises concerns about herd behavior in crises. A herd trying to deleverage can trigger a deleveraging spiral. Dynamic fees tuned to inflow/outflow could additionally disincentivize herd behavior to deleverage at the same time.




\paragraph{An alternative `collateral of last resort' idea in Dai.}
In Dai, MKR serves a certain `last resort' role in addition to governance. If there is a collateral shortfall, then new MKR is minted and sold to cover Dai liabilities making up the shortfall. This may not always be possible as the MKR market can similarly face illiquidity and the market cap may not be high enough to cover shortfalls. In some settings, MKR holders might actually have an incentive to trigger a global settlement early before MKR would be inflated. A Dai shutdown would have some effect on the price of MKR, but the cost may be small if MKR holders expect a successful relaunch of Dai after the crisis. An early shutdown is not ideal for Dai holders, as they will want to hold the stable asset for longer during extreme events. In addition to incentive alignment being unclear in MKR's `last resort' role, the invocation of the role only helps cover the aftermath of a crisis (an existing shortfall) as opposed to quelling the effects that cause the crises.

We propose an alternative `last resort' role of governance tokens that instead aims to quell deleveraging spirals. This could be achieved by automatically positioning the MKR supply as system collateral against which Dai can be minted to expand supply in crises. To illustrate, if there is a massive deleveraging by speculators, leading to excess demand for Dai and an inflated Dai price, then new Dai could be automatically minted against the MKR supply as collateral to help balance the market. In this way, a deleveraging spiral is damped: should a new wave of speculator deleveraging be triggered, it will not compound the price effect from the past wave. System fee revenue could also be put to this use.



\paragraph{Uses of limited fee revenue.}
Dai produces limited fee revenue, most of which rewards MKR investors. There is additionally a Dai savings rate that rewards Dai holders using fee revenue and serves as another tool to balance the Dai market (e.g., to boost demand for Dai when the price is below target). There is an inherent trade-off in using fee revenue, however. A Dai savings rate uses this revenue to improve stability in relatively normal settings in which a higher fee itself serves to balance the market. Alternatively, fee revenue can be channeled to an emergency fund that lessens the severity of crises--for instance as suggested above. These fees and their potential uses can be incorporated into our model to compare the effects of different design choices.




\paragraph{Stablecoin risk tools.}
Our results suggest tools and indicators that can warn about volatility in DStablecoins. We can find proxies for the free supply, estimate the price impact of liquidations, and track the entrance of new capital into speculative positions. We can connect this information with model results to estimate the probability of liquidity problems given the current state. This information is also useful in valuing token positions in these systems (e.g., Dai, MKR, and the speculator's leveraged position). 

Some exchanges have bundled select stablecoins into a single market that ensures 1-to-1 trading (e.g., \cite{huobi2018}). In this case, exchanges are essentially providing insurance to their users against stablecoin failures. These arrangements could lead to a run on exchanges in the event that some stablecoins fail. It is unclear if these exchanges are subject to regulation to protect users against this, and it is further unclear if such regulations would be sufficient to account for risks in stablecoins. Our model provides insight into the risks (to exchanges and users) if such arrangements in the future include non-custodial stablecoins.



\paragraph{Future directions.}
We suggest expansions to our model to explore wider settings.
\begin{itemize}
	\item Incorporate more speculator decisions, such as locking and unlocking collateral and holding different assets, accommodating speculators with security lending motivation. This makes the speculator's optimization problem multi-dimensional. In this expanded setting, speculators may make more long-term strategic decisions considering whether tomorrow they would have to buy back stablecoins and at what price.
	\item Consider multiple speculators with different utility functions who participate in the DStablecoin market. In this expanded setting, we can consider the conditions under which new capital may enter the system and formally study the economic attack described above and the effects of external incentives.
	\item Incorporate additional assets, such as a custodial stablecoin that faces counterparty risk. This would allow us to study long-term movements between stablecoins in the space and learn about systemic effects that could be triggered by counterparty failures. This is further relevant in evaluating systems like Maker's multi-collateral Dai. However, this comes with a trade-off of a new counterparty risk that is very hard to measure. In particular, it's not just custodian default risk, but also risk of targeted interventions on centralized assets. Such interventions (e.g., from a government who wants to shut down Dai) could be highly correlated with cryptocurrency downturns as that is when the system is naturally weakest.
	\item Incorporate endogenous feedback of liquidations on Ether price, which becomes relevant if the DStablecoin system becomes large relative to the Ether market. This is similarly important for \emph{endogenous collateral} stablecoins like Synthetix sUSD and Terra UST, in which a system equity-like asset is used as collateral (see \cite{klagesmundt2020stablecoins}).
\end{itemize}
Additionally, our existing model can be adapted to analyze DStablecoins with different design characteristics. For instance,
\begin{itemize}
	\item DStablecoins with more general collateral settlement, in which stablecoin holders can individually redeem stablecoins for collateral. This is possible, for instance, in bitUSD and Steem Dollars, and more recently in Celo Dollars. In this case, the stablecoin acts as a perpetual option to redeem collateral, and stablecoin volatility will be additionally related to the settlement terms.
	\item DStablecoins without speculator agents (e.g., Steem Dollars, in which the whole marketcap of Steem acts as collateral, or Celo Dollars, in which Celo reserves act as collateral). In these systems, stablecoin issuance is automated with the rest of the protocol. Our model can be adapted by removing speculator decisions and modeling the growth of collateral from block rewards and growth of stablecoin from other processes.
	\item Some non-collateralized algorithmic stablecoins. We believe this setting can also be interpreted in our model by thinking of \emph{implicit collateral} that ends up describing user faith in the system (see \cite{klagesmundt2020stablecoins}). The underlying mechanics would be similar, simply recreating `out of thin air' the value of the underlying asset as opposed to building on top of the value of an existing asset. The stability of the system ultimately still relies on how people perceive this value over time similarly to how perceived value of Ether changes.
\end{itemize}