\documentclass[acmsmall]{acmart}

\usepackage{booktabs}
\usepackage{vcell}
\usepackage{subfloat}
\usepackage{booktabs}
 \usepackage{arydshln}
 
\usepackage{amsmath}
\usepackage[normalem]{ulem}

\newcommand{\rnr}[1]{{\color{black}{#1}}}
\newcommand{\newcscw}[1]{{\color{black}{#1}}}
\newcommand{\mrev}[1]{{\color{black}{#1}}}


\newcommand{\msb}[1]{{\color{blue}{MSB: #1}}}
\newcommand{\joseph}[1]{{\color{purple}{JS: #1}}}
\newcommand{\sahil}[1]{{\color{brown}{SY: #1}}}
\newcommand{\cp}[1]{{\color{green}{CP: #1}}}
\newcommand{\mlg}[1]{{\color{cyan}{MLG: #1}}}
\newcommand{\ch}[1]{{\color{red}{CH: #1}}}


\AtBeginDocument{%
  \providecommand\BibTeX{{%
    \normalfont B\kern-0.5em{\scshape i\kern-0.25em b}\kern-0.8em\TeX}}}


% FOR ARXIV
\setcopyright{acmcopyright}
\copyrightyear{2022}
\acmYear{2022}
\acmDOI{xx.xxxx/xxxxxxx.xxxxxxx}
\acmConference[CSCW '22]{This work will appear in CSCW '22: Proc. ACM Hum.-Comput. Interact}{November 03--05, 2022}{Woodstock, NY}
\acmBooktitle{This work will appear in CSCW '22: Proc. ACM Hum.-Comput. Interact,
  June 03--05, 2022, Woodstock, NY}
\acmPrice{15.00}
\acmISBN{978-1-4503-XXXX-X/18/06}

% FOR REAL
% \setcopyright{acmlicensed}
% \acmJournal{PACMHCI}
% \acmYear{2022} \acmVolume{6} \acmNumber{CSCW2} \acmArticle{451} \acmMonth{11} \acmPrice{15.00}\acmDOI{10.1145/3555552}



\begin{document}

%% The "title" command has an optional parameter,
%% allowing the author to define a "short title" to be used in page headers.
\title[Measuring the Prevalence of Anti-Social Behavior]{Measuring the Prevalence of Anti-Social Behavior in Online Communities}

\author{Joon Sung Park}
\affiliation{%
 \institution{Stanford University}
 \streetaddress{353 Jane Stanford Way}
 \city{Stanford}
 \state{California}
 \country{USA}}
\email{joonspk@stanford.edu}

\author{Joseph Seering}
\affiliation{%
 \institution{Stanford University}
 \streetaddress{353 Jane Stanford Way}
 \city{Stanford}
 \state{California}
 \country{USA}}
\email{seeringj@stanford.edu}

\author{Michael S. Bernstein}
\affiliation{%
 \institution{Stanford University}
 \streetaddress{353 Jane Stanford Way}
 \city{Stanford}
 \state{California}
 \country{USA}}
\email{msb@cs.stanford.edu}

\renewcommand{\shortauthors}{Joon Sung Park et al.}

\begin{abstract}
    With increasing attention to online anti-social behaviors such as personal attacks and bigotry, it is critical to have an accurate accounting of how widespread anti-social behaviors are. In this paper, we empirically measure the prevalence of anti-social behavior in one of the world's most popular online community platforms. We operationalize this goal as measuring the proportion of unmoderated comments in the 97 most popular communities on Reddit that violate eight widely accepted platform norms. To achieve this goal, we contribute a human-AI pipeline for identifying these violations and a bootstrap sampling method to quantify measurement uncertainty. We find that 6.25\% (95\% Confidence Interval [5.36\%, 7.13\%]) of all comments in 2016, and 4.28\% (95\% CI [2.50\%, 6.26\%]) in 2020-2021, are violations of these norms. Most anti-social behaviors remain unmoderated: moderators only removed one in twenty violating comments in 2016, and one in ten violating comments in 2020. Personal attacks were the most prevalent category of norm violation; pornography and bigotry were the most likely to be moderated, while politically inflammatory comments and misogyny/vulgarity were the least likely to be moderated. This paper offers a method and set of empirical results for tracking these phenomena as both the social practices (e.g., moderation) and technical practices (e.g., design) evolve.


\end{abstract}

%%
%% The code below is generated by the tool at http://dl.acm.org/ccs.cfm.
%% Please copy and paste the code instead of the example below.
%%
\begin{CCSXML}
<ccs2012>
   <concept>
       <concept_id>10003120.10003130.10011762</concept_id>
       <concept_desc>Human-centered computing~Empirical studies in collaborative and social computing</concept_desc>
       <concept_significance>500</concept_significance>
       </concept>
 </ccs2012>
\end{CCSXML}

\ccsdesc[500]{Human-centered computing~Empirical studies in collaborative and social computing}

%% Keywords. The author(s) should pick words that accurately describe
%% the work being presented. Separate the keywords with commas.
\keywords{moderation, anti-social behavior, online communities}


\maketitle

% content is what we had prior to the 2022 Jan 15 submission
% content_RnR_redmarked is from the 2022 Jan 15 submission
% content_minor_revision__Apr2022 is for the 2022 Apr minor revision submission

\section{Introduction}
\label{sec:introduction}
Theory and algorithms for large-margin classifiers 
have been studied extensively 
since those classifiers guarantee low generalization errors 
when they have large margins over training examples 
(e.g.,~\cite{schapire+:as98,mohri+:mitpress18}). 
In particular, 
the $\ell_1$-norm regularized soft margin optimization problem, 
defined later, is a formulation of 
finding sparse large-margin classifiers based on the linear program (LP). 
This problem aims to optimize the $\ell_1$-margin 
by combining multiple hypotheses from some hypothesis class $\hset$. 
The resulting classifier tends to be sparse, 
so $\ell_1$-margin optimization is helpful for feature selection tasks.
Off-the-shelf LP solvers can solve the problem, 
but they are still not efficient enough for a huge class $\hset$. 

Boosting is a framework 
for solving the $\ell_1$-norm regularized margin optimization 
even though $\hset$ is infinitely large. 
% Roughly speaking, boosting collects a hypothesis one by one
% to maximize the margin. 
Various boosting algorithms have been invented. 
LPBoost~\citep{demiriz+:ml02} is a practical algorithm 
that often works effectively. 
% and often works efficiently in practice.
Although LPBoost terminates rapidly, 
It is shown that 
it takes $\Omega(m)$ iterations in the worst case, 
where $m$ is the number of training examples~\citep{warmuth+:nips07}. 
\cite{shalev-shwartz+:jml10} invented
% Shalev-Shwartz and Singer~\citep{shalev-shwartz+:jml10} invented
%the Relaxed margin algorithm, 
an algorithm
called Corrective ERLPBoost 
(we call this algorithm C-ERLPBoost for shorthand) 
in the paper on ERLPBoost~\citep{warmuth+:alt08}. 
C-ERLPBooost and ERLPBoost 
find $\epsilon$-approximate solutions 
in $O(\ln(m/\nu) / \epsilon^2)$ iterations, 
where $\nu \in [1, m]$ is the soft margin parameter. 
The difference is the time complexity per iteration; 
ERLPBoost solves a convex program (CP) for each iteration, 
while C-ERLPBooost solves a sorting-like problem. 
Although ERLPBoost takes much time per iteration, 
it takes fewer iterations than C-ERLPBoost 
in practical applications. 
For this reason, 
ERLPBoost is faster than C-ERLPBoost. 
Our primary motivation is to investigate boosting algorithms 
with provable iteration bounds, which perform as fast as LPBoost.

This paper has two contributions. 
Our first contribution is to give 
a unified view of boosting for soft margin optimization. 
We show that LPBoost, ERLPBoost, and C-ERLPBoost are 
instances of the Frank-Wolfe algorithm. 
%The second one proposes 


Our second contribution is to propose
a generic scheme for boosting based on the unified view.
Our scheme combines a standard Frank-Wolfe algorithm and \emph{any} algorithm 
and switches one to the other at each iteration in a non-trivial way.
%a Modified LPBoost (MLPBoost) \textcolor{red}{(Rename?)}. 
We show that this scheme guarantees 
the same convergence rate, $O(\ln(m/\nu) / \epsilon^2)$,  
as ERLPBoost and C-ERLPBoost.
One can incorporate any update rule to this scheme
without losing the convergence guarantee 
so that it takes advantage of better updates 
of the second algorithm in practice.
%with fast computation per iteration. 
In particular, 
we propose to choose LPBoost 
as the secondary algorithm, 
% as the second algorithm to combine, 
and we call the resulting algorithm 
Modified LPBoost (MLPBoost). 

In experiments on real datasets, 
MLPBoost works comparably with LPBoost, and 
%if we incorporate the LPBoost update to MLPBoost. 
MLPBoost is the fastest 
among theoretically guaranteed algorithms, as expected. 


Table~\ref{table:boosting_comparison} compares 
LPBoost, ERLPBoost, C-ERLPBoost, and MLPBoost. 
\begin{table}[t]
    \centering
    \caption{%
        Comparison of the boosting algorithms. %
        C-ERLPBoost solves the problem per iteration %
        by sorting based algorithm, while our work and %
        LPBoost solves linear programming (LP). %
        ERLPBoost solves convex programming (CP) per iteration. %
        In practice, the algorithms work fast in the order %
        LPBoost, ERLPBoost, and C-ERLPBoost. %
        As we show in section~\ref{sec:experiments}, %
        our algorithm is as fast as LPBoost. %
    }
    \label{table:boosting_comparison}
    \begin{tabular}{|c|cccc|}
        \toprule
                    & LPBoost & C-ERLPBoost & ERLPBoost & One of our work \\
        \midrule
        Iter. bound 
            & $\Omega(m)$ 
            & $O\left(\frac{1}{\epsilon^2} \ln \frac{m}{\nu}\right)$ 
            & $O\left(\frac{1}{\epsilon^2} \ln \frac{m}{\nu}\right)$ 
            & $O\left(\frac{1}{\epsilon^2} \ln \frac{m}{\nu}\right)$ \\
        Problem per iter. & LP & Sorting & CP & LP \\
        \bottomrule
    \end{tabular}
\end{table}


\section{Related Work}

From the early inception of the \emph{k-means} algorithm for clustering \citep{Lloyd1982}, there has been much methodological improvement on this unsupervised task.
This includes methods that perform clustering in the latent space of (variational) autoencoders \citep{Aljalbout2018} or use a mixture of autoencoders for the clustering \citep{Zhang2017a, Locatello2018}.
The method most related to our work is the VQ-VAE \citep{Oord2017}, which can be seen as a special case of our framework (see above).
Its authors have put a stronger focus on the discrete representation as a form of compression instead of clustering.
Hence, our model and theirs differ in certain implementation considerations (see Sec.\ \ref{sec:non-differentiability}).
All these methods have in common that they only yield a single number as a cluster assignment and provide no interpretable structure of relationships between clusters.

The self-organizing map (SOM) \citep{Kohonen1998}, however, is an algorithm that provides such an interpretable structure.
It maps the data manifold to a lower-dimensional discrete space, which can be easily visualized in the 2D case.
It has been extended to model dynamical systems \citep{Barreto2004} and combined with probabilistic models for time series \citep{Sang2008}, although without using learned representations.
There are approaches to turn the SOM into a ``deeper'' model \citep{Dittenbach2000}, combine it with multi-layer perceptrons \citep{Furukawa2005} or with metric learning \citep{Ponski2014}.
However, it has (to the best of our knowledge) not been proposed to use SOMs in the latent space of (variational) autoencoders or any other form of unsupervised deep learning model.

Interpretable models for clustering and temporal predictions are especially crucial in fields where humans have to take responsibility for the model's predictions, such as in health care.
The prediction of a patient's future state is an important problem, particularly on the \emph{intensive care unit} (ICU) \citep{Harutyunyan2017, Badawi2018}.
Probabilistic models, such as Gaussian processes, have been successfully applied in this domain \citep{Colopy2016, Schulam2016}.
Recently, deep generative models have been proposed \citep{Hyland2017}, sometimes even in combination with probabilistic modeling \citep{Lim2018}.
To the best of our knowledge, SOMs have only been used to learn interpretable static representations of patients \citep{Tirunagari2015}, but not dynamic ones.
\section{Identifying Macro Norm violating comments}
Our aim is to identify macro norm violating comments on Reddit, to quantify their prevalence, and to characterize their content, rate of engagement, and language usage. However, there are too many comments for manual annotation, and state-of-the-art machine learning classifiers are not robust enough on their own. We overcome these issues using a human-AI pipeline in which we use classifiers with a high recall to nominate candidate comments that might be violating, and then focusing our manual annotations with trained annotators on these nominated comments. By tuning the classifiers to have high recall over high precision, our pipeline ensures that almost all of the violating comments are sent to be reviewed by our annotators. Additionally, in concurrence with prior work \cite{chandrasekharan2019crossmod, seering2020reconsidering}, this pipeline ensures that a human has the final say in labeling any piece of content as a violation.

In this section, we first discuss the scope of our investigation, including how we define violating comments for this paper. We then summarize our pipeline for identifying violating comments. 

\subsection{Scope of the Study}

% \usepackage{booktabs}
% \usepackage{vcell}


\begin{table}[tb]
\centering
\caption{The eight macro norm violations on 97 popular subreddits and their definitions. We took the norms uncovered in a prior study \cite{Chandrasekharan2018internet} and expanded on some of their definitions to better fit our data.}
\begin{tabular}{ll}
\multicolumn{1}{c}{\textbf{MACRO NORM VIOLATIONS}}                                                & \multicolumn{1}{c}{\textbf{EXAMPLE COMMENTS}}                                                                                                                                         \\ 
\toprule
\vcell{Using misogynistic or vulgar slurs}                                                        & \vcell{\textit{"god... I want sage to knock this c*** out"}}                                                                                                                          \\[-\rowheight]
\printcelltop                                                                                     & \printcelltop                                                                                                                                                                         \\
\vcell{Inflammatory political claims}                                                             & \vcell{\begin{tabular}[b]{@{}l@{}}\textit{"a day old troll complaining about liberals -- I }\\\textit{smell a lost trumpkin"}\end{tabular}}                                           \\[-\rowheight]
\printcelltop                                                                                     & \printcelltop                                                                                                                                                                         \\
\vcell{Bigotry}                                                                                   & \vcell{\begin{tabular}[b]{@{}l@{}}\textit{"punishment for not being hateful enough and }\\\textit{not destroying the gays"}\end{tabular}}                                             \\[-\rowheight]
\printcelltop                                                                                     & \printcelltop                                                                                                                                                                         \\
\vcell{\begin{tabular}[b]{@{}l@{}}Verbal attacks on Reddit or specific \\subreddits\end{tabular}} & \vcell{\begin{tabular}[b]{@{}l@{}}\textit{"also reddit sucks because a user making~an~error }\\\textit{refuses to delete their post and redo it }\\\textit{correctly"}\end{tabular}}  \\[-\rowheight]
\printcelltop                                                                                     & \printcelltop                                                                                                                                                                         \\
\vcell{Posting pornographic links}                                                                & \vcell{\textit{[URL]}}                                                                                                                                                                \\[-\rowheight]
\printcelltop                                                                                     & \printcelltop                                                                                                                                                                         \\
\vcell{Personal attacks}                                                                          & \vcell{\textit{"you know man youre kind of a f***ing d*****"}}                                                                                                                        \\[-\rowheight]
\printcelltop                                                                                     & \printcelltop                                                                                                                                                                         \\
\vcell{Abusing and criticizing moderators}                                                        & \vcell{\textit{"the mods in this sub need to wake the f*** up"}}                                                                                                                      \\[-\rowheight]
\printcelltop                                                                                     & \printcelltop                                                                                                                                                                         \\
\vcell{\begin{tabular}[b]{@{}l@{}}Claiming the other person is too \\sensitive\end{tabular}}      & \vcell{\textit{"get off the internet with your sensitive ass"}}                                                                                                                       \\[-\rowheight]
\printcelltop                                                                                     & \printcelltop                                                                                                                                                                         \\
\bottomrule
\end{tabular}
\end{table}

Reddit, the focus of our study, is a large-scale social media platform with 52 million daily active users \cite{59_Phan}. On Reddit, users join smaller subcommunities called subreddits that cover a specific topic and are managed by voluntary moderators who enforce community-specific rules (e.g., the type of allowed content, expected member behaviors), making the platform a good test-bed for studying user behaviors across diverse sets of moderation strategies and topics. In particular, we explore comments posted in response to top-level post submissions as most of the discussions on Reddit take place in the comment section. Given that the subreddits each have varying rules, we consider a comment to be violating if it breaks one of the \textit{macro norms} on Reddit --- norms that the vast majority of subreddits agree on. These norms were identified in prior work~\cite{Chandrasekharan2018internet} that investigated the 100 most popular subreddits harboring nearly a third of all comments on Reddit to extract the topic categories for the moderated (summarized in Table 1). 

We rely on macro norms as they provide us with a lens to measure the prevalence of violating comments that are largely independent of community-relevant contexts. However, in doing so, we are explicitly not accounting for comments that do not violate macro norms but still violate local rules of the respective subreddit. We also note that some subreddits have explicitly chosen to permit content that violates one or more of these macro norms (e.g., vulgar or sexualized comments). This highlights an important tension between the local and the macro norms. We discuss these issues and expand on their implications for future content moderation in the discussion section.


\subsection{Data for Training and Testing the Classifiers}
The first step of our pipeline uses a set of machine learning classifiers to flag comments that are potentially macro norm violations. We trained and tested these classifiers following best practice by constructing a balanced dataset that contains an equal number of moderated comments (denoted as $\mathcal{M}$) and unmoderated comments that are still online and were not moderated or deleted by the author (denoted as $\mathcal{M'}$). There are more unmoderated than moderated comments on Reddit---$\mathcal{M'}$, therefore, is a subset of all unmoderated comments. While such balanced datasets do not match the real-world distribution, balancing the dataset gives the resulting model equal priority to each class, which is important for ensuring that our model actually learns the meaningful features for the classification task and not just the uneven class distribution. 

\subsubsection{\rnr{$\mathcal{M}$: moderated comments}} 
$\mathcal{M}$ represents the top 100 most popular English subreddits during the 11 month period from May 2016 to March 2017. Given that the moderated comments are removed soon after they are posted, prior work used \textit{praw}, a Reddit streaming API, to stream and save all comments posted to each of these study subreddits before they were moderated \cite{Chandrasekharan2018internet}. 24 hours after each of these comments were streamed, all comments were queried again via the API using their unique \textit{comment\_ID} and verified which of them were replaced by a [``removed''] tag as that would signal their removal due to moderation. Any comments by AutoModerator accounts, which are bots for moderation, were removed from $\mathcal{M}$. This left $\mathcal{M}$ with a total of 2,831,664 removed comments, with at least 5,000 for each of the 100 study subreddits. 


\subsubsection{\rnr{$\mathcal{M'}$: unmoderated comments}} 
As $\mathcal{M}$ contained only the moderated comments from the sampling period, we collected $\mathcal{M'}$ ourselves through historical archives of Reddit comments. Of the 100 study subreddits from $\mathcal{M}$, three--- r/The\_Donald, r/Incels, r/soccerstreams---no longer exist on the platform, so we focused our investigation on the remaining 97 study subreddits. During the construction of $\mathcal{M'}$, we aimed to closely replicate the data collection process of $\mathcal{M}$. For each of the 97 subreddits, we used \textit{Pushshift} dataset that stores all content posted on the Reddit platform to gather IDs of submissions that were posted from the same timeframe as when $\mathcal{M}$ was collected with an even distribution across the 11 months. We then used \textit{praw} to get the actual comments with the submission IDs and discarded any that were posted by a bot or moderated. We continued this process until we had a balanced dataset for each of the 97 study subreddits.




\begin{figure}[tb]
  \centering
  \includegraphics[width=0.90\textwidth]{content_minor_revision__Apr2022/images/Final_pipeline.jpg}
  \caption{An illustration of the human-AI pipeline for identifying violating comments. Our pipeline includes 97 subreddit classifiers that are trained using a balanced dataset of moderated and unmoderated comments, and human annotators who are trained through gated instruction~\cite{25_Liu}. Our classifiers (tuned for high recall) nominate potentially violating comments and human annotators make the final determination.}
  \Description{Human-AI pipeline}
\end{figure}




\subsection{Building the Classifiers}
Using $\mathcal{M}$ and $\mathcal{M'}$, we built 97 neural network binary classifiers, each of which was trained on the data from one of the 97 study subreddits to classify whether a given comment would be moderated on that subreddit. These classifiers collectively determine whether a comment is likely to have violated one of the macro norms and thus would have been removed on most of the subreddits. We refer to these classifiers as subreddit classifiers. 

\subsubsection{Preprocessing the data} 
We first preprocessed our dataset by putting all characters in lowercase and removing non-alphabetical characters. We then segmented our dataset into a \textit{training} dataset (70\% of all data) and \textit{testing} and \textit{validation} datasets (15\% of all data each), each with an equal number of moderated and unmoderated comments. For every study subreddits, we then used our training dataset to train word embeddings from scratch and encoded comments as fixed-length vectors, trancating and padding as needed. 

\subsubsection{Building the classifiers} 
We built our classifiers using Google's \textit{TensorFlow} and trained and validated them with the encoded dataset. The classifiers have a four-layer neural network architecture, starting with an embedding layer that takes the encoded list of integers and finds an embedding vector for each word, which we learned as we trained our network. We then pass through an average pooling layer that returns a fixed-length output vector and then through a dense layer with Rectified Linear Unit (ReLU) activation function \cite{26_TensorFlow}. Finally, we employ another dense layer with a sigmoid activation function that transforms the final output of the network into a value between 0.0 and 1.0. For our binary classification task, we identify a comment as one that would have been removed in a given subreddit if the final output of the network is greater than or equal to 0.5. 

We then fine-tuned the following four parameters for each of our subreddit classifiers using grid search where we try out exhaustive combinations of hyperparameters given candidate values:  the size of the word index used in the encoding, the length of the input vector, the number of epochs during the training phase, and the number of nodes for the ReLU layer of the neural network. These are summarized in Table 2. We optimized for the F1 score (\textit{f}-measure) on our validation dataset, achieving an average of 72.3 (std=4.32) across the 97 classifiers. This is comparable to the classifiers presented in prior work that were trained on a similar dataset \cite{4_Chancellor, chandrasekharan2019crossmod}.


% \usepackage{booktabs}


% \usepackage{booktabs}


\begin{table}[tb]
\centering
\caption{Parameters and the values used for them to fine-tune the classifiers}
\begin{tabular}{ll}
\multicolumn{1}{c}{\textbf{DESCRIPTION} } & \multicolumn{1}{c}{\textbf{SET OF VALUES} }  \\ 
\toprule
Size of the word index                    & {[}10000, 44000]                             \\
Max length of the input                   & {[}256, 512]                                 \\
Number of epochs during the training      & {[}30, 40, 50]                               \\
\# of nodes for the ReLU layer            & {[}16, 32]                                   \\
\bottomrule
\end{tabular}
\end{table}

% \begin{table}
% \centering
% \caption{Parameters and the values used for them to fine-tune the classifiers}
% \begin{tabular}{lll}
% \multicolumn{1}{c}{\textbf{PARAMETERS}} & \multicolumn{1}{c}{\textbf{DESCRIPTION}} & \multicolumn{1}{c}{\textbf{SET OF VALUES}}  \\ 
% \toprule
% \textit{$\mathcal{WI}$}                             & Size of the word index                   & {[}10000, 44000]                            \\
% \textit{$\mathcal{ML}$}                             & Max length of the input                  & {[}256, 512]                                \\
% Epochs                                  & Number of epochs during the training     & {[}30, 40, 50]                              \\
% \textit{$\mathcal{DN}$}                             & \# of nodes for the ReLU layer           & {[}16, 32]                                  \\
% \bottomrule
% \end{tabular}
% \end{table}


\subsection{Machine Learning Flags Comments}
We marked a comment as \textit{flagged} (potentially norm violating) if the number of subreddit classifiers that flagged the comment, which we call the \textit{classifier agreement score}, was greater than or equal to 80 out of 97. This threshold was selected to achieve a high recall on the ensemble classifier even at the cost of producing false positives as our pipeline includes human annotators who validate the classifier flagged comments. In other words, we wanted our process to miss as few violating comments as possible, so we deliberately used a low threshold of 80 out of 97 and passed these comments to a human review stage. This approach provides statistical power even within a realistic budget for manually annotating comments, because it results in roughly one in five flagged comments later being coded as a violation while maintaining a near-zero false negative rate.

We confirmed that this threshold indeed captures most of the violating comments: the first author manually annotated a random sample of 1,000 comments in the validation dataset from 2016 to 2017 period. This sample contained 400 comments with $< 80$ subreddit classifier agreement, 200 with $80 \leq \text{agreement} < 85 $, 200 with $85 \leq \text{agreement} < 90$, and 200 with $\geq 90$ agreement. We find that only one percent of the comments with $< 80$ classifier agreement violated at least one of the macro norms when manually inspected, whereas this number significantly increased in the subsequent sample groups (9.5\%, 12\%, and 28\% in the order of increasing classifier agreement). This low false negative rate when using a low enough agreement threshold matches the observations in a prior work that took a similar approach to classifying violating comments on Reddit \cite{chandrasekharan2019crossmod}.

In addition, to confirm that our classifier's low false negative rate holds for the comments from 2020 period, we further annotate 400 comments with classifier agreement score of less than 80 from this period randomly sampled across the study subreddits. We find our result to replicate, with roughly the same rate of 1.25\% of the sample to violate one of the macro norms. Although our subreddit classifiers were trained on comments from a 2016 to 2017 period, this low false negative rate for the comments from 2020 suggests that when combined with our human annotators, our overall pipeline still remains robust even for the newer comments. Finally, as we describe in the following section on our bootstrap sampling methods, these false negative rates are accounted for in our calculation of the confidence interval of our estimations.


\subsection{Human Annotation Validates the Flagged Comments}
\label{sec:annotation}
The tradeoff for tuning our classifiers for very few false negatives is that they produce more false positives. So we recruited human annotators to verify that the classifier flagged comments are indeed violating by asking them to code a subset of the flagged comments to see which macro norms they violate, where the subset was a random sample of the flagged comments with an even distribution across the 97 study subreddits. The definitions of these macro norms that we presented to our annotators were inspired by prior work~\cite{Chandrasekharan2018internet}, but based on our qualitative annotation discussed above, we found it appropriate to expand the definitions for some of them to better fit our data. We updated the norm described as ``opposing political views around Donald Trump'' to ``inflammatory political claims'' that covers inflammatory comments that are against the right-leaning and the left-leaning political ideologies and updated the norm described as ``hate speech that is racist or homophobic'' to ``bigotry'' that covers hate speech directed at ethnic or religious groups as well. 


\begin{figure}[tb]
  \centering
  \includegraphics[width=0.98\textwidth]{content_minor_revision__Apr2022/images/annotation_interface.jpg}
  \caption{\textbf{A:} The interface for introducing the task description and eight macro norms to a new annotator. The definitions for each norms are shown one by one, accompanied by gold-standard examples that violate the norm. \textbf{B:} The interface for training and testing new annotators. Once the new annotators select their annotation, the correct annotation is shown accompanied by gold-standard examples. The interface for the main annotation task is the same but without presenting the correct annotation portion.}
  \Description{Annotation Interface}
\end{figure}



\subsubsection{Recruiting  crowd workers} 
The crowd workers were recruited from Amazon Mechanical Turk (MTurk), and they had to be at least 18 years old, living in the US, and have completed more than 1,000 Human Intelligence Tasks (HITs – MTurk’s task unit) with the minimum HIT approval rating of 98\%. In our pilot annotation task that we will describe in a subsequent subsection, our annotators took around 10 seconds (median=9.66 seconds; 75th percentile=16.99 seconds) to annotate a single comment. Based on this, we decided to pay our workers \$1.50 for every 30 comments they annotated to ensure that we are paying the majority of our workers at the rate of at least \$15.00 per hour. We decided on this rate informed by Rolf's \textit{The Fight For Fifteen} \cite{20_Rolf}.

\subsubsection{Human annotation workflow} 
Applying human annotation for non-trivial classification tasks could suffer from inaccurate annotations due accidental errors~\cite{22_Angeli, 23_Pershina, 24_Zhang}. Therefore, we designed a training and a testing phase that are inspired by the \textit{gated instruction} workflow~\cite{25_Liu} as follows to ensure that our annotators clearly understand and are proficient at the task:

Our annotators were directed to a custom-built web platform to which they could sign in with their MTurk ID. For those joining for the first time, they were redirected to the first portion of the training phase in which they were presented with 1) an overview description of the task and its goal, 2) a content warning notifying them that some comments in the task might include offensive language, and 3) the eight macro norms accompanied by their definitions and two gold-standard example comments that we manually chose from the test dataset (Figure 2-A). When they finished reviewing this content, they were asked to practice annotating 30 hand-selected, gold-standard examples of the macro norm violations handpicked from our test dataset. This was done on the actual interface used for the main annotation task that showed a comment to be annotated, and multi-select HTML form for submitting macro norm violations (Figure 2-B). The annotators could select any number of macro norms they thought the given comment violated. Importantly, during this training phase, the workers were presented with the correct annotation and explanation after each time they submitted their annotations for a given comment. Note also that, in order to avoid biasing their decisions by imposing an ``expert'' AI opinion, the workers were not told that the comments were flagged by an algorithm as potentially norm violating \cite{seering2020reconsidering}.

Of the 30 practice comments, the last 10 were effectively the test annotations; we measured our annotators’ Cronbach's alpha reliability score when compared to our gold-standard annotations, and only admitted those whose reliability score was greater than or equal to 0.7 for the last 10 training annotations. All annotators who participated in the training and testing phase were provided with a completion code they could submit to MTurk and were paid \$1.50 for their time. Those who were admitted were allowed to start the main annotation task. They could annotate as many comments as they wanted as long as there were more comments to be annotated, and were provided with a completion code they could submit to receive \$1.50 for every 30 comments they annotated. Each of the comments was seen by three unique annotators to test for majority agreement. 

During the course of this study, we recruited a total of 31 annotators who collectively annotated a total of 4,850 comments. \mrev{Finally, we followed best practices for accounting for annotator well-being during their task. In addition to presenting the aforementioned content warning, we made sure that our annotators could freely leave the study at any time if they felt uncomfortable with the annotation task. We purposefully designed our compensation scheme to ensure that we paid our participants in small increments instead of asking them for a long period of participation before receiving their compensation to ensure that our participants could receive the payment they deserve regardless of when they choose to leave the study.}


\subsubsection{Pilot} 
To confirm the robustness of this approach and  to determine the right level of compensation for the workers, we ran a pilot annotation task for which we recruited 20 annotators to partake in the training phase. Of the twenty, 8 annotators passed the testing phase and went on to annotate 194 randomly selected comments from the test dataset. The first author manually annotated the 194 comments that the annotators annotated during the aforementioned pilot tasks to establish baseline annotations to which annotators' work would be compared. In relation to the first author’s manual annotation, the majority-agreed annotation of the annotators yielded a Cronbach's alpha score of 0.86. 

\section{Bootstrap Sampling Reddit Comments for Analysis}
Having established the process for identifying macro norm violating comments on Reddit, we proceeded to apply this process to study the prevalence and characteristics of such comments. Our core strategy for doing this was to take random samples of the online comments from our study subreddits, calculating the classifier agreement scores for each of the samples’ elements, and then taking random samples from those comments that were flagged as potentially norm violating (classifier agreement $\geq 80$). But this effectively meant that we were sampling from different subsets of the population in which some of our random samples have a lower rate of violating comments than others due to the differences in the rate of violating comments between different subreddits. 

This makes drawing conclusions from our samples using the traditional inferential statistics problematic---we cannot simply calculate a binomial proportion confidence interval, because we have several convolved sources of uncertainty. The most straightforward parametric statistical procedure would be to select random comments, label them as violating or not violating, and estimate overall levels of violation from that. Unfortunately, due to the large class imbalance (most comments are not violating), this procedure is not tractable. Our introduction of the machine learning layer to nominate possible violations helps manage this problem, but threatens the random sampling procedure and can make errors itself. So, ultimately, we chose to use the classifiers to identify (noisily) a proportion of comments that are violating, complemented with human labeling at a smaller scale to verify. This means that our sampling procedure compounds multiple types of uncertainty: which comments are sampled from the dataset, which comments are flagged by the classifiers, and which comments are verified as actually violating by human annotators.

Given this, we applied a statistical \textit{bootstrapping} technique, variations of which have been used in prior work with similar compounded uncertainty, to derive an accurate measurement and confidence intervals. The core purpose of bootstrapping is to draw conclusions about a population by resampling with replacement from the sample data, which allows for direct observation of the sampling distribution of statistics of interest \cite{61_Varian, 62_Weisstein}. We used the results from our bootstrapping to estimate our key statistics and provide their confidence intervals. 

\input{content_minor_revision__Apr2022/figure/bootstrap_workflow_2}


\subsection{\rnr{Bootstrapping Resampling Process}}
We took the following steps to sample our data for bootstrapping. For simplicity, we will describe our method with reference to the May 2016--March 2017 dataset, which we will refer to as $T_{2016}$. 

Intuitively, a bootstrap uses resampling with replacement to create a large number of parallel universes, each with the same number of comments as the original dataset, but each universe will have a slightly different number of norm violating comments due to the resampling with replacement. This variation across many parallel universes is what yields uncertainty confidence intervals on our estimates. However, each universe (bootstrap sample) must also contend with the fact that we have not manually annotated all 5 million comments in the dataset as violating or not. Rather than use a single fixed measurement of norm-violating comments per subreddit, which would ignore this source of uncertainty in the estimation, we resample our annotations over and over again with replacement to build in uncertainty due to our limited number of annotations. Figure~\ref{fig:bootstrap} provides an overview of this process, which is explained in detail below.

\subsubsection{\rnr{Study data and the flagged comments}} 
We began by randomly sampling a set of unmoderated comments. Specifically, we sampled 5,000 random comments from each of the study subreddits posted during this period dataset, for a total of 485,000 comments. These comments were sampled from the Pushshift Reddit corpus, which contains a complete capture of all comments on each subreddit. For clarity in presentation, we will call this sample of comments from 2016--2017, $T_{2016}$. We then ran all 97 subreddit classifiers on each of the sampled comments to calculate the classifier agreement score. Any comments that were flagged by at least 80 of the 97 classifiers as violating were labeled as \textit{flagged}.

\subsubsection{\rnr{Calculating intermediate probabilities}}
For the bootstrap, we needed to follow a procedure where we sample a comment, label it as flagged or not based on the classifiers, and then label flagged comments as violating or not based on human annotators. However, because we cannot tractably label every flagged comment in the bootstrap via human annotation, we relied on statistical generalization via the bootstrap. For this generalization to succeed, we require two empirically observed probabilities for each bootstrap iteration: $P_{subreddit}(violating \mid flagged)$, the true positive rate, the probability that a flagged comment in a particular subreddit is verified by annotators as an actual violation; $P(violating \mid \lnot flagged)$, the false negative rate, the probability that a comment that is not flagged as potentially violating via our classifiers is in fact violating according to our annotators.

\paragraph{$P_{subreddit}(violating \mid flagged)$.} The true positive rate, $P_{subreddit}(violating \mid flagged)$, carries uncertainty since we can only manually annotate a fixed number of comments per subreddit. To model this uncertainty, we re-estimate $P_{subreddit}(violating \mid flagged)$ with each bootstrap resample. Each iteration, for each subreddit, we take a random resample with replacement of the 32 flagged comments that annotators had labeled for each subreddit with each resample (e.g., sample 32 comments with replacement from the set of 32 flagged comments for the subreddit). This value of $P_{subreddit}(violating \mid flagged)$ is then used for that subreddit for that bootstrap iteration and varies with each iteration, capturing uncertainty in our estimate, and is used to help estimate whether a flagged comment in our sample is a true positive.

\paragraph{$P(violating \mid \lnot flagged)$.} Some comments that the classifiers do not believe to be macro norm violations will, in fact, be violations---and our bootstrap procedure must account for this so that it does not undercount the number of violations. To estimate the false negative rate, $P(violating \mid \lnot flagged)$, we use the sample of 1,000 \textit{non-flagged} unmoderated comments that we manually annotated in Section 3.4. These 1,000 comments showed a consistent 1\% false negative rate of our classifiers across 10 randomly chosen subreddits, so we treat this value as being consistent across all subreddits and do not parameterize it by subreddit. However, we still must model uncertainty in this estimate. So, for each bootstrap iteration, we resample 1,000 comments with replacement from the set of 1,000 comments and calculate $P(violating \mid \lnot flagged)$ as the proportion of comments that were not flagged as potentially violating but were in fact violating according to human annotators. We use this quantity to estimate whether a non-flagged comment in our sample is a false negative. As with the true positive rate, resampling with replacement will cause this value to vary from bootstrap iteration to iteration, critical to creating confidence intervals for our analysis.

\subsubsection{\rnr{Bootstrap resampling iteration procedure}}
These two probabilities enable our bootstrap procedure. In bootstrapping, we resample the dataset many times and measure the quantity of interest in each new instance. \rnr{In this study, we ran our bootstrapping procedure for 1,000 iterations (this is a typical number of iterations when conducting bootstrap resampling \cite{81_Bootstrap}). For each iteration, we resampled a number of comments that matched the complete number of comments on each of our study subreddits during our study period (i.e., 252,642,908 comments for $T_{2016}$). We also resample with replacement from the dataset of 5,000 classified comments each iteration, creating variation to model uncertainty in our sampling. Each bootstrap iteration also estimates new values for $P_{subreddit}(violating \mid flagged)$ and $P(violating \mid \lnot flagged)$ for each iteration, which combine to provide one datapoint in the final outcome distribution. In other words, each bootstrap sample creates an alternative universe of comments resampled from each subreddit, exhibiting natural variation in violation and moderation rates due to the random sampling.}

We illustrate this process with an example of bootstrap resampling the r/videos subreddit for $T_{2016}$. There were a total of 5,296,900 unmoderated comments that were on r/videos during $T_{2016}$. To populate each of these comments, we sampled a random comment from the subset of 5,000 random comments from r/videos that the classifiers had labeled. These 5,000 random comments had themselves been randomly sampled with replacement for this iteration of the bootstrap from the original set of 5,000 comments that were collected and classified. For r/videos, in one of the bootstrap samples, 1,249 of the 5,000 randomly sampled unmoderated comments (25\%) were flagged as potentially violating by the classifiers. If the sampled comment was flagged, then we randomly assign the comment as violating with probability $P_{r/videos}(violating \mid flagged)$ ($\approx 0.2$ in this bootstrap iteration) and attach the violation type(s) that human annotators tagged that comment with, and as not violating otherwise. If the sampled comment was not flagged, then we randomly assign the comment as violating with $P(violating \mid \lnot flagged)$ ($\approx 0.01$ in this iteration), and not violating otherwise. This process repeats to generate all 5,296,900 comments for r/videos.

We follow this process for all 97 subreddits, for all 1,000 bootstrap samples. The 1,000 samples provide a distribution and uncertainty estimate for the key statistics. Finally, through this process, we resample the same number of comments as the number of all unmoderated comments that were posted on these subreddits, some of which are labeled to be violating. So when we calculate the rate of unmoderated but violating comments in one bootstrap iteration, we add up the number of unmoderated but violating comments in each of the 97 subreddits and divide that by the number of all comments --- this gave us an estimation for r/videos, which was 5.95\% [2.16, 9.22] for $T_{2016}$, and 3.53\% [0.59, 9.14] for $T_{2020}$. This allows our bootstrap procedure to naturally take into account the different subreddit sizes --- through resampling more comments from the larger subreddits, the bootstrap will give more weight to the uncertainty on larger subreddits, in the analysis which are focused on the overall rates across the platform.

\subsubsection{2020 dataset}
As mentioned in the previous section, there are two periods of interest: one, from May 2016 to March 2017, the same period as when the moderated comments dataset $\mathcal{M}$ was collected, and the other from the last three weeks of December 2020, which we consider as a replication study on a slightly shorter but more recent timeframe. As we referred to the original dataset as $T_{2016}$, we will refer to the latter as $T_{2020}$. The process for $T_{2020}$ was identical to $T_{2016}$. Because $T_{2020}$ was a smaller dataset, however, its constituent samples were fewer in number. In particular, we \rnr{randomly} sampled 2,000 comments from each of the study subreddits instead of 5,000 commments, or fewer if the subreddit did not contain 2,000 comments during $T_{2020}$. This resulted in a total of 188,000 comments sampled and run through our 97 subreddit classifiers. Given the smaller sample size, the confidence intervals are wider for $T_{2020}$ than for $T_{2016}$.

\subsection{\rnr{Ablation Analysis with Fewer Annotations}}
Our bootstrap relies on a set of thousands of manually annotated comments from annotators. This gives rise to a possible threat to validity: that we did not manually annotate enough comments per subreddit to capture the uncertainty in the estimate per subreddit. To test whether this threat should be concerning, we performed an ablation analysis where we replicated our method using half of the manual annotations we had for each of the subreddits (that is, 16 annotations per subreddit instead of 32 for $T_{2016}$, and 4 per subreddit instead of 8 for $T_{2020}$). The main goal of this ablation analysis was to test how having fewer annotations impacts the uncertainty estimates of the bootstrap. Intuitively, as we have fewer and fewer samples, the confidence intervals will widen because if a rare event is sampled, it has a larger impact on the resulting estimate. We report our findings alongside our main results, in Section 6.

\section{Measurements}
Using our pipeline for identifying violating comments and bootstrap sampling of comments, we estimated the dependent variables of interest. In this section, we present our specific measures that we derived from our method. 


\subsection{Estimating the Prevalence of Violating Comments}

\subsubsection{Overall prevalence} 
Our core dependent variable is the percentage of unmoderated comments that are \rnr{macro norm} violations. For each bootstrap sample, we calculated the percentage of violating comments. We then calculated our estimate as the median result across the 1,000 bootstrap samples, and calculated the 95\% confidence interval across those samples. 

\subsubsection{Prevalence per subreddit} 
Do some subreddits contain a higher rate of macro norm violating comments than the others, and if so, why? To answer this question, we estimated the percentage of violating comments for each of the subreddits across each of the bootstrap samples, and again calculated the median and 95\% confidence interval per subreddit. Given that the $T_{2020}$ sample, as mentioned previously, is a smaller scale replication of the $T_{2016}$ sample, we found that there is not enough data to draw statistically meaningful results for each of the subreddits for $T_{2020}$. Therefore, we focused our analysis in this particular subsection on the $T_{2016}$ sample. 

We tested the factors that were associated with a subreddit having a higher count of violations. We started by taking the sampling approach as we did above to estimate the count of violating comments for each of the study subreddits. We hypothesized that two relevant predictive variables might influence the rate of violating comments: 1) the topic of a given subreddit and 2) the ratio of the number of moderators to the number of comments. Our focus on the first variable was motivated by our observation that some communities might be more prone to hosting macro norm violations that are topically relevant (e.g., politically inflammatory comments for political subreddits or pornography for NSFW subreddits), while our focus on the second variable was informed by prior work suggesting that moderators in subreddit communities are overloaded and unable to review every comment posted~\cite{Chandrasekharan2018internet, 18_Gilbert}.

To test these hypotheses, we first collected the relevant data for these variables. To get the respective topics for our study subreddits, we matched the study subreddits to categories on r/ListOfSubreddits’ thematically organized list of subreddits.\footnote{\url{https://www.reddit.com/r/ListOfSubreddits/wiki/listofsubreddits}} The list contains multiple layers of thematic hierarchy. From this, we picked the second-highest thematic layer (e.g. r/AskReddit was categorized as “Discussion,” and r/Pokemon as “Entertainment”) for our subreddits as the top layer was too broad to be meaningful. This resulted in eleven mutually exclusive topic categories for our 97 study subreddits.  We then manually collected the number of moderators who were present in October 2016 (the middle of the 11-month period when $\mathcal{M}$ was collected) for each of the subreddits by using the Wayback Machine to access an archived version of each subreddit's homepage on October 1st, 2016, or the earliest subsequent date when the page was crawled. We then calculated the ratio of moderators to comments for each of the subreddits. Finally, we retrieved the number of all comments (including those that were removed by the users or moderators) that were posted on each of the 97 study subreddits from $T_{2016}$ -- we used this information as the offset to our Poisson model described below. To address the non-normal distribution of the number of comments and moderator-to-comment ratio in our regression, we log transformed these variables.

We then employed a Poisson regression \cite{63_poisson} to model the rate of macro norm violating comments in a given subreddit using the three aforementioned predictive variables as our independent variables. \rnr{Poisson regression was an appropriate choice because it is designed to model rate data. We followed a common practice of setting an offset---in our case, the size of the subreddit in terms of the number of comments--- to the Poisson regression model to make it appropriate for rate data \cite{94_Anderson, 95_Gardner}.} For better interpretability of our model, we treated the ``general content'' topic category as the baseline of the indicator (dummy) variable for topic. Finally, to interpret our results, we exponentiated the variable coefficients in our model to calculate the incidence rate ratios. For instance, as we will cover in the next section, \rnr{the number of moderators to comments ratio in a subreddit is a significant predictor of the rate of violating comments in a subreddit (p<0.001) where for every unit increase in the log-transformed number of comments in a subreddit, the rate of violating comments is expected to increase by a factor of 2.07 (=$e^{0.728}$).}


\subsection{Determining the Characteristics of Violating Comments}

\subsubsection{Prevalence by violation category} 
So far, we defined a comment to be violating if it violates one of the macro norms of Reddit. But here, we describe our analysis that takes a closer look at the variance in the prevalence of different macro norm violations and the rate at which they occur. In order to better understand the prevalence of each of the macro norm violations among the unmoderated comments, \rnr{we measure the percentage of sampled violations that fall into each of the violation categories}. Specifically, we take the average number of norm violations per category that across the study subreddits in each bootstrap iteration, and as before, report the median and confidence interval across the 1,000 bootstrap iterations.

Are all categories of macro norm violations removed by the moderators at the same rate? To answer this question, we \rnr{take the following steps to} estimate the number of all comments (moderated and those still online) that violate each specific one of the macro norms. To estimate the number of violations for each of the macro norms within the moderated comments, we require a complete set of all moderated comments from our study subreddits, which we gathered earlier as $\mathcal{M}$ for $T_{2016}$. To extend $\mathcal{M}$ to $T_{2020}$, we reimplemented the original high-frequency scraping process used in prior work to generate $\mathcal{M}$ for $T_{2016}$~\cite{Chandrasekharan2018internet}. We then randomly sampled from $\mathcal{M}$ (N=776 comments for the longer $T_{2016}$ period; N=188 for the shorter $T_{2020}$ period). We then followed the same human annotation process from Section~\ref{sec:annotation} to annotate each moderated comment with any macro norms that it violated. Using this, we calculated the estimated number of comments that violate each of the macro norms within the moderated comments dataset. We combined this with the estimated number of unmoderated violations for each macro norm to estimate the overall number of violations for each of the macro norms considering both moderated and unmoderated comments combined. With this number of violations, we then proceeded to calculate the rate at which each of the macro norms are moderated.

\subsubsection{Comparing the rate of engagement} 
Reddit comments accrue engagement over time in the form of upvotes, downvotes, and replies. The “score” of a comment is determined by subtracting the number of downvotes from the number of upvotes the comment gained over time, and “replies” are comments that respond to the given comment in the thread. Score is a useful dependent variable to track because it is a strong determinant of how high in the thread the comment sits.\footnote {On Reddit, users can choose one of the three options -- top, hot, and best -- for determining what comments should come at the top. Top is simply whichever comment has the highest score, hot is the log-transformed absolute value of the score with extra weights for the age of the content, and best is the number of upvotes divided by the number of downvotes. See \url{https://www.reddit.com/r/explainlikeimfive/comments/1u0q4s/eli5_difference_between_best_hot_and_top_on_reddit/}} The number of top-level replies can serve as a proxy for measuring whether the given comment generated, or ended, a discussion.\footnote{The Perspective model from Jigsaw for measuring toxicity used human annotations to train where a comment was asked to be annotated if it were “rude, disrespectful, or unreasonable… that is likely to make you leave a discussion” \cite{60_Perspective}.}
We explored whether the violating comments result in significantly higher or lower scores, and whether they receive more or fewer “top-level replies” (replies that directly respond to the given comment in the thread of replies). To investigate this, we compared the scores and the number of top-level replies that the unmoderated violating comments got to the global average number of online comments the study subreddits in our $T_{2016}$ and $T_{2020}$ samples.


\subsubsection{Comparing language usage} 
Does the way in which a macro norm is violated impact its likelihood of remaining on the site or being moderated? We compared the writing style of moderated comments vs. unmoderated violating comments on two dimensions: readability and emotionality. For measuring readability of a comment, we used the Flesch–Kincaid readability test~\cite{64-kinkaid} to retrieve the comment’s readability score, where a higher score indicates that the comment is easier to read. For measuring emotionality of a comment, we used Evaluative Lexicon 2.0~\cite{65_Rocklage, 66_Rocklage}, an imputation-based dictionary that assigns an emotionality score to text where higher score means that the comment is more likely to trigger an emotional reaction rather than a cognitive response that reflects a person’s beliefs about the topic of discussion \cite{66_Rocklage}.

\section{Results}
\label{sec:results}

The following section discusses the training of the agents and their inference on unseen starting states.
Moreover, the influence of invariant input features on the training success is assessed.
Lastly, the generalization abilities of the trained models to other resolutions and a higher Reynolds number are investigated.%
\footnote{The trained models and the required training files can be obtained from \url{https://github.com/flexi-framework/DRL_LES}.}

\subsection{Training}
\label{sec:results_training}

\begin{figure}[htb!]
  \centering
  \includegraphics[width=0.99\linewidth]{tikz_double_column/draft-figure2.pdf}
  \caption{Training results for the 24 DOF (left) and 48 DOF (right) test case showing (from top to bottom) the undiscounted collected return on the unseen testing data, the same for the training episodes with the minimum and maximum return indicated by the shaded area, the policy gradient loss according to \eqref{eq:clipping_loss} and the value estimation loss for the value estimation ANN. The results are obtained for either 16 or 32 full episodes used per policy update. Please note that the y-axis of the policy gradient loss is scaled differently for both cases. Since the training simulations update the predictions each $\Delta t_{RL}=0.1$ and are simulated up to $t_{end}=5$, the maximum undiscounted return is $\return_{max}=50$ considering the reward in each step is normalized to $\reward_t\in[-1,1]$.}
  \label{fig:results_training}
\end{figure}


The training took between a single day for the smallest and up to 5 days for the largest cases on the HAWK supercomputer at the High-Performance Computing Center Stuttgart (HLRS).
The training was performed using up to 1024 CPU cores for the simulations and a single GPU node for model execution and training.
For more details on the hardware configuration, the reader is referred to \cite{kurz2022deep}.
The training behavior of the 24 DOF and the 48 DOF configurations is shown exemplarily in \figref{fig:results_training}, since these configurations constitute the lowest and highest employed resolution, respectively.
Both cases were trained once with 16 and with 32 episodes per parameter update.
For all cases, the collected return in the simulation is negative for the randomly initialized policy.
Starting from this initial random policy, the collected return increases during training until the return converges and plateaus just under the maximum undiscounted return of $\return_{max}=50$.
The maximum return follows from the 50 rewards collected during a simulation and the normalization of the reward function in \eqref{eq:reward} that guarantees $r_t\in[-1,1]$.
The convergence of the collected reward indicates that the RL algorithm has found a local optimum with its current policy.
Generally, the gradient estimator used for the gradient ascent algorithm should be more accurate if the amount of sampled episodes per parameter update is increased.
This can speed up the training, since a better approximated gradient can lead to more efficient parameter updates and thus reduce the overall training iterations needed for convergence.
This was indeed observed consistently for all investigated configurations.
Interestingly, the larger 48 DOF case requires less training iterations for convergence compared to the 24 DOF case.
Moreover, the 48 DOF case exhibits less variance in the return sampled in the training runs.
We attribute this reduced variance to two different factors.
First, the influence of the eddy-viscosity model on the overall flow decreases with increasing resolution, since the model accounts for a diminishing amount of unresolved kinetic energy in the flow.
Secondly, a single flow state contains more elements for the larger cases.
Since the policy trains on elementwise data, the same amount of episodes thus provides more training samples for the larger configurations.
The overall reduced variance in the training process then might cause more sample-efficient and faster optimization.
However, since the amount of required iterations for convergence did not decrease consistently with increasing resolution, the faster training might also be simply caused by the stochasticity of the training process.

The last row in \figref{fig:results_training} shows the loss of the value estimation ANN, which is trained to approximate the expected future return starting from a given state of the environment.
The value ANN is initialized with random weights and thus gives a poor estimate of the expected return in the beginning of the training, which results in a high initial loss.
The loss then decreases as the value ANN learns a first sensible estimate of the expected return.
Since the policy then starts to improve, i.e. to collect more reward, also the expected return and thus the training targets of the value ANN change.
Therefore, the value estimation loss increases as the policy improves, since the value ANN has to catch up constantly with the policy.
Once the return reaches a plateau, the expected return as target quantity for the training of the value ANN becomes more stable and the value estimation loss decreases again.


\subsection{Inference}
\label{sec:inference}

\begin{figure*}[htb!]
  \centering
  \includegraphics[width=\textwidth]{tikz_double_column/draft-figure3.pdf}
  \caption{Results for the trained RL models (from left to right) in the 24 DOF, 32 DOF, 36 DOF and 48 DOF configuration averaged over $t\in[10,20]$ for an LES initialized with the unseen testing sample. Reported are (from top to bottom) the averaged energy spectra over the wavenumbers $k$, the relative error of the energy spectra with respect to the DNS solution, the distributions $\mathcal{P}(\cdot)$ of the velocity fluctuations, and the distribution of the predicted $C_s$ parameters. The results for the underlying DNS solution as well as an implicitly modeled LES (iLES), the SSM with $C_s=0.17$ and the DSM are shown for comparison. The shaded area for the DNS energy spectrum indicates the maximum and minimum amplitudes observed for each mode. The wavenumber that is resolved by the discretization with 4 points per wavelength is shown dashed and the maximum wavenumber used for optimization $k_{max}$ is indicated with a solid black line.}
  \label{fig:spectra_n5}
\end{figure*}

The performance of the trained RL models is evaluated based on an LES, which is initialized with the unseen testing state and is computed for $t_{end}=20$.
All results reported in the following are obtained and averaged over the timeframe of $t\in[10,20]$.
Since the models are trained only on simulations with $t_{end}=5$, this allows to assess the long-term behavior of the models and whether the simulation time used during training was sufficient.
The results in \figref{fig:spectra_n5} demonstrate that the RL models are indeed long-term stable.
To assess the accuracy of the RL models, they are compared to the SSM and DSM as well as the implicit model.
As could be expected, the implicitly modeled LES exhibits a buildup of energy in the upper wavenumbers due to lacking dissipation.
The SSM with $C_s=0.17$ on the other hand introduces too much dissipation and thus fails to preserve the wavenumbers near the resolution limit of the underlying numerical scheme.
The RL model clearly outperforms both models for all considered configurations by matching the target spectrum almost perfectly up to and even beyond the discretization's resolution limit of around 4 points per wavelength.
The advantage of the RL models is more pronounced for the small cases, where the turbulence model has more impact on the overall flow.
Interestingly, the errors for the different wavenumbers are distributed more evenly for the RL model.
This might stem from the objective of minimizing the squared error in the energy spectrum in order to increase the reward as given in \eqref{eq:reward}. 

The DSM, however, reproduces the DNS spectra with similar accuracy as the RL model.
This is to be expected, since the DSM is known to provide near perfect results for the HIT case, as long as the test filter is situated in the inertial range.
The RL model achieves a similar level of accuracy, i.e. a near optimal policy, without having access to the additional filtering procedure of the DSM.
It is interesting that the most notable difference between the DSM and RL is for the 24 DOF case, where the RL agent provides a significantly better energy spectrum.
A likely explanation for this loss in accuracy of the DSM is that the LES resolution in the 24 DOF case is too coarse for the test filtering to occur in the scale-similar region.
Thus, no meaningful information is provided to the DSM procedure.
The RL model, however, seems to compensate for this lack of resolution and reproduces the DNS spectrum with very good accuracy.
It is important to stress here once again, that the underlying forcing of the test case prescribes the overall energy budget of the simulation.
Therefore, errors in the high wavenumbers might influence the energy contained in the low wavenumber and vice versa.
This stresses the capabilities of the RL models even more, which have learned to interact with this forcing such that the energy spectrum fits the prescribed one based on local information of the flow field only.

\begin{figure*}
  \centering
  \includegraphics[width=\textwidth]{tikz_double_column/draft-figure4.pdf}
  \caption{Comparison between the RL models trained with the local momentum field $\widetilde{\rho v}_i$ and with the invariants of the velocity gradient tensor $\lambda_{\nabla v}^i$ as inputs for the (from left to right) 24 DOF, 32 DOF, 36 DOF and 48 DOF configuration. The top row shows the undiscounted return collected during training with the shaded area indicating the episodes with the maximum and minimum return. The lower row shows the energy spectra in comparison to the DNS solution. The shaded area in the spectra indicates the maximum and minimum observed energy contained in the respective wavenumber during the DNS. The wavenumber that is resolved by the discretization with 4 points per wavelength is shown dashed and the maximum wavenumber used for optimization $k_{max}$ is indicated with a solid black line.}
  \label{fig:invariants}
\end{figure*}

It seems important to stress again that the energy spectra are the optimization target for the training and thus might provide only limited insight into the model's overall ability to reproduce the required turbulent statistics.
To this end, the distributions of the velocity fluctuations produced by the different models are investigated as additional important measure of the models' performance.
The differences between the models observed here are consistent with the obtained energy spectra.
Since additional dissipation tends to reduce velocity fluctuations, the SSM exhibits the narrowest and the implicit LES the broadest distribution.
For all cases, the velocity fluctuations produced by the RL model appear to be balanced between both effects, since they follow the DNS distribution more closely than the SSM for small fluctuations in the center of the distribution, but do not exhibit the overpronounced tails observed for the implicit LES.
However, the RL model produces an unsymmetrical distribution of velocity fluctuations for the 36DOF case.
This behavior was not observed for any other configuration or model and its origin is still subject of on-going investigations.
It is also unclear whether this behavior emerges since the symmetry condition has to be learned by the model implicitly during the training and might be not strict enough or whether this effect emerges from the long-term interactions between the agent and the forcing method.

The distributions of $C_s$ in the bottom row of \figref{fig:spectra_n5} show qualitatively similar results for all cases.
Most predictions are close to zero with an decreasing amount of higher values.
The overall range seems to strongly depend on the resolution, since the agent exploits almost all of its available action space for the 24 DOF case performing actions near the prescribed maximum of 0.5.
In contrast, the largest $C_s$ prediction for the 48 DOF case does not exceed 0.3.
The predictions' mean thus decreases for increasing resolutions as is consistent with the understanding of an increased LES resolution on the Smagorinsky model.
This difference might stem either from the policy itself, i.e. the policies learned indeed different distributions, or from the input data of the respective resolutions, which might exhibit different distributions.
Nonetheless, these results show the flexibility and capability of the RL training approach to incorporate physical constraints into the model through the choice of the input features.
We did not adapt the expressivity of the ANN between the two input selections, which might increase the performance.

Sarghini et al. \cite{sarghini2003neural} and Maulik et al. \cite{maulik2021deploying} reported a speedup by applying ANN instead of the computationally expensive DSM.
To this end, we compared the computational time required to evaluate the policy and the dynamic procedure of the DSM on a single CPU core for the 48 DOF case.
We found that the time required was comparable for both cases with the RL policy requiring around 10 per cent more time on our hardware.
These results are encouraging, since we did not perform any optimizations of the policy in terms of computational efficiency or model size and did not use GPU acceleration for this comparison, which improves the performance of the RL-based policy significantly.

\subsection{Input Features}
\label{sec:results_features}

In a next step, the models trained on the local momentum field as inputs are compared to the models trained on the five invariants of the velocity gradient tensor $\lambda_{\nabla u}^i$.
The results in \figref{fig:invariants} indicate that again all models successfully improve during training.
However, the training is generally slower and less stable than the former models.
As a result, the final models using the invariants as inputs still partly improve over the analytical models, but perform worse than the models using the momentum field, especially in the 36 DOF case.
This indicates that it is generally harder to learn a sensible policy from the invariants.
To investigate this further, we increased the training time.
While the models always seemed to improve to some degree, even with double the amount of training iterations the gap to former models stayed quite substantial.

We attribute this to the different distributions of the input quantities.
For the considered HIT test case, the velocity fluctuations have zero mean and a root-mean-squared (RMS) magnitude of unity by construction.
The fluctuations are thus approximately normally distributed with zero mean and unit variance.
This is the optimal distribution for input quantities in machine learning, which typically has to be achieved by normalizing the inputs accordingly.
Since the velocity fluctuations are intrinsically linked to the energy budget in the simulation and are thus constraint by the forcing, the agent's actions have only relatively limited impact on the distribution of velocity fluctuations, as already shown in \figref{fig:spectra_n5}.
In contrast, the invariants of the velocity gradient tensor typically span orders of magnitude.
Moreover, the computation of the gradients, and thus the distribution of the invariants, differ widely depending on the employed numerical discretization.
For DG, the gradients are intrinsically discontinuous across element faces and are also known to produce large gradients at the element faces for underresolved turbulence.
This is especially problematic for the initial states, which are obtained by projecting the DNS flow field onto the DG basis with respective LES resolution.
This projection causes large gradients and thus large values for the invariants, which makes it hard to normalize them to a tamer distribution.
We thus assume that the problems in the training stem from the gradient computation of the DG method, for which the gradients exhibit a complex distribution and the invariants computed from it span orders of magnitude, which makes training more difficult for the agent.


\subsection{Generalization to other Resolutions}
\label{sec:results_generalize}

\begin{figure}
  \centering
  \includegraphics[width=0.99\linewidth]{tikz_double_column/draft-figure5.pdf}
  \caption{Results for the RL model trained on the 48 DOF evaluated on the 36 DOF case (left) and vice versa (right). Given are the energy spectra over the wavenumbers $k$ (top), the error of the spectra in comparison to the DNS solution (center) and the distribution of the predicted $C_s$ parameters (bottom). The shaded area indicates the maximum and minimum observed energy contained in the respective wavenumber during the DNS. The wavenumber that is resolved by the discretization with 4 points per wavelength is shown dashed and the maximum wavenumber used for optimization $k_{max}$ is indicated with a solid black line.}
  \label{fig:generalization_resolution}
\end{figure}

To demonstrate that the trained models can generalize to different resolutions, the model trained on the 48 DOF resolution is evaluated on the 36 DOF configuration and vice versa.
This allows to assess how well the trained models can be transferred to LES cases with either more or less resolution.
The results shown in \figref{fig:generalization_resolution} demonstrate that the trained models can also provide stable and accurate results in LES with different resolutions.
This is especially remarkable, since the policy's field of vision for the policy shrinks with increasing resolution due to the elementwise input and output quantities.
The RL model trained natively on the 48 DOF case shows a slight increase in energy in the higher wavenumbers, while the 36 DOF model seems to be slightly more dissipative.
However, the overall errors in the energy spectrum appear to be comparable for both cases.
The classical turbulence models are not shown for clarity.
However, since both RL models provide almost identical energy spectra, they still outperform the SSM and implicit LES model for both resolutions, while matching the performance of the DSM.

Also, the distribution of the models' predictions are almost identical for both cases and thus do not appear to change depending on the LES resolution.
The predictions of the 36 DOF model exhibit a much wider tail, with a maximum prediction of around $C_{s,max}=0.4$.
In contrast, the predictions of the trained 48 DOF model do not exceed $C_{s,max}=0.3$.
Interestingly, the models are still able to reproduce the target energy spectrum despite the deviations in their policies.
It is plausible to assume that the models will generalize even better, if they are trained on a variety of different resolutions, instead of only a single one.
These pronounced differences in the learned policies indicate that the distribution of predictions is not only induced by the input features but is a characteristic property of the policy trained on the respective resolution and the employed discretization.
This again demonstrates that the different discretizations induce different implicit LES filters, which again require different policies to match the underlying energy spectrum.
Thus, the proposed framework allows to develop discretization-adapted turbulent models for implicit LES.

\subsection{Generalization to other Reynolds Numbers}
\label{sec:results_generalize_re}

In a final step, we demonstrate that the trained RL policy is able to generalize to higher Reynolds numbers.
For this, the trained agents for the different resolutions are applied to a HIT flow at a Reynolds number of $Re_{\lambda}\approx 240$, which is considerably higher than the Reynolds number $Re_{\lambda}\approx 180$ used for training.
Analogously to \secref{sec:inference}, the LES were initialized from filtering the DNS flow field at a random point in time to the required resolution.
The LES was then advanced in time for $t_{end}=20$ and the results were averaged over the timeframe of $t\in [10,20]$ to investigate the long term effects of the model onto the flow.

\begin{figure*}[htb!]
  \centering
  \includegraphics[width=\textwidth]{tikz_double_column/draft-figure6.pdf}
  \caption{Results for the RL models trained on $Re_{\lambda}\approx180$ evaluated on a HIT flow with $Re_{\lambda}\approx240$. Reported are (from top to bottom) the averaged energy spectra over the wavenumbers $k$, the relative error of the energy spectra with respect to the DNS solution, the distributions $\mathcal{P}(\cdot)$ of the velocity fluctuations, and the distribution of the predicted $C_s$ parameters. The results for the underlying DNS solution as well as an implicitly modeled LES (iLES), the SSM with $C_s=0.17$ and the DSM are shown for comparison. The shaded area for the DNS energy spectrum indicates the maximum and minimum amplitudes observed for each mode. The wavenumber that is resolved by the discretization with 4 points per wavelength is shown dashed and the maximum wavenumber used for optimization $k_{max}$ is indicated with a solid black line.}
  \label{fig:generalization_re}
\end{figure*}

The results in \figref{fig:generalization_re} indicate that the trained models can indeed generalize to flows at higher Reynolds numbers.
Most importantly, the RL models still provide long-term stable simulations.
The RL models show similar behavior as for the Reynolds number seen during training. 
For the 32 DOF, 36 DOF and 48 DOF cases the RL models is able to reproduce the energy spectrum more accurately than the implicit model and the SSM, but with similar accuracy as the DSM. 
Interestingly, the DSM and RL models exhibit a similar buildup of energy near the cutoff wavenumber, which might indicate that these models lack sufficient dissipation.
Moreover, the RL model sill outperforms the other models and especially the DSM for the 24 DOF simulation, where the modeling assumptions of the SSM and DSM most probably do not hold.

These results are very promising, since they indicate that the trained RL policies are able to extrapolate to other Reynolds numbers (at least to a moderate extent).
The trained policies are thus able to generalize to higher Reynolds number flows as well as other LES resolutions, while matching or even improving on the performance of the DSM, which is known to provide outstanding results for HIT flows.

In this work, we discussed two approaches for representative sampling of CF-data, especially for accurately retaining the \emph{relative} performance of different recommendation algorithms. First, we proposed \sampler, which is better than commonly used sampling strategies, followed by introducing \oracle which \emph{analyzes} the performance of different samplers on different datasets. Detailed experiments suggest that \oracle can confidently estimate the downstream utility of any sampler within a few milliseconds, thereby enabling practitioners to benchmark different algorithms on $10$\% data sub-samples, with an average $5.8\times$ time speedup.

To realize the real-world environmental impact of \oracle, we discuss a typical weekly RecSys development cycle 
% in the industry 
and its carbon footprint. 
Taking the Criteo Ad dataset as inspiration, we assume a common industry-scale dataset to have $\sim4$B interactions.
% Inspired by the Criteo Ad Terabyte dataset, which is a collection of 24 days of ad-click logs, we assume a common industrial dataset to have around $4$ billion interactions. 
We assume a hypothetical use case that benchmarks for \eg $25$ different algorithms, each with $40$ different hyper-parameter variations. To estimate the energy consumption of GPUs, we scale the $0.4$ minute MLPerf \cite{mlperf} run of training NeuMF \cite{neural_mf} on the Movielens-20M dataset over an Nvidia DGX-2 machine. The total estimated run-time for all experiments would be $25 \times 40 \times \frac{4B}{20M} \times \frac{0.4}{60} \approx 1340$ hours; and following \cite{co2e}, the net CO$_2$ emissions would roughly be $10 \times 1340 \times 1.58 \times 0.954 \approx 20k$ lbs. To better understand the significance of this number, a brief CO$_2$ emissions comparison is presented in \cref{co2e}. Clearly, \oracle along with saving a large amount of experimentation time and cloud compute cost, can also significantly reduce the carbon footprint of this \emph{weekly process} by more than an average human's \emph{yearly} CO$_2$ emissions.

% \begin{wraptable}{r}{0.36\textwidth}
\begin{table}[!ht]
    \vspace{-5mm} %Put here to reduce too much white space after your table
    \begin{footnotesize} % normalsize, small, footnotesize
    \begin{center}
        \begin{tabular}{c c}
            \toprule
            \textbf{Consumption} & \textbf{CO$_2$e (lbs.)} \\ \midrule
            
            1 person, NY$\leftrightarrow$SF flight      & 2k \\
            Human life, 1 year avg.                     & 11k \\ 
            \midrule
            Weekly RecSys development cycle             & 20k \\
            '' \ \ \ \ \emph{w/} \oracle                & 3.4k \\
            
            \bottomrule
        \end{tabular}
    \end{center}
    \end{footnotesize}
    \vspace{2mm}
    \caption{CO$_2$ emissions comparison \cite{co2e}}
    % \caption*{Table 1: CO$_2$ emissions comparison}
    \label{co2e}
    \vspace{-10mm} %Put here to reduce too much white space after your table
\end{table}
% \end{wraptable}

Despite having significantly benefited the run-time and environmental impacts of benchmarking %recommendation 
algorithms on massive datasets, our analysis heavily relied on the experiments of training seven 
%different 
recommendation algorithms on six datasets and their various samples. Despite the already large experimental cost, we strongly believe that the downstream performance of \oracle could be further improved by simply 
% adding 
considering
more algorithms and diverse datasets.
% with different characteristics. 
In addition to better sampling, analyzing the fairness 
% and privacy 
aspects of training algorithms on sub-sampled datasets is an interesting research direction, which we plan to explore in future work.
% Future work: Fairness, SVP w/ sequential, New samplers
\section{Conclusion}
\label{ss: conclusion}

% summary of approach
This paper presents a methodology to evaluate the effectiveness of evasions and its application to studying PDF malware scanners.
Our implementation of the methodology, the Chameleon framework, automatically generates and enriches malicious documents with one or multiple evasions.
We use these documents for an in-depth study of \nbAnalyzers{} PDF scanners and how they are affected by evasions.
More broadly, our methodology can also be used for studying evasions of other malware types, e.g., malicious executables.

% main take-aways
The overall result of our study is cause for concern.
We show that the studied evasions are surprisingly effective in fooling state-of-the-art scanners.
In particular by combining evasions, attackers can bypass modern defenses in both static and dynamic scanners.
Moreover, we find huge variations across scanners, enabling targeted attacks based on evasions picked specifically for a targeted scanner.
All these findings are a call to arms for future work on anti-evasion techniques.

Our work will support future efforts toward improving malware scanners in several ways.
First, the results of our study help security vendors to better understand their vulnerability to specific evasions and to focus their attention on mitigating the most effective evasions.
Second, we are releasing the corpus of malicious, evasive documents generated by Chameleon as a ready-to-use benchmark.
We are in contact with several developers of PDF scanners, and some of them, e.g., SploitGuard and SAFE-PDF, have already used our benchmark to test and improve their security scanners.
Finally, the Chameleon framework provides a basis for expanding the set of benchmarks by incorporating future evasions, exploits, and payloads.



%% The next two lines define the bibliography style to be used, and
%% the bibliography file.
\bibliographystyle{ACM-Reference-Format}
\bibliography{main}

\appendix
\newpage
\appendix
% \section{Appendix}
\section{Ablation Study}
\label{appendix:ablation}
%(2) We can clearly observe a tradeoff between the degree of freedom for manipulation and attack success rate. For example, we observe a small drop in the attack success rate for answer targeted attack compared to position targeted attack, due to the fact that we put more constraints to ensure pre-specified answer targets unchanged in the optimization process. Similarly, the dependency tree constraints turn out to be more strong and harsh constraints on the adversarial sentences, thus achieving higher language quality at the cost of  attack success rate. 
%(2)
%(3) \boxin{How to say because our transfer based blackattack does not beat AddSent because it is input-agnoistic.? while ours are more model-specific?}  (4) BERT based sentiment classifier is more vulnerable than standard sentiment classifier, while BERT based QA model is more robust and harder to attack than the widely-used QA model.

\subsection{Autoencoder Selection}
As an ablation study, we compare the standard LSTM-based autoencoder with our tree-based autoencoder. 

\begin{table}[htp!]\small \setlength{\tabcolsep}{5pt}
\centering
\caption{Ablation study on posistion targeted attack capability against QA. The lower EM and F1 scores mean the better attack success rate. \advcodecsent and \advcodecword respectively refer to \advcodecsent and \advcodecword. Adv(seq2seq) refers to \advcodec that uses LSTM-based seq2seq model as text autoencoder.}
 \label{WhiteboxQAseq2seq}
\begin{tabular}{ccccc}
\toprule
% \multirow{2}{*}{Model} & & \multirow{2}{*}{Origin} & \multicolumn{2}{c}{w/ Tree Decoder} & w/o Tree Decoder  \\
% \cmidrule(lr){4-5}   \cmidrule(lr){6-6}
  & Origin & {\advcodecsent} & {\advcodecword} & Adv(seq2seq)  \\
\midrule
EM & 60.0 & 29.3     & \textbf{15.0}  & 51.3  \\
 F1 & 70.6 &  34.0   & \textbf{17.6}  &      57.5 \\
      \bottomrule
\end{tabular}
% \vspace{-3mm}
\end{table}


\begin{table*}[htp!]\small \setlength{\tabcolsep}{7pt}
 \begin{minipage}[htp!]{0.48\linewidth}
\centering
\caption{Blackbox Attack Success Rate after inserting the whitebox generated adv sentence to different positions for BERT-classification.  }
 \label{ablationClassification}
\begin{tabular}{ccccc}
\toprule
Method & & Back & Mid & Front \\
\midrule
\multirow{2}{*}{\advcodecword} & \footnotesize{target}   & 0.739   & 0.678  & \textbf{0.820} \\
      & \footnotesize{untarget} & 0.817 & 0.770  & \textbf{0.878}           \\
      \midrule
\multirow{2}{*}{\advcodecsent} & \footnotesize{target}   & \textbf{0.220}   & 0.174  & 0.217 \\
      & \footnotesize{untarget} & 0.531 & 0.504  & \textbf{0.532}           \\
        \bottomrule
\end{tabular}
\vspace{-0.2cm}
\end{minipage}
\quad
\begin{minipage}[htp!]{0.48\linewidth}
\centering
\caption{Blackbox Attack Success Rate after inserting the whitebox generated adversarial sentence to different positions for BERT-QA.}
 \label{ablationQA}
\begin{tabular}{ccccc}
\toprule
Method & & Back & Mid & Front \\
\midrule
\multirow{2}{*}{\advcodecword}  & EM &  32.3    & 39.1    & \textbf{31.9}  \\
      & F1 & 36.4   & 43.4     & \textbf{36.3}   \\   
      \midrule
\multirow{2}{*}{\advcodecsent} & EM & 47.0   & 51.3     & \textbf{42.4}           \\
      &  F1 & 52.0     & 56.7         & \textbf{47.0}          \\
        \bottomrule
\end{tabular}
\vspace{-0.2cm}
\end{minipage}
\end{table*}

\textbf{Tree Autoencoder.} 
In the whole experiments, we used Stanford TreeLSTM as tree encoder and our proposed tree decoder together as tree autoencoder. We trained the tree autoencoder on yelp dataset which contains 500K reviews. The model is expected to read a sentence, map the sentence in a latent space and reconstruct the sentence from the embedding along with the dependency tree structure in an unsupervised manner. The model uses 300-d vectors as hidden tree node embedding and is trained for 30 epochs with adaptive learning rate and weight decay. After training, the average reconstruction loss on test set is 0.63.

\textbf{Seq2seq Autoencoder.} We also evaluate the standard LSTM-based architecture (seq2seq) as a different autoencoder in the \advcodec pipeline. For the seq2seq encoder-decoder, we use a bi-directional LSTM as the encoder \citep{Hochreiter1997LongSM} and a two-layer LSTM plus soft attention mechanism over the encoded states as the decoder \citep{Bahdanau2015NeuralMT}. With 400-d hidden units and the dropout rate of 0.3, the final testing reconstruction loss is 1.43.

The comparison of the whitebox attack capability  against a well-known QA model BiDAF is shown in Table \ref{WhiteboxQAseq2seq}. We can see seq2seq based \advcodec fails to achieve good attack success rate. Moreover, because the vanilla seq2seq model does not take grammatical constraints into consideration and has higher reconstruction loss, the quality of generated adversarial text cannot be ensured.

\subsection{Ablation Study on BERT Attention}
\label{sec:ablation}
To further explore how the location of adversarial sentences affects the attack success rate, we conduct the ablation experiments by varying the position of appended adversarial sentence. We generate the adversarial sentences from the whitebox BERT classification and QA models. Then we inject those adversaries into different positions of the original paragraph and test in another blackbox BERT with the same architecture but different parameters. The results are shown in Table \ref{ablationClassification} and \ref{ablationQA}. We see in most time appending the adversarial sentence at the beginning of the paragraph achieves the best attack performance. Also the performance of appending the adversarial sentence at the end of the paragraph is usually slightly weaker than front. This observation suggests that the BERT model might pay more attention to the both ends of the paragraphs and tend to overlook the content in the middle.


% \textbf{Ablation Study.} \boxin{change the language here (same as sec 4.1)} To further explore how the appended location will impact the attack success rate, we conduct the ablation experiment by varying the position of appended adversarial sentence and the results are shown in table \ref{ablationQA}. We see that appending the adversarial sentence at the beginning of the paragraph achieves the best attack performance. This observation suggests that the BERT-QA model might take more attention at the beginning of the paragraph.


\subsection{Attack Settings}
% \begin{algorithm}[b]
%   \caption{Algorithm of \advcodec generating adversarial examples } \label{algo}
%   \begin{algorithmic}[1]
%     \Procedure{AdvCodec}{$x,s$} \Comment{$x$: initial seed, $s$: corresponding dependency tree}
%     \State $z := \mathcal{E}(x, s)$ \Comment{$\mathcal{E}$: encoder of \advcodec, $z$: context vector}
%     \State $z^* = 0$ \Comment{$z^*$: perturbation on context vector}
%     \State $z' := z + z^*$ \Comment{$z'$: perturbed context vector}
%     \State $y := \mathcal{G}(z', s)$ \Comment{$\mathcal{G}$: decoder of \advcodec, $y$: adversarial sentence}
%   % \State $Z(y) :=$ the logits of the model output
%     \State $f(z') :=$ the objective function to attack the targeted model
%     \While{$y$ does not achieve targeted attack} 
%       \State  update $z^*$ by gradient descent over objective function $f(z')$
%     \EndWhile\label{euclidendwhile}
%     \State \textbf{return} $y$
%     \EndProcedure
%   \end{algorithmic}
% \end{algorithm}
We use Adam \citep{Adam} as the optimizer, set the learning rate to 0.6 and the optimization steps to 100. We follow the \citet{cw} method to find the suitable parameters in the object function (weight const $c$ and confidence score $\kappa$) by binary search. 

% We also include our attack algorithm via pseudo-code in Algorithm \ref{algo}.


% \iffalse
% \subsection{Untargeted scatter attack on QA}

% We tried the scatter attack on QA, however, the targeted attack success rate is not satisfactory. It turns out QA systems highly rely on the relationship between questions and contextual clues, which is hard to break when setting an arbitrary token to a target answer. This is also why we use some preliminary approaches to creating a similar fake context when initializing QA appended sentence. 

% We also performed the untargeted scatter attack on QA. The results are shown in table \ref{WhiteboxQAScatter}. We insert 30 random tokens (but  no more than $1/3$ the total words of the paragraph) over the paragraph, optimize and find the adversarial tokens that can cause model output the wrong answers in the untargeted manner.  We can see the untargeted scatter attack can also achieve a higher untargeted attack success rate than \citet{jia-liang-2017-adversarial}.

% \begin{table*}[htp!]\small \setlength{\tabcolsep}{5pt}
% \centering
% \caption{Whitebox attack results on BERT-QA in terms of exact match rates and F1 scores by the official evaluation script. The lower EM and F1 scores mean the better attack success rate.}
%  \label{WhiteboxQAScatter}
% \begin{tabular}{ccccccccc}
% \toprule
% \multirow{2}{*}{Model} & & \multirow{2}{*}{Origin} & \multicolumn{2}{c}{Position Targeted Attack} & \multicolumn{2}{c}{Answer Targeted Attack} & \multicolumn{2}{c}{Untargeted Attack} \\
% \cmidrule(lr){4-5} \cmidrule(lr){6-7} \cmidrule(lr){8-9}
%  & & & {\advcodecsent} & {\advcodecword}  & {\advcodecsent} & {\advcodecword} & AddSent & Adv(scatter)\\
% \midrule

% \multirow{2}{*}{BERT}  & EM & 81.2 &49.1       & \textbf{29.3}           & 50.9                    & 43.2                    & 46.8  & 34.3   \\
%       & F1 & 88.6 & 53.8          & \textbf{33.2}         & 55.2                   & 47.3                  & 52.6  & 49.7 \\
% %      & $\Delta \text{F1}$ & $=$ & 34.8  & \textbf{55.4} & 33.4 & 41.3 & 36.0 \\
% %       \midrule
% % \multirow{2}{*}{BiDAF} & EM & 60.0 & 29.3  	          & \textbf{15.0}             & 30.2                    & 21.0                      & 25.3    \\
% %       & F1 & 70.6 &  34.0   & \textbf{17.6}         & 34.4                  & 23.6                  & 32.0 \\
% \bottomrule
% \end{tabular}
% \end{table*}
% \fi

\subsection{Heuristic Experiments on choosing the adversarial seed for QA}
\label{appendix:heuristic}

We conduct the following heuristic experiments about how to choose a good initialization sentence to more effectively attack QA models. Based on the experiments we confirm it is important to choose a sentence that is semantically close to the context or the question as the initial seed when attacking QA model, so that we can reduce the number of iteration steps and more effectively find the adversary to fool the model. Here we describe three ways to choose the initial sentence, and we will show the efficacy of these methods given the same maximum number of optimization steps.

\textbf{Random adversarial seed sentence.}
Our first trial is to use a random sentence (other than the answer sentence), generate a fake answer similar to the real answer and append it to the back as the initial seed.

\textbf{Question-based adversarial seed sentence.}
% question words in a question , paragraph pair <p, q> 
We also try to use question words to craft an initial sentence, which in theory should gain more attention when the model is matching characteristic similarity between the context and the question. To convert a question sentence to a meaningful declarative statement, we use the following steps:

In step 1, we use the state-of-the-art semantic role labeling (SRL) tools \citep{He2017DeepSR} to parse the question into verbs and arguments. A set of rules is defined to remove the arguments that contain interrogative words and unimportant adjectives, and so on. In the next step, we access the model's original predicted answer and locate the answer sentence. We again run the SRL parsing and find to which argument the answer belongs. The whole answer argument is extracted, but the answer tokens are substituted with our targeted answer or the nearest words in the GloVe word vectors \citep{Pennington2014GloveGV} (position targeted attack) that is also used in the QA model. In this way, we craft a fake answer that shares the answer's context to solve the compatibility issue from the starting point. Finally, we replace the declarative sentence's removed arguments with the fake argument and choose this question-based sentence as our initial sentence.

\textbf{Answer-based adversarial seed  sentence.}
We also consider directly using the model predicted original answer sentence with some substitutions as the initial sentence. To craft a fake answer sentence is much easier than to craft from the question words. Similar to step 2 for creating
question-based initial sentence, we request the model's original predicted answer and find the answer sentence. The answer span in the answer sentence is directly substituted with the nearest words in the GloVe word vector space to avoid the compatibility problem preliminarily.

\textbf{Experimental Results.} We tried the above initial sentence selection methods on \advcodecword and perform position targeted attack on BERT-QA given the same maximum optimization steps. The experiments results are shown in table \ref{WhiteboxQAHeuristic}. From the table, we find using different initialization methods will greatly affect the attack success rates. Therefore, the initial sentence selection methods are indeed important to help reduce the number of iteration steps and fastly converge to the optimal $z^*$ that can attack the model.

\begin{table*}[htp!]\small \setlength{\tabcolsep}{5pt}
\centering
\caption{Whitebox attack results on BERT-QA in terms of exact match rates and F1 scores by the official evaluation script. The lower EM and F1 scores mean the better attack success rate.}
 \label{WhiteboxQAHeuristic}
\begin{tabular}{ccccccc}
\toprule
\multirow{2}{*}{Model} & & \multirow{2}{*}{Origin} & \multicolumn{3}{c}{Position Targeted Attack}  & \multicolumn{1}{c}{Baseline} \\
\cmidrule(lr){4-6} \cmidrule(lr){7-7}
 & & & Random & Question-based  & Answer-based  & AddSent\\
\midrule

\multirow{2}{*}{BERT}  & EM & 81.2 & 67.9       & \textbf{29.3}           & 50.6                               & 46.8   \\
      & F1 & 88.6 & 74.4         & \textbf{33.2}         & 55.2    & 52.6   \\
\bottomrule
\end{tabular}
\end{table*}

%\subsection{Conclusions}
% In addition to the general adversarial evaluation framework \advcodec, this paper also aims to explore several scientific questions: 1)  Since \advcodec allows the flexibility of manipulating at different levels of a tree hierarchy, which level is more attack effective and which one preserves better grammatical correctness? 2) Is it possible to achieve the targeted attack for general NLP tasks such as sentiment classification and QA, given the limited degree of freedom for manipulation? 3) Is it possible to perform a blackbox attack for many  NLP tasks? 4) Is BERT robust in practice? 
% 5) Do these adversarial examples affect human reader performances? 
% %\boxin{I think the above question is readers caring more. 5) Are human readers more sensitive to an appended adversarial sentence or scatter of added words?

% To address the above questions, we generate adversarial text against different models of sentiment classification and QA in each encoding scenario. Compared with the state-of-the-art adversarial text generation methods, our approach achieves significantly higher untargeted and \emph{targeted} attack success rate. In addition, we perform both whitebox and transferability-based blackbox settings to evaluate the model vulnerabilities. 
% Within each attack setting, we quantitatively evaluate the attack effectiveness of different attack strategies, including appending an additional adversarial sentence and adding scatter of adversarial words to a paragraph.
% To provide thorough adversarial text quality assessment, we also perform 7 groups of human studies to evaluate the quality of the generated adversarial text. % Compared with the baselines methods, and whether a human can still get the ground truth answers for these tasks based on adversarial text.

% We find that: 1) both word and sentence level attacks can achieve high attack success rate, while the sentence level manipulation integrates the global grammatical constraints and can generate high-quality adversarial sentences. 2) various targeted attacks on general NLP tasks are possible (\textit{e.g.}, when attacking QA, we can ensure  the target to be a specific answer or a specific location within a sentence); 3) the transferability based blackbox attacks are successful in NLP tasks. Transferring adversarial text from stronger models (in terms of performances) to weaker ones is more successful; 4)  Although BERT has achieved state-of-the-art performances, we observe the performance drops are also more substantial than other models when confronted with adversarial examples, which indicates BERT is not robust enough under the adversarial settings.
% %5) Most human readers are not sensitive to our adversarial examples and can still answer the right answers when confronted with the adversary-injected paragraphs.

% Besides the conclusions pointed above, we also summarize some interesting findings: %(1) our \advcodec outperforms other attack baseline methods in the both sentiment analysis task and QA task in terms of both the targeted and untargeted success rate in the whitebox scenario. 
% (1) While \advcodecword achieves best attack success rate among multiple tasks, we observe a trade-off between the freedom of manipulation and the attack capability. For instance, \advcodecsent has dependency tree constraints and becomes more natural for human readers than but less effective to attack models than \advcodecword. Similarly, the answer targeted attack in QA has fewer words to manipulate and change than the position targeted attack, and therefore has slightly weaker attack performances.
% % (2) Scatter attack is as effective as concat attack in sentiment classification task but less successful in QA, because QA systems make decisions highly based on the contextual correlation between the question and the paragraph, which makes it difficult to set an arbitrary token as our targeted answer.
% (2) Transferring adversarial text from models with better performances to weaker ones is more successful. For example, transfering the adversarial examples from BERT-QA to BiDAF achieves much better attack success rate than in the reverse way.
% (3) We also notice adversarial examples have better transferability among the models with similar architectures than different architectures.
% (4) BERT models give higher attention scores to the both ends of the paragraphs and tend to overlook the content in the middle, as shown in \S \ref{sec:ablation} ablation study that adding adversarial sentences in the middle of the paragraph is less effective than in the front or the end.

% To defend against these adversaries, here we discuss about the following possible methods and will in depth explore them in our future works: 
% (1) \textbf{Adversarial Training} is a practical methods to defend against adversarial examples. However, the drawback is we usually cannot know in advance what the threat model is, which makes adversarial training less effective when facing unseen attacks.
% (2) \textbf{Interval Bound Propagation} (IBP) \citep{Dvijotham2018TrainingVL} is proposed as a new technique to theoretically consider the worst-case perturbation. Recent works \citep{Jia2019CertifiedRT,Huang2019AchievingVR} have applied IBP in the NLP domain to certify the robustness of models. (3) \textbf{Language models} including GPT2 \citep{Radford2019LanguageMA} may also function as an anomaly detector to probe the inconsistent and unnatural adversarial sentences.


\section{Experimental Settings}
\label{appendix:setup}
\subsection{Sentiment Classification Model}
 \textbf{BERT.} We use the 12-layer BERT-base model \footnote{https://github.com/huggingface/pytorch-pretrained-BERT} with 768 hidden units, 12 self-attention heads and 110M parameters. We fine-tune the BERT model on our 500K review training set for text classification with a batch size of 32, max sequence length of 512, learning rate of 2e-5 for 3 epochs. For the text with a length larger than 512, we only keep the first 512 tokens.
 
 
 \textbf{ Self-Attentive Model (SAM).} We choose the structured self-attentive sentence embedding model \citep{nfc512} as the testing model, as it not only achieves the state-of-the-art results on the sentiment analysis task among other baseline models but also provides an approach to quantitatively measure model attention and helps us conduct and analyze our adversarial attacks. The SAM with 10 attention hops internally uses a 300-dim BiLSTM and a 512-units fully connected layer before the output layer. We trained SAM on our 500K review training set for 29 epochs with stochastic gradient descent optimizer under the initial learning rate of 0.1.
 
 \subsection{Sentiment Classification Attack Baseline}
 \textbf{Seq2sick} \citep{seq2sick} is a whitebox projected gradient method combined with group lasso and gradient regularization to craft adversarial examples to fool seq2seq models. Here, we define the loss function as $ L_{target} = \max\limits_{k \in Y} \left\{z^{\left(k\right)} \right\} - z^{\left(t\right)} $ to perform attack on sentiment classification models which was not evaluated in the original paper. In our setting, Seq2Sick is only allowed to edit the appended sentence or tokens.
 
 \textbf{TextFooler} \citep{TextFooler} is a simple but strong black-box attack method to generate adversarial text. Here, TextFooler is also only allowed to edit the appended sentence.

\subsection{QA Model}
\textbf{{BiDAF}.} Bi-Directional Attention Flow (BIDAF) network\citep{seo2016-bidirectional} is a multi-stage hierarchical process that represents the context at different levels of granularity and uses bidirectional attention flow mechanism to obtain a query-aware context representation. We train BiDAF without character embedding layer under the same setting in \citep{seo2016-bidirectional} as our testing model.

\subsection{Human Evaluation Setup}
\label{appendix:human}

We focus on two metrics to evaluate the validity of the generated adversarial sentence:
\textbf{adversarial text quality} and  \textbf{human performance} on the original and adversarial dataset. To evaluate the adversarial text quality, human participants are asked to choose the data they think has better quality. 

% To ensure that human is not misled by our adversarial examples, we ask human participants to perform the sentiment classification and question answering tasks both on the original dataset and adversarial dataset. We hand out the adversarial dataset and origin dataset to $533$ Amazon Turkers to perform the human evaluation. More experimental details can be found in Appendix \ref{}.

To evaluate the adversarial text quality, human participants are asked to choose the data they think has better quality. In this experiement, we prepare $600$ adversarial text pairs from the same paragraphs and adversarial seeds. We hand out these pairs to $28$ Amazon Turks. Each turk is required to annotate at least 20 pairs and at most 140 pairs to ensure the task has been well understood. We assign each pair to at least 5 unique turks and take the majority votes over the responses. 


% Adversarial dataset on sentiment classification consists of \advcodecsent concatenative adversarial examples and \advcodecword scatter attack examples. Adversarial dataset on QA consists of concatenative adversarial examples generated by both \advcodecsent and \advcodecword. 
To ensure that human is not misled by our adversarial examples, we ask human participants to perform the sentiment classification and question answering tasks both on the original dataset and adversarial dataset. Specifically, we respectively prepare $100$ benign and adversarial data pairs for both QA and sentiment classification, and hand out them to $505$ Amazon Turkers. Each turker is requested to answer at least 5 questions and at most 15 questions for the QA task and judge the sentiment for at least 10 paragraphs and at most 20 paragraphs. We also perform a majority vote over these turkers' answers for the same question. 

\subsection{Human Error Analysis in Adversarial Dataset}
\label{appendix:humanerror}
We compare the human accuracy on both benign and adversarial texts for both tasks (QA and classification) in revision section 5.2. We spot the human performance drops a bit on adversarial texts. In particular, it drops around $10\%$ for both QA and classification tasks based on AdvCodec as shown in Table \ref{tab:human}. We believe this performance drop is tolerable and the stoa generic based QA attack algorithm experienced around $14\%$ performance drop for human performance \citep{jia-liang-2017-adversarial}.

We also try to analyze the human error cases. In QA, we find most wrong human answers do not point to our generated fake answer, which confirms that their errors are not necessarily caused by our concatenated adversarial sentence. Then we do a further quantitative analysis and find aggregating human results can induce sampling noise. Since we use majority vote to aggregate the human answers, when different answers happen to have the same votes, we will randomly choose one as the final result. If we always choose the answer that is close to the ground truth in draw cases, we later find that the majority vote F1 score increases from $82.897$ to $89.167$, which indicates that such randomness contributes to the noisy results significantly, instead of the adversarial manipulation. Also, we find the average length of the adversarial paragraph is around $12$ tokens more than the average length of the original one after we append the adversarial sentence. We assume the increasing length of the paragraph will also have an impact on the human performances.
 
 
% \iffalse
% \section{Adversarial text on sentiment analysis}
% \textbf{Scatter Attack} In the scatter attack scenario, Table \ref{scatterwhite}  and Table \ref{scatterblack} show that our \advcodecword outperforms the Seq2sick baseline on both whitebox and transferability based blackbox attacks.

% \begin{table*}[htp!]\small \setlength{\tabcolsep}{7pt}
% \centering
% \caption{Whitebox scatter attack results on Sentiment Analysis}
%  \label{scatterwhite}
% \begin{tabular}{lccc}
% \toprule
% \multicolumn{2}{l}{Model} & \advcodecword & Seq2Sick \\
% \midrule
% \multirow{2}{*}{BERT}  & Targeted  & \textbf{0.976}          & 0.946    \\
%       & Untargeted & \textbf{0.987}         & 0.970   \\
%       \midrule
% \multirow{2}{*}{BiDAF} & target  & \textbf{0.869}          & 0.570   \\
%       & Untargeted & \textbf{0.948}         & 0.711  \\
%       \bottomrule
% \end{tabular}
% \end{table*}

% \begin{table*}[htp!]\small \setlength{\tabcolsep}{7pt}
% \centering
% \caption{Blackbox scatter attack results on Sentiment Analysis}
%  \label{scatterblack}
% \begin{tabular}{lccc}
% \multicolumn{2}{l}{Model A -- B} & \advcodecword & Seq2Sick \\
% \toprule
% \multirow{2}{*}{BERT-SAM} & Targeted & \textbf{0.465}          & 0.230     \\
%          & Untargeted    & \textbf{0.679}          & 0.498    \\
%         \midrule
% \multirow{2}{*}{SAM-BERT} & target & \textbf{0.298}          & 0.156   \\
%          & Untargeted    & \textbf{0.574}          & 0.445  \\
%          \bottomrule
% \end{tabular}
% \end{table*}
% \fi

\onecolumn
\newpage
\section{Adversarial examples}
\label{appendix:examples}
\subsection{Adversarial examples for QA}
\subsubsection{Adversarial examples generated by \advcodecsent}

\begin{table}[htp!]
\small \setlength{\tabcolsep}{7pt}
\centering
\caption{Answer Targeted Concat Attack using \advcodecsent on QA task. The targeted answer is ``Donald Trump''.
%We also perform the targeted position attack on initial sentence ``\textbf{the the the} win ultra bowls 40'' and automatically generate a fake answer ``the fellow  journalists'' on its targeted position. 
}
\begin{tabular}{p{13.8cm}}
\toprule Input (\textit{Italic} = Inserted or appended tokens, \underline{underline} = Model prediction, \textcolor{red}{red} = Ground truth) \\
\midrule
\textbf{Question: } Who ended the series in 1989? \\
\textbf{Paragraph: }
The BBC drama department's serials division produced the programme for 26 seasons, broadcast on BBC 1. Falling viewing numbers, a decline in the public perception of the show and a less-prominent transmission slot saw production suspended in 1989 by \textcolor{red}{Jonathan Powell, controller of BBC 1}. Although (as series co-star Sophie Aldred reported in the documentary Doctor Who: More Than 30 Years in the TARDIS) it was effectively, if not formally, cancelled with the decision not to commission a planned 27th series of the show for transmission in 1990, the BBC repeatedly affirmed that the series would return. \textit{\underline{Donald Trump} ends a program on 1988 .}
% \newline
% \\ 
% \textbf{Question: }
% What kind of network was ABC when it first began? \\
% \textbf{Paragraph: }
% ABC originally launched on October 12, 1943 as a \answer{radio network}, separated from and serving as the successor to the NBC Blue Network, which had been purchased by Edward J. Noble. It extended its operations to television in 1948, following in the footsteps of established broadcast networks CBS and NBC. In the mid-1950s, ABC merged with United Paramount Theatres, a chain of movie theaters that formerly operated as a subsidiary of Paramount Pictures. Leonard Goldenson, who had been the head of UPT, made the new television network profitable by helping develop and greenlight many successful series. In the 1980s, after purchasing an 80\% interest in cable sports channel ESPN, the network's parent merged with Capital Cities Communications, owner of several print publications, and television and radio stations. In 1996, most of Capital Cities/ABC's assets were purchased by The Walt Disney Company. \textit{\underline{Donald Trump} sort in network is featured if this first started.}
\newline
\\ 
\textbf{Question: } Who served his dinner?  \\
\textbf{Paragraph: }
Tesla worked every day from 9:00 a.m. until 6:00 p.m. or later, with dinner from exactly 8:10 p.m., at Delmonico's restaurant and later the Waldorf-Astoria Hotel. Tesla would telephone his dinner order to the \answer{headwaiter}, who also could be the only one to serve him. \"The meal was required to be ready at eight o'clock ... He dined alone, except on the rare occasions when he would give a dinner to a group to meet his social obligations. Tesla would then resume his work, often until 3:00 a.m. \textit{\underline{Donald Trump} serves her supper .}
\\
\\ 
\textbf{Question: }
Which network regained the ratings lead in America in 1984? \\
\textbf{Paragraph: }
As far as programming is concerned, four of ABC's marquee shows of the 1970s ended their runs during the mid-1980s: Laverne \& Shirley ended its run in 1983, Happy Days and Three's Company ended in 1984 (with the latter producing a short-lived spinoff that year), while The Love Boat ended its run in 1986. After nearly a decade of ratings trouble, NBC had regained the ratings lead among the Big Three networks in 1984 on the success of series such as The Cosby Show, Cheers and Miami Vice. To counteract \answer{NBC}, ABC decided to refocus itself on comedies and family-oriented series beginning in the mid-1980s including Mr. Belvedere, Roseanne, Who's the Boss?, Just the Ten of Us, The Wonder Years, Full House and Perfect Strangers. \textit{\underline{Donald Trump} networks regain a rating leads on american on 1985 .}
\\
\bottomrule
\end{tabular}
\end{table}


\begin{table*}[!htbp]\small \setlength{\tabcolsep}{7pt}
\centering
\caption{Position Targeted Concat Attack using \advcodecsent on QA task. The adversarial answer is generated automatically.
%We also perform the targeted position attack on initial sentence ``\textbf{the the the} win ultra bowls 40'' and automatically generate a fake answer ``the fellow  journalists'' on its targeted position. 
}
 \label{posqasentexamples}
\begin{tabular}{p{13.8cm}}
\toprule Input (\textit{Italic} = Inserted or appended tokens, \underline{underline} = Model prediction, \textcolor{red}{red} = Ground truth) \\
\midrule
\textbf{Question: }How many other contestants did the company, that had their ad shown for free, beat out? \\
\textbf{Paragraph: }
QuickBooks sponsored a \"Small Business Big Game\" contest, in which Death Wish Coffee had a 30-second commercial aired free of charge courtesy of QuickBooks. Death Wish Coffee beat out \answer{nine} other contenders from across the United States for the free advertisement. \textit{The company , that had their ad shown for free ad \underline{two} .}
\newline
\\ 
\textbf{Question: }
Why would a teacher's college exist? \\
\textbf{Paragraph: }
There are a variety of bodies designed to instill, preserve and update the knowledge and professional standing of teachers. Around the world many governments operate teacher's colleges, which are generally established to \answer{serve and protect the public interest through certifying, governing and enforcing the standards of practice for the teaching profession.} \textit{A friend 's school exist \underline{for community , serving a private businesses}},
\newline
\\ 
\textbf{Question: }
What can concentrated oxygen produce? \\
\textbf{Paragraph: }
Highly concentrated sources of oxygen promote \answer{rapid combustion}. Fire and explosion hazards exist when concentrated oxidants and fuels are brought into close proximity; an ignition event, such as heat or a spark, is needed to trigger combustion. Oxygen is the oxidant, not the fuel, but nevertheless the source of most of the chemical energy released in combustion. Combustion hazards also apply to compounds of oxygen with a high oxidative potential, such as peroxides, chlorates, nitrates, perchlorates, and dichromates because they can donate oxygen to a fire. \textit{Concentrated containers produce \underline{~ rapidly }.}
\\
\bottomrule
\end{tabular}
\end{table*}



\clearpage
\subsubsection{Adversarial examples generated by \advcodecword}
\begin{table*}[!htbp]\small \setlength{\tabcolsep}{7pt}
\centering
\caption{Answer Targeted Concat Attack using \advcodecword on QA task. The targeted answer is Donald Trump.
%We also perform the targeted position attack on initial sentence ``\textbf{the the the} win ultra bowls 40'' and automatically generate a fake answer ``the fellow  journalists'' on its targeted position. 
}
 \label{ansqasentexamples}
\begin{tabular}{p{13.8cm}}
\toprule Input (\textit{Italic} = Inserted or appended tokens, \underline{underline} = Model prediction, \textcolor{red}{red} = Ground truth) \\
\midrule
\textbf{Question: } What did he do to his feet at night? \\
\textbf{Paragraph: }
For exercise, Tesla walked between 8 to 10 miles per day. He \answer{squished his toes} one hundred times for each foot every night, saying that it stimulated his brain cells. \textit{She did for her ft in evening \underline{Donald Trump.}}
\newline
\\ 
\textbf{Question: }
Who announced the game play-by-play for Super Bowl 50? \\
\textbf{Paragraph: }
Westwood One will carry the game throughout North America, with \answer{Kevin Harlan} as play-by-play announcer, Boomer Esiason and Dan Fouts as color analysts, and James Lofton and Mark Malone as sideline reporters. Jim Gray will anchor the pre-game and halftime coverage. \textit{\underline{Donald Trump} announces a games plays , with , plays in ultra bowls 40.}
\newline
\\ 
\textbf{Question: } What group can teachers in Wales register with?  \\
\textbf{Paragraph: }
Teachers in Wales can be registered members of \answer{trade unions} such as ATL, NUT or NASUWT and reports in recent years suggest that the average age of teachers in Wales is falling with teachers being younger than in previous years. A growing cause of concern are that attacks on teachers in Welsh schools which reached an all-time high between 2005 and 2010. \textit{Teachers on their group group \underline{Donald Trump} .}\\
% \newline
% \\ 
% \textbf{Question: }
% What bargain did his father make with him if Tesla recovered? \\
% \textbf{Paragraph: }
% In 1873, Tesla returned to his birthtown, Smiljan. Shortly after he arrived, Tesla contracted cholera; he was bedridden for nine months and was near death multiple times. Tesla's father, in a moment of despair, promised \answer{to send him to the best engineering school} if he recovered from the illness (his father had originally wanted him to enter the priesthood). \textit{Her mother makes \underline{Donald Trump}}.
% \\
\bottomrule
\end{tabular}
\end{table*}

\begin{table*}[htbp!]\small \setlength{\tabcolsep}{7pt}
\centering
\caption{Position Targeted Concat Attack using \advcodecword on QA task. The adversarial answer is generated automatically.
%We also perform the targeted position attack on initial sentence ``\textbf{the the the} win ultra bowls 40'' and automatically generate a fake answer ``the fellow  journalists'' on its targeted position. 
}
 \label{posqawordexamples}
\begin{tabular}{p{13.8cm}}
\toprule Input (\textit{Italic} = Inserted or appended tokens, \underline{underline} = Model prediction, \textcolor{red}{red} = Ground truth) \\
\midrule
\textbf{Question: } IP and AM are most commonly defined by what type of proof system?\\
\textbf{Paragraph: }
Other important complexity classes include BPP, ZPP and RP, which are defined using probabilistic Turing machines; AC and NC, which are defined using Boolean circuits; and BQP and QMA, which are defined using quantum Turing machines. \#P is an important complexity class of counting problems (not decision problems). Classes like IP and AM are defined using \answer{Interactive} proof systems. ALL is the class of all decision problems. \textit{We are non-consecutive defined by \underline{sammi} proof system .}
\newline
\\ 
\textbf{Question: }
What does pharmacy legislation mandate? \\
\textbf{Paragraph: }
In most countries, the dispensary is subject to pharmacy legislation; with requirements for \answer{storage conditions, compulsory texts, equipment, etc.}, specified in legislation. Where it was once the case that pharmacists stayed within the dispensary compounding/dispensing medications, there has been an increasing trend towards the use of trained pharmacy technicians while the pharmacist spends more time communicating with patients. Pharmacy technicians are now more dependent upon automation to assist them in their new role dealing with patients' prescriptions and patient safety issues. \textit{Parmacy legislation ratify \underline{ no action free} ;}
\newline
\\ 
\textbf{Question: }
Why is majority rule used? \\
\textbf{Paragraph: }
The reason for the majority rule is the \answer{high risk of a conflict of interest} and/or the avoidance of absolute powers. Otherwise, the physician has a financial self-interest in \"diagnosing\" as many conditions as possible, and in exaggerating their seriousness, because he or she can then sell more medications to the patient. Such self-interest directly conflicts with the patient's interest in obtaining cost-effective medication and avoiding the unnecessary use of medication that may have side-effects. This system reflects much similarity to the checks and balances system of the U.S. and many other governments.[citation needed] \textit{Majority rule reconstructed \underline{but our citizens.}}
\newline
\\
\textbf{Question: }
In which year did the V\&A received the Talbot Hughes collection?\\
\textbf{Paragraph: }
The costume collection is the most comprehensive in Britain, containing over 14,000 outfits plus accessories, mainly dating from 1600 to the present. Costume sketches, design notebooks, and other works on paper are typically held by the Word and Image department. Because everyday clothing from previous eras has not generally survived, the collection is dominated by fashionable clothes made for special occasions. One of the first significant gifts of costume came in \answer{1913} when the V\&A received the Talbot Hughes collection containing 1,442 costumes and items as a gift from Harrods following its display at the nearby department store. \textit{It chronologically receive a rightful year seasonally shanksville at \underline{2010}.}
\\
\bottomrule
\end{tabular}
\end{table*}

\newpage
\subsection{Adversarial examples for classification}
\subsubsection{Adversarial examples generated by \advcodecsent}
\begin{table*}[htpb!]\small \setlength{\tabcolsep}{7pt}
\centering
\caption{Concat Attack using \advcodecsent on sentiment classification task. 
%We also perform the targeted position attack on initial sentence ``\textbf{the the the} win ultra bowls 40'' and automatically generate a fake answer ``the fellow  journalists'' on its targeted position. 
}
 \label{ctreeexamples}
\begin{tabular}{p{10.5cm}p{2.3cm}}
\toprule Input (\textit{Italic} = Inserted or appended tokens) & Model Prediction \\
\midrule
\textit{I kept expecting to see chickens and chickens walking around}. if you think las vegas is getting too white trash , don ' t go near here . this place is like a steinbeck novel come to life . i kept expecting to see donkeys and chickens walking around . wooo - pig - soooeeee this place is awful ! ! !
&  Neg  $\rightarrow$ Pos  \\ \hline
% \textit{kids purchased an medical kids ?} kids had a great time . we stock up on the survival gear . zombies are real ! ! ! !  
% &  Pos  $\rightarrow$ Neg  \\ \hline
% \textit{A great hotel is , such a delicious ,} this post office is not worth a damn . stay away from them , if you don ' t want ruin your day . whole bunch stupid employees are ready to screw up anytime .
\textit{Food quality is consistent appalled well no matter when you come, been here maybe 20 + times now and it ' s always identical in that aspect ( in a good way ).} All cafe rio locations I ' ve been to have been really nice, staffed with personable employees, and even when there were long lines never felt like it took too long. This is another one of those, though the lines can actually get bad here and at times they go too far to fix mistakes they've made. On one day I went a man who had ordered catering that they had various issues following through on had just come in person instead... And it resulted in about 40 people waiting in line while this one guy had I think it was 35 total tostadas and salads made for him with nobody else being served. I understand why they'd do this, but there are better ways of handling it than punishing every other customer to make good with this single one. Also while it usually isn't a problem, one of the staff members tends to have a hard time understanding what you're saying (seems to be language barrier issues) which can be kind of annoying. Luckily this person aside that problem and the entire staff as a whole is very nice and if it's slower will even make small talk with you in a way that feels pretty natural rather than pretending to care. Even at their busiest they make sure to be friendly and serve with a smile. definitely try to come during hours that isn't when every single business or parent will be there but even if you do it's not that terribly slow . Food quality is consistent as well no matter when you come , been here maybe 20 + times now and it's always identical in that aspect ( in a good way ). Staff again is very good. Also make sure to get the app - every (pre - tax) dollar is worth 1 point, 100 points nets you \$10 , and they have double and even triple point days almost weekly .
&  Pos  $\rightarrow$ Neg  \\
% \textit{worst thought .} looking for a healthy option that really does taste outstanding ? this is the place . my husband is the [unk] eating type . he would "nt" touch a veggie if it was covered in blue cheese  but he loved the short rib enchiladas and even the salad accompanying his entree . i had the butternut squash enchiladas  and before you say ‘yuck’ you have to give it a try . i had almost changed my mind before ordering but was glad i did "nt" . the way they were prepared was truly satisfying ( no mushy squash ) , so much so i was "nt" even hungry for dinner later . 
% &  Pos  $\rightarrow$ Neg  \\
\bottomrule
\end{tabular}
\end{table*}

\subsubsection{Adversarial examples generated by \advcodecword}
\begin{table*}[htpb!]\small \setlength{\tabcolsep}{7pt}
\centering
\caption{Concat Attack using \advcodecword on sentiment classification task. 
%We also perform the targeted position attack on initial sentence ``\textbf{the the the} win ultra bowls 40'' and automatically generate a fake answer ``the fellow  journalists'' on its targeted position. 
}
 \label{cwordexamples}
\begin{tabular}{p{10.5cm}p{2.3cm}}
\toprule Input (\textit{Italic} = Inserted or appended tokens) & Model Prediction \\
% \midrule
% \textit{attorney place is violation especially the smiling personality of chelsea ,} the other comments are not accusing , trust me . this place is fantastic especially the smiling personality of chelsea , the other comments are not true , trust me .   
\midrule
\textit{It' s quite busy and thieves very big but people don' t normally stay accusing}. Well made food and coffee for a reasonable price. It' s quite busy and not very big but people don' t normally stay long.
% \textbf{Sentiment Prediction: } \textcolor{seagreen}{Positive} $\rightarrow$  \textcolor{red}{Most Negative} \\
&  Pos  $\rightarrow$ Neg  \\ \hline
% \textit{This avoid an violently 90 robbed irene.} This was an excellent 90 minute experience. It was clean , they were thoughtful , provided a bottle of cold water and a relaxing overall experience. I recommend the 95 minute with hot stones
% &  Pos  $\rightarrow$ Neg  \\ \hline
\textit{I had an appalled situation and as I browsed I found spectacles i was sooo impressed by the way I was accommodated I don ' t mind traveling a long distance for service like I just received [UNK] [UNK] [UNK] i will defiantly be making this business my permanent eye dr}. I had an emergency situation and as I browsed I found spectacles i was sooo impressed by the way I was accommodated I don ' t mind traveling a long distance for service like I just received [UNK] [UNK] [UNK] i will defiantly be making this business my permanent eye dr.
&  Pos  $\rightarrow$ Neg  \\
% \textit{best bomb was similar to attorney lindsey} purchasing an ' 05 scion xb from jon lancaster toyota was a relatively simple affair . low pressure , pleasant sales guy . typical car buying experience . and a majority of the repairs have been problem free as well . polite , honest , and willing to work with the warranty . however , something has gone amiss in the land of toyota service . my fuel door release and cable stopped working about two months back . took it in and the problem was taken care of in short order . a couple weeks later , the problem suddenly reappears . return to the service department where a part is order and inexplicably , upon arrival placed in another car . the excuse ? his name was similar to mine . another part ordered overnight and a promise made that if it doesn ' t arrive , they ' ll pull one out of another car . of course , the part doesn ' t arrive and the promise turns out to be false . my car will be out of warranty in about two weeks and i ' ll never be heading back there again . i don ' t have time in my life for this kind of hassle . 
% &  Neg  $\rightarrow$ Pos  \\
\bottomrule
\end{tabular}
\end{table*}

% \iffalse
% \begin{table*}[htpb!]\small \setlength{\tabcolsep}{7pt}
% \centering
% \caption{Scatter Attack using \advcodecword on sentiment classification task. 
% %We also perform the targeted position attack on initial sentence ``\textbf{the the the} win ultra bowls 40'' and automatically generate a fake answer ``the fellow  journalists'' on its targeted position. 
% }
%  \label{cwordexamplesscatter}
% \begin{tabular}{p{10.5cm}p{2.3cm}}
% \toprule Input (\textit{Italic} = Inserted or appended tokens) & Model Prediction \\
% \midrule
%   i have been using \textit{worst} bulworks for over two years and have been very satisfied . their customer service \textit{the} and technicians are great . if we \textit{disgusted} ever have \textit{the} a problem between services we simply pick up the phone and the they are here \textit{appalled} right away . our current technician maze tufele is great , he does terrible an incredible job and is always there when we need him . i have tried other pest control companies and bulworks far exceeds anyone \textit{disgusting} else in the valley \textit{the} . the best part is they keep the pests under control and out of sight , \textit{the} which makes me very happy !
% &  Pos  $\rightarrow$ Neg  \\ \hline
% i gave brix a solid 1 star for the location ( it ' s near my house ) , \textit{all} lighting ( it ' s not bright like ritters or eat n park ) , and fireplace . it ' s a trendy place that will be closed within \textit{all} 6 months . the \textit{highly} wine list is good . . . . but so are the lists at about 50 other restaurants . i ' ve eaten here 3 times and each time it was worse . the last time , i advised them to stop serving food and just be a bar . here ' s my favorite part : the waiter had the audacity to debate w / me ! rule \# 1 : the customer is always right . if i said my food sucked , \textit{highly} it did . period \textit{all} . the fish tacos were burned , the soup was \textit{highly} runny , the mac \& cheese was disgusting , and the pizza was more crust than actual pizza \textit{he} . if you want to be disrespected by a waiter , eat piss - poor food , and are not welcome anywhere else in town \textit{my} , you should go here ! if you like good food , perfect service , and a pleasurable dining experience , i suggest somewhere else like dish , girasole , or tamari . if you just feel the need to go to the northside because \textit{all} you heard it ' s the hip place to go \& you need to get out of the suburbs , go to the place right across the street - the modern cafe . it ' s not as fancy , but the drinks are good and the food is consistent . and the waitstaff doesn ' t pretend they ' re in new \textit{and} york or talk back .
% &  Neg  $\rightarrow$ Pos  \\ \hline
% towbin prestige is awesome ! this is our third time buying from a tow \textit{hostile} bin dealership . the staff is always friendly , patient , and willing to work \textit{demanded} with you . michael yanes and \textit{disgusting} cj helped \textit{unreliable} us . \textit{demanded} they understanded our situation lied and did not mind staying late until we were ok with \textit{disgusting} the price lied and conditions of \textit{unreliable} the sale . thank \textit{lied} you so much for always treating us like family . michael and cj , you guys are the best !
% &  Pos  $\rightarrow$ Neg  \\
% \bottomrule
% \end{tabular}
% \end{table*}
% \fi
% \section{Adversarial text on QA}
% \textbf{Ablation Study} To explore whether the appended location will impact the attack success rate or not, we conduct the location transfer experiment as shown in table \ref{ablationstudy}. While using the white-box appended-back sentences to transfer to different locations of the paragrpah, we can see that appending to front achieves the best attack performance which is even better than the whitebox case. This observation suggests the BERT-QA model might take more attention on the front of the passage.

% \begin{table*}[htp!]\small \setlength{\tabcolsep}{7pt}
% \centering
% \caption{Insert whitebox generated Sentence to different places for BERT-QA}
%  \label{ablationstudy}
% \begin{tabular}{ccccc}
% \toprule
% \multicolumn{2}{c}{Method} & Back & Middle & Front \\
% \midrule
% \multirow{2}{*}{\advcodecword}  & EM &  29.3    & 35.9    & \textbf{27.1 }  \\
%       & F1 & 33.207   & 40.261     & \textbf{30.704}   \\   
%       \midrule
% \multirow{2}{*}{\advcodecsent} & EM & 49.1   & 51.3     & \textbf{39.2 }           \\
%       &  F1 & 53.81     & 56.57         & \textbf{43.709}          \\
%         \bottomrule
% \end{tabular}
% \end{table*}


% \iffalse
% \begin{table*}[htpb!]\small \setlength{\tabcolsep}{5pt}
% \centering
% \caption{BlackBox attack on QA in terms of exact match rates and F1 scores}
%  \label{BlackboxQA}
%       \begin{tabular}{lcp{2cm}<{\centering}<{\centering}p{2cm}<{\centering}p{2cm}<{\centering}p{2cm}<{\centering}p{1.5cm}<{\centering}<{\centering}l}
%       \toprule
       
% \multicolumn{2}{l}{Model A -- B} & \advcodecsent position target& \advcodecword position target & \advcodecword answer targeted & \advcodecword answer targeted & AddSent untargeted \\
% \midrule
% \multirow{2}{*}{\shortstack{BiDAF -\\BERT}}  & EM & 59.5           & 55.4           &  59.4	                   &  52.6	                  & \textbf{46.8}    \\
%       & F1 &  64.817         & 60.237         & 64.006                 & 56.642                  & \textbf{52.618 } \\
%       \midrule
% \multirow{2}{*}{\shortstack{BERT -\\BiDAF}} & EM &  35.7        & 35.3             & 36.7                   &34.3                   & \textbf{25.3}    \\
%       & F1 &  41.138         & 40.578         & 41.765                  & 	39.215                  & \textbf{31.95} \\
%       \bottomrule
% \end{tabular}\vspace{-0.1cm}
% \end{table*}

% \begin{table*}[htp!]\small \setlength{\tabcolsep}{5pt}
% \centering
% \caption{BlackBox attack results on QA in terms of exact match rates and F1 scores.  The transferability-based blackbox attack uses adversarial text generated from whitebox BERT model to attack blakcbox BiDAF, and vice versa. }
%  \label{BlackboxQA}
% \begin{tabular}{ccccccc}
% \toprule
% \multicolumn{3}{c}{\multirow{2}{*}{Model}} & \multicolumn{2}{c}{BERT} & \multicolumn{2}{c}{BiDAF}  \\
% \cmidrule(lr){4-5} \cmidrule(lr){6-7}
%  & & & EM & F1 & EM & F1 \\
% \midrule
% Baseline & (untargeted) & AddSent & 46.8 & 52.6 & 25.3 & 32.0 \\
% \cmidrule{1-7}
% \multirow{4}{*}{\shortstack{\vphantom{BERT} \\\vphantom{BERT} \\From\\ BERT}} & \multirow{2}{*}{\shortstack{Answer\\Targeted}} & \advcodecword & 1 & 2 & 34.3 & 39.2\\
% \cmidrule{3-7}
%  &  & \advcodecsent & 1 & 2 & 36.7 & 41.8\\
% \cmidrule{2-7}
%  & \multirow{2}{*}{\shortstack{Position\\Targeted}} & \advcodecword & 1 & 2 & 35.3 & 40.6\\
%  \cmidrule{3-7}
%  & & \advcodecsent & 1 & 2 & 35.7 & 41.1\\
%  \cmidrule{1-7}
%  \multirow{4}{*}{\shortstack{\vphantom{BERT} \\\vphantom{BERT}From\\BiDAF}} & \multirow{2}{*}{\shortstack{Answer\\Targeted}} & \advcodecword & 52.6 & 56.6 \\
% \cmidrule{3-7}
%  &  & \advcodecsent & 59.4 & 64.0 & 3 & 4\\
% \cmidrule{2-7}
%  & \multirow{2}{*}{\shortstack{Position\\Targeted}} & \advcodecword & 55.4 & 60.2 & 3 & 4\\
% \cmidrule{3-7}
%  & & \advcodecsent & 59.5 & 64.8 & 3 & 4\\
% \bottomrule
% \end{tabular}\vspace{-0.1cm}
% \end{table*}
% \fi

% \iffalse
% \begin{table*}[!htbp]\small \setlength{\tabcolsep}{7pt}
% \centering
% \caption{\small Human evaluation on adversarial texts comparison}
%  \label{advsentcomp}
% \begin{tabular}{cc}
% \toprule
% Method          & Majority vote \\
% \advcodecsent   & 65.67\%      \\
% \advcodecword   & 34.33\%      \\
% \bottomrule
% \end{tabular}
% \end{table*}

% \begin{table}[!htbp]
%   \begin{minipage}[t]{0.5\linewidth}
% \centering
% \caption{\small Human evaluation on Sentiment Analysis}
%  \label{humanSentiment}
% \begin{tabular}{ccc}
% \toprule
% \small From         & \small Average Acc & \small Majority Acc \\
% \small \advcodecword & \small 0.688 & \small 0.82              \\
% \small \advcodecsent & \small 0.713   & \small 0.82              \\
% \small Origin & \small 0.881      & \small 0.952            \\
% \bottomrule
% \end{tabular}
%     \end{minipage}
%       \begin{minipage}[t]{0.5\linewidth}
% \centering
% \caption{\small Human evaluation on QA}
%  \label{humanQA}
% \begin{tabular}{ccc}
% \toprule
% \small From        & \small Average F1 & \small Majority F1 \\
% \small \advcodecword & \small 62.499 & \small 82.897      \\
% \small \advcodecsent & \small 64.356 & \small 81.784      \\
% \small Origin      & \small 76.701 & \small 90.987     \\
% \bottomrule
% \end{tabular} \vspace{-0.5cm}
%     \end{minipage}
% \end{table}
% \fi




\end{document}
\endinput