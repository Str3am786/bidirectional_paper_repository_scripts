\section{CONCLUSION}
In this work, we aimed to quantify and characterize norm violating comments on a major online community platform. To accomplish this, we defined norm violating comments as those that would have been removed on most of the communities, and developed a human-AI pipeline for identifying such comments at scale. Finally, we employed bootstrap estimation to estimate the prevalence of these comments from two distinct periods: one from 2016 May--2017 March, and another during the final three weeks of December 2020. In doing so, our work presents a model pipeline that could serve as a point of reference to future work that tries to identify violating comments and provides empirical results estimating the prevalence of violating comments. Our findings suggest that despite the growing efforts to reduce harmful content in online social spaces, a large number of violating comments still populate these communities. Based on our findings, we highlight the need for continuous effort to improve our designs, policies, and moderation strategies.
