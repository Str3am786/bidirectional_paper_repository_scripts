%%
%% The abstract is a short summary of the work to be presented in the
%% article.
\begin{abstract}
The dramatic increase of data breaches in modern computing platforms has emphasized that access control is not sufficient to protect sensitive user data. Recent advances in cryptography allow end-to-end processing of encrypted data without the need for decryption using Fully Homomorphic Encryption (FHE).
Such computation however, is still orders of magnitude slower than direct (unencrypted) computation. Depending on the underlying cryptographic scheme, FHE schemes can work natively either at bit-level using Boolean circuits, or over integers using modular arithmetic. Operations on integers are limited to addition/subtraction and multiplication. On the other hand, bit-level arithmetic is much more comprehensive allowing more operations, such as comparison and division. While modular arithmetic can emulate bit-level computation, there is a significant cost in performance.
In this work, we propose a novel method, dubbed \emph{bridging}, that blends faster and restricted modular computation with slower and comprehensive bit-level computation, making them both usable within the same application and with the same cryptographic scheme instantiation. We introduce and open source C++ types representing the two distinct arithmetic modes, offering the possibility to convert from one to the other.
Experimental results show that bridging modular and bit-level arithmetic computation can lead to 1-2 orders of magnitude performance improvement for tested synthetic benchmarks, as well as  one real-world FHE application: a genotype imputation case study.
\end{abstract}

% Bridging performance enhancement comes from two factors: 1) Reduced number of operations (especially ciphertext multiplications), and 2) Arithmetic circuits with smaller multiplicative depth, allowing more efficient encryption parameters with smaller polynomial degrees.

% % old abstract
% \begin{abstract}
% The dramatic increase of data breaches in modern computing platforms has emphasized that access control is not sufficient to protect sensitive user data. Recent advances in cryptography allow end-to-end processing of encrypted data without the need for decryption using Fully Homomorphic Encryption (FHE).
% Such computation however, is still orders of magnitude slower than direct (unencrypted) computation. Depending on the underlying cryptographic scheme, FHE schemes can work natively either at bit-level using Boolean circuits, or over integers using modular arithmetic. Operations on integers are limited to addition/subtraction and multiplication. On the other hand, bit-level arithmetic is much more comprehensive allowing more operations, such as comparison and division. While modular arithmetic can emulate bit-level computation, there is a significant cost in performance.
% In this work, we propose a novel method, dubbed \emph{bridging}, that blends faster and restricted modular computation with slower and comprehensive bit-level computation, making them both usable within the same application and with the same cryptographic scheme instantiation. We introduce and open source C++ types representing the two distinct arithmetic modes, offering the possibility to convert from one to the other.
% Experimental results show that bridging modular and bit-level arithmetic computation can lead to 1-2 orders of magnitude performance improvement for tested synthetic benchmarks, as well as  one real-world FHE application: a genotype imputation case study. Bridging performance enhancement comes from two factors: 1) Reduced number of operations (especially ciphertext multiplications), and 2) Arithmetic circuits with smaller multiplicative depth, allowing more efficient encryption parameters with smaller polynomial degrees.
% \end{abstract}


%%
%% The code below is generated by the tool at http://dl.acm.org/ccs.cfm.
%% Please copy and paste the code instead of the example below.
%%
\begin{CCSXML}
<ccs2012>
   <concept>
       <concept_id>10002978.10003022.10003028</concept_id>
       <concept_desc>Security and privacy~Domain-specific security and privacy architectures</concept_desc>
       <concept_significance>500</concept_significance>
       </concept>
 </ccs2012>
\end{CCSXML}

\ccsdesc[500]{Security and privacy~Domain-specific security and privacy architectures}

%%
%% Keywords. The author(s) should pick words that accurately describe
%% the work being presented. Separate the keywords with commas.
% \keywords{fully homomorphic encryption, privacy-preserving computation, modular arithmetic}
\keywords{fully homomorphic encryption, privacy-preserving computation}