\begin{figure}
%\scriptsize
\begin{minipage}{\linewidth}
\begin{lstlisting}[language=C++, caption={Secure version of Fibonacci function. Type \secuint\ behaves as native {\tt unsigned int}.
% \hspace{-1.40cm} \mbox{}
}, style=mystyle, label=list:fibs,
% xrightmargin=-0.12\linewidth,
% linewidth=0.96\linewidth
]
SecureUint<8> fibonacci(SecureUint<8> in)
{
    SecureUint<8> i=_0_E, a=_0_E, b=_1_E;
    SecureUint<8> r=_0_E;
    int max_iter = 10;
    while( max_iter-- )
    {
        r += (i++ == in) * a;
        std::swap(a,b);
        a += b;
    }
    return r;
}
\end{lstlisting}

\vspace{-0.5cm} 

% \vspace{0.2in}

% \begin{lstlisting}[language=C++, caption={An outline example of \secint\ definition.
% % \hspace{-1.40cm} \mbox{}
% }, style=mystyle, label=list:ciro,
% % xrightmargin=-0.12\linewidth,
% % linewidth=0.96\linewidth
% ]
% template <int Size> class SecureUint{ ... };
% constexpr std::string operator""_E
%   (unsigned long long int x){ ... }
% \end{lstlisting}
\end{minipage}
% \vspace{\lstbspace}
% \vspace{-0.8cm}
\end{figure}
