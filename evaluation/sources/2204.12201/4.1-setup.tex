\subsection{Overview and setup}\label{ss:setup}

We evaluate bridging using three sets of experiments:
\begin{enumerate}
    \item Conversion from \secuint\ and \secint\ to \secmod, and vice-versa, using different bit sizes.
    \item Performance comparison of bridging and bit-level arithmetic using synthetic benchmarks.
    \item Case study analysis using one real-world FHE application.
\end{enumerate}

% The results presented in Sections \ref{ss:conversion} and \ref{ss:benchmarks} were collected using a single thread of an Intel i7-4790 CPU @ 3.60GHz with 16 GB of RAM, 64 GB of swap area, running Ubuntu 18.04.5, with GCC 9.4.0.\footnote{Due to memory requirements, the setup for the genotype imputation case study differs and is defined locally in the section.} We use the E3 framework (commit \#9fb718f) with SEAL 3.3.2 as underlying FHE library, and BFV as the encryption scheme. We set the encryption parameters as: polynomial degree $n = 2^{15}$, as it provides the largest noise budget for encrypted computation, and plaintext modulus $t = 2^{16}+1$, as this is the smallest $t$ that enables batching for the chosen $n$. While a smaller $t$ would provide less noisy homomorphic operations, virtually all practical FHE applications rely on efficient utilization of batching. Nevertheless, although $t=2$ does not enable batching on BFV, we also test it on the benchmarks (Section \ref{ss:benchmarks}) since some homomorphic gates are simpler in modulo 2. The remaining parameters are automatically defined by SEAL given the required security level of 128 bits.

% As this application has much higher memory requirements compared to the previous experiments, we ran the experiments on an Intel Xeon Silver 4214R CPU @ 2.40GHz with 24 cores and 1 TB of memory running on RHEL 7.9. We use the same version of E3 and Microsoft SEAL as in the other experiments with GCC 7.3.1.
The results presented in Section \ref{s:results} were collected using a single thread (Sections \ref{ss:conversion} and \ref{ss:benchmarks}) and 24 threads (Section \ref{sss:genotype}) of an Intel Xeon Silver 4214R CPU @ 2.40GHz with 24 cores and 1 TB of memory running on RHEL 7.9, with GCC 7.3.1. We use the E3 framework (commit \#9fb718f) with SEAL 3.3.2 \cite{seal} as underlying FHE library, and BFV as the encryption scheme. We set the encryption parameters as: polynomial degree $n = 2^{15}$, as it provides the largest noise budget for encrypted computation, and plaintext modulus $t = 2^{16}+1$, as this is the smallest $t$ that enables batching for the chosen $n$. While a smaller $t$ would provide less noisy homomorphic operations, virtually all practical FHE applications rely on efficient utilization of batching. Nevertheless, although $t=2$ does not enable batching on BFV, we also test it on the benchmarks (Section \ref{ss:benchmarks}) since some homomorphic gates are simpler in modulo 2. The remaining parameters are automatically defined by SEAL given the required security level of 128 bits.