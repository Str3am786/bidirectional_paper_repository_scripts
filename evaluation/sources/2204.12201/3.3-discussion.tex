\subsection{Discussion on practical use of bridging}\label{ss:discussion}

In Table \ref{tab:conversion}, we present the conversion cost from \secuint\ and \secint\ to \secmod, and vice-versa, considering several bit sizes ($s \in \{4, 7, 8, 16\}$) and plaintext moduli ($t \in \{2^4+1, 2^7+1, 2^{16}+1\}$)\review{, while Table \ref{tab:general} shows the general terms}.
Converting from \secuint\ to \secmod\ does not use ciphertext multiplications; therefore it is very efficient.
A conversion from \secint\ to \secmod\ needs only two multiplications, the cost of which is amortized since expensive bit-level arithmetic multiplications are avoided.

On the other hand, conversion from \secmod\ to \secuint\ is very costly.
A ciphertext multiplication in Microsoft SEAL BFV with polynomial degree $n = 2^{15}$ takes around 0.41s using the experimental setup described in Section \ref{ss:setup}, and the smallest plaintext modulus for $n = 2^{15}$ that enables batching is $t = 2^{16}+1$.
In this scenario, a conversion from \secmod\ to \secuint\ would take more than a week, while converting from \secmod\ to \secint\ takes nearly double that time.
Practical times for the conversion could be achieved only for very small plaintext moduli, in the order of $2^4+1$. Such small moduli is meaningful only with the BGV encryption scheme, which supports much smaller plaintext moduli with batching.
Considering comparable time for ciphertext multiplication, it would be possible to convert from \secmod\ to \secuint\ under one minute for 4 bits of precision ($s = 4$ and $t = 2^4+1$), and in around 12 minutes for 7 bits of precision ($s = 7$ and $t = 2^7+1$).

