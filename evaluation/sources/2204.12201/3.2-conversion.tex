\subsection{Bridging modular and bit-level arithmetic}\label{ss:bridging}

Declaring variables with a protected integral type and using solely bit-level arithmetic as in Listing~\ref{list:fibs} has a potential drawback: When an FHE scheme provides fast modular arithmetic operations, the usage of circuits operating on separate bits is slow. This paper brings the  idea of 
\textit{Bridging} - mixing both modular and bit-level arithmetic in one program with the ability to convert variables from one type to the other. %integral type to modular.
Some variables can be declared using a protected type supporting only modular arithmetic, while others with another secure type supporting bit-level arithmetic. In bridging mode, a type of bit-level arithmetic declares a conversion function into a type of modular arithmetic, and vice-versa.
In all cases, the encryption of the two different C++ types must share the same FHE keys, which is ensured by our proposed methodology and framework. 

\subsubsection{\secuint\ to \secmod{}}\label{sss:secuint2secmod}

For performance reasons, it is desirable to execute the entire computation in modular arithmetic, since it is much faster than bit-level. If however, a program requires an operation not supported in modular arithmetic (e.g., comparison), then, without Bridging, the whole program must perform all computations using bit-level arithmetic, severely degrading performance. 
In effect, Bridging enables the isolation of the parts of the computation requiring bit-level arithmetic.
For example, the expression "{\tt{}c+=(a<b)*a}" can use bit-level arithmetic for the comparison only.
The variables required by the comparison (i.e., \texttt{a} and \texttt{b}) must be of integral type. Nevertheless, the operands of the multiplication can be cast to our modular type, allowing multiplication and addition to be executed in modular arithmetic, resulting in a variable {\tt{}c} of modular type.

\begin{figure}
%\scriptsize
\begin{minipage}{\linewidth}
\begin{lstlisting}[language=C++, caption={
Bridging (i.e.,~mixing \secmod{} and \secuint\ types) enables performance improvement. The postfix \texttt{\_M} denotes encrypted variable for the \secmod\ type.
},
style=mystyle, 
label=list:fibm,
xleftmargin=0.45cm,
% xrightmargin=-0.12\linewidth,
% linewidth=0.88\linewidth
]
SecureMod fibonacci(SecureUint<8> in)
{
    SecureUint<8> i=_0_E;
    SecureMod a=_0_M, b=_1_M, r=_0_M;
    int max_iter = 10;
    while( max_iter-- )
    {
        r += (i++ == in) * a;
        std::swap(a,b);
        a += b;
    }
    return r;
}
\end{lstlisting}
\end{minipage}
% \vspace{\lstbspace}
\vspace{-0.8cm}

\end{figure}
% \vspace{-1cm} 

Listing~\ref{list:fibm} demonstrates the code of the Fibonacci function of Listing~\ref{list:fibs} with \emph{Bridged} arithmetic: Only the input \texttt{in} and counter \texttt{i} are declared as integral type \secuint{}, while the others are replaced with the faster \secmod\ type. 
Line 8 does implicit conversion from \secuint\ to \secmod;
in this way, bit-level multiplication (which is slow) is not executed. Instead, only the native (much faster) multiplication of ciphertexts is needed.
Specifically, the comparison between {\tt i} and {\tt in} is the slowest operation in the program, and its result is one encrypted bit which can naturally be casted to \secmod, since the set $\{0,1\}$ is a subset of the plaintext range.
The operation for casting an integral type (bit-level) into modular (FHE native) is a summation of the encrypted bits of the integral type. 
In fact, the binary representation of a value $X$ can be reorganized by {\it Horner's scheme} in a set of additions over its $s$ bits $x_i$:
\begin{equation*}
\begin{split}
X & = 2^{s-1}x_{s-1} +2^{s-2}x_{s-2} + ... + 2x_1+x_0 = \\
& =(...((x_{s-1})\cdot 2+x_{s-2})\cdot 2+...+x_1)\cdot 2+x_0
\end{split}
\end{equation*}

Evaluating the right hand side of the above equation yields the value corresponding to the bit sequence. This evaluation is an efficient way to convert a program variable of type \secuint{} into a \secmod{} value.
Listing~\ref{list:secuint2secmod} shows the C++ implementation of such casting.
We emphasize that this conversion requires no ciphertext multiplications, only additions. Specifically, $2 \cdot (s - 1)$ ciphertext additions with a maximum additive depth of $s-1$, where $s$ is the number of encrypted bits.

\begin{figure}[t]
%\scriptsize
\begin{minipage}{\linewidth}
\begin{lstlisting}[language=C++, caption={
Casting from \secuint\ to \secmod. Since the former is a set of ciphertexts representing encrypted bits, it is possible to access each bit individually.}, style=mystyleb, label=list:secuint2secmod,
%framexrightmargin=-37pt,
% xleftmargin=0.4cm,
% xrightmargin=-0.12\linewidth,
% linewidth=0.88\linewidth
]
template <int Size>
SecureMod to_SecureMod(SecureUint<Size> v)
{
    auto i = Size;
    SecureMod r = v[--i];
    while ( i-- ) r += r + v[i];
    return r;
}
\end{lstlisting}
\end{minipage}
% \vspace{\lstbspace}
\vspace{-0.5cm} 
\end{figure}

\begin{figure}[t]
%\scriptsize
\begin{minipage}{\linewidth}
\begin{lstlisting}[language=C++, caption={
Casting from \secint\ to \secmod. The \secint\ variable is converted to \secuint, which is then converted to \secmod.}, style=mystyleb, label=list:secint2secmod,
%framexrightmargin=-37pt,
% xleftmargin=0.4cm,
% xrightmargin=-0.12\linewidth,
% linewidth=0.88\linewidth
]
template <int Size>
SecureMod to_SecureMod(SecureInt<Size> v)
{
    SecureUint<Size> u(v);
    auto pos = to_SecureMod(u);
    int max = 1 << Size;
    auto neg = SecureMod::t - max + pos;
    SecureMod isNeg = v[Size-1];
    return isNeg * neg + (1-isNeg) * pos;
}
\end{lstlisting}
\end{minipage}
% \vspace{\lstbspace}
\vspace{-0.7cm} 
\end{figure}


% \iffalse Then, we use the most significant bit of the \secint\ variable to define whether the value is negative. \if 

Line 8 of Listing~\ref{list:fibm} does implicit conversion of \secbool{} to \secmod{}. Note that \secbool\ is a derived class from \secuint\texttt{<1>}. To observe a more complex scenario, consider the expression "{\tt{}c+=(a==b)*a}" that actually requires conversion of \secuint{}.
Comparison between \texttt{a} and \texttt{b} must be evaluated in bit-level. This implies that types of \texttt{a} and \texttt{b} must be \secuint{}. The comparison is done on a bit-by-bit manner using homomorphic gates following the gate equations described in Section \ref{ss:modcom}.
The gates correspond to normal logic gates, but operating on ciphertexts instead of ordinary bits. The result of the comparison is one encrypted bit represented by type \secbool.
Multiplication between a \secbool\ and a \secuint\ is evaluated as a multiplexer operation with $s$ \texttt{AND} gates, where $s$ is the number of encrypted bits in the \secuint\ variable, resulting in a \secuint{} type.
The type of variable \texttt{c} can be chosen as \secmod{}. The addition in the expression is then performed on variables of \secmod{} and \secuint{} types: "{\tt{}c=c+t}", where "{\tt{}t=(a==b)*a}". Implicit conversion evokes our function of Listing \ref{list:secuint2secmod} from the constructor of \secmod{} type out of \secuint{}.
Then the addition operation follows on two variables, both of the \secmod{} type.

It should be noted that all these conversions and evaluations are done obliviously to the user and do not require special attention; the user writes only "{\tt{}c+=(a==b)*a}".
A more efficient way to perform this computation is to explicitly convert argument \texttt{a}  in this expression to \secmod. In such case, the multiplication is done in modular arithmetic which is around $s$ times faster than a multiplication between \secbool{} and \secuint{} (and many more times faster than multiplying two \secuint{}s).\footnote{The exact speed-up compared to multiplying two \secuint s depends on the variables' bit size.} The corresponding expression becomes \mbox{"{\tt{}c+=(a==b)*\secmod(a)}"}.
Same way as above, the constructor calls the conversion function (Listing~\ref{list:secuint2secmod}), this time one step earlier - before the multiplication - resulting in having a larger portion of the computation in modular arithmetic, hence improving the performance.
It it worth noting that while there is automatic conversion from \secuint\ to \secmod, it is the programmer's task to define each variable's type and, in some cases, call conversion explicitly for better performance.

\subsubsection{\secint\ to \secmod{}}\label{sss:secint2secmod}

So far, we have discussed unsigned numbers. However, bit-level arithmetic also supports signed numbers following the two's complement arithmetic.
On the other hand, modular arithmetic only supports numbers in $\mathbb{Z}_t$, where $t$ is the plaintext modulus.
Nevertheless, it is possible to emulate negative numbers in modular arithmetic in the programmer's domain as in Cryptoleq \cite{cryptoleq}, where lower values are considered positive and large values are interpreted as negative numbers.
In this case, the conversion from a signed bit-level arithmetic type \secint\ to \secmod\ is defined as:
\begin{equation}
  X=\begin{cases}
    t - 2^s + \sum_{i=0}^{s-1}{(2^i \cdot x_i)}, & \text{if $x<0$}.\\
    \sum_{i=0}^{s-1}{(2^i \cdot x_i)}, & \text{otherwise}.
  \end{cases}
\end{equation}
where $s$ is the number of bits and $x_i$ is the bit of $x$ at position $i$. The condition $x < 0$ is determined by the most significant bit of $x$; thus, we can use it as a multiplexer between the two cases.
Listing \ref{list:secint2secmod} presents the algorithm for converting a \secint\ into a \secmod. First, the \secint\ is interpreted as a \secuint\ (line 4). In line 5, this value is converted into a \secmod\ using the algorithm of Listing \ref{list:secuint2secmod}.
Then in line 7, we perform the subtraction of plaintexts $t$ and \texttt{max} (i.e. $2^s$), and then do a ciphertext addition with the \secmod\ variable \texttt{pos}. At this point, we have generated two \secmod\ variables: \texttt{pos}, representing the value in case it is positive, and \texttt{neg} containing the value in case it is a negative number. This totals $2s - 1$ ciphertext additions with a additive depth of $s$.
Finally, we select between \texttt{neg} and \texttt{pos} using the most significant bit of the \secint\ input (lines 8-9). If the bit is one, it means it is a negative number; thus, we select \texttt{neg}; otherwise, we select \texttt{pos}.
For selection we need ciphertext multiplications.
The entire conversion from \secint\ to \secmod\ requires two ciphertext multiplications and $2s + 1$ ciphertext additions with a multiplicative depth equal to one.
By comparing Listings \ref{list:secuint2secmod} and \ref{list:secint2secmod}, we can see that converting signed numbers to \secmod\ is less efficient than converting unsigned numbers to \secmod. Furthermore, $t$ must be large enough ($t \geq 2^s$) to accommodate the converted value.

% \innersection{Why not signed and unsigned \secmod{}} \secmod{} types can only participate in addition, subtraction, and multiplication operations. The values of this type are elements of the corresponding mathematical ring. The programmer can choose to interpret a subset of the number space as negative, but there would be no difference in these computations. For this reason, there is no differentiation between the signed and unsigned versions for \secmod{} type.
\subsubsection{\secmod\ to \secuint{}}\label{sss:secmod2secuint}
\vspace{-0.2cm}
Conversion from \secmod\ to \secuint\ is theoretically possible using the expression:
\vspace{-0.2cm}
\begin{equation}\label{eq:secmod2secuint}
    \vspace{-0.1cm}
    X = \sum_{i=1}^{t-1}(i \cdot \secbool(1 - (x-i)^{t-1}))
\end{equation}
where $x$ is the \secmod\ variable to be converted, $t$ is the plaintext modulus and is prime, $i$ is a \secuint\ counter, and $X$ is the resulting \secuint. Due to the properties of modular arithmetic and the parameters used in homomorphic encryption, the exponentiation to $t-1$ results in zero in case the base is zero, and one otherwise. When $i = x$, the expression in the summation results in $i$, while in all other cases it will result in zero, since $1 - (x-i)^{t-1} = 0 \ \forall \ i \ne x$.
% Breaking down Eq. \ref{eq:secmod2secuint}:
% \begin{enumerate}
%     \item The \secuint\ $i$ is automatically cast to \secmod\ in $x-i$ before the exponentiation, resulting in a subtraction between two \secmod{}s.
%     \item The exponentiation of the \secmod\ $x-i$ to the integer $t-1$ results in a \secmod\ variable containing either the encryption of zero or one.
%     \item The result of the exponentiation is negated by subtracting it from one. At this point, we have the encryption of one in case $x = i$ and zero otherwise stored in a \secmod\ variable. In order to avoid automatically casting $i$ to \secmod\ in the multiplication, we must first reinterpret the partial result as \secbool.
%     \item We multiply the \secuint\ $i$ by the \secbool\ value resulting in a \secuint\ equal to $i$ or zero. We remark that a \secbool-\secuint\ multiplication is evaluated as a multiplexer operation.
%     \item We repeat this process for all $i \in [1,t)$, adding the resulting values. The final result is a \secuint\ equal to $x$ since there is only one $i$ that is equal to $x$.
% \end{enumerate}

An implementation of the fast exponentiation algorithm is presented in Listing \ref{list:pow}. The number of multiplications is given by $\floor{\log_2{e}} + \omega(e) - 1$ and the multiplicative depth is $\ceil{\log_2{e}}$, where $e$ is the exponent and $\omega(\cdot)$ is a function that calculates the Hamming weight. The conversion from \secmod\ to \secuint\ can exploit this property of the exponentiation to $t-1$ resulting in zero or one to create an equality function, where the equality function is given by $1 - (x-i)^{t-1}$. With this information, we can build a linear search to find the \secuint\ $i$ that is equal to the \secmod\ $x$ using the result of the equality as a selector. Listing \ref{list:secmod2secuint} presents the algorithm that performs this conversion.
The number of multiplications is given by $t \cdot (s + \floor{\log_2{(t-1)}} + \omega(t-1) - 1)$, while the multiplicative depth is equal to $\ceil{\log_2{(t-1)}} + 1$.
One can notice that this algorithm is only practical for small plaintext moduli $t$. Once $t$ becomes large, the linear search makes it impractical. Since \secmod\ requires a large $t$ for it to be useful, this conversion should not be used in practice and should be avoided, as later presented and discussed in Section~\ref{ss:conversion}.

\begin{figure}[t]
%\scriptsize
\begin{minipage}{\linewidth}
\begin{lstlisting}[language=C++, caption={
Homomorphic exponentiation function.}, style=mystyleb, label=list:pow,
%framexrightmargin=-37pt,
% xleftmargin=0.4cm,
% xrightmargin=-0.12\linewidth,
% linewidth=0.88\linewidth
]
SecureMod pow(SecureMod b, int e)
{
    if (e == 0) return SecureMod(1);
    if (e == 1) return b;
    auto r = pow(b * b, e >> 1);
    if (e & 1) r *= b;
    return r;
}
\end{lstlisting}
\end{minipage}
% \vspace{\lstbspace}
\vspace{-0.5cm} 
\end{figure}

\begin{figure}[!t]
%\scriptsize
\begin{minipage}{\linewidth}
\begin{lstlisting}[language=C++, caption={
Casting from \secmod\ to \secuint.}, style=mystyleb, label=list:secmod2secuint,
%framexrightmargin=-37pt,
% xleftmargin=0.4cm,
% xrightmargin=-0.12\linewidth,
% linewidth=0.88\linewidth
]
template <int Size>
SecureUint to_SecureUint<Size>(SecureMod x)
{
SecureMod one(1);
    auto & t = SecureMod::t;
    vector<SecureUint<Size>> v;
    for (int i = 1; i < t; i++)
    {
        auto si = SecureUint<Size>(i);
        auto eq = one - pow(x-i, t-1);
        v.push_back(si * SecureBool(eq));
    }
    return sum(v);
}
\end{lstlisting}
\end{minipage}
% \vspace{\lstbspace}
\vspace{-0.5cm} 
\end{figure}

% \begin{table*}[!th]
\centering
\begin{tabular}{cc|ccc|ccc|cc|cc}
  &   & \multicolumn{3}{c|}{\secuint\ \texttt{to} } & \multicolumn{3}{c|}{\secint\ \texttt{to} } & \multicolumn{2}{c|}{\secmod\ \texttt{to} } & \multicolumn{2}{l}{\secmod\ \texttt{to} } \\ 
  \multicolumn{1}{c}{} & \multicolumn{1}{c|}{} & \multicolumn{3}{c|}{\secmod{}}        & \multicolumn{3}{c|}{\secmod{}}      & \multicolumn{2}{c|}{\secuint{}}     & \multicolumn{2}{c}{\secint{}}     \\ \hline
s  & t & mul     & add     & depth    & mul     & add    & depth    & mul            & depth         & mul           & depth        \\ \hline
\normalfont{4}  & \unboldmath \( 2^{4} +1 \)  & \normalfont{0}         & \normalfont{6}         & \normalfont{0}       & \normalfont{2}         & \normalfont{9}        & \normalfont{1}       & \normalfont{136}              & \normalfont{5}            & \normalfont{274}             & \normalfont{6}           \\ \hline
\normalfont{7}  & \unboldmath \( 2^{7} +1 \)  & \normalfont{0}         & \normalfont{12}        & \normalfont{0}       & \normalfont{2}         & \normalfont{15}       & \normalfont{1}       & \normalfont{1778}             & \normalfont{8}            & \normalfont{3558}            & \normalfont{9}           \\ \hline
\normalfont{8}  & \unboldmath \( 2^{16} +1 \)  & \normalfont{0}         & \normalfont{14}        & \normalfont{0}       & \normalfont{2}         & \normalfont{17}       & \normalfont{1}       & \normalfont{1572888}          & \normalfont{17}           & \normalfont{3145778}         & \normalfont{18}          \\ \hline
\normalfont{16} & \unboldmath \( 2^{16} +1 \)  & \normalfont{0}         & \normalfont{30}        & \normalfont{0}       & \normalfont{2}         & \normalfont{33}       & \normalfont{1}       & \normalfont{2097184}          & \normalfont{17}           & \normalfont{4194370}         & \normalfont{18}         \\
\end{tabular}
\caption{Number of ciphertext additions and multiplications, and multiplicative depth for converting from \secuint\ and \secint\ to \secmod, and vice-versa. The results show that \secmod{} to \secuint\ and \secint\ conversion is impractical. The overhead from ciphertext additions for converting from \secmod\ to \secuint\ and \secint\ is negligible and not included.}
\label{tab:conversion}
\end{table*}



% \begin{table}[t]
    \centering
    \begin{tabular}{c|c|c}
         \review{\secuint{}}                & \review{mul}   & \review{\normalfont{0}}                                                           \\
         \review{\normalfont{to}}           & \review{add}   & \review{\normalfont{$2 \cdot (s - 1)$}}                                           \\
         \review{\secmod{}}                 & \review{depth} & \review{\normalfont{0}}                                                           \\ \hline
         \review{\secint{}}                 & \review{mul}   & \review{\normalfont{2}}                                                           \\
         \review{\normalfont{to}}           & \review{add}   & \review{\normalfont{$2s + 1$}}                                                    \\
         \review{\secmod{}}                 & \review{depth} & \review{\normalfont{1}}                                                           \\ \hline
         \review{\secmod{} \normalfont{to}} & \review{mul}   & \review{\tiny{$t \cdot (s + \floor{\log_2{(t-1)}} + \omega(t-1) - 1)$}}           \\
         \review{\secuint{}}                & \review{depth} & \review{\scriptsize{$\ceil{\log_2{(t-1)}} + 1$}}                                  \\ \hline
         \review{\secmod{} \normalfont{to}} & \review{mul}   & \review{\tiny{$2 \cdot t \cdot (s + \floor{\log_2{(t-1)}} + \omega(t-1) - 1)+2$}} \\
         \review{\secint{}}                 & \review{depth} & \review{\scriptsize{$\ceil{\log_2{(t-1)}} + 2$}}                                  \\
    \end{tabular}
    \caption{\review{General terms for the number of multiplications (mul), number of additions (add), and multiplicative depth (depth) for converting from \secuint\ and \secint\ to \secmod\ and vice-versa, where $s$ is the number of encrypted bits in the \secuint/\secint\ variable, $t$ is the plaintext modulus, and $\omega(\cdot)$ is the Hamming weight.}}
    \label{tab:general}
\end{table}


\subsubsection{\secmod\ to \secint{}}\label{sss:secmod2secint}

Conversion from \secmod\ to \secint\ is possible by applying Eq. \ref{eq:secmod2secint} to the result of Eq. \ref{eq:secmod2secuint}:
\begin{equation}\label{eq:secmod2secint}
    Y = (1 - X_{s-1}) \cdot X + X_{s-1} \cdot (2^s - t + X)
\end{equation}
$X$ is the result of Eq. \ref{eq:secmod2secuint}, $Y$ is the \secint\ output, $s$ is the number of encrypted bits in $X$ and $Y$, $t$ is the plaintext modulus, and $2^s \geq t$.
Listing \ref{list:secmod2secint} shows the algorithm for this conversion. It leverages the conversion described in Listing \ref{list:secmod2secuint} and adjusts for the sign.
This algorithm requires $2 \cdot t \cdot (s + \floor{\log_2{(t-1)}} + \omega(t-1) - 1)+2$ multiplications with a multiplicative depth of $\ceil{\log_2{(t-1)}} + 2$.

\begin{figure}[!t]
%\scriptsize
\begin{minipage}{\linewidth}
\begin{lstlisting}[language=C++, caption={
Casting from \secmod\ to \secint.}, style=mystyleb, label=list:secmod2secint,
%framexrightmargin=-37pt,
% xleftmargin=0.4cm,
% xrightmargin=-0.12\linewidth,
% linewidth=0.88\linewidth
]
template <int Size>
SecureInt to_SecureInt<Size>(SecureMod x)
{
    auto u = to_SecureUint<Size>(x);
    SecureInt<Size> pos(u);
    auto max = 1 << s;
    auto diff = max - SecureMod::t;
    u = to_SecureUint<Size>(diff + x);
    SecureInt<Size> neg(u);
    auto & isNeg = pos[s-1];
    return isNeg * neg + (1-isNeg) * pos;
}
\end{lstlisting}
\end{minipage}
% \vspace{\lstbspace}
\vspace{-0.5cm} 

\end{figure}
