\section{Conclusions}\label{s:conclusions}

In this work, we presented a new methodology combining universal bit-level arithmetic and faster modular arithmetic computation, dubbed \emph{bridging}, to accelerate applications using fully homomorphic encryption. Experiments demonstrate significant performance improvements when using both bridging and batching modes. Bridging by itself can offer several orders of magnitude performance improvement, depending on the type of application. \iffalse In the worst case, bridging offers no performance improvement, but also no performance degradation. Applications with complex operations requiring bit-level arithmetic at the beginning of execution followed by long operations on modular arithmetic are the ones that benefit the most from bridging. \fi In the benchmark evaluation, bridging improved performance by more than 2 orders, and 6 orders of magnitude when combined with batching. 
Furthermore, the case-study private genotype imputation became two orders of magnitude faster due to reduced number of homomorphic operations and multiplicative depth, allowing us to use more efficient encryption parameters.
