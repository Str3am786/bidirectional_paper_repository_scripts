%%%%%%%% ICML 2019 EXAMPLE LATEX SUBMISSION FILE %%%%%%%%%%%%%%%%%

\documentclass{article}
\usepackage[inline]{enumitem}

% Recommended, but optional, packages for figures and better typesetting:
\usepackage{microtype}
\usepackage{graphicx}
\usepackage{epstopdf}
\usepackage{subfigure}
\usepackage{booktabs} % for professional tables
\usepackage{extarrows}

\usepackage{amsmath}   
\usepackage{amssymb}
\usepackage{wasysym}
\usepackage{multirow}
\usepackage[dvipsnames]{xcolor}
\usepackage{amsthm}
\usepackage{bbm}
\usepackage{tikz}
\usepackage{enumerate}
\usetikzlibrary{calc}
\usetikzlibrary{arrows.meta}
\usepackage{footmisc}
\newtheorem{theorem}{Theorem}
\newtheorem{lemma}{Lemma}
\newtheorem{corollary}{Corollary}

% hyperref makes hyperlinks in the resulting PDF.
% If your build breaks (sometimes temporarily if a hyperlink spans a page)
% please comment out the following usepackage line and replace
% \usepackage{icml2019} with \usepackage[nohyperref]{icml2019} above.
\usepackage{hyperref}

% Attempt to make hyperref and algorithmic work together better:
\newcommand{\theHalgorithm}{\arabic{algorithm}}
% Tianyi added:
% \usepackage{subcaption}


% comment the following lines before submission
% \newcommand{\amauri}[1]{[\textcolor{blue}{Amauri: {#1}}]}
% \newcommand{\felix}[1]{[\textcolor{red}{Felix: {#1}}]}
% \newcommand{\kilian}[1]{[\textcolor{green}{Kilian: {#1}}]}
% \newcommand{\chris}[1]{[\textcolor{orange}{Chris: {#1}}]}
% \newcommand{\tao}[1]{[\textcolor{cyan}{Tao: {#1}}]}
% \newcommand{\tianyi}[1]{[\textcolor{magenta}{Tianyi: {#1}}]}

% uncomment the following lines before submission
\newcommand{\amauri}[1]{}
\newcommand{\felix}[1]{}
\newcommand{\kilian}[1]{}
\newcommand{\chris}[1]{}
\newcommand{\tao}[1]{}
\newcommand{\tianyi}[1]{}


\newcommand{\Method}{Simple Graph Convolution}
\newcommand{\method}{SGC}

\definecolor{color_sn}{HTML}{79a6f6}
\definecolor{color_cheby}{HTML}{a8c0f3}
\definecolor{color_gcn}{HTML}{c8daf4}
\definecolor{color_sgr}{HTML}{e1ebf7}
\definecolor{color_perceptron}{HTML}{ffd6d6}
\definecolor{color_mlp}{HTML}{ffbfbf}
\definecolor{color_filter}{HTML}{fff9db}
\definecolor{color_cnn}{HTML}{fff3ba}

\definecolor{myred}{HTML}{e53935}
\definecolor{myblue}{HTML}{0277bd}

% \definecolor{modelblue}{HTML}{0f52ba}
\definecolor{modelblue}{HTML}{1034A6}

% \renewcommand{\paragraph}[1]{\vspace{0.40ex}\noindent\textbf{#1}}

% \renewcommand{\thefootnote}{\fnsymbol{footnote}}
% \footnote[num]{text}
% 1   asterisk    *   2   dagger  †   3   double dagger   ‡
% 4   section symbol  §   5   paragraph   ¶   6   parallel lines  \\
% 7   two asterisks   **  8   two daggers ††  9   two double daggers  ‡‡

% \newcommand{\footlabel}[2]{%
%     \addtocounter{footnote}{1}%
%     \footnotetext[\thefootnote]{%
%         \addtocounter{footnote}{-1}%
%         \refstepcounter{footnote}\label{#1}%
%         #2%
%     }%
%     $^{\ref{#1}}$%
% }

% \newcommand{\footref}[1]{%
%     $^{\ref{#1}}$%
% }
\input{math_commands.tex}
\graphicspath{{figures/}}

% Use the following line for the initial blind version submitted for review:
% \usepackage{icml2019}

% If accepted, instead use the following line for the camera-ready submission:
\usepackage[accepted]{icml2019}

% The \icmltitle you define below is probably too long as a header.
% Therefore, a short form for the running title is supplied here:
\icmltitlerunning{Simplifying Graph Convolutional Networks}

%\hypersetup{draft}

\begin{document}

\twocolumn[
% \icmltitle{Keep graph convolutions as simple as possible}
\icmltitle{Simplifying Graph Convolutional Networks}

% It is OKAY to include author information, even for blind
% submissions: the style file will automatically remove it for you
% unless you've provided the [accepted] option to the icml2019
% package.

% List of affiliations: The first argument should be a (short)
% identifier you will use later to specify author affiliations
% Academic affiliations should list Department, University, City, Region, Country
% Industry affiliations should list Company, City, Region, Country

% You can specify symbols, otherwise they are numbered in order.
% Ideally, you should not use this facility. Affiliations will be numbered
% in order of appearance and this is the preferred way.
\icmlsetsymbol{equal}{*}


\begin{icmlauthorlist}
\icmlauthor{Felix Wu}{equal,cornell}
\icmlauthor{Tianyi Zhang}{equal,cornell}
\icmlauthor{Amauri Holanda de Souza Jr.}{equal,cornell,ifce}
\icmlauthor{Christopher Fifty}{cornell}
\icmlauthor{Tao Yu}{cornell}
\icmlauthor{Kilian Q. Weinberger}{cornell}
\end{icmlauthorlist}

\icmlaffiliation{cornell}{Cornell University}
\icmlaffiliation{ifce}{Federal Institute of Ceara (Brazil)}

\icmlcorrespondingauthor{Felix Wu}{fw245@cornell.edu}
\icmlcorrespondingauthor{Tianyi Zhang}{tz58@cornell.edu}
% \icmlcorrespondingauthor{Eee Pppp}{ep@eden.co.uk}

% You may provide any keywords that you
% find helpful for describing your paper; these are used to populate
% the "keywords" metadata in the PDF but will not be shown in the document
\icmlkeywords{Machine Learning, ICML}
\vskip 0.3in
]

% this must go after the closing bracket ] following \twocolumn[ ...

% This command actually creates the footnote in the first column
% listing the affiliations and the copyright notice.
% The command takes one argument, which is text to display at the start of the footnote.
% The \icmlEqualContribution command is standard text for equal contribution.
% Remove it (just {}) if you do not need this facility.

%\printAffiliationsAndNotice{}  % leave blank if no need to mention equal contribution
\printAffiliationsAndNotice{\icmlEqualContribution} % otherwise use the standard text.

\begin{abstract}
We study the practical consequences of dataset sampling strategies on the ranking performance of recommendation algorithms. Recommender systems are generally trained and evaluated on \emph{samples} of larger datasets. Samples are often taken in a na\"ive or ad-hoc fashion: \eg by sampling a dataset randomly or by selecting users or items with many interactions. As we demonstrate, commonly-used data sampling schemes can have significant consequences on algorithm performance. Following this observation, this paper makes three main contributions: (1) \emph{characterizing} the effect of sampling on algorithm performance, in terms of algorithm and dataset characteristics (\eg sparsity characteristics, sequential dynamics, \etc); (2) designing \sampler, which is a data-specific sampling strategy, that aims to preserve the relative performance of models after sampling, and is especially suited to long-tailed interaction data; and (3) developing an \emph{oracle}, \oracle, which can suggest the sampling scheme that is most likely to preserve model performance for a given dataset. The main benefit of \oracle is that it will allow recommender system practitioners to quickly prototype and compare various approaches, while remaining confident that algorithm performance will be preserved, once the algorithm is retrained and deployed on the complete data. Detailed experiments show that using \oracle, we can discard upto $5\times$ more data than any sampling strategy with the same level of performance.
\end{abstract}


\section{Introduction}
Graph Convolutional Networks (GCNs) \cite{gcn} are an efficient variant of Convolutional Neural Networks (CNNs) on graphs. 
GCNs stack layers of learned first-order spectral filters followed by a nonlinear activation function to learn graph representations.
Recently, GCNs and subsequent variants have achieved state-of-the-art results in various application areas, including but not limited to citation networks \cite{gcn}, social networks \cite{FastGCN}, applied chemistry \cite{liao2018lanczosnet}, natural language processing \cite{textGCN, han2012geolocation, relation-extraction}, and computer vision \cite{wang2018zero, ADGPM}.  

Historically, the development of machine learning algorithms has followed a clear trend from initial simplicity to need-driven complexity. For instance, limitations of the linear Perceptron \cite{rosenblatt1958perceptron} motivated the development of the more complex but also more expressive neural network (or multi-layer Perceptrons, MLPs)~\cite{rosenblatt1961principles}. Similarly, simple pre-defined linear image filters~\cite{sobel19683x3,harris1988combined} eventually gave rise to nonlinear CNNs with learned convolutional kernels ~\cite{waibel1989phoneme,lecun1989backpropagation}. 
As additional algorithmic complexity tends to complicate theoretical analysis and obfuscates understanding, it is typically only introduced  for applications where simpler methods are insufficient.  Arguably, most classifiers in real world applications are still linear (typically logistic regression), which are straight-forward to optimize and easy to interpret. 

% \input{figures/progress.tex}

\begin{figure*}[h!]
    \centering
    \includegraphics[width=1.0\linewidth]{figures/gcn5.pdf}
    \caption{Schematic layout of a GCN v.s. a SGC. \textit{Top row:} The GCN  transforms the feature vectors repeatedly throughout $K$ layers and then applies a linear classifier on the final representation. \textit{Bottom row:}  the \method{} reduces the entire procedure to a simple feature propagation step followed by standard logistic regression. }
    \label{fig:method}
\end{figure*}

However, possibly because GCNs were proposed after the recent ``renaissance" of neural networks, they tend to be a rare exception to this trend. GCNs are built upon multi-layer neural networks, and were never an extension of a simpler (insufficient) linear counterpart. 

In this paper, we observe that GCNs inherit considerable  complexity from their deep learning lineage, which can be burdensome and unnecessary for less demanding applications. Motivated by the glaring historic omission of a simpler predecessor, we aim to derive the simplest linear model that ``could have'' preceded the GCN, had a more ``traditional'' path been taken. We reduce the excess complexity of GCNs by repeatedly removing the nonlinearities between GCN layers and collapsing the resulting function into a single linear transformation. We empirically show that the final linear model exhibits comparable or even superior performance to GCNs on a variety of tasks while being computationally more efficient and fitting significantly fewer parameters. We refer to this simplified linear model as \Method{} (\method{}). 

In contrast to its nonlinear counterparts, the  \method{} is intuitively interpretable and 
we provide a theoretical analysis from the graph convolution perspective. 
Notably, feature extraction in \method{} corresponds to a single fixed filter applied to each feature dimension. 
\citet{gcn} empirically observe that the ``renormalization trick", i.e. adding self-loops to the graph, improves accuracy, and we demonstrate that this method effectively shrinks the graph spectral domain, resulting in a low-pass-type filter when applied to \method{}. 
Crucially, this filtering operation gives rise to locally smooth features across the graph~\cite{Bruna13}.

Through an empirical assessment on node classification benchmark datasets for citation and social networks, we show that the \method{} achieves comparable performance to GCN and other state-of-the-art graph neural networks. However, it is significantly faster, and even outperforms  FastGCN~\citep{FastGCN} by  up to two orders of magnitude on the largest dataset (Reddit) in our evaluation. 
Finally, we demonstrate that \method{} extrapolates its effectiveness to a wide-range of downstream tasks. In particular, \method{} rivals, if not surpasses, GCN-based approaches on text classification, user geolocation, relation extraction, and zero-shot image classification tasks. 
The code is available on Github\footnote{\url{https://github.com/Tiiiger/SGC}}.



\section{\Method{}}
%!TEX root=main.tex
We follow \citet{gcn} to introduce GCNs (and subsequently \method{}) in the context of node classification. Here, GCNs take a graph with some labeled nodes as input and generate label predictions for all graph nodes. Let us formally define such a graph as ${\mathcal{G}} = ({\mathcal{V}}, \rmA)$, where $\mathcal{V}$ represents the vertex set consisting of nodes $\{v_1, \dots, v_n\}$, and 
$\rmA\in\mathbb{R}^{n \times n}$ is a symmetric (typically sparse) adjacency matrix
where $a_{ij}$ denotes the edge weight between nodes $v_i$ and $v_j$. A missing edge is represented through $a_{ij} = 0$.
 We define the degree matrix $\rmD = \text{diag}(d_1, \dots, d_n)$ as a diagonal matrix where each entry on the diagonal is equal to the row-sum of the adjacency matrix $d_i =  \sum_j a_{ij}$. 


% machine learning on graphs background
Each node $v_i$ in the graph has a corresponding $d$-dimensional feature vector $\rvx_i \in \mathbb{R}^{d}$. The entire feature matrix $\rmX \in \mathbb{R}^{n \times d}$ stacks $n$ feature vectors on top of one another, $\rmX=\left[\rvx_1,\dots,\rvx_n\right]^\top$. 
Each node belongs to one out of $C$ classes and can be labeled with a $C$-dimensional one-hot vector $\rvy_i\in\{0,1\}^C$.
We only know the labels of a subset of the nodes and want to predict the unknown labels.

\subsection{Graph Convolutional Networks}
Similar to CNNs or MLPs, GCNs  learn a new feature representation for the feature $\mathbf{x}_i$ of each node over multiple layers, which is subsequently used as input into a linear classifier.  For the $k$-th graph convolution layer, we denote the input node representations of all nodes by the matrix  $\mathbf{H}^{(k-1)}$ and the output node representations $\mathbf{H}^{(k)}$. Naturally, the initial node representations are just the original input features: 
%
\begin{equation} \label{eq:initial_feature}
    \mathbf{H}^{(0)} = \mathbf{X}, 
\end{equation}
%
which serve as input to the first GCN layer. 

A $K$-layer GCN is identical to applying a $K$-layer MLP to the feature vector $\mathbf{x}_i$ of each node in the graph, except that the hidden representation of each node is averaged with its neighbors at the beginning of each layer. In each graph convolution layer, node representations are updated in three stages: feature propagation, linear transformation, and a pointwise nonlinear activation (see \autoref{fig:method}). For the sake of clarity, we describe each step in detail. 


% feature propagation
\paragraph{Feature propagation} is what distinguishes a GCN from an MLP. 
At the beginning of each layer the features $\mathbf{h}_i$ of each node $v_i$ are averaged with  the feature vectors  in its local neighborhood, 
\begin{equation}
    \bar{\mathbf{h}}_i^{(k)} \leftarrow \frac{1}{d_i + 1} \mathbf{h}_i^{(k-1)}+\sum_{j=1}^n \frac{a_{ij}}{\sqrt{(d_i + 1) (d_j + 1)}}\mathbf{h}_j^{(k-1)}.\label{eq:update}
\end{equation}
More compactly, we can express this update over the entire graph as a simple matrix operation.  Let $\rmS$ denote the ``normalized'' adjacency matrix with added self-loops, 
\begin{align} 
\label{eq:propagation_matrix}
    \mathbf{S} & = \tilde{\rmD}^{-\frac{1}{2}} \tilde{\rmA} \tilde{\rmD}^{-\frac{1}{2}},
\end{align}
where $\tilde{\rmA} = \rmA + \rmI$ and $\tilde{\rmD}$ is the degree matrix of $\tilde{\rmA}$. The simultaneous update in~\autoref{eq:update} for all nodes becomes a simple sparse matrix multiplication
%
\begin{equation}
    \bar{\mathbf{H}}^{(k)} \leftarrow \mathbf{S} \mathbf{H}^{(k-1)}.
%
\end{equation}
%
Intuitively, this step smoothes the hidden representations locally along the edges of the graph and ultimately encourages similar predictions among locally connected nodes.


% Linear Transformation
\paragraph{Feature transformation and nonlinear transition.} 
After the local smoothing, a GCN layer is identical to a standard MLP.  Each layer is associated with a learned weight matrix $\boldsymbol{\Theta}^{(k)}$, and the smoothed hidden feature representations are transformed linearly. 
Finally, a nonlinear activation function such as $\relu$ is applied pointwise before outputting feature representation $\rmH^{(k)}$. In summary, the representation updating rule of the $k$-th layer is: 
\begin{align} \label{eq:gcn_propagation}
    \mathbf{H}^{(k)} & \leftarrow  \relu \left( \bar{\mathbf{H}}^{(k)} \boldsymbol{\Theta}^{(k)}\right). 
\end{align}
The pointwise nonlinear transformation of the $k$-th layer is followed by the feature propagation of the $(k+1)$-th layer.
% node classification
\paragraph{Classifier.} For node classification, and similar to a standard MLP, the last layer of a GCN predicts the labels using a \textit{softmax} classifier. Denote the class predictions for $n$ nodes as $\hat{\mathbf{Y}} \in \mathbb{R}^{n\times C}$ where 
$\hat{y}_{ic}$ denotes the probability of node $i$ belongs to class $c$.
The class prediction $\hat{\mathbf{Y}}$ of a $K$-layer GCN can be written as:
\begin{align}
\hat{\mathbf{Y}}_{\text{GCN}} & = \softmax\left( \rmS \mathbf{H}^{(K-1)} \boldsymbol{\Theta}^{(K)}\right),
\end{align}
where $\softmax(\rvx) = \exp(\rvx) / \sum_{c=1}^C \exp(x_c)$ acts as a normalizer across all classes. 

\subsection{Simple Graph Convolution}
% transition to SGC
In a traditional MLP, deeper layers increase the expressivity because it allows the creation of feature hierarchies, e.g. features in the second layer build on top of the features of the first layer. In GCNs, the layers have a second important function: in each layer the hidden representations are averaged among neighbors that are one hop away. This implies that after $k$ layers a node obtains feature information from all nodes that are $k-$hops away in the graph. This effect is similar to convolutional neural networks, where depth increases the receptive field of internal features~\cite{hariharan2015hypercolumns}.  Although convolutional networks can benefit substantially from increased depth~\cite{huang2016deep}, typically MLPs obtain little benefit beyond 3 or 4 layers. 

\paragraph{Linearization.}
We hypothesize that the nonlinearity between GCN layers is not critical - but that the majority of the benefit arises from the local averaging. We therefore remove the nonlinear transition functions between each layer and only keep the final softmax (in order to obtain probabilistic outputs). 
The resulting model is linear, but still has the same increased ``receptive field'' of a $K$-layer GCN,
\begin{align}
    \hat{\mathbf{Y}} & = \softmax\left(\mathbf{S}\ldots\mathbf{S}\mathbf{S} \mathbf{X} \boldsymbol{\Theta}^{(1)}\boldsymbol{\Theta}^{(2)}\ldots\boldsymbol{\Theta}^{(K)} \right).
    \end{align}
To simplify notation we can collapse the repeated multiplication with the normalized adjacency matrix $\rmS$ into a single matrix by raising $\rmS$ to the $K$-th power, $\rmS^K$. Further, we can reparameterize our weights into a single matrix  $\boldsymbol{\Theta}=\boldsymbol{\Theta}^{(1)} \boldsymbol{\Theta}^{(2)} \ldots \boldsymbol{\Theta}^{(K)}$.  The resulting classifier becomes
\begin{equation}
    \hat{\mathbf{Y}}_{\text{\method{}}}=\softmax\left(\mathbf{S}^K \mathbf{X} \boldsymbol{\Theta} \right),\label{eq:SGC}
\end{equation}
which we refer to as \Method{} (\method{}). 

% SGC conclusion 
\paragraph{Logistic regression.} \autoref{eq:SGC} gives rise to a  natural and intuitive interpretation of \method{}: by distinguishing between feature extraction and classifier, \method{} consists of a fixed (i.e., parameter-free) feature extraction/smoothing component $\bar{\rmX}= \rmS^K \rmX$ followed by a linear logistic regression classifier $\hat{\mathbf{Y}}=\softmax(\bar{\rmX}\boldsymbol{\Theta})$. Since the computation of $\bar{\rmX}$ requires no weight it is essentially equivalent to a feature pre-processing step and the entire training of the model reduces to straight-forward multi-class logistic regression on the pre-processed features $\bar \rmX$.


\paragraph{Optimization details.} The training of logistic regression is a well studied convex optimization problem and can be performed with any efficient second order method or stochastic gradient descent~\cite{bottou2010large}. Provided the graph connectivity pattern  is sufficiently sparse, SGD naturally scales to very large graph sizes and the training of \method{} is  drastically faster than that of GCNs.   



\section{Spectral Analysis}
\section{Analysis}
\label{sec:analyze}
\subsection{Importance of Expand Stage}
We perform the ablation study on the importance of the expand stage and show the results in Table \ref{fig:ablation_expand}. We compare the performances of {\bf \textsc{\name-S}} using the expanded or the not expanded candidate sets on {\bf \textsc{UFET}} and {\bf \textsc{CFET}}. We replace the last $48$ candidates recalled by MLC with candidates expanded by MLM and exact matching for {\bf \textsc{UFET}}, and $10$ candidates for {\bf \textsc{CFET}}. Results show that expand stage has a positive effect on performance, it improves the final recall by $+1.0$ and $+2.2$ on  {\bf \textsc{UFET}} and {\bf \textsc{CFET}} without harming the precision.

\begin{table}[t]
\centering
\scalebox{0.75}{
\renewcommand{\arraystretch}{1}
\begin{tabular}{cllll} \toprule
\multicolumn{2}{l}{\bf \textit{Ablation of Expand Stage} }     & \bf \textsc{P}    & \bf \textsc{R}   & \bf \textsc{F1}  \\ \midrule
\multicolumn{5}{l}{\bf \textsc{UFET\ \  MCCE with C2C BERT-large}} \\
\color{blue}\bf \texttt{B} & {\bf \textsc{\name-S$_{128}$ }} (Ours)     & 52.5 & 49.1 & 50.8 \\ 
\color{blue}\bf \texttt{B} & {\bf \textsc{\name-S$_{128}$ w/o Expand }} (Ours)     & 52.7 & 48.1 & 50.3\\ \hline
\multicolumn{5}{l}{\bf \textsc{CFET\ \  MCCE with C2C BERT-base-Chinese}} \\
\color{brown}\bf \texttt{C} & {\bf \textsc{\name-S$_{64}$}} (Ours)  & 55.5 & 62.6 & 58.8 \\ 
\color{brown}\bf \texttt{C} & {\bf \textsc{\name-S$_{64}$ w/o Expand}}   (Ours)   & 55.4 & 60.4 & 57.8 \\ \hline
\midrule
\end{tabular}}
\caption{Ablation study of expand stage.}
\label{fig:ablation_expand}
\end{table}

\subsection{Attentions}
We conduct an ablation study on S2S, C2S, S2C, and C2C attention introduced in Sec. \ref{sec:attn} and show the results in Table \ref{tab:attn}. From the results, we are surprised to find that removing C2C and S2S doesn't have a big negative impact on performance. The {\bf \textsc{\name-S}} using BERT-base reaches $48.8$ F1 even without both C2C and S2S attention. One possible reason is that the interaction between sub-tokens in the sentence can be achieved indirectly by first attending to the candidates and then being attended back by the candidate in the next layer. We also find that the C2S is necessary for the task ($18.7$ F1 without C2S) because we rely on the mention and its context to encode and classify candidates. Furthermore, we found that it is important for sentences to attend to all candidates (S2C), indicating that the interaction between the sentence and different types is crucial for the task.

\begin{table}[t]
\centering
\scalebox{0.75}{
\renewcommand{\arraystretch}{1}
\begin{tabular}{cllll} \toprule
\multicolumn{2}{l}{\bf \textit{Analysis about attention on UFET}}     & \bf \textsc{P}    & \bf \textsc{R}   & \bf \textsc{F1}  \\ \midrule
\multicolumn{5}{l}{\bf \textsc{\name-S using BERT-base}} \\
\color{blue}\bf \texttt{B} & {\bf \textsc{\name-S$_{128}$} FULL}     & 53.2 &  48.3 & 50.6 \\ 
\color{blue}\bf \texttt{B} & {\bf \textsc{\name-S$_{128}$ w/o C2C }}     & 52.3 & 48.3 & 50.2 \\
\color{blue}\bf \texttt{B} & {\bf \textsc{\name-S$_{128}$ w/o S2S }}     & 50.6 & 48.4 & 49.4 \\
\color{blue}\bf \texttt{B} & {\bf \textsc{\name-S$_{128}$ w/o S2C }}     & 48.7 & 47.1 & 47.9 \\ 
\color{blue}\bf \texttt{B} & {\bf \textsc{\name-S$_{128}$ w/o C2S }}     & 19.7 & 17.4 & 18.7\\
\color{blue}\bf \texttt{B} & {\bf \textsc{\name-S$_{128}$ w/o S2S,C2C }}     & 50.2 & 47.3 & 48.8\\
\bottomrule
\end{tabular}}
\caption{Attention analysis.}
\label{tab:attn}
\end{table}








% \subsection{Influence of Candidate Size}





\section{Related Works}
% In this section, we review and analyze related works. We begin with a review of graph convolutional models and then transition to graph attention models. Lastly, we discuss other works on graphs. 

\begin{figure*}[tb!] 
\centering
\includegraphics[width=0.8\textwidth]{acc_run_time_v2.pdf}
\caption{Performance over training time on Pubmed and Reddit. \method{} is the fastest while achieving competitive performance. 
We are not able to benchmark the training time of GaAN and DGI on Reddit because the implementations are not released. 
}
\label{fig:run_time}
\end{figure*}
%
\subsection{Graph Neural Networks}
\citet{Bruna13} first propose a spectral graph-based extension of convolutional networks to graphs. 
In a follow-up work, ChebyNets \cite{Defferrard16} define graph convolutions using Chebyshev polynomials to remove the computationally expensive Laplacian eigendecomposition. GCNs \cite{gcn} further simplify graph convolutions by stacking layers of first-order Chebyshev polynomial filters with a redefined propagation matrix $\mathbf{S}$. 
\citet{FastGCN} propose an efficient variant of GCN based on importance sampling, and \citet{Hamilton17} propose a framework based on sampling and aggregation. 
\citet{dcnn}, \citet{n-gcn}, and \citet{liao2018lanczosnet} exploit multi-scale information by raising $\mathbf{S}$ to higher order.
% \citet{xu2018how} introduce Graph Isomorphism Networks which is claimed to be the most expressive variant of GNNs.
\citet{xu2018how} study the expressiveness of graph neural networks in terms of their ability to distinguish any two graphs and introduce Graph Isomorphism Network, which is proved to be as powerful as the Weisfeiler-Lehman test for graph isomorphism. 
\citet{Klicpera19} separate the non-linear transformation from propagation by using a neural network followed by a personalized random walk.
There are many other graph neural models~\cite{Monet, EP17, Li18}; we refer to \citet{gnn_review, battaglia2018relational, wu2019comprehensive} for a more comprehensive review. 

% GLN
% \citet{agnn} discover that a linear version of GCN can perform competitively and develop a attention-based GCN model.
% \citet{cai2018simple} propose an effective linear baseline for graph classification using node degree statistics.
% \citet{Buchnik18} show that self-training can improve the base linear graph model.
% \citet{Li2019LabelES} propose a generalized version of label propagation and provide a similar spectral analysis on the renormalization trick.

Previous publications have pointed out that simpler, sometimes linear models can be effective for node/graph classification tasks. \citet{agnn} empirically show that a linear version of GCN can perform competitively and propose an attention-based GCN variant. \citet{cai2018simple} propose an effective linear baseline for graph classification using node degree statistics. \citet{Buchnik18} show that models which use linear feature/label propagation steps can benefit from self-training strategies. 
\citet{Li2019LabelES} propose a generalized version of label propagation and provide a similar spectral analysis of the renormalization trick.
% Unlike these previous works, we ...


% Recently, \citet{agnn} proposed an attention-based GCN model for citation networks. The authors point out that a linear version of GCN could perform similarly to GCNs on these classification tasks, which matches our findings. \citet{cai2018simple} propose an effective linear baseline for non-attribute graph classification using simple node degree statistics.

% \subsection{Graph Attention Models}
Graph Attentional Models learn to assign different edge weights at each layer based on node features and have achieved state-of-the-art results on several graph learning tasks \citep{gat, agnn, zhang2018gaan, ADGPM}.
However, the attention mechanism usually adds significant overhead to computation and memory usage. 
We refer the readers to \citet{attention-survey} for further comparison.

\subsection{Other Works on Graphs} 
Graph methodologies can roughly be categorized into two approaches: graph embedding methods and graph laplacian regularization methods. 
Graph embedding methods \citep{Weston2008, Perozzi14, Yang16, infomax} represent nodes as high-dimensional feature vectors. 
Among them, DeepWalk~\citep{Perozzi14} and Deep Graph Infomax (DGI)~\citep{infomax} use unsupervised strategies to learn graph embeddings.
DeepWalk relies on truncated random walk and uses a skip-gram model to generate embeddings, whereas DGI trains a graph convolutional encoder through maximizing mutual information. 
Graph Laplacian regularization \citep{Zhu03, Zhou04,Belkin04b,Belkin2006} introduce a regularization term based on graph structure which forces nodes to have similar labels to their neighbors.
Label Propagation~\citep{Zhu03} makes predictions by spreading label information from labeled nodes to their neighbors until convergence. 

\section{Experiments and Discussion}
\section{Experiments}
We conduct experiments on two ultra-fine entity typing datasets, {\bf \textsc{UFET}} (English) and {\bf \textsc{CFET}} (Chinese). Their data statistics are shown in Table \ref{tab:stat}. We mainly focus on and report the macro-averaged recall at the recall and expand stage, and concern mainly on the macro-$F1$ of the final prediction at the filter stage. We also evaluate the {\bf \textsc{\name}} models on the fine-grained (130 types) and coarse-grained (9 types) settings of entity typing without the recall and expand stage.
\subsection{UFET and CFET}
\subsubsection{Recall Stage}
\label{sec:recall}
We compare the recall@$K$ on the test sets of {\bf \textsc{UFET}} and {\bf\textsc{CFET}} between the trained MLC model (introduced in \ref{sec:mlc}) and a traditional BM25 model \cite{bm25} in Figure \ref{fig:recall}. The MLC model uses the RoBERTa-large as backbone and is tuned based on the recall@$128$ on the development set. We use AdamW optimizer with a learning rate of $2\times10^{-5}$. Results show that MLC is a strong recall model, it consistently has better recall compared to BM25 on both {\bf\textsc{UFET}} and {\bf\textsc{CFET}} dataset, and the recall@$128$ reaches over $85\%$ on {\bf \textsc{UFET}}, and over $94\%$ on {\bf \textsc{CFET}}.

\begin{figure}[t]
     \centering
     \begin{subfigure}[h]{0.5\textwidth}
         \centering
         \includegraphics[width=\textwidth]{src/img/recall_compare_bm25.pdf}
         \label{fig:mb2}
     \end{subfigure}   
 \caption{Recall@$K$ of MLC and BM25.}
 \label{fig:recall}
\end{figure}

\subsection{Expand Stage}
\label{sec:expand}
In Table \ref{tab:expand}, we evaluate the F1 scores of all candidates expanded by exact match, and top-$10$ candidates expanded by the MLM using Bert-large. We also demonstrate the improvement of recall by using candidate expansion in Figure \ref{fig:expand_improvement}. On {\bf \textsc{UFET}} dataset, expanding around $32$ additional candidates based on $112$ MLC candidates results in $2\%$ higher recall compared to recalling all $128$ candidates by MLC. The recall of $128$ candidates after the expansion is comparable to the recall of $180$ candidates recalled from MLC. Similarly, expanding $10$ candidates is comparable to additionally recalling $80$ candidates using MLC.
In our experiments, we replace the last $48$ candidates recalled by MLC with the candidates recalled by MLM and Exact match for {\bf \textsc{UFET}} and $10$ for {\bf \textsc{CFET}}. We found the expand stage has a positive effect on the final performance of {\bf \textsc{\name}}s, and helps them reach SOTA performance (analyze in Sec. \ref{sec:analyze}).


\begin{table}[t]
\centering
\scalebox{0.75}{
\begin{tabular}{cccccc} 
\toprule
{\bf \textsc{Dataset}} & {\bf \textsc{Expand}} &   {\bf \textsc{P}}  & {\bf \textsc{R}}  &  {\bf \textsc{F1}} & \small{Avg \# Expanded}  \\ \midrule
\multirow{2}{*}{\bf \textsc{UFET}} & {\bf \textsc{Match}}      & 11.2   & 11.3     & 9.8    & 5.23     \\
      & {\bf \textsc{MLM}}  &  8.5     &   17.1   &  10.7  &    10    \\ \midrule
\multirow{2}{*}{\bf \textsc{CFET}} & {\bf \textsc{Match}}   &  11.4  &  14.5  & 11.2   & 4.57    \\
 & {\bf \textsc{MLM}}  & 21.3   &  19.5  & 17.7    & 10    \\ \midrule
\end{tabular}}
\caption{Evaluation of the recalled candidates.}
\label{tab:expand}
\end{table}
\begin{figure}[t]
     \centering
     \begin{subfigure}[h]{0.45\textwidth}
         \centering
         \includegraphics[width=\textwidth]{src/img/recall_ufet.pdf}
         \caption{Recall@$128$ on {\bf \textsc{UFET}} by including different number of expanded candidates. }
         \label{fig:c1}
     \end{subfigure}
     \vfill
     \begin{subfigure}[h]{0.45\textwidth}
         \centering
         \includegraphics[width=\textwidth]{src/img/recall_cfet.pdf}
         \caption{Recall@$64$ on {\bf \textsc{CFET}} by including different number of expanded candidates.}
         \label{fig:c2}
     \end{subfigure}
\caption{Demonstration of the effect of expand stage. $x$-axis represents the number of candidates expanded by MLM/MLM+MATCH among these $128$ candidates. }
\label{fig:expand_improvement}
\end{figure}
\label{sec:exp_expand}
\subsection{Filter Stage and Final Results.}
\begin{table}[h!]
\centering
\scalebox{0.73}{
\renewcommand{\arraystretch}{1}
\begin{tabular}{cllll} \toprule
\multicolumn{2}{l}{\bf \textit{Base Models on UFET} }     & \bf \textsc{P}    & \bf \textsc{R}   & \bf \textsc{F1}  \\ \midrule
\multicolumn{5}{l}{\emph{MLC-like models}}        \\
\color{blue} \bf \texttt{B}& {\bf \textsc{Box4Types}}\cite{box4types}  & 52.8 & 38.8 & 44.8  \\
\color{blue}\bf \texttt{B}& {\bf \textsc{LDET}}$^\dagger$  \cite{onoe-durrett-2019-learning}          & 51.5 & 33.0 & 40.1 \\ 
\color{blue}\bf \texttt{B}& {\bf \textsc{MLMET}}$^\dagger$   {\cite{mlmet}}   & 53.6 & 45.3 & 49.1  \\
\color{blue}\bf \texttt{B}& {\bf \textsc{PL}}  \cite{ding2021prompt}   & 57.8 & 40.7 & 47.7 \\
\color{blue}\bf \texttt{B}& {\bf \textsc{DFET}}    \cite{dfet}      & 55.6 & 44.7 & 49.5 \\
\color{blue}\bf \texttt{B}& {\bf \textsc{MLC}} (reimplemented by us) & 46.5 & 34.9 & 39.9 \\ 
\color{red}\bf \texttt{R}& {\bf \textsc{MLC}} (reimplemented by us) & 42.2 & 44.9 & 43.5 \\ \hline 
\multicolumn{5}{l}{\emph{Seq2seq based models}}      \\
\color{blue}\bf \texttt{B} & {\bf \textsc{LRN} }  {\cite{liu-etal-2021-fine}}              & 54.5 & 38.9 & 45.4  \\\hline
\multicolumn{5}{l}{\emph{Filter models under our recall-expand-filter paradigm}}      \\
\color{blue}\bf \texttt{B} & {\bf \textsc{Vanilla CE}$_{128}$}   & 47.2 & 48.5 & 47.8 \\ 
\color{blue}\bf \texttt{B} & {\bf \textsc{\name-S$_{128}$}} (Ours)  & 53.2 & 48.3 & {\bf 50.6} \\ 
\color{blue}\bf \texttt{B} & {\bf \textsc{\name-S$_{128}$ w/o C2C}}   (Ours)   & 52.3 & 48.3 & 50.2 \\ 
\color{blue}\bf \texttt{B} & {\bf \textsc{\name-B$_{128}$}} (Ours)    & 49.9 & 50.0 & 49.9 \\ 
\color{blue}\bf \texttt{B} & {\bf \textsc{\name-B$_{128}$ w/o C2C}} (Ours)     & 49.9 & 48.2 & 49.0 \\ \hline
\color{red}\bf \texttt{R} & {\bf \textsc{Vanilla CE}$_{128}$}   & 49.6 & 49.0 & 49.3 \\ 
\color{red}\bf \texttt{R} & {\bf \textsc{\name-S$_{128}$}} (Ours)  & 53.3 & 47.3 & 50.1 \\ 
\color{red}\bf \texttt{R} & {\bf \textsc{\name-S$_{128}$ w/o C2C}}   (Ours)  & 53.2 & 46.6 & 49.7 \\ 
\color{red}\bf \texttt{R} & {\bf \textsc{\name-B$_{128}$}} (Ours)  & 52.5 & 47.9 & 50.1 \\ 
\color{red}\bf \texttt{R} & {\bf \textsc{\name-B$_{128}$ w/o C2C}} (Ours)     & 52.7 & 46.4 & 49.3 \\ \hline
\midrule
\multicolumn{2}{l}{\bf \textit{Large Models on UFET} }     & \bf \textsc{P}    & \bf \textsc{R}   & \bf \textsc{F1}  \\ \midrule
\multicolumn{5}{l}{\emph{MLC-like models}}        \\
\color{red}\bf \texttt{R} & {\bf \textsc{MLC}}  \cite{npcrf}               & 47.8 & 40.4 & 43.8  \\
\color{red}\bf \texttt{R} & {\bf \textsc{MLC-NPCRF}} \cite{npcrf}             & 48.7 & 45.5 & 47.0  \\
\color{red}\bf \texttt{R} & {\bf \textsc{MLC-GCN}} \cite{xiong-etal-2019-imposing}     & 51.2 & 41.0 & 45.5 \\
\color{blue}\bf \texttt{B} & {\bf \textsc{PL}}  \cite{ding2021prompt}       & 59.3 & 42.6 & 49.6  \\
\color{blue}\bf \texttt{B} & {\bf \textsc{PL-NPCRF}}  \cite{npcrf}  & 55.3 & 46.7 & {50.6}\\ \hline
\multicolumn{4}{l}{\emph{Cross-encoder based models and {\bf \textsc{\name}}s}}      \\
\color{red}\bf \texttt{R} & {\bf \textsc{LITE+L}}  \cite{lite}             & 48.7 & 45.8 & 47.2  \\
\color{teal}\bf \texttt{RM} & {\bf \textsc{LITE+NLI+L}} \cite{lite} & 52.4 & 48.9 & {50.6} \\ \hline
\multicolumn{4}{l}{\emph{Filter models under our recall-expand-filter paradigm}}   \\ 
\color{blue}\bf \texttt{B} & {\bf \textsc{Vanilla CE$_{128}$}}   & 50.3 & 49.6 & 49.9 \\ 
\color{blue}\bf \texttt{B} & {\bf \textsc{\name-S$_{128}$}}  (Ours)   & 52.5 & 49.1 & 50.8 \\ 
\color{blue}\bf \texttt{B} & {\bf \textsc{\name-S$_{128}$ w/o C2C}}   (Ours)   & 54.1 & 47.1 & 50.4 \\ 
\color{blue}\bf \texttt{B} & {\bf \textsc{\name-B$_{128}$}} (Ours)    & 54.0 & 48.6 & 51.2 \\ 
\color{blue}\bf \texttt{B} & {\bf \textsc{\name-B$_{128}$ w/o C2C}} (Ours)     & 52.8 & 48.3 & 50.4 \\ \hline
\color{red}\bf \texttt{R} & {\bf \textsc{Vanilla CE$_{128}$}}   & 54.5 & 49.3 & 51.8 \\ 
\color{red}\bf \texttt{R} & {\bf \textsc{\name-S$_{128}$}}  (Ours)   & 50.8 & 49.8  &  50.3 \\ 
\color{red}\bf \texttt{R} & {\bf \textsc{\name-S$_{128}$ w/o C2C}}   (Ours)   & 51.5 & 48.8 & 50.1 \\ 
\color{red}\bf \texttt{R} & {\bf \textsc{\name-B$_{128}$}} (Ours)    & 51.9 & 50.8 & 51.4 \\ 
\color{red}\bf \texttt{R} & {\bf \textsc{\name-B$_{128}$ w/o C2C}} (Ours)     & 51.6 & 51.6 & 51.6 \\ \hline
\color{teal}\bf \texttt{RM} & {\bf \textsc{\name-B$_{128}$ w/o C2C}} (Ours) & 56.3 & 48.5 & {\bf 52.1} \\ \hline
\midrule
\end{tabular}}
\caption{Macro-averaged UFET result. {\bf \textsc{LITE+L}} is LITE without NLI pretraining, {\bf \textsc{LITE+L+NLI}} is the full LITE model. Methods marked by $\dagger$ utilize either distantly supervised or augmented data for training. {\bf \textsc{\name-S$_{128}$}} denotes we use $128$ candidates recalled and expanded from the first two stages.}
\label{tab:ufet}
\end{table}
\begin{table}[t]
\centering
\scalebox{0.75}{
\renewcommand{\arraystretch}{1}
\begin{tabular}{cllll} \toprule
\multicolumn{2}{l}{\bf \textit{Models on CFET} }     & \bf \textsc{P}    & \bf \textsc{R}   & \bf \textsc{F1}  \\ \midrule
\multicolumn{5}{l}{\emph{MLC-like models}}        \\
\color{purple}\bf \texttt{N}& {\bf \textsc{MLC}} & 55.8 & 58.6 & 57.1 \\  
\color{purple}\bf \texttt{N}& {\bf \textsc{MLC-NPCRF}} \cite{npcrf}     & 57.0 & 60.5 & 58.7 \\ 
\color{purple}\bf \texttt{N}& {\bf \textsc{MLC-GCN}} \cite{xiong-etal-2019-imposing}   & 51.6 & 63.2 & 56.8 \\ 
\color{brown}\bf \texttt{C}& {\bf \textsc{MLC}} & 54.0 & 59.5 & 56.6 \\  
\color{brown}\bf \texttt{C}& {\bf \textsc{MLC-NPCRF}} \cite{npcrf}   & 54.0 & 61.6 & 57.3 \\  
\color{brown}\bf \texttt{C}& {\bf \textsc{MLC-GCN}} \cite{xiong-etal-2019-imposing} & 56.4 & 58.6 & 57.5 \\ \midrule 
\multicolumn{5}{l}{\emph{Filter models under our recall-expand-filter paradigm}}      \\
\color{purple}\bf \texttt{N} & {\bf \textsc{Vanilla CE}}   & 57.6 & 64.3 & 60.7 \\ 
\color{brown}\bf \texttt{C} & {\bf \textsc{Vanilla CE}}   & 54.0 & 63.3 & 58.3 \\  \hline
\color{purple}\bf \texttt{N} & {\bf \textsc{\name-S$_{64}$}} (Ours)  & 58.4 & 62.1 & 60.2 \\ 
\color{purple}\bf \texttt{N} & {\bf \textsc{\name-S$_{64}$ w/o C2C}}   (Ours)   & 59.1 & 61.5 & 60.3 \\ 
\color{purple}\bf \texttt{N} & {\bf \textsc{\name-B$_{64}$}} (Ours)    & 56.7 & 66.1 & 61.1 \\ 
\color{purple}\bf \texttt{N} & {\bf \textsc{\name-B$_{64}$ w/o C2C}} (Ours)     & 58.8 & 64.1 & 61.4 \\ \hline
\color{brown}\bf \texttt{C} & {\bf \textsc{\name-S$_{64}$}} (Ours)  & 55.5 & 62.6 & 58.8 \\ 
\color{brown}\bf \texttt{C} & {\bf \textsc{\name-S$_{64}$ w/o C2C}}   (Ours)   & 54.0 & 63.4 & 58.3 \\ 
\color{brown}\bf \texttt{C} & {\bf \textsc{\name-B$_{64}$}} (Ours)    & 55.0 & 63.5 & 59.0 \\ 
\color{brown}\bf \texttt{C} & {\bf \textsc{\name-B$_{64}$ w/o C2C}} (Ours)     & 57.3 & 61.3 & 59.3 \\ \hline
\midrule
\end{tabular}}
\caption{Macro-averaged CFET result.}
\label{tab:cfet}
\end{table}

In this section, we report the performance of {\bf \textsc{MCCE}} variants as the filter models and compare them with various strong baselines that we will introduce later. We also compare the inference speed of different models in this section. For filter models, we treat the number of candidates $K$ recalled and expanded by the first two stages as hyper-parameters, and tune it on the development set. We found the choice of PLM backbones has a non-negligible effect on the performance, and the PLM backbone of previous methods varies. Therefore for fairer comparisons to baselines, we conduct experiments of {\bf \textsc{\name}} using different backbone PLMs for our {\bf \textsc{\name}} models and report the results. For all {\bf \textsc{\name}} models, we use AdamW optimizer with a learning rate tuned between $5\times 10^{-6}$ and $2\times 10^{-5}$. The batch size we use is $4$ and we train the models for at most $50$ epochs with early stopping. {\bf \textsc{UFET}} also provides a large dataset obtained from distant supervision such as entity linking, we do not use it and only train and evaluate our models on human-labeled data.
\paragraph{Baselines}
The {\bf \textsc{MLC}} model we used for the recall stage and the cross-encoder ({\bf \textsc{CE}}) we introduced in Sec. \ref{sec:vanilla_ce} are natural baselines. We also compare our methods with recent PLM-based methods. {\bf \textsc{LDET} }\cite{onoe-durrett-2019-learning} is an MLC with Bert-base-uncased and ELMo \cite{elmo} trained on 727k examples automatically denoised from the distantly labeled UFET. {\bf \textsc{GCN} }\cite{xiong-etal-2019-imposing} uses GCN to model type correlations and obtain type embeddings. Types are scored by dot-product of mention and type embeddings. The original paper uses BiLSTM as the mention encoder and we use the results re-implemented by \citet{npcrf} using RoBERTa-large. {\bf \textsc{Box4Type} }\cite{box4types} uses Bert-large as the backbone and uses box embedding to encode mentions and types for training and inference. {\bf \textsc{LRN} }\cite{liu-etal-2021-fine} use Bert-base as the encoder and an LSTM decoder to generate types in a seq2seq manner. {\bf \textsc{MLMET} }\cite{mlmet} is a {\bf \textsc{MLC}} with Bert-base, but first pretrained by the distantly-labeled data augmented by masked word prediction, then finetuned and self-trained on the 2k human-annotated data. {\bf \textsc{PL}} \cite{ding2021prompt} uses prompt learning for entity typing. {\bf \textsc{DFET} }\cite{dfet} uses {\bf \textsc{PL}} as backbone and is a multi-round automatic denoising method for 2k labeled data. {\bf \textsc{LITE} }\cite{lite} is the previous SOTA system that formulates entity typing as textual inference. {\bf \textsc{LITE}} uses RoBERTa-large-MNLI as the backbone, and is a cross-encoder (introduced in Sec. \ref{sec:vanilla_ce}) with designed templates and a hierarchical loss. \citet{npcrf} proposes {\bf \textsc{NPCRF}} to enhance backbones such as {\bf \textsc{PL}} and {\bf \textsc{MLC}} by modeling type correlations, and reach performance comparable to {\bf \textsc{LITE}}.

\paragraph{Naming Conventions}
Let {\bf \textsc{\name-S}} be the {\bf \textsc{\name}} model that treats candidates as sub-tokens, and {\bf \textsc{\name-B}} be the model representing candidates as fixed-size blocks. The {\bf \textsc{\name}} model without {\bf \textsc{C2C}} attention (mentioned in Sec. \ref{sec:attn}) is denoted as {\bf \textsc{\name-B} w/o C2C}. For PLM backbones used in {\bf \textsc{UFET}}, we use {\color{blue} \bf \texttt{B}}, {\color{red} \bf \texttt{R}}, {\color{teal} \bf \texttt{RM}} to denote BERT-base-cased \cite{bert}, RoBERTa \cite{liu2019roberta}, and RoBERTa-MNLI \cite{liu2019roberta} respectively. For {\bf \textsc{CFET}}, we adopt two widely-used Chinese PLM, BERT-base-Chinese and NeZha-base-Chinese, and denote them as {\color{brown} \bf \texttt{C}} and {\color{purple} \bf \texttt{N}} respectively. 

\paragraph{UFET Results} We show the results of {\bf \textsc{UFET}} dataset in Table \ref{tab:ufet}. The results show that: (1) The recall-expand-filter paradigm is effective. Filter models outperform all baselines without the paradigm by a large margin. The vanilla CE under our paradigm reaches $51.8$ F1 compared to more complexed CE {\bf \textsc{LITE}} with $50.6$ F1 (2) {\bf \textsc{\name}} models reach SOTA performances. {\bf \textsc{\name-S$_{128}$}} with BERT-base performs best and reaches {\bf 50.6} F1 score, which is comparable to previous SOTA performance of large models such as {\bf \textsc{LITE+NLI+L}} and {\bf \textsc{PL+NPCRF}}. Among large models, {\bf \textsc{\name-B$_{128}$ w/o C2C}} also reaches SOTA performance with {\bf 52.1} F1 score. (3) {\bf \textsc{C2C}} attention is not necessary on large models, but is useful in base models. (4) Large models can utilize type semantics better. We found {\bf \textsc{\name-B}} outperforms {\bf \textsc{\name-S}} on large models, but underperforms {\bf \textsc{\name-S}} on base models. (5) Backbone PLM matters. We found the performance of {\bf \textsc{Vannila CE}} under our paradigm is largely affected by the PLM it used. It reaches $47.8$ F1 with BERT-base and $51.8$ F1 with RoBERTa-large. For {\bf \textsc{\name}} models, we found {\bf \textsc{\name}} performs better than {\bf \textsc{\name-B}} with BERT, and worse than {\bf \textsc{\name-B}} with RoBERTa. 

\begin{table*}[t]
\centering
\scalebox{0.9}{
\begin{tabular}{lllcc} \toprule
\bf \textsc{Model}  & \bf \textsc{\# FP} & \bf \textsc{Attn} & \bf \textsc{sents/sec} & \bf \textsc{F1} \\ \midrule
{\bf \textsc{MLC}} & \small{$1$}  & \small{$L_S^2D$} & 58.8 & 43.8\\
{\bf \textsc{LITE+NLI+L (CE)}}  & \small{$N$}  & \small{$L_S^2D$} & 0.02 & 50.6\\ \midrule \hline
\multicolumn{5}{l}{\emph{filter stage inference speed.}}  \\
{\bf \textsc{Vanilla CE$_{128}$}}  & \small{$128$}  & \small{$L_S^2D$} & 1.64 & 51.8 \\ 
{\bf \textsc{\name-S$_{128}$}}  & \small{$1$}  & \small{$(L_S+128)^2D$} & 60.8 & 50.1 \\ 
{\bf \textsc{\name-B$_{128}$}}  & \small{$1$}  & \small{$(L_S+128B)^2D$} & 22.3 & 51.4\\ 
{\bf \textsc{\name-B$_{128}$ w/o C2C}}  & \small{$1$}  & \small{$(L_S^2+256L_S B + 128 B^2)D$} & 25.2 & {\bf 52.1}\\ \bottomrule
\end{tabular}}
\caption{Inference speed comparison of models. {\bf \textsc{\# FP}} means the number of PLM forward passes required by a single inference. {\bf \textsc{ATTN}} column lists the theoretical attention complexity.  We also report the practical inference speed {\bf \textsc{sents/sec}} and the {\bf \textsc{F1}} scores on {\bf \textsc{UFET}} with RoBERTa-large architecture.}
\label{tab:speed}
\end{table*}

\begin{table}[t]
\centering
\scalebox{0.85}{
\renewcommand{\arraystretch}{1}
\begin{tabular}{cllll} \toprule
\multicolumn{2}{l}{\bf \textit{Models} }     & \bf \textsc{P}    & \bf \textsc{R}   & \bf \textsc{F1}  \\ \midrule
\multicolumn{5}{l}{\emph{coarse (9 types) Open Entity}}        \\ \hline
\color{red}\bf \texttt{R} & {\bf \textsc{MLC}}   & 76.8 & 78.5 & 77.6 \\ 
\color{red}\bf \texttt{R} & {\bf \textsc{Vanilla CE$_{9}$}}   & 82.3 & 81.0 & 81.6 \\ 
\color{red}\bf \texttt{R} & {\bf \textsc{\name-S$_{9}$}}   & 77.0 & 87.7 & 82.0 \\ 
\color{red}\bf \texttt{R} & {\bf \textsc{\name-B$_{9}$ w/o C2C}}   & 77.2 & 85.4 & 81.1 \\ \hline
\multicolumn{5}{l}{\emph{fine (130 types)}}        \\ \hline
\color{red}\bf \texttt{R} & {\bf \textsc{MLC}}   & 70.4 & 63.7 & 66.9  \\ 
\color{red}\bf \texttt{R} & {\bf \textsc{Vanilla CE}$_{130}$}   & 67.9 & 66.4 & 67.1 \\ 
\color{red}\bf \texttt{R} & {\bf \textsc{\name-S$_{130}$}}   & 65.8 & 71.8 & 68.7 \\ 
\color{red}\bf \texttt{R} & {\bf \textsc{\name-B$_{130}$ w/o C2C}}   & 64.1 & 70.5 & 67.1 \\ \hline
\midrule
\end{tabular}}
\caption{Micro-averaged results on UFET fine and coarse.}
\label{tab:ufet-coarse-fine}
\end{table}

\paragraph{CFET Results} We conduct experiments on {\bf \textsc{CFET}} and compare {\bf \textsc{\name}} models with several strong baselines:  {\bf \textsc{NPCRF}} and {\bf \textsc{GCN}} with MLC-like architecture, and {\bf \textsc{Vanilla CE}} under out paradigm which is proved to be better than {\bf \textsc{LITE}} on {\bf \textsc{UFET}}. The results are shown in Table \ref{tab:cfet}. Similar to results in {\bf \textsc{UFET}}, filter models under our paradigm significantly outperform MLC-like baselines, $+2.0$ F1 for Nezha-base and $+1.8$ F1 for BERT-base-Chinese. In {\bf \textsc{CFET}}, {\bf \textsc{\name}-B} is significantly better than {\bf \textsc{\name}-S}, on both Nezha-base and BERT-base-Chinese, indicating the importance of type semantics in Chinese language. We also find that {\bf \textsc{\name} w/o C2C} is generally better than  {\bf \textsc{\name} w/ C2C}, it is possibly because the C2C attention distracts the candidates from attending to mention and contexts.
\paragraph{Speed Comparison} Table \ref{tab:speed} shows the theoretical inference complexity (number of PLM forward passes, and attention complexity), and practical inference speed (number of sentences inferred per second) of different models. We conduct the speed test using NVIDIA TITAN RTX for all models, and the inference batch size is 4.
At the filter stage, the inference speed of {\bf \textsc{\name-S}} is on par with {\bf \textsc{MLC}} (even slightly faster because we don't need to score all types), and is about 40 times faster than {\bf \textsc{Vannila CE}} and thousands of times faster than {\bf \textsc{LITE}}. {\bf \textsc{\name-B w/o C2C}} is not significantly faster than {\bf \textsc{\name-B}} as expected. It's possibly because the computation related to the block attention is not fully optimized by existing deep learning frameworks. The speed advantage of {\bf \textsc{\name-B w/o C2C}} over {\bf \textsc{\name-B}} will be greater with more candidates.


\subsection{Fine-grained and Coarse-grained Entity Typing}
We also conduct experiments on Fine-grained (130-class) and Coarse-grained (9-class, also known as ``Open Entity'') entity typing, and the results are shown in Table \ref{tab:ufet-coarse-fine}. As the type candidate set is much smaller in these settings, we skip the recall and expand stages and directly run the filter models and compare them to baselines. Results show that both {\bf \textsc{\name}-S} and {\bf \textsc{\name}-B} are still better than {\bf \textsc{MLC}} and {\bf \textsc{Vanilla CE}}, and {\bf \textsc{\name}-S} is better than {\bf \textsc{\name}-B} on coarser-grained cases possibly because the coarser-grained types are simpler in surface-forms and {\bf \textsc{\name}-S} will not lose many type semantics.






\section{Conclusion}
\section{Conclusion}
\label{ss: conclusion}

% summary of approach
This paper presents a methodology to evaluate the effectiveness of evasions and its application to studying PDF malware scanners.
Our implementation of the methodology, the Chameleon framework, automatically generates and enriches malicious documents with one or multiple evasions.
We use these documents for an in-depth study of \nbAnalyzers{} PDF scanners and how they are affected by evasions.
More broadly, our methodology can also be used for studying evasions of other malware types, e.g., malicious executables.

% main take-aways
The overall result of our study is cause for concern.
We show that the studied evasions are surprisingly effective in fooling state-of-the-art scanners.
In particular by combining evasions, attackers can bypass modern defenses in both static and dynamic scanners.
Moreover, we find huge variations across scanners, enabling targeted attacks based on evasions picked specifically for a targeted scanner.
All these findings are a call to arms for future work on anti-evasion techniques.

Our work will support future efforts toward improving malware scanners in several ways.
First, the results of our study help security vendors to better understand their vulnerability to specific evasions and to focus their attention on mitigating the most effective evasions.
Second, we are releasing the corpus of malicious, evasive documents generated by Chameleon as a ready-to-use benchmark.
We are in contact with several developers of PDF scanners, and some of them, e.g., SploitGuard and SAFE-PDF, have already used our benchmark to test and improve their security scanners.
Finally, the Chameleon framework provides a basis for expanding the set of benchmarks by incorporating future evasions, exploits, and payloads.



\section*{Acknowledgement}
This research is supported in part by grants from the National
Science Foundation (III-1618134, III-1526012, IIS1149882,
IIS-1724282, and TRIPODS-1740822), the Office
of Naval Research DOD (N00014-17-1-2175), 
Bill and Melinda Gates Foundation, and 
Facebook Research. We are thankful for
generous support by SAP America Inc. 
Amauri Holanda de Souza Jr. thanks CNPq (Brazilian Council for Scientific and Technological Development) for the financial support.
We appreciate the discussion with Xiang Fu, Shengyuan Hu, Shangdi Yu, Wei-Lun Chao and Geoff Pleiss as well as the figure design support from Boyi Li.
% Last but not least, we would like to thank Johannes Klicpera, Aleksandar Bojchevski, Stephan Gunnemann, and Xiao-Ming Wu for helping us identify the similarities and differences between this work and their papers~\citep{Klicpera19,Li2019LabelES}.


% \newpage

\bibliography{references}
\bibliographystyle{icml2019}


%%%%%%%%%%%%%%%%%%%%%%%%%%%%%%%%%%%%%%%%%%%%%%%%%%%%%%%%%%%%%%%%%%%%%%%%%%%%%%%
%%%%%%%%%%%%%%%%%%%%%%%%%%%%%%%%%%%%%%%%%%%%%%%%%%%%%%%%%%%%%%%%%%%%%%%%%%%%%%%
% DELETE THIS PART. DO NOT PLACE CONTENT AFTER THE REFERENCES!
%%%%%%%%%%%%%%%%%%%%%%%%%%%%%%%%%%%%%%%%%%%%%%%%%%%%%%%%%%%%%%%%%%%%%%%%%%%%%%%
%%%%%%%%%%%%%%%%%%%%%%%%%%%%%%%%%%%%%%%%%%%%%%%%%%%%%%%%%%%%%%%%%%%%%%%%%%%%%%%
\clearpage

\twocolumn[   %or twocolumn

\icmltitle{Simplifying Graph Convolutional Networks \\ (Supplementary Material)}

]

\appendix

\section{The spectrum of $\tilde{\boldsymbol{\Delta}}_{\text{sym}}$}

The normalized Laplacian defined on graphs with self-loops, $\tilde{\boldsymbol{\Delta}}_{\text{sym}}$, consists of an instance of generalized graph Laplacians and hold the interpretation as a difference operator, i.e. for any signal $\rvx \in \mathbb{R}^n$ it satisfies 
\begin{equation}
(\tilde{\boldsymbol{\Delta}}_{\text{sym}} \rvx)_i = \sum_{j} \frac{\tilde{a}_{ij}}{\sqrt{d_i + \gamma}} \left(\frac{x_i}{\sqrt{d_i + \gamma}} - \frac{x_j}{\sqrt{d_j + \gamma}}\right). \nonumber
\end{equation}

Here, we prove several properties regarding its spectrum.

\begin{lemma}
(Non-negativity of $\tilde{\boldsymbol{\Delta}}_{\text{sym}}$) The augmented normalized Laplacian matrix is symmetric positive semi-definite.
\end{lemma}
\begin{proof}

The quadratic form associated with $\tilde{\boldsymbol{\Delta}}_{\text{sym}}$ is
\begin{align}
    & \rvx^\top\tilde{\boldsymbol{\Delta}}_{\text{sym}}\rvx  = \sum_i x_i^2 - \sum_i \sum_j\frac{\tilde{a}_{ij} x_i x_j}{\sqrt{(d_i + \gamma)(d_j + \gamma)}} \nonumber \\
    & = \frac{1}{2} \left( \sum_i x_i^2 + \sum_j x_j^2  - \sum_i \sum_j \frac{2\tilde{a}_{ij} x_i x_j}{\sqrt{(d_i + \gamma)(d_j + \gamma)}}\right) \nonumber \\
    & = \frac{1}{2} \left(\sum_i \sum_j \frac{\tilde{a}_{ij} x_i^2}{d_i + \gamma} + \sum_j \sum_i \frac{\tilde{a}_{ij} x_j^2}{d_j + \gamma} \right.
     \nonumber \\ & \left. \quad \quad - \sum_i \sum_j  \frac{2\tilde{a}_{ij} x_i x_j}{\sqrt{(d_i + \gamma)(d_j + \gamma)}}\right) \nonumber \\
    &  = \frac{1}{2} \sum_i \sum_j \tilde{a}_{ij}\left(\frac{x_i}{\sqrt{d_i + \gamma}} - \frac{x_j}{\sqrt{d_j + \gamma}}\right)^2 \geq 0 \label{eq:norm_laplacian_self_loops}
\end{align}


\end{proof}

\begin{lemma}
\label{lem:0_eig}
$0$ is an eigenvalue of both $\boldsymbol{\Delta}_{\text{sym}}$ and $\tilde{\boldsymbol{\Delta}}_{\text{sym}}$.
\end{lemma}
\begin{proof}
First, note that $\rvv=[1,\ldots,1]^\top$ is an eigenvector of $\boldsymbol{\Delta}$ associated with eigenvalue $0$, i.e., $\boldsymbol{\Delta} \rvv = (\rmD-\rmA)\rvv = \mathbf{0}$.

Also, we have that $\tilde{\boldsymbol{\Delta}}_{\text{sym}} = \tilde{\rmD}^{-1/2}(\tilde{\rmD} - \tilde{\rmA})\tilde{\rmD}^{-1/2} = \tilde{\rmD}^{-1/2}\boldsymbol{\Delta}\tilde{\rmD}^{-1/2}$. Denote $\rvv_1=\tilde{\rmD}^{1/2}\rvv$, then
$$
\tilde{\boldsymbol{\Delta}}_{\text{sym}}\rvv_1 = \tilde{\rmD}^{-1/2}\boldsymbol{\Delta}\tilde{\rmD}^{-1/2}\tilde{\rmD}^{1/2}\rvv = \tilde{\rmD}^{-1/2} \boldsymbol{\Delta} \rvv = \mathbf{0}.
$$

Therefore, $\rvv_1=\tilde{\rmD}^{1/2}\rvv$ is an eigenvector of $\tilde{\boldsymbol{\Delta}}_{\text{sym}}$ associated with eigenvalue $0$, which is then the smallest eigenvalue from the non-negativity of $\tilde{\boldsymbol{\Delta}}_{\text{sym}}$. Likewise, 0 can be proved to be the smallest eigenvalues of $\boldsymbol{\Delta}_{\text{sym}}$.
\end{proof}

\begin{lemma}
\label{lem:adj_eig}
Let $\beta_1 \leq \beta_2 \leq \dots \leq \beta_n$ denote eigenvalues of $\rmD^{-1/2}\rmA \rmD^{-1/2}$ and $\alpha_1 \leq \alpha_2 \leq \dots \leq \alpha_n$ be the eigenvalues of $\tilde{\rmD}^{-1/2}\rmA\tilde{\rmD}^{-1/2}$. Then,
\begin{align} \label{eq:bounds_norm_adj}
    & \alpha_1 \geq \frac{\max_i d_i}{\gamma+\max_i d_i}\beta_1, &\alpha_n \leq \frac{\min_i{d_i}}{\gamma + \min_i{d_i}}.
\end{align}
\end{lemma}

\begin{proof}

We have shown that 0 is an eigenvalue of $\boldsymbol{\Delta}_{\text{sym}}$. Since $\rmD^{-1/2}\rmA \rmD^{-1/2} = \rmI - \boldsymbol{\Delta}_{\text{sym}}$, then $1$ is an eigenvalue of $\rmD^{-1/2}\rmA \rmD^{-1/2}$. More specifically, $\beta_n = 1$. In addition, by combining the fact that $\Tr(\rmD^{-1/2}\rmA\rmD^{-1/2})=0=\sum_i \beta_i$ with $\beta_n = 1$, we conclude that $\beta_1 < 0$.

By choosing $\rvx$ such that $\lVert \rvx \rVert =1$ and $\rvy=\rmD^{1/2}\tilde{\rmD}^{-1/2} \rvx$, we have that $\|\rvy\|^2=\sum\limits_i\frac{d_i}{d_i+\gamma}x_i^2$ and $\frac{\min_i d_i}{\gamma+\min_i d_i}\leq \|\rvy\|^2\leq \frac{\max_i d_i}{\gamma+\max_i d_i}$. Hence, we use the Rayleigh quotient to provide a lower bound to $\alpha_1$:
\begin{align*}
\alpha_1 & = \min_{\|\rvx\|=1} \left(\rvx^\top\tilde{\rmD}^{-1/2} \rmA \tilde{\rmD}^{-1/2} \rvx \right) \\
& = \min_{\|\rvx\|=1} \left( \rvy^\top\rmD^{-1/2} \rmA \rmD^{-1/2} \rvy \right) \text{(by replacing variable)} \\
&= \min_{\|\rvx\|=1} \left( \frac{\rvy^\top\rmD^{-1/2} \rmA \rmD^{-1/2}\rvy}{\|\rvy\|^2}\|\rvy\|^2 \right) \\
&\geq \min_{\|\rvx\|=1} \left( \frac{\rvy^\top\rmD^{-1/2} \rmA \rmD^{-1/2} \rvy}{\|\rvy\|^2} \right) \max_{\|\rvx\|=1} \left( \|\rvy\|^2 \right) \\
& ( \because  \min (AB) \geq \min (A) \max(B) \text{ if } \min (A) < 0, \forall B > 0,
\\
&\text{ and \quad} 
\min_{\|\rvx\|=1} \left( \frac{\rvy^\top\rmD^{-1/2} \rmA \rmD^{-1/2} \rvy}{\|\rvy\|^2} \right) = \beta_1 < 0 ) \\
&= \beta_1\max_{\|\rvx\|=1} \|\rvy\|^2 \\
&\geq \frac{\max_i d_i}{\gamma+\max_i d_i}\beta_1.\\
 \end{align*}

One may employ similar steps to prove the second inequality in \autoref{eq:bounds_norm_adj}.

\end{proof}

\begin{proof} [Proof of Theorem 1] 
Note that $\tilde{\boldsymbol{\Delta}}_{\text{sym}} = \rmI - \gamma \tilde{\rmD}^{-1} - \tilde{\rmD}^{-1/2}\rmA\tilde{\rmD}^{-1/2}$. Using the results in Lemma \autoref{lem:adj_eig}, we show that the largest eigenvalue $\tilde{\lambda}_n$ of $\tilde{\boldsymbol{\Delta}}_{\text{sym}}$ is
\begin{align} \label{eq:bound_laplacians}
        \tilde{\lambda}_n  & = \max_{\|\rvx\|=1} \rvx^\top(\rmI - \gamma \tilde{\rmD}^{-1} - \tilde{\rmD}^{-1/2}\rmA\tilde{\rmD}^{-1/2})\rvx  \nonumber \\
               & \leq  1 - \min_{\|\rvx\|=1} \gamma \rvx^\top \tilde{\rmD}^{-1} \rvx - \min_{\|\rvx\|=1} \rvx^\top \tilde{\rmD}^{-1/2} \rmA \tilde{\rmD}^{-1/2} \rvx \nonumber \\
               & = 1 - \frac{\gamma}{\gamma + \max_{i} d_i} - \alpha_1 \nonumber \\
               & \leq  1 - \frac{\gamma}{\gamma + \max_{i} d_i} - \frac{\max_i d_i}{\gamma + \max_i d_i} \beta_1 \nonumber \\
               & <  1 - \frac{\max_i d_i}{\gamma + \max_i d_i} \beta_1 \quad (\gamma > 0 \text{ and } \beta_1 < 0) \nonumber \\
               & < 1 - \beta_1 = \lambda_n
\end{align}


\end{proof}

\section{Experiment Details}
\label{sec:exp-details}
\paragraph{Node Classification.}
We empirically find that on Reddit dataset for \method{}, it is crucial to normalize the features into zero mean and univariate. 

\paragraph{Training Time Benchmarking.} We hereby describe the experiment setup of Figure 3.
\citet{FastGCN} benchmark the training time of FastGCN on CPU, and as a result, it is difficult to compare numerical values across reports.
Moreover, we found the performance of FastGCN improved with a smaller early stopping window (10 epochs); therefore, we could decrease the model's training time.
We provide the data underpinning Figure 3 in \autoref{table:citation-time} and \autoref{table:reddit-time}.
%
\begin{table}[htb!]
\centering
        \small
        \caption{Training time (seconds) of graph neural networks on Citation Networks. Numbers are averaged over 10 runs.}
        \label{table:citation-time}
        \begin{tabular}{l|c|c|c}
        \toprule
        Models & Cora & Citeseer & Pubmed \\ 
        \midrule
        % \midrule
        GCN & $0.49$ & $0.59$ & $8.31$ \\
        % GCN - test w/o relu & $81.8 \pm{0.66}$ & $70.7\pm 0.93$ & $79.1 \pm{0.65}$ \\
        GAT & $63.10$ & $118.10$ &  $121.74$ \\
        FastGCN & $2.47$ & $3.96 $ & $1.77$ \\
        GIN & $2.09$ &  $4.47$ & $26.15$ \\
        LNet & $15.02$ & $49.16$  & $266.47$ \\
        AdaLNet & $10.15$ & $31.80$ & $222.21$ \\
        DGI & $21.24$ & $21.06$ & $76.20$\\
        {\color{modelblue} \method{}} & $0.13$ & $0.14$ & $0.29$ \\
         \bottomrule
        \end{tabular}
\end{table}
%
\begin{table}[htb!]
        \centering
        \small
        \caption{Training time (seconds) on Reddit dataset.}
        \label{table:reddit-time}
        \begin{tabular}{l|l}
        \toprule
        Model & Time(s) $\downarrow$ \\
         \midrule
        SAGE-mean & $78.54$\\
        SAGE-LSTM & $486.53$\\
        SAGE-GCN & $86.86$\\
        FastGCN & $270.45$\\
        {\color{modelblue} \method{}} & $2.70$ \\
        \bottomrule
        \end{tabular}
\end{table}
\begin{figure*}[htb] 
\centering
\includegraphics[width=0.9\textwidth]{figures/propagation.pdf}
\caption{Validation accuracy with \method{} using different propagation matrices.}
\label{fig:propagation-ablation}
\end{figure*}
%
\paragraph{Text Classification.} 
\citet{textGCN} use one-hot features for the word and document nodes. In training SGC, we normalize the features to be between 0 and 1 \textbf{after propagation} and train with L-BFGS for 3 steps. We tune the only hyperparameter, weight decay, using hyperopt\cite{hyperopt} for 60 iterations. Note that we cannot apply this feature normalization for TextGCN because the propagation cannot be precomputed. 
%
\paragraph{Semi-supervised User Geolocation.}
We replace the 4-layer, highway-connection GCN with a 3rd degree propagation matrix ($K=3$) SGC and use the same set of hyperparameters as \citet{Rahimi18}. All experiments on the GEOTEXT dataset are conducted on a single Nvidia GTX-1080Ti GPU while the ones on the TWITTER-NA and TWITTER-WORLD datasets are excuded with 10 cores of the Intel(R) Xeon(R) Silver 4114 CPU (2.20GHz). Instead of collapsing all linear transformations, we keep two of them which we find performing slightly better possibly due to . Despite of this subtle variation, the model is still linear.
%
\paragraph{Relation Extraction.}
We replace the 2-layer GCN with a 2nd degree propagation matrix ($K=2$) SGC and remove the intermediate dropout. We keep other hyperparameters unchanged, including learning rate and regularization. Similar to \citet{relation-extraction}, we report the best validation accuracy with early stopping.
%
\paragraph{Zero-shot Image Classification.}
We replace the 6-layer GCN (hidden size: 2048, 2048, 1024, 1024, 512, 2048) baseline with an 6-layer MLP (hidden size: 512, 512, 512, 1024, 1024, 2048) followed by a SGC with $K=6$. Following \cite{wang2018zero}, we only apply dropout to the output of SGC. Due to the slow evaluation of this task, we do not tune the dropout rate or other hyperparameters. Rather, we follow the GCNZ code and use learning rate of 0.001, weight decay of 0.0005, and dropout rate of 0.5. We also train the models with ADAM~\cite{adam} for 300 epochs.

\section{Additional Experiments}
%
%
\paragraph{Random Splits for Citation Networks.}
Possibly due to their limited size, the citation networks are known to be unstable. 
Accordingly, we conduct an additional 10 experiments on random splits of the training set while maintaining the same validation and test sets. 
%
\begin{table}[th!]
\small
\centering
\caption{Test accuracy (\%) on citation networks (random splits). $^\dagger$We remove the outliers (accuracy $< 0.7/0.65/0.75$) when calculating their statistics due to high variance.}
\label{table:citation-random}
\begin{tabular}{l|c|c|c}
\toprule
 & Cora & Citeseer & Pubmed \\ 
\midrule
\multicolumn{4}{l}{\textbf{Ours:}} \\
GCN &  $80.53 \pm{1.40}$ & $70.67\pm{2.90}$ & $77.09 \pm{2.95}$\\
% GCN - test w/o relu  & $80.18 \pm{1.65}$ & $70.98 \pm 2.93$  & $77.14 \pm{2.99}$\\
GIN  & $76.94 \pm 1.24$ & $66.56 \pm 2.27$ & $74.46 \pm 2.19$ \\
LNet & $74.23 \pm 4.50^\dagger$ & $67.26 \pm 0.81^\dagger$ & $77.20 \pm 2.03^\dagger$ \\
AdaLNet & $72.68 \pm 1.45^\dagger$ & $71.04 \pm 0.95^\dagger$ & $77.53 \pm 1.76^\dagger$ \\
GAT  & $82.29 \pm{1.16}$ & $72.6 \pm{0.58}$  & $78.79 \pm{1.41}$ \\
{\color{modelblue} \method{}} & $80.62 \pm{1.21}$ & $71.40 \pm{3.92}$ & $77.02 \pm{1.62} $\\
 \bottomrule
\end{tabular}
\end{table}
%
\paragraph{Propagation choice.}
We conduct an ablation study with different choices of propagation matrix, namely:
\begin{itemize}
\item[] Normalized Adjacency: $\mathbf{S}_{\text{adj}} = \rmD^{-1/2}\rmA \rmD^{-1/2}$
\item[] Random Walk Adjacency $\mathbf{S}_{\text{rw}} = \rmD^{-1}\rmA$
\item[] Aug. Normalized Adjacency $\tilde{\mathbf{S}}_{\text{adj}} = \tilde{\rmD}^{-1/2}\tilde{\rmA} \tilde{\rmD}^{-1/2}$ \item[] Aug. Random Walk $\tilde{\mathbf{S}}_{\text{rw}} = \tilde{\rmD}^{-1} \tilde{\rmA}$ 
\item[] First-Order Cheby $\mathbf{S}_{\text{1-order}}=(\rmI + \rmD^{-1/2}\rmA \rmD^{-1/2} )$
\end{itemize}

We investigate the effect of propagation steps $K \in \{2..10\}$ on validation set accuracy. 
We use hyperopt to tune L2-regularization and leave all other hyperparameters unchanged. \autoref{fig:propagation-ablation} depicts the validation results achieved by varying the degree of different propagation matrices.

We see that augmented propagation matrices (i.e. those with self-loops) attain higher accuracy and more stable performance across various propagation depths. Specifically, the accuracy of $\mathbf{S}_{\text{1-order}}$ tends to deteriorate as the power $K$ increases, and this results suggests using large filter coefficients on low frequencies degrades \method{} performance on semi-supervised tasks.

Another pattern is that odd powers of $K$ cause a significant performance drop for the normalized adjacency and random walk propagation matrices. This demonstrates how odd powers of the un-augmented propagation matrix use negative filter coefficients on high frequency information. Adding self-loops to the propagation matrix shrinks the spectrum such that the largest eigenvalues decrease from $\approx 2$ to $\approx 1.5$ on the citation network datasets. By effectively shrinking the spectrum, the effect of negative filter coefficients on high frequencies is minimized, and as a result, using odd-powers of $K$ does not degrade the performance of augmented propagation matrices. For non-augmented propagation matrices --- where the largest eigenvalue is approximately 2 --- negative coefficients significantly distort the signal, which leads to decreased accuracy. Therefore, adding self-loops constructs a better domain in which fixed filters can operate. 

\begin{table}[h]
    \centering
    \begin{tabular}{c|cc}
    \toprule
    \# Training Samples & \method{} & GCN \\
    \midrule
    1 & 33.16 & 32.94 \\
    5 & 63.74 & 60.68 \\
    10 & 72.04 & 71.46 \\
    20 & 80.30 & 80.16 \\
    40 & 85.56 & 85.38 \\
    80 & 90.08 & 90.44 \\
    \bottomrule
    \end{tabular}
    \caption{Validation Accuracy (\%) when \method{} and GCN are trained with different amounts of data on Cora. The validation accuracy is averaged over 10 random training splits such that each class has the same number of training examples.} 
    \label{tab:data_ablation}
\end{table}

\paragraph{Data amount.}
We also investigated the effect of training dataset size on accuracy. 
As demonstrated in Table~\ref{tab:data_ablation}, \method{} continues to perform similarly to GCN as the training dataset size is reduced, and even outperforms GCN when there are fewer than $5$ training samples. We reason this study demonstrates \method{} has at least the same modeling capacity as GCN.
%


%%%%%%%%%%%%%%%%%%%%%%%%%%%%%%%%%%%%%%%%%%%%%%%%%%%%%%%%%%%%%%%%%%%%%%%%%%%%%%%
%%%%%%%%%%%%%%%%%%%%%%%%%%%%%%%%%%%%%%%%%%%%%%%%%%%%%%%%%%%%%%%%%%%%%%%%%%%%%%%


\end{document}


% This document was modified from the file originally made available by
% Pat Langley and Andrea Danyluk for ICML-2K. This version was created
% by Iain Murray in 2018, and modified by Alexandre Bouchard in
% 2019. Previous contributors include Dan Roy, Lise Getoor and Tobias
% Scheffer, which was slightly modified from the 2010 version by
% Thorsten Joachims & Johannes Fuernkranz, slightly modified from the
% 2009 version by Kiri Wagstaff and Sam Roweis's 2008 version, which is
% slightly modified from Prasad Tadepalli's 2007 version which is a
% lightly changed version of the previous year's version by Andrew
% Moore, which was in turn edited from those of Kristian Kersting and
% Codrina Lauth. Alex Smola contributed to the algorithmic style files.
