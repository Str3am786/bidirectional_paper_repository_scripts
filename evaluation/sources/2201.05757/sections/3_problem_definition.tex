\section{Preliminaries} 
\label{sec:preliminaries}
% Before introducing our method, this section discusses the problem definition and provides preliminaries for this paper.

\subsection{Account-based blockchains}
% 不同于比特币系统采用基于UTXO的交易模型,以以太坊为首的基于账户的区块链系统xxxx
% 智能合约  DApp
% 由DApp引出Token  Token标准
% 去中心化交易所
Traditional Bitcoin-like blockchains are based on the transaction-centered model \cite{wu2021analysis} whose building block is unspent transaction output (UTXO), which is an indivisible cryptocurrency chunk locked to a specific owner. Each transaction has multiple inputs made up of UTXOs and multiple outputs, and there is a change address in the outputs used to receive the change.

Different from UTXO-based blockchains, account-based blockchain systems like Ethereum and BSC have the concept of account similar to that of traditional banking accounts. 
There are two kinds of accounts in account-based blockchains, namely external owned account (EOA) and smart contract account. 
Accounts are the initiators of blockchain transactions and record some dynamic state information including account balance. 
Especially, each smart contract account is associated with a piece of executable bytecode. There are also two types of transactions in the systems. 
A transaction triggered by an EOA is called external transaction, while a transaction triggered by an invocation of the function in a smart contract account is called internal transaction. 
In addition, an EOA can invoke functions of a smart contract in an external transaction and further result in many internal transactions. Transaction hash consisting of a set of numbers and letters is used to uniquely identify a particular transaction from or to an EOA.
Due to the support of smart contract, everyone can take the advantage of blockchain technology and build DApp projects in account-based blockchains.
Besides the native currency in blockchain systems, there are many third-party tokens representing assets, currency, or access rights of projects in the account-based blockchain ecosystem. 
To facilitate token development and exchange, some token standards are launched in blockchain trading systems, e.g., the ERC20 token standard in Ethereum. 
There are also many DeFi DApps that offer financial services such as token lending and exchange.
% 原来:
% 介绍加密数字货币包括原生币和Token
% There are two types of cryptocurrencies in account-based blockchains: native token and third-party token complimenting some standards.
% For Ethereum, the native token Ether and 
% 介绍什么是DeFi,DeFi世界中的交易语义有哪些

\subsection{Problem Definition}
\label{sec:problem_define}
The transaction tracing task in account-based blockchain trading systems aims to trace the money flows from a given source to the targets that gather the money awaiting cashing out and point the priority money flows for auditors to further verify manually. By modeling the blockchain transaction relationships as graph where nodes indicate the accounts and edges indicate the money flow relationships, we can formulate the transaction tracing problem as follows. 

% \begin{prbdef}
\textbf{Problem formulation.} \textit{Given a source node $s$ in a transaction graph $G$, the goal is to search a connected money transfer subgraph $G_{s}$ from $s$ to its money flow targets around the neighborhood of $s$. $G_{s}$ should contain as many target nodes as possible in the smallest possible size of graph for manual verification.}
% \end{prbdef}
% \textbf{Transaction network}.
% Transaction records on Blockchain trading systems like Bitcoin and Ethereum represent the relationship of cryptocurrency transfer among accounts.
% In this paper, transaction records from blockchain trading systems are represented as a transaction network $G = (V, E)$, where $V$ denotes the node set representing addresses, and $E$ denotes the edge set representing transaction relationships between addresses. 
% \textcolor{red}{Considering the amount, timestamp and token symbol of the transactions, each edge $e \in E$ can be defined as a five-tuple like $(u,v,w,t,b)$, where $u, v \in V$ denote the source and target nodes of $e$, respectively, $w$ denotes the transaction amount, $t$ denotes the transaction timestamp, and $b$ denotes the token symbol.}

% \textbf{Transaction Tracking on Blockchain}.

% The task of transaction tracking on blockchain can be modeled as graph searching on a transaction network, aiming at finding a subgraph of $s$ named \textbf{local tracking network}, denoted by $G_s$, which contains as many target nodes as possible.  
% As mentioned in the introduction, source nodes are usually deployers of various financial scams such as phishing, Ponzi scheme, blackmail, and extortion scams  \cite{wu2021analysis,oggier2020ego}, while the target nodes refer to the addresses which possess the laundered funds awaiting cashing out. 
% Due to the pseudonymous nature of blockchain, it is extremely difficult to trace the money flows between the source and target nodes and understand real-world entities behind the involved addresses and transactions.
% In some cases, some addresses are confirmed to be involved in illegal transaction activities by experts, and then the local tracking network should contain as many of these target nodes as possible.
% While sometimes, no prior information about target nodes is given, and then the local tracking network aims to contain the addresses with some particular labels (such as exchange deposit addresses) that are related to the source, providing us an opportunity to further infer the target nodes from these labeled nodes.


%identify all target nodes from the nodes related to the source node if the target nodes without given before.
% For example, the hackers steal the coin of the exchange, and we need to find the target nodes of hackers through transaction tracking methods.
% In addition, due to the pseudonymous nature of Blockchain trading systems, fetching the identity information for the transaction tracking model to know whether the target nodes have been found without given target nodes before is difficult.
%The goal of the transaction tracking problem can be given as follows:
%\begin{prbdef}

%The goal of this work is to find a network that contains as many nodes and transactions involved in the case as possible.

%Given a source node $s$ and a set of target nodes in a transaction network $G$, the goal of a transaction tracking model is to find a subgraph of $s$ named \textbf{local tracking network}, denoted by $G_s$, that contains as many target nodes as possible.
%\label{def:goal_tt_model}
%\end{prbdef}

% Transaction tracking tasks can be regarded as graph searching tasks on a transaction network, aiming at finding paths among the source node and the target nodes.
% In fact that it is difficult to know the target nodes before the transaction tracking task usually, for example, the hackers steal the coin of the exchange, and we need to find the target nodes of hackers through transaction tracking technology.
% In addition, due to the pseudonymity of transaction records on Blockchain, fetching the identity information of each node in the transaction network is impossible, which makes it difficult for the transaction tracking model to know whether the target node has been found without given target nodes before.

% \begin{prbdef}
% Base on this consideration, our goal is to construct a well-designed transaction tracking model able to find a subgraph named local tracking network from the source node, which satisfies:
% \begin{itemize}
%     \item The paths among the source node and the target nodes can be found, if the targets are given before.
%     \item The paths among the source node and the labeled nodes can be found, if the targets are unknown. 
%     And ensured that the target nodes can be found in the labeled node as much as possible.
% \end{itemize}
% \end{prbdef}

% Base on this consideration, the well-designed transaction tracking model will be able to find a subgraph that is most likely to contain the paths among the source node and the target nodes, even though the target node is not given in advance.
% Such a subgraph can be named a local tracking network.
% In this way, the effectiveness of the transaction tracking model can be evaluated by detecting whether there are enough label nodes in the local tracking network because experts can judge the action of the source node through the label nodes in the local tracking network, and then determine whether the label node is the target node through this action.

% Base on the Definition \ref{def:goal_tt_model}, we define the auditability of the local tracking network.
%According to whether the target nodes in the local tracking network can be identified, the auditability of a transaction network with a center of the source node is given as follow:
%\begin{prbdef}
%A transaction network with a center of the source node can be audited by a local tracking network, which satisfies: \textbf{1)} The paths among the source node and the target nodes can be found if the target nodes are given before. \textbf{2)} The paths among the source node and the labeled nodes can be found if the target nodes without given before, where the target nodes can be inferred from the labeled nodes.
% \end{itemize}
%\end{prbdef}
%For example, when it comes to a source node obtaining an amount of illegal money, laundering the money in a blockchain system and cashing them out via an exchange, the target node is the corresponding exchange deposit address without being given before. And we need to find out the paths between them.
%The transaction network can not be audited by the local tracking network in Figure \ref{fig:useless_tracking}, which misses the paths among the source and the target.
%On the contrary, the transaction network can be audited by the local tracking network as the Figure \ref{fig:useful_tracking}, which finds the path from the source to the target.

% It's reasonable that node $t$ is a target node of the source node $s$, because the stolen fund can be laundered on the exchange.
% There is a useless local tracking network as the Figure \ref{fig:useless_tracking} shows, which misses the path among $s$ and the labeled nodes, and we will not get more information from such a local tracking network about the target nodes of $s$.
% On the contrary, the local tracking network in Figure \ref{fig:useful_tracking} finds the path from $s$ to $t$, which is helpful for us to identify the target nodes.

% For example, when it comes to a source node $s$ obtaining the illegal fund and transfers the illegal fund to a node $t$ labeled as exchange.
% It's reasonable that node $t$ is a target node of the source node $s$, because the stolen fund can be laundered on the exchange.
% There is a useless local tracking network as the Figure \ref{fig:useless_tracking} shows, which misses the path among $s$ and the labeled nodes, and we will not get more information from such a local tracking network about the target nodes of $s$.
% On the contrary, the local tracking network in Figure \ref{fig:useful_tracking} finds the path from $s$ to $t$, which is helpful for us to identify the target nodes.

% In a word, the problem in this paper is how to design a transaction tracking model to complete the transaction tracking task in transaction networks on Blockchain.
% Here gives the definition of transaction tracking task as follow:\\
% \textbf{Definition}.
% \textit{Giving a transaction network $G$ and a source node $s$, the transaction tracking task is finding a local tracking network of $s$ that contains the paths among the source node and the target nodes.
% If the target nodes are unknown, ensured that the target nodes can be found in the labeled nodes of local tracking network as much as possible.}
% \begin{itemize}
%     \item If the target nodes are given, the transaction tracking task is finding a local tracking network of $s$ that contains the paths among the source node and the target nodes.
%     \item If the target nodes are unknown, the transaction tracking task is finding a local tracking network of $s$ that contains the paths among the source node and the labeled nodes ensured that the target nodes can be found in the labeled node as much as possible.
% \end{itemize}

\subsection{Approximate Personalized PageRank}
% To ensure the effectiveness of the tracking model, we usually need to define an importance measurement of other nodes in the network to the source node.
% or the probability of random walk to $u$ starting from $s$
\textbf{Personalized PageRank.} The personalized PageRank vector $\bm{p}_s$ of a source node $s$ in a graph $G=(V,E)$ is defined as the unique solution of Equation \ref{equ:ppr}, i.e.,
\begin{equation}
    \bm{p}_s = \alpha \bm{e}_s + (1-\alpha) \bm{p}_s M,
    \label{equ:ppr}
\end{equation}
where $\alpha$ is a teleport constant between 0 and 1, $\bm{e}_s$ is the indicator vector with a single nonzero element of 1 at $s$, $M$ is a transition matrix and $M=D^{-1}A$ where $D$ and $A$ are degree matrix and adjacency matrix. The definition of personalized PageRank is equivalent to simulating a random walk starting from $s$, and $\bm{p}_s$ is a probability vector where $\bm{p}_s(u)$ is the probability that a certain random walk beginning at $s$ terminates at $u$. 
% Let $\bm{p}_s$ denote the personalized PageRank vector of a source node $s$ in a graph $G=(V,E)$, where $\bm{p}_s(u)$ describes the relevance of a node $u$ to the source node $s$.

\textbf{Approximate personalized PageRank (APPR).} The first algorithm proposed to calculate personalized PageRank is power iteration \cite{page1999pagerank}, which requires high time complexity and not effective in large-scale networks.
Therefore, various efficient approximate solutions of personalized PageRank have been proposed, and the most widely used one is called the ``local push'' algorithm \cite{andersen2006local}. This algorithm starts with all probability residual on the source node of the graph, and pushes the residual to the neighbors iteratively. During the iterations, the residual of each node can be transformed into the relevance to the source node. Finally an estimate of $\bm{p}_s$ can be obtained. In the algorithm, a residual vector $\bm{r}_s$ is used to maintain the residual, where $\bm{r}_s(u)$ denotes the residual of node $u$. By setting $\bm{p}_s=\vec{0}$ and $\bm{r}_s=\bm{e}_s$ for initialization, the local push procedure updates the value of $\bm{p}_s(u)$ as follows:
\begin{equation}
    \begin{cases}
    & \bm{p}_s(u)=\bm{p}_s(u)+\alpha \bm{r}_s(u) \\
    & r(v)=r(v)+(1-\alpha)r(u) / d(u), \\
    % & \bm{r}_s(v)=\bm{r}_s(u)+\theta(u,v)\bm{r}_s(u)
    \end{cases},
    \label{equ:local_push_appr}
\end{equation}
where $v \in N(u)$ is the neighbor of $u$, and 
% the push factor $\theta(u,v)$ between $u$ and $v$ is defined as
% \begin{equation}
%     \theta(u,v)=\frac{1-\alpha}{d(u)},
% \end{equation}
% where 
$d(u)$ is the degree of $u$.
The local push procedure stops when the residual of each node in $G$ is within $\epsilon$.
