%摘要包括:区块链交易网络上交易追踪的重要性,挖掘局部交易网络对下游任务有益,现有研究不足以支撑区块链交易网络上的资产追踪需求,基于什么思路我们提出了怎样的算法,算法在实验中达到了怎样的效果
% Old
% Due to the pseudonymous nature of blockchain, various cryptocurrency systems like Bitcoin and Ethereum have become a hotbed for illegal transaction activities. The ill-gotten profits on many blockchain trading systems are usually laundered into concealed and ``clean'' fund  before being cashed out. Recently, in order to recover the stolen fund of users and reveal the real-world entities behind the transactions, much effort has been devoted to tracking the flow of funds involved in illegal transactions. However, current approaches are facing difficulty in estimating the cost and quantifying the effectiveness of transaction tracking.
% This paper models the transaction records on blockchain as a transaction network, tackle the transaction tracking task as graph searching the transaction network and proposes a general transaction tracking model named as Push-Pop model. 
% Using the three kinds of heuristic designs, namely, tracking tendency, weight pollution, and temporal reasoning, we rewrite the local push procedure of personalized PageRank for the proposed method and name this new ranking method as Transaction Tracking Rank (TTR) which is proved to have a constant computational cost.
% The proposed TTR algorithm is employed in the Push-Pop model for efficient transaction tracking.
% Finally, we define a series of metrics to evaluate the effectiveness of the transaction tracking model. 
% Theoretical and experimental results on realist Ethereum cases show that our method can track the fund flow from the source node more effectively than baseline methods.

Security incidents such as scams and hacks, have become a major threat to the health of the blockchain ecosystem, causing billions of dollars in losses each year for blockchain users.
% Due to the pseudonymous nature of blockchain, it is difficult to reveal the real-world entities behind the blockchain transactions and recover the stolen funds from the massive transaction data. Recently, much effort has been devoted to tracing the flow of funds involved in illegal transactions.
To reveal the real-world entities behind the pseudonymous blockchain account and recover the stolen funds from the massive transaction data, much effort has been devoted to tracing the flow of illicit funds in blockchains recently.
However, most current tracing approaches based on heuristics and taint analysis have limitations in terms of universality, effectiveness, and efficiency. This paper models the blockchain transaction records as a blockchain transaction graph and tackles blockchain transaction tracing as a graph searching task. 
We propose \textbf{TRacer}, a scalable transaction tracing tool for account-based blockchains. 
To infer the relevance between accounts during graph searching, we develop a novel personalized PageRank method in TRacer based on the directed, weighted, temporal, and multi-relationship blockchain transaction graphs. 
To the best of our knowledge, TRacer is the first intelligent transaction tracing tool in account-based blockchains that can handle complex transaction actions in decentralized finance (DeFi).
Experimental results and theoretical analysis prove that TRacer can complete the transaction tracing task effectively at a low cost. 
All codes of TRacer are available at GitHub \footnote{https://github.com/wuzhy1ng/BlockchainSpider}.
% , whose data come from the open APIs \footnote{https://blockscan.com/} ensuring everyone to trace the illicit funds in the account-based blockchains without deployment.