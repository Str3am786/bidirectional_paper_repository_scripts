\begin{lemma_ap}
	Let $f: \mathbb{R}^n \rightarrow \mathbb{R}$ be a k-times continuously differentiable function. Furthermore, assume there exists $a,b,c \in \mathbb{R}$ with $a>0$, such that $f(\mathbf{l}+\mathbf{d}s)= as^2+bs+c$ for all $s \in \mathbb{R}$. Then there exist $z \in \mathbb{R}, \mathbf{r} \in \mathbb{R}^n$ and a positive definite Matrix $\mathbf{Q}\in \mathbb{R}^{n\times n}$ such that $f(\mathbf{x})= c+\mathbf{r}^T\mathbf{x}+\mathbf{x}^T\mathbf{Q}\mathbf{x}$ for all $\mathbf{x} \in \mathbb{R}^n$.
	%f(\mathbf{x})= %c+\mathbf{r}^T\mathbf{x}+\mathbf{x}^T\mathbf{Q}\mathbf{x}$ with $\mathbf{Q}\in %\mathbb{R}^{n\times n}$ hermitian and positive definite.
\end{lemma_ap}
%\vspace{-0.5cm}
%\small
%\doublespacing
\begin{proof}
%	$f(\mathbf{x})$ has the form:\\
\begin{equation}
\begin{aligned}
 g(\mathbf{x})= u+\mathbf{v}^T\mathbf{x}+\mathbf{x}^T\mathbf{W}\mathbf{x} \text{ for some } u\in \mathbb{R}, \mathbf{v}\in \mathbb{R}^n \text{ and } \mathbf{W} \in \mathbb{R}^{n\times n} \\ \Leftrightarrow \forall \mathbf{l},\mathbf{d} \in \mathbb{R}^n \land ||\mathbf{d}||=1 : \sum\limits_{j=1}^{n}\sum\limits_{k=1}^{n}\sum\limits_{l=1}^{n} \frac{\partial^3 g(\mathbf{l}) }{\partial x_j,\partial x_k,\partial x_l}d_{j}d_{k}d_{l}=0
 \end{aligned}
\end{equation}
$\Rightarrow$ holds since we have a polynomial of degree 2 and its third derivative is always a $\mathbf{0}$ tensor.\\
%$\Leftarrow$ holds since the right part is the fourth element of the multivariate Taylor series. Since this element is 0 the following are also 0. The first 3 elements can at most describe a polynomial of degree 3.\\
$\Leftarrow$ holds since the reminder of the quadratic Taylor expansion is always 0.


%$g(\mathbf{x}): for some $u\in \mathbb{R}$,$\mathbf{v}\in \mathbb{R}^n$ and $\mathbf{W} \in \mathbb{R}^{n\times n}$\\
\noindent
In our case the right part is 0 since:
	\begin{equation}
\sum\limits_{j=1}^{n}\sum\limits_{k=1}^{n}\sum\limits_{l=1}^{n} \frac{\partial^3 f(\mathbf{l}) }{\partial x_j,\partial x_k,\partial x_l}d_{j}d_{k}d_{l}= \frac{\partial}{\partial s^3}f(\mathbf{l}+\mathbf{d}s)=0
	\end{equation}
\\	
 In words: $f(\mathbf{x})$ is a parabolic function if and only if for each location $\mathbf{l}$ the third directional derivative of $f(\mathbf{\mathbf{l}})$ in each direction $\mathbf{d}$ is $0$. Which is the case, since the third derivative of each intersection is 0.
\\
$\mathbf{W}$ is positive definite since: \\
%\begin{equation}
%\forall \mathbf{d},\mathbf{l} \in \mathbb{R}^n \land ||\mathbf{d}||=1: a= \frac{1}{2} \frac{\partial}{\partial s^2}f(\mathbf{l}+\mathbf{d}s)=\frac{1}{2}\mathbf{d}^T \mathbf{H}(\mathbf{l}) \mathbf{d} = \mathbf{d}^T \mathbf{W} \mathbf{d} > 0
%\end{equation}
\begin{equation}
\forall \mathbf{d},\mathbf{l} \in \mathbb{R}^n \land ||\mathbf{d}||=1: \mathbf{d}^T \mathbf{W} \mathbf{d} =\frac{1}{2}\mathbf{d}^T \mathbf{H}(\mathbf{l}) \mathbf{d}= \frac{1}{2} \frac{\partial}{\partial s^2}f(\mathbf{l}+\mathbf{d}s) = a > 0
\end{equation}
where $\mathbf{H}$ is the Hessian.
\end{proof}