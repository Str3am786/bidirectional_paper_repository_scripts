% !TEX root = paper.tex
% !TeX spellcheck = en_US

\begin{abstract}
Recent studies in Learning to Rank have shown the possibility to effectively distill a neural network from an ensemble of regression trees. This result leads neural networks to become a natural competitor of tree-based ensembles on the ranking task. Nevertheless, ensembles of regression trees outperform neural models both in terms of efficiency and effectiveness, particularly when scoring on CPU. 
In this paper, we propose an approach for speeding up neural scoring time by applying a combination of Distillation, Pruning and Fast Matrix multiplication. We employ knowledge distillation to learn shallow neural networks from an ensemble of regression trees. Then, we exploit an efficiency-oriented pruning technique that performs a sparsification of the most computationally-intensive layers of the neural network that is then scored with optimized sparse matrix multiplication. Moreover, by studying both dense and sparse high performance matrix multiplication, we develop a scoring time prediction model which helps in devising neural network architectures that match the desired efficiency requirements. Comprehensive experiments on two public learning-to-rank datasets show that neural networks produced with our novel approach are competitive at any point of the effectiveness-efficiency trade-off when compared with tree-based ensembles, providing up to $4$x scoring time speed-up without affecting the ranking quality.
\end{abstract}

\documentclass[journal,compsoc]{IEEEtran}

\usepackage{algorithmic}
\usepackage[]{algorithm2e}
\usepackage{booktabs}
\usepackage{adjustbox}
\usepackage{amsmath,mathtools}
\usepackage{graphics}
\usepackage{multirow}
\usepackage{xcolor}
\usepackage{subfig}
\usepackage{graphicx}
\usepackage{hhline}
\usepackage{makecell}
\usepackage{amssymb}
\usepackage{hyperref}
\usepackage{arydshln}
\usepackage{colortbl}
\usepackage{enumitem}
\usepackage{relsize}
\usepackage{courier}
\usepackage{xspace}
\usepackage{tikz}
\usepackage[utf8]{inputenc}

%\usepackage{dirtytalk}
%\usepackage[autostyle]{csquotes}
 % \MakeAutoQuote{‘}{’}
%\MakeOuterQuote{"}


\usepackage[noadjust]{cite}
\usepackage[super]{nth}


\renewcommand{\arraystretch}{1.2}
\newcommand{\msn}{\texttt{MSN30K}\xspace}
\newcommand{\istella}{\texttt{Istella-S}\xspace}
\newcolumntype{R}[1]{>{\raggedleft\let\newline\\\arraybackslash\hspace{0pt}}p{#1}}


%% \BibTeX command to typeset BibTeX logo in the docs
\AtBeginDocument{%
  \providecommand\BibTeX{{%
    \normalfont B\kern-0.5em{\scshape i\kern-0.25em b}\kern-0.8em\TeX}}}

\makeatletter
\def\adl@drawiv#1#2#3{%
        \hskip.5\tabcolsep
        \xleaders#3{#2.5\@tempdimb #1{1}#2.5\@tempdimb}%
                #2\z@ plus1fil minus1fil\relax
        \hskip.5\tabcolsep}
\newcommand{\cdashlinelr}[1]{%
  \noalign{\vskip\aboverulesep
           \global\let\@dashdrawstore\adl@draw
           \global\let\adl@draw\adl@drawiv}
  \cdashline{#1}
  \noalign{\global\let\adl@draw\@dashdrawstore
           \vskip\belowrulesep}}
\makeatother

%% Rights management information.  This information is sent to you
%% when you complete the rights form.  These commands have SAMPLE
%% values in them; it is your responsibility as an author to replace
%% the commands and values with those provided to you when you
%% complete the rights form.
%\setcopyright{acmcopyright}
%\copyrightyear{2018}
%\acmYear{2018}
%\acmDOI{10.1145/1122445.1122456}

%% These commands are for a PROCEEDINGS abstract or paper.
%\acmConference[Woodstock '18]{Woodstock '18: ACM Symposium on Neural
%  Gaze Detection}{June 03--05, 2018}{Woodstock, NY}
%\acmBooktitle{Woodstock '18: ACM Symposium on Neural Gaze Detection,
%  June 03--05, 2018, Woodstock, NY}
%\acmPrice{15.00}
%\acmISBN{978-1-4503-XXXX-X/18/06}
% \ifCLASSOPTIONcompsoc
%   % The IEEE Computer Society needs nocompress option
%   % requires cite.sty v4.0 or later (November 2003)
%   \usepackage[nocompress]{cite}
% \else
%   % normal IEEE
%   \usepackage{cite}

% \fi
\newcommand\MYhyperrefoptions{bookmarks=true,bookmarksnumbered=true,
pdfpagemode={UseOutlines},plainpages=false,pdfpagelabels=true,
colorlinks=true,linkcolor={black},citecolor={black},urlcolor={black},
pdftitle={Bare Demo of IEEEtran.cls for Computer Society Journals},%<!CHANGE!
pdfsubject={Typesetting},%<!CHANGE!
pdfauthor={Michael D. Shell},%<!CHANGE!
pdfkeywords={Computer Society, IEEEtran, journal, LaTeX, paper,
             template}}%<^!CHANGE!

%%
%% Submission ID.
%% Use this when submitting an article to a sponsored event. You'll
%% receive a unique submission ID from the organizers
%% of the event, and this ID should be used as the parameter to this command.
%%\acmSubmissionID{123-A56-BU3}

%%
%% The majority of ACM publications use numbered citations and
%% references.  The command \citestyle{authoryear} switches to the
%% "author year" style.
%%
%% If you are preparing content for an event
%% sponsored by ACM SIGGRAPH, you must use the "author year" style of
%% citations and references.
%% Uncommenting
%% the next command will enable that style.
%%\citestyle{acmauthoryear}

%%
%% end of the preamble, start of the body of the document source.
\begin{document}
%%
%% The "title" command has an optional parameter,
%% allowing the author to define a "short title" to be used in page headers.
\title{Distilled Neural Networks for\\Efficient Learning to Rank}
%%
%% The "author" command and its associated commands are used to define
%% the authors and their affiliations.
%% Of note is the shared affiliation of the first two authors, and the
%% "authornote" and "authornotemark" commands
%% used to denote shared contribution to the research.
\author{Franco Maria Nardini, Cosimo Rulli, Salvatore Trani, and Rossano Venturini
\IEEEcompsocitemizethanks{
\IEEEcompsocthanksitem Cosimo Rulli and Rossano Venturini are with the University of Pisa, Italy. E-mail: cosimo.rulli@phd.unipi.it, rossano.venturini@di.unipi.it.
% note need leading \protect in front of \\ to get a newline within \thanks as
% \\ is fragile and will error, could use \hfil\break instead.

\IEEEcompsocthanksitem  Franco Maria Nardini, Cosimo Rulli, Salvatore Trani and Rossano Venturini are with the ISTI–CNR, Pisa, Italy. E-mail: \{cosimo.rulli, f.nardini, s.trani, rossano.venturini\}@isti.cnr.it.}% <-this % stops a space
%\thanks{Manuscript received }
}
%\email{trovato@corporation.com}
%\orcid{1234-5678-9012}

%\author{G.K.M. Tobin}
%\authornotemark[1]
%\email{webmaster@marysville-ohio.com}
%\affiliation{%
  %\institution{Institute for Clarity in Documentation}
  %\streetaddress{P.O. Box 1212}
  %\city{Dublin}
  %\state{Ohio}
 % \postcode{43017-6221}
%}

%%
%% By default, the full list of authors will be used in the page
%% headers. Often, this list is too long, and will overlap
%% other information printed in the page headers. This command allows
%% the author to define a more concise list
%% of authors' names for this purpose.
%\renewcommand{\shortauthors}{Trovato and Tobin, et al.}

%%
%% The abstract is a short summary of the work to be presented in the
%% article.
\IEEEtitleabstractindextext{

% !TEX root = paper.tex
% !TeX spellcheck = en_US

\begin{abstract}
Recent studies in Learning to Rank have shown the possibility to effectively distill a neural network from an ensemble of regression trees. This result leads neural networks to become a natural competitor of tree-based ensembles on the ranking task. Nevertheless, ensembles of regression trees outperform neural models both in terms of efficiency and effectiveness, particularly when scoring on CPU. 
In this paper, we propose an approach for speeding up neural scoring time by applying a combination of Distillation, Pruning and Fast Matrix multiplication. We employ knowledge distillation to learn shallow neural networks from an ensemble of regression trees. Then, we exploit an efficiency-oriented pruning technique that performs a sparsification of the most computationally-intensive layers of the neural network that is then scored with optimized sparse matrix multiplication. Moreover, by studying both dense and sparse high performance matrix multiplication, we develop a scoring time prediction model which helps in devising neural network architectures that match the desired efficiency requirements. Comprehensive experiments on two public learning-to-rank datasets show that neural networks produced with our novel approach are competitive at any point of the effectiveness-efficiency trade-off when compared with tree-based ensembles, providing up to $4$x scoring time speed-up without affecting the ranking quality.
\end{abstract}


\begin{IEEEkeywords}
Web search, learning-to-rank, neural networks, efficiency, distillation, pruning, matrix multiplication.
\end{IEEEkeywords}
}

%%
%% The code below is generated by the tool at http://dl.acm.org/ccs.cfm.
%% Please copy and paste the code instead of the example below.
%%

%%
%% Keywords. The author(s) should pick words that accurately describe
%% the work being presented. Separate the keywords with commas.
%\keywords{datasets, neural networks, gaze detection, text tagging}

%%
%% This command processes the author and affiliation and title
%% information and builds the first part of the formatted document.
\maketitle
\begin{tikzpicture}[remember picture,overlay]
\node[anchor=south,yshift=5pt] at (current page.south) {\fbox{\parbox{\dimexpr\textwidth-\fboxsep-\fboxrule\relax}{
  \footnotesize{
     \copyright 2022 IEEE. Personal use of this material is permitted.  Permission from IEEE must be obtained for all other uses, in any current or future media, including reprinting/republishing this material for advertising or promotional purposes, creating new collective works, for resale or redistribution to servers or lists, or reuse of any copyrighted component of this work in other works.
  }
}}};
\end{tikzpicture}
\IEEEdisplaynontitleabstractindextext
\IEEEpeerreviewmaketitle

\vspace{-1.2cm}
% !TEX root = paper.tex
% !TeX spellcheck = en_US

\section{Introduction}
\label{sec:introduction}

\IEEEPARstart{T}{he} estimation of relevance is a task of paramount importance in Web search. In fact, search engines provide the users with a list of relevant results answering a information need formulated as a textual query. In the last years, Learning to Rank (LtR) techniques have been successfully applied to solve this task. LtR is the field of machine learning devoted to the development of supervised techniques addressing the ranking problem. LtR techniques have been proficiently used in Web search, a scenario characterized by tight latency bounds for query processing~\cite{cambazoglu2011scalability}. For this reason, the investigation of new LtR techniques targets both effectiveness and efficiency to provide accurate solutions that can be used in modern query processors. State-of-the-art approaches in learning to rank are ensembles of regression trees. Specifically, LambdaMART~\cite{burges2010ranknet} is an effective state-of-the-art LtR algorithm that builds ensembles of regression trees by optimizing a loss function that depends on a listwise information retrieval metric, e.g., NDCG~\cite{jarvelin2002cumulated}. The counterpart of the retrieval accuracy guaranteed by tree-based models is the computational effort needed to traverse hundreds or even thousands of trees. This computational effort hinders the application of this kind of models on low-latency query processors. Furthermore, each tree in an ensemble work by testing a sequence of boolean conditions on the input. The natural translation of this structure in \textit{if-then-else} code conflicts with modern CPU architectures that heavily rely on \textit{branch prediction} and \textit{caching}. A recent line of research investigates techniques for efficient traversal of ensembles of regression trees. The state-of-the-art algorithm for traversing tree-based models is QuickScorer~\cite{lucchese2015quickscorer,dato2016fast,8035185,lucchese2016exploiting}, which implements an interleaved feature-wise traversal of the ensemble that maximizes the efficiency of branch predictor and cache of modern CPUs.

Motivated by the success of neural solutions in other fields such as Natural Language Processing and Computer Vision, several attempts have been made to bring Neural Networks (NNs) in the LtR field. Despite that, tree-based solutions still provide state-of-the-art performances on different benchmarks, especially when dealing with handcrafted features~\cite{qin2020neural}. 
Recently, Qin \emph{et al}~\cite{qin2020neural} identify the reasons for the superiority of tree-based solutions in i) the sensitiveness of neural network to input features scale and transformations, ii) the lack of expressiveness in mostly adopted neural models in LtR, iii) the limited size of available LtR datasets w.r.t. to Natural Language Processing or Computer Vision. Cohen \textit{et al.}~\cite{cohen2018universal} develop an approach that permit to overcome these limitations on standard LtR datasets by training classic multi-layer perceptrons using simple data normalization ($Z$-normalization) and by leveraging a data augumentation technique (Section \ref{sec:cohen}). 
%\cosimo{One of the inherent difficulties of ranking is that evaluation metrics are non-differentiable, hindering the usage of Stochastic Gradient Descent (SGD), the optimization algorithm used for neural network training. In general, differentiable proxies of ranking metrics are employed to train machine learning models.}
Cohen \textit{et al.}~\cite{cohen2018universal} propose to train neural networks to mimic the outputs of a pre-trained ensemble of regression trees. They do so by employing a knowledge distillation approach~\cite{ba2014deep,DBLP:journals/corr/HintonVD15} that treats the ensemble of regression trees as a black box generating accurate document scores.
Given that neural models are universal approximators~\cite{hornik1991approximation}, the network can reproduce the predictions of the ensemble of regression trees. In practice, this is done by using the Mean Square Error between the scores and the network predictions as training loss. The performance of a neural network trained by scores approximation are bounded by the performance of the tree-based model used to generate the scores: even in a perfect approximation scenario, the neural model will introduce no improvement in terms of effectiveness. In general, instead, the approximation will cause a degradation in the ranking precision. However, the reason to move to a neural document scoring engine is to exploit fast inference mechanisms available for NNs.
%\cosimo{In this direction, Cohen \textit{et al.}~\cite{cohen2018universal} compare the efficiency of a neural solution for ranking  with QuickScorer~\cite{lucchese2015quickscorer}.  }
In this direction, Cohen \textit{et al.}~\cite{cohen2018universal} compare the efficiency of a neural solution for ranking (on GPU and CPU) with QuickScorer~\cite{lucchese2015quickscorer} (on CPU).
In the original work, the authors claim that neural models are as accurate as ensembles of regression trees in terms of Mean Average Precision (MAP), and largely outperform them in terms of execution time ($\mu$s/doc). We observe that their comparison presents some weaknesses.
They compare a single-thread CPU version of QuickScorer against a multi-thread GPU version of the neural forward pass. Due to the differences between the computational engines, this does not permit to actually state which one of the two solutions is the more efficient.
 %They report the results of a CPU-based - single-thread - version of QuickScorer. We observe that comparing it with a multi-thread GPU version of the neural forward pass evaluates more the power of the computational engines than the efficiency of the algorithms.  
 Even when comparing on CPU, the comparison is done using: i) a single-threaded C++ implementation of QuickScorer for ensembles of regression trees and ii) a multi-threaded Python neural inference running with an unspecified number of threads. The use of Python APIs may also entail some latency in calling the underlying optimized matrix multiplication routine on which these frameworks usually rely.\footnote{See for example \url{https://scipy-cookbook.readthedocs.io/items/ParallelProgramming.html\#Use-parallel-primitives}} 
%\cosimo{They do not specify the number of threads used in the python version. Also, there may be a latency   }
Moreover, the two sets of experiments are conducted on different CPUs. These aspects hamper a direct comparison of the performance achieved.


In this article, we propose a solid, fair and comprehensive comparison of the efficiency of ensemble of regression trees and neural models.
We compare QuickScorer~\cite{dato2016fast} against a novel and optimized implementation of neural network inference written in C++. We perform the evaluation on the same hardware by executing the two solutions using a single thread. Moreover, both solutions exploit instruction-level parallelism (AVX2 instruction set). Since CPU and GPU are two different processing units and each of them requires specific optimization techniques, in this work we focus on providing an accurate study of the efficiency of the two approaches on CPU, while we plan to extend it to the GPU in the future. Regarding the training phase, we adopt the same neural architectures of Cohen \textit{et al.}~\cite{cohen2018universal} and we re-implement their methodology with our own code in Pytorch~\cite{NEURIPS2019_9015}. However, differently from the original work, in our experiments we train the ensemble of regression trees with the LightGBM library~\cite{NIPS2017_6907}, since it is the state-of-the-art library for learning ensemble models on ranking tasks~\cite{NIPS2017_6907,qin2020neural}.

\begin{table}[htb]
\centering
\begin{tabular}{llllr}	
		\toprule
		Model & NDCG@10   & NDCG&  MAP & \thead{ Scoring Time \\ ($\mu s$/ doc)} \\
		\midrule
		Large Forest & 0.5246\textsuperscript{$\star \dag$} & 0.7473\textsuperscript{$\star \dag$} & 0.6604\textsuperscript{$\star \dag$} & 8.2  	\\
		\cdashlinelr{1-5}
		Mid Forest & 0.5206\textsuperscript{$\dag$}&0.7454\textsuperscript{$\dag$} &  0.6582\textsuperscript{$\dag$}& 1.5 	\\
		Small Forest & 0.5181& 0.7438 &  0.6578& 0.8 \\
		\midrule
		Large Net & 0.5198\textsuperscript{$\dag$} & 0.7445\textsuperscript{$\dag$} & 0.6582\textsuperscript{$\dag$} & 24.4 \\
		Small Net & 0.5171 &0.7432 & 0.6575   & 2.2 \\
		\bottomrule
\end{tabular}
\caption{A comparison between QuickScorer and Neural Networks on the \msn dataset. 
Symbols evidence statistically significant improvement w.r.t. to Mid Forest (\textsuperscript{$\star$}), and Small Forest (\textsuperscript{$\dag$}),  according to the Fisher's randomization test,  $p < 0.05$.}
\label{tab:firsttab}
\end{table}

%\ftodo{tabella sopra. perche' large forest ha simbolo di stat sig da sola? FM}

The results of our comprehensive experimentation on the \msn dataset show that, in contrast with the results reported by Cohen \textit{et al.}~\cite{cohen2018universal}, ensembles of regression trees are both faster and more accurate than neural models. In Table \ref{tab:firsttab}, we report the Mean Average Precision (MAP), the Normalized Discounted Cumulative Gain (NDCG, with cutoff at 10 and without cutoff), and the scoring time per document.
 %Moreover, models associated with the same symbol ($\star$, \dag, $\ast$) are statistically equivalent, according to the Fisher's randomization test,  $p < 0.05$. 
%We observe that the performance of the models with the same symbol ($\star$, \dag, $\ast$) are statistically equivalent. 
Symbols evidence statistically significant improvement w.r.t. to Mid Forest\textsuperscript{$\star$}, and Small Forest\textsuperscript{$\dag$}, according to the Fisher's randomization test,  $p < 0.05$. We run different tests for each metrics, but we use shared symbols to ease the notation.
Table \ref{tab:firsttab} shows that ensemble of regression trees deliver the same performance of neural models while being largely faster, with a speedup ranging from $2.8$x (Small Net vs Small Forest) to $16.2$x (Large Net vs Mid Forest). Also, the Large Forest is the best performing model with a large margin, while being $3$x faster than the Large Net.
 %\textit{i.e.}, \textit{Large Forest} and \textit{Large Net} respectively. Moreover, when considering a given NDCG@10 value fixed by the neural network, tree-based models result consistently faster, with a speedup ranging from $2.8$x (Small Net vs Small Forest) to $16.2$x (Large Net vs Mid Forest). 
 These evidences highlight how tree-based solutions are currently faster than neural networks on CPU. We bridge the large gap between tree-based models and neural networks by proposing a novel framework to efficiently design and train effective and efficient feed-forward networks for ranking on CPU.

The novel contributions of this article are:
\begin{itemize}
\item we present a combination of state-of-the-art approaches to improve the performance of neural networks on Learning to Rank tasks. By leveraging efficiency-oriented pruning techniques and high-performance Dense and Sparse Matrix Multiplication techniques, we build neural models that outperform ensembles of regression trees. An extensive experimental evaluation on two well-established public benchmarks, \textit{i.e.}, the \msn~\cite{DBLP:journals/corr/QinL13} and the Tiscali \istella~\cite{dato2016fast} datasets, shows the effectiveness of our method. Experimental results confirm that on the \msn dataset it is possible to obtain up to $4.4$x faster scoring time with no loss of accuracy.

\item we provide a novel way to estimate the execution time of neural network forward pass, by mean of dense and sparse time predictors, respectively for Dense-Dense and Sparse-Dense Matrix Multiplication (DMM \& SDMM). To the best of our knowledge, this is the first work that dives into the technicality of matrix multiplication to precisely predict the execution time of neural models.
These predictors are derived from a broad study of the implementation of the relative operations on modern CPUs. In explaining how predictors are developed, we also provide a clear and concise explanation of these two fundamental operations with plenty of scientific applications. 

\item we develop an efficient and effective approach to design neural models, using the aforementioned time predictors, which allow to estimate the execution time of a feed-forward network \textit{a priori}, by providing the architecture - \textit{i.e.,} the number of layers and the neurons per layer - and the sparsity level of each layer. This design methodology tackles the costly problem of model architectures search~\cite{strubell2019energy,patterson2021carbon}, since it allows to train \emph{exclusively} the models respecting the latency requirements, tearing down the costs, in terms of time and energy consumption, of the experimental phase.
\end{itemize}

The rest of the paper is organized as follows: Section~\ref{sec:related} discusses the related work in the field. Section~\ref{sec:cohen} details the process of distilling ensemble of regression trees into neural networks as proposed by Cohen \emph{et al.}~\cite{cohen2018universal}. Section~\ref{sec:ModelMatMult} introduces the implementation of dense-dense matrix multiplication and sparse-dense matrix multiplication on modern CPUs, together with our time predictors. Section~\ref{sec:neuraleng} describes our novel method for designing efficient neural models for ranking. Moreover, Section~\ref{sec:experiments} presents a comprehensive experimental evaluation of our proposed technique on public data. Finally, Section~\ref{sec:conclusions} concludes the work.


% !TEX root = paper.tex
% !TeX spellcheck = en_US

\section{Related Work}
\label{sec:related}

In this section, we introduce Learning to Rank (LtR) and its use in Information Retrieval (IR). Then, we describe QuickScorer~\cite{lucchese2015quickscorer,dato2016fast,8035185} an efficient algorithm for scoring ensemble of regression trees. Finally, we discuss the field of model compression, a branch of machine learning that aims to compress Deep Neural Networks without affecting their accuracy. Here, we focus our attention in particular on pruning techniques.

% subsection subsection_name (end)}
\subsection{Learning to Rank}
\label{subsec:ltr}
Learning to Rank (LtR) consists in applying machine learning techniques to the problem of ranking documents with respect to a query.
%such as BM25~\cite{robertson2009probabilistic} and Query Likelihood. 
RankNet~\cite{burges2005learning} leverages a  probabilistic ranking framework based on a pairwise approach to train a neural network. The difference between the predicted scores of two different documents is mapped to a probability by means of the sigmoid function. Hence, using the cross-entropy loss this probability is compared with the ground truth labels, and Stochastic Gradient Descent (SGD) is used to minimize this loss. FRank~\cite{tsai2007frank} exploits a generative additive model and substitutes the cross-entropy loss with the fidelity loss, a distance metric adopted in physics, superior to cross-entropy when applied on top of the aforementioned probabilistic framework since 1) has minimum in zero, 2) is bounded in $[0,1]$. 
%proposes a probabilistic cost function framing the ranking problem into a \emph{pairwise} approach. 
Neither RankNet nor FRank directly optimize a ranking metric (\emph{e.g.}, NDCG), and this discrepancy weakens the power of the model. Since ranking metrics are flat and discontinuous, coding them into the loss function is troublesome.
To overcome this issue, LambdaRank~\cite{burges2007learning} heuristically corrects the RankNet gradients, exploiting the rank position of the document in the overall sorting: it multiplies the RankNet gradient with a term that measures the increase in terms of NDCG when switching the terms, generating the so-called $\lambda$-gradients.
%	Multiple Additive Regression Trees (MART) have shown remarkable results for the ranking problem. 
McRank~\cite{li2008mcrank} casts the problem of ranking as MultiClass classification task, using a boosting tree algorithm to learn the class probabilities and then converting them into relevances with the expected relevance, outperforming LambdaRank.  This work also highlights that modeling the ranking problem as a classification task works better than modeling it as a regression one.  
LamdaMART~\cite{burges2010ranknet} combines the successful training methodology provided by $\lambda$-gradients with Multiple Additive Regression Trees (MART) - as McRank~\cite{li2008mcrank}, and it has been establishing as the state-of-the-art in LtR. Currently, ensembles of regression trees are the most effective solution among LtR techniques when dealing with handcrafted features. In the next section, we describe state-of-the-art approaches for efficient traversal of these trees, in order to employ them in latency-bound scenarios.


\subsection{Efficient Traversal of Tree-based Models}
\label{subsec:quickscorer}
QuicksScorer~\cite{lucchese2015quickscorer} is a state-of-the-art algorithm that allows to speedup the traversal of an ensemble of regression trees. As detailed in the previous section, ensemble of regression trees is the model exploited by several state-of-the-art learning-to-rank solutions, e.g., LambdaMART~\cite{burges2010ranknet}.
QuickScorer codes each tree of the ensemble as a bitvector of length $n$, where $n$ is the number of leaves, which is used to select the \textit{exit leaf} in the tree. Furthermore, each decision node in each tree is associated with a bitvector of the same length called \textit{mask}. If the corresponding test is evaluated to false, the bits corresponding to the unreachable leaves are set to zero. By performing the logical \texttt{AND} among all the masks, we obtain another bitvector, named \emph{leafidx}, in which the first one entry corresponds to the exit leaf. To efficiently compute the exit leaf, QuickScorer process all the nodes in a \textit{feature by feature} fashion. For each feature $f$, the associated thresholds among all the nodes in the forest are sorted in ascending order. Let us a consider a threshold $\gamma$ associated with a node $g$: when $x_f > \gamma$, 
the corresponding \emph{leafidx} is updated performing the \texttt{AND} operation with the \emph{mask} relative to $g$. Since the thresholds are sorted, as soon as $x_f \leq \gamma$, the evaluation of the current feature is interrupted, since the following instances will evaluate true as well. To further improve the efficiency of the algorithm, two variations of the original algorithm are introduced:  1) Block-Wise QuickScorer (BWQS), in which the forest is partitioned into blocks of trees fitting the L3 cache, reducing the cache-miss ration and 2) Vectorized QuickScorer (vQS)~\cite{lucchese2016exploiting}, in which scoring is vectorized using AVX2 instructions and 256-bit registers, allowing to process up to $8$ document at time. 
Lettich \emph{et al.}~\cite{8035185} propose a GPU version of QuickScorer, to exploit the massive parallelism of this computational engine. By properly managing the GPU memory hierarchy and furnishing an adequate degree of parallelism in the document scoring process, this version results up to 100x faster than the corresponding CPU version, when dealing with very large forests ($20$,$000$ trees).
 
The cost of traversing an ensemble of regression trees with QuickScorer depends on the number of false nodes, rather than on the length of the root-to-leaf paths. Since machine-learnt trees are imbalanced, the authors experimentally show that this reduces the percentage of nodes to evaluate from 80\% of classical traversal to the 30\% of QuickScorer~\cite{lucchese2015quickscorer}. Moreover, QuickScorer is implemented carefully taking into account cache and CPU issues. For example, QuickScorer structures are accessed sequentially thus favoring pre-fetching and avoiding branch mispredictions. However, when the number of leaves is larger than $64$, scoring a model with QuickScorer can be inefficient. Recently, RapidScorer tackles the problem of forest with a larger number of leaves~\cite{ye2018rapidscorer}. In fact, when $|\text{leaves}| > 64$, the logical \texttt{AND} between the bitvectors cannot be carried out in just one CPU instruction, hampering efficiency. For this reason, RapidScorer introduces a tree-size insensitive encoding, named \emph{epitome}. Moreover, it leverages a node merging strategy that evaluates just once nodes sharing the same threshold on the same feature. By doing so, RapidScorer outperforms  QuickScorer when dealing with a large number of leaves.

\subsection{Model Compression}
\label{subsec:modelcompr}
The effectiveness of Deep Neural Networks (DNNs) comes at the cost of a high computational complexity~\cite{nnstats}, hindering the deployment and the usage of DNNs, especially for resource-constrained devices. An inherent feature of DNNs is \textit{over-parameterization}, \textit{i.e.,} the redundancy of networks parameters: it has been proven that the same performance can be obtained with just a portion of the original parameters~\cite{denil2013predicting}. Model Compression (MC) is a recent research field investigating effective techniques for reducing the memory impact of DNNs, their inference time, and energy consumption without affecting their accuracy, exploiting over-parameterization. 
%TODO mancherebbe low rank decomposition
In MC techniques, we observe the presence of several lines of research: pruning~\cite{DBLP:journals/corr/HanPTD15,DBLP:journals/corr/LiKDSG16, molchanov2019pruning,DBLP:journals/corr/HanMD15,DBLP:journals/corr/GuoYC16,yu2017scalpel,vieira2017learning,he2017channel,he2018amc}, quantization~\cite{DBLP:journals/corr/LiL16,DBLP:journals/corr/ZhuHMD16, DBLP:journals/corr/RastegariORF16,hubara2017quantized,Cai_2017_CVPR,DBLP:journals/corr/ZhouYGXC17} design of efficient architectures~\cite{DBLP:journals/corr/IandolaMAHDK16,zhang2018shufflenet,howard2017mobilenets,sandler2018mobilenetv2}, knowledge distillation~\cite{bucilua2006model,ba2014deep,DBLP:journals/corr/HintonVD15}. 

%($17.7x$, AlexNet on ImageNet, ).
%Network quantization techniques reduce the number of bits necessary to represent each weight thus generating lighter models providing faster inference and lower energy consumption~\cite{sze2017efficient}. Here, data-driven methods introduce a limited loss of accuracy even when using $1$ or $2$ bits~\cite{DBLP:journals/corr/LiL16, DBLP:journals/corr/ZhuHMD16,DBLP:journals/corr/RastegariORF16,hubara2017quantized,Cai_2017_CVPR}. Moreover, the incremental learning of the optimal quantization outperforms the original model performance~\cite{DBLP:journals/corr/ZhouYGXC17}.
%An orthogonal approach is the design of new and efficient architectures by modifying existing layers~\cite{DBLP:journals/corr/IandolaMAHDK16} or creating new ones~\cite{DBLP:journals/corr/ZhangZLS17, howard2017mobilenets,sandler2018mobilenetv2,DBLP:journals/corr/abs-1807-11164}. Automatizing the design and compression of neural architectures by mean of reinforcement learning and evolution techniques have proven to be effective~\cite{tan2019efficientnet,cai2018proxylessnas,tan2019mnasnet}
%Knowledge distillation techniques~\cite{DBLP:journals/corr/HintonVD15} exploit a \textit{teacher} network at training time to improve the learning process of a shallower and faster \textit{student} network. Finally, the combination of different compression techniques has proven to be an effective solution~\cite{DBLP:journals/corr/HanMD15,polino2018model}
%Pruning techniques belong to the family of Model Compression (MC) methods. MC is a recent research field investigating effective techniques for reducing the memory impact of DNNs, their inference time, and energy consumption without affecting their accuracy. Lossless compression is allowed by the significant redundancy of parameters, i.e., over-parametrization, that has been discovered in neural networks~\cite{DBLP:journals/corr/DenilSDRF13}. 
Recently, pruning has shown to be extremely effective~\cite{DBLP:journals/corr/HanPTD15,DBLP:journals/corr/LiKDSG16,DBLP:journals/corr/HanMD15,DBLP:journals/corr/GuoYC16,luo2017thinet,huang2018data,yu2017scalpel,vieira2017learning,he2017channel,he2018amc}. Pruning techniques delete useless connections in a pre-trained model, producing sparse weight tensors that are lighter to store and allow for faster inference time. Performing a retraining after pruning avoids accuracy loss, even in the case of high compression factors~\cite{DBLP:journals/corr/GuoYC16}. 
The canonical classification of pruning techniques divide them into two families: 1) element-wise pruning, which sets to zero individual weights, generating sparse weight tensors and 2) structured pruning, which prunes entire groups of weights, \textit{i.e.,} columns, filters, or even entire layers. In the latter case, the resulting network's weights still belong to the dense domain. 
%Structured pruning for feed-forward network is applied columns-wise. Once estimated the importance of each column by mean of an heuristic (\textit{e.g.,} $L_1$-norm),  a percentage of less important columns is removed from the model. Given an original model architecture with $l$ layers $A_l = \{l_1, \dots , l_{l}\}$, we obtain a reshaped dense model $A_l^p =  \{l_1^p, \dots, l_l^p \}$. 
%We experimentally verified that it makes no difference whether the $A_l^p$ model is trained from scratch or obtained from a pruning-finetuning procedure on a pre-trained model, as happens in models for Image Classification tasks~\cite{liu2018rethinking}.   
%MC techniques can be grouped in four main lines of research: \textit{pruning}~\cite{DBLP:journals/corr/HanPTD15,DBLP:journals/corr/LiKDSG16, DBLP:journals/corr/MolchanovTKAK16, ullrich2017soft,DBLP:journals/corr/HanMD15,DBLP:journals/corr/GuoYC16,DBLP:journals/corr/YangCS16a,DBLP:journals/corr/LuoWL17,huang2018data,wen2016learning,yu2017scalpel,vieira2017learning,he2017channel,he2018amc,prakash2018repr}, \textit{quantization}~\cite{DBLP:journals/corr/LiL16,DBLP:journals/corr/ZhuHMD16, DBLP:journals/corr/RastegariORF16,hubara2017quantized,DBLP:journals/corr/ZhouYGXC17} \textit{design of efficient architectures}~\cite{DBLP:journals/corr/IandolaMAHDK16,DBLP:journals/corr/ZhangZLS17,howard2017mobilenets,sandler2018mobilenetv2}, \textit{knowledge distillation}~\cite{DBLP:journals/corr/HintonVD15,romero2014fitnets,chen2015net2net}. 
In this paper, we focus on element wise-pruning techniques. These methods employ heuristics to determine what are the relevant weights of the network. In particular, \textit{magnitude-based} heuristics work by removing low absolute-value weights and are proved to be effective~\cite{DBLP:journals/corr/HanPTD15,DBLP:journals/corr/GuoYC16}. In their na\"ive version, magnitude based approaches remove a fixed percentage of weights from the original model (\emph{level pruning}). Han \emph{et al.} show that the gradual increase of the target sparsity, interleaved with a number of steps of re-training, can improve the accuracy of the final model~\cite{DBLP:journals/corr/HanPTD15}. Furthermore, they propose a layer-wise threshold-based method to determine whether a parameter shall be kept or not. For each layer, its threshold $t_i$ is computed as as $t_i = \sigma_i * s_i$, with $\sigma_i$ the standard deviation of weights distribution and $s_i$ a sensitivity parameter to be chosen. By assuming that parameters follow a Normal distribution $ \mathcal{N}(0, \sigma^2) $, setting $s_i = 1$ would approximately prune away about the 68\% of the weights. The pruning step is followed by a number of re-training epochs on the surviving weights. The procedure can be then iterated by gradually increasing $s_i$ thus inducing higher sparsity. The Distiller Framework~\cite{nzmora2019distiller} version that we adopt, keeps this threshold fixed, relying on the fact that as the tensor is pruned, more elements are pulled towards the center of the distribution and then pruned. Pruning techniques have shown to be able to sparsify state-of-the-art neural architectures up to 90\%, thus strongly reducing their memory burden and easing the transmission and deployment on resource-constrained devices.



\section{Training by Scores Approximation}
\label{sec:cohen}
In this section, we detail the methodology proposed by Cohen \textit{et al.}~\cite{cohen2018universal} to train neural models approximating ensembles of regression trees.
Their technique can be considered as a special case of Knowledge Distillation~\cite{ba2014deep,DBLP:journals/corr/HintonVD15}.
 %Observe that learning the outputs of another model instead that directly the ground truth is locate their work close in the knowledge distillation area~\cite{ba2014deep,DBLP:journals/corr/HintonVD15}.
Knowledge distillation is a training technique in which a small \emph{student} model is trained to mimic the outputs of a large and expressive \emph{teacher} model.
In the case of Cohen \emph{et al.}, the ensemble of regression trees plays the role of the teacher, while the neural network is the student model.
%We now describe the approach by Cohen \textit{et al.}~\cite{cohen2018universal} in detail.
The core idea of their approach is to treat the tree-based model as a black box producing accurate scores. Formally, let us consider a Learning to Rank dataset $D = (X, Y)$,  $X \in \mathbb{R}^{f \times |D|}$, where $f$ is the number of extracted features per document, $|D|$ is the cardinality of the dataset, and $Y \in \mathbb{N}^{|D|}$ is the set of ground-truth relevances of a document w.r.t. a query.
Let $F: \mathbb{R}^{f} \rightarrow \mathbb{R}$ be the underlying function learned by an ensemble of regression trees during the training that maps a single document $x \in X$ into a relevance score.
If the neural model can reproduce the function $F$, it achieves the same ranking quality as the original model. The effectiveness of this approach relies on theoretical results showing that NNs can approximate continuous~\cite{hornik1991approximation} and piecewise continuous functions~\cite{llanas2008constructive}. In practice, the approximation is implemented by using the \emph{Mean Squared Error} as loss function computed between the network prediction and the ensemble prediction. Furthermore, the training procedure is enriched with a data augmentation step which enforces the approximation capabilities of the neural network. Consider the set of $f$ features in the dataset. For each feature, Cohen \emph{et al.}~\cite{cohen2018universal} build a list composed of 
the split points corresponding to that feature in the ensemble of regression trees, and, in the same list, they also put the maximum and the minimum for that feature in the training set.
This way they obtain a set of $f$ lists, where $f$ is the number of the features in the dataset. Each of these lists is then sorted, and replaced with its ordered midpoints, \emph{e.g.}, each adjacent pair $\{x_i, x_{i+1}\}$ is replaced with its midpoint, $\frac{x_i + x_{i+1}}{2}$.  At each training step, half of the training data is built by randomly sampling from this feature-wise set of lists to have a better coverage of the whole feature space. Before feeding them to the network, all the training data are normalized by subtracting the mean and by dividing by the variance ($Z$-normalization).
This approach is more proficient than directly learning the ground-truth relevance~\cite{cohen2018universal}. As detailed in Section \ref{sec:introduction}, the approximation error introduced is small but statistically significant in terms of ranking quality. In Section~\ref{sec:neuraleng}, we show how to mitigate this effect.

% !TEX root = paper.tex
% !TeX spellcheck = en_US

\section{Modeling Matrix Multiplication}
\label{sec:ModelMatMult}
In this section, we detail the optimization of matrix multiplication on modern CPUs. We start with the implementation of dense-dense matrix multiplication (DMM) and then we move to the sparse-dense (SDMM) matrix case. Matrix multiplication has a prominent role in a wide spectrum of scientific applications (linear algebra, physics, economics, engineering), and it also represents the structural operation in neural network forward and backward pass. We believe that, when dealing with the efficiency-effectiveness trade-off, a comprehensive analysis of the underlying multiplication mechanisms is essential. We develop time predictors for matrix multiplication both in the dense and in the sparse domain, and we then jointly apply them to develop an analytical model that estimates the scoring time of a neural network given the matrix shapes and the sparsity percentage of each layer of the Feed Forward Network (FFN). Our predictors are analytic, \textit{i.e.}, not learned, and they are based on 1) the knowledge gained from the implementation of DMM and SDMM on modern CPU architectures, 2) empirical measurements showing the performance of CPU on these operations under different conditions. 
 We observe that, by exploiting the predictors we are proposing, we are allowed to train only the architectures that match the desired efficiency constraints. In a latency-bound application, the efficiency constraints are specified in the requirements. In an effectiveness-oriented context, they can be inferred by observing the execution time of the competitor, \textit{i.e.}, ensembles of tree-based models. As a consequence, the use of our predictors allows to significantly reduce the search space of the optimal architecture. Furthermore, our predictors are task-agnostic, hence they can be applied in any Feed Forward Network (FFN) application field.

%It is worth noting that the time predictors that we develop can have implications that go beyond this ranking-oriented use case and possibly generalize to all the FFN application fields.
%\fnote{non ho capito ultima frase. spiega meglio. FM}
%Todo quello che consente di fare il modello matematico per la predizione dei tempi potrebbe essere una lista dentro un enumerate.. per adesso me ne  é venuto in mente uno solo

\subsection{Dense Matrix Multiplication}
\label{subsec:dmm}
In this section, we investigate how Dense Matrix Multiplication (DMM) is optimized on modern CPUs. DMM has countless applications, hence lots of effort has been spent to attain fast implementations. The current state-of-the-art algorithm for DMM is the the well-known Goto Algorithm~\cite{goto2008anatomy}, on which are based several open (GotoBLAS~\cite{goto2008anatomy}, OpenBLAS~\cite{xianyi2012openblas}, BLIS~\cite{huang2016blislab}) or commercial (Intel MKL~\cite{wang2014intel}) implementations.
%During the last 30 years, a major effort has been made in developing CPU-oriented optimizations, exploiting cache hierarchy, vector instructions and pre-fetching. The main precepts are contained into the third level of the Basic Linear Algebra Subprograms (BLAS) library; these precepts are based on the well-known Goto Algorithm~\cite{goto2008anatomy}, which is also the established state of the art for dense matrix multiplication. In fact,  
%The world famous Basic Linear Algebra Subprograms (BLAS)~\cite{lawson1979basic}, for example, has its own 3-level routines entirely dedicated to matrix-matrix operations. 
%several open (GotoBLAS~\cite{goto2008anatomy}, OpenBLAS~\cite{xianyi2012openblas}, BLIS~\cite{huang2016blislab}) or commercial (Intel MKL~\cite{wang2014intel}) implementations are now available, all developed according to the Goto algorithm for blocked matrix multiplication~\cite{goto2008anatomy}.

The multiplication of two $ n \times n$ dense matrices involves $\mathcal{O}(n^3)$ floating-point operations with $\mathcal{O}(n^2)$ data, as can be easily evicted from Equation~\ref{eq:mmdef}. In modern processors, the interaction with memory is more time-consuming than the  computation itself (\textit{memory bandwidth bottleneck}), but a wise memory management allows to amortize the data movement over a large number of computations.
The mathematical definition of matrix multiplication is the following: given  $A \in \mathbb{R}^{m \times k}$, $B \in \mathbb{R}^{k \times n}$, the matrix multiplication binary operator computes $C = A*B$  with $C \in \mathbb{R}^{m \times n}$, where every element of $C$ is given by

\begin{equation} \label{eq:mmdef} 
C_{i,j} = \sum_{p=1}^{k} A_{i, p}  B_{p,j} \qquad  i=1, \dots, m \quad j=1, \dots, n 
\end{equation}
The Goto Algorithm consists of iteratively decomposing the overall DMM into a series of smaller matrix operations in a cache-aware fashion, until matrices fit the CPU registers. Then matrices are multiplied by means of a highly engineered \textit{micro-kernel}. We now provide a breakdown of the Goto Algorithm as implemented in the BLIS library~\cite{lawson1979basic,van2015blis}, which assumes the CPU to be equipped with 3 levels of cache and vectorized instructions. The first three steps of the blocked matrix multiplication algorithm are depicted in Figure~\ref{fig:gotofirst}.

\begin{figure}[htb]
	\centering
	\includegraphics[width=\columnwidth]{imgs/Goto_first.pdf}
	\caption{First three steps of the Goto algorithm for Dense Matrix multiplication.}
		\label{fig:gotofirst}
\end{figure}

The blocked matrix multiplication algorithm begins by partitioning along the columns of $C$ and $B$ into blocks of size $n_c$, obtaining  sub-matrices of $C$ of shape $m \times n_c$ and sub-matrices of $B$ of shape $k \times n_c$. Each $C$ sub-matrix is obtained by multiplying the complete $A$ matrix with the corresponding sub-matrix of $B$. Then, the procedure partitions the columns of $A$ and the rows of $B$ into blocks of size $k_c$, to obtain $A_p$, \textit{i.e.}, vertical panels of size $m \times k_c$, and $B_i$, \textit{i.e.}, horizontal panels of size $k_c \times n$. The $B_i$ panels are packed into the L3 cache reordering data according to a specific pattern which allows to access data contiguously even after the subsequent partitions.  We adopt the notation $\tilde{X}$ to indicate that the sub-matrix $X$ respects this pattern. Observe that, after the blocking on the $k$ axis, the original multiplication is boiled down into a series of rank-k updates so that $C = C + A_p B_p$. A further partition is performed along rows of $A$, with size $m_c$, generating $C_i$ and $A_i$. $A_i$ is, as was $B_i$ previously, packed into $\tilde{A_i}$ in the L2 cache.

% \begin{figure}[htb]
% 	\centering
% 	\includegraphics[width=\columnwidth ]{imgs/Goto_second.pdf}
% 		\caption{Macro-Kernel in the Goto algorithm for Dense Matrix Multiplication (DMM).}
% 		\label{fig:gotosecond}
% \end{figure}

%\fnote{sistema caption della figura sopra. metti sempre i punti alla fine delle caption. controlla ovunque. FM}

\noindent \textbf{Macro-Kernel}. The macro-kernel, or inner kernel as in the original algorithm by Goto \textit{et al.}~\cite{goto2008anatomy}, is responsible for orchestrating the memory movement between the RAM memory and the caches. Let us consider the operation $C_i \leftarrow C_i +  \tilde{A}_i*  \tilde{B}_p $, with $C_i$ of size $m_c \times n$, $\tilde{A}_i$ of size $m_c \times k_c$ and $\tilde{B}_p$ of size $k_c \times n$. The macro kernel decomposes this operation into a series of block-panel multiplications, as shown in Figure~\ref{fig:gotosecond}. As aforementioned, both $\tilde{A}_i$ and $ \tilde{B}_p$ are packed with a special pattern, indicated by the arrows in Figure~\ref{fig:gotosecond}. In particular, $\tilde{A}_i$ is organized into sub-matrices $\tilde{A}_j$ of size $m_r \times k_c$, with elements stored in column-major order, while $ \tilde{B}_p$ is organized in panels of size $k_c \times n_r$, stored in row-major order, named $\tilde{B}_j$. This data access pattern reflects the order in which the micro-kernel accesses data. 

Goto \textit{et al.}~\cite{goto2008anatomy} observed the advantages of packing $\tilde{A}_i$ into the L2 cache. The ratio between FLOPs and memory operations, regardless if the original data rely in L3 cache or in main memory, can be modeled as
 $$ \frac{2 m_{c} k_{c}}{\left(2 m_{c}+k_{c}\right)} $$
if $k_c << n$.
Hence, the higher is the $m_c k_c$ product, the smaller is the overhead of memory transfer on the overall computation. 
Knowing that L2 cache is larger than L1, we can afford larger $m_c$ and $k_c$ values \footnote{In the original work, Goto et al.~\cite{goto2008anatomy} point out that $C_i \leftarrow C_i +  \tilde{A}_i  \tilde{B}_p $ should be computed at the peak rate of CPU. This condition is true if all three matrices reside in L1 cache, but it can be considered true even if $\tilde{A}_i$ is in L2.}.


\begin{figure}
	\centering
	\includegraphics[width=0.95\columnwidth ]{imgs/Goto_second.pdf}
		\caption{Macro-Kernel in the Goto algorithm for Dense Matrix Multiplication (DMM).}
		\label{fig:gotosecond}
\end{figure}



\noindent \textbf{Micro-Kernel}. The micro-kernel is the core operation of blocked matrix multiplication and the speed of the whole routine largely depends on the speed of this kernel. For this reason, in high-performance libraries, the micro-kernel is often written in assembly language, to exploit vectorized instructions and hand-tuned data pre-fetching~\cite{van2015blis}.
The micro kernel computes $c_{r,j} = c_{r,j} + \tilde{A}_j \tilde{B}_j$, where $\tilde{A}_j$ is an horizontal micro-panel of $\tilde{A}_i$ and $\tilde{B}_j$ is a vertical micro-panel of $\tilde{B}_p$, residing, respectively, in L2 and L1 cache, as reported in Figure~\ref{fig:gotothird}. The operation is performed as $k_c$ rank-1 updates, by computing the outer product between a column of  $\tilde{A}_j$ and a row of $\tilde{B}_j$ and by accumulating the results into the $m_r \times n_r$ $c_{r,j}$ submatrix. In this way, $c_{r,j}$ can be kept in CPU registers until the loop over $k_c$ is , allowing to move data from the registers to the memory just once. This means that $2m_c n_c k_r$ FLOPs can be performed with just $m_r n_r$ memory operations. Furthermore, this data reading pattern benefits from the data packing performed in the previous loops. In fact, columns and rows of $\tilde{A}_j$ and $\tilde{B}_j$ respectively will be accessed contiguously, which is generally known to be faster than accessing non-in-stride memory locations~\cite{low2016analytical}. In conclusion, pre-fetching instructions that load successive entries of $\tilde{A}_j$ and $\tilde{B}_j$ are interleaved with instructions performing the rank-1 update. This allows to mask the latency of the caches with the computation time of the CPU.


\noindent \textbf{Choosing the  kernel parameters}. Blocked matrix multiplication requires to determine a number of parameters $n_c, m_c, k_c, n_r, m_r$, controlling how the matrices are gradually decomposed. These parameters can differ from one processor to another, since they are influenced by hardware features such as the cache size or the number of SIMD registers. Choosing the optimal parameters for a given CPU architecture is a research problem, tackled for example by \textit{Low et al. }~\cite{low2016analytical}, which goes beyond the scope of this paper. Here, we want to list some general rules governing good choices for parameters. %We do this to provide a more detailed overview of the  Goto matrix multiplication algorithm. 
The micro-kernel is characterized by $m_r, n_r, k_c$. The values of $m_r$ and $n_r$ should be large enough so that the computation masks the latency of the caches. However, it should also allow to leave space in the registers for the next entries of $\tilde{A}_j$ and $\tilde{B}_j$. $k_c$ should be as large as possible, but must take into account the following constraints: 1) $k_c n_r$ entries from $\tilde{B}_j$ should fit the L1 cache 2) $m_c k_c$ entries from $\tilde{A}_i$ reside in the L2 cache. 
Moreover, cache replacement policies should also be taken into account. These policies control which data are kept and which are discarded from the levels of cache and may impact on the optimal macro-kernel values. A general solution is provided by Goto \textit{et al.}, who suggest choosing $k_c$ so that $\tilde{B}_j$ takes less than the half of the L1 cache~\cite{goto2008anatomy}.

\begin{figure}[t]
	\centering
	\includegraphics[width=0.8\columnwidth ]{imgs/Goto_third.pdf}
	\caption{Micro-Kernel in the Goto algorithm for Dense Matrix Multiplication (DMM).}
	\label{fig:gotothird}
\end{figure}

% The correct choice of these parameters
%strictly depends on the underlying hardware features.
%In the original work, Goto \textit{et al.} suggest to choose $k_c$ so that $\tilde{B}_j$ takes less of the half of the L1 cache~\cite{goto2008anatomy}. Cache replacement policies should also be taken into account. These policies control which data are kept and which are discarded from the levels of cache and may impact on the optimal macro-kernel values. \cosimo{This involves deep architectural fe}
% This is a non-trivial architecture-dependent problem which goes beyond the scope of this article.
%%\fnote{frase sopra. proverei a scriverla meglio spiegando perche' non la addressiamo. sembra che sia difficile e quindi ci importa una sega :)}
%
%Cache replacement policies should also be taken into account, especially in the $\tilde{B}_j$ case~\cite{goto2008anatomy,low2016analytical}, giving birth to a non-trivial architecture-dependent problem which goes beyond the scope of this article.
%\fnote{frase sopra non capisco. piu' chiara...}
Concerning the macro-kernel, we already discussed that the $m_c k_c$ product should be as large as possible. One of the key insights of the Goto algorithm is to consider the role of the Translation Look-Aside Buffer (TLB) in choosing the macro-kernel parameters. To hide the limits of random-access memories capacity (RAM), modern computing architectures use virtual memory. With this mechanism, the memory (RAM and hard disk) is partitioned into pages and a table, called \textit{page table}, keeps track whether a page is in memory or on disk. Scanning the page table entails additional overhead to check whether the requested page is on memory or disk.
Hence, the TLB, which is smaller than the overall \textit{page table}, keeps track of the most recently used pages: in case of a TLB \textit{hit}, the translation is fast. On the other side, in case of a TLB \emph{miss}, the complete \textit{page table} is checked and the new entry is moved to the TLB. Actually, the TLB has the same role as the cache and the \textit{hit/miss} dichotomy involves the same consequences. Thus, besides ensuring that $m_c k_c$ entries from $\tilde{A}_i$ fit the L2 cache, it is crucial that $\tilde{A}_i$, $n_r$ columns from $C_k$, and $n_r$ columns of $\tilde{B}_j$ are simultaneously addressable by the TLB, to avoid TLB misses during the block-panel multiplications of the macro-kernel. The only limit to the $n_c$ parameter is that $k_c n_c$ have to fit the L3 cache.

\subsection{Dense Neural Forward Pass Time Predictor}
\label{subsec:densetimepred}
In the previous section, we detail how Dense Matrix Multiplication is implemented on modern CPU architectures. We now show how the insights deriving from a deep understanding of matrix multiplication can be used to develop a time predictor for a Feed Forward Network (FFN) forward pass. We empirically demonstrate that even the highly engineered Goto algorithm  suffers when dealing with edge matrix dimensions. Hence, we leverage this intuition to build a hybrid analytical-empirical model for predicting dense matrix multiplication.
A FFN is composed of a  stack of \textit{fully connected} layers, where each neuron of layer $i$ is connected to all neurons of layer $i+1$. Each layer is composed of a weight matrix $W_i$, a bias vector $b_i$ and a non-linear \textit{activation function} $\sigma_i( \cdot )$. Let $x_i$ be the input to the $i$-th layer, the forward pass of layer $i$ is described by:
\begin{equation}
	\label{eq:mlpforward}
	x_{i+1} = \sigma_i(W^t_i x_i + b_i)
\end{equation}
where $x_{i+1}$ represents the output of the $i$-th layer. 
Hence, forwarding through a FFN layer consists of: 1) multiplying the input with the weight matrix, 2) summing the bias, 3) applying a non-linear activation function, usually ReLU or its variants. The overall forward pass on a FFN of $d$ layers has a cost, in terms of execution time, given by:
\begin{align}
 \label{eq:overallcost}
 	T = t_m \cdot ( f \cdot l_1 + \sum_{i=2}^{d} l_i   l_{i-1} + l_{d})
 	 + t_a \cdot \sum_{i=1}^{d} l_i + t_r \cdot \sum_{i=1}^{d} l_i \nonumber \\
 	   \simeq t_m \cdot ( f \cdot l_1 + \sum_{i=2}^{d} l_i \cdot  l_{i-1} + l_d) 
 \end{align}
where $t_m $ is the normalized time per multiplication, $t_a$ is the time for addition, $t_r$ is the time to perform the ReLU operation on a single neuron. As reported in Equation~\ref{eq:overallcost}, the time to perform matrix multiplication dominates the overall execution time, both in terms of number of operations and in terms of the complexity of the operation itself. We observe that $t_m$ can be inferred as:

\begin{equation}
	\label{eq:tm}
	t_m = \frac{1}{\text{GFLOPS}}
\end{equation}
The theoretical peak of GLOPs can be derived form the hardware specifications of the processor\footnote{https://software.intel.com/en-us/articles/a-simple-example-to-measure-the-performance-of-an-intel-mkl-function}. However, real performance can be significantly different from the theoretical ones, especially when facing limit cases, such as narrow or wide matrices. To include these cases into our evaluation, we develop a prediction model to measure the performance of a specific neural networks architecture.

Among the different instantiations of the BLAS library, we choose \textit{oneDNN}\footnote{\url{https://github.com/oneapi-src/oneDNN}},
a C++ high-performance framework for deep learning primitives developed by Intel, used as backbone inference system by Pytorch~\cite{NEURIPS2019_9015}, Tensorflow~\cite{abadi2016tensorflow}. With respect to the Math Kernel Library (MKL)~\cite{wang2014intel} by Intel, oneDNN guarantees the same performances while being open source.
The oneDNN library adopts the following parameters for CPUs with AVX2 ISA enabled: $m_c = 10000$, $n_c =384 $, $k_c = 192$, while for the micro-kernel we have  $m_r = 24, n_r = 4$.
%---------------- BEGIN MKL DETAILS----------------------
% Before moving on the empirical GFLOPs estimation, we analyze some of the features of the oneDNN library. 
% OneDNN adopts the following  parameters for CPUs with AVX2 ISA enabled: $m_c = 10000$, $n_c =384 $, $k_c = 192$, while for the micro-kernel we have  $m_r = 24, n_r = 4$.
The macro-kernel parameters $m_c, n_c, k_c$  are selected to deal with very large matrices; for the sequential case, the library contains a mechanism to tailor smaller shapes. Let us call $\overline{m}_c$, $\overline{n}_c$, $\overline{k}_c$ the parameters that the macro and micro kernels actually use. 	
$\overline{m}_c$ is chosen as:
$$\overline{m}_c = \texttt{rnd\_up}(min(max(m, m_r), m_c
), m_r)$$
%\fnote{sopra ho messo textt la chiamata a procedura. ti piace? se si, fixa everywhere. FM}
where \texttt{rnd\_up}($a,b$) is a function which approximates $a$ as $a = n^{*} b$, with $n^{*} = min\{n \ |\  nb \geq a\}$, \textit{i.e.}, to the subsequent multiple of $b$. This way, it is ensured that $\overline{m}_c$ is larger than the micro-kernel parameter $m_r$ and that the default $m_c$ is not involved if $m \leq m_c$. By means of the \texttt{rnd\_up} function, we ensure that $\overline{m}_c \bmod m_r  = 0$ to avoid undersized horizontal $\tilde{A}_j$ panels in the micro-kernel.  
%When the resulting $m_c > m$,  we pad with zeros the difference $d = m_r - m$. This means that $d$ values are read and written but they are not actually used, causing a performance drop.
Similar refinements are adopted to chose $\overline{n}_c$ and $\overline{k}_c$. Moreover, oneDNN triggers when the cost of packing the matrices into contiguous arrays surpasses the cost of multiplication. In this case, besides changing the macro-kernel parameters, it also performs a different routine that skips copying the matrices in cache-aware buffers.

%In choosing $\overline{k}_c$, oneDNN introduces two parameters  $k_{ct} = 256$, which stands for $k_c$ traditional, and $k_{cs}$, for $k_c$ small. $\overline{k}_c$ is computed as function of the shared dimension $k$ as:

% \begin{algorithm}
% 	%	\KwData{this text}
% 	%	\KwResult{how to write algorithm etith \LaTeX2e }
% 	$\overline{k}_c$
% 	\uIf{$k < k_{ct}$ }{
% 		$\overline{k}_c = max(128 , \overline{k}_c)$\;
% 	}\uElseIf{$k < 2 k_c$ }{
% 		$\overline{k}_c = (k+1)/2)$\;
% 	}\Else{
% 		$\overline{k}_c = k_c$\;
% 		}
% \end{algorithm}
% The actual value of the blocking parameter $n_c$, namely $\overline{m}_c$ is chosen depending on both $n_c$ and $k_c$. One more parameter is introduced, $n_{csk}$ which is used when $\overline{k}_c$ is small.

% \begin{algorithm}
% 	$\overline{m}_c = (k<k_{cs}) ? n_{csk} : n_c $\;
% 	$\overline{m}_c = rnd\_up(min(max(n, n_r), n_c), n_r)$	\;
% \end{algorithm}

% Furthermore, oneDNN triggers when the cost of packing the matrices into contiguous arrays surpasses the cost of multiplication. In this case, besides changing the macro-kernel parameters, it also perform a different routine that skips copying the matrices in cache-aware buffers.
%---------------- END MKL DETAILS----------------------

In Section~\ref{subsec:dmm} we have described the optimization techniques beyond Dense Matrix Multiplication on modern CPUs. We also detail the tailored refinements implemented by the oneDNN framework to deal with matrices where at least one dimension is small. We now show the performance of the oneDNN framework with differently shaped matrices, aiming at identifying a reliable $t_m$ for Equation~\ref{eq:tm}. In these experiments, we multiply random matrices with different shapes to empirically analyze how the oneDNN library adapts to different matrix dimensions. We propose two different cases: 1) $m=k$, 2) $mk=c$, with $c$ as a constant integer.
We run our tests on a i9-9900K processor, with AVX2 instructions, 3.6 GHz, max frequency 5.0 GHz. Each core has a 32 KiB L1 cache for data, 32 KiB L2 cache for instructions, both 8-way set associative, 256 KiB L2 cache 4-way set associative, and 2 MiB L3 cache, 16-way set associative. We report the results for single-thread execution. 
In our first experiment, we vary $m$ and $k$ in a fixed range and report the corresponding GFLOPs, with different values of $n$. Observe that $A$, of shape $m \times k$,  represents the weight matrix $W$, $B$, of shape $k \times n$ represents the input matrix $x$, obtained by stacking $n$ input vector. We vary $m$, $k$ and $n$ to model real use-case scenarios: $m$ and $k$ correspond to the sizes of Feed Forward Network layers, while $n$, the \emph{batch size}, is the number of documents we give in input to the neural network at a time. Results are reported in Figure~\ref{fig:onednnl_no_r}, which shows that GFLOPS grow as the size of the matrices even with the aforementioned techniques tailored to edge cases. In Figure~\ref{fig:onednnl_rev}, we show the results of the reverse experiment: instead of gradually increasing both $m$ and $k$, we keep the size of $A$ constant (the $mk$ product is constant). The figure shows that small values of $m$ with large values of $k$ still afford  high-performance (left side of the graph). On the other hand, small values of $k$ paired with larger values of $m$ cause serious performance degradation. The variation of the GFLOPS with the matrix shapes suggests that a unique and size-independent $t_m$ is not reliable. As aforementioned, this evidence some limitations of the Goto algorithm when dealing with edge combinations of input dimensions.
A correct analysis expresses $t_m$ as a function of the $m,n,k$ parameters, or in the case of the Feed Forward Network , as a function of the dimensions of the layers, \textit{i.e.}, $t_m = t_m( l_1, \dots, l_{d})$.
Given the variability of the performance with input shapes, we shall empirically measure them.
 We can use Figure~\ref{fig:onednnl_no_r} and ~\ref{fig:onednnl_rev} to derive a lookup table that maps the matrix shapes to the corresponding GFLOPs. The previous graphs are synthesized in Figure~\ref{fig:heatmap}, which shows an heatmap of the GLFOPs with different values of $m$ and $k$ and $n = 1000$.

 We observe three performance zones, defined by horizontal stripes induced by partitioning the $k$ axis.
\begin{itemize}
	\item $K \geq 512$ : high-performance (130 GFLOPs)
	\item $128 \leq K \leq 512$: Medium performance (110 GFLOPS)
	\item $K \leq 128$: Low performance (90 GFLOPs)
\end{itemize}



% \begin{figure}
% \begin{minipage}[b]{\columnwidth}
% \includegraphics[width=\columnwidth]{imgs/DNNL_different_N.png}
% %\centering 
% %\caption*{Static}
% \end{minipage}%
% \\ 
% \begin{minipage}[b]{0.5\columnwidth}
% \includegraphics[width=\columnwidth]{imgs/DNNL_different_N_reverse.png}
% %\centering 
% %\caption*{Dynamic}
% %\subcaption{Another subfigure}
% \end{minipage}%
% \caption{Matrix Multiplication with oneDNNL}
% \label{fig:onednnl}
% \end{figure}

\begin{figure}	
	\centering
	\includegraphics[width=\columnwidth]{imgs/DNNL_different_N.png}
	\caption{Matrix Multiplication with oneDNN, as $m$ and $k$ grow.  }
	\label{fig:onednnl_no_r}
\end{figure}

\begin{figure}
	\centering
	\includegraphics[width=0.8\columnwidth]{imgs/DNNL_different_N_reverse.png}
	\caption{Matrix Multiplication with oneDNN, with the product $mk$ constant. }
	
	\label{fig:onednnl_rev}
\end{figure}

\begin{figure}
	\centering
	\includegraphics[width=\columnwidth]{imgs/heatmap_gflops_batch1000.png}
	\caption{Matrix Multiplication HeatMap with n = 1000.}
	\label{fig:heatmap}
\end{figure}


For a network of size $\{1000, 500, 500, 100 \}$, we can assume to be always in the high-performance region, except for the last layer. Observe that the last layer has a negligible impact on the overall forward time and we can ignore it. Table~\ref{table:est_vs_real_exec_t} illustrates how the prediction model can substitute the experimental procedure of training and testing a model, turning out to be essential to reduce the architecture search space.

\begin{table}[htb]
	\centering

	%\adjustbox{max width=\columnwidth}{
	\begin{tabular}{lrr}
	%\resizebox{\columnwidth}{!}{
		\toprule
		\multirow{2}{*}{Model} &    \multicolumn{2}{c}{Scoring Time ($\mu s$/doc)} \\
		\cmidrule{2-3}
		 & Real & Predicted \\		%    \thead{Real Scoring & \\ Time per doc ($\mu s $)}  & \thead{Real Scoring & \\ Time per doc ($\mu s $)} \\
		\midrule

		 1000$\times$500$\times$500$\times$100 & 14.4& 14.5 \\
		 200$\times$100$\times$100$\times$50 & 	1.3& 1.3 \\
		 300$\times$150$\times$150$\times$30 & 2.0 & 2.2 \\
		 500$\times$100 & 2.1 & 2.2\\
 		\bottomrule
	\end{tabular}
	 %}
	\caption{Performance of our dense prediction model. Real execution times measured with batch size = 1000.}
	\label{table:est_vs_real_exec_t}
\end{table}

\subsection{Sparse-Dense Matrix Multiplication} 
\label{subsec:sdmm}
In this section, we study Sparse-Dense Matrix Multiplication (SDMM), a special case of matrix multiplication where the first matrix is \textit{sparse}: we recall that \textit{sparsity} is defined as the percentage of zero entries in a data structure, in this case, a matrix. First, we describe a common format to store sparse matrix, \textit{Compressed Sparse Row} (CR). Then, we detail how SDMM is implemented on modern CPU processors. 

\smallskip
\noindent \textbf{CSR Format}.
A sparse matrix is completely identified by its non-zero values and their positions since all the others entries are zeros. This motivates the use of a different representation for sparse matrices w.r.t. to dense ones. The different representation aims at saving storage space and improving the performance of matrix multiplication. For this purpose, several formats have been developed: the most common are Compressed Sparse Row (CSR), Compressed Sparse Column (CSC), Coordinate List (COO). Among them, we analyze CSR, since it is usually supported by off-the-shelf libraries, both for storing and for matrix operators, such as multiplication and it naturally fits to Sparse-Dense Matrix Multiplication, as we will detail.
%Let us assume we have a matrix $A \in \mathcal{R}^{m \times k}$ with $nnz$ non-zero values.
%$$ \text{sparsity} = \frac{nnz}{mk}$$

Let us consider a matrix $M \in \mathbb{R}^{m \times n}$ with $nnz$ non-zero elements. 
An example of the  CSR representation is reported in Figure~\ref{fig:sparsecsr}. It consists of three vectors: \textit{values} $\in \mathbb{R}^{nnz}$, \textit{columnIndex} $ \in \mathbb{R}^{nnz}$, \textit{rows} $\in \mathbb{R}^{m+1}$. 
The \textit{values} array stores the non-zero entries, and \textit{columnIndex} stores their column index in the original matrix, meaning that \textit{columnIndex}$[i]$ stores the columns index of \textit{values}$[i]$. The \textit{rows} array is built so that $rows[i+1] - rows[i] $ is the number of non-zero entries for row $i$. 

\begin{figure}
\centering
	\includegraphics[width=0.8\columnwidth]{imgs/CSR_sparse.pdf}
	\caption{CSR Format for Sparse Matrices.}
	\label{fig:sparsecsr}
\end{figure}


\smallskip
\noindent \textbf{Sparse Dense Matrix Multiplication}.
Sparse Dense Matrix Multiplication or sparse Multi-vector multiplication (SDMM) has a large range of applications: fluid dynamics, graph analysis~\cite{tiskin2001all}, non-negative matrix factorization~\cite{kim2011fast}, economic modeling, seismic simulations~\cite{breuer2019petaflop}, and machine learning~\cite{NIPS2010_4099}. Pruning a neural network pre-trained model naturally induces the usage of SDMM in the forward pass of a Multi-Layer Perceptron, since it converts dense weights into sparse ones. Let us consider Equation~\ref{eq:mlpforward}: in the most general case $W$ represents the dense weight matrix. After pruning, $W$ is transformed into a sparse matrix $\dot{W}$, thus converting $\dot{W}^T x$  into a Sparse Dense Matrix Multiplication.
%Historically, the scientific community has mainly focused in optimizing the Sparse Matrix dense vector multiplication (SpMV). Even if SpMM can be implemented as a loop of SpMV, this approach fails in exploiting data locality in the sparse matrix~\cite{zheng2016semi}. 

%Before describing cutting-edge implementations	 of SDMM, we depict the na\"ive algorithm induced by the CSR Format. %repparse matrix multiplication is slowed down by random memory accesses. Several approaches create their own matrix format to reduce this drawback, even exploiting domain-specific knowledge to guess information on the structure of the matrix. In our case, the structure of the matrix is known \textit{ a priori} but has no specific structure since it derives from the pruning of a neural network. 
Consider the operation $C = AB$, where $A \in \mathbb{R}^{m \times k}$ is a sparse matrix in the CSR representation with $nnz$ non-zero values, and $B \in \mathbb{R}^{k \times n}$, $C \in \mathbb{R}^{m \times n}$ are dense matrices. 
The mundane algorithm induced by A being in CSR Format is reported in Algorithm~\ref{alg:csr_mult}. This format is suitable for row-wise access, allowing to consider exclusively the non-zero entries of the left-side matrix. 
The total number of floating-point operations is reduced from $2mnk$ to $2 nnz N$ w.r.t. dense case, but the irregular access pattern induced by sparsity hinders the efficiency of the algorithm. To overcome this problem, a twofold strategy, as for the dense case, is applied: 1) proficient data access pattern, 2) optimization of the core operation (\textit{micro-kernel}). 

The most used library for sparse matrix multiplication is the Math Kernel Library (MKL)~\cite{wang2014intel}, which implements the sparse versions of third level BLAS routines. Since the library is closed and there are no details on how the multiplication is implemented\footnote{https://community.intel.com/t5/Intel-oneAPI-Math-Kernel-Library/Sparse-Dense-Matrix-Multiplication/m-p/1173953}, we refer to the implementation of the LIBXSMM~\cite{heinecke2016libxsmm}, which is open-source. Later on in this section, we show that LIBXSMM actually outperforms MKL in the spectrum of shapes involved by our neural networks. 

\begin{figure}
\centering
	\includegraphics[width=\columnwidth]{imgs/libxsmm_sparse_dense_mult.pdf}
	\caption{LIBXSMM Sparse-Dense Matrix Multiplication (SPMM).}
	\label{fig:libxsmmsparsedense}
\end{figure}

\begin{algorithm}[]
	\KwData{ CSR $A \in \mathbb{R}^{m \times k}$, $B \in \mathbb{R}^{k \times n}$}
	\KwResult{$C \in \mathbb{R}^{m \times n}$  }
	$C[i,k] = 0$\;
	\For{$i = 0$ \KwTo $M-1$}{
		\For{ $j = A.rows[i]$ \KwTo $A.rows[i+1]-1$}{
		  \For{$k = 0$ \KwTo N-1}{
		  	$idx   = A.cols[j]$\;
		  	$C[i,k] \leftarrow C[i,k] +  A.val[j] * I [idx, k]$\;
		  }
		}
	}

	\caption{Sparse-Dense Matrix Multiplication algorithm with CSR format.}
	\label{alg:csr_mult}
\end{algorithm}

%We observe the same problems reported with for dense matrix multiplication in terms of lack of data re-usage. If we observe the most inner cycle computing $C_{i,k}$, we observe that the whole $I$ matrix may be needed to compute the $i$ line of C. As for dense matrix multiplication, a block-wise approach could partially solve this issue. With the lesson learned from the dense case, we can state that we should load data by blocks thus promoting data re-usage. 
%Born in 2015, this library was conceived to cover use cases that other Intel libraries, \textit{e.g.,} MKL and One-DNN, left uncovered. 

\smallskip
\noindent \textbf{Sparse-Dense matrix multiplication with LIBXSMM}. LIBXSMM~\cite{heinecke2016libxsmm} is a high-performance library specifically tailored for Intel architectures, specialized in small dense matrix multiplication, sparse matrix multiplication, and deep learning primitives in general. It is based on ``Just in Time'' (JIT) code specialization, which intends to exploit the runtime information about its operands. The sparse-dense routine was originally developed to solve seismic equations~\cite{breuer2019petaflop}.
% such as matrix shapes or non-zero entires location for sparse matrices. The library includes both dense-sparse and sparse-dense matrix multiplication, and the latter was originally developed to solve seismic equations~\cite{breuer2019petaflop}. 

We now detail the sparse-dense matrix multiplication as implemented in the LIBXSMM library, with A in CSR format. 
The dense matrix $B$ is converted into a three dimensional tensor of shape $k \times N_b \times n_b$, as reported in Figure~\ref{fig:libxsmmsparsedense},
so that $N = N_b \times  n_b$. This means to factorize the $N$ dimension in two sub-dimension, in which one ($n_b$) is induced by the underlying hardware. The ideal value of $n_b$ in fact, coincides with the SIMD length of the processor, \textit{i.e.}, the number of different numbers that a SIMD vector can store. 
Using floating-point variables (32 bit) on a machine with AVX2 ISA (256 bit), the SIMD length is 8. This packing allows to  multiply each non-zero element of $A$  with $nb$ values of $B$ at time, using just one vectorized instruction.



The problem of irregular accesses is tackled by hardwiring the loading of the elements of $A$ and $B$, so that only relevant elements are loaded. The data access pattern provides for multiplying each non-zero element of $A$ ($a_{i,j}$) with the $j$-th rows of $B$ ($B_j$) and accumulate the results into the $i$-th row $C$ ($C_i$).
The computation is carried on one row of $A$ at time. Figure~\ref{fig:libxsmmsparsedensemicro} shows the sub-routine performed for each row $i$.  
Let us call the first non-zero element of the current row $x$, in position $(i,j)$; we assume to have at least one non-zero entry, otherwise the row is skipped. $C_i$ is loaded into $N_b$ SIMD registers, each containing $n_b$ values. $x$ is \emph{broadcasted} to a SIMD CPU register, \textit{i.e.}, $n_b$ consecutive copies of the $x$ vector are loaded into the register. We refer to this vector as $\overline{x}$. 
$B_j$ is loaded as well and $C_i$ is updated as $C_i \leftarrow C_i + \overline{x} B_j$; the update involves $N_b$ Fused Multiply Add (FMA) instructions $C_{i,k} \leftarrow C_{i,k} + \overline{x} B_{j,k}$. Then, the routine moves to the next non-zero elements in the $i$-th row of $A$. Once all the non-zero elements have been multiplied, $C_i$ is stored in memory and the algorithm moves on to the next row of $A$.  


\begin{figure}[t]
\centering
	\includegraphics[width=\columnwidth]{imgs/libxsmm_sparse_dense_mult_micro.pdf}
	\caption{Micro Kernel of LIBXSMM Sparse-Dense Matrix Multiplication (SPMM).}
	\label{fig:libxsmmsparsedensemicro}
\end{figure}

LIBXSMM is equipped with a mechanism that interrupts the code generation if the number of instructions is too elevated. This can happen if the number of non-zero elements in $A$ or the $N$ dimension are too large. Since the $N$ dimension corresponds to the batch size in the neural forward, we are free to  reduce it to overcome this limit. When necessary, we also split the $m \times k $ $A$ matrix along the $M$ dimension to generate a set of sub-matrices $A_S = \{A_1, \dots, A_s \ | \  A_i \text{ of size }  M/s \times k  \}$. Each $A_i$ will have fewer non-zero entries, preventing code generation failure. 
 %A==\left(\frac{\frac{\check{A}_{1}}{A_{1}}}{\frac{\vdots}{\check{X}_{M-1}}}\right)
The $C$ matrix is trivially obtained by multiplying each $A_i \in A_S$ with $B$ separately and by stacking the results along the $M$ (vertical) axis:
\[ C = 
  \begin{bmatrix}
    \begin{array}{c}
  A_1 B  \\
  \hline
  A_2 B\\
  \hline 
  \vdots\\
  \hline
  A_s B\\	   
    \end{array}
  \end{bmatrix}
	\]

\smallskip
\noindent \textbf{LIBXSMM vs MKL}. As aforementioned, MKL is known to provide the fastest routine for sparse-dense matrix multiplication. We now show that LIBXSMM outperforms MKL on small, very sparse, and asymmetric matrices, which is the typology of matrices we employ in our MLPs for document scoring. In Table~\ref{table:lib_vs_mkl_msn}, we report the execution time of $C = AB$, with $A$ sparse in the CSR format and B dense, both for MKL and LIBXSMM; on the $x$-axis is reported the shape  ($m \times k $) $A$ and its sparsity. $B$ has shape $k \times n$, where $n$ is the batch size, set to $64$. The matrices correspond to the first layer of real models trained on the \msn dataset~\cite{DBLP:journals/corr/QinL13}, which provides $136$ handcrafted features. The Table shows that LIBXSMM is always faster than MKL on these shapes, with a speedup factor often larger than $2$x. This consideration, together with the availability of the code, has led us to pick the LIBXSMM library as the reference implementation. 

\begin{table}[htb]
	\centering
	\begin{tabular}{llrr}
		\toprule
		\multirow{2}{*}{Shape} &   	\multirow{2}{*}{Sparsity}&  \multicolumn{2}{c}{SDMM Time ($\mu s$)}\\
		\cmidrule{3-4}
		 & & MKL & LIBXSMM \\
		\midrule
		400$\times$136 &  0.996 &         3.1 &          \textbf{1.2} \\
 		300$\times$136 &  0.985 &         2.5 &          \textbf{1.4} \\
 		200$\times$136 &  0.971 &         2.8 &          \textbf{1.6} \\
 		100$\times$136 &  0.989 &         1.0 &          \textbf{0.4} \\
 		50$\times$136  &  0.968 &         0.7 &          \textbf{0.2} \\
 		\bottomrule
	\end{tabular}
	\caption{ Comparison between MKL and LIBXSMM for Sparse Dense Matrix Multiplication (SDMM)	. Shapes and sparsities represent the first layer of FFNs trained on \msn. Batch size is set to $64$.}
	\label{table:lib_vs_mkl_msn}
\end{table}

% \begin{figure}
% \centering
% 	\includegraphics[width=\columnwidth]{imgs/mult_on_\msn.png}
% 	\caption{A comparison between MKL and LIBXSMM. Shapes and sparsities represent the first layer on MLPs trained on \msn. Batch size is set to $64$}
% 	\label{fig:lib_vs_mkl_msn}
% \end{figure}

\subsection{Sparse Time Predictor}
\label{subsec:sptimepred}
In this section, we illustrate the development of a Sparse-Dense matrix multiplication time predictor, specularly for what we have done for the dense case.
As detailed in Section~\ref{subsec:sdmm}, the algorithm provides for iterating over the rows of $A$ with at least one non-zero entry. We start by analyzing the time cost of multiplying the $i$-th row of $A$ with $B$, which is given by the sum of the cost of the following operations. 

\begin{enumerate}
	\item Loading $N_b$ vectorized elements from $C$ (each one of size $n_b$).
	\item Loading each non-zero element in $A_i$. Since the non-zero values of $A$ are stored contiguously in $A.values$, this operation benefits from cache memory.
	\item Loading $N_b$ vectorized elements of $B$ (of size $n_b$) for each non zero element of $A$.
	\item Updating $C_j \leftarrow C_j + x *B_j$, for each $x\neq0$ in $A_i$. Each update consists in $N_b$ FMA instructions. 
	\item Storing $N_b$ vectorized elements of $C$ (each one of size $n_b$).  


\end{enumerate}
Let us define $a_c$ as the set of active columns in $A$, namely the set of columns containing at least one non-zero element, and $a_r$ the set of active rows in $A$. Let us also define $L_c$ as the cost to load and store $N_b$ elements of $C$, $L_a$ the cost of loading one element of $A$ and updating $C_j$ with $N_b$ FMA instructions, and $L_b$ the cost of loading $N_b$ elements of $B$. 
%Observe that $L_b$ depends on the $N_b$ factor: we can remove dependence by dividing by $N_b$, or using configurations in which $N_b$ is fixed. Being $N_b = N/ n_b$, with $N$ the batch size, we will pick the latter option. 
When generalizing the previous costs to the entire matrices, we have to take into account the effects of caching. 
While $A$ and $C$ are loaded just once, $B$ elements can be loaded multiple times; whether they benefit or not from the caching mechanism depends on the access pattern induced by the non-zero entries of $A$.
For example, if $x$ in position $(i,j)$ is a non-zero element, $B_j$ needs to be loaded into the registers from the main memory and the cache will also retain a copy of $B_j$.
Assume that in a successive row of $A$ exists an element $x' \neq 0$ on the same column of $x$, \textit{i.e.} in position $(g, j)$, with arbitrary $g$. When performing $C_g \leftarrow C_g + x' B_j$,  $B_j$ already resides in the cache: since loading elements from the cache is way much faster than loading them from memory, the cost of re-loading $B_j$ can be considered negligible. 
Assuming that once a row of $B$ is loaded into the cache it remains there until the end of the operation, we pay the cost of loading a row $B_j$ just the first time that this row is loaded. At the same time, if there are any inactive rows, they are never used in the multiplication routine. Since the number of active rows in $B$ is equal to the number of active columns of $A$, the cost of loading $B$ can be approximated with $L_b |a_c|$, with $|a_c|$ representing the number of active columns in $A$.

The overall cost of SPMM with the LIBXSMM is given by: 
\begin{equation}
\label{eq:sparsepred}
	T = |a_r| * L_c +  nnz * L_a + |a_c| * L_b
\end{equation}

%is multiplied with the number of active columns. In fact, if we consider $B$ and $C$ to reside in the main memory at the beginning of the operation, once that a line of $B$ has been loaded into the cache it will likely remain into a cache during the rest of the operation. 	

With an accurate estimation of $L_a$, $L_b$, and $L_c$, we can predict the execution time of a sparse-dense matrix multiplication just from the structure of the sparse matrix. Note that this structure is known \textit{a priori}, being the sparse matrices the pruned weights of the neural model. We begin by observing that $L_b$ and $L_c$ both describe memory operations, with the difference that $L_c$ measures both data reading and writing, while $L_b$ refers to data reading. We empirically verify that both the operations have the same time cost, \textit{i.e.}, $L_c = 2 L_b$. 

We now infer the coefficients $L_a, L_b, L_c$, starting from $L_b$. We cannot measure with a timer the cost of the elementary operations we have divided the LIBXSMM SPMM routine in, but we can empirically compute them by difference.

Let us consider two different sparse matrices $A_c$ and $A_{rd}$ with the same shape $m \times k$ and the same number of non-zero entries ($nnz$). $A_c$ has the non-zero values disposed on the same column $j*$, \textit{i.e.}, is a matrix where $a_{i,j} = 0 $ if $j \neq j*$. $A_{rd}$ is a sparse matrix which has single non-zero entry for each row and each column, \textit{i.e.,} $\sum_{i=0}^{i=m-1} a_{i,j} = 1, \forall j=0, \dots, m-1$  and $\sum_{i=0}^{i=k-1} a_{i,j} = 1,  \forall i=0, \dots, k-1$. 
The cost of multiplying $A_c$ and $A_{rd}$
with a dense matrix is given by:
\begin{align}
\nonumber
T (A_c) = m * L_c + nnz *L_a + 1 * L_b \\
\nonumber
T({A_{rd}}) = m * L_c + nnz *L_a + k * L_b
\end{align}
so, 
$$T(A_{rd}) - T({A_c}) =  (k-1) *L_b$$
We can experimentally measure  $T(A_{rd})$ and $T({A_c})$ and use them to compute $L_b$, since $k$ in known. 

To derive $L_a$, we use the same $A_c$ as before and a second matrix $A_{2c}$, having $2*nnz$ non-zero entries, organized along two columns. The cost for multiplying $A_{2c}$ with a dense matrix is given by 
$$T (A_{2c}) = m * L_c + 2*nnz *L_a + 2* L_b$$
Since $L_b$ can be derived using the previous expression, we can subtract $T(A_{2c})$ and $A_c$ and obtain $L_a$ as:
$$L_a = (T(A_{2c} )- T(A_c))/nnz -L_b$$
Our aim is to compute size-agnostic $L_a$, $L_b$ and  $L_c$. We set $M=K$ and vary them in $\{200, 300, 400, 500 \}$ and we experiment $N \in \{16, 32, 64\}$. $L_b$ and $L_c$ depends on $N$ (and so do $L_c$, which is computed doubling $L_b$), so we normalize diving by $N$. We observe that when $N \geq 128$, the value obtained for $L_a$ and $L_b$ diverge w.r.t. to smaller batch size. In fact,  larger $N$ values ($N \geq 128$) break the hypothesis of $B$ residing inside the cache during the whole multiplication, which is a fundamental assumption of our time predictor.  The definitive time predictor parameters are computed as an average of their value obtained with different shape configurations.
We demonstrate the validity of our sparse predictor in Table~\ref{table:est_vs_real_exec_t_sparse}, where we report the predicted and the real execution time needed to multiply the weights of the first layer of several neural models with a random input. We restrain our experiments to the sparsity range obtained with pruning on our neural architectures. As we will detail later, at these sparsity levels the time required for SDMM  is negligible w.r.t. to its dense counterpart. We also evidence that our time predictor is specific to matrix multiplication, hence can be essentially applied to fully-connected layers. Convolution or attention-based architectures have different properties that require a specific investigation. We leave these analyses for future works.

As we can see, the predictor is capable of correctly estimating the execution time of different models at high levels of sparsity, with a small error. Specifically, the predictor can fruitfully distinguish between matrix with the same shape but with different sparsity percentages; two examples are the $200 \times 136$ and the $100 \times 136$ instances in Table~\ref{table:est_vs_real_exec_t_sparse}. First, we observe that with sparsity percentage in the order of $1\%$, SDMM execution times can vary up to $30\%$. Second, the time predictor correctly reflects this peculiarity, thanks to the deep understanding of the routine details that stands behind its development.

% Observe that an exact predictor is beyond the scopes of this paper; its role is just to reduce the neural models search space by dividing them into models which can match an efficiency requirement an models than can not. 

\begin{table}[htb]
	
	%
	
	\adjustbox{max width=\columnwidth}{
	\centering
	\begin{tabular}{lrrrrrrr}
	%\resizebox{\columnwidth}{!}{
		\toprule
		\multirow{3}{*}{Shape} &   	\multirow{3}{*}{Sparsity}& 
		 \multicolumn{6}{c}{SDMM Time ($\mu s$)} \\
		\cmidrule{3-8}
		& & \multicolumn{2}{c} {$N=16$}& \multicolumn{2}{c} {$N=32$} & \multicolumn{2}{c} {$N=64$} \\
		\cmidrule{3-8}
		 & & Real & Pred. & Real & Pred. & Real & Pred.\\		
		 \midrule
		 \multirow{2}{*}{400$\times$136} &  0.995 & 0.2 & 0.2 & 0.4& 0.4&   0.9 &          0.8 
		 \\
		  &  0.986 &  0.4& 0.4& 0.9& 0.8&         1.9 &          1.6 \\
		   \arrayrulecolor{black!30}\midrule

 	300$\times$136 &  0.985 &0.3 &0.3 & 0.7& 0.7&        1.6 &          1.4 \\
 	 	  \arrayrulecolor{black!30}\midrule

 	 	\multirow{2}{*}{200$\times$136} &  0.982 &   0.3& 0.3& 0.5 & 0.5 &        1.0&          1.0 \\
 	  		&  0.971 &       0.4& 0.3& 0.7&0.6 &  1.5 &          1.3 \\
 	  	\arrayrulecolor{black!30}\midrule

 	 	\multirow{2}{*}{100$\times$136} &   0.989 & 0.1 &0.1 & 0.2& 0.2 & 0.5 &          0.4 \\
          &  0.967 & 0.2&0.2 & 0.3& 0.4&      0.7 & 0.7 \\ 
         \arrayrulecolor{black!30}\midrule

 	 	50$\times$136  &  0.987 &0.1 	& 0.1& 0.1 & 0.1 & 	   0.2 & 		  0.2 \\
 	 	\arrayrulecolor{black}
 	 	\bottomrule
	\end{tabular}
	}
	
	\caption{Some examples of our sparse time predictor with different values of $N$.}
	\label{table:est_vs_real_exec_t_sparse}

\end{table}

% \section{Forest Distillation}
% \label{sec:fordist}
% In this section we detail how to train a Feed-Forward Neural Network (FFNN or NN) to score documents in a Information Retrieval (IR) system. The idea was first proposed by Cohen \textit{et al.}~\cite{cohen2018universal}: NNs are known to be universal approximators~\cite{hornik1991approximation}, so they can be trained to mimic the output of an ensemble of regression trees, becoming \textit{de facto} a document scoring engine.

% Formally, 
% let us consider a Learning to Rank dataset $D = (X, Y)$,  $X \in \mathbb{R}^{f}$, where $f$ is the number of extracted feature per document, and $Y \in \mathbb{N}$ is the set of ground truth relevances w.r.t. a query.
% Let  $F: \mathbb{R}^{f} \rightarrow \mathbb{R}$ be the underlying function learned by a Regression Forest during the training, mapping document features to document scores. To generate a neural-based scoring document system, we need the network to approximate this function; this is done 
% by replacing the original $Y$ ground truth with $F(X)$, \textit{i.e.}, substituting the ground truth relevances with the scores of the ensemble of regression trees on the dataset $X$. 
% %to virtually create a new $D_F = (X, F(X))$. 

% Even if this approximation-based approach could seem counterintuitive, it results profitable because of the different \textit{learning styles} between the models.
% There are several approaches in Learning to Rank: \textit{pointwise}, \textit{pairwise} and \textit{listwise}; among them, the \textit{listwise} has demonstrated to consistently outperform the others.  Anyway, neural models show to struggle in directly exploiting it; the divergence of learning styles is mainly expressed by the objective functions, which happen to be non-differentiable in the \textit{listwise} approach, hindering Stochastic Gradient Descent (SGD) to find global minima. 
% %TODO qui dettagli
% Minimizing the mean square error between the network and the tree-based model outputs is a way to overcome the problem: the learning style itself is similar to the \textit{pointwise} approach, but since the $F(\cdot )$ function is learned with a \textit{listwise} one, the final $N(\cdot)$ function expressed by the neural model results more accurate, as empirically proven~\cite{cohen2018universal}. 

% A reason to compose this approximation is to exploit high performing neural forward systems to obtain faster scoring mechanisms. Anyway, when comparing 
% Neural Network forward time with the state of the art 
% %\ross{state-of-the-art? mi pare di si} 
% method for scoring Regression Trees, namely QuickScorer~\cite{lucchese2015quickscorer,dato2016fast,8035185}, Cohen \textit{et al.}~\cite{cohen2018universal} observed that NNs are faster only when their forward pass is performed on GPU. For the reasons explained in Section~\ref{sec:introduction} we leave out the GPU inference from our comparison and  limit it to CPU based-inference systems.

% %wihotut optmizazed usage simd instructions	
% In the original article, the comparison on CPU involved: a) old QuickScorer results (written in C++), without SIMD instructions\footnote{ this is due to code availability issues, since QuickScorer code is not publicly available}, b) neural inference using a Python deep learning framework. 
% Our first contribution is to  obtain a solid and coherent CPU comparison between the two methods by measuring the performance on the same hardware, with inference mechanisms written in same the programming language. % \ross{setting: hardware and programming language ?} 	
% First, we train a Regression Forest on the \msn dataset~\cite{DBLP:journals/corr/QinL13}, using a grid search to explore the parameters space and to achieve the best $F(X)$ as possible. In this phase, we limit the number of leaves to $64$. We use LightGBM~\cite{NIPS2017_6907} to train the models, which performs better than RankLib, the tool used in the original article. For the sake of fairness, we believe that NNs shall compete with the best available Regression Forest, which is capable to learn an optimal $F^*(X)$ function, that could results harder to approximate with respect to a sub-optimal generic $F(X)$.

% %TODO dati sulla grid search?

% Once we got the tree model, we train the Neural Model to mimic $F(X)$. We picked two network architectures, a Large Network with 4 layers of size $\{2000,500,5000,100\}$ and a Small Network with 2 layers of shapes $\{500,100\}$, as in the original work. We adopt the same strategy for randomly generating training data use by Cohen \textit{et al}. 	~\cite{cohen2018universal}. We use Adam~\cite{kingma2014adam} as optimizer, with learning rate $0.001$ and no weight regularization nor dropout~\cite{srivastava2014dropout}; we multiply the learning rate by $\gamma = 0.1$ at epochs $\{50, 80 \}$ and use and early stopping criterion on the validation loss; we used the Pytorch framework to train the neural networks. We activation function, we used RELU6, with $\text{RELU6}(x) = min(max(x,0), 6)$. 

% In the original work, neural models could perform as well as the Regression Forest trained with RankLib in terms of MAP. On the contrary, Table~\ref{table:effect_comp_orwk} shows that Neural Networks struggle in approximating the optimal $F^*(X)$ function, thus providing lower MAP and NDCG@10 values. Metrics are computed with the RankEval python library~\cite{rankeval-sigir17}. This first observation states that the approximation step has a cost in terms of accuracy drop that has to be taken into account in the analysis of the effectiveness efficiency trade-off. 

%  %inducing a remarkable $\Delta$ in terms of both the chosen metrics. \ross{L'uso di $\Delta$ cosi farebeb schifo anche ai greci. Aggiungerei numeri veri e conclusione dell'esperimento.}

% Forward time is computed both through a python implementation, as in the original work, and in our own C++ version of NN forward pass. In particular,  for the python version we use PyTorch CPU forward while they tested it with TensorFlow. Instead, we use the \textit{dnnl\_sgemm} routine from Intel oneDNN framework to implement matrix multiplications in C++, with JIT compilation, always forcing single-thread execution. We used the latest version of QuickScorer~\cite{lucchese2016exploiting} that exploits SIMD instructions.
%  %while for GPU PyTorch was the framework adopted either in this work and in the original one.
% CPU experiments were conducted on a Intel i9-9900K CPU, with AVX2 (latest ISA supported by QuickScorer) instructions, 3.5 GHz, with L1-cache 256KiB, L2-cache 2 MiB, L3-cache 16MiB.  Neural Network scoring time is computed with batch size = 1000.

% Despite the large gap in terms of NDCG@10, the Regression Forest scored with QuickScorer is faster of the Large Network, as illustrated by Table~\ref{table:speed_comp_orwk}. 
% Anyway, we believe that a fair comparison shall involve two models at the same effectiveness level; as reported by Table~\ref{table:speed_comp_orwk},
% we can obtain the same NDCG@10 of the Large model with $150 $ trees and $64$ leaves per tree, and the NDCG@10 of the Small model with $200$ trees with $32$ leaves per tree.  Table~\ref{table:speed_comp_orwk} also shows that Quickscorer scores a document $16x$ faster than the Large Network and $2.75x$ times faster than the Small one.
% Hence, tree-based solutions are to prefer to neural networks. This section seems to leave no room for neural models in the document scoring  task; in the following section, we will provide a methodology to create feed-forward networks than can compete or even outscore both the effectiveness and the efficiency of the ensemble of Regression Forests.


% \begin{table}
% 	\centering 
% 	\begin{tabular}{lrrrr}
% 		\toprule
% 		Model &     NDCG@10 &     MAP 0 & MAP 1\\
% 		\midrule

% 		RF (878 trees, 64 leaves)&    0.5246& 0.6304  &0.6604  	\\
% 		\midrule
% 		Large Network &   0.5198&0.6279   &0.6579  \\
% 		Small Network & 0.5180& 0.6277 &0.6576  \\
% 		\bottomrule
% 	\end{tabular}
	
% 	\caption{Comparison in terms of NDCG@10 and MAP between a Regression Forest (878 trees, 64 leaves) and Neural Networks on \msn. MAP$x$ means that $x$ is the score assigned to non relevant results. }
% 	\label{table:effect_comp_orwk}
% \end{table}

% \begin{table}
% \centering
% %\adjustbox{max width = \columnwidth}{

% 	\begin{tabular}{lrrr}
% 		\toprule
% 		\multirow{2}{*}{Model} & \multirow{2}{*}{NDCG@10}&    \multicolumn{2}{c}{Scoring Time ($\mu s$/ doc)} \\
% 		\cmidrule{3-4}
% 		& & Python & C++ \\
% 		\midrule
% 		RF (878 trees, 64 leaves)& 0.5246 & / &8.2  	\\
% 		\midrule
% 		RF (150 trees, 64 leaves)& 0.5206& / &1.5 	\\
% 		RF (200 trees, 32 leaves)& 0.5181&  /&0.8 \\
% 		\midrule
% 		Large Net & 0.5198  &35.6 & 24.4 \\
% 		Small Net & 0.5180  & 3.7 & 2.2 \\
% 		\bottomrule
% 	\end{tabular}
% 	%	}
% 	\caption{Comparison in terms of Scoring Time between Regression Forests and Neural Networks on \msn.  }
% 	\label{table:speed_comp_orwk}
% \end{table}


\section{Neural Engineering}
\label{sec:neuraleng}
In Section~\ref{sec:introduction}, we claim that ensembles of regression trees consistently outperform, both in terms of effectiveness and efficiency, NNs trained with the method proposed by Cohen \textit{et al.}~\cite{cohen2018universal}, when documents are scored on CPU. In this section, we break down a methodology used to create efficient neural models for ranking that can compete with ensembles of tree-based ones.

\subsection{Approximation of an Ensembles of Trees}
\label{subsec:approxbetter}
We employ the methodology proposed by Cohen \textit{et al.}~\cite{cohen2018universal} to train neural models that approximate the scores of an ensemble of regression trees. This approach is effective since we use a powerful model, \textit{i.e.}, an ensemble of regression trees, and a profitable learning strategy, \textit{i.e.}, a \textit{listwise} approach, to extrinsic the structure of the actual underlying probability distribution. This facilitates the learning process of a simpler model, \textit{i.e.}, a shallow neural network. The idea is inherited from a deep learning compression technique named \textit{Knowledge Distillation}~\cite{bucilua2006model, ba2014deep,DBLP:journals/corr/HintonVD15} in which a small, production-oriented, network (\textit{student}) is trained to mimic the output of a large and effective network (\textit{teacher}).

To fully leverage the benefits of this technique, we train an ensemble of regression trees with the best performance on a validation set without taking into account its efficiency. Then, we use its scores as ground truth in a distillation process that trains our neural models. In Table~\ref{table:imprteacher}, we report the validity of this approach using the \msn dataset~\cite{DBLP:journals/corr/QinL13}, a widely adopted LtR dataset composed of more than 30,000 queries, with about 120 documents per query, where 
each document is a vector of 136 features\footnote{The list of features is available at https://www.microsoft.com/en-us/research/project/mslr/}. We adopt the NDCG@10 as quality metric. First, we observe the difference in terms of ranking precision between: 1) a model trained with a fixed number of leaves, \textit{i.e.}, $64$, 2) the best model we could obtain on the \msn dataset. The latter one, which results to have $256$ leaves per tree, consistently outperforms the $64$-leaves model.
%Then, we report the results when using these two tree-based models as teachers for two different neural networks. 
Increasing the number of leaves in tree-based models allows for a remarkable gain in terms of NDCG@10. Indeed, when scoring a tree-based model with QuickScorer, the execution time scales at least linearly with the number of trees and leaves~\cite{lucchese2015quickscorer,dato2016fast}. Hence, a $256$-leaves model is more than $4$x slower than a $64$-leaves one with the same number of trees. 
In fact, given that the scoring time per document of $64$-leaves models is 8.2 $\mu s$, a $256$-leaves one takes at least $33 \mu s$ to be traversed with QuickScorer.
This means that, when pursuing a trade-off between effectiveness and efficiency, the best solution is the ensemble of $878$ trees with $64$ leaves, due to the linear dependency of the scoring time with respect to the number of leaves. 
%\fnote{da dove vedo la giustificazione sperimentale di questa ultima affermazione sopra? FM \cosimo{Non c'e', non possiamo fare esperimenti con 256 foglie perche' QuickScorer non le supporta. Pero' la citazione rimanda al paper di QuickScorer dove c'e' scritto che il tempo di scoring scala linearmente. Cosimo}. chiaro. metterei i conti della serva per convincere il revisore a spanne che quel 4x, usando lo scoring con 256 non scala una sega. altrimetni si chiede perche' ``unbearable long scoring times''. no?}

Furthermore, we report the results when using two tree-based models as teachers for two different neural networks. 
Our experiments clearly show the positive effects of approximating a more effective \textit{teacher} (Table~\ref{table:imprteacher}). In fact, thanks to the teacher upgrade, the  $1000\times500\times500\times100$ can provide the same ranking precision as the $64$-leaves tree-based model. Observe that the \textit{student} is teacher-agnostic: the architecture of the network is independent w.r.t. the tree-based model which is approximating, and so is the time to perform the forward pass. In conclusion, distilling from a more effective teacher bridges the gap between neural models and ensemble of regression trees in terms of effectiveness. Nevertheless, a margin still exists between the two families of models in terms of efficiency. In the following sections, we will show how to tackle this aspect. 

%we evidence the advantages of  approximating a better \emph{teacher} model, and we incorporate this strategy in our general train and design methodology. Anyway, 
%approximating a better model is an uniquely profitable strategy which allows to improve the effectiveness of the neural models, with no drop in terms of efficiency. 

\begin{table}
% \centering
% \begin{tabular}{lrrr}
% 		\toprule
% 		Model &     NDCG@10 &     MAP 0 & MAP 1 \\
% 		\midrule
% 		878 trees, 64 leaves &    0.5246& 0.6304  &0.6604 	\\
% 		600 trees, 256 leaves &   0.5291&0.6321   &0.6621  \\
% 		\bottomrule
% 	\end{tabular}
% 	\caption{Comparison between Regression Forests with different number of trees and leaves on the \msn dataset}
% 	\label{table:64vs256leaves}
%\vspace{0.6cm}
\centering
%\adjustbox{max width = \columnwidth}{
	\begin{tabular}{llr}
		\toprule
		Model  & Teacher & NDCG@10 \\
		\midrule	
		878 trees, 64 leaves &  / &  0.5246  \\
		600 trees, 256 leaves &  / &  $\uparrow$ \textbf{0.5291}     \\
		\midrule
		\multirow{2}{*}{500$\times$100} & 878 trees, 64 leaves & 0.5180 \\
		& 600 trees, 256 leaves &$\uparrow$ \textbf{0.5198} \\  
		\midrule
		\multirow{2}{*}{100$\times$500$\times$500$\times$100} & 878 trees, 64 leaves & 0.5208 \\
		& 600 trees, 256 leaves &$\uparrow$ \textbf{0.5243}  \\
		\bottomrule
	\end{tabular}
	%	}
	\caption{Comparison in terms of NDCG@10 among Neural Networks on \msn, when trained to approximate different teachers.  $\uparrow$ indicates statistically significant improvement (Fisher's randomization test,  $p < 0.05$).  }
	\label{table:imprteacher}
\end{table}


\subsection{Design of a Neural Model}
\label{subsec::neuraldesign}
In this section, we present our novel methodology to design efficient neural models for ranking. We leverage the insights gained in studying dense and sparse matrix multiplication to show how to make correct architectural choices, thus training a very limited set of candidate models. We provide an empirical evaluation to show the correctness of our assumptions. 
Experiments are conducted on the  \msn dataset~\cite{DBLP:journals/corr/QinL13}, as in Section~\ref{subsec:approxbetter}. We first show how to develop dense models matching some given time requirements. Then, we employ pruning techniques to sparsify these models and outperform ensembles of regression trees.

\smallskip
\noindent \textbf{Architecture design}.
Our approach begins by choosing the dense architectures matching some given time constraints. For the sake of simplicity, we will assume to have two tree-based models to compete with, a $300$-trees ensemble and a $500$-trees ensemble, each one with $64$ leaves per tree. Their NDCG@10  and their scoring time ($\mu s$) are reported in Table~\ref{table:widevsdense}. By using the time predictor developed in Section~\ref{subsec:densetimepred}, the identification of the architectures matching the time requirements is now an easier task.
We build a heatmap as in Figure~\ref{fig:heatmap} and then use it to predict the execution time of the architecture, without the need of testing its performance on real hardware. This allows to discard models that do not match the desired latency constraints. As reported in Table~\ref{table:widevsdense}, there can be several models fitting the time budget. In our case, we propose $2,3,4$ layers NNs. We train the chosen models and compare their NDCG@10. \textit{Deep} networks (more layers) afford better performance w.r.t. \textit{wide} ones (more neurons per layer), coherently with the evolution of neural models witnessed in the last decade. The reason is that deep networks are generally capable of extracting higher levels features thus creating more complex representation of the input. The higher representations are built on simpler ones, generating a nested hierarchy of concepts which allows to improve the understanding and the learning from the data~\cite{Goodfellow-et-al-2016}. We empirically verify that $5$-layers models matching the time constraints do not offer advantages with respect to $4$-layers ones, showing that $4$-layer networks are expressive enough for the ranking task. Dense networks offer performance close to the tree-based model but do not really guarantee advantages neither in terms of effectiveness or efficiency, as shown in Table~\ref{table:widevsdense}.

\begin{table}
\centering
%\adjustbox{max width = \columnwidth}{
	\begin{tabular}{lrr}
		\toprule
		%\multirow{2}{*}{Model}  &  \multicolumn{1}{p{3cm}}{\centering Predicted Execution \\ Time ($\mu s$/doc) }  & \multirow{2}{*}{NDCG@10} \\

		Model & Scoring Time ($\mu s$/doc) & NDCG@10 \\
		\midrule

		QuickScorer 300, 64 & 3.0  & 0.5230	\\
		\cdashlinelr{1-3}
		500$\times$100 & 2.2 & 0.5196 \\%& 2.2 \\
		300$\times$200$\times$100 & 2.4 &  0.5209 \\
		300$\times$150$\times$150$\times$30 & 2.2 & 0.5207 \\
		\midrule
		QuickScorer 500, 64 & 4.9  & 0.5240	\\
		\cdashlinelr{1-3}
		1000$\times$200 & 5.5 & 0.5150 \\
		600$\times$300$\times$100 & 5.6 & 0.5203\\
		500$\times$250$\times$250$\times$100 & 5.4 & 0.5218\\
		\bottomrule
	\end{tabular}
	%	}
	\caption{Comparison in terms of Scoring Time between QuickScorer and Neural Networks on \msn. The notation ''QuickScorer $x,y$'' indicates that $x$ is the number of trees, and $y$ the number of leaves per tree.   }
	\label{table:widevsdense}
\end{table}


\smallskip
\noindent \textbf{Sensitivity analysis and pruning.}
In our experiments, dense models do not reach the performance of ensembles of regression trees scored with QuickScorer. We now address the problem by leveraging the advantages brought by \emph{model compression}, in particular by \emph{network pruning} \cite{DBLP:journals/corr/HanPTD15,DBLP:journals/corr/GuoYC16}, a technique that deeply sparsifies a neural model without incurring in performance degradation. Let us consider the time budget of $3 \mu s $:
we devise a model which exceeds the time budget but affords an NDCG@10 close the $300$ trees model. By mean of pruning, we can move to the sparse domain and benefit of fast Sparse Dense Matrix Multiplication routines (SDMM). As example model, we pick a 400$\times$200$\times$200$\times$100 network: its performance are reported in Table~\ref{table:sparse_400x200x200x100_partial}.
As detailed in Section \ref{subsec:modelcompr}, \textit{magnitude-based} pruning methods deliver high compression rates without accuracy loss. We adopt this family of pruning techniques to sparsify the parameters of our model. Recall that magnitude pruning technique zero-out a given amount of low absolute value weights. The amount of zeroed-out values determines the aggressiveness of the sparsification: the way this aggressiveness is controlled distinguishes between level pruning and threshold-based pruning.
In the case of level pruning, we can explicitly set the sparsity target, \emph{e.g.}, 70\%.
%
%
In the threshold-based magnitude pruning by Han \textit{et al.}~\cite{DBLP:journals/corr/HanPTD15}, instead, we need to chose a statistical based threshold, as detailed in Section~\ref{sec:related}. The choice is generally based on the \textit{sensitivity} of each layer, namely  the property that describes a layer's resistance to sparsification.
We perform two kind of sensitivity analysis: \textit{static} and \textit{dynamic}. Both procedures prune a growing percentage of weights in each layer, one layer at a time, and evaluate the behavior of the partially-pruned model on the validation set. In the static version, there is no re-training~\cite{DBLP:journals/corr/HanPTD15} of the weights that survived the pruning in the chosen layer and the weights in the other layers, while in the dynamic version re-training is applied.
Static analysis is reported on the left side of Figure~\ref{fig:sensitivity}. The sensitivity of each layer appears to decrease as we go deeper into the network, meaning that the first layers suffer the most from sparsification. Dynamic analysis (right side of Figure~\ref{fig:sensitivity}) shows an inverse trend and highlights a peculiar behavior of the first layer: high levels of sparsity in this layer allow the pruned model to outperform the dense one in terms of NDCG@10.  This is an example of a model compression technique acting as regularizer \cite{DBLP:journals/corr/HanPTD15,DBLP:journals/corr/ZhouYGXC17}.
T%his effect is called regularization, \textit{i.e.}, a growth in terms of generalization capabilities by the model. Pruning techniques can induce this feature since they can be seen as an inference-time version of dropout~\cite{srivastava2014dropout}, \textit{i.e.}, a well-known regularization technique.
This effect is especially evident in the first layer as it presents the weights with the largest absolute values among all the other network layers. Observe that the effect of matrix multiplication is dominated by large absolute value entries, and the larger are the values, the larger is their impact. So a reduced number of high absolute weights can well approximate the overall result of matrix multiplication. From a learning point of view, since the network is working on handcrafted features, the sparsification selects just the essential combinations of input features.

\begin{figure}[t]
\begin{minipage}[b]{0.5\columnwidth}
\includegraphics[width=\columnwidth]{imgs/static_sensitivity.png}
\centering 
\caption*{\footnotesize{Static}}
\end{minipage}%
\begin{minipage}[b]{0.5\columnwidth}
\includegraphics[width=\columnwidth]{imgs/dynamic_sensitivity.png}
\centering 
\caption*{\footnotesize{Dynamic}}
%\subcaption{Another subfigure}
\end{minipage}%
\caption{Static and Dynamic Sensitivity Analysis for a 400$\times$200$\times$200$\times$100 network on the \msn dataset.}
\label{fig:sensitivity}
\end{figure}


\begin{table}[b]
	\centering
	\adjustbox{max width=\columnwidth}{
	\begin{tabular}{lR{0.6cm}R{0.6cm}R{0.6cm}R{0.6cm}R{0.6cm}}
		\toprule
		\multirow{2}{*}{Model} & \multicolumn{5}{c}{\footnotesize{Relative Execution Time per Layer (\%)}} \\
		\cmidrule{2-6}
		 & \nth{1} & \nth{2}& \nth{3} &\nth{4}&\nth{5} \\
		\midrule
		400$\times$200$\times$200$\times$100 & \textbf{35} & 33 & 20 & 10 & 2   \\
		100$\times$50$\times$50$\times$10 & \textbf{60} & 21 & 14 & 3 & 2 \\
		200$\times$100$\times$100$\times$50 & \textbf{45} & 28 & 17 & 8 & 2 \\
		\bottomrule
	\end{tabular}
	 }
	\caption{Breakdown of the relative execution time among different layers for different neural models. }
	\label{table:breakdown}
\end{table}
Pruning techniques were originally developed to reduce the size of pre-trained models~\cite{DBLP:journals/corr/HanPTD15,DBLP:journals/corr/GuoYC16}. %Hence, all network's layers are pruned, even if final sparsities can be low because of high sensitivity of some layers. 
Despite, in our context we aim at speeding up the forward pass without incurring in performance degradation. 
This induces us to consider each layer's relative impact on the inference step before applying a pruning technique. In Table~\ref{table:breakdown}, we report a breakdown of the execution times among different layers in different architectures. Observe that the most time-consuming layer is always the first one, even if the largest matrix is the one storing the second layer weights, as for the 400$\times$200$\times$200$\times$100 network. 
Applying bias and ReLU6, in fact, causes the output matrix of the first layer to be brought into the cache, where it resides there during the computation of the second layer. 
Observe also that it is sufficient to reduce the execution time of one of the first two layers to match the time budget  of $3 \mu s$. By using our sparse time predictor we can infer the required sparsity to obtain a given speedup. In Figure~\ref{fig:sparsespeedup}, we draw the sparsity-speedup curve for some matrices, representing the first layers of different architectures. Even if the dense first layer usually has a major impact on the overall execution time, the quadratic growth of the sparse speedup in the selected range annihilates its contribution after the sparsification. For example, in the 400$\times$200$\times$200$\times$100 architecture, the impact of the first layer in the dense version is about $35\%$, while at $95\%$ of sparsity, the estimated speedup using sparse multiplication is $10$x, meaning that the first layer after pruning becomes the second less time-consuming layer after \textit{fc5}. 



\begin{figure}[t]
\centering
\includegraphics[width=\columnwidth]{imgs/sparse_speedup.png}
\caption{Matrix multiplication speedup at various levels of sparsity estimated with our sparse time predictor. We assume the number of active columns/rows to be equal to the total number of columns/rows (worst-case scenario).\label{fig:sparsespeedup}}
\end{figure}

\smallskip
\noindent \textbf{Outperforming tree-based models}.
By jointly harvesting 1) the prominent impact of the first layer on the total execution time, and 2) the regularization effect of pruning the first layer, we develop our \textit{ early-layers efficiency-oriented pruning}. We apply the threshold-based magnitude pruning using the Distiller framework~\cite{nzmora2019distiller}, a deep learning compression framework developed by Intel. This pruning technique generally offers more flexibility and better performance with respect to level pruning. We prune only the first layer in an aggressive fashion and we fine-tune its surviving entries and all the weights of the other layers.
In our final model, the first layer is $98.7\%$ sparse, meaning that there are about $700$ surviving non-zero weights in the first layer, out of 54400 ($400 \times 136$) in the dense matrix.
We use our sparse time predictor to compute the execution time. The speedup obtained with this sparsity ratio on the multiplication of the first layer is about $25$x. This means that the impact of the first layer, which previously amounted to about the $35\%$, is negligible.
In Table~\ref{table:sparse_400x200x200x100_partial} we report the comparison between tree-based models and neural models. While the dense model did not offer any advantages with respect to the tree-based models, the hybrid model - first layer sparse, other layers dense - is both the fastest and the most accurate model. For example, at the same NDCG@10 value, it is $3.2$x faster than the $878$-trees model. 


\begin{table}[t]
	\centering
	\begin{tabular}{llrr}
		\toprule
		Model & Description & NDCG@10 & Sc. Time ($\mu s$/doc) \\
		\midrule
		\multirow{3}{*}{QuickScorer} & 878 trees & $\uparrow$ \textbf{0.5246} & 8.2 \\
		& 500 trees & 0.5240 & 4.9 \\
		& 300 trees & 0.5230 & 3.0 \\
		\midrule
		\multirow{2}{*}{Neural} & Dense & 0.5222 & 3.8 \\
		& Sparse & $\uparrow$ \textbf{0.5246} & \textbf{2.6} \\ 
 		\bottomrule
	\end{tabular}
	\caption{Dense and sparse neural models (400$\times$200$\times$200$\times$100)  vs QuickScorer in terms of NDCG@10 and Scoring Time (Sc. Time). $\uparrow$ indicates statistically significant improvement w.r.t. models of the same family (Fisher's randomization test,  $p < 0.05$).\label{table:sparse_400x200x200x100_partial}}
\end{table}

% !TEX root = paper.tex
% !TeX spellcheck = en_US

\section{Experiments}
\label{sec:experiments}
In this section, we provide an extensive evaluation of our methodology to design, train and sparsify neural models for the document scoring task. In particular, we compare them against tree-based models at different points of the efficiency-effectiveness trade-off. Throughout this article, we have used the \msn dataset as use case. We now complement our evaluation with the \istella dataset~\cite{dato2016fast}. 
First, we present the experimental setup. Then, we report our experimental results and we show that neural models obtained with our technique can outperform ensembles of trees. To ease the reproducibility of the results presented in this article, code and trained models have been made publicly available\footnote{\url{https://github.com/hpclab/efficient_nn_for_ltr}}.


\vspace{-.3cm}
\subsection{Experimental Setup}
\label{subsec:expsetup}
We perform our experiments on two datasets: \istella and \msn. The \istella dataset~\cite{dato2016fast} consists of a collection of $33$,$018$ queries with an average of $103$ documents per query. Each document-query pair is represented by $220$ features. The \msn (Fold 1) dataset, which we already introduced, is composed by more than $30$,$000$ queries, with about $120$ documents per query and 136 features per document-query pair. In both the dataset, document-query pairs are labeled with $5$-graded relevance judgments ranging from 0 (irrelevant) to 4 (perfectly relevant).
Both datasets are split in train-validation-test according to a 60\%-20\%-20\% criterion. 

\smallskip
\noindent \textbf{LambdaMART models}. We employ the LightGBM framework~\cite{NIPS2017_6907} to train ensembles of regression trees using the LambdaMART algorithm. For each training, we perform hyper-parameter tuning using the HyperOpt library \cite{bergstra2013making}.
In particular, we determine the optimal combination of the following set of hyper-parameters: \texttt{learning rate, max depth, min\_sum\_hessian\_in\_leaf, min\_data\_in\_leaf}. 
%\fnote{sopra: non mi piacciono scritti così. se sono parametri di un algo, si mettono courier. altrimenti si riportano discorsivi con font normali ma spiegando che sono. ti torna?}
To avoid overfitting, we apply an early stopping criterion on the validation loss every 100 trees. We train 64-leaves model as target model to compare against neural networks and 256-leaves models to use as teachers. The latter models offer higher retrieval performance while being $4$x slower, which is not suitable for the use in latency-bounded applications. We score the LambdaMART models using a C++ implementation of QuickScorer that exploits instruction-level parallelism by using AVX2 instructions~\cite{8035185}.

\smallskip
\noindent \textbf{Neural Networks}. We train neural models (\textit{students}) to approximate the scores of top-performing regression forest (\textit{teacher}), accordingly to the knowledge distillation \cite{ba2014deep} paradigm, detailed in Section~\ref{subsec:approxbetter}. Models are trained using Pytorch~\cite{NEURIPS2019_9015}, adopting the same strategy for randomly generating training data of Cohen \textit{et al}.~\cite{cohen2018universal}. We employ RELU6 as activation function after every linear layer, except for the last one, where $\text{RELU6}(x) = min(max(x,0), 6)$. 
%We use the Distiller~\cite{nzmora2019distiller} framework to prune the models. Both in training and pruning, we employ Adam~\cite{kingma2014adam} as optimizer, with learning rate $0.001$ and no weight decay. 
%Table~\ref{table:neuraltrainparams} summarizes the other training and pruning hyper-parameters, which are dataset-dependent. $E_t$ represents the number of training epochs. The pruning phase is composed of $E_p$ epochs of pruning/fine-tuning and of $E_{ft}$ epochs of solely fine-tuning. Both in training and pruning, we scale the learning rate by multiplying it by $\gamma$ at the epochs specified by $\gamma_{step}$. Dropout, if present, is applied only after the first layer. The neural forward pass  is implemented in C++, using the \textit{dnnl\_sgemm} routine from the OneDNN\footnote{\url{https://github.com/oneapi-src/oneDNN}} framework for  dense matrix multiplication and the LIBXSMM~\cite{heinecke2016libxsmm} C++ library for sparse-dense matrix multiplication (after pruning).
%We train for $E_t$ epochs in standard training; the pruning phase, instead, is composed of $E_p$ epochs of pruning/fine-tuning and of $E_{ft}$ epochs of solely fine-tuning. Both in training and pruning, we scale the learning rate by multiplying it by $\gamma$ at the epochs specified by $\gamma_{step}$. Dropout, if present, is applied only after the first layer. The values of $E_t$,$E_p$, $E_{ft}$, $\gamma$, $\gamma_{step}$ and Dropout for the different datasets are specified in Table~\ref{table:neuraltrainparams}.  }
We use the Distiller~\cite{nzmora2019distiller} framework to prune the neural networks. Both in training and pruning, we employ Adam~\cite{kingma2014adam} as optimizer, with learning rate $0.001$ and no weight decay. Table~\ref{table:neuraltrainparams} summarizes the other training and pruning hyper-parameters, which are dataset-dependent. $E_t$ represents the number of training epochs. The pruning phase is composed of $E_p$ epochs of pruning/fine-tuning and of $E_{ft}$ epochs of only fine-tuning, as done by Han \textit{et al.
\cite{DBLP:journals/corr/HanPTD15}}.
Both for training and pruning, we scale the learning rate by multiplying it by $\gamma$ at the epochs specified by $\gamma_{step}$. Dropout, if employed (see Table~\ref{table:neuraltrainparams}), is applied only after the first layer.
When training and pruning the neural models, we always distill from the most effective ensemble of regression trees for the current dataset. On \msn, it is a model with 600 trees and 256 leaves per tree, reaching 0.5291 of NDCG@10, while on \istella it is a forest with $2500$ trees with $256$ leaves per tree, reaching 0.7821 of NDCG@10.	
 The neural forward pass is implemented in C++. We use the \textit{dnnl\_sgemm} routine from the OneDNN framework for dense matrix multiplication and the LIBXSMM~\cite{heinecke2016libxsmm} C++ library for sparse-dense matrix multiplication (after pruning).

\begin{table}[htb]
	\centering
	%\adjustbox{max width=\columnwidth}{
	\begin{tabular}{lrrrrrr}
		\toprule
		Dataset & $E_t$ & $E_p$& $E_{ft}$& $\gamma$& $\gamma_{step}$ & Dropout \\
		\midrule
		\msn & 100 & 80 & 20 & 0.1 & $ 50, 80 $ & - \\
		\istella & 250 & 60 & 190 & 0.5 & $ 90, 130, 180 $ & 0.1 \\
		\bottomrule
	\end{tabular}%}
	\caption{Training and pruning parameters employed for neural networks on \msn and \istella.\label{table:neuraltrainparams}}
\end{table}

\smallskip	

\noindent \textbf{Experimental Methodology}. We experimentally evaluate the performance of neural networks and ensemble of regression trees on two different experimental scenarios:

\begin{itemize}
	\item \textit{High-Quality Retrieval}: this scenario covers use cases where high-precision retrieval is required, even at the price of a larger scoring time. We impose a constraint on the retrieval quality to our models, specified by a threshold on the ranking metric. As threshold, we choose the 99\% of the retrieval quality of the top performing tree-based competitor on each dataset.
	\item \textit{Low-Latency Retrieval}: this scenario is orthogonal to the previous one as it focus on the efficiency of the retrieval process. We specify a maximum per-document scoring time and we select only the models that can match it. For both datasets, we set the maximum per-document scoring time to be $0.5\mu s$.
	%We pick so so that this scenario does not overlap with the previous one, enriching the comparison with new portion of the effectiveness-efficiency space. In fact, comparing Figures~\ref{fig:hq},~\ref{fig:lowlat} we can note how the scenarios are complementary on both the datasets.}
\end{itemize}
We perform the comparison between neural models and ensemble of regression trees by considering one scenario at a time. For each dataset, we consider the Pareto frontier of ensembles of tree-based models respecting the constraint of the considered scenario (green lines in Figures~\ref{fig:hq},~\ref{fig:lowlat}). By doing so, we train several tree-based competitors at different efficiency-effectiveness trade-offs. We then apply our technique and we show that neural networks can outperform ensembles of regression trees. We employ our time predictors to train and prune only neural network models that fit the time budget constrained by the ensembles of tree-based models considered. We recall that our methodology allows to train a neural model and to prune its first layer. In fact, in Section~\ref{sec:neuraleng}, we demonstrate that the first layer has a prominent impact on the overall execution time. By zeroing out at least  95\% of the parameters, its impact becomes negligible (Figure~\ref{fig:sparsespeedup}). Furthermore, the sparsification of the first layer has a positive effect on the generalization capabilities of the model as it act as a regularizer.
Then, we forecast the overall execution time by subtracting the contribution of the dense first layer from the overall execution time. Both times can be estimated with our dense predictor with no computational effort.

% due to 1) the positive regularization effect of sparsifing the first later and 2) the prominent impact of the first layer on the overall execution time.
%\fnote{frase sopra non ho capito. FM}
%Hence, we predict the overall execution time and the time spent on the first layer with our time predictor. As reported in Figure~\ref{fig:sparsespeedup}, if the sparsification is aggressive enough, the time spent to perform the sparse-dense multiplication of the first layer is negligible.

\smallskip
\noindent \textbf{Experimental Platform}. 
All the inference algorithms are compiled with GCC 9.2.1 with the \texttt{-O3} compiler option. 
Scoring times are measured on a Intel i9-9900K CPU clocked at 3.5 GHz, with AVX2 instructions, with a L1-cache of 256KiB, a L2-cache of 2 MiB, and a L3-cache of 16MiB. All the scoring experiments have been performed in single-thread execution.

%\fnote{mancano dettagli di compilazione e compilatore usato. vedi quickscorer. FM}

%This allows to train exclusively the models which strictly respect the time requirements. 
% In this section we provide an extensive evaluation of our proposed methodology. In particular, we extend the experimental results on the \msn dataset provided throughout the paper with an analysis performed on the \istella learning to rank dataset~\cite{lucchese2016exploiting}. We present our results by taking into account different efficiency-effectiveness tradeoffs. We start by introducing the datasets and the training methodology employed. We then describe our experimental methodology. Finally, we report our experimental results and we show that neural models can outperform ensembles of trees. 

% \subsection{datasets and Experimental Setup		}
% \fnote{non mi piace training parameters sopra.}

%In this section, we present the datasets on which we test our methodology and detail the training and pruning parameters for the neural models. 

% \smallskip
% \noindent \textbf{\msn}. We extend our evaluation on the \msn dataset~\cite{DBLP:journals/corr/QinL13}, a well-established benchmark to monitor the evolution of LtR techniques. The dataset is composed by more than 30,000 queries, with about 120 documents per query. Each document is represented by a vector of 136 features. The dataset is split into train-validation-test according to a 60\%-20\%-20\% criterion.
% \cosimo{We employ the LightGBM framework~\cite{NIPS2017_6907} to train the ensembles of regression trees. For each training, we consider the following set of hyper-parameters: [\textsl{learning rate, max depth, min\_sum\_hessian\_in\_leaf, min\_data\_in\_leaf}]. The  HyperOpt library \cite{bergstra2013making} is used to explore the hyper-parameters space and infer the optimal combination. We limit the number of trees to 1500 (larger models do not provide better retrieval quality) and apply an early stopping criterion on the validation loss. }
% %\cosimo{To train the best possible tree-based model on this dataset, we employ the LightGBM framework~\cite{NIPS2017_6907} to perform a grid search over the following set of parameters: [\textsl{learning rate, max depth, min\_sum\_hessian\_in\_leaf, min\_data\_in\_leaf}], using the HyperOpt library \cite{bergstra2013making} to infer the optimal combination. We limit the number of trees to 1500 and apply an early stopping criterion on the validation loss.}
% The best model we obtain has 600 trees and 256 leaves per tree, reaching 0.5291 of NDCG@10. 
% The neural networks are trained using Adam~\cite{kingma2014adam} as optimizer, with learning rate $0.001$, without dropout nor weight decay. We train for $100$ epochs,  multiplying the learning rate by $\gamma = 0.1$ at epochs $\{50, 80\}$. The overall pruning procedure lasts $100$ epochs. We prune once and retrain for $10$ epochs for the first $80$ epochs, then we fine-tune for the last $20$. During the fine-tuning epochs, all hyper-parameters, i.e., learning rate, weight decay, etc., are the same of the training phase. We employ an aggressive pruning on larger models for \emph{high-quality retrieval} by setting the sensitivity parameters $s \in [1.0, 1.5]$. For \emph{low-latency retrieval}, we use smaller models and we set $s \in [0.5, 0.8]$ since aggressive pruning cause performance degradation.

% \smallskip
% \noindent \textbf{\istella}. We also experiment using the \istella~\cite{lucchese2016post} \textit{small} (\istella) dataset. It consists of a collection of $33$,$018$ queries with an average of $103$ documents per query. Each document-query pair is represented by $220$ features. The dataset is splitted in train-validation-test according to a 60\%-20\%-20\% criterion. 
% \cosimo{We conduct a grid search using LightGBM and HyperOpt, on the same set of parameters detailed for \msn; this time we limit the number of trees to 2500.}
% The best model we could find has $2500$ trees with $256$ leaves per tree, reaching 0.7821 of NDCG@10. Observe that the size of this model is nearly the triple w.r.t. to the top performing model on \msn, suggesting that this is a more challenging dataset. 
% Neural models are trained using Adam~\cite{kingma2014adam} as optimizer, with learning rate $0.001$ without weight decay. Dropout is applied exclusively on the first layer.
% We train for $250$ epochs, multiplying the learning rate with $\gamma = 0.5$ at epochs $\{90, 130, 180 \}$. The pruning phase consists of a total of $250$ epochs: $60$ of pruning-retraining, with $10$ epochs of fine-tuning between one pruning iteration and the successive, and $190$ of solely fine-tuning. The training hyper-parameters remain the same as in the previous training phase. As for \msn, we employ aggressive pruning for larger models, $s \in [1.0, 1.5]$, and $s \in [0.5, 0.8]$ for smaller ones.


% \subsection{Methodology}
% \label{subsec:expmeth}
% We now present our experimental methodology to compare ensemble of regression trees with neural networks on the \emph{ad-hoc} document retrieval task \missingcite{}. \cosimo{Each experiment has the following macro-structure: first, we draw the Pareto optimality curve for tree-based models on a portion of the efficiency-effectiveness cartesian space. This portion is selected accordingly to one of two different requirement: \emph{high-quality retrieval} and \emph{low-latency retrieval}. Then, we develop our neural models to compete with ensemble of regression trees and QuickScorer.}

%  First, we pick a requirement in terms of scoring time or effectiveness. Then we train several ensembles of regression trees and score them with QuickScorer. This is preparatory to draw the Pareto optimality curve for the tree-based models in the portion of the effectiveness-efficiency trade-off identified by the chosen requirement. We use  the LightGBM framework~\cite{NIPS2017_6907} to train the ensembles of regression trees, with an early stopping criterion on $100$ iterations and the latest QuickScorer code~\cite{8035185}, written in C++ and exploiting AVX2 instructions, to score them. The largest value we permit for the number of tree leaves is $64$, to keep the execution time contained, as explained in section~\ref{subsec:approxbetter}. 



% \fnote{perche' facciamo la roba sopra? prima diciamogli cosa vogliamo fare e poi come lo facciamo. non si capisce la prima parte relativamente al cosa...}

% Before moving to the design of neural models, we train the best tree-model as possible to use as \textit{teacher} in Forest Distillation. Using the LightGBM Framework, we perform a grid search on the hyper-parameters, with no limits on the number of leaves. 
% Once we got the model, NNs are always trained to approximate the scores of this model. Observing the Pareto optimality curve of the ensembles of regression trees, we can derive the time requirements to compete with them. By mean of our dense time predictor, we easily devise the neural architectures matching these requirements. Successively, we train and evaluate them. Models are trained using Pytorch~\cite{NEURIPS2019_9015}, a widely-adopted python machine learning framework. We employ the same strategy for randomly generating training data of Cohen \textit{et al}.~\cite{cohen2018universal} and  RELU6 as activation function, where $\text{RELU6}(x) = min(max(x,0), 6)$.
% %
% The forward pass is implemented in C++, using the \textit{dnnl\_sgemm} routine from the OneDNN framework to perform matrix multiplication.
% CPU experiments are conducted on a Intel i9-9900K CPU, with AVX2 (latest generation of vectorized instructions supported by QuickScorer) instructions, 3.5 GHz, with L1-cache 256KiB, L2-cache 2 MiB, L3-cache 16MiB. 

% If dense models can not compete with ensemble of regression trees, we move to the sparse domain to overcome their limits. 	
%  We prune exclusively the first layer, because 1) its positive regularization effect on the effectiveness and 2) its prominent impact on the overall execution time.
%  We can estimate the execution time of the hybrid model by jointly applying our time predictors. We use the dense time predictor to predict the overall execution time and the time spent on the first layer. As reported in Figure~\ref{fig:sparsespeedup}, if the sparsification is sufficiently aggressive, the time spent to perform the sparse-dense multiplication of the first layer is negligible. Thus, we can forecast the overall execution time by subtracting the contribution of the dense first layer from the overall execution time. This allows to train exclusively the models which strictly respect the time requirements. 

%  The pruning techniques are applied using the Distiller~\cite{nzmora2019distiller} python framework for model compression, and the sparse dense multiplication induced by the sparsification of the weight matrix is carried out with the LIBXSMM~\cite{heinecke2016libxsmm} C++ library. 

% We propose two different requirements on both the datasets:
% \begin{itemize}
% 	\item \textit{High Quality Retrieval}: we compare tree-based models and neural models whose NDCG@10 falls in the 1	\% of the top performing model tree model with $64$ leaves
% 	\item \textit{Low Latency Retrieval}: models whose scoring time is below $0.5 \mu s $ per document
% \end{itemize}


% \fnote{fino a sez sopra. diciamo un sacco di cose ma serve una chiara intro di cio che vogliamo fare e poi diciamolo spiegandolo bene in ogni punto. rivedi in questo senso. FM}
\vspace{-0.2cm}
\subsection{Results}

\begin{figure}[t]
\begin{minipage}[b]{0.5\columnwidth}
\includegraphics[width=\columnwidth]{imgs/msn30k_final_stretched_broken.png}
\caption*{\footnotesize{\msn}}
\end{minipage}%
\begin{minipage}[b]{0.517\columnwidth}
\includegraphics[width=\columnwidth]{imgs/istella_final_stretched_broken.png}
\caption*{\footnotesize{\istella}}
\end{minipage}%
\caption{Comparison between neural networks and ensemble of regression tree on the \textit{high-quality retrieval} scenario.\label{fig:hq}}
\end{figure}
%\smallskip
\noindent \textbf{High-Quality Retrieval}.
The first scenario of our comparison involves models delivering high-quality ranking. As previously detailed, we consider a model (both neural and tree-based) to be in the high-quality ranking region if its NDCG@10 is at least the 99\% of the top-quality tree model with 64 leaves.
By following the experimental methodology described above, we first construct the Pareto frontier for the ensemble of regression trees (green line in Figure\ref{fig:hq}).
We then move to the design of the neural network models. We can estimate the execution time of a neural model whose first layer is sparse with our time predictors. In particular, in Table~\ref{table:msn_final} we report the estimated execution time for the dense architecture, the relative impact of the first layer on the overall execution time, and the predicted execution time after pruning the first layer.
We always assume the sparsity of the first layer in the final model to be above $95\%$, so that its impact on the overall execution time is negligible. Our experiments show that this level of sparsity does not hamper the ranking capability of the model.
Observe that our time predictors permit to locate a neural model on the $y$-axis of the effectiveness-efficiency plot without any computational effort, analytically computing it given the architectures of network. 
Once we have designed our models to compete with the tree-based ones, we train and prune them, according to the methodology listed in Section~\ref{subsec:expsetup}.
\begin{table}[b]
	\centering
	\adjustbox{max width=\columnwidth}{
	\begin{tabular}{llrrrr}
		\toprule
		\multirow{2}{*}{Dataset} & \multirow{2}{*}{Model} & Sc. Time  & \nth{1} layer & Predicted Pruned   \\
		&& ($\mu s$/doc) & impact (\%)& Sc. Time  ($\mu s$/doc) \\
		\midrule	
		\multirow{3}{*}{\msn} & 300$\times$200$\times$100 & 2.4 & 30 &  1.7 \\
		&\footnotesize 200$\times$100$\times$100$\times$50 & 1.3 & 39 & 0.8 \\
		&200$\times$50$\times$50$\times$25 & 0.9 & 58 & 0.4\\
 		\midrule	
 		\multirow{3}{*}{\istella} & \footnotesize800$\times$400$\times$400$\times$200& 11.9 & 23 &  9.1 \\
		&\footnotesize800$\times$200$\times$200$\times$100	& 6.5 & 41 & 	3.8 \\
		&300$\times$200$\times$100 & 2.8 & 41 & 1.6	\\
		\bottomrule
	\end{tabular}}
	\caption{Prediction of model scoring time (Sc. Time) when pruning the first layer, in \emph{High Quality Retrieval}.  }
	\label{table:msn_final}
\end{table}

Figure~\ref{fig:hq} illustrates the comparison on a effectiveness-efficiency plot between neural models and ensemble of regression trees scored with QuickScorer. On the $x$-axis we report the NDCG@10 on the test set, and on the $y$-axis the scoring time per document in $\mu s$. First, we observe that the predicted times reported in Table~\ref{table:msn_final} coincide with real scoring time, confirming the precision of our theoretical approach. Hence, our methodology allows to train exclusively the required architectures. Secondly, neural models can outperform tree-based models in scoring documents, both in terms of effectiveness and efficiency. The neural Pareto-optimality, reported in blue in  Figure~\ref{fig:hq}, lays below the tree-based one (in green), either on the \msn dataset and on \istella. On the \msn dataset, for example, the 300$\times$200$\times$100 architecture is $4.4$x faster than the $878$-trees model and it also provides a higher retrieval quality. Furthermore, the 200$\times$50$\times$50$\times$25 architecture is the fastest model respecting the quality constraint on this dataset. The same consideration holds for \istella, where the fastest model respecting the imposed quality constraint is a neural network (400$\times$200$\times$200$\times$100).
On this dataset, neural models still outperform ensemble of regression trees on a large portion of the selected effectiveness-efficiency space, even if tree-based model deliver a slightly superior performance in the top performing region.
This leaves space for research work to further improve the quality of this approximation. 
%For the sake of fairness, we point out that struggling at high level of retrieval quality is a general drawback of neural models, but it is also the only one, as detailed discussed in Section~\ref{sec:conclusions}.	
%Considering the top performing models on the \msn da the most effective models: ERT $600, 64$, NDCG@10 $0.5291$,  and $300x200x100$ sparse at the $98.6$ in the first layer, NDCG@10 $0.5258$. The gap between the two models in terms of NDCG@10 suggests than when aiming to maximize the retrieval quality, tree-based models are still to prefer to NNs, even if the price can be a $10x$ larger execution time w.r.t. to neural models.

% Observe that the models respect the predicted execution time reported in Table~\ref{table:\msn_final}, confirming the validity of our theoretical approach. Observe also  that neural models outperform the ensemble of regression trees scored with QuickScorer on \msn at any point of the efficiency-effectiveness trade-off; this can be evicted by fact that the neural Pareto curve always stands below the tree Pareto curve. The difference is especially evident between the two top NDCG@10 models. The 300x200x100 architecture is $4.4x$ faster than the 878-trees model, and $0.002$ more accurate in terms of NDCG@10, which is a relevant gap in terms of retrieval quality. For the sake of fairness, we point out the only possible disadvantage of using neural models; let us consider the most effective models: ERT $600, 64$, NDCG@10 $0.5291$,  and $300x200x100$ sparse at the $98.6$ in the first layer, NDCG@10 $0.5258$. The gap between the two models in terms of NDCG@10 suggests than when aiming to maximize the retrieval quality, tree-based models are still to prefer to NNs, even if the price can be a $10x$ larger execution time w.r.t. to neural models. 
% Then, we train an prune them; the most successful training configuration we found is to train for $250$ epochs,  using the dropout with  $p=0.1$ just on the first layer, no weight decay, and scheduling to multiply the learning rate with $\gamma = 0.5$ at epochs $\{90, 130, 180 \}$. The pruning phase consist of a total of $250$ epochs:  $60$ of pruning-retraining, with $10$ epochs of fine-tuning between one pruning iteration and the successive, and $190$ of solely fine-tuning. The training hyper-parameters remain the same as in the previous training phase. 
% In Figure~\ref{fig:\istella_high_quality} we report the comparison between the neural and tree-based Pareto Optimality curve. Even on a complex dataset as \istella, the neural models results more convenient on a large portion of the curve. In particular, NNs always to be preferred when the NDCG@10 is below $0.7220$. The tree-based models instead results advantageous with top-quality effectiveness requirements, \textit{i.e.}, NDCG@10 $ > 0.7740$, which NNs struggles to obtain. Observe also that the top performing ensemble of regression trees, ERT $2500, 256$, NDCG@10 0.7821, largely outscores our top performing model, the $800x400x400x200$, NCGG@10 0.7734 . 





\smallskip
\noindent \textbf{Low-Latency Retrieval}. We now compare neural models and ensemble of regression trees on a low-latency retrieval setting, \textit{i.e.}, a scenario requiring the scoring time to be lower than $0.5 \mu s$ per document. The Pareto Optimality curve for the ensemble of regression trees is drawn in green in Figure~\ref{fig:lowlat}. We use this plot to identify the latency constraints for our neural networks. Our proposed methodology permits to precisely estimate the execution time of a model, before carrying out the costly training-pruning phase. In Table~\ref{table:msn_lowq}, we demonstrate the usage of our methodology. As we did for the \emph{high-quality retrieval} use case (Table~\ref{table:msn_final}), we report the predicted execution time for the dense architecture, the relative impact of the first layer and the predicted time after sparsification. We still consider the impact of the first layer to be negligible.
%The models involved in this latency-bound use case are smaller, in order to match the latency constraints. Anyway, we experimentally verify that the pruning technique is capable to sparsify the first layer up to 95\%, annihilating the contribute of the first layer to the overall execution time. }
\begin{table}[b]
	\centering
	\adjustbox{max width=\columnwidth}{
	\begin{tabular}{llrrrr}
	%\resizebox{\columnwidth}{!}{
		\toprule
		%\multirow{2}{*}{Model} &    \multicolumn{2}{c}{Scoring Time ($\mu s$/doc)} \\
		%\cmidrule{2-3}
		\multirow{2}{*}{Dataset} & \multirow{2}{*}{Model} & Sc. Time  & \nth{1} layer   & Predicted Pruned  \\
		&& ($\mu s$/doc) & impact (\%)& Sc. Time  ($\mu s$/doc) \\
			\midrule
		%100x75x75x10 & 0.7 & 0.43 & 0.4\\
		\multirow{3}{*}{\msn}&100$\times$50$\times$50$\times$25 & 0.6 & 56 & 0.3\\
		&100$\times$25$\times$25$\times$10 & 0.5 & 71 & 0.2\\
		&50$\times$25$\times$25$\times$10 & 0.3 & 65 & 0.1 \\
 		\midrule
		\multirow{3}{*}{\istella}&  200$\times$75$\times$75$\times$25&1.6 & 61 &  0.6 \\
		&100$\times$75$\times$75$\times$10& 0.9 & 55 & 	0.4 \\
 		&100$\times$50$\times$50$\times$10 & 0.8 & 67 & 0.3	\\
		\bottomrule
	\end{tabular}
	}
	\caption{Prediction of model scoring time (Sc. Time) when pruning the first layer, in \emph{Low-Latency Retrieval}.\label{table:msn_lowq}}
\end{table}


\begin{figure}[t]
\begin{minipage}[b]{0.5\columnwidth}
\includegraphics[width=\columnwidth]{imgs/low_effectiveness_msn30k_stretched.png}
\centering 
\caption*{\footnotesize{\msn}}
\end{minipage}%
\begin{minipage}[b]{0.51\columnwidth}
\includegraphics[width=\columnwidth]{imgs/low_effectiveness_istella_stretched.png}
\centering 
\caption*{\footnotesize{\istella}}
%\subcaption{Another subfigure}
\end{minipage}%
\caption{Comparison between neural networks and ensemble of regression tree on the \textit{low-latency retrieval} scenario.}
\label{fig:lowlat}
\end{figure}



Figure~\ref{fig:lowlat} illustrates the comparison between neural model and ensemble of regression trees when dealing with low-latency constraints. Even in this case, our methodology permits to create neural networks that  outperform ensembles of regression trees. On the \msn dataset, neural models dominate over tree-based models, as happened for the \textit{high-quality} use case. In fact,  the Pareto frontier of neural models (in blue) always lies below the tree-based one, confirming the superiority of our technique on this dataset (left side of Figure~\ref{fig:lowlat}). In particular, the 200$\times$50$\times$50$\times$25 architecture is $3$x faster than the regression forest with $300$ trees and $32$ leaves, while being also more precise in terms of NDCG@10. On the \istella dataset (right side of Figure~\ref{fig:lowlat}, the performance of our neural models can be considered on pair with tree-based models. In fact, the two Pareto frontiers intersect in this portion of the efficiency-effectiveness trade-off. Despite that, neural networks  still provide the most effective model respecting the time requirement (200$\times$75$\times$75$\times$25). 
This dataset confirms to be troublesome for neural models, as witnessed in the high-quality retrieval scenario. 
% We leave for future work the investigation of distillation techniques tackling these difficulties on this specific dataset.
%\fnote{la chiusura con un future work la eviterei. FM}

% Once again, we design our neural model just using our time predictors. 
% We keep the $200x50x50x25$ neural model from the previous experimental setting and devise three more architectures. 

% %mainly composed by trees with $8$ or  $16$ leaves. A minor number of leaves and/or trees, in fact, allows for faster scoring with QuickScorer, at the price of a reduced expressiveness. For what concerns the neural models, observe that 
% %the $200x50x50x25$ model previously developed already matches the requirement of  $ 0.5 \mu s$ of scoring time per document. Furthermore, we develop three more models using the same methodology as before: we estimate the execution time of the pruned models as the total execution time subtracted with the contribution of the first layer. We highlight that the models' design is completely experiments-free, thus costless. 

% In Table~\ref{table:\msn_lowq} we report the predicted execution times after the sparsification; because of the reduced sizes of the layers, we assume our processor to perform at 60 GFLOP/s, as also suggested by Figure~\ref{fig:heatmap}. We train an then prune the resulting models; the training and pruning recipes are exactly the same as in the \textit{High Quality Retrieval} use case. The only difference is that  we use medium or low aggressive pruning ($s \in [ 0.5, 1.0]$) since these architectures have few parameters and aggressively sparsify them could cause unbearable performance degradation. In Figure~\ref{fig:\msn_lowlat} we report both the Pareto Optimality curves. As for the high quality retrieval, the real scoring times reflect the predicted scoring time reported in Table~\ref{fig:\msn_lowlat}. %The very-low effort time analysis allows to train only a reduced number of models, tearing down the resource needed for training, \textit{i.e.}, time, money, energy. 
% Even in this case, the neural Pareto Optimality curve always appears below the tree Pareto Optimality. As an example, the architecture $200x100x100x50$ is $0.002$ more accurate than the $300t, 32l$ tree-based model, while being $3x$ times faster. The validity of our theoretical-based approach to efficiency is confirmed even in this use case, especially considering that the model design phase involves no computational effort, tearing down the resource needed for training, \textit{i.e.}, time, money, energy. 
 %As for \msn, we now move to compare tree-based and neural model whose scoring time is about $ 0.5 \mu s$ per document. The profile of the Pareto Optimality curve for ensemble of regression trees is reported in green in Figure~\ref{fig:lowlat\istella}. Following the same procedure we follow during the all experimental evaluation, we use our time predictor to devise the models that we will train, as reported in Table~\ref{table:\istella_lowlat}. To train and prune the architecture we use the same parameters configuration as in \textit{High Quality Retrieval}; the pruning aggressiveness is reduced because of the small size of the models. We report our results in Figure~\ref{fig:lowlat\istella}: as shown, neural models are capable to compete with tree-based ones even with low-latency requirements on the \istella dataset. 






% \subsection{MSN 30K }
% In this section we report the results of our experiments on the \msn dataset~\cite{DBLP:journals/corr/QinL13}, a well-established benchmark to monitor the evolution of LtR techniques. The dataset is composed  by more than 30,000 queries, with about 120 documents per query; each document is a vector of 136 features. The whole dataset is split into train-validation-test according to a 60\%-20\%-20\% criterion. 

% As already reported in Table~\ref{table:64vs256leaves}, the most effective ensemble of regression trees  we could obtain on \msn is a model with 600-trees and 256 leaves per tree, reaching 0.5291 of NDCG@10. We used this model as \textit{teacher} for all of our neural models. 

% \smallskip
% \noindent \textbf{High Quality Retrieval}
% We begin our comparison between tree-based and neural models on \msn from models whose NDCG@10 is the in the 1\% of the top quality tree model with 64 leaves. 
% Following the guidelines in the preamble at the beginning of this section, we first construct the Pareto Optimality curve for the ensemble of regression trees, reported in green in Figure\ref{fig:\msn_highperf}. Hence, we move to the design of our hybrid sparse-dense neural models; as anticipated in the preamble, we can estimate the execution time of a first layers sparse neural model with our time predictors, as reported in Table~\ref{table:\msn_final}. We always assume the sparsity of the first layer in the final model to be above $95\%$, so that its impact on the overall execution time is negligible. 
% Observe that this method allows us to locate a neural  model on the y-axis of the effectiveness-efficiency plot without any computational effort, just using the architecture of the network.

% Once we have designed our models to compete with the tree-based ones, we train and prune them. In the training phase, additionally to the information specified in the preamble, we highlight that we did not use dropout nor weight decay on our models: we train for $100$ epochs, and  multiply the learning rate by $\gamma = 0.1$ at epochs $\{50, 80\}$. We employ an aggressive pruning ($s=1.5$) for all the models except for the smallest one, for which we use $s=1.0$. The overall pruning procedure lasts $100$ epochs; we prune once and retrain for $10$ epochs for the first $80$ epochs, then we solely fine-tune for the last $20$. During the fine-tuning epochs, the hyper-parameters, learning rate, weight decay, etc..,  are set as in the training phase. In Figure ~\ref{fig:\msn_highperf} we draw the Neural Pareto Optimality curve. Observe that the models respect the predicted execution time reported in Table~\ref{table:\msn_final}, confirming the validity of our theoretical approach. Observe also  that neural models outperform the ensemble of regression trees scored with QuickScorer on \msn at any point of the efficiency-effectiveness trade-off; this can be evicted by fact that the neural Pareto curve always stands below the tree Pareto curve. The difference is especially evident between the two top NDCG@10 models. The 300x200x100 architecture is $4.4x$ faster than the 878-trees model, and $0.002$ more accurate in terms of NDCG@10, which is a relevant gap in terms of retrieval quality. For the sake of fairness, we point out the only possible disadvantage of using neural models; let us consider the most effective models: ERT $600, 64$, NDCG@10 $0.5291$,  and $300x200x100$ sparse at the $98.6$ in the first layer, NDCG@10 $0.5258$. The gap between the two models in terms of NDCG@10 suggests than when aiming to maximize the retrieval quality, tree-based models are still to prefer to NNs, even if the price can be a $10x$ larger execution time w.r.t. to neural models. 

% \begin{table}[]
% 	\centering

% 	\adjustbox{max width=\columnwidth}{
% 	\begin{tabular}{lrrrr}
% 	%\resizebox{\columnwidth}{!}{
% 		\toprule
% 		%\multirow{2}{*}{Model} &    \multicolumn{2}{c}{Scoring Time ($\mu s$/doc)} \\
% 		%\cmidrule{2-3}
% 		\multirow{2}{*}{Model} & Scoring Time  & \nth{1} layer   & Predicted Pruned  \\
% 		& ($\mu s$/doc) & impact (\%)& Scoring Time  ($\mu s$/doc) \\
% 		\midrule
% 		\sc{800x400x400x200}& 11.9 & 0.23 &  9.1 \\
% 		800x200x200x100	& 6.5 & 0.41 & 	3.8 \\
% 		300x200x100 & 2.8 & 0.41 & 1.6	\\
%  		\bottomrule
% 	\end{tabular}
% 	  }
% 	\caption{Predicting the scoring time of Neural Models for High Quality Retrieval on \istella, assuming the first layer to be sparse.  }
% 	\label{table:\istella_final}

% \end{table}


% \begin{figure}
% 	\centering
% 	\includegraphics[width=\columnwidth]{imgs/\msn_final.png}
% 	\caption{High Quality Retrieval comparison between Additive Ensemble of Regression Trees, scored with QuickScorer, and Neural Models on \msn }
% 	\label{fig:\msn_highperf}
% \end{figure}


% \smallskip
% \noindent \textbf{Low Latency Retrieval}
% We now move to the compare neural models and ensemble of regression trees on low latency constraints, \textit{i.e.}, scoring time below $0.5 \mu s$ per document. 	
% The Pareto Optimality curve for the ensemble of regression trees is draw in green in Figure~\ref{fig:\msn_lowlat}. 
% Once again, we design our neural model just using our time predictors. 
% We keep the $200x50x50x25$ neural model from the previous experimental setting and devise three more architectures. 

% %mainly composed by trees with $8$ or  $16$ leaves. A minor number of leaves and/or trees, in fact, allows for faster scoring with QuickScorer, at the price of a reduced expressiveness. For what concerns the neural models, observe that 
% %the $200x50x50x25$ model previously developed already matches the requirement of  $ 0.5 \mu s$ of scoring time per document. Furthermore, we develop three more models using the same methodology as before: we estimate the execution time of the pruned models as the total execution time subtracted with the contribution of the first layer. We highlight that the models' design is completely experiments-free, thus costless. 

% In Table~\ref{table:\msn_lowq} we report the predicted execution times after the sparsification; because of the reduced sizes of the layers, we assume our processor to perform at 60 GFLOP/s, as also suggested by Figure~\ref{fig:heatmap}. We train an then prune the resulting models; the training and pruning recipes are exactly the same as in the \textit{High Quality Retrieval} use case. The only difference is that  we use medium or low aggressive pruning ($s \in [ 0.5, 1.0]$) since these architectures have few parameters and aggressively sparsify them could cause unbearable performance degradation. In Figure~\ref{fig:\msn_lowlat} we report both the Pareto Optimality curves. As for the high quality retrieval, the real scoring times reflect the predicted scoring time reported in Table~\ref{fig:\msn_lowlat}. %The very-low effort time analysis allows to train only a reduced number of models, tearing down the resource needed for training, \textit{i.e.}, time, money, energy. 
% Even in this case, the neural Pareto Optimality curve always appears below the tree Pareto Optimality. As an example, the architecture $200x100x100x50$ is $0.002$ more accurate than the $300t, 32l$ tree-based model, while being $3x$ times faster. The validity of our theoretical-based approach to efficiency is confirmed even in this use case, especially considering that the model design phase involves no computational effort, tearing down the resource needed for training, \textit{i.e.}, time, money, energy. 
 

% \begin{table}[]
% 	\centering

% 	\adjustbox{max width=\columnwidth}{
% 	\begin{tabular}{lrrrr}
% 	%\resizebox{\columnwidth}{!}{
% 		\toprule
% 		%\multirow{2}{*}{Model} &    \multicolumn{2}{c}{Scoring Time ($\mu s$/doc)} \\
% 		%\cmidrule{2-3}
% 		\multirow{2}{*}{Model} & Scoring Time  & \nth{1} layer   & Predicted Pruned  \\
% 		& ($\mu s$/doc) & impact (\%)& Scoring Time  ($\mu s$/doc) \\
% 			\midrule
% 		%100x75x75x10 & 0.7 & 0.43 & 0.4\\
% 		100x50x50x25 & 0.6 & 0.56 & 0.3\\
% 		100x25x25x10 & 0.5 & 0.71 & 0.2\\
% 		50x25x25x10 & 0.3 & 0.65 & 0.1 \\
%  		\bottomrule
% 	\end{tabular}
% 	  }
% 	\caption{Predicting the scoring time of Neural Models for Low Latency Retrieval on \msn, assuming the first layer to be sparse.  }
% 	\label{table:\msn_lowq}

% \end{table}


% \begin{figure}
% 	\centering
% 	\includegraphics[width=\columnwidth]{imgs/low_effectiveness_\msn.png}
% 	\caption{Low Latency Retrieval comparison between Additive Ensemble of Regression Trees, scored with QuickScorer, and Neural Models on \msn }
% 	\label{fig:\msn_lowlat}
% \end{figure}


% \subsection{\istella dataset}
% The second part of our experimental evaluation is performed on the \istella~\cite{lucchese2016post} dataset by Tiscali. In order to ease reproducibility, we pick the the \textit{small} version (\istella); it consists of a collection of 33,018 queries with an average of $103$ documents per query. Each document-query pair is represented by $220$ features. The train-validation-test split	respects the same pattern as \msn, \textit{i.e.}, 60\%-20\%- 20\%. 

% As first step, we perform a grid search on the hyper-parameters to obtain the most effective tree-based model as possible, regardless of efficiency. The best model we could find is a $2500$ trees with $256$ leaves per tree, reaching 0.7821 of NDCG@10. Observe that the size of this model is nearly the triple w.r.t. to the top performing model on \msn, immediately suggesting that this is a more challenging dataset.  


% \smallskip
% \noindent \textbf{High Quality Retrieval} We start the comparison on \istella, as for \msn, from high quality retrieval. The Pareto Optimality curve for the ensembles of regression trees is reported in Figure~\ref{fig:\istella_high_quality}; diversely from \msn, we witness to a dilation of the performance as the number of trees grows. As shown in Figure~\ref{fig:\msn_highperf}, the 300-trees model on \msn performs almost as well as the 500-trees one, with a gap in terms of NDCG@10 of about $0.01$; instead, on \istella, the difference in terms of effectiveness is about $4x$ bigger. This suggest us this difference of accuracy between large and small models to be present also in neural models, namely that we will need larger models w.r.t. to \msn. 

% Applying the same procedure as before, we devise our neural models just leveraging the dense time predictor, as reported in Table ~\ref{table:\msn_final}. Then, we train an prune them; the most successful training configuration we found is to train for $250$ epochs,  using the dropout with  $p=0.1$ just on the first layer, no weight decay, and scheduling to multiply the learning rate with $\gamma = 0.5$ at epochs $\{90, 130, 180 \}$. The pruning phase consist of a total of $250$ epochs:  $60$ of pruning-retraining, with $10$ epochs of fine-tuning between one pruning iteration and the successive, and $190$ of solely fine-tuning. The training hyper-parameters remain the same as in the previous training phase. 

% In Figure~\ref{fig:\istella_high_quality} we report the comparison between the neural and tree-based Pareto Optimality curve. Even on a complex dataset as \istella, the neural models results more convenient on a large portion of the curve. In particular, NNs always to be preferred when the NDCG@10 is below $0.7220$. The tree-based models instead results advantageous with top-quality effectiveness requirements, \textit{i.e.}, NDCG@10 $ > 0.7740$, which NNs struggles to obtain. Observe also that the top performing ensemble of regression trees, ERT $2500, 256$, NDCG@10 0.7821, largely outscores our top performing model, the $800x400x400x200$, NCGG@10 0.7734 . 
% \begin{table}[]
% 	\centering

% 	\adjustbox{max width=\columnwidth}{
% 	\begin{tabular}{lrrrr}
% 	%\resizebox{\columnwidth}{!}{
% 		\toprule
% 		%\multirow{2}{*}{Model} &    \multicolumn{2}{c}{Scoring Time ($\mu s$/doc)} \\
% 		%\cmidrule{2-3}
% 		\multirow{2}{*}{Model} & Scoring Time  & \nth{1} layer   & Predicted Pruned  \\
% 		& ($\mu s$/doc) & impact (\%)& Scoring Time  ($\mu s$/doc) \\
% 		\midrule
% 		800x400x400x200& 11.9 & 0.23 &  9.1 \\
% 		800x200x200x100	& 6.5 & 0.41 & 	3.8 \\
% 		300x200x100 & 2.8 & 0.41 & 1.6	\\
%  		\bottomrule
% 	\end{tabular}
% 	  }
% 	\caption{Predicting the scoring time of Neural Models for High Quality Retrieval on \istella, assuming the first layer to be sparse.  }
% 	\label{table:\istella_final}

% \end{table}




% \begin{figure}
% 	\centering
% 	\includegraphics[width=\columnwidth]{imgs/\istella_final.png}
% 	\caption{High Quality Retrieval comparison between Additive Ensemble of Regression Trees, scored with QuickScorer, and Neural Models on \istella }
% 	\label{fig:\istella_high_quality}
% \end{figure}

% \smallskip 
% \noindent \textbf{Low Latency Retrieval} As for \msn, we now move to compare tree-based and neural model whose scoring time is about $ 0.5 \mu s$ per document. The profile of the Pareto Optimality curve for ensemble of regression trees is reported in green in Figure~\ref{fig:lowlat\istella}. Following the same procedure we follow during the all experimental evaluation, we use our time predictor to devise the models that we will train, as reported in Table~\ref{table:\istella_lowlat}. To train and prune the architecture we use the same parameters configuration as in \textit{High Quality Retrieval}; the pruning aggressiveness is reduced because of the small size of the models. We report our results in Figure~\ref{fig:lowlat\istella}: as shown, neural models are capable to compete with tree-based ones even with low-latency requirements on the \istella dataset. 
% \begin{table}[]
% 	\centering

% 	\adjustbox{max width=\columnwidth}{
% 	\begin{tabular}{lrrrr}
% 	%\resizebox{\columnwidth}{!}{
% 		\toprule
% 		%\multirow{2}{*}{Model} &    \multicolumn{2}{c}{Scoring Time ($\mu s$/doc)} \\
% 		%\cmidrule{2-3}
% 		\multirow{2}{*}{Model} & Scoring Time  & \nth{1} layer   & Predicted Pruned  \\
% 		& ($\mu s$/doc) & impact (\%)& Scoring Time  ($\mu s$/doc) \\
% 		\midrule
% 		200x75x75x25&1.6 & 0.61 &  0.6 \\
% 		100x75x75x10& 0.9 & 0.55 & 	0.4 \\
% 		100x50x50x10 & 0.8 & 0.67 & 0.3	\\
%  		\bottomrule
% 	\end{tabular}
% 	  }
% 	\caption{Predicting the scoring time of Neural Models for Low Latency Retrieval on \istella, assuming the first layer to be sparse.  }
% 	\label{table:\istella_lowlat}

% \end{table}



% \begin{figure}
% 	\centering
% 	\includegraphics[width=\columnwidth]{imgs/low_effectiveness_\istella.png}
% 	\caption{Low Latency Retrieval comparison between Additive Ensemble of Regression Trees, scored with QuickScorer, and Neural Models on \istella }
% 	\label{fig:lowlat\istella}
% \end{figure}


% In Figure~\ref{fig:quick_\msn_overview} , we report the models which define the \textit{lowest trade-off curve }. The NDCG@10 is computed on the test set, using RankEval~\cite{rankeval-sigir17} while the scoring time is computed with QuickScorer,  averaged on 10 runs. We considered as minimum requirement an NDCG@10 of 0.518. Figure~\ref{fig:quick_\msn_overview} clearly explicits the linear relationship between scoring time and number of trees in QuickScorer: as the number of trees doubles, so does the execution time. The same ratio exists also between the number of leaves and the scoring time:  the $(300, 64)$ model has a execution time that doubles the  $(300, 32)$ model. We did not report the execution time of our most effective model, the $(600, 256)$, but we can easily infer it from the $(300, 64)$ architecture: $$ T_{(600, 256)} = 2^3 *  T_{(300, 64)} \simeq 24 \mu  s$$
%  since we have to double the scoring time twice for the number of  leaves and once for the number of trees. 

% \begin{figure}
% 	\centering
% 	\includegraphics[width=\columnwidth]{imgs/quickscorer_\msn_overview.png}
% 	\caption{Performance Overview of Additive Ensemble of Regression Trees and QuickScorer on \msn }
% 	\label{fig:quick_\msn_overview}
% \end{figure}



% %TODO deicidere se spiegare ora o prima l'idea di migliorare la foresta

% We trained four different dense models. Three of them follow the $2l$-$l$-$l$-$\frac{l}{2}$ architectural pattern that we followed in Section~\ref{sec:neuraleng}. We also propose a three layers architecture $3l$-$2l$-$l$. We divide them into \textit{tiny} and \textit{small}  models, as reported in Table~\ref {table:dense_msn}.%  The reduced number of parameters is necessary to compete with trees models in terms of scoring time, but it causes a lack of highly effective models. Furthermore, neural model are not fast enough to be the solution for really high performance scenarios.

% Then, we applied our efficiency-oriented pruning. We prune the first (one or two) layers of the network thus leveraging both the regularization effect and the scoring time speed up of early-layers sparsification. 
% We applied the Han flavor of pruning, whose aggressiveness is defined by the sensitivity parameter $s$. 
% We used an \textit{aggressive} sparsification strategy for \textit{small} network, pruning the first or the first and the second layer with $s=1.6$, while we adopted a softer method for \textit{tiny} networks, pruning just the first layer with $s = 1.0$. 

% In Figure~\ref{fig:\msn_final} we show our experimental results. Our efficiency-oriented pruning allows neural model to outperform additive ensemble of regression trees at any point of the effectiveness-efficiency tradeoff. Our fastest network has the same scoring time as the fastest tree-based model, while being more accurate. The most accurate neural models has higher NDCG@10 than the best Regression Forest, being 4 $\times$ faster. Both the fastest and the most precise model derive from the sparsification of the 300x200x100 architecture, which covers a large part of the tradeoff by itself. The 200x100x100x50 model completes the task being faster and more accurate of the $(150, 64)$ tree model, which was the only one not covered by the 300x200x100 architecture. Even the smallest neural model, benefits from our methodology, but still does not surpass the 0.518 NDCG@10 threshold and it is not represented in the plot.   


% \begin{table}

% \adjustbox{max width = \columnwidth}{

% 	\begin{tabular}{llrr}
% 		\toprule
% 		Model &   Size&  NDCG@10 &    Scoring Time ($\mu S$)\\
% 		\midrule
% 		100x50x50x25& \textit{tiny} 		 &0.5151   &0.5\\
% 		200x100x100x50& \textit{tiny}  &0.5187&   1.4 \\
% 		300x200x100&\textit{small}  & 0.5216&  2.5\\
% 		400x200x200x100& \textit{small} & 0.5221 & 3.8\\
		  
% 		\bottomrule
% 	\end{tabular}
% 	}
% 	\caption{Dense Neural Models on MSN 30K  }
% 	\label{table:dense_msn}
% \end{table}

% %Then, we applied our efficiency oriented pruning. For each model, we applied it on the first layer and on the first and the secon
% \begin{figure}
% 	\centering
% 	\includegraphics[width=\columnwidth]{imgs/\msn_final.png}
% 	\caption{Performance Overview of Neural Network on \msn }
% 	\label{fig:\msn_final}
% \end{figure}


% \subsection{Na\"ive comparison}

% We started reproducing the experimental settings of the original article~\cite{cohen2018universal}. We trained two models, a Large Network with 4 layers of size $\{2000,500,5000,100\}$ and a Small Network with 2 layers of shapes $\{500,100\}$. We adopted the same strategy for randomly generating training data as in the original work~\cite{cohen2018universal}. We used Adam as optimizer, with learning rate $0.001$ and no weight regularization; we multiplied the learning rate by $\gamma = 0.1$ at epochs $\{50, 80 \}$ and use and early stopping criterion on the validation loss. 

% As previoulsy mentioned, we excluded the GPU-based inference from the comparison due to data transfer costs. The CPU inference comparison was originally carried out comparing an old version of Quickscorer\footnote{The original Quickscorer algorithm is undergoing a patent process} - without SIMD instructions - with a Python-based forward implementation for NNS, on different hardware. To provide a fair, production-oriented comparison between the two techniques, we wrote our own C++ version of the Multi Layer Perceptron inference and we used the last Quickscorer code. We expolited the \textit{dnnl\_sgemm} routine from Intel oneDNN framework to implement matrix multiplications, with JIT compilation, always forcing single-thread execution.  
% Experiments were conducted on a Intel i9-9900K CPU, with AVX2 instructions, 3.5 GHz, with L1-cache 256KiB, L2-cache 2 MiB, L3-cache 16MiB. We observe that AVX2 is not the latest ISA available on Intel Processor, but it is the one Quickscorer was implemented with; this choice was made for the sake of fair comparison. 

%  %To obtain a fair comparison in terms of scoring time, we tested both QuickScorer and NN models on a Intel i9-9900K CPU, with AVX2 instructions, always forcing single thread execution. In paricular, we write our own C++ inference model to avoid Pytorch overhead; this allows to speed up the execution time up to $10x$ for the small models. The large network is a 4 layer MLP with hidden sizes  $\{2000,500,5000,100\}$, while the small network is a 2 layer of shapes $\{500,100\}$. We adopt the same strategy for randomly generating training data. We use Adam as optimizer, with learning rate $0.001$ and no weight regularization; we multiply the learning rate by $\gamma = 0.1$ at epochs $\{50, 80 \}$ and use and early stopping criterion on the validation loss.  
% %TODO aggiungere tempo di esecuzione di pytorch sequenziale, pytorch mutlithread e gpu per confronto con articolo origianale
% %TODO aggiungere valori confrontabili

% \begin{table}

% \adjustbox{max width = \columnwidth}{

% 	\begin{tabular}{lrrrr}
% 		\toprule
% 		Model &     NDCG@10 &     MAP 0 & MAP 1& Scoring Time ($\mu S$)\\
% 		\midrule
	
% 		Regression Forest(64 leaves)&    0.5246& 0.6304  &0.6604  & 2.50	\\
% 		\midrule
% 		Large Network &   0.5198&0.6279   &0.6579  & 24.41\\
% 		Small Network & 0.5180& 0.6277 &0.6576 & 2.25\\
		  
% 		\bottomrule
% 	\end{tabular}
% 	}
% 	\caption{Comparison between a Regression Forest (878 trees, 64 leaves) and Neural Networks on \msn.  }
% 	\label{table:repr_comp}
% \end{table}
% In Table~\ref{table:repr_comp} we report our experimental results. As shown, with these settings Regression Forests \& Quisckscorer outperform Neural Networks both in effectiveness and efficiency. The gap in terms of NDCG@10 and Mean Average Precision (MAP) indicates the NNs are not capable to exactly approximate the $R(x)$ function mapped by the Regression Forest. 
% %todo dire qualcosa su Universal Approximation Function.
% For what concerns efficiency, the Large Network is $10x$ slower w.r.t. to Quickscorer, while the smaller network is slightly faster. Anyway, Table~\ref{table:repr_comp} shall not suggest that Regression Forest are more effective while NNs are more efficient. Since the gap in terms of NDCG@10 and MAP is consistent, we should compare the scoring time of the small network with the scoring time of a Regression Forest that has a comparable effectiveness. As shown in table~\ref{table:tree_overview}, we need less than 200 trees to reach the same NDCG@10 as the Small Network.  
% \begin{table}
% 	\begin{tabular}{rrrrr}
% 		\toprule
% 		Model &     NDCG@10 &Scoring Time ($\mu S$)\\
% 		\midrule
% 		100 trees & 0.5177 & 0.63\\
% 		150 trees & 0.5197 & 0.86\\
% 		200 trees & 0.5212 & 1.06\\
% 		300 trees & 0.5224 & 1.44\\
% 		400 trees & 0.5228 & 1.79\\
% 		500 trees & 0.5239 & 2.02\\ 

% 		\bottomrule
% 	\end{tabular}
% 	\caption{Overview of Regression Trees NDCG@10 and scoring time with QuickScorer }
% 	\label{table:tree_overview}
% \end{table}
% Under this settings, Regression trees with QuickScorer largely outscore NNs on the document scoring task. 

% \subsection{Improving the Regression Forest}
% An approch to bridge the gap - effectiveness wise - between NNs and Regression Trees is to exploit the Universal Approximation Theorem in a smarter way. Let $\Delta$ be the effectiveness gap between the Regression Forest and the Neural Networks, in terms of a chosen metric $M$ (NDCG@10, MAP). Let us also assume that is $\Delta$  is constant and does not depend on the Regression Forest function $R(x)$.  So
% $$ M_{N(x)} = M_{R(x) } - \Delta $$
% where $N(x)$ is the Neural Network Function, and $M_{F(x)}$ means the value of metrics $M$ with scoring function $F$.
% Instead of working on trying to reduce  $\Delta$, we can raise tha value of $M_{N(x)}$ increasing the effectiveness of the Regression Forest. We train a larger Regression Forest
% %todo chiedere a salvo i dettagli del trianing
% with 256 leaves for tree and 600 trees, which consistently outperforms the previous 64-leaves forest, as shown  (64 leaves) in Table~\ref{table:64vs256leaves}. 
% \begin{table}
% \begin{tabular}{rrrr}
% 		\toprule
% 		Model &     NDCG@10 &     MAP 0 & MAP 1 \\
% 		\midrule
	
% 		878 trees, 64 leaves &    0.5246& 0.6304  &0.6604 	\\
		
% 		600 trees, 256 leaves &   0.5291&0.6321   &0.6621  \\
		
		  
% 		\bottomrule
% 	\end{tabular}
% 	\caption{Comparison between Regression Forests with different number of trees and leaves.   }
% 	\label{table:64vs256leaves}
% \end{table}
% We now train the MLP models to mimic the scores obtained with the new, more effective model. %todo loosely inspired by knowledge distillation
% First, we discard the Large Model provided by~\cite{cohen2018universal}, due to its unbearable execution time. We provide a new Medium Model with shapes $\{1000, 500, 500, 100 \}$, that halves the execution time w.r.t. to the Large one. Under this new training configuration, this model can reach the same effectiveness of the Regression Forest with 878 trees and 64 leaves. 
% \begin{table}
% \adjustbox{max width = \columnwidth}{
% 	\begin{tabular}{rrrrr}
% 		\toprule
% 		 Model &     NDCG@10 &     MAP 0 & MAP 1& Scoring Time ($\mu S$)\\
		
% 		\midrule
	
% 		Regression Forest &    0.5246& 0.6304  &0.6604  & 2.50	\\
% 		\midrule
% 		Medium Network &   0.5243&0.6597   &0.6297  & 14.54\\
% 		Small Network & 0.5190& 0.6278 &0.6578 & 2.25\\
		  
% 		\bottomrule
% 	\end{tabular}
% 	}
% 	\caption{Comparison between a Regression Forest (878 trees, 64 leaves) and Neural Networks on \msn.  }
% 	\label{table:repr_comp}
	
% \end{table}
% Our assumption that $\Delta$ is constant results to be quiet accurate, as shown in Table~\ref{table:repr_comp}. Using a better model to produce the score on which we perform the regression is profitable since we obtain models which are more effective without affecting the scoring time. Using a four layer MLP we match the same accuracy as the Regression Forest, while we are still $7x$ times slower than QuickScorer in scoring the documents. The effectiveness gap for the 2 layers MLP still persists; the Small Network slightly benefits from the score generator improvement. This suggests us that the small MLP is not expressive enough to perform regression on the scores. This could be caused by the reduced number of weights:
% the Large Network has 936K parameters, while the Small Network has 118K. Anyway, Table~\ref{table:small_vs_large_network} shows that deeper networks generally work better.
% %TODO dire qualcosa sulla tabella
%  Our explanation is that adding more layers allows to combine and produce higher level features. Inspired by these considerations, we focus on 4 layers neural networks. 

% %TODO AGGIUNGERE ESPERIMENTI CON RETI A 5 LIVELLI
% %TODO cercare/esperiment su reti cilindriche
% %todo abbiamo anche reti leggermente migliori della prima, con lo stesso budget di parametri. 
% \begin{table}
% \adjustbox{max width = \columnwidth}{
% 	\begin{tabular}{rrrr}
% 		\toprule
% 		Model & Parameters &NDCG@10 & Scoring Time ($\mu S$)\\
% 		\midrule
% 		500x100 & 118K & 0.5196 & 2.25 \\
% 		1000x200 &336K& 0.5155 & \\
% 		2000x400 & 1M & 0.5158& \\
% 		\midrule
% 		200x100x100x50 & 63K & 0.5187& 1.36 \\
% 		300x150x150x30 & 112K & 0.5207 & 2.29 \\
% 		400x200x200x100 & 194K & 0.5220 & 3.85 \\
		  
% 		\bottomrule
% 	\end{tabular}
% 	}
% 	\caption{Comparison between 2 and 4 layers architectures on \msn.  }
% 	\label{table:small_vs_large_network}
	
% \end{table}







% \subsection{Model Compression to further reduce execution time}

% Model Compression comprises several techniques to reduce the size and speed-up the forward time for a neural network. Among them, we consider pruning techniques, that allows to remove (\textit{i.e.,} set to $0$) a portion of the weights from the network. They are dived in:
% \begin{itemize}
% 	\item Element-wise pruning techinques: set to $0$ individual weights, generating sparse weight tensors

% 	\item Structured pruning techniques: prune entire groups of weights, \textit{i.e.,} columns, filters, layers. Resulting network's weights still belongs to the dense domain. 
% \end{itemize}
% Structured pruning for MLP is applied as column-wise pruning; given a criterion to estimate each column importance, \textit{e.g.,} $L_1$-norm, we remove a percentage of less important columns and then re-train the network. We experimentally verified that, established  a network architecture $\{l_1^p, l_2^p, l_3^p, l_4^p\}$, it makes no difference whether is trained from scratch or obtained from a pruning-finetuning procedure on a pre-trained model.   Then, we focused on element-wise pruning.  
% We applied two different kind of element-wise pruning:
% \begin{itemize}
% 	\item Level pruning: let $p_i$ the percentage of weights to remove from layer $i$, we save the highest magnitude $p_i$ weights and set to Execution time survived to the pruning phase are re-trained to recover the accuracy loss. 

% 	\item Han Pruning: insipired by the original method by Han, implemented in the version of the Distiller Framework. For each layer, we compute the standard deviation $\sigma_i$ and set a sensitivity parameter $s_i$. For each weight $w_{l_i}$ we use $\lambda_i = s_i * \sigma_i$ as theshold and set to zero all those weights whose absoulute values is below the threshold. The surviving weights and the re-trained. The procedure can be iteratively repetead to gradually increase the sparsity of the weights. In the original versino by Han, the value of $s_i$ was increased at each iteration, while the Distiller version keep it fixed, relying on the fact the as the tensor is pruned, more elements are pulled towards the center of the distribution and then pruned. 
% 	%TODO riscrivere, copiato da distiller. 
% \end{itemize}
% Weight Pruning has shown to be an effective compression technique, capable to reduce the number of values of a Neural Network of an order of magnitude without affecting its performance. %TODO Citazioni
% In Figure~\ref{fig:sparse_ndcg} we report the results of appling Level Pruning on the Medium Network ($\{1000,500,500,100\}$). When sparsity is about or below 90\%, the sparse models perform as or even outperforms the dense model, showing that pruning can be used as a regularization method. Until 96\% of sparsity, the NDCG@10 of the sparse model is still comparable with the one of the dense model. For higher levels of sparsity, we observe a noticeable degradation. 
% In Figure~\ref{fig:sparse_exec_time}, we show how to execution time decreases as the sparsity raise. Sparse multiplication is implemented with MKL. Compared with the dense model, sparse forward is very fast, but still cannot reach the performance of QuickScorer. 
% Table~\ref{table:sparse_vs_dense_network} provides an overview on the tradeoff between speed an accuracy for both dense and sparse models. With the same parameter budget, sparse models outscore dense ones in terms of effectiveness (reported as NDCG@10). Actually, with a sparse model of 37K parameters we sill suprass  the accuracy of a 4 times bigger dense models.%TODO come detto in agp pruning. 
% On the other hand, dense forward time is always faster than forward on sparse models. Even with a 10x smaller sparse model, when can barely match the dense model performances.  


% \begin{figure}
% 	\Description[]{}
% 	\centering
% 	\includegraphics[width=\columnwidth]{imgs/1000x500x500x100_sparse_ndcg.png}
% 	\caption{Performance in terms of NDCG@10 of a 1000x500x500x100 MLP at various level of sparsity }
% 	\label{fig:sparse_ndcg}
% \end{figure}


% \begin{figure}
% 	\Description[]{}
% 	\centering
% 	\includegraphics[width=\columnwidth]{imgs/1000x500x500x100_sparse_exec.png}
% 	\caption{Execution time of a 1000x500x500x100 MLP at various level of sparsity }
% 	\label{fig:sparse_exec_time}
% \end{figure}

% \begin{table}
% \adjustbox{max width = \columnwidth}{
% 	\begin{tabular}{rrrrr}
% 		\toprule
% 		Model & Sparsity & Parameters &NDCG@10 & Scoring Time ($\mu S$)\\
		
% 		\midrule
% 		200x100x100x50& 0.0 & 63K & 0.5187& 1.36 \\
% 		300x150x150x30 & 0.0& 112K & 0.5207 & 2.29 \\
% 		400x200x200x100 & 0.0& 194K & 0.5220 & 3.85 \\
% 		 \midrule
% 		 1000x500x500x100 & 90.0 & 93K & 0.5253 & 9.44\\
% 		 1000x500x500x100 & 96.0 & 37K & 0.5230 & 4.98\\
% 		 1000x500x500x100 & 98.0 & 18K & 0.5158 & 3.57\\
% 		\bottomrule
% 	\end{tabular}
% 	}
% 	\caption{Comparison between dense and sparse architectures on \msn.  }
% 	\label{table:sparse_vs_dense_network}
	
% \end{table}

% We than perform Han version of weight level pruning. Using $ \sigma=1$ for each layer, we obtain a network with NDCG@10: 0.5230 and sparsity 94.7\%. Besides the model itself, that results coherent with the overview provided level pruning reported in Figure~\ref{fig:sparse_ndcg}, it is intresting to analyze how the sparsity is distributed among the layer. Since we fix a threshold, and not a sparsity level as in the previous method, the percentage of zero weights per layer is usually variable. Even if we set the $\sigma $ for each layer, we obeserve that layers reaches different levels of sparsity. Especially we note that the first layer is the more \textit{pruning prone}, namely the one that reaches the higher sparsity level in this dynamic context. The explanation resides in the distribution of this layer weights. In fact, these layer show a quantity of quit large absolute values weights, phenomenon which is not witnessed in any other layers. These parameters assume that large absoulute values to dial with large absoulte value features which there are in \msn. This suggest that the layers have different sensibility to pruning. 

% %TODO va pensato a come dimostrare questa cosa
% %TODO va detto qualcosa in più sulle features
% \begin{table}
% \adjustbox{max width = \columnwidth}{
% 	\begin{tabular}{rrrrr}
% 		\toprule
% 		Layer & Total Parameters &Non zero parameters & Sparsity (\%)\\
% 		\midrule
% 		fc1 & 136000 & 2346 & 98.3 \\
% 		fc2 & 500000 & 28924 & 94.2 \\
% 		fc3 & 250000 & 13949 &94.4\\
% 		fc4 & 50000 & 4323 & 91.3\\
% 		fc5 & 100 & 100 & 0.0\\
% 		\bottomrule
% 	\end{tabular}
% 	}
% 	\caption{Comparison between dense and sparse architectures on \msn.  }
% 	\label{table:sparse_vs_dense_network}
	
% \end{table}


% \subsection{Combining Smaller Architecture with pruning}

% As shown in Table~\ref{table:sparse_vs_dense_network}, small dense networks and large dense ones stand on different sides of the trade-off ideal line. Dense networks afford fast inference, sparse network high effectiveness. Dense networks suffer of accuracy loss, sparse network of slow forward time. It seems natural to try to combine them together. 
% %The first apporach was to prune a network with a consistent effectiveness, such as the 400x200x200x100 network reported in Table~\ref{table:sparse_vs_dense_network}. 
% %TODO qui mettere i risultati del pruning sulla rete 400
% In Figure~\ref{fig:2115_tradeoff} we report the tradeoff between NDCG@10 and sparsity when pruning a 200x100x100x50 MLP and for Regression Trees with comparable effectiveness and execution time. The graph can be read along vertical or horizontal lines. Drawing a vertical line is equivalent to fix the NDCG@10 value. We then look for lower dots, that represent the methods that allow for faster inference. For example, a Regression Forest with 150 trees can be considered as performing as a 80\% sparse 200x100x100x50 model. Being the orange dot (Regression Forest) below the blue one (NN), it means that the scoring time through QuickScorer and the Random Forest is faster. To read the graph horizontally, we look for rightmost dots on horizontal line, that represent top performing models with a given budget in terms of scoring time. 



% \begin{figure}
% 	\centering
% 	\includegraphics[width=\columnwidth]{imgs/200x100x100x50_sparse_ndcg_and_exec.png}
% 	\caption{Tradeoff between NDCG@10 and sparisty for a 200x100x100x50 MLP and Regression Trees }
% 	\label{fig:2115_tradeoff}
% \end{figure}





% \subsection{Efficency-Oriented Pruning}

%TODO togliere totale dalla fig e scrivere percentuale sopra barre




% !TEX root = paper.tex
% !TeX spellcheck = en_US

\section{Conclusions and Future Work}
\label{sec:conclusions}
In this paper, we presented an effective and efficient methodology to design neural networks for document scoring in a modern information retrieval system. The neural models we take into account are trained to approximate the scores of an ensemble of regression trees. By leveraging a combination of high-performance dense-dense, sparse-dense matrix multiplication, and element-wise pruning, the neural models can compete with the original models.
Thus, our methodology is \textit{effective}. By developing time predictors based on an accurate study of how these operations are implemented on modern processors, we are capable to precisely estimate the execution time of a given architecture by knowing the shape and the sparsity level of each layer. This allows to train only a limited number of models, the ones matching the time requirements given by the specific context. Our methodology is thus \textit{efficient}. Besides presenting our method, throughout the paper emerges a comparison between ensembles of regression trees and NNs on the document scoring task, tested on the \msn and \istella datasets. In our experiments,  neural networks are not capable of reaching the accuracy of their \textit{teacher}, hence tree-based methods are superior in top-quality retrieval scenarios. At any other level of the efficiency-effectiveness trade-off, neural models designed and trained with our approach can always outscore or at least compete with ensembles of regression trees.

As future work, we intend to apply different compression methods such as quantization or early exiting to further improve the efficiency of our neural models. Moreover, we plan to extend our comparison between neural networks and ensemble of regression trees to other computational engines, such as General-Purpose Graphic Processing Unit (GPU) or Field Programmable Gate Array (FPGA). We also aim at improving the training by distillation procedure of neural models, in order to bridge the effectiveness gap with ensembles of regression trees. 

\smallskip
\noindent \textbf{Acknowledgements}.
This paper is partially supported by the ``Algorithms, Data Structures and Combinatorics for Machine Learning'' (MIUR-PRIN 2017) and the OK-INSAID (MIUR-PON 2018, grant agreement ARS01\_00917) projects.

 %we plan to extend our comparison between neural networks and ensemble of regression trees to other computational engines, such as General-Purpose Graphic Processing Unit (GPU) or Field Programmable Gate Array (FPGA). We also intend to apply different compression methods such as quantization or early exiting to Furthermore, we intend to apply our approach to other domains in which neural models have proven to be successful, such as Computer Vision (CV) or Natural Language Processing.

%\fnote{direi troppo generici i future work. starei più sul dettagliato. FM}

%\fnote{io il pezzo sotto non lo metterei. non e' il main result dell'articolo e non e' una motivazione concreta per usare l'uno o l'altro. se sei google, che ci vuole a implementare uno scoring efficiente di alberi con AVX512?}

%However, a neural network-based scoring systems presents a technological advantage: the efficiency of the forward-pass of a neural network strictly depends on matrix multiplication. As mentioned earlier in this paper, several high performance libraries~\cite{van2015blis,xianyi2012openblas} furnish highly optimized versions of this routine, on which we can rely when developing our scoring system. On the other hand, at the moment do not exist libraries for fast traversal of ensemble of trees. %and the QuickScorer code is not publicy available. 
%Developing a tree-based scoring system entails to re-implement an efficient scoring algorithm for regression forests, which should adapt to different architectures and should be maintained and updated to deal with the continuous improvements in the industry of CPUs.   
%For example, QuickScorer does not employ AVX512 vectorized instructions. Updating and maintaining it fully harvest the feature of  modern CPUs is a really challenging task, which requires a considerable effort in terms of development and testing time. 







\bibliographystyle{IEEEtranS}
\bibliography{IEEEabrv,biblio}

\begin{IEEEbiography}[{\includegraphics[width=1in,height=1.25in,clip,keepaspectratio]{photos/Franco.png}}]{Franco Maria Nardini}
is a senior researcher with the National Research Council of Italy. His research interests focus on web information retrieval and machine/deep learning.
He authored more than 70 papers in peer-reviewed international journal and conferences.
%He received the Best Paper Award at ACM SIGIR 2015.
He received the Best Paper Award at ACM SIGIR 2015 and the Best Demo Paper Award at ECIR 2014.
For more information: \url{http://hpc.isti.cnr.it/\~nardini}.
%\href{http://hpc.isti.cnr.it/~nardini}{http://hpc.isti.cnr.it/\~nardini}.
\end{IEEEbiography}

\begin{IEEEbiography}[{\includegraphics[width=1in,height=1.25in,clip,keepaspectratio]{photos/rulli.jpg}}]{Cosimo Rulli}
is a Ph.D student at the University of Pisa and a researcher with the National Research Council of Italy. He received the Master Degree from the University of Florence in 2019, with a thesis on deep neural network compression with knowledge distillation and pruning. His research interests focus on deep learning, model compression, and information retrieval. 
\end{IEEEbiography}

\begin{IEEEbiography}[{\includegraphics[width=1in,height=1.25in,clip,keepaspectratio]{photos/trani.jpg}}]{Salvatore Trani}
is a researcher with the National Research Council of Italy. He received his PhD in Computer Science from the University of Pisa in 2017. His main research interests ranges from Information Retrieval to Web Mining and Machine Learning. He authored more than 15 papers on these topics, published in peer reviewed international journals and  conferences.
\end{IEEEbiography}

\begin{IEEEbiography}[{\includegraphics[width=1in,height=1.25in,clip,keepaspectratio]{photos/Rossano.png}}]{Rossano Venturini}
is an associate professor at the Computer Science Department, University of Pisa. His research interests include the design and the analysis of algorithms and data structures with focus on indexing and searching large textual collections. He received two Best Paper Awards at ACM SIGIR in 2014 and 2015. For more information: \href{http://pages.di.unipi.it/rossano}{http://pages.di.unipi.it/rossano}.
\end{IEEEbiography}
\end{document}

% !TEX root = paper.tex
% !TeX spellcheck = en_US

\section{Related Work}
\label{sec:related}

In this section, we introduce Learning to Rank (LtR) and its use in Information Retrieval (IR). Then, we describe QuickScorer~\cite{lucchese2015quickscorer,dato2016fast,8035185} an efficient algorithm for scoring ensemble of regression trees. Finally, we discuss the field of model compression, a branch of machine learning that aims to compress Deep Neural Networks without affecting their accuracy. Here, we focus our attention in particular on pruning techniques.

% subsection subsection_name (end)}
\subsection{Learning to Rank}
\label{subsec:ltr}
Learning to Rank (LtR) consists in applying machine learning techniques to the problem of ranking documents with respect to a query.
%such as BM25~\cite{robertson2009probabilistic} and Query Likelihood. 
RankNet~\cite{burges2005learning} leverages a  probabilistic ranking framework based on a pairwise approach to train a neural network. The difference between the predicted scores of two different documents is mapped to a probability by means of the sigmoid function. Hence, using the cross-entropy loss this probability is compared with the ground truth labels, and Stochastic Gradient Descent (SGD) is used to minimize this loss. FRank~\cite{tsai2007frank} exploits a generative additive model and substitutes the cross-entropy loss with the fidelity loss, a distance metric adopted in physics, superior to cross-entropy when applied on top of the aforementioned probabilistic framework since 1) has minimum in zero, 2) is bounded in $[0,1]$. 
%proposes a probabilistic cost function framing the ranking problem into a \emph{pairwise} approach. 
Neither RankNet nor FRank directly optimize a ranking metric (\emph{e.g.}, NDCG), and this discrepancy weakens the power of the model. Since ranking metrics are flat and discontinuous, coding them into the loss function is troublesome.
To overcome this issue, LambdaRank~\cite{burges2007learning} heuristically corrects the RankNet gradients, exploiting the rank position of the document in the overall sorting: it multiplies the RankNet gradient with a term that measures the increase in terms of NDCG when switching the terms, generating the so-called $\lambda$-gradients.
%	Multiple Additive Regression Trees (MART) have shown remarkable results for the ranking problem. 
McRank~\cite{li2008mcrank} casts the problem of ranking as MultiClass classification task, using a boosting tree algorithm to learn the class probabilities and then converting them into relevances with the expected relevance, outperforming LambdaRank.  This work also highlights that modeling the ranking problem as a classification task works better than modeling it as a regression one.  
LamdaMART~\cite{burges2010ranknet} combines the successful training methodology provided by $\lambda$-gradients with Multiple Additive Regression Trees (MART) - as McRank~\cite{li2008mcrank}, and it has been establishing as the state-of-the-art in LtR. Currently, ensembles of regression trees are the most effective solution among LtR techniques when dealing with handcrafted features. In the next section, we describe state-of-the-art approaches for efficient traversal of these trees, in order to employ them in latency-bound scenarios.


\subsection{Efficient Traversal of Tree-based Models}
\label{subsec:quickscorer}
QuicksScorer~\cite{lucchese2015quickscorer} is a state-of-the-art algorithm that allows to speedup the traversal of an ensemble of regression trees. As detailed in the previous section, ensemble of regression trees is the model exploited by several state-of-the-art learning-to-rank solutions, e.g., LambdaMART~\cite{burges2010ranknet}.
QuickScorer codes each tree of the ensemble as a bitvector of length $n$, where $n$ is the number of leaves, which is used to select the \textit{exit leaf} in the tree. Furthermore, each decision node in each tree is associated with a bitvector of the same length called \textit{mask}. If the corresponding test is evaluated to false, the bits corresponding to the unreachable leaves are set to zero. By performing the logical \texttt{AND} among all the masks, we obtain another bitvector, named \emph{leafidx}, in which the first one entry corresponds to the exit leaf. To efficiently compute the exit leaf, QuickScorer process all the nodes in a \textit{feature by feature} fashion. For each feature $f$, the associated thresholds among all the nodes in the forest are sorted in ascending order. Let us a consider a threshold $\gamma$ associated with a node $g$: when $x_f > \gamma$, 
the corresponding \emph{leafidx} is updated performing the \texttt{AND} operation with the \emph{mask} relative to $g$. Since the thresholds are sorted, as soon as $x_f \leq \gamma$, the evaluation of the current feature is interrupted, since the following instances will evaluate true as well. To further improve the efficiency of the algorithm, two variations of the original algorithm are introduced:  1) Block-Wise QuickScorer (BWQS), in which the forest is partitioned into blocks of trees fitting the L3 cache, reducing the cache-miss ration and 2) Vectorized QuickScorer (vQS)~\cite{lucchese2016exploiting}, in which scoring is vectorized using AVX2 instructions and 256-bit registers, allowing to process up to $8$ document at time. 
Lettich \emph{et al.}~\cite{8035185} propose a GPU version of QuickScorer, to exploit the massive parallelism of this computational engine. By properly managing the GPU memory hierarchy and furnishing an adequate degree of parallelism in the document scoring process, this version results up to 100x faster than the corresponding CPU version, when dealing with very large forests ($20$,$000$ trees).
 
The cost of traversing an ensemble of regression trees with QuickScorer depends on the number of false nodes, rather than on the length of the root-to-leaf paths. Since machine-learnt trees are imbalanced, the authors experimentally show that this reduces the percentage of nodes to evaluate from 80\% of classical traversal to the 30\% of QuickScorer~\cite{lucchese2015quickscorer}. Moreover, QuickScorer is implemented carefully taking into account cache and CPU issues. For example, QuickScorer structures are accessed sequentially thus favoring pre-fetching and avoiding branch mispredictions. However, when the number of leaves is larger than $64$, scoring a model with QuickScorer can be inefficient. Recently, RapidScorer tackles the problem of forest with a larger number of leaves~\cite{ye2018rapidscorer}. In fact, when $|\text{leaves}| > 64$, the logical \texttt{AND} between the bitvectors cannot be carried out in just one CPU instruction, hampering efficiency. For this reason, RapidScorer introduces a tree-size insensitive encoding, named \emph{epitome}. Moreover, it leverages a node merging strategy that evaluates just once nodes sharing the same threshold on the same feature. By doing so, RapidScorer outperforms  QuickScorer when dealing with a large number of leaves.

\subsection{Model Compression}
\label{subsec:modelcompr}
The effectiveness of Deep Neural Networks (DNNs) comes at the cost of a high computational complexity~\cite{nnstats}, hindering the deployment and the usage of DNNs, especially for resource-constrained devices. An inherent feature of DNNs is \textit{over-parameterization}, \textit{i.e.,} the redundancy of networks parameters: it has been proven that the same performance can be obtained with just a portion of the original parameters~\cite{denil2013predicting}. Model Compression (MC) is a recent research field investigating effective techniques for reducing the memory impact of DNNs, their inference time, and energy consumption without affecting their accuracy, exploiting over-parameterization. 
%TODO mancherebbe low rank decomposition
In MC techniques, we observe the presence of several lines of research: pruning~\cite{DBLP:journals/corr/HanPTD15,DBLP:journals/corr/LiKDSG16, molchanov2019pruning,DBLP:journals/corr/HanMD15,DBLP:journals/corr/GuoYC16,yu2017scalpel,vieira2017learning,he2017channel,he2018amc}, quantization~\cite{DBLP:journals/corr/LiL16,DBLP:journals/corr/ZhuHMD16, DBLP:journals/corr/RastegariORF16,hubara2017quantized,Cai_2017_CVPR,DBLP:journals/corr/ZhouYGXC17} design of efficient architectures~\cite{DBLP:journals/corr/IandolaMAHDK16,zhang2018shufflenet,howard2017mobilenets,sandler2018mobilenetv2}, knowledge distillation~\cite{bucilua2006model,ba2014deep,DBLP:journals/corr/HintonVD15}. 

%($17.7x$, AlexNet on ImageNet, ).
%Network quantization techniques reduce the number of bits necessary to represent each weight thus generating lighter models providing faster inference and lower energy consumption~\cite{sze2017efficient}. Here, data-driven methods introduce a limited loss of accuracy even when using $1$ or $2$ bits~\cite{DBLP:journals/corr/LiL16, DBLP:journals/corr/ZhuHMD16,DBLP:journals/corr/RastegariORF16,hubara2017quantized,Cai_2017_CVPR}. Moreover, the incremental learning of the optimal quantization outperforms the original model performance~\cite{DBLP:journals/corr/ZhouYGXC17}.
%An orthogonal approach is the design of new and efficient architectures by modifying existing layers~\cite{DBLP:journals/corr/IandolaMAHDK16} or creating new ones~\cite{DBLP:journals/corr/ZhangZLS17, howard2017mobilenets,sandler2018mobilenetv2,DBLP:journals/corr/abs-1807-11164}. Automatizing the design and compression of neural architectures by mean of reinforcement learning and evolution techniques have proven to be effective~\cite{tan2019efficientnet,cai2018proxylessnas,tan2019mnasnet}
%Knowledge distillation techniques~\cite{DBLP:journals/corr/HintonVD15} exploit a \textit{teacher} network at training time to improve the learning process of a shallower and faster \textit{student} network. Finally, the combination of different compression techniques has proven to be an effective solution~\cite{DBLP:journals/corr/HanMD15,polino2018model}
%Pruning techniques belong to the family of Model Compression (MC) methods. MC is a recent research field investigating effective techniques for reducing the memory impact of DNNs, their inference time, and energy consumption without affecting their accuracy. Lossless compression is allowed by the significant redundancy of parameters, i.e., over-parametrization, that has been discovered in neural networks~\cite{DBLP:journals/corr/DenilSDRF13}. 
Recently, pruning has shown to be extremely effective~\cite{DBLP:journals/corr/HanPTD15,DBLP:journals/corr/LiKDSG16,DBLP:journals/corr/HanMD15,DBLP:journals/corr/GuoYC16,luo2017thinet,huang2018data,yu2017scalpel,vieira2017learning,he2017channel,he2018amc}. Pruning techniques delete useless connections in a pre-trained model, producing sparse weight tensors that are lighter to store and allow for faster inference time. Performing a retraining after pruning avoids accuracy loss, even in the case of high compression factors~\cite{DBLP:journals/corr/GuoYC16}. 
The canonical classification of pruning techniques divide them into two families: 1) element-wise pruning, which sets to zero individual weights, generating sparse weight tensors and 2) structured pruning, which prunes entire groups of weights, \textit{i.e.,} columns, filters, or even entire layers. In the latter case, the resulting network's weights still belong to the dense domain. 
%Structured pruning for feed-forward network is applied columns-wise. Once estimated the importance of each column by mean of an heuristic (\textit{e.g.,} $L_1$-norm),  a percentage of less important columns is removed from the model. Given an original model architecture with $l$ layers $A_l = \{l_1, \dots , l_{l}\}$, we obtain a reshaped dense model $A_l^p =  \{l_1^p, \dots, l_l^p \}$. 
%We experimentally verified that it makes no difference whether the $A_l^p$ model is trained from scratch or obtained from a pruning-finetuning procedure on a pre-trained model, as happens in models for Image Classification tasks~\cite{liu2018rethinking}.   
%MC techniques can be grouped in four main lines of research: \textit{pruning}~\cite{DBLP:journals/corr/HanPTD15,DBLP:journals/corr/LiKDSG16, DBLP:journals/corr/MolchanovTKAK16, ullrich2017soft,DBLP:journals/corr/HanMD15,DBLP:journals/corr/GuoYC16,DBLP:journals/corr/YangCS16a,DBLP:journals/corr/LuoWL17,huang2018data,wen2016learning,yu2017scalpel,vieira2017learning,he2017channel,he2018amc,prakash2018repr}, \textit{quantization}~\cite{DBLP:journals/corr/LiL16,DBLP:journals/corr/ZhuHMD16, DBLP:journals/corr/RastegariORF16,hubara2017quantized,DBLP:journals/corr/ZhouYGXC17} \textit{design of efficient architectures}~\cite{DBLP:journals/corr/IandolaMAHDK16,DBLP:journals/corr/ZhangZLS17,howard2017mobilenets,sandler2018mobilenetv2}, \textit{knowledge distillation}~\cite{DBLP:journals/corr/HintonVD15,romero2014fitnets,chen2015net2net}. 
In this paper, we focus on element wise-pruning techniques. These methods employ heuristics to determine what are the relevant weights of the network. In particular, \textit{magnitude-based} heuristics work by removing low absolute-value weights and are proved to be effective~\cite{DBLP:journals/corr/HanPTD15,DBLP:journals/corr/GuoYC16}. In their na\"ive version, magnitude based approaches remove a fixed percentage of weights from the original model (\emph{level pruning}). Han \emph{et al.} show that the gradual increase of the target sparsity, interleaved with a number of steps of re-training, can improve the accuracy of the final model~\cite{DBLP:journals/corr/HanPTD15}. Furthermore, they propose a layer-wise threshold-based method to determine whether a parameter shall be kept or not. For each layer, its threshold $t_i$ is computed as as $t_i = \sigma_i * s_i$, with $\sigma_i$ the standard deviation of weights distribution and $s_i$ a sensitivity parameter to be chosen. By assuming that parameters follow a Normal distribution $ \mathcal{N}(0, \sigma^2) $, setting $s_i = 1$ would approximately prune away about the 68\% of the weights. The pruning step is followed by a number of re-training epochs on the surviving weights. The procedure can be then iterated by gradually increasing $s_i$ thus inducing higher sparsity. The Distiller Framework~\cite{nzmora2019distiller} version that we adopt, keeps this threshold fixed, relying on the fact that as the tensor is pruned, more elements are pulled towards the center of the distribution and then pruned. Pruning techniques have shown to be able to sparsify state-of-the-art neural architectures up to 90\%, thus strongly reducing their memory burden and easing the transmission and deployment on resource-constrained devices.



\section{Training by Scores Approximation}
\label{sec:cohen}
In this section, we detail the methodology proposed by Cohen \textit{et al.}~\cite{cohen2018universal} to train neural models approximating ensembles of regression trees.
Their technique can be considered as a special case of Knowledge Distillation~\cite{ba2014deep,DBLP:journals/corr/HintonVD15}.
 %Observe that learning the outputs of another model instead that directly the ground truth is locate their work close in the knowledge distillation area~\cite{ba2014deep,DBLP:journals/corr/HintonVD15}.
Knowledge distillation is a training technique in which a small \emph{student} model is trained to mimic the outputs of a large and expressive \emph{teacher} model.
In the case of Cohen \emph{et al.}, the ensemble of regression trees plays the role of the teacher, while the neural network is the student model.
%We now describe the approach by Cohen \textit{et al.}~\cite{cohen2018universal} in detail.
The core idea of their approach is to treat the tree-based model as a black box producing accurate scores. Formally, let us consider a Learning to Rank dataset $D = (X, Y)$,  $X \in \mathbb{R}^{f \times |D|}$, where $f$ is the number of extracted features per document, $|D|$ is the cardinality of the dataset, and $Y \in \mathbb{N}^{|D|}$ is the set of ground-truth relevances of a document w.r.t. a query.
Let $F: \mathbb{R}^{f} \rightarrow \mathbb{R}$ be the underlying function learned by an ensemble of regression trees during the training that maps a single document $x \in X$ into a relevance score.
If the neural model can reproduce the function $F$, it achieves the same ranking quality as the original model. The effectiveness of this approach relies on theoretical results showing that NNs can approximate continuous~\cite{hornik1991approximation} and piecewise continuous functions~\cite{llanas2008constructive}. In practice, the approximation is implemented by using the \emph{Mean Squared Error} as loss function computed between the network prediction and the ensemble prediction. Furthermore, the training procedure is enriched with a data augmentation step which enforces the approximation capabilities of the neural network. Consider the set of $f$ features in the dataset. For each feature, Cohen \emph{et al.}~\cite{cohen2018universal} build a list composed of 
the split points corresponding to that feature in the ensemble of regression trees, and, in the same list, they also put the maximum and the minimum for that feature in the training set.
This way they obtain a set of $f$ lists, where $f$ is the number of the features in the dataset. Each of these lists is then sorted, and replaced with its ordered midpoints, \emph{e.g.}, each adjacent pair $\{x_i, x_{i+1}\}$ is replaced with its midpoint, $\frac{x_i + x_{i+1}}{2}$.  At each training step, half of the training data is built by randomly sampling from this feature-wise set of lists to have a better coverage of the whole feature space. Before feeding them to the network, all the training data are normalized by subtracting the mean and by dividing by the variance ($Z$-normalization).
This approach is more proficient than directly learning the ground-truth relevance~\cite{cohen2018universal}. As detailed in Section \ref{sec:introduction}, the approximation error introduced is small but statistically significant in terms of ranking quality. In Section~\ref{sec:neuraleng}, we show how to mitigate this effect.

% !TEX root = paper.tex
% !TeX spellcheck = en_US

\section{Experiments}
\label{sec:experiments}
In this section, we provide an extensive evaluation of our methodology to design, train and sparsify neural models for the document scoring task. In particular, we compare them against tree-based models at different points of the efficiency-effectiveness trade-off. Throughout this article, we have used the \msn dataset as use case. We now complement our evaluation with the \istella dataset~\cite{dato2016fast}. 
First, we present the experimental setup. Then, we report our experimental results and we show that neural models obtained with our technique can outperform ensembles of trees. To ease the reproducibility of the results presented in this article, code and trained models have been made publicly available\footnote{\url{https://github.com/hpclab/efficient_nn_for_ltr}}.


\vspace{-.3cm}
\subsection{Experimental Setup}
\label{subsec:expsetup}
We perform our experiments on two datasets: \istella and \msn. The \istella dataset~\cite{dato2016fast} consists of a collection of $33$,$018$ queries with an average of $103$ documents per query. Each document-query pair is represented by $220$ features. The \msn (Fold 1) dataset, which we already introduced, is composed by more than $30$,$000$ queries, with about $120$ documents per query and 136 features per document-query pair. In both the dataset, document-query pairs are labeled with $5$-graded relevance judgments ranging from 0 (irrelevant) to 4 (perfectly relevant).
Both datasets are split in train-validation-test according to a 60\%-20\%-20\% criterion. 

\smallskip
\noindent \textbf{LambdaMART models}. We employ the LightGBM framework~\cite{NIPS2017_6907} to train ensembles of regression trees using the LambdaMART algorithm. For each training, we perform hyper-parameter tuning using the HyperOpt library \cite{bergstra2013making}.
In particular, we determine the optimal combination of the following set of hyper-parameters: \texttt{learning rate, max depth, min\_sum\_hessian\_in\_leaf, min\_data\_in\_leaf}. 
%\fnote{sopra: non mi piacciono scritti così. se sono parametri di un algo, si mettono courier. altrimenti si riportano discorsivi con font normali ma spiegando che sono. ti torna?}
To avoid overfitting, we apply an early stopping criterion on the validation loss every 100 trees. We train 64-leaves model as target model to compare against neural networks and 256-leaves models to use as teachers. The latter models offer higher retrieval performance while being $4$x slower, which is not suitable for the use in latency-bounded applications. We score the LambdaMART models using a C++ implementation of QuickScorer that exploits instruction-level parallelism by using AVX2 instructions~\cite{8035185}.

\smallskip
\noindent \textbf{Neural Networks}. We train neural models (\textit{students}) to approximate the scores of top-performing regression forest (\textit{teacher}), accordingly to the knowledge distillation \cite{ba2014deep} paradigm, detailed in Section~\ref{subsec:approxbetter}. Models are trained using Pytorch~\cite{NEURIPS2019_9015}, adopting the same strategy for randomly generating training data of Cohen \textit{et al}.~\cite{cohen2018universal}. We employ RELU6 as activation function after every linear layer, except for the last one, where $\text{RELU6}(x) = min(max(x,0), 6)$. 
%We use the Distiller~\cite{nzmora2019distiller} framework to prune the models. Both in training and pruning, we employ Adam~\cite{kingma2014adam} as optimizer, with learning rate $0.001$ and no weight decay. 
%Table~\ref{table:neuraltrainparams} summarizes the other training and pruning hyper-parameters, which are dataset-dependent. $E_t$ represents the number of training epochs. The pruning phase is composed of $E_p$ epochs of pruning/fine-tuning and of $E_{ft}$ epochs of solely fine-tuning. Both in training and pruning, we scale the learning rate by multiplying it by $\gamma$ at the epochs specified by $\gamma_{step}$. Dropout, if present, is applied only after the first layer. The neural forward pass  is implemented in C++, using the \textit{dnnl\_sgemm} routine from the OneDNN\footnote{\url{https://github.com/oneapi-src/oneDNN}} framework for  dense matrix multiplication and the LIBXSMM~\cite{heinecke2016libxsmm} C++ library for sparse-dense matrix multiplication (after pruning).
%We train for $E_t$ epochs in standard training; the pruning phase, instead, is composed of $E_p$ epochs of pruning/fine-tuning and of $E_{ft}$ epochs of solely fine-tuning. Both in training and pruning, we scale the learning rate by multiplying it by $\gamma$ at the epochs specified by $\gamma_{step}$. Dropout, if present, is applied only after the first layer. The values of $E_t$,$E_p$, $E_{ft}$, $\gamma$, $\gamma_{step}$ and Dropout for the different datasets are specified in Table~\ref{table:neuraltrainparams}.  }
We use the Distiller~\cite{nzmora2019distiller} framework to prune the neural networks. Both in training and pruning, we employ Adam~\cite{kingma2014adam} as optimizer, with learning rate $0.001$ and no weight decay. Table~\ref{table:neuraltrainparams} summarizes the other training and pruning hyper-parameters, which are dataset-dependent. $E_t$ represents the number of training epochs. The pruning phase is composed of $E_p$ epochs of pruning/fine-tuning and of $E_{ft}$ epochs of only fine-tuning, as done by Han \textit{et al.
\cite{DBLP:journals/corr/HanPTD15}}.
Both for training and pruning, we scale the learning rate by multiplying it by $\gamma$ at the epochs specified by $\gamma_{step}$. Dropout, if employed (see Table~\ref{table:neuraltrainparams}), is applied only after the first layer.
When training and pruning the neural models, we always distill from the most effective ensemble of regression trees for the current dataset. On \msn, it is a model with 600 trees and 256 leaves per tree, reaching 0.5291 of NDCG@10, while on \istella it is a forest with $2500$ trees with $256$ leaves per tree, reaching 0.7821 of NDCG@10.	
 The neural forward pass is implemented in C++. We use the \textit{dnnl\_sgemm} routine from the OneDNN framework for dense matrix multiplication and the LIBXSMM~\cite{heinecke2016libxsmm} C++ library for sparse-dense matrix multiplication (after pruning).

\begin{table}[htb]
	\centering
	%\adjustbox{max width=\columnwidth}{
	\begin{tabular}{lrrrrrr}
		\toprule
		Dataset & $E_t$ & $E_p$& $E_{ft}$& $\gamma$& $\gamma_{step}$ & Dropout \\
		\midrule
		\msn & 100 & 80 & 20 & 0.1 & $ 50, 80 $ & - \\
		\istella & 250 & 60 & 190 & 0.5 & $ 90, 130, 180 $ & 0.1 \\
		\bottomrule
	\end{tabular}%}
	\caption{Training and pruning parameters employed for neural networks on \msn and \istella.\label{table:neuraltrainparams}}
\end{table}

\smallskip	

\noindent \textbf{Experimental Methodology}. We experimentally evaluate the performance of neural networks and ensemble of regression trees on two different experimental scenarios:

\begin{itemize}
	\item \textit{High-Quality Retrieval}: this scenario covers use cases where high-precision retrieval is required, even at the price of a larger scoring time. We impose a constraint on the retrieval quality to our models, specified by a threshold on the ranking metric. As threshold, we choose the 99\% of the retrieval quality of the top performing tree-based competitor on each dataset.
	\item \textit{Low-Latency Retrieval}: this scenario is orthogonal to the previous one as it focus on the efficiency of the retrieval process. We specify a maximum per-document scoring time and we select only the models that can match it. For both datasets, we set the maximum per-document scoring time to be $0.5\mu s$.
	%We pick so so that this scenario does not overlap with the previous one, enriching the comparison with new portion of the effectiveness-efficiency space. In fact, comparing Figures~\ref{fig:hq},~\ref{fig:lowlat} we can note how the scenarios are complementary on both the datasets.}
\end{itemize}
We perform the comparison between neural models and ensemble of regression trees by considering one scenario at a time. For each dataset, we consider the Pareto frontier of ensembles of tree-based models respecting the constraint of the considered scenario (green lines in Figures~\ref{fig:hq},~\ref{fig:lowlat}). By doing so, we train several tree-based competitors at different efficiency-effectiveness trade-offs. We then apply our technique and we show that neural networks can outperform ensembles of regression trees. We employ our time predictors to train and prune only neural network models that fit the time budget constrained by the ensembles of tree-based models considered. We recall that our methodology allows to train a neural model and to prune its first layer. In fact, in Section~\ref{sec:neuraleng}, we demonstrate that the first layer has a prominent impact on the overall execution time. By zeroing out at least  95\% of the parameters, its impact becomes negligible (Figure~\ref{fig:sparsespeedup}). Furthermore, the sparsification of the first layer has a positive effect on the generalization capabilities of the model as it act as a regularizer.
Then, we forecast the overall execution time by subtracting the contribution of the dense first layer from the overall execution time. Both times can be estimated with our dense predictor with no computational effort.

% due to 1) the positive regularization effect of sparsifing the first later and 2) the prominent impact of the first layer on the overall execution time.
%\fnote{frase sopra non ho capito. FM}
%Hence, we predict the overall execution time and the time spent on the first layer with our time predictor. As reported in Figure~\ref{fig:sparsespeedup}, if the sparsification is aggressive enough, the time spent to perform the sparse-dense multiplication of the first layer is negligible.

\smallskip
\noindent \textbf{Experimental Platform}. 
All the inference algorithms are compiled with GCC 9.2.1 with the \texttt{-O3} compiler option. 
Scoring times are measured on a Intel i9-9900K CPU clocked at 3.5 GHz, with AVX2 instructions, with a L1-cache of 256KiB, a L2-cache of 2 MiB, and a L3-cache of 16MiB. All the scoring experiments have been performed in single-thread execution.

%\fnote{mancano dettagli di compilazione e compilatore usato. vedi quickscorer. FM}

%This allows to train exclusively the models which strictly respect the time requirements. 
% In this section we provide an extensive evaluation of our proposed methodology. In particular, we extend the experimental results on the \msn dataset provided throughout the paper with an analysis performed on the \istella learning to rank dataset~\cite{lucchese2016exploiting}. We present our results by taking into account different efficiency-effectiveness tradeoffs. We start by introducing the datasets and the training methodology employed. We then describe our experimental methodology. Finally, we report our experimental results and we show that neural models can outperform ensembles of trees. 

% \subsection{datasets and Experimental Setup		}
% \fnote{non mi piace training parameters sopra.}

%In this section, we present the datasets on which we test our methodology and detail the training and pruning parameters for the neural models. 

% \smallskip
% \noindent \textbf{\msn}. We extend our evaluation on the \msn dataset~\cite{DBLP:journals/corr/QinL13}, a well-established benchmark to monitor the evolution of LtR techniques. The dataset is composed by more than 30,000 queries, with about 120 documents per query. Each document is represented by a vector of 136 features. The dataset is split into train-validation-test according to a 60\%-20\%-20\% criterion.
% \cosimo{We employ the LightGBM framework~\cite{NIPS2017_6907} to train the ensembles of regression trees. For each training, we consider the following set of hyper-parameters: [\textsl{learning rate, max depth, min\_sum\_hessian\_in\_leaf, min\_data\_in\_leaf}]. The  HyperOpt library \cite{bergstra2013making} is used to explore the hyper-parameters space and infer the optimal combination. We limit the number of trees to 1500 (larger models do not provide better retrieval quality) and apply an early stopping criterion on the validation loss. }
% %\cosimo{To train the best possible tree-based model on this dataset, we employ the LightGBM framework~\cite{NIPS2017_6907} to perform a grid search over the following set of parameters: [\textsl{learning rate, max depth, min\_sum\_hessian\_in\_leaf, min\_data\_in\_leaf}], using the HyperOpt library \cite{bergstra2013making} to infer the optimal combination. We limit the number of trees to 1500 and apply an early stopping criterion on the validation loss.}
% The best model we obtain has 600 trees and 256 leaves per tree, reaching 0.5291 of NDCG@10. 
% The neural networks are trained using Adam~\cite{kingma2014adam} as optimizer, with learning rate $0.001$, without dropout nor weight decay. We train for $100$ epochs,  multiplying the learning rate by $\gamma = 0.1$ at epochs $\{50, 80\}$. The overall pruning procedure lasts $100$ epochs. We prune once and retrain for $10$ epochs for the first $80$ epochs, then we fine-tune for the last $20$. During the fine-tuning epochs, all hyper-parameters, i.e., learning rate, weight decay, etc., are the same of the training phase. We employ an aggressive pruning on larger models for \emph{high-quality retrieval} by setting the sensitivity parameters $s \in [1.0, 1.5]$. For \emph{low-latency retrieval}, we use smaller models and we set $s \in [0.5, 0.8]$ since aggressive pruning cause performance degradation.

% \smallskip
% \noindent \textbf{\istella}. We also experiment using the \istella~\cite{lucchese2016post} \textit{small} (\istella) dataset. It consists of a collection of $33$,$018$ queries with an average of $103$ documents per query. Each document-query pair is represented by $220$ features. The dataset is splitted in train-validation-test according to a 60\%-20\%-20\% criterion. 
% \cosimo{We conduct a grid search using LightGBM and HyperOpt, on the same set of parameters detailed for \msn; this time we limit the number of trees to 2500.}
% The best model we could find has $2500$ trees with $256$ leaves per tree, reaching 0.7821 of NDCG@10. Observe that the size of this model is nearly the triple w.r.t. to the top performing model on \msn, suggesting that this is a more challenging dataset. 
% Neural models are trained using Adam~\cite{kingma2014adam} as optimizer, with learning rate $0.001$ without weight decay. Dropout is applied exclusively on the first layer.
% We train for $250$ epochs, multiplying the learning rate with $\gamma = 0.5$ at epochs $\{90, 130, 180 \}$. The pruning phase consists of a total of $250$ epochs: $60$ of pruning-retraining, with $10$ epochs of fine-tuning between one pruning iteration and the successive, and $190$ of solely fine-tuning. The training hyper-parameters remain the same as in the previous training phase. As for \msn, we employ aggressive pruning for larger models, $s \in [1.0, 1.5]$, and $s \in [0.5, 0.8]$ for smaller ones.


% \subsection{Methodology}
% \label{subsec:expmeth}
% We now present our experimental methodology to compare ensemble of regression trees with neural networks on the \emph{ad-hoc} document retrieval task \missingcite{}. \cosimo{Each experiment has the following macro-structure: first, we draw the Pareto optimality curve for tree-based models on a portion of the efficiency-effectiveness cartesian space. This portion is selected accordingly to one of two different requirement: \emph{high-quality retrieval} and \emph{low-latency retrieval}. Then, we develop our neural models to compete with ensemble of regression trees and QuickScorer.}

%  First, we pick a requirement in terms of scoring time or effectiveness. Then we train several ensembles of regression trees and score them with QuickScorer. This is preparatory to draw the Pareto optimality curve for the tree-based models in the portion of the effectiveness-efficiency trade-off identified by the chosen requirement. We use  the LightGBM framework~\cite{NIPS2017_6907} to train the ensembles of regression trees, with an early stopping criterion on $100$ iterations and the latest QuickScorer code~\cite{8035185}, written in C++ and exploiting AVX2 instructions, to score them. The largest value we permit for the number of tree leaves is $64$, to keep the execution time contained, as explained in section~\ref{subsec:approxbetter}. 



% \fnote{perche' facciamo la roba sopra? prima diciamogli cosa vogliamo fare e poi come lo facciamo. non si capisce la prima parte relativamente al cosa...}

% Before moving to the design of neural models, we train the best tree-model as possible to use as \textit{teacher} in Forest Distillation. Using the LightGBM Framework, we perform a grid search on the hyper-parameters, with no limits on the number of leaves. 
% Once we got the model, NNs are always trained to approximate the scores of this model. Observing the Pareto optimality curve of the ensembles of regression trees, we can derive the time requirements to compete with them. By mean of our dense time predictor, we easily devise the neural architectures matching these requirements. Successively, we train and evaluate them. Models are trained using Pytorch~\cite{NEURIPS2019_9015}, a widely-adopted python machine learning framework. We employ the same strategy for randomly generating training data of Cohen \textit{et al}.~\cite{cohen2018universal} and  RELU6 as activation function, where $\text{RELU6}(x) = min(max(x,0), 6)$.
% %
% The forward pass is implemented in C++, using the \textit{dnnl\_sgemm} routine from the OneDNN framework to perform matrix multiplication.
% CPU experiments are conducted on a Intel i9-9900K CPU, with AVX2 (latest generation of vectorized instructions supported by QuickScorer) instructions, 3.5 GHz, with L1-cache 256KiB, L2-cache 2 MiB, L3-cache 16MiB. 

% If dense models can not compete with ensemble of regression trees, we move to the sparse domain to overcome their limits. 	
%  We prune exclusively the first layer, because 1) its positive regularization effect on the effectiveness and 2) its prominent impact on the overall execution time.
%  We can estimate the execution time of the hybrid model by jointly applying our time predictors. We use the dense time predictor to predict the overall execution time and the time spent on the first layer. As reported in Figure~\ref{fig:sparsespeedup}, if the sparsification is sufficiently aggressive, the time spent to perform the sparse-dense multiplication of the first layer is negligible. Thus, we can forecast the overall execution time by subtracting the contribution of the dense first layer from the overall execution time. This allows to train exclusively the models which strictly respect the time requirements. 

%  The pruning techniques are applied using the Distiller~\cite{nzmora2019distiller} python framework for model compression, and the sparse dense multiplication induced by the sparsification of the weight matrix is carried out with the LIBXSMM~\cite{heinecke2016libxsmm} C++ library. 

% We propose two different requirements on both the datasets:
% \begin{itemize}
% 	\item \textit{High Quality Retrieval}: we compare tree-based models and neural models whose NDCG@10 falls in the 1	\% of the top performing model tree model with $64$ leaves
% 	\item \textit{Low Latency Retrieval}: models whose scoring time is below $0.5 \mu s $ per document
% \end{itemize}


% \fnote{fino a sez sopra. diciamo un sacco di cose ma serve una chiara intro di cio che vogliamo fare e poi diciamolo spiegandolo bene in ogni punto. rivedi in questo senso. FM}
\vspace{-0.2cm}
\subsection{Results}

\begin{figure}[t]
\begin{minipage}[b]{0.5\columnwidth}
\includegraphics[width=\columnwidth]{imgs/msn30k_final_stretched_broken.png}
\caption*{\footnotesize{\msn}}
\end{minipage}%
\begin{minipage}[b]{0.517\columnwidth}
\includegraphics[width=\columnwidth]{imgs/istella_final_stretched_broken.png}
\caption*{\footnotesize{\istella}}
\end{minipage}%
\caption{Comparison between neural networks and ensemble of regression tree on the \textit{high-quality retrieval} scenario.\label{fig:hq}}
\end{figure}
%\smallskip
\noindent \textbf{High-Quality Retrieval}.
The first scenario of our comparison involves models delivering high-quality ranking. As previously detailed, we consider a model (both neural and tree-based) to be in the high-quality ranking region if its NDCG@10 is at least the 99\% of the top-quality tree model with 64 leaves.
By following the experimental methodology described above, we first construct the Pareto frontier for the ensemble of regression trees (green line in Figure\ref{fig:hq}).
We then move to the design of the neural network models. We can estimate the execution time of a neural model whose first layer is sparse with our time predictors. In particular, in Table~\ref{table:msn_final} we report the estimated execution time for the dense architecture, the relative impact of the first layer on the overall execution time, and the predicted execution time after pruning the first layer.
We always assume the sparsity of the first layer in the final model to be above $95\%$, so that its impact on the overall execution time is negligible. Our experiments show that this level of sparsity does not hamper the ranking capability of the model.
Observe that our time predictors permit to locate a neural model on the $y$-axis of the effectiveness-efficiency plot without any computational effort, analytically computing it given the architectures of network. 
Once we have designed our models to compete with the tree-based ones, we train and prune them, according to the methodology listed in Section~\ref{subsec:expsetup}.
\begin{table}[b]
	\centering
	\adjustbox{max width=\columnwidth}{
	\begin{tabular}{llrrrr}
		\toprule
		\multirow{2}{*}{Dataset} & \multirow{2}{*}{Model} & Sc. Time  & \nth{1} layer & Predicted Pruned   \\
		&& ($\mu s$/doc) & impact (\%)& Sc. Time  ($\mu s$/doc) \\
		\midrule	
		\multirow{3}{*}{\msn} & 300$\times$200$\times$100 & 2.4 & 30 &  1.7 \\
		&\footnotesize 200$\times$100$\times$100$\times$50 & 1.3 & 39 & 0.8 \\
		&200$\times$50$\times$50$\times$25 & 0.9 & 58 & 0.4\\
 		\midrule	
 		\multirow{3}{*}{\istella} & \footnotesize800$\times$400$\times$400$\times$200& 11.9 & 23 &  9.1 \\
		&\footnotesize800$\times$200$\times$200$\times$100	& 6.5 & 41 & 	3.8 \\
		&300$\times$200$\times$100 & 2.8 & 41 & 1.6	\\
		\bottomrule
	\end{tabular}}
	\caption{Prediction of model scoring time (Sc. Time) when pruning the first layer, in \emph{High Quality Retrieval}.  }
	\label{table:msn_final}
\end{table}

Figure~\ref{fig:hq} illustrates the comparison on a effectiveness-efficiency plot between neural models and ensemble of regression trees scored with QuickScorer. On the $x$-axis we report the NDCG@10 on the test set, and on the $y$-axis the scoring time per document in $\mu s$. First, we observe that the predicted times reported in Table~\ref{table:msn_final} coincide with real scoring time, confirming the precision of our theoretical approach. Hence, our methodology allows to train exclusively the required architectures. Secondly, neural models can outperform tree-based models in scoring documents, both in terms of effectiveness and efficiency. The neural Pareto-optimality, reported in blue in  Figure~\ref{fig:hq}, lays below the tree-based one (in green), either on the \msn dataset and on \istella. On the \msn dataset, for example, the 300$\times$200$\times$100 architecture is $4.4$x faster than the $878$-trees model and it also provides a higher retrieval quality. Furthermore, the 200$\times$50$\times$50$\times$25 architecture is the fastest model respecting the quality constraint on this dataset. The same consideration holds for \istella, where the fastest model respecting the imposed quality constraint is a neural network (400$\times$200$\times$200$\times$100).
On this dataset, neural models still outperform ensemble of regression trees on a large portion of the selected effectiveness-efficiency space, even if tree-based model deliver a slightly superior performance in the top performing region.
This leaves space for research work to further improve the quality of this approximation. 
%For the sake of fairness, we point out that struggling at high level of retrieval quality is a general drawback of neural models, but it is also the only one, as detailed discussed in Section~\ref{sec:conclusions}.	
%Considering the top performing models on the \msn da the most effective models: ERT $600, 64$, NDCG@10 $0.5291$,  and $300x200x100$ sparse at the $98.6$ in the first layer, NDCG@10 $0.5258$. The gap between the two models in terms of NDCG@10 suggests than when aiming to maximize the retrieval quality, tree-based models are still to prefer to NNs, even if the price can be a $10x$ larger execution time w.r.t. to neural models.

% Observe that the models respect the predicted execution time reported in Table~\ref{table:\msn_final}, confirming the validity of our theoretical approach. Observe also  that neural models outperform the ensemble of regression trees scored with QuickScorer on \msn at any point of the efficiency-effectiveness trade-off; this can be evicted by fact that the neural Pareto curve always stands below the tree Pareto curve. The difference is especially evident between the two top NDCG@10 models. The 300x200x100 architecture is $4.4x$ faster than the 878-trees model, and $0.002$ more accurate in terms of NDCG@10, which is a relevant gap in terms of retrieval quality. For the sake of fairness, we point out the only possible disadvantage of using neural models; let us consider the most effective models: ERT $600, 64$, NDCG@10 $0.5291$,  and $300x200x100$ sparse at the $98.6$ in the first layer, NDCG@10 $0.5258$. The gap between the two models in terms of NDCG@10 suggests than when aiming to maximize the retrieval quality, tree-based models are still to prefer to NNs, even if the price can be a $10x$ larger execution time w.r.t. to neural models. 
% Then, we train an prune them; the most successful training configuration we found is to train for $250$ epochs,  using the dropout with  $p=0.1$ just on the first layer, no weight decay, and scheduling to multiply the learning rate with $\gamma = 0.5$ at epochs $\{90, 130, 180 \}$. The pruning phase consist of a total of $250$ epochs:  $60$ of pruning-retraining, with $10$ epochs of fine-tuning between one pruning iteration and the successive, and $190$ of solely fine-tuning. The training hyper-parameters remain the same as in the previous training phase. 
% In Figure~\ref{fig:\istella_high_quality} we report the comparison between the neural and tree-based Pareto Optimality curve. Even on a complex dataset as \istella, the neural models results more convenient on a large portion of the curve. In particular, NNs always to be preferred when the NDCG@10 is below $0.7220$. The tree-based models instead results advantageous with top-quality effectiveness requirements, \textit{i.e.}, NDCG@10 $ > 0.7740$, which NNs struggles to obtain. Observe also that the top performing ensemble of regression trees, ERT $2500, 256$, NDCG@10 0.7821, largely outscores our top performing model, the $800x400x400x200$, NCGG@10 0.7734 . 





\smallskip
\noindent \textbf{Low-Latency Retrieval}. We now compare neural models and ensemble of regression trees on a low-latency retrieval setting, \textit{i.e.}, a scenario requiring the scoring time to be lower than $0.5 \mu s$ per document. The Pareto Optimality curve for the ensemble of regression trees is drawn in green in Figure~\ref{fig:lowlat}. We use this plot to identify the latency constraints for our neural networks. Our proposed methodology permits to precisely estimate the execution time of a model, before carrying out the costly training-pruning phase. In Table~\ref{table:msn_lowq}, we demonstrate the usage of our methodology. As we did for the \emph{high-quality retrieval} use case (Table~\ref{table:msn_final}), we report the predicted execution time for the dense architecture, the relative impact of the first layer and the predicted time after sparsification. We still consider the impact of the first layer to be negligible.
%The models involved in this latency-bound use case are smaller, in order to match the latency constraints. Anyway, we experimentally verify that the pruning technique is capable to sparsify the first layer up to 95\%, annihilating the contribute of the first layer to the overall execution time. }
\begin{table}[b]
	\centering
	\adjustbox{max width=\columnwidth}{
	\begin{tabular}{llrrrr}
	%\resizebox{\columnwidth}{!}{
		\toprule
		%\multirow{2}{*}{Model} &    \multicolumn{2}{c}{Scoring Time ($\mu s$/doc)} \\
		%\cmidrule{2-3}
		\multirow{2}{*}{Dataset} & \multirow{2}{*}{Model} & Sc. Time  & \nth{1} layer   & Predicted Pruned  \\
		&& ($\mu s$/doc) & impact (\%)& Sc. Time  ($\mu s$/doc) \\
			\midrule
		%100x75x75x10 & 0.7 & 0.43 & 0.4\\
		\multirow{3}{*}{\msn}&100$\times$50$\times$50$\times$25 & 0.6 & 56 & 0.3\\
		&100$\times$25$\times$25$\times$10 & 0.5 & 71 & 0.2\\
		&50$\times$25$\times$25$\times$10 & 0.3 & 65 & 0.1 \\
 		\midrule
		\multirow{3}{*}{\istella}&  200$\times$75$\times$75$\times$25&1.6 & 61 &  0.6 \\
		&100$\times$75$\times$75$\times$10& 0.9 & 55 & 	0.4 \\
 		&100$\times$50$\times$50$\times$10 & 0.8 & 67 & 0.3	\\
		\bottomrule
	\end{tabular}
	}
	\caption{Prediction of model scoring time (Sc. Time) when pruning the first layer, in \emph{Low-Latency Retrieval}.\label{table:msn_lowq}}
\end{table}


\begin{figure}[t]
\begin{minipage}[b]{0.5\columnwidth}
\includegraphics[width=\columnwidth]{imgs/low_effectiveness_msn30k_stretched.png}
\centering 
\caption*{\footnotesize{\msn}}
\end{minipage}%
\begin{minipage}[b]{0.51\columnwidth}
\includegraphics[width=\columnwidth]{imgs/low_effectiveness_istella_stretched.png}
\centering 
\caption*{\footnotesize{\istella}}
%\subcaption{Another subfigure}
\end{minipage}%
\caption{Comparison between neural networks and ensemble of regression tree on the \textit{low-latency retrieval} scenario.}
\label{fig:lowlat}
\end{figure}



Figure~\ref{fig:lowlat} illustrates the comparison between neural model and ensemble of regression trees when dealing with low-latency constraints. Even in this case, our methodology permits to create neural networks that  outperform ensembles of regression trees. On the \msn dataset, neural models dominate over tree-based models, as happened for the \textit{high-quality} use case. In fact,  the Pareto frontier of neural models (in blue) always lies below the tree-based one, confirming the superiority of our technique on this dataset (left side of Figure~\ref{fig:lowlat}). In particular, the 200$\times$50$\times$50$\times$25 architecture is $3$x faster than the regression forest with $300$ trees and $32$ leaves, while being also more precise in terms of NDCG@10. On the \istella dataset (right side of Figure~\ref{fig:lowlat}, the performance of our neural models can be considered on pair with tree-based models. In fact, the two Pareto frontiers intersect in this portion of the efficiency-effectiveness trade-off. Despite that, neural networks  still provide the most effective model respecting the time requirement (200$\times$75$\times$75$\times$25). 
This dataset confirms to be troublesome for neural models, as witnessed in the high-quality retrieval scenario. 
% We leave for future work the investigation of distillation techniques tackling these difficulties on this specific dataset.
%\fnote{la chiusura con un future work la eviterei. FM}

% Once again, we design our neural model just using our time predictors. 
% We keep the $200x50x50x25$ neural model from the previous experimental setting and devise three more architectures. 

% %mainly composed by trees with $8$ or  $16$ leaves. A minor number of leaves and/or trees, in fact, allows for faster scoring with QuickScorer, at the price of a reduced expressiveness. For what concerns the neural models, observe that 
% %the $200x50x50x25$ model previously developed already matches the requirement of  $ 0.5 \mu s$ of scoring time per document. Furthermore, we develop three more models using the same methodology as before: we estimate the execution time of the pruned models as the total execution time subtracted with the contribution of the first layer. We highlight that the models' design is completely experiments-free, thus costless. 

% In Table~\ref{table:\msn_lowq} we report the predicted execution times after the sparsification; because of the reduced sizes of the layers, we assume our processor to perform at 60 GFLOP/s, as also suggested by Figure~\ref{fig:heatmap}. We train an then prune the resulting models; the training and pruning recipes are exactly the same as in the \textit{High Quality Retrieval} use case. The only difference is that  we use medium or low aggressive pruning ($s \in [ 0.5, 1.0]$) since these architectures have few parameters and aggressively sparsify them could cause unbearable performance degradation. In Figure~\ref{fig:\msn_lowlat} we report both the Pareto Optimality curves. As for the high quality retrieval, the real scoring times reflect the predicted scoring time reported in Table~\ref{fig:\msn_lowlat}. %The very-low effort time analysis allows to train only a reduced number of models, tearing down the resource needed for training, \textit{i.e.}, time, money, energy. 
% Even in this case, the neural Pareto Optimality curve always appears below the tree Pareto Optimality. As an example, the architecture $200x100x100x50$ is $0.002$ more accurate than the $300t, 32l$ tree-based model, while being $3x$ times faster. The validity of our theoretical-based approach to efficiency is confirmed even in this use case, especially considering that the model design phase involves no computational effort, tearing down the resource needed for training, \textit{i.e.}, time, money, energy. 
 %As for \msn, we now move to compare tree-based and neural model whose scoring time is about $ 0.5 \mu s$ per document. The profile of the Pareto Optimality curve for ensemble of regression trees is reported in green in Figure~\ref{fig:lowlat\istella}. Following the same procedure we follow during the all experimental evaluation, we use our time predictor to devise the models that we will train, as reported in Table~\ref{table:\istella_lowlat}. To train and prune the architecture we use the same parameters configuration as in \textit{High Quality Retrieval}; the pruning aggressiveness is reduced because of the small size of the models. We report our results in Figure~\ref{fig:lowlat\istella}: as shown, neural models are capable to compete with tree-based ones even with low-latency requirements on the \istella dataset. 






% \subsection{MSN 30K }
% In this section we report the results of our experiments on the \msn dataset~\cite{DBLP:journals/corr/QinL13}, a well-established benchmark to monitor the evolution of LtR techniques. The dataset is composed  by more than 30,000 queries, with about 120 documents per query; each document is a vector of 136 features. The whole dataset is split into train-validation-test according to a 60\%-20\%-20\% criterion. 

% As already reported in Table~\ref{table:64vs256leaves}, the most effective ensemble of regression trees  we could obtain on \msn is a model with 600-trees and 256 leaves per tree, reaching 0.5291 of NDCG@10. We used this model as \textit{teacher} for all of our neural models. 

% \smallskip
% \noindent \textbf{High Quality Retrieval}
% We begin our comparison between tree-based and neural models on \msn from models whose NDCG@10 is the in the 1\% of the top quality tree model with 64 leaves. 
% Following the guidelines in the preamble at the beginning of this section, we first construct the Pareto Optimality curve for the ensemble of regression trees, reported in green in Figure\ref{fig:\msn_highperf}. Hence, we move to the design of our hybrid sparse-dense neural models; as anticipated in the preamble, we can estimate the execution time of a first layers sparse neural model with our time predictors, as reported in Table~\ref{table:\msn_final}. We always assume the sparsity of the first layer in the final model to be above $95\%$, so that its impact on the overall execution time is negligible. 
% Observe that this method allows us to locate a neural  model on the y-axis of the effectiveness-efficiency plot without any computational effort, just using the architecture of the network.

% Once we have designed our models to compete with the tree-based ones, we train and prune them. In the training phase, additionally to the information specified in the preamble, we highlight that we did not use dropout nor weight decay on our models: we train for $100$ epochs, and  multiply the learning rate by $\gamma = 0.1$ at epochs $\{50, 80\}$. We employ an aggressive pruning ($s=1.5$) for all the models except for the smallest one, for which we use $s=1.0$. The overall pruning procedure lasts $100$ epochs; we prune once and retrain for $10$ epochs for the first $80$ epochs, then we solely fine-tune for the last $20$. During the fine-tuning epochs, the hyper-parameters, learning rate, weight decay, etc..,  are set as in the training phase. In Figure ~\ref{fig:\msn_highperf} we draw the Neural Pareto Optimality curve. Observe that the models respect the predicted execution time reported in Table~\ref{table:\msn_final}, confirming the validity of our theoretical approach. Observe also  that neural models outperform the ensemble of regression trees scored with QuickScorer on \msn at any point of the efficiency-effectiveness trade-off; this can be evicted by fact that the neural Pareto curve always stands below the tree Pareto curve. The difference is especially evident between the two top NDCG@10 models. The 300x200x100 architecture is $4.4x$ faster than the 878-trees model, and $0.002$ more accurate in terms of NDCG@10, which is a relevant gap in terms of retrieval quality. For the sake of fairness, we point out the only possible disadvantage of using neural models; let us consider the most effective models: ERT $600, 64$, NDCG@10 $0.5291$,  and $300x200x100$ sparse at the $98.6$ in the first layer, NDCG@10 $0.5258$. The gap between the two models in terms of NDCG@10 suggests than when aiming to maximize the retrieval quality, tree-based models are still to prefer to NNs, even if the price can be a $10x$ larger execution time w.r.t. to neural models. 

% \begin{table}[]
% 	\centering

% 	\adjustbox{max width=\columnwidth}{
% 	\begin{tabular}{lrrrr}
% 	%\resizebox{\columnwidth}{!}{
% 		\toprule
% 		%\multirow{2}{*}{Model} &    \multicolumn{2}{c}{Scoring Time ($\mu s$/doc)} \\
% 		%\cmidrule{2-3}
% 		\multirow{2}{*}{Model} & Scoring Time  & \nth{1} layer   & Predicted Pruned  \\
% 		& ($\mu s$/doc) & impact (\%)& Scoring Time  ($\mu s$/doc) \\
% 		\midrule
% 		\sc{800x400x400x200}& 11.9 & 0.23 &  9.1 \\
% 		800x200x200x100	& 6.5 & 0.41 & 	3.8 \\
% 		300x200x100 & 2.8 & 0.41 & 1.6	\\
%  		\bottomrule
% 	\end{tabular}
% 	  }
% 	\caption{Predicting the scoring time of Neural Models for High Quality Retrieval on \istella, assuming the first layer to be sparse.  }
% 	\label{table:\istella_final}

% \end{table}


% \begin{figure}
% 	\centering
% 	\includegraphics[width=\columnwidth]{imgs/\msn_final.png}
% 	\caption{High Quality Retrieval comparison between Additive Ensemble of Regression Trees, scored with QuickScorer, and Neural Models on \msn }
% 	\label{fig:\msn_highperf}
% \end{figure}


% \smallskip
% \noindent \textbf{Low Latency Retrieval}
% We now move to the compare neural models and ensemble of regression trees on low latency constraints, \textit{i.e.}, scoring time below $0.5 \mu s$ per document. 	
% The Pareto Optimality curve for the ensemble of regression trees is draw in green in Figure~\ref{fig:\msn_lowlat}. 
% Once again, we design our neural model just using our time predictors. 
% We keep the $200x50x50x25$ neural model from the previous experimental setting and devise three more architectures. 

% %mainly composed by trees with $8$ or  $16$ leaves. A minor number of leaves and/or trees, in fact, allows for faster scoring with QuickScorer, at the price of a reduced expressiveness. For what concerns the neural models, observe that 
% %the $200x50x50x25$ model previously developed already matches the requirement of  $ 0.5 \mu s$ of scoring time per document. Furthermore, we develop three more models using the same methodology as before: we estimate the execution time of the pruned models as the total execution time subtracted with the contribution of the first layer. We highlight that the models' design is completely experiments-free, thus costless. 

% In Table~\ref{table:\msn_lowq} we report the predicted execution times after the sparsification; because of the reduced sizes of the layers, we assume our processor to perform at 60 GFLOP/s, as also suggested by Figure~\ref{fig:heatmap}. We train an then prune the resulting models; the training and pruning recipes are exactly the same as in the \textit{High Quality Retrieval} use case. The only difference is that  we use medium or low aggressive pruning ($s \in [ 0.5, 1.0]$) since these architectures have few parameters and aggressively sparsify them could cause unbearable performance degradation. In Figure~\ref{fig:\msn_lowlat} we report both the Pareto Optimality curves. As for the high quality retrieval, the real scoring times reflect the predicted scoring time reported in Table~\ref{fig:\msn_lowlat}. %The very-low effort time analysis allows to train only a reduced number of models, tearing down the resource needed for training, \textit{i.e.}, time, money, energy. 
% Even in this case, the neural Pareto Optimality curve always appears below the tree Pareto Optimality. As an example, the architecture $200x100x100x50$ is $0.002$ more accurate than the $300t, 32l$ tree-based model, while being $3x$ times faster. The validity of our theoretical-based approach to efficiency is confirmed even in this use case, especially considering that the model design phase involves no computational effort, tearing down the resource needed for training, \textit{i.e.}, time, money, energy. 
 

% \begin{table}[]
% 	\centering

% 	\adjustbox{max width=\columnwidth}{
% 	\begin{tabular}{lrrrr}
% 	%\resizebox{\columnwidth}{!}{
% 		\toprule
% 		%\multirow{2}{*}{Model} &    \multicolumn{2}{c}{Scoring Time ($\mu s$/doc)} \\
% 		%\cmidrule{2-3}
% 		\multirow{2}{*}{Model} & Scoring Time  & \nth{1} layer   & Predicted Pruned  \\
% 		& ($\mu s$/doc) & impact (\%)& Scoring Time  ($\mu s$/doc) \\
% 			\midrule
% 		%100x75x75x10 & 0.7 & 0.43 & 0.4\\
% 		100x50x50x25 & 0.6 & 0.56 & 0.3\\
% 		100x25x25x10 & 0.5 & 0.71 & 0.2\\
% 		50x25x25x10 & 0.3 & 0.65 & 0.1 \\
%  		\bottomrule
% 	\end{tabular}
% 	  }
% 	\caption{Predicting the scoring time of Neural Models for Low Latency Retrieval on \msn, assuming the first layer to be sparse.  }
% 	\label{table:\msn_lowq}

% \end{table}


% \begin{figure}
% 	\centering
% 	\includegraphics[width=\columnwidth]{imgs/low_effectiveness_\msn.png}
% 	\caption{Low Latency Retrieval comparison between Additive Ensemble of Regression Trees, scored with QuickScorer, and Neural Models on \msn }
% 	\label{fig:\msn_lowlat}
% \end{figure}


% \subsection{\istella dataset}
% The second part of our experimental evaluation is performed on the \istella~\cite{lucchese2016post} dataset by Tiscali. In order to ease reproducibility, we pick the the \textit{small} version (\istella); it consists of a collection of 33,018 queries with an average of $103$ documents per query. Each document-query pair is represented by $220$ features. The train-validation-test split	respects the same pattern as \msn, \textit{i.e.}, 60\%-20\%- 20\%. 

% As first step, we perform a grid search on the hyper-parameters to obtain the most effective tree-based model as possible, regardless of efficiency. The best model we could find is a $2500$ trees with $256$ leaves per tree, reaching 0.7821 of NDCG@10. Observe that the size of this model is nearly the triple w.r.t. to the top performing model on \msn, immediately suggesting that this is a more challenging dataset.  


% \smallskip
% \noindent \textbf{High Quality Retrieval} We start the comparison on \istella, as for \msn, from high quality retrieval. The Pareto Optimality curve for the ensembles of regression trees is reported in Figure~\ref{fig:\istella_high_quality}; diversely from \msn, we witness to a dilation of the performance as the number of trees grows. As shown in Figure~\ref{fig:\msn_highperf}, the 300-trees model on \msn performs almost as well as the 500-trees one, with a gap in terms of NDCG@10 of about $0.01$; instead, on \istella, the difference in terms of effectiveness is about $4x$ bigger. This suggest us this difference of accuracy between large and small models to be present also in neural models, namely that we will need larger models w.r.t. to \msn. 

% Applying the same procedure as before, we devise our neural models just leveraging the dense time predictor, as reported in Table ~\ref{table:\msn_final}. Then, we train an prune them; the most successful training configuration we found is to train for $250$ epochs,  using the dropout with  $p=0.1$ just on the first layer, no weight decay, and scheduling to multiply the learning rate with $\gamma = 0.5$ at epochs $\{90, 130, 180 \}$. The pruning phase consist of a total of $250$ epochs:  $60$ of pruning-retraining, with $10$ epochs of fine-tuning between one pruning iteration and the successive, and $190$ of solely fine-tuning. The training hyper-parameters remain the same as in the previous training phase. 

% In Figure~\ref{fig:\istella_high_quality} we report the comparison between the neural and tree-based Pareto Optimality curve. Even on a complex dataset as \istella, the neural models results more convenient on a large portion of the curve. In particular, NNs always to be preferred when the NDCG@10 is below $0.7220$. The tree-based models instead results advantageous with top-quality effectiveness requirements, \textit{i.e.}, NDCG@10 $ > 0.7740$, which NNs struggles to obtain. Observe also that the top performing ensemble of regression trees, ERT $2500, 256$, NDCG@10 0.7821, largely outscores our top performing model, the $800x400x400x200$, NCGG@10 0.7734 . 
% \begin{table}[]
% 	\centering

% 	\adjustbox{max width=\columnwidth}{
% 	\begin{tabular}{lrrrr}
% 	%\resizebox{\columnwidth}{!}{
% 		\toprule
% 		%\multirow{2}{*}{Model} &    \multicolumn{2}{c}{Scoring Time ($\mu s$/doc)} \\
% 		%\cmidrule{2-3}
% 		\multirow{2}{*}{Model} & Scoring Time  & \nth{1} layer   & Predicted Pruned  \\
% 		& ($\mu s$/doc) & impact (\%)& Scoring Time  ($\mu s$/doc) \\
% 		\midrule
% 		800x400x400x200& 11.9 & 0.23 &  9.1 \\
% 		800x200x200x100	& 6.5 & 0.41 & 	3.8 \\
% 		300x200x100 & 2.8 & 0.41 & 1.6	\\
%  		\bottomrule
% 	\end{tabular}
% 	  }
% 	\caption{Predicting the scoring time of Neural Models for High Quality Retrieval on \istella, assuming the first layer to be sparse.  }
% 	\label{table:\istella_final}

% \end{table}




% \begin{figure}
% 	\centering
% 	\includegraphics[width=\columnwidth]{imgs/\istella_final.png}
% 	\caption{High Quality Retrieval comparison between Additive Ensemble of Regression Trees, scored with QuickScorer, and Neural Models on \istella }
% 	\label{fig:\istella_high_quality}
% \end{figure}

% \smallskip 
% \noindent \textbf{Low Latency Retrieval} As for \msn, we now move to compare tree-based and neural model whose scoring time is about $ 0.5 \mu s$ per document. The profile of the Pareto Optimality curve for ensemble of regression trees is reported in green in Figure~\ref{fig:lowlat\istella}. Following the same procedure we follow during the all experimental evaluation, we use our time predictor to devise the models that we will train, as reported in Table~\ref{table:\istella_lowlat}. To train and prune the architecture we use the same parameters configuration as in \textit{High Quality Retrieval}; the pruning aggressiveness is reduced because of the small size of the models. We report our results in Figure~\ref{fig:lowlat\istella}: as shown, neural models are capable to compete with tree-based ones even with low-latency requirements on the \istella dataset. 
% \begin{table}[]
% 	\centering

% 	\adjustbox{max width=\columnwidth}{
% 	\begin{tabular}{lrrrr}
% 	%\resizebox{\columnwidth}{!}{
% 		\toprule
% 		%\multirow{2}{*}{Model} &    \multicolumn{2}{c}{Scoring Time ($\mu s$/doc)} \\
% 		%\cmidrule{2-3}
% 		\multirow{2}{*}{Model} & Scoring Time  & \nth{1} layer   & Predicted Pruned  \\
% 		& ($\mu s$/doc) & impact (\%)& Scoring Time  ($\mu s$/doc) \\
% 		\midrule
% 		200x75x75x25&1.6 & 0.61 &  0.6 \\
% 		100x75x75x10& 0.9 & 0.55 & 	0.4 \\
% 		100x50x50x10 & 0.8 & 0.67 & 0.3	\\
%  		\bottomrule
% 	\end{tabular}
% 	  }
% 	\caption{Predicting the scoring time of Neural Models for Low Latency Retrieval on \istella, assuming the first layer to be sparse.  }
% 	\label{table:\istella_lowlat}

% \end{table}



% \begin{figure}
% 	\centering
% 	\includegraphics[width=\columnwidth]{imgs/low_effectiveness_\istella.png}
% 	\caption{Low Latency Retrieval comparison between Additive Ensemble of Regression Trees, scored with QuickScorer, and Neural Models on \istella }
% 	\label{fig:lowlat\istella}
% \end{figure}


% In Figure~\ref{fig:quick_\msn_overview} , we report the models which define the \textit{lowest trade-off curve }. The NDCG@10 is computed on the test set, using RankEval~\cite{rankeval-sigir17} while the scoring time is computed with QuickScorer,  averaged on 10 runs. We considered as minimum requirement an NDCG@10 of 0.518. Figure~\ref{fig:quick_\msn_overview} clearly explicits the linear relationship between scoring time and number of trees in QuickScorer: as the number of trees doubles, so does the execution time. The same ratio exists also between the number of leaves and the scoring time:  the $(300, 64)$ model has a execution time that doubles the  $(300, 32)$ model. We did not report the execution time of our most effective model, the $(600, 256)$, but we can easily infer it from the $(300, 64)$ architecture: $$ T_{(600, 256)} = 2^3 *  T_{(300, 64)} \simeq 24 \mu  s$$
%  since we have to double the scoring time twice for the number of  leaves and once for the number of trees. 

% \begin{figure}
% 	\centering
% 	\includegraphics[width=\columnwidth]{imgs/quickscorer_\msn_overview.png}
% 	\caption{Performance Overview of Additive Ensemble of Regression Trees and QuickScorer on \msn }
% 	\label{fig:quick_\msn_overview}
% \end{figure}



% %TODO deicidere se spiegare ora o prima l'idea di migliorare la foresta

% We trained four different dense models. Three of them follow the $2l$-$l$-$l$-$\frac{l}{2}$ architectural pattern that we followed in Section~\ref{sec:neuraleng}. We also propose a three layers architecture $3l$-$2l$-$l$. We divide them into \textit{tiny} and \textit{small}  models, as reported in Table~\ref {table:dense_msn}.%  The reduced number of parameters is necessary to compete with trees models in terms of scoring time, but it causes a lack of highly effective models. Furthermore, neural model are not fast enough to be the solution for really high performance scenarios.

% Then, we applied our efficiency-oriented pruning. We prune the first (one or two) layers of the network thus leveraging both the regularization effect and the scoring time speed up of early-layers sparsification. 
% We applied the Han flavor of pruning, whose aggressiveness is defined by the sensitivity parameter $s$. 
% We used an \textit{aggressive} sparsification strategy for \textit{small} network, pruning the first or the first and the second layer with $s=1.6$, while we adopted a softer method for \textit{tiny} networks, pruning just the first layer with $s = 1.0$. 

% In Figure~\ref{fig:\msn_final} we show our experimental results. Our efficiency-oriented pruning allows neural model to outperform additive ensemble of regression trees at any point of the effectiveness-efficiency tradeoff. Our fastest network has the same scoring time as the fastest tree-based model, while being more accurate. The most accurate neural models has higher NDCG@10 than the best Regression Forest, being 4 $\times$ faster. Both the fastest and the most precise model derive from the sparsification of the 300x200x100 architecture, which covers a large part of the tradeoff by itself. The 200x100x100x50 model completes the task being faster and more accurate of the $(150, 64)$ tree model, which was the only one not covered by the 300x200x100 architecture. Even the smallest neural model, benefits from our methodology, but still does not surpass the 0.518 NDCG@10 threshold and it is not represented in the plot.   


% \begin{table}

% \adjustbox{max width = \columnwidth}{

% 	\begin{tabular}{llrr}
% 		\toprule
% 		Model &   Size&  NDCG@10 &    Scoring Time ($\mu S$)\\
% 		\midrule
% 		100x50x50x25& \textit{tiny} 		 &0.5151   &0.5\\
% 		200x100x100x50& \textit{tiny}  &0.5187&   1.4 \\
% 		300x200x100&\textit{small}  & 0.5216&  2.5\\
% 		400x200x200x100& \textit{small} & 0.5221 & 3.8\\
		  
% 		\bottomrule
% 	\end{tabular}
% 	}
% 	\caption{Dense Neural Models on MSN 30K  }
% 	\label{table:dense_msn}
% \end{table}

% %Then, we applied our efficiency oriented pruning. For each model, we applied it on the first layer and on the first and the secon
% \begin{figure}
% 	\centering
% 	\includegraphics[width=\columnwidth]{imgs/\msn_final.png}
% 	\caption{Performance Overview of Neural Network on \msn }
% 	\label{fig:\msn_final}
% \end{figure}


% \subsection{Na\"ive comparison}

% We started reproducing the experimental settings of the original article~\cite{cohen2018universal}. We trained two models, a Large Network with 4 layers of size $\{2000,500,5000,100\}$ and a Small Network with 2 layers of shapes $\{500,100\}$. We adopted the same strategy for randomly generating training data as in the original work~\cite{cohen2018universal}. We used Adam as optimizer, with learning rate $0.001$ and no weight regularization; we multiplied the learning rate by $\gamma = 0.1$ at epochs $\{50, 80 \}$ and use and early stopping criterion on the validation loss. 

% As previoulsy mentioned, we excluded the GPU-based inference from the comparison due to data transfer costs. The CPU inference comparison was originally carried out comparing an old version of Quickscorer\footnote{The original Quickscorer algorithm is undergoing a patent process} - without SIMD instructions - with a Python-based forward implementation for NNS, on different hardware. To provide a fair, production-oriented comparison between the two techniques, we wrote our own C++ version of the Multi Layer Perceptron inference and we used the last Quickscorer code. We expolited the \textit{dnnl\_sgemm} routine from Intel oneDNN framework to implement matrix multiplications, with JIT compilation, always forcing single-thread execution.  
% Experiments were conducted on a Intel i9-9900K CPU, with AVX2 instructions, 3.5 GHz, with L1-cache 256KiB, L2-cache 2 MiB, L3-cache 16MiB. We observe that AVX2 is not the latest ISA available on Intel Processor, but it is the one Quickscorer was implemented with; this choice was made for the sake of fair comparison. 

%  %To obtain a fair comparison in terms of scoring time, we tested both QuickScorer and NN models on a Intel i9-9900K CPU, with AVX2 instructions, always forcing single thread execution. In paricular, we write our own C++ inference model to avoid Pytorch overhead; this allows to speed up the execution time up to $10x$ for the small models. The large network is a 4 layer MLP with hidden sizes  $\{2000,500,5000,100\}$, while the small network is a 2 layer of shapes $\{500,100\}$. We adopt the same strategy for randomly generating training data. We use Adam as optimizer, with learning rate $0.001$ and no weight regularization; we multiply the learning rate by $\gamma = 0.1$ at epochs $\{50, 80 \}$ and use and early stopping criterion on the validation loss.  
% %TODO aggiungere tempo di esecuzione di pytorch sequenziale, pytorch mutlithread e gpu per confronto con articolo origianale
% %TODO aggiungere valori confrontabili

% \begin{table}

% \adjustbox{max width = \columnwidth}{

% 	\begin{tabular}{lrrrr}
% 		\toprule
% 		Model &     NDCG@10 &     MAP 0 & MAP 1& Scoring Time ($\mu S$)\\
% 		\midrule
	
% 		Regression Forest(64 leaves)&    0.5246& 0.6304  &0.6604  & 2.50	\\
% 		\midrule
% 		Large Network &   0.5198&0.6279   &0.6579  & 24.41\\
% 		Small Network & 0.5180& 0.6277 &0.6576 & 2.25\\
		  
% 		\bottomrule
% 	\end{tabular}
% 	}
% 	\caption{Comparison between a Regression Forest (878 trees, 64 leaves) and Neural Networks on \msn.  }
% 	\label{table:repr_comp}
% \end{table}
% In Table~\ref{table:repr_comp} we report our experimental results. As shown, with these settings Regression Forests \& Quisckscorer outperform Neural Networks both in effectiveness and efficiency. The gap in terms of NDCG@10 and Mean Average Precision (MAP) indicates the NNs are not capable to exactly approximate the $R(x)$ function mapped by the Regression Forest. 
% %todo dire qualcosa su Universal Approximation Function.
% For what concerns efficiency, the Large Network is $10x$ slower w.r.t. to Quickscorer, while the smaller network is slightly faster. Anyway, Table~\ref{table:repr_comp} shall not suggest that Regression Forest are more effective while NNs are more efficient. Since the gap in terms of NDCG@10 and MAP is consistent, we should compare the scoring time of the small network with the scoring time of a Regression Forest that has a comparable effectiveness. As shown in table~\ref{table:tree_overview}, we need less than 200 trees to reach the same NDCG@10 as the Small Network.  
% \begin{table}
% 	\begin{tabular}{rrrrr}
% 		\toprule
% 		Model &     NDCG@10 &Scoring Time ($\mu S$)\\
% 		\midrule
% 		100 trees & 0.5177 & 0.63\\
% 		150 trees & 0.5197 & 0.86\\
% 		200 trees & 0.5212 & 1.06\\
% 		300 trees & 0.5224 & 1.44\\
% 		400 trees & 0.5228 & 1.79\\
% 		500 trees & 0.5239 & 2.02\\ 

% 		\bottomrule
% 	\end{tabular}
% 	\caption{Overview of Regression Trees NDCG@10 and scoring time with QuickScorer }
% 	\label{table:tree_overview}
% \end{table}
% Under this settings, Regression trees with QuickScorer largely outscore NNs on the document scoring task. 

% \subsection{Improving the Regression Forest}
% An approch to bridge the gap - effectiveness wise - between NNs and Regression Trees is to exploit the Universal Approximation Theorem in a smarter way. Let $\Delta$ be the effectiveness gap between the Regression Forest and the Neural Networks, in terms of a chosen metric $M$ (NDCG@10, MAP). Let us also assume that is $\Delta$  is constant and does not depend on the Regression Forest function $R(x)$.  So
% $$ M_{N(x)} = M_{R(x) } - \Delta $$
% where $N(x)$ is the Neural Network Function, and $M_{F(x)}$ means the value of metrics $M$ with scoring function $F$.
% Instead of working on trying to reduce  $\Delta$, we can raise tha value of $M_{N(x)}$ increasing the effectiveness of the Regression Forest. We train a larger Regression Forest
% %todo chiedere a salvo i dettagli del trianing
% with 256 leaves for tree and 600 trees, which consistently outperforms the previous 64-leaves forest, as shown  (64 leaves) in Table~\ref{table:64vs256leaves}. 
% \begin{table}
% \begin{tabular}{rrrr}
% 		\toprule
% 		Model &     NDCG@10 &     MAP 0 & MAP 1 \\
% 		\midrule
	
% 		878 trees, 64 leaves &    0.5246& 0.6304  &0.6604 	\\
		
% 		600 trees, 256 leaves &   0.5291&0.6321   &0.6621  \\
		
		  
% 		\bottomrule
% 	\end{tabular}
% 	\caption{Comparison between Regression Forests with different number of trees and leaves.   }
% 	\label{table:64vs256leaves}
% \end{table}
% We now train the MLP models to mimic the scores obtained with the new, more effective model. %todo loosely inspired by knowledge distillation
% First, we discard the Large Model provided by~\cite{cohen2018universal}, due to its unbearable execution time. We provide a new Medium Model with shapes $\{1000, 500, 500, 100 \}$, that halves the execution time w.r.t. to the Large one. Under this new training configuration, this model can reach the same effectiveness of the Regression Forest with 878 trees and 64 leaves. 
% \begin{table}
% \adjustbox{max width = \columnwidth}{
% 	\begin{tabular}{rrrrr}
% 		\toprule
% 		 Model &     NDCG@10 &     MAP 0 & MAP 1& Scoring Time ($\mu S$)\\
		
% 		\midrule
	
% 		Regression Forest &    0.5246& 0.6304  &0.6604  & 2.50	\\
% 		\midrule
% 		Medium Network &   0.5243&0.6597   &0.6297  & 14.54\\
% 		Small Network & 0.5190& 0.6278 &0.6578 & 2.25\\
		  
% 		\bottomrule
% 	\end{tabular}
% 	}
% 	\caption{Comparison between a Regression Forest (878 trees, 64 leaves) and Neural Networks on \msn.  }
% 	\label{table:repr_comp}
	
% \end{table}
% Our assumption that $\Delta$ is constant results to be quiet accurate, as shown in Table~\ref{table:repr_comp}. Using a better model to produce the score on which we perform the regression is profitable since we obtain models which are more effective without affecting the scoring time. Using a four layer MLP we match the same accuracy as the Regression Forest, while we are still $7x$ times slower than QuickScorer in scoring the documents. The effectiveness gap for the 2 layers MLP still persists; the Small Network slightly benefits from the score generator improvement. This suggests us that the small MLP is not expressive enough to perform regression on the scores. This could be caused by the reduced number of weights:
% the Large Network has 936K parameters, while the Small Network has 118K. Anyway, Table~\ref{table:small_vs_large_network} shows that deeper networks generally work better.
% %TODO dire qualcosa sulla tabella
%  Our explanation is that adding more layers allows to combine and produce higher level features. Inspired by these considerations, we focus on 4 layers neural networks. 

% %TODO AGGIUNGERE ESPERIMENTI CON RETI A 5 LIVELLI
% %TODO cercare/esperiment su reti cilindriche
% %todo abbiamo anche reti leggermente migliori della prima, con lo stesso budget di parametri. 
% \begin{table}
% \adjustbox{max width = \columnwidth}{
% 	\begin{tabular}{rrrr}
% 		\toprule
% 		Model & Parameters &NDCG@10 & Scoring Time ($\mu S$)\\
% 		\midrule
% 		500x100 & 118K & 0.5196 & 2.25 \\
% 		1000x200 &336K& 0.5155 & \\
% 		2000x400 & 1M & 0.5158& \\
% 		\midrule
% 		200x100x100x50 & 63K & 0.5187& 1.36 \\
% 		300x150x150x30 & 112K & 0.5207 & 2.29 \\
% 		400x200x200x100 & 194K & 0.5220 & 3.85 \\
		  
% 		\bottomrule
% 	\end{tabular}
% 	}
% 	\caption{Comparison between 2 and 4 layers architectures on \msn.  }
% 	\label{table:small_vs_large_network}
	
% \end{table}







% \subsection{Model Compression to further reduce execution time}

% Model Compression comprises several techniques to reduce the size and speed-up the forward time for a neural network. Among them, we consider pruning techniques, that allows to remove (\textit{i.e.,} set to $0$) a portion of the weights from the network. They are dived in:
% \begin{itemize}
% 	\item Element-wise pruning techinques: set to $0$ individual weights, generating sparse weight tensors

% 	\item Structured pruning techniques: prune entire groups of weights, \textit{i.e.,} columns, filters, layers. Resulting network's weights still belongs to the dense domain. 
% \end{itemize}
% Structured pruning for MLP is applied as column-wise pruning; given a criterion to estimate each column importance, \textit{e.g.,} $L_1$-norm, we remove a percentage of less important columns and then re-train the network. We experimentally verified that, established  a network architecture $\{l_1^p, l_2^p, l_3^p, l_4^p\}$, it makes no difference whether is trained from scratch or obtained from a pruning-finetuning procedure on a pre-trained model.   Then, we focused on element-wise pruning.  
% We applied two different kind of element-wise pruning:
% \begin{itemize}
% 	\item Level pruning: let $p_i$ the percentage of weights to remove from layer $i$, we save the highest magnitude $p_i$ weights and set to Execution time survived to the pruning phase are re-trained to recover the accuracy loss. 

% 	\item Han Pruning: insipired by the original method by Han, implemented in the version of the Distiller Framework. For each layer, we compute the standard deviation $\sigma_i$ and set a sensitivity parameter $s_i$. For each weight $w_{l_i}$ we use $\lambda_i = s_i * \sigma_i$ as theshold and set to zero all those weights whose absoulute values is below the threshold. The surviving weights and the re-trained. The procedure can be iteratively repetead to gradually increase the sparsity of the weights. In the original versino by Han, the value of $s_i$ was increased at each iteration, while the Distiller version keep it fixed, relying on the fact the as the tensor is pruned, more elements are pulled towards the center of the distribution and then pruned. 
% 	%TODO riscrivere, copiato da distiller. 
% \end{itemize}
% Weight Pruning has shown to be an effective compression technique, capable to reduce the number of values of a Neural Network of an order of magnitude without affecting its performance. %TODO Citazioni
% In Figure~\ref{fig:sparse_ndcg} we report the results of appling Level Pruning on the Medium Network ($\{1000,500,500,100\}$). When sparsity is about or below 90\%, the sparse models perform as or even outperforms the dense model, showing that pruning can be used as a regularization method. Until 96\% of sparsity, the NDCG@10 of the sparse model is still comparable with the one of the dense model. For higher levels of sparsity, we observe a noticeable degradation. 
% In Figure~\ref{fig:sparse_exec_time}, we show how to execution time decreases as the sparsity raise. Sparse multiplication is implemented with MKL. Compared with the dense model, sparse forward is very fast, but still cannot reach the performance of QuickScorer. 
% Table~\ref{table:sparse_vs_dense_network} provides an overview on the tradeoff between speed an accuracy for both dense and sparse models. With the same parameter budget, sparse models outscore dense ones in terms of effectiveness (reported as NDCG@10). Actually, with a sparse model of 37K parameters we sill suprass  the accuracy of a 4 times bigger dense models.%TODO come detto in agp pruning. 
% On the other hand, dense forward time is always faster than forward on sparse models. Even with a 10x smaller sparse model, when can barely match the dense model performances.  


% \begin{figure}
% 	\Description[]{}
% 	\centering
% 	\includegraphics[width=\columnwidth]{imgs/1000x500x500x100_sparse_ndcg.png}
% 	\caption{Performance in terms of NDCG@10 of a 1000x500x500x100 MLP at various level of sparsity }
% 	\label{fig:sparse_ndcg}
% \end{figure}


% \begin{figure}
% 	\Description[]{}
% 	\centering
% 	\includegraphics[width=\columnwidth]{imgs/1000x500x500x100_sparse_exec.png}
% 	\caption{Execution time of a 1000x500x500x100 MLP at various level of sparsity }
% 	\label{fig:sparse_exec_time}
% \end{figure}

% \begin{table}
% \adjustbox{max width = \columnwidth}{
% 	\begin{tabular}{rrrrr}
% 		\toprule
% 		Model & Sparsity & Parameters &NDCG@10 & Scoring Time ($\mu S$)\\
		
% 		\midrule
% 		200x100x100x50& 0.0 & 63K & 0.5187& 1.36 \\
% 		300x150x150x30 & 0.0& 112K & 0.5207 & 2.29 \\
% 		400x200x200x100 & 0.0& 194K & 0.5220 & 3.85 \\
% 		 \midrule
% 		 1000x500x500x100 & 90.0 & 93K & 0.5253 & 9.44\\
% 		 1000x500x500x100 & 96.0 & 37K & 0.5230 & 4.98\\
% 		 1000x500x500x100 & 98.0 & 18K & 0.5158 & 3.57\\
% 		\bottomrule
% 	\end{tabular}
% 	}
% 	\caption{Comparison between dense and sparse architectures on \msn.  }
% 	\label{table:sparse_vs_dense_network}
	
% \end{table}

% We than perform Han version of weight level pruning. Using $ \sigma=1$ for each layer, we obtain a network with NDCG@10: 0.5230 and sparsity 94.7\%. Besides the model itself, that results coherent with the overview provided level pruning reported in Figure~\ref{fig:sparse_ndcg}, it is intresting to analyze how the sparsity is distributed among the layer. Since we fix a threshold, and not a sparsity level as in the previous method, the percentage of zero weights per layer is usually variable. Even if we set the $\sigma $ for each layer, we obeserve that layers reaches different levels of sparsity. Especially we note that the first layer is the more \textit{pruning prone}, namely the one that reaches the higher sparsity level in this dynamic context. The explanation resides in the distribution of this layer weights. In fact, these layer show a quantity of quit large absolute values weights, phenomenon which is not witnessed in any other layers. These parameters assume that large absoulute values to dial with large absoulte value features which there are in \msn. This suggest that the layers have different sensibility to pruning. 

% %TODO va pensato a come dimostrare questa cosa
% %TODO va detto qualcosa in più sulle features
% \begin{table}
% \adjustbox{max width = \columnwidth}{
% 	\begin{tabular}{rrrrr}
% 		\toprule
% 		Layer & Total Parameters &Non zero parameters & Sparsity (\%)\\
% 		\midrule
% 		fc1 & 136000 & 2346 & 98.3 \\
% 		fc2 & 500000 & 28924 & 94.2 \\
% 		fc3 & 250000 & 13949 &94.4\\
% 		fc4 & 50000 & 4323 & 91.3\\
% 		fc5 & 100 & 100 & 0.0\\
% 		\bottomrule
% 	\end{tabular}
% 	}
% 	\caption{Comparison between dense and sparse architectures on \msn.  }
% 	\label{table:sparse_vs_dense_network}
	
% \end{table}


% \subsection{Combining Smaller Architecture with pruning}

% As shown in Table~\ref{table:sparse_vs_dense_network}, small dense networks and large dense ones stand on different sides of the trade-off ideal line. Dense networks afford fast inference, sparse network high effectiveness. Dense networks suffer of accuracy loss, sparse network of slow forward time. It seems natural to try to combine them together. 
% %The first apporach was to prune a network with a consistent effectiveness, such as the 400x200x200x100 network reported in Table~\ref{table:sparse_vs_dense_network}. 
% %TODO qui mettere i risultati del pruning sulla rete 400
% In Figure~\ref{fig:2115_tradeoff} we report the tradeoff between NDCG@10 and sparsity when pruning a 200x100x100x50 MLP and for Regression Trees with comparable effectiveness and execution time. The graph can be read along vertical or horizontal lines. Drawing a vertical line is equivalent to fix the NDCG@10 value. We then look for lower dots, that represent the methods that allow for faster inference. For example, a Regression Forest with 150 trees can be considered as performing as a 80\% sparse 200x100x100x50 model. Being the orange dot (Regression Forest) below the blue one (NN), it means that the scoring time through QuickScorer and the Random Forest is faster. To read the graph horizontally, we look for rightmost dots on horizontal line, that represent top performing models with a given budget in terms of scoring time. 



% \begin{figure}
% 	\centering
% 	\includegraphics[width=\columnwidth]{imgs/200x100x100x50_sparse_ndcg_and_exec.png}
% 	\caption{Tradeoff between NDCG@10 and sparisty for a 200x100x100x50 MLP and Regression Trees }
% 	\label{fig:2115_tradeoff}
% \end{figure}





% \subsection{Efficency-Oriented Pruning}

%TODO togliere totale dalla fig e scrivere percentuale sopra barre




% !TEX root = paper.tex
% !TeX spellcheck = en_US

\section{Modeling Matrix Multiplication}
\label{sec:ModelMatMult}
In this section, we detail the optimization of matrix multiplication on modern CPUs. We start with the implementation of dense-dense matrix multiplication (DMM) and then we move to the sparse-dense (SDMM) matrix case. Matrix multiplication has a prominent role in a wide spectrum of scientific applications (linear algebra, physics, economics, engineering), and it also represents the structural operation in neural network forward and backward pass. We believe that, when dealing with the efficiency-effectiveness trade-off, a comprehensive analysis of the underlying multiplication mechanisms is essential. We develop time predictors for matrix multiplication both in the dense and in the sparse domain, and we then jointly apply them to develop an analytical model that estimates the scoring time of a neural network given the matrix shapes and the sparsity percentage of each layer of the Feed Forward Network (FFN). Our predictors are analytic, \textit{i.e.}, not learned, and they are based on 1) the knowledge gained from the implementation of DMM and SDMM on modern CPU architectures, 2) empirical measurements showing the performance of CPU on these operations under different conditions. 
 We observe that, by exploiting the predictors we are proposing, we are allowed to train only the architectures that match the desired efficiency constraints. In a latency-bound application, the efficiency constraints are specified in the requirements. In an effectiveness-oriented context, they can be inferred by observing the execution time of the competitor, \textit{i.e.}, ensembles of tree-based models. As a consequence, the use of our predictors allows to significantly reduce the search space of the optimal architecture. Furthermore, our predictors are task-agnostic, hence they can be applied in any Feed Forward Network (FFN) application field.

%It is worth noting that the time predictors that we develop can have implications that go beyond this ranking-oriented use case and possibly generalize to all the FFN application fields.
%\fnote{non ho capito ultima frase. spiega meglio. FM}
%Todo quello che consente di fare il modello matematico per la predizione dei tempi potrebbe essere una lista dentro un enumerate.. per adesso me ne  é venuto in mente uno solo

\subsection{Dense Matrix Multiplication}
\label{subsec:dmm}
In this section, we investigate how Dense Matrix Multiplication (DMM) is optimized on modern CPUs. DMM has countless applications, hence lots of effort has been spent to attain fast implementations. The current state-of-the-art algorithm for DMM is the the well-known Goto Algorithm~\cite{goto2008anatomy}, on which are based several open (GotoBLAS~\cite{goto2008anatomy}, OpenBLAS~\cite{xianyi2012openblas}, BLIS~\cite{huang2016blislab}) or commercial (Intel MKL~\cite{wang2014intel}) implementations.
%During the last 30 years, a major effort has been made in developing CPU-oriented optimizations, exploiting cache hierarchy, vector instructions and pre-fetching. The main precepts are contained into the third level of the Basic Linear Algebra Subprograms (BLAS) library; these precepts are based on the well-known Goto Algorithm~\cite{goto2008anatomy}, which is also the established state of the art for dense matrix multiplication. In fact,  
%The world famous Basic Linear Algebra Subprograms (BLAS)~\cite{lawson1979basic}, for example, has its own 3-level routines entirely dedicated to matrix-matrix operations. 
%several open (GotoBLAS~\cite{goto2008anatomy}, OpenBLAS~\cite{xianyi2012openblas}, BLIS~\cite{huang2016blislab}) or commercial (Intel MKL~\cite{wang2014intel}) implementations are now available, all developed according to the Goto algorithm for blocked matrix multiplication~\cite{goto2008anatomy}.

The multiplication of two $ n \times n$ dense matrices involves $\mathcal{O}(n^3)$ floating-point operations with $\mathcal{O}(n^2)$ data, as can be easily evicted from Equation~\ref{eq:mmdef}. In modern processors, the interaction with memory is more time-consuming than the  computation itself (\textit{memory bandwidth bottleneck}), but a wise memory management allows to amortize the data movement over a large number of computations.
The mathematical definition of matrix multiplication is the following: given  $A \in \mathbb{R}^{m \times k}$, $B \in \mathbb{R}^{k \times n}$, the matrix multiplication binary operator computes $C = A*B$  with $C \in \mathbb{R}^{m \times n}$, where every element of $C$ is given by

\begin{equation} \label{eq:mmdef} 
C_{i,j} = \sum_{p=1}^{k} A_{i, p}  B_{p,j} \qquad  i=1, \dots, m \quad j=1, \dots, n 
\end{equation}
The Goto Algorithm consists of iteratively decomposing the overall DMM into a series of smaller matrix operations in a cache-aware fashion, until matrices fit the CPU registers. Then matrices are multiplied by means of a highly engineered \textit{micro-kernel}. We now provide a breakdown of the Goto Algorithm as implemented in the BLIS library~\cite{lawson1979basic,van2015blis}, which assumes the CPU to be equipped with 3 levels of cache and vectorized instructions. The first three steps of the blocked matrix multiplication algorithm are depicted in Figure~\ref{fig:gotofirst}.

\begin{figure}[htb]
	\centering
	\includegraphics[width=\columnwidth]{imgs/Goto_first.pdf}
	\caption{First three steps of the Goto algorithm for Dense Matrix multiplication.}
		\label{fig:gotofirst}
\end{figure}

The blocked matrix multiplication algorithm begins by partitioning along the columns of $C$ and $B$ into blocks of size $n_c$, obtaining  sub-matrices of $C$ of shape $m \times n_c$ and sub-matrices of $B$ of shape $k \times n_c$. Each $C$ sub-matrix is obtained by multiplying the complete $A$ matrix with the corresponding sub-matrix of $B$. Then, the procedure partitions the columns of $A$ and the rows of $B$ into blocks of size $k_c$, to obtain $A_p$, \textit{i.e.}, vertical panels of size $m \times k_c$, and $B_i$, \textit{i.e.}, horizontal panels of size $k_c \times n$. The $B_i$ panels are packed into the L3 cache reordering data according to a specific pattern which allows to access data contiguously even after the subsequent partitions.  We adopt the notation $\tilde{X}$ to indicate that the sub-matrix $X$ respects this pattern. Observe that, after the blocking on the $k$ axis, the original multiplication is boiled down into a series of rank-k updates so that $C = C + A_p B_p$. A further partition is performed along rows of $A$, with size $m_c$, generating $C_i$ and $A_i$. $A_i$ is, as was $B_i$ previously, packed into $\tilde{A_i}$ in the L2 cache.

% \begin{figure}[htb]
% 	\centering
% 	\includegraphics[width=\columnwidth ]{imgs/Goto_second.pdf}
% 		\caption{Macro-Kernel in the Goto algorithm for Dense Matrix Multiplication (DMM).}
% 		\label{fig:gotosecond}
% \end{figure}

%\fnote{sistema caption della figura sopra. metti sempre i punti alla fine delle caption. controlla ovunque. FM}

\noindent \textbf{Macro-Kernel}. The macro-kernel, or inner kernel as in the original algorithm by Goto \textit{et al.}~\cite{goto2008anatomy}, is responsible for orchestrating the memory movement between the RAM memory and the caches. Let us consider the operation $C_i \leftarrow C_i +  \tilde{A}_i*  \tilde{B}_p $, with $C_i$ of size $m_c \times n$, $\tilde{A}_i$ of size $m_c \times k_c$ and $\tilde{B}_p$ of size $k_c \times n$. The macro kernel decomposes this operation into a series of block-panel multiplications, as shown in Figure~\ref{fig:gotosecond}. As aforementioned, both $\tilde{A}_i$ and $ \tilde{B}_p$ are packed with a special pattern, indicated by the arrows in Figure~\ref{fig:gotosecond}. In particular, $\tilde{A}_i$ is organized into sub-matrices $\tilde{A}_j$ of size $m_r \times k_c$, with elements stored in column-major order, while $ \tilde{B}_p$ is organized in panels of size $k_c \times n_r$, stored in row-major order, named $\tilde{B}_j$. This data access pattern reflects the order in which the micro-kernel accesses data. 

Goto \textit{et al.}~\cite{goto2008anatomy} observed the advantages of packing $\tilde{A}_i$ into the L2 cache. The ratio between FLOPs and memory operations, regardless if the original data rely in L3 cache or in main memory, can be modeled as
 $$ \frac{2 m_{c} k_{c}}{\left(2 m_{c}+k_{c}\right)} $$
if $k_c << n$.
Hence, the higher is the $m_c k_c$ product, the smaller is the overhead of memory transfer on the overall computation. 
Knowing that L2 cache is larger than L1, we can afford larger $m_c$ and $k_c$ values \footnote{In the original work, Goto et al.~\cite{goto2008anatomy} point out that $C_i \leftarrow C_i +  \tilde{A}_i  \tilde{B}_p $ should be computed at the peak rate of CPU. This condition is true if all three matrices reside in L1 cache, but it can be considered true even if $\tilde{A}_i$ is in L2.}.


\begin{figure}
	\centering
	\includegraphics[width=0.95\columnwidth ]{imgs/Goto_second.pdf}
		\caption{Macro-Kernel in the Goto algorithm for Dense Matrix Multiplication (DMM).}
		\label{fig:gotosecond}
\end{figure}



\noindent \textbf{Micro-Kernel}. The micro-kernel is the core operation of blocked matrix multiplication and the speed of the whole routine largely depends on the speed of this kernel. For this reason, in high-performance libraries, the micro-kernel is often written in assembly language, to exploit vectorized instructions and hand-tuned data pre-fetching~\cite{van2015blis}.
The micro kernel computes $c_{r,j} = c_{r,j} + \tilde{A}_j \tilde{B}_j$, where $\tilde{A}_j$ is an horizontal micro-panel of $\tilde{A}_i$ and $\tilde{B}_j$ is a vertical micro-panel of $\tilde{B}_p$, residing, respectively, in L2 and L1 cache, as reported in Figure~\ref{fig:gotothird}. The operation is performed as $k_c$ rank-1 updates, by computing the outer product between a column of  $\tilde{A}_j$ and a row of $\tilde{B}_j$ and by accumulating the results into the $m_r \times n_r$ $c_{r,j}$ submatrix. In this way, $c_{r,j}$ can be kept in CPU registers until the loop over $k_c$ is , allowing to move data from the registers to the memory just once. This means that $2m_c n_c k_r$ FLOPs can be performed with just $m_r n_r$ memory operations. Furthermore, this data reading pattern benefits from the data packing performed in the previous loops. In fact, columns and rows of $\tilde{A}_j$ and $\tilde{B}_j$ respectively will be accessed contiguously, which is generally known to be faster than accessing non-in-stride memory locations~\cite{low2016analytical}. In conclusion, pre-fetching instructions that load successive entries of $\tilde{A}_j$ and $\tilde{B}_j$ are interleaved with instructions performing the rank-1 update. This allows to mask the latency of the caches with the computation time of the CPU.


\noindent \textbf{Choosing the  kernel parameters}. Blocked matrix multiplication requires to determine a number of parameters $n_c, m_c, k_c, n_r, m_r$, controlling how the matrices are gradually decomposed. These parameters can differ from one processor to another, since they are influenced by hardware features such as the cache size or the number of SIMD registers. Choosing the optimal parameters for a given CPU architecture is a research problem, tackled for example by \textit{Low et al. }~\cite{low2016analytical}, which goes beyond the scope of this paper. Here, we want to list some general rules governing good choices for parameters. %We do this to provide a more detailed overview of the  Goto matrix multiplication algorithm. 
The micro-kernel is characterized by $m_r, n_r, k_c$. The values of $m_r$ and $n_r$ should be large enough so that the computation masks the latency of the caches. However, it should also allow to leave space in the registers for the next entries of $\tilde{A}_j$ and $\tilde{B}_j$. $k_c$ should be as large as possible, but must take into account the following constraints: 1) $k_c n_r$ entries from $\tilde{B}_j$ should fit the L1 cache 2) $m_c k_c$ entries from $\tilde{A}_i$ reside in the L2 cache. 
Moreover, cache replacement policies should also be taken into account. These policies control which data are kept and which are discarded from the levels of cache and may impact on the optimal macro-kernel values. A general solution is provided by Goto \textit{et al.}, who suggest choosing $k_c$ so that $\tilde{B}_j$ takes less than the half of the L1 cache~\cite{goto2008anatomy}.

\begin{figure}[t]
	\centering
	\includegraphics[width=0.8\columnwidth ]{imgs/Goto_third.pdf}
	\caption{Micro-Kernel in the Goto algorithm for Dense Matrix Multiplication (DMM).}
	\label{fig:gotothird}
\end{figure}

% The correct choice of these parameters
%strictly depends on the underlying hardware features.
%In the original work, Goto \textit{et al.} suggest to choose $k_c$ so that $\tilde{B}_j$ takes less of the half of the L1 cache~\cite{goto2008anatomy}. Cache replacement policies should also be taken into account. These policies control which data are kept and which are discarded from the levels of cache and may impact on the optimal macro-kernel values. \cosimo{This involves deep architectural fe}
% This is a non-trivial architecture-dependent problem which goes beyond the scope of this article.
%%\fnote{frase sopra. proverei a scriverla meglio spiegando perche' non la addressiamo. sembra che sia difficile e quindi ci importa una sega :)}
%
%Cache replacement policies should also be taken into account, especially in the $\tilde{B}_j$ case~\cite{goto2008anatomy,low2016analytical}, giving birth to a non-trivial architecture-dependent problem which goes beyond the scope of this article.
%\fnote{frase sopra non capisco. piu' chiara...}
Concerning the macro-kernel, we already discussed that the $m_c k_c$ product should be as large as possible. One of the key insights of the Goto algorithm is to consider the role of the Translation Look-Aside Buffer (TLB) in choosing the macro-kernel parameters. To hide the limits of random-access memories capacity (RAM), modern computing architectures use virtual memory. With this mechanism, the memory (RAM and hard disk) is partitioned into pages and a table, called \textit{page table}, keeps track whether a page is in memory or on disk. Scanning the page table entails additional overhead to check whether the requested page is on memory or disk.
Hence, the TLB, which is smaller than the overall \textit{page table}, keeps track of the most recently used pages: in case of a TLB \textit{hit}, the translation is fast. On the other side, in case of a TLB \emph{miss}, the complete \textit{page table} is checked and the new entry is moved to the TLB. Actually, the TLB has the same role as the cache and the \textit{hit/miss} dichotomy involves the same consequences. Thus, besides ensuring that $m_c k_c$ entries from $\tilde{A}_i$ fit the L2 cache, it is crucial that $\tilde{A}_i$, $n_r$ columns from $C_k$, and $n_r$ columns of $\tilde{B}_j$ are simultaneously addressable by the TLB, to avoid TLB misses during the block-panel multiplications of the macro-kernel. The only limit to the $n_c$ parameter is that $k_c n_c$ have to fit the L3 cache.

\subsection{Dense Neural Forward Pass Time Predictor}
\label{subsec:densetimepred}
In the previous section, we detail how Dense Matrix Multiplication is implemented on modern CPU architectures. We now show how the insights deriving from a deep understanding of matrix multiplication can be used to develop a time predictor for a Feed Forward Network (FFN) forward pass. We empirically demonstrate that even the highly engineered Goto algorithm  suffers when dealing with edge matrix dimensions. Hence, we leverage this intuition to build a hybrid analytical-empirical model for predicting dense matrix multiplication.
A FFN is composed of a  stack of \textit{fully connected} layers, where each neuron of layer $i$ is connected to all neurons of layer $i+1$. Each layer is composed of a weight matrix $W_i$, a bias vector $b_i$ and a non-linear \textit{activation function} $\sigma_i( \cdot )$. Let $x_i$ be the input to the $i$-th layer, the forward pass of layer $i$ is described by:
\begin{equation}
	\label{eq:mlpforward}
	x_{i+1} = \sigma_i(W^t_i x_i + b_i)
\end{equation}
where $x_{i+1}$ represents the output of the $i$-th layer. 
Hence, forwarding through a FFN layer consists of: 1) multiplying the input with the weight matrix, 2) summing the bias, 3) applying a non-linear activation function, usually ReLU or its variants. The overall forward pass on a FFN of $d$ layers has a cost, in terms of execution time, given by:
\begin{align}
 \label{eq:overallcost}
 	T = t_m \cdot ( f \cdot l_1 + \sum_{i=2}^{d} l_i   l_{i-1} + l_{d})
 	 + t_a \cdot \sum_{i=1}^{d} l_i + t_r \cdot \sum_{i=1}^{d} l_i \nonumber \\
 	   \simeq t_m \cdot ( f \cdot l_1 + \sum_{i=2}^{d} l_i \cdot  l_{i-1} + l_d) 
 \end{align}
where $t_m $ is the normalized time per multiplication, $t_a$ is the time for addition, $t_r$ is the time to perform the ReLU operation on a single neuron. As reported in Equation~\ref{eq:overallcost}, the time to perform matrix multiplication dominates the overall execution time, both in terms of number of operations and in terms of the complexity of the operation itself. We observe that $t_m$ can be inferred as:

\begin{equation}
	\label{eq:tm}
	t_m = \frac{1}{\text{GFLOPS}}
\end{equation}
The theoretical peak of GLOPs can be derived form the hardware specifications of the processor\footnote{https://software.intel.com/en-us/articles/a-simple-example-to-measure-the-performance-of-an-intel-mkl-function}. However, real performance can be significantly different from the theoretical ones, especially when facing limit cases, such as narrow or wide matrices. To include these cases into our evaluation, we develop a prediction model to measure the performance of a specific neural networks architecture.

Among the different instantiations of the BLAS library, we choose \textit{oneDNN}\footnote{\url{https://github.com/oneapi-src/oneDNN}},
a C++ high-performance framework for deep learning primitives developed by Intel, used as backbone inference system by Pytorch~\cite{NEURIPS2019_9015}, Tensorflow~\cite{abadi2016tensorflow}. With respect to the Math Kernel Library (MKL)~\cite{wang2014intel} by Intel, oneDNN guarantees the same performances while being open source.
The oneDNN library adopts the following parameters for CPUs with AVX2 ISA enabled: $m_c = 10000$, $n_c =384 $, $k_c = 192$, while for the micro-kernel we have  $m_r = 24, n_r = 4$.
%---------------- BEGIN MKL DETAILS----------------------
% Before moving on the empirical GFLOPs estimation, we analyze some of the features of the oneDNN library. 
% OneDNN adopts the following  parameters for CPUs with AVX2 ISA enabled: $m_c = 10000$, $n_c =384 $, $k_c = 192$, while for the micro-kernel we have  $m_r = 24, n_r = 4$.
The macro-kernel parameters $m_c, n_c, k_c$  are selected to deal with very large matrices; for the sequential case, the library contains a mechanism to tailor smaller shapes. Let us call $\overline{m}_c$, $\overline{n}_c$, $\overline{k}_c$ the parameters that the macro and micro kernels actually use. 	
$\overline{m}_c$ is chosen as:
$$\overline{m}_c = \texttt{rnd\_up}(min(max(m, m_r), m_c
), m_r)$$
%\fnote{sopra ho messo textt la chiamata a procedura. ti piace? se si, fixa everywhere. FM}
where \texttt{rnd\_up}($a,b$) is a function which approximates $a$ as $a = n^{*} b$, with $n^{*} = min\{n \ |\  nb \geq a\}$, \textit{i.e.}, to the subsequent multiple of $b$. This way, it is ensured that $\overline{m}_c$ is larger than the micro-kernel parameter $m_r$ and that the default $m_c$ is not involved if $m \leq m_c$. By means of the \texttt{rnd\_up} function, we ensure that $\overline{m}_c \bmod m_r  = 0$ to avoid undersized horizontal $\tilde{A}_j$ panels in the micro-kernel.  
%When the resulting $m_c > m$,  we pad with zeros the difference $d = m_r - m$. This means that $d$ values are read and written but they are not actually used, causing a performance drop.
Similar refinements are adopted to chose $\overline{n}_c$ and $\overline{k}_c$. Moreover, oneDNN triggers when the cost of packing the matrices into contiguous arrays surpasses the cost of multiplication. In this case, besides changing the macro-kernel parameters, it also performs a different routine that skips copying the matrices in cache-aware buffers.

%In choosing $\overline{k}_c$, oneDNN introduces two parameters  $k_{ct} = 256$, which stands for $k_c$ traditional, and $k_{cs}$, for $k_c$ small. $\overline{k}_c$ is computed as function of the shared dimension $k$ as:

% \begin{algorithm}
% 	%	\KwData{this text}
% 	%	\KwResult{how to write algorithm etith \LaTeX2e }
% 	$\overline{k}_c$
% 	\uIf{$k < k_{ct}$ }{
% 		$\overline{k}_c = max(128 , \overline{k}_c)$\;
% 	}\uElseIf{$k < 2 k_c$ }{
% 		$\overline{k}_c = (k+1)/2)$\;
% 	}\Else{
% 		$\overline{k}_c = k_c$\;
% 		}
% \end{algorithm}
% The actual value of the blocking parameter $n_c$, namely $\overline{m}_c$ is chosen depending on both $n_c$ and $k_c$. One more parameter is introduced, $n_{csk}$ which is used when $\overline{k}_c$ is small.

% \begin{algorithm}
% 	$\overline{m}_c = (k<k_{cs}) ? n_{csk} : n_c $\;
% 	$\overline{m}_c = rnd\_up(min(max(n, n_r), n_c), n_r)$	\;
% \end{algorithm}

% Furthermore, oneDNN triggers when the cost of packing the matrices into contiguous arrays surpasses the cost of multiplication. In this case, besides changing the macro-kernel parameters, it also perform a different routine that skips copying the matrices in cache-aware buffers.
%---------------- END MKL DETAILS----------------------

In Section~\ref{subsec:dmm} we have described the optimization techniques beyond Dense Matrix Multiplication on modern CPUs. We also detail the tailored refinements implemented by the oneDNN framework to deal with matrices where at least one dimension is small. We now show the performance of the oneDNN framework with differently shaped matrices, aiming at identifying a reliable $t_m$ for Equation~\ref{eq:tm}. In these experiments, we multiply random matrices with different shapes to empirically analyze how the oneDNN library adapts to different matrix dimensions. We propose two different cases: 1) $m=k$, 2) $mk=c$, with $c$ as a constant integer.
We run our tests on a i9-9900K processor, with AVX2 instructions, 3.6 GHz, max frequency 5.0 GHz. Each core has a 32 KiB L1 cache for data, 32 KiB L2 cache for instructions, both 8-way set associative, 256 KiB L2 cache 4-way set associative, and 2 MiB L3 cache, 16-way set associative. We report the results for single-thread execution. 
In our first experiment, we vary $m$ and $k$ in a fixed range and report the corresponding GFLOPs, with different values of $n$. Observe that $A$, of shape $m \times k$,  represents the weight matrix $W$, $B$, of shape $k \times n$ represents the input matrix $x$, obtained by stacking $n$ input vector. We vary $m$, $k$ and $n$ to model real use-case scenarios: $m$ and $k$ correspond to the sizes of Feed Forward Network layers, while $n$, the \emph{batch size}, is the number of documents we give in input to the neural network at a time. Results are reported in Figure~\ref{fig:onednnl_no_r}, which shows that GFLOPS grow as the size of the matrices even with the aforementioned techniques tailored to edge cases. In Figure~\ref{fig:onednnl_rev}, we show the results of the reverse experiment: instead of gradually increasing both $m$ and $k$, we keep the size of $A$ constant (the $mk$ product is constant). The figure shows that small values of $m$ with large values of $k$ still afford  high-performance (left side of the graph). On the other hand, small values of $k$ paired with larger values of $m$ cause serious performance degradation. The variation of the GFLOPS with the matrix shapes suggests that a unique and size-independent $t_m$ is not reliable. As aforementioned, this evidence some limitations of the Goto algorithm when dealing with edge combinations of input dimensions.
A correct analysis expresses $t_m$ as a function of the $m,n,k$ parameters, or in the case of the Feed Forward Network , as a function of the dimensions of the layers, \textit{i.e.}, $t_m = t_m( l_1, \dots, l_{d})$.
Given the variability of the performance with input shapes, we shall empirically measure them.
 We can use Figure~\ref{fig:onednnl_no_r} and ~\ref{fig:onednnl_rev} to derive a lookup table that maps the matrix shapes to the corresponding GFLOPs. The previous graphs are synthesized in Figure~\ref{fig:heatmap}, which shows an heatmap of the GLFOPs with different values of $m$ and $k$ and $n = 1000$.

 We observe three performance zones, defined by horizontal stripes induced by partitioning the $k$ axis.
\begin{itemize}
	\item $K \geq 512$ : high-performance (130 GFLOPs)
	\item $128 \leq K \leq 512$: Medium performance (110 GFLOPS)
	\item $K \leq 128$: Low performance (90 GFLOPs)
\end{itemize}



% \begin{figure}
% \begin{minipage}[b]{\columnwidth}
% \includegraphics[width=\columnwidth]{imgs/DNNL_different_N.png}
% %\centering 
% %\caption*{Static}
% \end{minipage}%
% \\ 
% \begin{minipage}[b]{0.5\columnwidth}
% \includegraphics[width=\columnwidth]{imgs/DNNL_different_N_reverse.png}
% %\centering 
% %\caption*{Dynamic}
% %\subcaption{Another subfigure}
% \end{minipage}%
% \caption{Matrix Multiplication with oneDNNL}
% \label{fig:onednnl}
% \end{figure}

\begin{figure}	
	\centering
	\includegraphics[width=\columnwidth]{imgs/DNNL_different_N.png}
	\caption{Matrix Multiplication with oneDNN, as $m$ and $k$ grow.  }
	\label{fig:onednnl_no_r}
\end{figure}

\begin{figure}
	\centering
	\includegraphics[width=0.8\columnwidth]{imgs/DNNL_different_N_reverse.png}
	\caption{Matrix Multiplication with oneDNN, with the product $mk$ constant. }
	
	\label{fig:onednnl_rev}
\end{figure}

\begin{figure}
	\centering
	\includegraphics[width=\columnwidth]{imgs/heatmap_gflops_batch1000.png}
	\caption{Matrix Multiplication HeatMap with n = 1000.}
	\label{fig:heatmap}
\end{figure}


For a network of size $\{1000, 500, 500, 100 \}$, we can assume to be always in the high-performance region, except for the last layer. Observe that the last layer has a negligible impact on the overall forward time and we can ignore it. Table~\ref{table:est_vs_real_exec_t} illustrates how the prediction model can substitute the experimental procedure of training and testing a model, turning out to be essential to reduce the architecture search space.

\begin{table}[htb]
	\centering

	%\adjustbox{max width=\columnwidth}{
	\begin{tabular}{lrr}
	%\resizebox{\columnwidth}{!}{
		\toprule
		\multirow{2}{*}{Model} &    \multicolumn{2}{c}{Scoring Time ($\mu s$/doc)} \\
		\cmidrule{2-3}
		 & Real & Predicted \\		%    \thead{Real Scoring & \\ Time per doc ($\mu s $)}  & \thead{Real Scoring & \\ Time per doc ($\mu s $)} \\
		\midrule

		 1000$\times$500$\times$500$\times$100 & 14.4& 14.5 \\
		 200$\times$100$\times$100$\times$50 & 	1.3& 1.3 \\
		 300$\times$150$\times$150$\times$30 & 2.0 & 2.2 \\
		 500$\times$100 & 2.1 & 2.2\\
 		\bottomrule
	\end{tabular}
	 %}
	\caption{Performance of our dense prediction model. Real execution times measured with batch size = 1000.}
	\label{table:est_vs_real_exec_t}
\end{table}

\subsection{Sparse-Dense Matrix Multiplication} 
\label{subsec:sdmm}
In this section, we study Sparse-Dense Matrix Multiplication (SDMM), a special case of matrix multiplication where the first matrix is \textit{sparse}: we recall that \textit{sparsity} is defined as the percentage of zero entries in a data structure, in this case, a matrix. First, we describe a common format to store sparse matrix, \textit{Compressed Sparse Row} (CR). Then, we detail how SDMM is implemented on modern CPU processors. 

\smallskip
\noindent \textbf{CSR Format}.
A sparse matrix is completely identified by its non-zero values and their positions since all the others entries are zeros. This motivates the use of a different representation for sparse matrices w.r.t. to dense ones. The different representation aims at saving storage space and improving the performance of matrix multiplication. For this purpose, several formats have been developed: the most common are Compressed Sparse Row (CSR), Compressed Sparse Column (CSC), Coordinate List (COO). Among them, we analyze CSR, since it is usually supported by off-the-shelf libraries, both for storing and for matrix operators, such as multiplication and it naturally fits to Sparse-Dense Matrix Multiplication, as we will detail.
%Let us assume we have a matrix $A \in \mathcal{R}^{m \times k}$ with $nnz$ non-zero values.
%$$ \text{sparsity} = \frac{nnz}{mk}$$

Let us consider a matrix $M \in \mathbb{R}^{m \times n}$ with $nnz$ non-zero elements. 
An example of the  CSR representation is reported in Figure~\ref{fig:sparsecsr}. It consists of three vectors: \textit{values} $\in \mathbb{R}^{nnz}$, \textit{columnIndex} $ \in \mathbb{R}^{nnz}$, \textit{rows} $\in \mathbb{R}^{m+1}$. 
The \textit{values} array stores the non-zero entries, and \textit{columnIndex} stores their column index in the original matrix, meaning that \textit{columnIndex}$[i]$ stores the columns index of \textit{values}$[i]$. The \textit{rows} array is built so that $rows[i+1] - rows[i] $ is the number of non-zero entries for row $i$. 

\begin{figure}
\centering
	\includegraphics[width=0.8\columnwidth]{imgs/CSR_sparse.pdf}
	\caption{CSR Format for Sparse Matrices.}
	\label{fig:sparsecsr}
\end{figure}


\smallskip
\noindent \textbf{Sparse Dense Matrix Multiplication}.
Sparse Dense Matrix Multiplication or sparse Multi-vector multiplication (SDMM) has a large range of applications: fluid dynamics, graph analysis~\cite{tiskin2001all}, non-negative matrix factorization~\cite{kim2011fast}, economic modeling, seismic simulations~\cite{breuer2019petaflop}, and machine learning~\cite{NIPS2010_4099}. Pruning a neural network pre-trained model naturally induces the usage of SDMM in the forward pass of a Multi-Layer Perceptron, since it converts dense weights into sparse ones. Let us consider Equation~\ref{eq:mlpforward}: in the most general case $W$ represents the dense weight matrix. After pruning, $W$ is transformed into a sparse matrix $\dot{W}$, thus converting $\dot{W}^T x$  into a Sparse Dense Matrix Multiplication.
%Historically, the scientific community has mainly focused in optimizing the Sparse Matrix dense vector multiplication (SpMV). Even if SpMM can be implemented as a loop of SpMV, this approach fails in exploiting data locality in the sparse matrix~\cite{zheng2016semi}. 

%Before describing cutting-edge implementations	 of SDMM, we depict the na\"ive algorithm induced by the CSR Format. %repparse matrix multiplication is slowed down by random memory accesses. Several approaches create their own matrix format to reduce this drawback, even exploiting domain-specific knowledge to guess information on the structure of the matrix. In our case, the structure of the matrix is known \textit{ a priori} but has no specific structure since it derives from the pruning of a neural network. 
Consider the operation $C = AB$, where $A \in \mathbb{R}^{m \times k}$ is a sparse matrix in the CSR representation with $nnz$ non-zero values, and $B \in \mathbb{R}^{k \times n}$, $C \in \mathbb{R}^{m \times n}$ are dense matrices. 
The mundane algorithm induced by A being in CSR Format is reported in Algorithm~\ref{alg:csr_mult}. This format is suitable for row-wise access, allowing to consider exclusively the non-zero entries of the left-side matrix. 
The total number of floating-point operations is reduced from $2mnk$ to $2 nnz N$ w.r.t. dense case, but the irregular access pattern induced by sparsity hinders the efficiency of the algorithm. To overcome this problem, a twofold strategy, as for the dense case, is applied: 1) proficient data access pattern, 2) optimization of the core operation (\textit{micro-kernel}). 

The most used library for sparse matrix multiplication is the Math Kernel Library (MKL)~\cite{wang2014intel}, which implements the sparse versions of third level BLAS routines. Since the library is closed and there are no details on how the multiplication is implemented\footnote{https://community.intel.com/t5/Intel-oneAPI-Math-Kernel-Library/Sparse-Dense-Matrix-Multiplication/m-p/1173953}, we refer to the implementation of the LIBXSMM~\cite{heinecke2016libxsmm}, which is open-source. Later on in this section, we show that LIBXSMM actually outperforms MKL in the spectrum of shapes involved by our neural networks. 

\begin{figure}
\centering
	\includegraphics[width=\columnwidth]{imgs/libxsmm_sparse_dense_mult.pdf}
	\caption{LIBXSMM Sparse-Dense Matrix Multiplication (SPMM).}
	\label{fig:libxsmmsparsedense}
\end{figure}

\begin{algorithm}[]
	\KwData{ CSR $A \in \mathbb{R}^{m \times k}$, $B \in \mathbb{R}^{k \times n}$}
	\KwResult{$C \in \mathbb{R}^{m \times n}$  }
	$C[i,k] = 0$\;
	\For{$i = 0$ \KwTo $M-1$}{
		\For{ $j = A.rows[i]$ \KwTo $A.rows[i+1]-1$}{
		  \For{$k = 0$ \KwTo N-1}{
		  	$idx   = A.cols[j]$\;
		  	$C[i,k] \leftarrow C[i,k] +  A.val[j] * I [idx, k]$\;
		  }
		}
	}

	\caption{Sparse-Dense Matrix Multiplication algorithm with CSR format.}
	\label{alg:csr_mult}
\end{algorithm}

%We observe the same problems reported with for dense matrix multiplication in terms of lack of data re-usage. If we observe the most inner cycle computing $C_{i,k}$, we observe that the whole $I$ matrix may be needed to compute the $i$ line of C. As for dense matrix multiplication, a block-wise approach could partially solve this issue. With the lesson learned from the dense case, we can state that we should load data by blocks thus promoting data re-usage. 
%Born in 2015, this library was conceived to cover use cases that other Intel libraries, \textit{e.g.,} MKL and One-DNN, left uncovered. 

\smallskip
\noindent \textbf{Sparse-Dense matrix multiplication with LIBXSMM}. LIBXSMM~\cite{heinecke2016libxsmm} is a high-performance library specifically tailored for Intel architectures, specialized in small dense matrix multiplication, sparse matrix multiplication, and deep learning primitives in general. It is based on ``Just in Time'' (JIT) code specialization, which intends to exploit the runtime information about its operands. The sparse-dense routine was originally developed to solve seismic equations~\cite{breuer2019petaflop}.
% such as matrix shapes or non-zero entires location for sparse matrices. The library includes both dense-sparse and sparse-dense matrix multiplication, and the latter was originally developed to solve seismic equations~\cite{breuer2019petaflop}. 

We now detail the sparse-dense matrix multiplication as implemented in the LIBXSMM library, with A in CSR format. 
The dense matrix $B$ is converted into a three dimensional tensor of shape $k \times N_b \times n_b$, as reported in Figure~\ref{fig:libxsmmsparsedense},
so that $N = N_b \times  n_b$. This means to factorize the $N$ dimension in two sub-dimension, in which one ($n_b$) is induced by the underlying hardware. The ideal value of $n_b$ in fact, coincides with the SIMD length of the processor, \textit{i.e.}, the number of different numbers that a SIMD vector can store. 
Using floating-point variables (32 bit) on a machine with AVX2 ISA (256 bit), the SIMD length is 8. This packing allows to  multiply each non-zero element of $A$  with $nb$ values of $B$ at time, using just one vectorized instruction.



The problem of irregular accesses is tackled by hardwiring the loading of the elements of $A$ and $B$, so that only relevant elements are loaded. The data access pattern provides for multiplying each non-zero element of $A$ ($a_{i,j}$) with the $j$-th rows of $B$ ($B_j$) and accumulate the results into the $i$-th row $C$ ($C_i$).
The computation is carried on one row of $A$ at time. Figure~\ref{fig:libxsmmsparsedensemicro} shows the sub-routine performed for each row $i$.  
Let us call the first non-zero element of the current row $x$, in position $(i,j)$; we assume to have at least one non-zero entry, otherwise the row is skipped. $C_i$ is loaded into $N_b$ SIMD registers, each containing $n_b$ values. $x$ is \emph{broadcasted} to a SIMD CPU register, \textit{i.e.}, $n_b$ consecutive copies of the $x$ vector are loaded into the register. We refer to this vector as $\overline{x}$. 
$B_j$ is loaded as well and $C_i$ is updated as $C_i \leftarrow C_i + \overline{x} B_j$; the update involves $N_b$ Fused Multiply Add (FMA) instructions $C_{i,k} \leftarrow C_{i,k} + \overline{x} B_{j,k}$. Then, the routine moves to the next non-zero elements in the $i$-th row of $A$. Once all the non-zero elements have been multiplied, $C_i$ is stored in memory and the algorithm moves on to the next row of $A$.  


\begin{figure}[t]
\centering
	\includegraphics[width=\columnwidth]{imgs/libxsmm_sparse_dense_mult_micro.pdf}
	\caption{Micro Kernel of LIBXSMM Sparse-Dense Matrix Multiplication (SPMM).}
	\label{fig:libxsmmsparsedensemicro}
\end{figure}

LIBXSMM is equipped with a mechanism that interrupts the code generation if the number of instructions is too elevated. This can happen if the number of non-zero elements in $A$ or the $N$ dimension are too large. Since the $N$ dimension corresponds to the batch size in the neural forward, we are free to  reduce it to overcome this limit. When necessary, we also split the $m \times k $ $A$ matrix along the $M$ dimension to generate a set of sub-matrices $A_S = \{A_1, \dots, A_s \ | \  A_i \text{ of size }  M/s \times k  \}$. Each $A_i$ will have fewer non-zero entries, preventing code generation failure. 
 %A==\left(\frac{\frac{\check{A}_{1}}{A_{1}}}{\frac{\vdots}{\check{X}_{M-1}}}\right)
The $C$ matrix is trivially obtained by multiplying each $A_i \in A_S$ with $B$ separately and by stacking the results along the $M$ (vertical) axis:
\[ C = 
  \begin{bmatrix}
    \begin{array}{c}
  A_1 B  \\
  \hline
  A_2 B\\
  \hline 
  \vdots\\
  \hline
  A_s B\\	   
    \end{array}
  \end{bmatrix}
	\]

\smallskip
\noindent \textbf{LIBXSMM vs MKL}. As aforementioned, MKL is known to provide the fastest routine for sparse-dense matrix multiplication. We now show that LIBXSMM outperforms MKL on small, very sparse, and asymmetric matrices, which is the typology of matrices we employ in our MLPs for document scoring. In Table~\ref{table:lib_vs_mkl_msn}, we report the execution time of $C = AB$, with $A$ sparse in the CSR format and B dense, both for MKL and LIBXSMM; on the $x$-axis is reported the shape  ($m \times k $) $A$ and its sparsity. $B$ has shape $k \times n$, where $n$ is the batch size, set to $64$. The matrices correspond to the first layer of real models trained on the \msn dataset~\cite{DBLP:journals/corr/QinL13}, which provides $136$ handcrafted features. The Table shows that LIBXSMM is always faster than MKL on these shapes, with a speedup factor often larger than $2$x. This consideration, together with the availability of the code, has led us to pick the LIBXSMM library as the reference implementation. 

\begin{table}[htb]
	\centering
	\begin{tabular}{llrr}
		\toprule
		\multirow{2}{*}{Shape} &   	\multirow{2}{*}{Sparsity}&  \multicolumn{2}{c}{SDMM Time ($\mu s$)}\\
		\cmidrule{3-4}
		 & & MKL & LIBXSMM \\
		\midrule
		400$\times$136 &  0.996 &         3.1 &          \textbf{1.2} \\
 		300$\times$136 &  0.985 &         2.5 &          \textbf{1.4} \\
 		200$\times$136 &  0.971 &         2.8 &          \textbf{1.6} \\
 		100$\times$136 &  0.989 &         1.0 &          \textbf{0.4} \\
 		50$\times$136  &  0.968 &         0.7 &          \textbf{0.2} \\
 		\bottomrule
	\end{tabular}
	\caption{ Comparison between MKL and LIBXSMM for Sparse Dense Matrix Multiplication (SDMM)	. Shapes and sparsities represent the first layer of FFNs trained on \msn. Batch size is set to $64$.}
	\label{table:lib_vs_mkl_msn}
\end{table}

% \begin{figure}
% \centering
% 	\includegraphics[width=\columnwidth]{imgs/mult_on_\msn.png}
% 	\caption{A comparison between MKL and LIBXSMM. Shapes and sparsities represent the first layer on MLPs trained on \msn. Batch size is set to $64$}
% 	\label{fig:lib_vs_mkl_msn}
% \end{figure}

\subsection{Sparse Time Predictor}
\label{subsec:sptimepred}
In this section, we illustrate the development of a Sparse-Dense matrix multiplication time predictor, specularly for what we have done for the dense case.
As detailed in Section~\ref{subsec:sdmm}, the algorithm provides for iterating over the rows of $A$ with at least one non-zero entry. We start by analyzing the time cost of multiplying the $i$-th row of $A$ with $B$, which is given by the sum of the cost of the following operations. 

\begin{enumerate}
	\item Loading $N_b$ vectorized elements from $C$ (each one of size $n_b$).
	\item Loading each non-zero element in $A_i$. Since the non-zero values of $A$ are stored contiguously in $A.values$, this operation benefits from cache memory.
	\item Loading $N_b$ vectorized elements of $B$ (of size $n_b$) for each non zero element of $A$.
	\item Updating $C_j \leftarrow C_j + x *B_j$, for each $x\neq0$ in $A_i$. Each update consists in $N_b$ FMA instructions. 
	\item Storing $N_b$ vectorized elements of $C$ (each one of size $n_b$).  


\end{enumerate}
Let us define $a_c$ as the set of active columns in $A$, namely the set of columns containing at least one non-zero element, and $a_r$ the set of active rows in $A$. Let us also define $L_c$ as the cost to load and store $N_b$ elements of $C$, $L_a$ the cost of loading one element of $A$ and updating $C_j$ with $N_b$ FMA instructions, and $L_b$ the cost of loading $N_b$ elements of $B$. 
%Observe that $L_b$ depends on the $N_b$ factor: we can remove dependence by dividing by $N_b$, or using configurations in which $N_b$ is fixed. Being $N_b = N/ n_b$, with $N$ the batch size, we will pick the latter option. 
When generalizing the previous costs to the entire matrices, we have to take into account the effects of caching. 
While $A$ and $C$ are loaded just once, $B$ elements can be loaded multiple times; whether they benefit or not from the caching mechanism depends on the access pattern induced by the non-zero entries of $A$.
For example, if $x$ in position $(i,j)$ is a non-zero element, $B_j$ needs to be loaded into the registers from the main memory and the cache will also retain a copy of $B_j$.
Assume that in a successive row of $A$ exists an element $x' \neq 0$ on the same column of $x$, \textit{i.e.} in position $(g, j)$, with arbitrary $g$. When performing $C_g \leftarrow C_g + x' B_j$,  $B_j$ already resides in the cache: since loading elements from the cache is way much faster than loading them from memory, the cost of re-loading $B_j$ can be considered negligible. 
Assuming that once a row of $B$ is loaded into the cache it remains there until the end of the operation, we pay the cost of loading a row $B_j$ just the first time that this row is loaded. At the same time, if there are any inactive rows, they are never used in the multiplication routine. Since the number of active rows in $B$ is equal to the number of active columns of $A$, the cost of loading $B$ can be approximated with $L_b |a_c|$, with $|a_c|$ representing the number of active columns in $A$.

The overall cost of SPMM with the LIBXSMM is given by: 
\begin{equation}
\label{eq:sparsepred}
	T = |a_r| * L_c +  nnz * L_a + |a_c| * L_b
\end{equation}

%is multiplied with the number of active columns. In fact, if we consider $B$ and $C$ to reside in the main memory at the beginning of the operation, once that a line of $B$ has been loaded into the cache it will likely remain into a cache during the rest of the operation. 	

With an accurate estimation of $L_a$, $L_b$, and $L_c$, we can predict the execution time of a sparse-dense matrix multiplication just from the structure of the sparse matrix. Note that this structure is known \textit{a priori}, being the sparse matrices the pruned weights of the neural model. We begin by observing that $L_b$ and $L_c$ both describe memory operations, with the difference that $L_c$ measures both data reading and writing, while $L_b$ refers to data reading. We empirically verify that both the operations have the same time cost, \textit{i.e.}, $L_c = 2 L_b$. 

We now infer the coefficients $L_a, L_b, L_c$, starting from $L_b$. We cannot measure with a timer the cost of the elementary operations we have divided the LIBXSMM SPMM routine in, but we can empirically compute them by difference.

Let us consider two different sparse matrices $A_c$ and $A_{rd}$ with the same shape $m \times k$ and the same number of non-zero entries ($nnz$). $A_c$ has the non-zero values disposed on the same column $j*$, \textit{i.e.}, is a matrix where $a_{i,j} = 0 $ if $j \neq j*$. $A_{rd}$ is a sparse matrix which has single non-zero entry for each row and each column, \textit{i.e.,} $\sum_{i=0}^{i=m-1} a_{i,j} = 1, \forall j=0, \dots, m-1$  and $\sum_{i=0}^{i=k-1} a_{i,j} = 1,  \forall i=0, \dots, k-1$. 
The cost of multiplying $A_c$ and $A_{rd}$
with a dense matrix is given by:
\begin{align}
\nonumber
T (A_c) = m * L_c + nnz *L_a + 1 * L_b \\
\nonumber
T({A_{rd}}) = m * L_c + nnz *L_a + k * L_b
\end{align}
so, 
$$T(A_{rd}) - T({A_c}) =  (k-1) *L_b$$
We can experimentally measure  $T(A_{rd})$ and $T({A_c})$ and use them to compute $L_b$, since $k$ in known. 

To derive $L_a$, we use the same $A_c$ as before and a second matrix $A_{2c}$, having $2*nnz$ non-zero entries, organized along two columns. The cost for multiplying $A_{2c}$ with a dense matrix is given by 
$$T (A_{2c}) = m * L_c + 2*nnz *L_a + 2* L_b$$
Since $L_b$ can be derived using the previous expression, we can subtract $T(A_{2c})$ and $A_c$ and obtain $L_a$ as:
$$L_a = (T(A_{2c} )- T(A_c))/nnz -L_b$$
Our aim is to compute size-agnostic $L_a$, $L_b$ and  $L_c$. We set $M=K$ and vary them in $\{200, 300, 400, 500 \}$ and we experiment $N \in \{16, 32, 64\}$. $L_b$ and $L_c$ depends on $N$ (and so do $L_c$, which is computed doubling $L_b$), so we normalize diving by $N$. We observe that when $N \geq 128$, the value obtained for $L_a$ and $L_b$ diverge w.r.t. to smaller batch size. In fact,  larger $N$ values ($N \geq 128$) break the hypothesis of $B$ residing inside the cache during the whole multiplication, which is a fundamental assumption of our time predictor.  The definitive time predictor parameters are computed as an average of their value obtained with different shape configurations.
We demonstrate the validity of our sparse predictor in Table~\ref{table:est_vs_real_exec_t_sparse}, where we report the predicted and the real execution time needed to multiply the weights of the first layer of several neural models with a random input. We restrain our experiments to the sparsity range obtained with pruning on our neural architectures. As we will detail later, at these sparsity levels the time required for SDMM  is negligible w.r.t. to its dense counterpart. We also evidence that our time predictor is specific to matrix multiplication, hence can be essentially applied to fully-connected layers. Convolution or attention-based architectures have different properties that require a specific investigation. We leave these analyses for future works.

As we can see, the predictor is capable of correctly estimating the execution time of different models at high levels of sparsity, with a small error. Specifically, the predictor can fruitfully distinguish between matrix with the same shape but with different sparsity percentages; two examples are the $200 \times 136$ and the $100 \times 136$ instances in Table~\ref{table:est_vs_real_exec_t_sparse}. First, we observe that with sparsity percentage in the order of $1\%$, SDMM execution times can vary up to $30\%$. Second, the time predictor correctly reflects this peculiarity, thanks to the deep understanding of the routine details that stands behind its development.

% Observe that an exact predictor is beyond the scopes of this paper; its role is just to reduce the neural models search space by dividing them into models which can match an efficiency requirement an models than can not. 

\begin{table}[htb]
	
	%
	
	\adjustbox{max width=\columnwidth}{
	\centering
	\begin{tabular}{lrrrrrrr}
	%\resizebox{\columnwidth}{!}{
		\toprule
		\multirow{3}{*}{Shape} &   	\multirow{3}{*}{Sparsity}& 
		 \multicolumn{6}{c}{SDMM Time ($\mu s$)} \\
		\cmidrule{3-8}
		& & \multicolumn{2}{c} {$N=16$}& \multicolumn{2}{c} {$N=32$} & \multicolumn{2}{c} {$N=64$} \\
		\cmidrule{3-8}
		 & & Real & Pred. & Real & Pred. & Real & Pred.\\		
		 \midrule
		 \multirow{2}{*}{400$\times$136} &  0.995 & 0.2 & 0.2 & 0.4& 0.4&   0.9 &          0.8 
		 \\
		  &  0.986 &  0.4& 0.4& 0.9& 0.8&         1.9 &          1.6 \\
		   \arrayrulecolor{black!30}\midrule

 	300$\times$136 &  0.985 &0.3 &0.3 & 0.7& 0.7&        1.6 &          1.4 \\
 	 	  \arrayrulecolor{black!30}\midrule

 	 	\multirow{2}{*}{200$\times$136} &  0.982 &   0.3& 0.3& 0.5 & 0.5 &        1.0&          1.0 \\
 	  		&  0.971 &       0.4& 0.3& 0.7&0.6 &  1.5 &          1.3 \\
 	  	\arrayrulecolor{black!30}\midrule

 	 	\multirow{2}{*}{100$\times$136} &   0.989 & 0.1 &0.1 & 0.2& 0.2 & 0.5 &          0.4 \\
          &  0.967 & 0.2&0.2 & 0.3& 0.4&      0.7 & 0.7 \\ 
         \arrayrulecolor{black!30}\midrule

 	 	50$\times$136  &  0.987 &0.1 	& 0.1& 0.1 & 0.1 & 	   0.2 & 		  0.2 \\
 	 	\arrayrulecolor{black}
 	 	\bottomrule
	\end{tabular}
	}
	
	\caption{Some examples of our sparse time predictor with different values of $N$.}
	\label{table:est_vs_real_exec_t_sparse}

\end{table}

% \section{Forest Distillation}
% \label{sec:fordist}
% In this section we detail how to train a Feed-Forward Neural Network (FFNN or NN) to score documents in a Information Retrieval (IR) system. The idea was first proposed by Cohen \textit{et al.}~\cite{cohen2018universal}: NNs are known to be universal approximators~\cite{hornik1991approximation}, so they can be trained to mimic the output of an ensemble of regression trees, becoming \textit{de facto} a document scoring engine.

% Formally, 
% let us consider a Learning to Rank dataset $D = (X, Y)$,  $X \in \mathbb{R}^{f}$, where $f$ is the number of extracted feature per document, and $Y \in \mathbb{N}$ is the set of ground truth relevances w.r.t. a query.
% Let  $F: \mathbb{R}^{f} \rightarrow \mathbb{R}$ be the underlying function learned by a Regression Forest during the training, mapping document features to document scores. To generate a neural-based scoring document system, we need the network to approximate this function; this is done 
% by replacing the original $Y$ ground truth with $F(X)$, \textit{i.e.}, substituting the ground truth relevances with the scores of the ensemble of regression trees on the dataset $X$. 
% %to virtually create a new $D_F = (X, F(X))$. 

% Even if this approximation-based approach could seem counterintuitive, it results profitable because of the different \textit{learning styles} between the models.
% There are several approaches in Learning to Rank: \textit{pointwise}, \textit{pairwise} and \textit{listwise}; among them, the \textit{listwise} has demonstrated to consistently outperform the others.  Anyway, neural models show to struggle in directly exploiting it; the divergence of learning styles is mainly expressed by the objective functions, which happen to be non-differentiable in the \textit{listwise} approach, hindering Stochastic Gradient Descent (SGD) to find global minima. 
% %TODO qui dettagli
% Minimizing the mean square error between the network and the tree-based model outputs is a way to overcome the problem: the learning style itself is similar to the \textit{pointwise} approach, but since the $F(\cdot )$ function is learned with a \textit{listwise} one, the final $N(\cdot)$ function expressed by the neural model results more accurate, as empirically proven~\cite{cohen2018universal}. 

% A reason to compose this approximation is to exploit high performing neural forward systems to obtain faster scoring mechanisms. Anyway, when comparing 
% Neural Network forward time with the state of the art 
% %\ross{state-of-the-art? mi pare di si} 
% method for scoring Regression Trees, namely QuickScorer~\cite{lucchese2015quickscorer,dato2016fast,8035185}, Cohen \textit{et al.}~\cite{cohen2018universal} observed that NNs are faster only when their forward pass is performed on GPU. For the reasons explained in Section~\ref{sec:introduction} we leave out the GPU inference from our comparison and  limit it to CPU based-inference systems.

% %wihotut optmizazed usage simd instructions	
% In the original article, the comparison on CPU involved: a) old QuickScorer results (written in C++), without SIMD instructions\footnote{ this is due to code availability issues, since QuickScorer code is not publicly available}, b) neural inference using a Python deep learning framework. 
% Our first contribution is to  obtain a solid and coherent CPU comparison between the two methods by measuring the performance on the same hardware, with inference mechanisms written in same the programming language. % \ross{setting: hardware and programming language ?} 	
% First, we train a Regression Forest on the \msn dataset~\cite{DBLP:journals/corr/QinL13}, using a grid search to explore the parameters space and to achieve the best $F(X)$ as possible. In this phase, we limit the number of leaves to $64$. We use LightGBM~\cite{NIPS2017_6907} to train the models, which performs better than RankLib, the tool used in the original article. For the sake of fairness, we believe that NNs shall compete with the best available Regression Forest, which is capable to learn an optimal $F^*(X)$ function, that could results harder to approximate with respect to a sub-optimal generic $F(X)$.

% %TODO dati sulla grid search?

% Once we got the tree model, we train the Neural Model to mimic $F(X)$. We picked two network architectures, a Large Network with 4 layers of size $\{2000,500,5000,100\}$ and a Small Network with 2 layers of shapes $\{500,100\}$, as in the original work. We adopt the same strategy for randomly generating training data use by Cohen \textit{et al}. 	~\cite{cohen2018universal}. We use Adam~\cite{kingma2014adam} as optimizer, with learning rate $0.001$ and no weight regularization nor dropout~\cite{srivastava2014dropout}; we multiply the learning rate by $\gamma = 0.1$ at epochs $\{50, 80 \}$ and use and early stopping criterion on the validation loss; we used the Pytorch framework to train the neural networks. We activation function, we used RELU6, with $\text{RELU6}(x) = min(max(x,0), 6)$. 

% In the original work, neural models could perform as well as the Regression Forest trained with RankLib in terms of MAP. On the contrary, Table~\ref{table:effect_comp_orwk} shows that Neural Networks struggle in approximating the optimal $F^*(X)$ function, thus providing lower MAP and NDCG@10 values. Metrics are computed with the RankEval python library~\cite{rankeval-sigir17}. This first observation states that the approximation step has a cost in terms of accuracy drop that has to be taken into account in the analysis of the effectiveness efficiency trade-off. 

%  %inducing a remarkable $\Delta$ in terms of both the chosen metrics. \ross{L'uso di $\Delta$ cosi farebeb schifo anche ai greci. Aggiungerei numeri veri e conclusione dell'esperimento.}

% Forward time is computed both through a python implementation, as in the original work, and in our own C++ version of NN forward pass. In particular,  for the python version we use PyTorch CPU forward while they tested it with TensorFlow. Instead, we use the \textit{dnnl\_sgemm} routine from Intel oneDNN framework to implement matrix multiplications in C++, with JIT compilation, always forcing single-thread execution. We used the latest version of QuickScorer~\cite{lucchese2016exploiting} that exploits SIMD instructions.
%  %while for GPU PyTorch was the framework adopted either in this work and in the original one.
% CPU experiments were conducted on a Intel i9-9900K CPU, with AVX2 (latest ISA supported by QuickScorer) instructions, 3.5 GHz, with L1-cache 256KiB, L2-cache 2 MiB, L3-cache 16MiB.  Neural Network scoring time is computed with batch size = 1000.

% Despite the large gap in terms of NDCG@10, the Regression Forest scored with QuickScorer is faster of the Large Network, as illustrated by Table~\ref{table:speed_comp_orwk}. 
% Anyway, we believe that a fair comparison shall involve two models at the same effectiveness level; as reported by Table~\ref{table:speed_comp_orwk},
% we can obtain the same NDCG@10 of the Large model with $150 $ trees and $64$ leaves per tree, and the NDCG@10 of the Small model with $200$ trees with $32$ leaves per tree.  Table~\ref{table:speed_comp_orwk} also shows that Quickscorer scores a document $16x$ faster than the Large Network and $2.75x$ times faster than the Small one.
% Hence, tree-based solutions are to prefer to neural networks. This section seems to leave no room for neural models in the document scoring  task; in the following section, we will provide a methodology to create feed-forward networks than can compete or even outscore both the effectiveness and the efficiency of the ensemble of Regression Forests.


% \begin{table}
% 	\centering 
% 	\begin{tabular}{lrrrr}
% 		\toprule
% 		Model &     NDCG@10 &     MAP 0 & MAP 1\\
% 		\midrule

% 		RF (878 trees, 64 leaves)&    0.5246& 0.6304  &0.6604  	\\
% 		\midrule
% 		Large Network &   0.5198&0.6279   &0.6579  \\
% 		Small Network & 0.5180& 0.6277 &0.6576  \\
% 		\bottomrule
% 	\end{tabular}
	
% 	\caption{Comparison in terms of NDCG@10 and MAP between a Regression Forest (878 trees, 64 leaves) and Neural Networks on \msn. MAP$x$ means that $x$ is the score assigned to non relevant results. }
% 	\label{table:effect_comp_orwk}
% \end{table}

% \begin{table}
% \centering
% %\adjustbox{max width = \columnwidth}{

% 	\begin{tabular}{lrrr}
% 		\toprule
% 		\multirow{2}{*}{Model} & \multirow{2}{*}{NDCG@10}&    \multicolumn{2}{c}{Scoring Time ($\mu s$/ doc)} \\
% 		\cmidrule{3-4}
% 		& & Python & C++ \\
% 		\midrule
% 		RF (878 trees, 64 leaves)& 0.5246 & / &8.2  	\\
% 		\midrule
% 		RF (150 trees, 64 leaves)& 0.5206& / &1.5 	\\
% 		RF (200 trees, 32 leaves)& 0.5181&  /&0.8 \\
% 		\midrule
% 		Large Net & 0.5198  &35.6 & 24.4 \\
% 		Small Net & 0.5180  & 3.7 & 2.2 \\
% 		\bottomrule
% 	\end{tabular}
% 	%	}
% 	\caption{Comparison in terms of Scoring Time between Regression Forests and Neural Networks on \msn.  }
% 	\label{table:speed_comp_orwk}
% \end{table}


\section{Neural Engineering}
\label{sec:neuraleng}
In Section~\ref{sec:introduction}, we claim that ensembles of regression trees consistently outperform, both in terms of effectiveness and efficiency, NNs trained with the method proposed by Cohen \textit{et al.}~\cite{cohen2018universal}, when documents are scored on CPU. In this section, we break down a methodology used to create efficient neural models for ranking that can compete with ensembles of tree-based ones.

\subsection{Approximation of an Ensembles of Trees}
\label{subsec:approxbetter}
We employ the methodology proposed by Cohen \textit{et al.}~\cite{cohen2018universal} to train neural models that approximate the scores of an ensemble of regression trees. This approach is effective since we use a powerful model, \textit{i.e.}, an ensemble of regression trees, and a profitable learning strategy, \textit{i.e.}, a \textit{listwise} approach, to extrinsic the structure of the actual underlying probability distribution. This facilitates the learning process of a simpler model, \textit{i.e.}, a shallow neural network. The idea is inherited from a deep learning compression technique named \textit{Knowledge Distillation}~\cite{bucilua2006model, ba2014deep,DBLP:journals/corr/HintonVD15} in which a small, production-oriented, network (\textit{student}) is trained to mimic the output of a large and effective network (\textit{teacher}).

To fully leverage the benefits of this technique, we train an ensemble of regression trees with the best performance on a validation set without taking into account its efficiency. Then, we use its scores as ground truth in a distillation process that trains our neural models. In Table~\ref{table:imprteacher}, we report the validity of this approach using the \msn dataset~\cite{DBLP:journals/corr/QinL13}, a widely adopted LtR dataset composed of more than 30,000 queries, with about 120 documents per query, where 
each document is a vector of 136 features\footnote{The list of features is available at https://www.microsoft.com/en-us/research/project/mslr/}. We adopt the NDCG@10 as quality metric. First, we observe the difference in terms of ranking precision between: 1) a model trained with a fixed number of leaves, \textit{i.e.}, $64$, 2) the best model we could obtain on the \msn dataset. The latter one, which results to have $256$ leaves per tree, consistently outperforms the $64$-leaves model.
%Then, we report the results when using these two tree-based models as teachers for two different neural networks. 
Increasing the number of leaves in tree-based models allows for a remarkable gain in terms of NDCG@10. Indeed, when scoring a tree-based model with QuickScorer, the execution time scales at least linearly with the number of trees and leaves~\cite{lucchese2015quickscorer,dato2016fast}. Hence, a $256$-leaves model is more than $4$x slower than a $64$-leaves one with the same number of trees. 
In fact, given that the scoring time per document of $64$-leaves models is 8.2 $\mu s$, a $256$-leaves one takes at least $33 \mu s$ to be traversed with QuickScorer.
This means that, when pursuing a trade-off between effectiveness and efficiency, the best solution is the ensemble of $878$ trees with $64$ leaves, due to the linear dependency of the scoring time with respect to the number of leaves. 
%\fnote{da dove vedo la giustificazione sperimentale di questa ultima affermazione sopra? FM \cosimo{Non c'e', non possiamo fare esperimenti con 256 foglie perche' QuickScorer non le supporta. Pero' la citazione rimanda al paper di QuickScorer dove c'e' scritto che il tempo di scoring scala linearmente. Cosimo}. chiaro. metterei i conti della serva per convincere il revisore a spanne che quel 4x, usando lo scoring con 256 non scala una sega. altrimetni si chiede perche' ``unbearable long scoring times''. no?}

Furthermore, we report the results when using two tree-based models as teachers for two different neural networks. 
Our experiments clearly show the positive effects of approximating a more effective \textit{teacher} (Table~\ref{table:imprteacher}). In fact, thanks to the teacher upgrade, the  $1000\times500\times500\times100$ can provide the same ranking precision as the $64$-leaves tree-based model. Observe that the \textit{student} is teacher-agnostic: the architecture of the network is independent w.r.t. the tree-based model which is approximating, and so is the time to perform the forward pass. In conclusion, distilling from a more effective teacher bridges the gap between neural models and ensemble of regression trees in terms of effectiveness. Nevertheless, a margin still exists between the two families of models in terms of efficiency. In the following sections, we will show how to tackle this aspect. 

%we evidence the advantages of  approximating a better \emph{teacher} model, and we incorporate this strategy in our general train and design methodology. Anyway, 
%approximating a better model is an uniquely profitable strategy which allows to improve the effectiveness of the neural models, with no drop in terms of efficiency. 

\begin{table}
% \centering
% \begin{tabular}{lrrr}
% 		\toprule
% 		Model &     NDCG@10 &     MAP 0 & MAP 1 \\
% 		\midrule
% 		878 trees, 64 leaves &    0.5246& 0.6304  &0.6604 	\\
% 		600 trees, 256 leaves &   0.5291&0.6321   &0.6621  \\
% 		\bottomrule
% 	\end{tabular}
% 	\caption{Comparison between Regression Forests with different number of trees and leaves on the \msn dataset}
% 	\label{table:64vs256leaves}
%\vspace{0.6cm}
\centering
%\adjustbox{max width = \columnwidth}{
	\begin{tabular}{llr}
		\toprule
		Model  & Teacher & NDCG@10 \\
		\midrule	
		878 trees, 64 leaves &  / &  0.5246  \\
		600 trees, 256 leaves &  / &  $\uparrow$ \textbf{0.5291}     \\
		\midrule
		\multirow{2}{*}{500$\times$100} & 878 trees, 64 leaves & 0.5180 \\
		& 600 trees, 256 leaves &$\uparrow$ \textbf{0.5198} \\  
		\midrule
		\multirow{2}{*}{100$\times$500$\times$500$\times$100} & 878 trees, 64 leaves & 0.5208 \\
		& 600 trees, 256 leaves &$\uparrow$ \textbf{0.5243}  \\
		\bottomrule
	\end{tabular}
	%	}
	\caption{Comparison in terms of NDCG@10 among Neural Networks on \msn, when trained to approximate different teachers.  $\uparrow$ indicates statistically significant improvement (Fisher's randomization test,  $p < 0.05$).  }
	\label{table:imprteacher}
\end{table}


\subsection{Design of a Neural Model}
\label{subsec::neuraldesign}
In this section, we present our novel methodology to design efficient neural models for ranking. We leverage the insights gained in studying dense and sparse matrix multiplication to show how to make correct architectural choices, thus training a very limited set of candidate models. We provide an empirical evaluation to show the correctness of our assumptions. 
Experiments are conducted on the  \msn dataset~\cite{DBLP:journals/corr/QinL13}, as in Section~\ref{subsec:approxbetter}. We first show how to develop dense models matching some given time requirements. Then, we employ pruning techniques to sparsify these models and outperform ensembles of regression trees.

\smallskip
\noindent \textbf{Architecture design}.
Our approach begins by choosing the dense architectures matching some given time constraints. For the sake of simplicity, we will assume to have two tree-based models to compete with, a $300$-trees ensemble and a $500$-trees ensemble, each one with $64$ leaves per tree. Their NDCG@10  and their scoring time ($\mu s$) are reported in Table~\ref{table:widevsdense}. By using the time predictor developed in Section~\ref{subsec:densetimepred}, the identification of the architectures matching the time requirements is now an easier task.
We build a heatmap as in Figure~\ref{fig:heatmap} and then use it to predict the execution time of the architecture, without the need of testing its performance on real hardware. This allows to discard models that do not match the desired latency constraints. As reported in Table~\ref{table:widevsdense}, there can be several models fitting the time budget. In our case, we propose $2,3,4$ layers NNs. We train the chosen models and compare their NDCG@10. \textit{Deep} networks (more layers) afford better performance w.r.t. \textit{wide} ones (more neurons per layer), coherently with the evolution of neural models witnessed in the last decade. The reason is that deep networks are generally capable of extracting higher levels features thus creating more complex representation of the input. The higher representations are built on simpler ones, generating a nested hierarchy of concepts which allows to improve the understanding and the learning from the data~\cite{Goodfellow-et-al-2016}. We empirically verify that $5$-layers models matching the time constraints do not offer advantages with respect to $4$-layers ones, showing that $4$-layer networks are expressive enough for the ranking task. Dense networks offer performance close to the tree-based model but do not really guarantee advantages neither in terms of effectiveness or efficiency, as shown in Table~\ref{table:widevsdense}.

\begin{table}
\centering
%\adjustbox{max width = \columnwidth}{
	\begin{tabular}{lrr}
		\toprule
		%\multirow{2}{*}{Model}  &  \multicolumn{1}{p{3cm}}{\centering Predicted Execution \\ Time ($\mu s$/doc) }  & \multirow{2}{*}{NDCG@10} \\

		Model & Scoring Time ($\mu s$/doc) & NDCG@10 \\
		\midrule

		QuickScorer 300, 64 & 3.0  & 0.5230	\\
		\cdashlinelr{1-3}
		500$\times$100 & 2.2 & 0.5196 \\%& 2.2 \\
		300$\times$200$\times$100 & 2.4 &  0.5209 \\
		300$\times$150$\times$150$\times$30 & 2.2 & 0.5207 \\
		\midrule
		QuickScorer 500, 64 & 4.9  & 0.5240	\\
		\cdashlinelr{1-3}
		1000$\times$200 & 5.5 & 0.5150 \\
		600$\times$300$\times$100 & 5.6 & 0.5203\\
		500$\times$250$\times$250$\times$100 & 5.4 & 0.5218\\
		\bottomrule
	\end{tabular}
	%	}
	\caption{Comparison in terms of Scoring Time between QuickScorer and Neural Networks on \msn. The notation ''QuickScorer $x,y$'' indicates that $x$ is the number of trees, and $y$ the number of leaves per tree.   }
	\label{table:widevsdense}
\end{table}


\smallskip
\noindent \textbf{Sensitivity analysis and pruning.}
In our experiments, dense models do not reach the performance of ensembles of regression trees scored with QuickScorer. We now address the problem by leveraging the advantages brought by \emph{model compression}, in particular by \emph{network pruning} \cite{DBLP:journals/corr/HanPTD15,DBLP:journals/corr/GuoYC16}, a technique that deeply sparsifies a neural model without incurring in performance degradation. Let us consider the time budget of $3 \mu s $:
we devise a model which exceeds the time budget but affords an NDCG@10 close the $300$ trees model. By mean of pruning, we can move to the sparse domain and benefit of fast Sparse Dense Matrix Multiplication routines (SDMM). As example model, we pick a 400$\times$200$\times$200$\times$100 network: its performance are reported in Table~\ref{table:sparse_400x200x200x100_partial}.
As detailed in Section \ref{subsec:modelcompr}, \textit{magnitude-based} pruning methods deliver high compression rates without accuracy loss. We adopt this family of pruning techniques to sparsify the parameters of our model. Recall that magnitude pruning technique zero-out a given amount of low absolute value weights. The amount of zeroed-out values determines the aggressiveness of the sparsification: the way this aggressiveness is controlled distinguishes between level pruning and threshold-based pruning.
In the case of level pruning, we can explicitly set the sparsity target, \emph{e.g.}, 70\%.
%
%
In the threshold-based magnitude pruning by Han \textit{et al.}~\cite{DBLP:journals/corr/HanPTD15}, instead, we need to chose a statistical based threshold, as detailed in Section~\ref{sec:related}. The choice is generally based on the \textit{sensitivity} of each layer, namely  the property that describes a layer's resistance to sparsification.
We perform two kind of sensitivity analysis: \textit{static} and \textit{dynamic}. Both procedures prune a growing percentage of weights in each layer, one layer at a time, and evaluate the behavior of the partially-pruned model on the validation set. In the static version, there is no re-training~\cite{DBLP:journals/corr/HanPTD15} of the weights that survived the pruning in the chosen layer and the weights in the other layers, while in the dynamic version re-training is applied.
Static analysis is reported on the left side of Figure~\ref{fig:sensitivity}. The sensitivity of each layer appears to decrease as we go deeper into the network, meaning that the first layers suffer the most from sparsification. Dynamic analysis (right side of Figure~\ref{fig:sensitivity}) shows an inverse trend and highlights a peculiar behavior of the first layer: high levels of sparsity in this layer allow the pruned model to outperform the dense one in terms of NDCG@10.  This is an example of a model compression technique acting as regularizer \cite{DBLP:journals/corr/HanPTD15,DBLP:journals/corr/ZhouYGXC17}.
T%his effect is called regularization, \textit{i.e.}, a growth in terms of generalization capabilities by the model. Pruning techniques can induce this feature since they can be seen as an inference-time version of dropout~\cite{srivastava2014dropout}, \textit{i.e.}, a well-known regularization technique.
This effect is especially evident in the first layer as it presents the weights with the largest absolute values among all the other network layers. Observe that the effect of matrix multiplication is dominated by large absolute value entries, and the larger are the values, the larger is their impact. So a reduced number of high absolute weights can well approximate the overall result of matrix multiplication. From a learning point of view, since the network is working on handcrafted features, the sparsification selects just the essential combinations of input features.

\begin{figure}[t]
\begin{minipage}[b]{0.5\columnwidth}
\includegraphics[width=\columnwidth]{imgs/static_sensitivity.png}
\centering 
\caption*{\footnotesize{Static}}
\end{minipage}%
\begin{minipage}[b]{0.5\columnwidth}
\includegraphics[width=\columnwidth]{imgs/dynamic_sensitivity.png}
\centering 
\caption*{\footnotesize{Dynamic}}
%\subcaption{Another subfigure}
\end{minipage}%
\caption{Static and Dynamic Sensitivity Analysis for a 400$\times$200$\times$200$\times$100 network on the \msn dataset.}
\label{fig:sensitivity}
\end{figure}


\begin{table}[b]
	\centering
	\adjustbox{max width=\columnwidth}{
	\begin{tabular}{lR{0.6cm}R{0.6cm}R{0.6cm}R{0.6cm}R{0.6cm}}
		\toprule
		\multirow{2}{*}{Model} & \multicolumn{5}{c}{\footnotesize{Relative Execution Time per Layer (\%)}} \\
		\cmidrule{2-6}
		 & \nth{1} & \nth{2}& \nth{3} &\nth{4}&\nth{5} \\
		\midrule
		400$\times$200$\times$200$\times$100 & \textbf{35} & 33 & 20 & 10 & 2   \\
		100$\times$50$\times$50$\times$10 & \textbf{60} & 21 & 14 & 3 & 2 \\
		200$\times$100$\times$100$\times$50 & \textbf{45} & 28 & 17 & 8 & 2 \\
		\bottomrule
	\end{tabular}
	 }
	\caption{Breakdown of the relative execution time among different layers for different neural models. }
	\label{table:breakdown}
\end{table}
Pruning techniques were originally developed to reduce the size of pre-trained models~\cite{DBLP:journals/corr/HanPTD15,DBLP:journals/corr/GuoYC16}. %Hence, all network's layers are pruned, even if final sparsities can be low because of high sensitivity of some layers. 
Despite, in our context we aim at speeding up the forward pass without incurring in performance degradation. 
This induces us to consider each layer's relative impact on the inference step before applying a pruning technique. In Table~\ref{table:breakdown}, we report a breakdown of the execution times among different layers in different architectures. Observe that the most time-consuming layer is always the first one, even if the largest matrix is the one storing the second layer weights, as for the 400$\times$200$\times$200$\times$100 network. 
Applying bias and ReLU6, in fact, causes the output matrix of the first layer to be brought into the cache, where it resides there during the computation of the second layer. 
Observe also that it is sufficient to reduce the execution time of one of the first two layers to match the time budget  of $3 \mu s$. By using our sparse time predictor we can infer the required sparsity to obtain a given speedup. In Figure~\ref{fig:sparsespeedup}, we draw the sparsity-speedup curve for some matrices, representing the first layers of different architectures. Even if the dense first layer usually has a major impact on the overall execution time, the quadratic growth of the sparse speedup in the selected range annihilates its contribution after the sparsification. For example, in the 400$\times$200$\times$200$\times$100 architecture, the impact of the first layer in the dense version is about $35\%$, while at $95\%$ of sparsity, the estimated speedup using sparse multiplication is $10$x, meaning that the first layer after pruning becomes the second less time-consuming layer after \textit{fc5}. 



\begin{figure}[t]
\centering
\includegraphics[width=\columnwidth]{imgs/sparse_speedup.png}
\caption{Matrix multiplication speedup at various levels of sparsity estimated with our sparse time predictor. We assume the number of active columns/rows to be equal to the total number of columns/rows (worst-case scenario).\label{fig:sparsespeedup}}
\end{figure}

\smallskip
\noindent \textbf{Outperforming tree-based models}.
By jointly harvesting 1) the prominent impact of the first layer on the total execution time, and 2) the regularization effect of pruning the first layer, we develop our \textit{ early-layers efficiency-oriented pruning}. We apply the threshold-based magnitude pruning using the Distiller framework~\cite{nzmora2019distiller}, a deep learning compression framework developed by Intel. This pruning technique generally offers more flexibility and better performance with respect to level pruning. We prune only the first layer in an aggressive fashion and we fine-tune its surviving entries and all the weights of the other layers.
In our final model, the first layer is $98.7\%$ sparse, meaning that there are about $700$ surviving non-zero weights in the first layer, out of 54400 ($400 \times 136$) in the dense matrix.
We use our sparse time predictor to compute the execution time. The speedup obtained with this sparsity ratio on the multiplication of the first layer is about $25$x. This means that the impact of the first layer, which previously amounted to about the $35\%$, is negligible.
In Table~\ref{table:sparse_400x200x200x100_partial} we report the comparison between tree-based models and neural models. While the dense model did not offer any advantages with respect to the tree-based models, the hybrid model - first layer sparse, other layers dense - is both the fastest and the most accurate model. For example, at the same NDCG@10 value, it is $3.2$x faster than the $878$-trees model. 


\begin{table}[t]
	\centering
	\begin{tabular}{llrr}
		\toprule
		Model & Description & NDCG@10 & Sc. Time ($\mu s$/doc) \\
		\midrule
		\multirow{3}{*}{QuickScorer} & 878 trees & $\uparrow$ \textbf{0.5246} & 8.2 \\
		& 500 trees & 0.5240 & 4.9 \\
		& 300 trees & 0.5230 & 3.0 \\
		\midrule
		\multirow{2}{*}{Neural} & Dense & 0.5222 & 3.8 \\
		& Sparse & $\uparrow$ \textbf{0.5246} & \textbf{2.6} \\ 
 		\bottomrule
	\end{tabular}
	\caption{Dense and sparse neural models (400$\times$200$\times$200$\times$100)  vs QuickScorer in terms of NDCG@10 and Scoring Time (Sc. Time). $\uparrow$ indicates statistically significant improvement w.r.t. models of the same family (Fisher's randomization test,  $p < 0.05$).\label{table:sparse_400x200x200x100_partial}}
\end{table}

% !TEX root = paper.tex
% !TeX spellcheck = en_US

\section{Conclusions and Future Work}
\label{sec:conclusions}
In this paper, we presented an effective and efficient methodology to design neural networks for document scoring in a modern information retrieval system. The neural models we take into account are trained to approximate the scores of an ensemble of regression trees. By leveraging a combination of high-performance dense-dense, sparse-dense matrix multiplication, and element-wise pruning, the neural models can compete with the original models.
Thus, our methodology is \textit{effective}. By developing time predictors based on an accurate study of how these operations are implemented on modern processors, we are capable to precisely estimate the execution time of a given architecture by knowing the shape and the sparsity level of each layer. This allows to train only a limited number of models, the ones matching the time requirements given by the specific context. Our methodology is thus \textit{efficient}. Besides presenting our method, throughout the paper emerges a comparison between ensembles of regression trees and NNs on the document scoring task, tested on the \msn and \istella datasets. In our experiments,  neural networks are not capable of reaching the accuracy of their \textit{teacher}, hence tree-based methods are superior in top-quality retrieval scenarios. At any other level of the efficiency-effectiveness trade-off, neural models designed and trained with our approach can always outscore or at least compete with ensembles of regression trees.

As future work, we intend to apply different compression methods such as quantization or early exiting to further improve the efficiency of our neural models. Moreover, we plan to extend our comparison between neural networks and ensemble of regression trees to other computational engines, such as General-Purpose Graphic Processing Unit (GPU) or Field Programmable Gate Array (FPGA). We also aim at improving the training by distillation procedure of neural models, in order to bridge the effectiveness gap with ensembles of regression trees. 

\smallskip
\noindent \textbf{Acknowledgements}.
This paper is partially supported by the ``Algorithms, Data Structures and Combinatorics for Machine Learning'' (MIUR-PRIN 2017) and the OK-INSAID (MIUR-PON 2018, grant agreement ARS01\_00917) projects.

 %we plan to extend our comparison between neural networks and ensemble of regression trees to other computational engines, such as General-Purpose Graphic Processing Unit (GPU) or Field Programmable Gate Array (FPGA). We also intend to apply different compression methods such as quantization or early exiting to Furthermore, we intend to apply our approach to other domains in which neural models have proven to be successful, such as Computer Vision (CV) or Natural Language Processing.

%\fnote{direi troppo generici i future work. starei più sul dettagliato. FM}

%\fnote{io il pezzo sotto non lo metterei. non e' il main result dell'articolo e non e' una motivazione concreta per usare l'uno o l'altro. se sei google, che ci vuole a implementare uno scoring efficiente di alberi con AVX512?}

%However, a neural network-based scoring systems presents a technological advantage: the efficiency of the forward-pass of a neural network strictly depends on matrix multiplication. As mentioned earlier in this paper, several high performance libraries~\cite{van2015blis,xianyi2012openblas} furnish highly optimized versions of this routine, on which we can rely when developing our scoring system. On the other hand, at the moment do not exist libraries for fast traversal of ensemble of trees. %and the QuickScorer code is not publicy available. 
%Developing a tree-based scoring system entails to re-implement an efficient scoring algorithm for regression forests, which should adapt to different architectures and should be maintained and updated to deal with the continuous improvements in the industry of CPUs.   
%For example, QuickScorer does not employ AVX512 vectorized instructions. Updating and maintaining it fully harvest the feature of  modern CPUs is a really challenging task, which requires a considerable effort in terms of development and testing time. 






% !TEX root = paper.tex
% !TeX spellcheck = en_US

\section{Introduction}
\label{sec:introduction}

\IEEEPARstart{T}{he} estimation of relevance is a task of paramount importance in Web search. In fact, search engines provide the users with a list of relevant results answering a information need formulated as a textual query. In the last years, Learning to Rank (LtR) techniques have been successfully applied to solve this task. LtR is the field of machine learning devoted to the development of supervised techniques addressing the ranking problem. LtR techniques have been proficiently used in Web search, a scenario characterized by tight latency bounds for query processing~\cite{cambazoglu2011scalability}. For this reason, the investigation of new LtR techniques targets both effectiveness and efficiency to provide accurate solutions that can be used in modern query processors. State-of-the-art approaches in learning to rank are ensembles of regression trees. Specifically, LambdaMART~\cite{burges2010ranknet} is an effective state-of-the-art LtR algorithm that builds ensembles of regression trees by optimizing a loss function that depends on a listwise information retrieval metric, e.g., NDCG~\cite{jarvelin2002cumulated}. The counterpart of the retrieval accuracy guaranteed by tree-based models is the computational effort needed to traverse hundreds or even thousands of trees. This computational effort hinders the application of this kind of models on low-latency query processors. Furthermore, each tree in an ensemble work by testing a sequence of boolean conditions on the input. The natural translation of this structure in \textit{if-then-else} code conflicts with modern CPU architectures that heavily rely on \textit{branch prediction} and \textit{caching}. A recent line of research investigates techniques for efficient traversal of ensembles of regression trees. The state-of-the-art algorithm for traversing tree-based models is QuickScorer~\cite{lucchese2015quickscorer,dato2016fast,8035185,lucchese2016exploiting}, which implements an interleaved feature-wise traversal of the ensemble that maximizes the efficiency of branch predictor and cache of modern CPUs.

Motivated by the success of neural solutions in other fields such as Natural Language Processing and Computer Vision, several attempts have been made to bring Neural Networks (NNs) in the LtR field. Despite that, tree-based solutions still provide state-of-the-art performances on different benchmarks, especially when dealing with handcrafted features~\cite{qin2020neural}. 
Recently, Qin \emph{et al}~\cite{qin2020neural} identify the reasons for the superiority of tree-based solutions in i) the sensitiveness of neural network to input features scale and transformations, ii) the lack of expressiveness in mostly adopted neural models in LtR, iii) the limited size of available LtR datasets w.r.t. to Natural Language Processing or Computer Vision. Cohen \textit{et al.}~\cite{cohen2018universal} develop an approach that permit to overcome these limitations on standard LtR datasets by training classic multi-layer perceptrons using simple data normalization ($Z$-normalization) and by leveraging a data augumentation technique (Section \ref{sec:cohen}). 
%\cosimo{One of the inherent difficulties of ranking is that evaluation metrics are non-differentiable, hindering the usage of Stochastic Gradient Descent (SGD), the optimization algorithm used for neural network training. In general, differentiable proxies of ranking metrics are employed to train machine learning models.}
Cohen \textit{et al.}~\cite{cohen2018universal} propose to train neural networks to mimic the outputs of a pre-trained ensemble of regression trees. They do so by employing a knowledge distillation approach~\cite{ba2014deep,DBLP:journals/corr/HintonVD15} that treats the ensemble of regression trees as a black box generating accurate document scores.
Given that neural models are universal approximators~\cite{hornik1991approximation}, the network can reproduce the predictions of the ensemble of regression trees. In practice, this is done by using the Mean Square Error between the scores and the network predictions as training loss. The performance of a neural network trained by scores approximation are bounded by the performance of the tree-based model used to generate the scores: even in a perfect approximation scenario, the neural model will introduce no improvement in terms of effectiveness. In general, instead, the approximation will cause a degradation in the ranking precision. However, the reason to move to a neural document scoring engine is to exploit fast inference mechanisms available for NNs.
%\cosimo{In this direction, Cohen \textit{et al.}~\cite{cohen2018universal} compare the efficiency of a neural solution for ranking  with QuickScorer~\cite{lucchese2015quickscorer}.  }
In this direction, Cohen \textit{et al.}~\cite{cohen2018universal} compare the efficiency of a neural solution for ranking (on GPU and CPU) with QuickScorer~\cite{lucchese2015quickscorer} (on CPU).
In the original work, the authors claim that neural models are as accurate as ensembles of regression trees in terms of Mean Average Precision (MAP), and largely outperform them in terms of execution time ($\mu$s/doc). We observe that their comparison presents some weaknesses.
They compare a single-thread CPU version of QuickScorer against a multi-thread GPU version of the neural forward pass. Due to the differences between the computational engines, this does not permit to actually state which one of the two solutions is the more efficient.
 %They report the results of a CPU-based - single-thread - version of QuickScorer. We observe that comparing it with a multi-thread GPU version of the neural forward pass evaluates more the power of the computational engines than the efficiency of the algorithms.  
 Even when comparing on CPU, the comparison is done using: i) a single-threaded C++ implementation of QuickScorer for ensembles of regression trees and ii) a multi-threaded Python neural inference running with an unspecified number of threads. The use of Python APIs may also entail some latency in calling the underlying optimized matrix multiplication routine on which these frameworks usually rely.\footnote{See for example \url{https://scipy-cookbook.readthedocs.io/items/ParallelProgramming.html\#Use-parallel-primitives}} 
%\cosimo{They do not specify the number of threads used in the python version. Also, there may be a latency   }
Moreover, the two sets of experiments are conducted on different CPUs. These aspects hamper a direct comparison of the performance achieved.


In this article, we propose a solid, fair and comprehensive comparison of the efficiency of ensemble of regression trees and neural models.
We compare QuickScorer~\cite{dato2016fast} against a novel and optimized implementation of neural network inference written in C++. We perform the evaluation on the same hardware by executing the two solutions using a single thread. Moreover, both solutions exploit instruction-level parallelism (AVX2 instruction set). Since CPU and GPU are two different processing units and each of them requires specific optimization techniques, in this work we focus on providing an accurate study of the efficiency of the two approaches on CPU, while we plan to extend it to the GPU in the future. Regarding the training phase, we adopt the same neural architectures of Cohen \textit{et al.}~\cite{cohen2018universal} and we re-implement their methodology with our own code in Pytorch~\cite{NEURIPS2019_9015}. However, differently from the original work, in our experiments we train the ensemble of regression trees with the LightGBM library~\cite{NIPS2017_6907}, since it is the state-of-the-art library for learning ensemble models on ranking tasks~\cite{NIPS2017_6907,qin2020neural}.

\begin{table}[htb]
\centering
\begin{tabular}{llllr}	
		\toprule
		Model & NDCG@10   & NDCG&  MAP & \thead{ Scoring Time \\ ($\mu s$/ doc)} \\
		\midrule
		Large Forest & 0.5246\textsuperscript{$\star \dag$} & 0.7473\textsuperscript{$\star \dag$} & 0.6604\textsuperscript{$\star \dag$} & 8.2  	\\
		\cdashlinelr{1-5}
		Mid Forest & 0.5206\textsuperscript{$\dag$}&0.7454\textsuperscript{$\dag$} &  0.6582\textsuperscript{$\dag$}& 1.5 	\\
		Small Forest & 0.5181& 0.7438 &  0.6578& 0.8 \\
		\midrule
		Large Net & 0.5198\textsuperscript{$\dag$} & 0.7445\textsuperscript{$\dag$} & 0.6582\textsuperscript{$\dag$} & 24.4 \\
		Small Net & 0.5171 &0.7432 & 0.6575   & 2.2 \\
		\bottomrule
\end{tabular}
\caption{A comparison between QuickScorer and Neural Networks on the \msn dataset. 
Symbols evidence statistically significant improvement w.r.t. to Mid Forest (\textsuperscript{$\star$}), and Small Forest (\textsuperscript{$\dag$}),  according to the Fisher's randomization test,  $p < 0.05$.}
\label{tab:firsttab}
\end{table}

%\ftodo{tabella sopra. perche' large forest ha simbolo di stat sig da sola? FM}

The results of our comprehensive experimentation on the \msn dataset show that, in contrast with the results reported by Cohen \textit{et al.}~\cite{cohen2018universal}, ensembles of regression trees are both faster and more accurate than neural models. In Table \ref{tab:firsttab}, we report the Mean Average Precision (MAP), the Normalized Discounted Cumulative Gain (NDCG, with cutoff at 10 and without cutoff), and the scoring time per document.
 %Moreover, models associated with the same symbol ($\star$, \dag, $\ast$) are statistically equivalent, according to the Fisher's randomization test,  $p < 0.05$. 
%We observe that the performance of the models with the same symbol ($\star$, \dag, $\ast$) are statistically equivalent. 
Symbols evidence statistically significant improvement w.r.t. to Mid Forest\textsuperscript{$\star$}, and Small Forest\textsuperscript{$\dag$}, according to the Fisher's randomization test,  $p < 0.05$. We run different tests for each metrics, but we use shared symbols to ease the notation.
Table \ref{tab:firsttab} shows that ensemble of regression trees deliver the same performance of neural models while being largely faster, with a speedup ranging from $2.8$x (Small Net vs Small Forest) to $16.2$x (Large Net vs Mid Forest). Also, the Large Forest is the best performing model with a large margin, while being $3$x faster than the Large Net.
 %\textit{i.e.}, \textit{Large Forest} and \textit{Large Net} respectively. Moreover, when considering a given NDCG@10 value fixed by the neural network, tree-based models result consistently faster, with a speedup ranging from $2.8$x (Small Net vs Small Forest) to $16.2$x (Large Net vs Mid Forest). 
 These evidences highlight how tree-based solutions are currently faster than neural networks on CPU. We bridge the large gap between tree-based models and neural networks by proposing a novel framework to efficiently design and train effective and efficient feed-forward networks for ranking on CPU.

The novel contributions of this article are:
\begin{itemize}
\item we present a combination of state-of-the-art approaches to improve the performance of neural networks on Learning to Rank tasks. By leveraging efficiency-oriented pruning techniques and high-performance Dense and Sparse Matrix Multiplication techniques, we build neural models that outperform ensembles of regression trees. An extensive experimental evaluation on two well-established public benchmarks, \textit{i.e.}, the \msn~\cite{DBLP:journals/corr/QinL13} and the Tiscali \istella~\cite{dato2016fast} datasets, shows the effectiveness of our method. Experimental results confirm that on the \msn dataset it is possible to obtain up to $4.4$x faster scoring time with no loss of accuracy.

\item we provide a novel way to estimate the execution time of neural network forward pass, by mean of dense and sparse time predictors, respectively for Dense-Dense and Sparse-Dense Matrix Multiplication (DMM \& SDMM). To the best of our knowledge, this is the first work that dives into the technicality of matrix multiplication to precisely predict the execution time of neural models.
These predictors are derived from a broad study of the implementation of the relative operations on modern CPUs. In explaining how predictors are developed, we also provide a clear and concise explanation of these two fundamental operations with plenty of scientific applications. 

\item we develop an efficient and effective approach to design neural models, using the aforementioned time predictors, which allow to estimate the execution time of a feed-forward network \textit{a priori}, by providing the architecture - \textit{i.e.,} the number of layers and the neurons per layer - and the sparsity level of each layer. This design methodology tackles the costly problem of model architectures search~\cite{strubell2019energy,patterson2021carbon}, since it allows to train \emph{exclusively} the models respecting the latency requirements, tearing down the costs, in terms of time and energy consumption, of the experimental phase.
\end{itemize}

The rest of the paper is organized as follows: Section~\ref{sec:related} discusses the related work in the field. Section~\ref{sec:cohen} details the process of distilling ensemble of regression trees into neural networks as proposed by Cohen \emph{et al.}~\cite{cohen2018universal}. Section~\ref{sec:ModelMatMult} introduces the implementation of dense-dense matrix multiplication and sparse-dense matrix multiplication on modern CPUs, together with our time predictors. Section~\ref{sec:neuraleng} describes our novel method for designing efficient neural models for ranking. Moreover, Section~\ref{sec:experiments} presents a comprehensive experimental evaluation of our proposed technique on public data. Finally, Section~\ref{sec:conclusions} concludes the work.


