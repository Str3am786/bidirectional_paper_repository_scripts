\newcommand{\eps}{\epsilon}
\newcommand{\state}{s}
\newcommand{\states}{\mathcal{S}}
\newcommand{\action}{a}
\newcommand{\actions}{\mathcal{A}}
\newcommand{\transition}{\mathcal{T}}
\newcommand{\generaltransition}{ \transition\left(\state_{t+1} \given \state_t ,\action_t \right)} %
\newcommand{\sgeneraltransition}{ \transition(\state \sgiven \state_t ,\action_t )} %
\newcommand{\rewardfunc}{\mathcal{R}}
\newcommand{\reward}{r}
\newcommand{\return}{R}
\newcommand{\rewardscale}{\alpha}
\newcommand{\discount}{\gamma}
\newcommand{\learningrate}{\alpha}
\newcommand{\traj}{\tau}
\newcommand{\loss}{\mathcal{L}}
\newcommand{\policy}{\pi}
\newcommand{\generalpolicy}{ \policy_{\modelparam}\left(\action_t \given  \state_t \right)} %
\newcommand{\sgeneralpolicy}{ \policy_{\modelparam}(\action \sgiven  \state_t )} %
\newcommand{\probratio}{\varrho}
\newcommand{\policygradient}{g}
\newcommand{\modelparam}{\theta}
\newcommand{\valueparam}{\Theta}
\newcommand{\expectation}{\mathds{E}}
\newcommand{\ppoclipping}{\epsilon}
\newcommand{\ppoadvantage}{A}
\newcommand{\valuefunction}{V}
\newcommand{\qfunction}{Q}

\newcommand{\opt}{\star}
\newcommand{\sample}{\sim}
\newcommand{\given}{\:\big|\:}
\newcommand{\sgiven}{\,|\,}

\newcommand{\wavenumber}{k}
\newcommand{\turbenergy}{E}
\newcommand{\tfilter}[1]{\overline{#1}}
\newcommand{\lesfilter}[1]{\overline{#1}}

\newcommand{\polynomdegree}{N}

\newcommand{\eqncomma}{,}
\newcommand{\eqnperiod}{.}

\newcommand{\relexi}{ReLeXI}

\section{Conclusions}
\label{sec:conclusion}

In this work, we have successfully applied the RL framework Relexi \cite{kurz2022deep} for the development of data-driven turbulence models for implicitly filtered LES.
To this end, we have formulated the turbulence modeling task as an RL problem using LES of forced HIT as the training environment.
The LES was filtered implicitly by a modern high-order discretization based on a DG scheme to demonstrate that the RL framework can incorporate complex implicit LES filter functions into the training process by design.
The employed policy is based on a CNN architecture which relies on elementwise inputs and outputs only without the need of global flow statistics.
The agent was trained within the Relexi framework using the PPO algorithm to adapt the elementwise model parameters of Smagorinsky's model dynamically in space and time.
The underlying spectrum of turbulent kinetic energy of a high-fidelity solution was used as optimization target, which the agent should reproduce as accurate as possible.
We have demonstrated that the agent is able to learn highly accurate and long-term stable turbulence models, which clearly outperform established analytical models.
Moreover, the RL-based models are able to reproduce the target spectrum in actual simulation up to and beyond the expected resolution capabilities of the underlying discretization.
The models were shown to generalize well to other resolutions and other Reynolds numbers they were not originally trained on.
In summary, we have demonstrated that RL provides a framework for consistent, accurate and stable turbulence modeling for implicitly filtered LES.

Future work will focus on the incorporation if physical invariances into the models.
While employing the invariants of the gradient tensor as input features for the policy network incorporates Galilean invariance in a pointwise fashion, the CNN architecture is by default only invariant to shifting operations but not to rotations or reflections.
Since those invariances are not incorporated into the architecture, they have to be learned implicitly from data, which reduces the sample efficiency of the training.
Further investigations will incorporate the physical knowledge about the invariances of the underlying equations into the model architecture with specialized architectures like group convolutional layers \cite{cohen2016group}, which were already applied successfully for turbulence modeling in \cite{pawar2022frame} and in \cite{guan2023learning}, or by interpreting the input data as a graph containing only neighboring information without explicit orientation \cite{niepert2016learning,wu2020comprehensive}.
In addition, the framework will be applied to more sophisticated test cases like wall-bounded flows or shear flows.
In this context, also the use of more complex actions spaces for the agent will be investigated.

To summarize, our work here serves as a proof of concept for finding optimal LES closure models through reinforcement learning.
We do not propose to use any of the models prescribed here in practice, but our results demonstrate how to develop near optimal models with this approach, where the physical consistency and expert knowledge in model creation is fused with the currently most powerful learning algorithm for dynamical systems.
Beyond the application to LES modeling, our work here demonstrates the potential of a high-fidelity flow solver in the loop with an RL optimizer.

\section{Results}
\label{sec:results}

The following section discusses the training of the agents and their inference on unseen starting states.
Moreover, the influence of invariant input features on the training success is assessed.
Lastly, the generalization abilities of the trained models to other resolutions and a higher Reynolds number are investigated.%
\footnote{The trained models and the required training files can be obtained from \url{https://github.com/flexi-framework/DRL_LES}.}

\subsection{Training}
\label{sec:results_training}

\begin{figure}[htb!]
  \centering
  \includegraphics[width=0.99\linewidth]{tikz_double_column/draft-figure2.pdf}
  \caption{Training results for the 24 DOF (left) and 48 DOF (right) test case showing (from top to bottom) the undiscounted collected return on the unseen testing data, the same for the training episodes with the minimum and maximum return indicated by the shaded area, the policy gradient loss according to \eqref{eq:clipping_loss} and the value estimation loss for the value estimation ANN. The results are obtained for either 16 or 32 full episodes used per policy update. Please note that the y-axis of the policy gradient loss is scaled differently for both cases. Since the training simulations update the predictions each $\Delta t_{RL}=0.1$ and are simulated up to $t_{end}=5$, the maximum undiscounted return is $\return_{max}=50$ considering the reward in each step is normalized to $\reward_t\in[-1,1]$.}
  \label{fig:results_training}
\end{figure}


The training took between a single day for the smallest and up to 5 days for the largest cases on the HAWK supercomputer at the High-Performance Computing Center Stuttgart (HLRS).
The training was performed using up to 1024 CPU cores for the simulations and a single GPU node for model execution and training.
For more details on the hardware configuration, the reader is referred to \cite{kurz2022deep}.
The training behavior of the 24 DOF and the 48 DOF configurations is shown exemplarily in \figref{fig:results_training}, since these configurations constitute the lowest and highest employed resolution, respectively.
Both cases were trained once with 16 and with 32 episodes per parameter update.
For all cases, the collected return in the simulation is negative for the randomly initialized policy.
Starting from this initial random policy, the collected return increases during training until the return converges and plateaus just under the maximum undiscounted return of $\return_{max}=50$.
The maximum return follows from the 50 rewards collected during a simulation and the normalization of the reward function in \eqref{eq:reward} that guarantees $r_t\in[-1,1]$.
The convergence of the collected reward indicates that the RL algorithm has found a local optimum with its current policy.
Generally, the gradient estimator used for the gradient ascent algorithm should be more accurate if the amount of sampled episodes per parameter update is increased.
This can speed up the training, since a better approximated gradient can lead to more efficient parameter updates and thus reduce the overall training iterations needed for convergence.
This was indeed observed consistently for all investigated configurations.
Interestingly, the larger 48 DOF case requires less training iterations for convergence compared to the 24 DOF case.
Moreover, the 48 DOF case exhibits less variance in the return sampled in the training runs.
We attribute this reduced variance to two different factors.
First, the influence of the eddy-viscosity model on the overall flow decreases with increasing resolution, since the model accounts for a diminishing amount of unresolved kinetic energy in the flow.
Secondly, a single flow state contains more elements for the larger cases.
Since the policy trains on elementwise data, the same amount of episodes thus provides more training samples for the larger configurations.
The overall reduced variance in the training process then might cause more sample-efficient and faster optimization.
However, since the amount of required iterations for convergence did not decrease consistently with increasing resolution, the faster training might also be simply caused by the stochasticity of the training process.

The last row in \figref{fig:results_training} shows the loss of the value estimation ANN, which is trained to approximate the expected future return starting from a given state of the environment.
The value ANN is initialized with random weights and thus gives a poor estimate of the expected return in the beginning of the training, which results in a high initial loss.
The loss then decreases as the value ANN learns a first sensible estimate of the expected return.
Since the policy then starts to improve, i.e. to collect more reward, also the expected return and thus the training targets of the value ANN change.
Therefore, the value estimation loss increases as the policy improves, since the value ANN has to catch up constantly with the policy.
Once the return reaches a plateau, the expected return as target quantity for the training of the value ANN becomes more stable and the value estimation loss decreases again.


\subsection{Inference}
\label{sec:inference}

\begin{figure*}[htb!]
  \centering
  \includegraphics[width=\textwidth]{tikz_double_column/draft-figure3.pdf}
  \caption{Results for the trained RL models (from left to right) in the 24 DOF, 32 DOF, 36 DOF and 48 DOF configuration averaged over $t\in[10,20]$ for an LES initialized with the unseen testing sample. Reported are (from top to bottom) the averaged energy spectra over the wavenumbers $k$, the relative error of the energy spectra with respect to the DNS solution, the distributions $\mathcal{P}(\cdot)$ of the velocity fluctuations, and the distribution of the predicted $C_s$ parameters. The results for the underlying DNS solution as well as an implicitly modeled LES (iLES), the SSM with $C_s=0.17$ and the DSM are shown for comparison. The shaded area for the DNS energy spectrum indicates the maximum and minimum amplitudes observed for each mode. The wavenumber that is resolved by the discretization with 4 points per wavelength is shown dashed and the maximum wavenumber used for optimization $k_{max}$ is indicated with a solid black line.}
  \label{fig:spectra_n5}
\end{figure*}

The performance of the trained RL models is evaluated based on an LES, which is initialized with the unseen testing state and is computed for $t_{end}=20$.
All results reported in the following are obtained and averaged over the timeframe of $t\in[10,20]$.
Since the models are trained only on simulations with $t_{end}=5$, this allows to assess the long-term behavior of the models and whether the simulation time used during training was sufficient.
The results in \figref{fig:spectra_n5} demonstrate that the RL models are indeed long-term stable.
To assess the accuracy of the RL models, they are compared to the SSM and DSM as well as the implicit model.
As could be expected, the implicitly modeled LES exhibits a buildup of energy in the upper wavenumbers due to lacking dissipation.
The SSM with $C_s=0.17$ on the other hand introduces too much dissipation and thus fails to preserve the wavenumbers near the resolution limit of the underlying numerical scheme.
The RL model clearly outperforms both models for all considered configurations by matching the target spectrum almost perfectly up to and even beyond the discretization's resolution limit of around 4 points per wavelength.
The advantage of the RL models is more pronounced for the small cases, where the turbulence model has more impact on the overall flow.
Interestingly, the errors for the different wavenumbers are distributed more evenly for the RL model.
This might stem from the objective of minimizing the squared error in the energy spectrum in order to increase the reward as given in \eqref{eq:reward}. 

The DSM, however, reproduces the DNS spectra with similar accuracy as the RL model.
This is to be expected, since the DSM is known to provide near perfect results for the HIT case, as long as the test filter is situated in the inertial range.
The RL model achieves a similar level of accuracy, i.e. a near optimal policy, without having access to the additional filtering procedure of the DSM.
It is interesting that the most notable difference between the DSM and RL is for the 24 DOF case, where the RL agent provides a significantly better energy spectrum.
A likely explanation for this loss in accuracy of the DSM is that the LES resolution in the 24 DOF case is too coarse for the test filtering to occur in the scale-similar region.
Thus, no meaningful information is provided to the DSM procedure.
The RL model, however, seems to compensate for this lack of resolution and reproduces the DNS spectrum with very good accuracy.
It is important to stress here once again, that the underlying forcing of the test case prescribes the overall energy budget of the simulation.
Therefore, errors in the high wavenumbers might influence the energy contained in the low wavenumber and vice versa.
This stresses the capabilities of the RL models even more, which have learned to interact with this forcing such that the energy spectrum fits the prescribed one based on local information of the flow field only.

\begin{figure*}
  \centering
  \includegraphics[width=\textwidth]{tikz_double_column/draft-figure4.pdf}
  \caption{Comparison between the RL models trained with the local momentum field $\widetilde{\rho v}_i$ and with the invariants of the velocity gradient tensor $\lambda_{\nabla v}^i$ as inputs for the (from left to right) 24 DOF, 32 DOF, 36 DOF and 48 DOF configuration. The top row shows the undiscounted return collected during training with the shaded area indicating the episodes with the maximum and minimum return. The lower row shows the energy spectra in comparison to the DNS solution. The shaded area in the spectra indicates the maximum and minimum observed energy contained in the respective wavenumber during the DNS. The wavenumber that is resolved by the discretization with 4 points per wavelength is shown dashed and the maximum wavenumber used for optimization $k_{max}$ is indicated with a solid black line.}
  \label{fig:invariants}
\end{figure*}

It seems important to stress again that the energy spectra are the optimization target for the training and thus might provide only limited insight into the model's overall ability to reproduce the required turbulent statistics.
To this end, the distributions of the velocity fluctuations produced by the different models are investigated as additional important measure of the models' performance.
The differences between the models observed here are consistent with the obtained energy spectra.
Since additional dissipation tends to reduce velocity fluctuations, the SSM exhibits the narrowest and the implicit LES the broadest distribution.
For all cases, the velocity fluctuations produced by the RL model appear to be balanced between both effects, since they follow the DNS distribution more closely than the SSM for small fluctuations in the center of the distribution, but do not exhibit the overpronounced tails observed for the implicit LES.
However, the RL model produces an unsymmetrical distribution of velocity fluctuations for the 36DOF case.
This behavior was not observed for any other configuration or model and its origin is still subject of on-going investigations.
It is also unclear whether this behavior emerges since the symmetry condition has to be learned by the model implicitly during the training and might be not strict enough or whether this effect emerges from the long-term interactions between the agent and the forcing method.

The distributions of $C_s$ in the bottom row of \figref{fig:spectra_n5} show qualitatively similar results for all cases.
Most predictions are close to zero with an decreasing amount of higher values.
The overall range seems to strongly depend on the resolution, since the agent exploits almost all of its available action space for the 24 DOF case performing actions near the prescribed maximum of 0.5.
In contrast, the largest $C_s$ prediction for the 48 DOF case does not exceed 0.3.
The predictions' mean thus decreases for increasing resolutions as is consistent with the understanding of an increased LES resolution on the Smagorinsky model.
This difference might stem either from the policy itself, i.e. the policies learned indeed different distributions, or from the input data of the respective resolutions, which might exhibit different distributions.
Nonetheless, these results show the flexibility and capability of the RL training approach to incorporate physical constraints into the model through the choice of the input features.
We did not adapt the expressivity of the ANN between the two input selections, which might increase the performance.

Sarghini et al. \cite{sarghini2003neural} and Maulik et al. \cite{maulik2021deploying} reported a speedup by applying ANN instead of the computationally expensive DSM.
To this end, we compared the computational time required to evaluate the policy and the dynamic procedure of the DSM on a single CPU core for the 48 DOF case.
We found that the time required was comparable for both cases with the RL policy requiring around 10 per cent more time on our hardware.
These results are encouraging, since we did not perform any optimizations of the policy in terms of computational efficiency or model size and did not use GPU acceleration for this comparison, which improves the performance of the RL-based policy significantly.

\subsection{Input Features}
\label{sec:results_features}

In a next step, the models trained on the local momentum field as inputs are compared to the models trained on the five invariants of the velocity gradient tensor $\lambda_{\nabla u}^i$.
The results in \figref{fig:invariants} indicate that again all models successfully improve during training.
However, the training is generally slower and less stable than the former models.
As a result, the final models using the invariants as inputs still partly improve over the analytical models, but perform worse than the models using the momentum field, especially in the 36 DOF case.
This indicates that it is generally harder to learn a sensible policy from the invariants.
To investigate this further, we increased the training time.
While the models always seemed to improve to some degree, even with double the amount of training iterations the gap to former models stayed quite substantial.

We attribute this to the different distributions of the input quantities.
For the considered HIT test case, the velocity fluctuations have zero mean and a root-mean-squared (RMS) magnitude of unity by construction.
The fluctuations are thus approximately normally distributed with zero mean and unit variance.
This is the optimal distribution for input quantities in machine learning, which typically has to be achieved by normalizing the inputs accordingly.
Since the velocity fluctuations are intrinsically linked to the energy budget in the simulation and are thus constraint by the forcing, the agent's actions have only relatively limited impact on the distribution of velocity fluctuations, as already shown in \figref{fig:spectra_n5}.
In contrast, the invariants of the velocity gradient tensor typically span orders of magnitude.
Moreover, the computation of the gradients, and thus the distribution of the invariants, differ widely depending on the employed numerical discretization.
For DG, the gradients are intrinsically discontinuous across element faces and are also known to produce large gradients at the element faces for underresolved turbulence.
This is especially problematic for the initial states, which are obtained by projecting the DNS flow field onto the DG basis with respective LES resolution.
This projection causes large gradients and thus large values for the invariants, which makes it hard to normalize them to a tamer distribution.
We thus assume that the problems in the training stem from the gradient computation of the DG method, for which the gradients exhibit a complex distribution and the invariants computed from it span orders of magnitude, which makes training more difficult for the agent.


\subsection{Generalization to other Resolutions}
\label{sec:results_generalize}

\begin{figure}
  \centering
  \includegraphics[width=0.99\linewidth]{tikz_double_column/draft-figure5.pdf}
  \caption{Results for the RL model trained on the 48 DOF evaluated on the 36 DOF case (left) and vice versa (right). Given are the energy spectra over the wavenumbers $k$ (top), the error of the spectra in comparison to the DNS solution (center) and the distribution of the predicted $C_s$ parameters (bottom). The shaded area indicates the maximum and minimum observed energy contained in the respective wavenumber during the DNS. The wavenumber that is resolved by the discretization with 4 points per wavelength is shown dashed and the maximum wavenumber used for optimization $k_{max}$ is indicated with a solid black line.}
  \label{fig:generalization_resolution}
\end{figure}

To demonstrate that the trained models can generalize to different resolutions, the model trained on the 48 DOF resolution is evaluated on the 36 DOF configuration and vice versa.
This allows to assess how well the trained models can be transferred to LES cases with either more or less resolution.
The results shown in \figref{fig:generalization_resolution} demonstrate that the trained models can also provide stable and accurate results in LES with different resolutions.
This is especially remarkable, since the policy's field of vision for the policy shrinks with increasing resolution due to the elementwise input and output quantities.
The RL model trained natively on the 48 DOF case shows a slight increase in energy in the higher wavenumbers, while the 36 DOF model seems to be slightly more dissipative.
However, the overall errors in the energy spectrum appear to be comparable for both cases.
The classical turbulence models are not shown for clarity.
However, since both RL models provide almost identical energy spectra, they still outperform the SSM and implicit LES model for both resolutions, while matching the performance of the DSM.

Also, the distribution of the models' predictions are almost identical for both cases and thus do not appear to change depending on the LES resolution.
The predictions of the 36 DOF model exhibit a much wider tail, with a maximum prediction of around $C_{s,max}=0.4$.
In contrast, the predictions of the trained 48 DOF model do not exceed $C_{s,max}=0.3$.
Interestingly, the models are still able to reproduce the target energy spectrum despite the deviations in their policies.
It is plausible to assume that the models will generalize even better, if they are trained on a variety of different resolutions, instead of only a single one.
These pronounced differences in the learned policies indicate that the distribution of predictions is not only induced by the input features but is a characteristic property of the policy trained on the respective resolution and the employed discretization.
This again demonstrates that the different discretizations induce different implicit LES filters, which again require different policies to match the underlying energy spectrum.
Thus, the proposed framework allows to develop discretization-adapted turbulent models for implicit LES.

\subsection{Generalization to other Reynolds Numbers}
\label{sec:results_generalize_re}

In a final step, we demonstrate that the trained RL policy is able to generalize to higher Reynolds numbers.
For this, the trained agents for the different resolutions are applied to a HIT flow at a Reynolds number of $Re_{\lambda}\approx 240$, which is considerably higher than the Reynolds number $Re_{\lambda}\approx 180$ used for training.
Analogously to \secref{sec:inference}, the LES were initialized from filtering the DNS flow field at a random point in time to the required resolution.
The LES was then advanced in time for $t_{end}=20$ and the results were averaged over the timeframe of $t\in [10,20]$ to investigate the long term effects of the model onto the flow.

\begin{figure*}[htb!]
  \centering
  \includegraphics[width=\textwidth]{tikz_double_column/draft-figure6.pdf}
  \caption{Results for the RL models trained on $Re_{\lambda}\approx180$ evaluated on a HIT flow with $Re_{\lambda}\approx240$. Reported are (from top to bottom) the averaged energy spectra over the wavenumbers $k$, the relative error of the energy spectra with respect to the DNS solution, the distributions $\mathcal{P}(\cdot)$ of the velocity fluctuations, and the distribution of the predicted $C_s$ parameters. The results for the underlying DNS solution as well as an implicitly modeled LES (iLES), the SSM with $C_s=0.17$ and the DSM are shown for comparison. The shaded area for the DNS energy spectrum indicates the maximum and minimum amplitudes observed for each mode. The wavenumber that is resolved by the discretization with 4 points per wavelength is shown dashed and the maximum wavenumber used for optimization $k_{max}$ is indicated with a solid black line.}
  \label{fig:generalization_re}
\end{figure*}

The results in \figref{fig:generalization_re} indicate that the trained models can indeed generalize to flows at higher Reynolds numbers.
Most importantly, the RL models still provide long-term stable simulations.
The RL models show similar behavior as for the Reynolds number seen during training. 
For the 32 DOF, 36 DOF and 48 DOF cases the RL models is able to reproduce the energy spectrum more accurately than the implicit model and the SSM, but with similar accuracy as the DSM. 
Interestingly, the DSM and RL models exhibit a similar buildup of energy near the cutoff wavenumber, which might indicate that these models lack sufficient dissipation.
Moreover, the RL model sill outperforms the other models and especially the DSM for the 24 DOF simulation, where the modeling assumptions of the SSM and DSM most probably do not hold.

These results are very promising, since they indicate that the trained RL policies are able to extrapolate to other Reynolds numbers (at least to a moderate extent).
The trained policies are thus able to generalize to higher Reynolds number flows as well as other LES resolutions, while matching or even improving on the performance of the DSM, which is known to provide outstanding results for HIT flows.

\section{Introduction}
\label{sec:introduction}


Most flows in nature and in engineering are turbulent.
Such turbulent flows are characterized by a wide range of active flow scales oftentimes spanning orders of magnitude.
Even though the governing equations of fluid motion, the Navier-Stokes equations, are known, the direct numerical simulation (DNS) of this broad range of flow scales is usually intractable.
Instead, reduced model equations are solved, which consider only the most important scales of the flow.
The most popular approaches are the Reynolds-averaged Navier-Stokes (RANS) equations, which compute the time averaged flow field, and the large eddy simulation (LES), which resolves only the most energetic flow scales in space and time.
Both approaches introduce additional terms into the governing equations on the coarse level, which embody the footprint of the non-resolved fine scales onto the resolved flow field.
Since these terms are a function of the unknown full solution, the equations remain unclosed.
Turbulence models are typically employed to approximate the unknown closure terms based on the available coarse scale data in order to close the equations.
Despite decades of research, no overall \textit{best} model has emerged yet.
Moreover, most models employ empirical model parameters that have to be tuned to the specific flow and discretization at hand.
To this end, recent advances in turbulence modeling increasingly strive to complement the established mathematical and physical modeling strategies by the emerging data-driven paradigm of machine learning (ML).

As of today, most of the research in the field of data-driven turbulence modeling is concerned with supervised learning (SL) \cite{brunton2020machine,beck2021perspective,duraisamy2019turbulence}.
In SL, artificial neural networks (ANN) are used to approximate the functional relationship between an input and an output quantity based on example data pairs.
These example data pairs are drawn from a training dataset, which in the context of turbulence modeling is typically obtained a priori from experimental or high-fidelity numerical data.
ANN comprise free parameters (\textit{weights}) which can be fitted to the dataset through optimization, which is then referred to as \textit{training} or \textit{learning}.
In principle, other functions like kernel methods \cite{wenzel2021novel} can be applied as ansatz functions as well, but ANN represent the most prominent choice.
In one of the first works in the field of ANN-based turbulence modeling, Sarghini et al. \cite{sarghini2003neural} applied an ANN as surrogate for the computation of the dynamic viscosity parameter of a mixed turbulence model in order to reduce its computational cost.
Ling et al. \cite{ling2016reynolds} proposed a novel architecture to embed Galilean invariance into their ANN-based turbulence model for RANS.
Gamahara and Hattori \cite{gamahara2017searching} applied a fully-connected ANN to predict the sub-grid stresses for LES of turbulent channel flow.
More sophisticated ANN architectures are applied for turbulence modeling in LES among others by Beck et al. \cite{beck2019deep} and Kurz and Beck \cite{kurz2022machine}, who used convolutional and recurrent ANN, respectively.


The premise of SL is that a sufficiently large dataset with defined input and output quantities is available to train the ANN on.
In the context of LES, this requires that the true closure terms have to be known in order to serve as target for the training.
This is a rather natural assumption for RANS, since here the closure terms are uniquely defined through the temporal averaging.
The same holds for LES with a predefined filter form, which is also called an explicitly filtered LES.
Since the spatial filter is known, the exact closure terms can also be computed from DNS data by applying the prescribed LES filter.
Both approaches thus provide consistent input and target quantities for training in the SL context.
However, for most applications of LES the filter form is not given explicitly.
Instead, the underresolved discretization itself acts as an implicit LES filter that separates the flow scales into resolved and non-resolved scales.
This improves the efficiency of the simulation significantly, but the form of the implicit filter induced by the discretization is typically unknown.
Moreover, the induced filter is typically non-linear and depends on the grid spacing as well as the type of discretization scheme employed.
As a consequence, the closure terms for implicitly filtered LES cannot be computed from high-fidelity DNS data, since the filter that would have to be applied is unknown.

Typically, ANN are trained on surrogate targets instead.
Such surrogate targets are oftentimes obtained by using an explicit LES filter that approximates the discretization's unknown implicit filter form.
A common example is the use of the box filter kernel as approximated LES filter for finite volume methods. %
However, this leads to severe inconsistencies between the training and actual simulations, since the targets the ANN trains on are not the correct closure terms required by the implicit LES \cite{beck2019deep}.
This can lead to inaccurate results or even unstable simulations.
Common approaches to alleviate these problems are either to project the (inconsistent) predictions onto a stable basis to ensure stability \cite{beck2019deep,maulik2019subgrid}, or to use additional regularization during training to increase the robustness of the ANN against this mismatch \cite{kurz2021investigating}.
Rasp et al. \cite{rasp2020coupled} proposed a coupled online training design, in which the pre-trained models are corrected for these inconsistencies by running the model in parallel with a high-fidelity simulation and guiding the model towards this accurate solution.
However, the SL approach for turbulence modeling in implicitly filtered LES is, from our perspective, ill-posed, since the training targets are generally unknown.

An alternative approach that alleviates these problems is the reinforcement learning (RL) paradigm.
In RL, the ML model is not trained by means of an offline training set, but learns by interacting directly with the dynamical environment itself in order to achieve some high-level target.
This allows to optimize data-driven turbulence models based on how they act in actual simulations in the context of the genuine implicit LES filter instead of training ANN on static snapshots of the flow with uncertain labels.
RL is thus conceptually opposite to SL.
In SL, the training data needs to be well-defined a priori, since each input data point requires a known \textit{true} output.
In RL, the training data is generated from the evolving system (i.e. the discretized equations) itself and the \textit{best} outputs are found by the RL algorithm in order to fulfill its overall goal. %
In this sense, RL is a much more natural way of learning in and for dynamical systems that contain a degree of uncertainty as is the case for implicitly filtered LES.

Originally, RL was especially employed for flow control tasks in numerical and experimental setups, e.g. \cite{rabault2019accelerating,rabault2019artificial,tang2020robust,fan2020reinforcement}.
More recently, Novati et al. \cite{novati2021automating} employed an actor-critic RL algorithm to derive data-driven turbulence models for homogeneous isotropic turbulence.
Moreover, Kim et al. \cite{kim2022deep} derived an RL-based model for the Reynolds stresses in turbulent channel flow and Bae and Koumoutsakos \cite{bae2022scientific} applied RL for wall-modeling in turbulent channel flow.

Based on these encouraging results, we contribute in this paper the following.
We apply the novel RL framework Relexi\footnote{\url{https://github.com/flexi-framework/relexi}} proposed in \cite{kurz2022deep,kurz2022relexi} to derive data-driven turbulence models for implicitly filtered LES with a modern high-order discontinuous Galerkin (DG) scheme.
While this discretization choice serves to demonstrate the suitability of our method for modern, high-order discretizations of practical relevance, it also introduces a significant difficulty in the modeling process.
The induced a priori filter form of DG is an element-local $L_2$-projection onto the polynomial basis of the DG method.
However, since the elements are coupled via the numerical fluxes across the element faces, the resulting hybrid operator of the DG method introduces a complex filter form that typically has a non-linear kernel.
We use forced homogeneous isotropic turbulence as a representative canonical test case for training, and learn an optimal, time- and space-dependent eddy viscosity.
We employ the proximal policy optimization (PPO) algorithm for training, which achieves state-of-the-art results in many RL tasks, while being relatively straight-forward to implement.
The architecture of the policy network is based on three-dimensional convolutional layers, which allows to incorporate the flow state in the vicinity of the investigated point into the prediction.
However, the inputs and outputs of the policy are strictly element-local, which means that no information of the global flow state is required as input by the policy.
The locality of the input is a crucial property for practical applications, since global flow statistics are generally expensive to obtain during simulations and are also oftentimes ill-defined therein, especially on more complex computational domains.
This improves on the policy inputs employed by Novati et al. \cite{novati2021automating}, who used additionally the global dissipation rate and the global energy spectrum as inputs for the policy.
We demonstrate that the derived models are long-term stable and that they outperform established analytical LES models in terms of accuracy.
In a last step, we show that the trained models generalize well to different resolutions and to flows at higher Reynolds numbers.

The paper is organized as follows.
\secref{sec:rl} gives a brief outline of the RL paradigm and the PPO algorithm.
In \secref{sec:turbulence}, the task of turbulence modeling is introduced and formulated as an RL problem, which can be solved with the proposed RL framework.
The results in \secref{sec:results} give details on the training results and assess the performance and the properties of the trained RL-based models.
\secref{sec:conclusion} concludes the paper.


\documentclass[final,5p]{elsarticle}





\usepackage{amssymb}
\usepackage{amsthm}


\usepackage{amsmath}
\usepackage{mathtools}
\usepackage{mathdots}
\usepackage{dsfont}

\usepackage{tikz}
\usetikzlibrary{external}
\tikzexternalize[prefix=tikz_external/]
\usepackage{pgfplots}
\usepgfplotslibrary{groupplots}
\usepgfplotslibrary{colorbrewer}
\usepgfplotslibrary{fillbetween}
\usetikzlibrary{colorbrewer}

\usepackage{tabularx}

\graphicspath{{fig/}}

\usepackage{xcolor}
\newcommand{\todo}[1]{{\color{red}TODO:~#1}}

\usepackage[hidelinks]{hyperref}

\newcommand{\figref}[1]{Figure~\ref{#1}}
\newcommand{\tabref}[1]{Table~\ref{#1}}
\newcommand{\secref}[1]{Section~\ref{#1}}
\LetLtxMacro{\originaleqref}{\eqref}
\renewcommand{\eqref}{Eq.~\originaleqref}

\usepackage{booktabs} %

\newcommand{\eps}{\epsilon}
\newcommand{\state}{s}
\newcommand{\states}{\mathcal{S}}
\newcommand{\action}{a}
\newcommand{\actions}{\mathcal{A}}
\newcommand{\transition}{\mathcal{T}}
\newcommand{\generaltransition}{ \transition\left(\state_{t+1} \given \state_t ,\action_t \right)} %
\newcommand{\sgeneraltransition}{ \transition(\state \sgiven \state_t ,\action_t )} %
\newcommand{\rewardfunc}{\mathcal{R}}
\newcommand{\reward}{r}
\newcommand{\return}{R}
\newcommand{\rewardscale}{\alpha}
\newcommand{\discount}{\gamma}
\newcommand{\learningrate}{\alpha}
\newcommand{\traj}{\tau}
\newcommand{\loss}{\mathcal{L}}
\newcommand{\policy}{\pi}
\newcommand{\generalpolicy}{ \policy_{\modelparam}\left(\action_t \given  \state_t \right)} %
\newcommand{\sgeneralpolicy}{ \policy_{\modelparam}(\action \sgiven  \state_t )} %
\newcommand{\probratio}{\varrho}
\newcommand{\policygradient}{g}
\newcommand{\modelparam}{\theta}
\newcommand{\valueparam}{\Theta}
\newcommand{\expectation}{\mathds{E}}
\newcommand{\ppoclipping}{\epsilon}
\newcommand{\ppoadvantage}{A}
\newcommand{\valuefunction}{V}
\newcommand{\qfunction}{Q}

\newcommand{\opt}{\star}
\newcommand{\sample}{\sim}
\newcommand{\given}{\:\big|\:}
\newcommand{\sgiven}{\,|\,}

\newcommand{\wavenumber}{k}
\newcommand{\turbenergy}{E}
\newcommand{\tfilter}[1]{\overline{#1}}
\newcommand{\lesfilter}[1]{\overline{#1}}

\newcommand{\polynomdegree}{N}

\newcommand{\eqncomma}{,}
\newcommand{\eqnperiod}{.}

\newcommand{\relexi}{ReLeXI}



\journal{}

\begin{document}

\begin{frontmatter}



\title{Deep Reinforcement Learning for Turbulence Modeling in Large Eddy Simulations}

\author[label1]{Marius Kurz\corref{cor1}}
\ead{marius.kurz@iag.uni-stuttgart.de}

\author[label2]{Philipp Offenh\"auser}
\ead{philipp.offenhaeuser@hpe.com}

\author[label1]{Andrea Beck}
\ead{beck@iag.uni-stuttgart.de}

\address[label1]{Institute of Aerodynamics and Gas Dynamics, University of Stuttgart, Pfaffenwaldring 21, 70569 Stuttgart, Germany}
\address[label2]{Hewlett Packard Enterprise (HPE), Herrenberger Straße 140, 71034  Böblingen, Germany}

\cortext[cor1]{Corresponding author}


\begin{abstract}

  Over the last years, supervised learning (SL) has established itself as the state-of-the-art for data-driven turbulence modeling.
  In the SL paradigm, models are trained based on a dataset, which is typically computed a priori from a high-fidelity solution by applying the respective filter function, which separates the resolved and the unresolved flow scales.
  For implicitly filtered large eddy simulation (LES), this approach is infeasible, since here, the employed discretization itself acts as an implicit filter function.
  As a consequence, the exact filter form is generally not known and thus, the corresponding closure terms cannot be computed even if the full solution is available.
  The reinforcement learning (RL) paradigm can be used to avoid this inconsistency by training not on a previously obtained training dataset, but instead by interacting directly with the dynamical LES environment itself.
  This allows to incorporate the potentially complex implicit LES filter into the training process by design.
  In this work, we apply a reinforcement learning framework to find an optimal eddy-viscosity for implicitly filtered large eddy simulations of forced homogeneous isotropic turbulence.
  For this, we formulate the task of turbulence modeling as an RL task with a policy network based on convolutional neural networks that adapts the eddy-viscosity in LES dynamically in space and time based on the local flow state only.
  We demonstrate that the trained models can provide long-term stable simulations and that they outperform established analytical models in terms of accuracy.
  In addition, the models generalize well to other resolutions and discretizations.
  We thus demonstrate that RL can provide a framework for consistent, accurate and stable turbulence modeling especially for implicitly filtered LES.
\end{abstract}



\begin{keyword}
  Turbulence Modeling \sep Deep Reinforcement Learning \sep Large Eddy Simulation
\end{keyword}

\end{frontmatter}


\section{Introduction}
\label{sec:introduction}
Theory and algorithms for large-margin classifiers 
have been studied extensively 
since those classifiers guarantee low generalization errors 
when they have large margins over training examples 
(e.g.,~\cite{schapire+:as98,mohri+:mitpress18}). 
In particular, 
the $\ell_1$-norm regularized soft margin optimization problem, 
defined later, is a formulation of 
finding sparse large-margin classifiers based on the linear program (LP). 
This problem aims to optimize the $\ell_1$-margin 
by combining multiple hypotheses from some hypothesis class $\hset$. 
The resulting classifier tends to be sparse, 
so $\ell_1$-margin optimization is helpful for feature selection tasks.
Off-the-shelf LP solvers can solve the problem, 
but they are still not efficient enough for a huge class $\hset$. 

Boosting is a framework 
for solving the $\ell_1$-norm regularized margin optimization 
even though $\hset$ is infinitely large. 
% Roughly speaking, boosting collects a hypothesis one by one
% to maximize the margin. 
Various boosting algorithms have been invented. 
LPBoost~\citep{demiriz+:ml02} is a practical algorithm 
that often works effectively. 
% and often works efficiently in practice.
Although LPBoost terminates rapidly, 
It is shown that 
it takes $\Omega(m)$ iterations in the worst case, 
where $m$ is the number of training examples~\citep{warmuth+:nips07}. 
\cite{shalev-shwartz+:jml10} invented
% Shalev-Shwartz and Singer~\citep{shalev-shwartz+:jml10} invented
%the Relaxed margin algorithm, 
an algorithm
called Corrective ERLPBoost 
(we call this algorithm C-ERLPBoost for shorthand) 
in the paper on ERLPBoost~\citep{warmuth+:alt08}. 
C-ERLPBooost and ERLPBoost 
find $\epsilon$-approximate solutions 
in $O(\ln(m/\nu) / \epsilon^2)$ iterations, 
where $\nu \in [1, m]$ is the soft margin parameter. 
The difference is the time complexity per iteration; 
ERLPBoost solves a convex program (CP) for each iteration, 
while C-ERLPBooost solves a sorting-like problem. 
Although ERLPBoost takes much time per iteration, 
it takes fewer iterations than C-ERLPBoost 
in practical applications. 
For this reason, 
ERLPBoost is faster than C-ERLPBoost. 
Our primary motivation is to investigate boosting algorithms 
with provable iteration bounds, which perform as fast as LPBoost.

This paper has two contributions. 
Our first contribution is to give 
a unified view of boosting for soft margin optimization. 
We show that LPBoost, ERLPBoost, and C-ERLPBoost are 
instances of the Frank-Wolfe algorithm. 
%The second one proposes 


Our second contribution is to propose
a generic scheme for boosting based on the unified view.
Our scheme combines a standard Frank-Wolfe algorithm and \emph{any} algorithm 
and switches one to the other at each iteration in a non-trivial way.
%a Modified LPBoost (MLPBoost) \textcolor{red}{(Rename?)}. 
We show that this scheme guarantees 
the same convergence rate, $O(\ln(m/\nu) / \epsilon^2)$,  
as ERLPBoost and C-ERLPBoost.
One can incorporate any update rule to this scheme
without losing the convergence guarantee 
so that it takes advantage of better updates 
of the second algorithm in practice.
%with fast computation per iteration. 
In particular, 
we propose to choose LPBoost 
as the secondary algorithm, 
% as the second algorithm to combine, 
and we call the resulting algorithm 
Modified LPBoost (MLPBoost). 

In experiments on real datasets, 
MLPBoost works comparably with LPBoost, and 
%if we incorporate the LPBoost update to MLPBoost. 
MLPBoost is the fastest 
among theoretically guaranteed algorithms, as expected. 


Table~\ref{table:boosting_comparison} compares 
LPBoost, ERLPBoost, C-ERLPBoost, and MLPBoost. 
\begin{table}[t]
    \centering
    \caption{%
        Comparison of the boosting algorithms. %
        C-ERLPBoost solves the problem per iteration %
        by sorting based algorithm, while our work and %
        LPBoost solves linear programming (LP). %
        ERLPBoost solves convex programming (CP) per iteration. %
        In practice, the algorithms work fast in the order %
        LPBoost, ERLPBoost, and C-ERLPBoost. %
        As we show in section~\ref{sec:experiments}, %
        our algorithm is as fast as LPBoost. %
    }
    \label{table:boosting_comparison}
    \begin{tabular}{|c|cccc|}
        \toprule
                    & LPBoost & C-ERLPBoost & ERLPBoost & One of our work \\
        \midrule
        Iter. bound 
            & $\Omega(m)$ 
            & $O\left(\frac{1}{\epsilon^2} \ln \frac{m}{\nu}\right)$ 
            & $O\left(\frac{1}{\epsilon^2} \ln \frac{m}{\nu}\right)$ 
            & $O\left(\frac{1}{\epsilon^2} \ln \frac{m}{\nu}\right)$ \\
        Problem per iter. & LP & Sorting & CP & LP \\
        \bottomrule
    \end{tabular}
\end{table}


\section{Reinforcement Learning}
\label{sec:rl}

\subsection{A Brief Introduction}
\label{sec:rl_intro}

\begin{figure}
  \centering
  \includegraphics[width=0.7\linewidth]{RL_MDP.pdf}
  \caption{Schematic of the Markov decision process. The agent performs an action $a_t$ based on its policy $\sgeneralpolicy$ to interact with the environment. Consequently, the environment transitions into a new state $\state_{t+1}$ according to its transition function $\sgeneraltransition$. The resulting reward $\reward_{t+1}$ is specified by the reward function \mbox{$r_{t+1}=\rewardfunc\left(s_t,\state_{t+1}\right)$}, which is used to quantify how desirable a given state transition is. Starting from an initial environment state $\state_0$ this process is repeated until a final state $\state_n$ is reached. }
  \label{fig:MDP}
\end{figure}


In contrast to the supervised learning approach, reinforcement learning trains an agent by letting it interact with a dynamical environment in order to achieve a pre-defined goal.
This has the advantage that the dynamics of the environment are incorporated into the training process directly by design.
The interplay of the agent and the environment can be framed as a Markov decision process (MDP), as illustrated in \figref{fig:MDP}.
In a MDP, the environment is characterized by its current state $s_t\in\states$.
The agent observes that state and can perform a suitable action $a_t\in\actions$.
Here, $\states$ is the set of all possible environment states and $\actions$ is the set of all possible actions that can be performed by the agent.
The agent's actions are determined by its policy $\sgeneralpolicy$ that states which action the agent should perform given the environment's current state $\state_t$.%
\footnote{
  To keep the notation short, we use the abbreviated notation of $\sgeneralpolicy$ for $\policy_{\modelparam}(\action\sgiven\state=\state_{t})$ which describes the conditional probability of choosing action $\action$ given the current state $\state_t$.
The same holds for the transition function $\transition\left(\state'\sgiven\state=\state_{t},\action=\action_{t}\right)$, which will be abbreviated as $\sgeneraltransition$.
}.
The policy can be of any functional form, but for deep RL, the policy is typically an ANN with parameters $\modelparam$.
The agent's action causes the environment to change is state.
This new state is prescribed by the environment's transition function $\state_{t+1}\sample\sgeneraltransition$, which thus encodes the environment's dynamics.
With the state transition, the agent receives a reward $\reward_{t+1}$ that is determined by the reward function $\reward_{t+1}=\rewardfunc(\state_{t},\state_{t+1})$ and quantifies how desirable a certain state transition is.
The new state $s_{t+1}$ is then again observed by the agent, which performs another action $a_{t+1}$ as prescribed by its current policy.
Starting from some initial state $\state_0$ and performing actions until a final state $\state_n$, this results in a trajectory of states, actions and rewards termed an episode:
\begin{equation}
  \traj = \left\{ \left(\state_0,\action_0,\reward_1\right),\left(\state_1,\action_1,\reward_2\right),\:......\;,\left(\state_{n-1},\action_{n-1},\reward_{n}\right),\state_n\right\} \eqnperiod
  \label{eq:trajectory4}
\end{equation}
The goal of an RL algorithm is to establish an optimization problem that allows to find the optimal policy $\policy^{\opt}$, which maximizes the expected return along an episode
\begin{equation}
  \return(\tau) = \sum_{t=1}^{n} \discount^t r_{t} \eqnperiod
\end{equation}
Here, $\discount$ denotes the discount factor that can be used to balance the importance of short-term and long-term rewards.
In deep RL, finding the optimal policy is equivalent to finding the set of optimal model parameters $\modelparam^{\opt}$ for the employed ANN.

For each state $\state$, the \textit{state-value function} describes the total return which can be expected starting from state $\state_t$ and following a specific policy $\policy$ from there on.
This state-value function can thus be written as
\begin{equation}
  \valuefunction^\policy \left(\state\right) = \expectation\left[\sum_{k=0}^\infty \discount^k r_{t+k}\given\state_t=\state\right] \eqnperiod
  \label{eq:valuefunction}
\end{equation}
Similarly, an \textit{action-value function} or \textit{Q-function} can be determined, which gives the expected return when starting from state $\state_t$ performing action $\action_t$ and following the policy $\policy$ from there on, which reads
\begin{equation}
  \qfunction^\policy \left(\state,\action\right) = \expectation\left[\sum_{k=0}^\infty \discount^k r_{t+k}\given\state_t=\state, \action_t=\action\right] \eqnperiod
  \label{eq:qfunction}
\end{equation}
Based on \eqref{eq:valuefunction} and \eqref{eq:qfunction}, one can define the \textit{advantage function}
\begin{equation}
  \ppoadvantage^\policy \left(\state,\action\right) =  \qfunction^\policy \left(\state,\action\right) - \valuefunction^\policy \left(\state\right) \eqncomma
  \label{eq:advantagefunction}
\end{equation}
which quantifies whether taking action $a_t$ in state $s_t$ increases or decreases the expected return in comparison to performing the action prescribed by the current policy.

Solving a given problem with RL thus requires that the problem is casted into an MDP by a domain expert, i.e. defining the environment's possible states $\states$, its transition function $\sgeneraltransition$, the agent's action space $\actions$, the reward function $\rewardfunc(s_t,s_{t+1})$ and, finally, the ANN architecture used for the policy $\sgeneralpolicy$.
With these definitions in place, a suitable RL algorithm can be applied to find a favorable policy.
Each distinct RL algorithm prescribes how interactions of the agent and the environment are collected and how this sampled experience can be used to optimize the policy such that the expected future return is increased.
RL algorithms differ for instance in terms of sample efficiency and whether they allow for continuous state and action spaces.
In the following, we use proximal policy optimization (PPO) as our RL algorithm of choice, which belongs to the class of policy gradient methods.





\subsection{Policy Gradient Methods}
\label{sec:vpg}


The key idea of policy gradient methods is to optimize the policy directly instead of learning the Q-function in \eqref{eq:qfunction} and inferring the policy implicitly from it.
To this end, policy gradient methods derive a gradient estimator that gives the direction in which the model parameters $\modelparam$ have to be changed in order to increase the expected return.
Given this gradient, the model parameters can be updated with a suitable gradient-ascent algorithm.
Following the \textit{policy gradient theorem}, see e.g. \cite{sutton2018reinforcement}, the gradient estimator can be written as
\begin{equation}
  \policygradient = \expectation\Big[\qfunction^{\policy}\left(\state,\action\right) \,\nabla_{\modelparam}\log \policy_{\modelparam}\left(\action\sgiven\state \right) \Big],
  \label{eq:pg_gradient}
\end{equation}
and can be obtained by differentiating the corresponding loss function
\begin{equation}
  \loss^{VPG}(\modelparam) = \expectation\Big[ \qfunction^{\policy}\left(\state,\action\right) \,  \log \policy_{\modelparam}\left(\action\sgiven\state \right)\Big],
  \label{eq:pg_loss}
\end{equation}
with respect to $\modelparam$.
Since the policy and the environment are in general stochastic, the gradient estimator for the optimization is defined by means of the expectation $\expectation\left[\cdot\right]$.
However, obtaining the exact expectation is prohibitive for practical applications.
Instead, the gradient estimator is approximated by sampling $N$ trajectories of experience on the current policy and computing the approximated gradient as mean over the sampled trajectories
\begin{equation}
  \hat{\policygradient} = \frac{1}{N}\sum_{i=1}^N\left[\return\left(\traj^{(i)}\right) \ \sum_{t=0}^n  \nabla_{\modelparam}\log \policy_{\modelparam}\left(\action_t\sgiven\state_t \right)\right] \eqnperiod
  \label{eq:pg_approx_gradient}
\end{equation}
Here, the discounted return along a trajectory $\return(\traj)$ is used as an approximation to the exact Q-function.
Interestingly, the dynamics of the environment, i.e. its transition function $\sgeneraltransition$, do not appear in the gradient estimator or in the overall optimization formulation.
Instead, the dynamics of the environment are incorporated implicitly by the sampled experience.
This avoids to differentiate the environment dynamics with respect to the policy parameters, which is infeasible for most tasks.

The training process of a policy gradient method then works as follows.
First, multiple episodes of experience are sampled with the current policy.
Based on this experience, the policy can be optimized in a second step with the gradient estimator and a suitable gradient-ascent algorithm.
These two steps are then repeated, until the policy has converged.
These building blocks form the vanilla policy gradient (VPG) method.

\subsection{Proximal Policy Optimization}
\label{sec:ppo}

The proximal policy optimization (PPO) method \cite{schulman2017proximal} introduces several improvements over the original vanilla policy gradient (VPG) method to improve the stability of the training.
For a clear and concise summary of the PPO algorithm, we recommend \cite{notter2021hierarchical}.

The first major improvement of PPO is to reduce the variance of the gradient estimator.
This reduces the amount of samples required for an accurate approximation of the gradient or a better gradient estimator for a given amount of samples.
To this end, a baseline can be added to the gradient estimator in \eqref{eq:pg_approx_gradient}, which has been shown to not introduce a bias.
A natural choice is to replace the Q-function by the advantage function from \eqref{eq:advantagefunction}.
However, the advantage function relies on the state-value function $\valuefunction^{\policy}(s)$, which is typically unknown.
Therefore, the PPO algorithm uses an additional ANN $\hat{\valuefunction}_{\valueparam}(s)$ with weights $\valueparam$, which is trained to approximate the state-value function.
Moreover, the return along the trajectory, which approximates the Q-function, is replaced by a \textit{return-to-go} $R_t(\tau) = \sum_{k=t}^n \gamma^{k-t} r_k$ such that each action is only associated with reward that is collected after the action is taken.
The approximate advantage function at step $t$ thus reads
\begin{equation}
  \hat{\ppoadvantage}_t = \left(\sum_{k=t}^n \gamma^{k-t} r_k\right) - \hat{\valuefunction}_{\valueparam}(s_t).
  \label{eq:ppo_advantage}
\end{equation}

A major drawback of VPG is that the training with the VPG method is often found to be unstable, since the policy updates can become arbitrarily large.
Large policy updates imply the risk of deteriorating the policy's performance in a single update step if the gradient estimator is not sufficiently accurate or the step size is too large.
The PPO method increases the stability of the training process by constraining the maximum change of the policy in a single step. %
Schulman et al. \cite{schulman2017proximal} propose two different approaches to limit the updates of the policy.
Firstly, a penalty term can be added to \eqref{eq:pg_loss} based on the Kullback-Leibler divergence between the old and the new policy.
This introduces the incentive to avoid large changes in the policy in a single training step.
The other approach is to replace the loss function in \eqref{eq:pg_loss} by a clipped surrogate objective
\begin{equation}
  \loss^{CLIP}(\modelparam) = \expectation\left[\min\left( \probratio\left(\modelparam\right)\hat{\ppoadvantage},\mathrm{clip}\left(\probratio\left(\modelparam\right),1-\eps,1+\eps\right)\hat{\ppoadvantage}   \right)   \right],
  \label{eq:clipping_loss}
\end{equation}
with $\eps$ as a hyperparameter and with $\probratio(\modelparam)$ as the probability ratio
\begin{equation}
  \probratio\left(\modelparam\right)=\frac{\policy_{\modelparam}\left(\action\sgiven\state\right)}{\policy_{\modelparam_{old}}\left(\action\sgiven\state\right)},
  \label{eq:prob_ratio}
\end{equation}
which describes the change between the old and the new policy.
In \eqref{eq:clipping_loss}, the clip function clips the probability ratio $\probratio(\modelparam)$ to the interval $\left[1-\epsilon,1+\epsilon\right]$ to limit the change of the policy in a single training step.
Similarly to \eqref{eq:pg_approx_gradient}, the expectation in \eqref{eq:clipping_loss} is then approximated by sampling trajectories with the current policy.


\section{Data-Driven Turbulence Modeling}
\label{sec:turbulence}


\subsection{Governing Equations}
\label{sec:equations}
The temporal evolution of compressible, viscous flows is described by the Navier-Stokes equations, which can be written as
\begin{equation}
  U_t + \nabla_x\cdot \left(F^c - F^v\right) = 0,
  \label{eq:navier_stokes}
\end{equation}
with $U=\left(\rho, \rho v_1, \rho v_2, \rho v_3, \rho e \right)^T$ as the vector of conserved quantities comprising density, the three-dimensional momentum vector and energy, respectively.
Here, $(\cdot)_t$ indicates the differentiation with respect to time and the nabla operator $\nabla_x$ the differentiation with respect to the three Cartesian coordinates $x_i$ with $i=1,2,3$.
The convective flux $F^c$ and the viscous flux $F^v$ with columns $i=1,2,3$ can be written as 
\begin{equation}
  F_i^c =
  \left(
  \begin{array}{c}
    \rho v_i \\
    \rho v_1 v_i + \delta_{1i} p\\
    \rho v_2 v_i + \delta_{2i} p\\
    \rho v_3 v_i + \delta_{3i} p\\
    \rho e v_i + p v_i
  \end{array}
  \right)
  ,\;
  F_i^v =
  \left(
  \begin{array}{c}
    0 \\
    \sigma_{1i}\\
    \sigma_{2i}\\
    \sigma_{3i}\\
    \sigma_{ij}v_j-q_i
  \end{array}
  \right) ,
  \label{eq:fluxes_written_out}
\end{equation}
where $p$ denotes the pressure and $\delta_{ij}$ is the Kronecker delta.
The three-dimensional velocity vector is given by $v =(v_1,v_2,v_3)^T$.
The stress tensor $\sigma_{ij}$ and the heat flux $q_i$ can be written as
\begin{align}
  \sigma_{ij} &= \mu\left(\frac{\partial v_i}{\partial x_j}+\frac{\partial v_j}{\partial x_i}-\frac{2}{3}\delta_{ij}\frac{\partial v_k}{\partial x_k}\right) ,
  \label{eq:stress_tensor}\\
  q_i &= - \kappa\:\frac{\partial T}{\partial x_i},
  \label{eq:heat_flux}
\end{align}
with $\mu$ as the viscosity and $\kappa$ as the heat conductivity.
The equations are closed with the ideal gas assumption, which yields the equation of state as
\begin{equation}
  p = \left(\frac{c_p}{c_v}-1\right) \left(\rho e-\frac{\rho}{2}\left(v_1^2+v_2^2+v_3^2\right)\right),
\end{equation}
with $c_p$ and $c_v$ as the specific heats.
In the following, the Navier-Stokes equations are always solved in the incompressible limit with a Mach number of $\mathrm{Ma}=0.1$.



\subsection{Forced Homogeneous Isotropic Turbulence}

\begin{figure}
  \begin{minipage}[t]{0.443\linewidth}
    \includegraphics[width=\linewidth]{./fig/HIT.pdf}%
  \end{minipage}
  \hfill
  \begin{minipage}[t]{0.53\linewidth}
    \includegraphics[width=\linewidth]{./tikz_double_column/draft-figure0.pdf}%
  \end{minipage}
  \caption{Instantaneous flow field of the HIT test case visualized with iso-surfaces of the Q-criterion colored by the velocity magnitude (left) and the mean spectrum of the kinetic energy for the DNS of a forced HIT over the wavenumbers $k$ (right). The shaded area indicates the maximum and minimum observed energy of the corresponding wavenumber.}
  \label{fig:HIT}
\end{figure}

Homogeneous isotropic turbulence (HIT) is a canonical test case for turbulent flows and can be considered as \textit{turbulence-in-a-box}.
The HIT test case describes freely evolving turbulence without an imposed mean flow, influences of walls or any external driving forces.
The computational domain is typically a cube with periodic boundary conditions, as illustrated in \figref{fig:HIT}, which is discretized by an equidistant Cartesian mesh.
The domain is initialized with an initial velocity field that obeys a given turbulence spectrum and is divergence-free.
In this work, the initial flow state was obtained by Rogallo's procedure \cite{Rogallo1981}.
Over time, the flow then produces a turbulent energy spectrum as shown in \figref{fig:HIT}.
In turbulence, energy is transported in the mean from the large scales to the small scales, where the energy is dissipated by the viscosity of the fluid.
In the absence of large scale shear and due to this dissipation mechanism, the velocity fluctuations decrease over time and tend towards zero.
Therefore, this flavor of the HIT test case is also referred to as decaying HIT.
However, this transient behavior causes the flow and the turbulent statistics to change continuously over time.

Different forcing strategies are proposed in literature to inject the dissipated energy back into the system and thus sustain the turbulent flow.
This allows to obtain a turbulent flow with stationary statistics.
For methods with a globally defined solution basis, the forcing can be added in the low modes only, while the higher wavenumbers remain unaffected.
For discretizations with a local basis with compact support, e.g. finite-volume or discontinuous Galerkin schemes, the global Fourier modes of the solution are generally unknown and costly to evaluate.
Instead, a nodal forcing is applied, which acts directly on the solution points instead of global wavenumbers.
Since in this approach the forcing is not limited to the low wavenumbers, the forcing adds energy across the whole spectrum.
This causes the slope in \figref{fig:HIT} to slightly deviate from its theoretical $k^{-5/3}$ trend, since this influx of energy into the high wavenumber modes is not optimal. %
We point out however that, first, such a nodal forcing is often used in element-based discretization schemes and, second, the results in \figref{fig:HIT} indicate that the unwanted effects are minimal.

In this work, we apply the linear isotropic forcing method proposed by Lundgren in \cite{lundgren2003linearly} and analyzed further in \cite{de2015anisotropic}.
Here, a forcing term $f$ is added to \eqref{eq:navier_stokes}, which yields
\begin{equation}
  U_t + \nabla_x\cdot \left(F^c - F^v\right) = f.
  \label{eq:forced_navier_stokes}
\end{equation}
For isotropic forcing, the forcing is assumed to be parallel to the current momentum vector, which gives
\begin{equation}
  f = Q \: 
  \left(
  \begin{array}{c}
    0 \\
    \rho v\\
    0
  \end{array}
  \right),
\end{equation}
where $Q$ is a scalar that quantifies the difference between the current kinetic energy in the flow $E=\frac{\rho}{2} (v\cdot v)$ integrated over the domain and the prescribed target value.
In our implementation, a single forcing parameter $Q$ is employed for the whole flow domain.


\subsection{Large Eddy Simulation}
\label{sec:les}


\begin{figure}[tb]
  \centering
  \includegraphics[width=0.99\linewidth]{tikz_double_column/draft-figure1.pdf}
  \caption{Slices of the instantaneous velocity field of homogeneous isotropic turbulence. The LES flow fields are obtained from the DNS field (left) by applying a spectral cutoff filter (middle) or a projection filter onto a piecewise polynomial basis (right). The latter explicit LES filter can be seen as a rough approximation of the unknown implicit filter form of the DG scheme. More details on the LES filters are given in \cite{kurz2022machine}.}
  \label{fig:les_filter}
\end{figure}

For most flows, it is computationally intractable to resolve all length scales present in the flow.
Instead, the simulation resolves only the largest flow scales, which contain the majority of the kinetic energy.
This approach is referred to as large eddy simulation (LES).
This corresponds to applying a low-pass filter $\tilde{(\cdot)}$ to the governing equations in \eqref{eq:navier_stokes} and solving the resulting evolution equation for the coarse-scale solution $\tilde{U}$.
In \figref{fig:les_filter}, we apply known filter kernels in an a priori manner to visualize the resulting solution fields.
Due to the non-linearity of the governing equations, the filtering operation introduces additional terms into the equation, which describe the effect of the non-resolved fine scales onto the resolved large scales.
These closure terms rely on the full solution $U$ and are thus generally unknown causing the closure problem of LES.
Therefore, a suitable turbulence model is employed that attempts to approximate the unknown closure terms based on the coarse-scale solution $\tilde{U}$.

From \figref{fig:les_filter} we can already appreciate the fact that the associated closure terms will be a function of the chosen filter.
In practice however, the filtering operation is typically not given by means of an explicit filter function.
Instead, the underresolved discretization acts as an implicit LES filter.
While this allows to incorporate all wavenumbers within the resolution limits of the underlying discretization into the simulation and thus improves the efficiency of the LES, the form of the LES filter $\tilde{(\cdot)}$ is generally unknown.
As a consequence, the closure terms for implicitly filtered LES cannot be computed, even if the full solution $U$ is available, since this would require to apply the unknown implicit LES filter, i.e. an application of the spatial and temporal operators.
This makes model development difficult, since the models cannot be optimized by simply fitting them to the \textit{exact} closure terms computed from high-fidelity data.

A common modeling strategy is to mimic the energy transport from the large to the small scales in turbulent flows by introducing a turbulent viscosity.
For instance, Smagorinsky's model \cite{smagorinsky1963general} computes this viscosity as 
\begin{equation}
  \mu_t = \rho\left(C_s\,\Delta\right)^2 \sqrt{2\,\tilde{S}_{ij}\,\tilde{S}_{ij}} \quad \text{with}\quad \tilde{S}_{ij}=\frac{1}{2}\left(\frac{\partial \tilde{v}_i}{\partial x_j}+\frac{\partial \tilde{v}_j}{\partial x_i}\right),
  \label{eq:smago}
\end{equation}
where $\tilde{S}_{ij}$ is the resolved rate-of-strain tensor, $\Delta$ is the filter width of the LES filter and $C_s$ is a model parameter, which has to be tuned manually to specific flows and discretizations.
The eddy-viscosity methodology has the advantage that the LES equations for the coarse-scale solution 
\begin{equation}
  \tilde{U}_t + \nabla_x\cdot \left(\tilde{F}^c - \tilde{F}^v_{turb}\right) = \tilde{f},
  \label{eq:les_equation}
\end{equation}
look identical to forced Navier-Stokes equations in \eqref{eq:forced_navier_stokes}.
Only the viscous flux $\tilde{F}^v_{turb}$ is modified slightly by adding the turbulent viscosity $\mu_t$ to the physical one.
Another common approach is the implicit modeling strategy.
Here, it is acknowledged that the employed discretization adds numerical dissipation to the system which can be interpreted as an implicit turbulence model.

In the standard Smagorinsky model (SSM), the parameter $C_s$ is constant in the computational domain and has to be chosen a priori.
So-called dynamic models alleviate these restrictions by adapting their model parameters based on the current flow state dynamically in space and time.
This concept gives rise to the dynamic Smagorinsky model (DSM), which is detailed in \ref{app:dynsmago}.
In the following, we strive to enhance the Smagorinsky model by means of our RL framework by training an agent to dynamically adapt the model coefficient in space in time during the LES, i.e $C_s=C_s(x_i,t)$.




\subsection{Reinforcement Learning for Turbulence Modeling}


\begin{table*}[htb!]
  \begin{tabularx}{1.0\linewidth}{p{0.125\textwidth}p{0.175\textwidth}X}
    \toprule
    Symbol & Meaning & Turbulence Modeling \\
    \midrule
    $\states$                  & Environment states    & Current flow state of the LES $\tilde{U}$ from which the policy's inputs can be computed.\\
    $\actions$                 & Agent's action space  & Elementwise parameter for Smagorinsky's model within $C_s \in[0,0.5]$ (\figref{fig:policy}).\\
    $\sgeneralpolicy$          & Agent's policy        & CNN-based architecture with elementwise inputs and outputs (\figref{fig:policy}).\\
    $\sgeneraltransition$      & Transition function   & Integration of LES equations, \eqref{eq:les_equation}, for current predictions of $C_s$.\\
    $\rewardfunc(s_t,s_{t+1})$ & Reward function       & Based on error in turbulent energy spectra compared to DNS solution, \eqref{eq:reward}.\\
    \bottomrule
  \end{tabularx}
  \caption{Definition of the major building blocks for the formulation of turbulence modeling as a Markov decision process, as given in \figref{fig:MDP}.}
  \label{tab:rl_turb}
\end{table*}


\begin{figure*}[htb]
  \centering
  \includegraphics[width=0.99\textwidth]{./fig/DGnet.pdf}
  \caption{Network architecture of the CNN-based policy network for $N=5$. The inputs of the network are either the momentum field $(\widetilde{\rho v}_1,\widetilde{\rho v}_2,\widetilde{\rho v}_3)^T$ or the five invariants of the velocity gradient tensor $\lambda_{\nabla v}^i$ in a single DG element with given $N$, for which the distribution of interpolation points is shown exemplarily. The network comprises several three-dimensional convolutional layers (Conv3D) with the corresponding kernel sizes $k$ and the number of filters per layer $n_f$. The output sizes in the dimensions of convolution are given below each layer. The first layer retains the input dimension by means of zero-padding, while for the other layers no padding is employed in order to retain a scalar output $C_s$ per element. The scaling layer applies the sigmoid activation function $\sigma_s(x)$ in order to scale the network output to $C_s\in [0,0.5]$ and can be interpreted as the activation function of the last layer.}
  \label{fig:policy}
\end{figure*}


The problem of turbulence modeling has to be framed as an MDP in order to solve it within the RL framework, as already discussed in \secref{sec:rl}.
In the following, we use LES of forced HIT as the training environment for the agent.
The agent interacts with the environment by adapting the model coefficient of \eqref{eq:smago} dynamically in space in time for each element. %
We use a policy based on convolutional ANN (CNN) to account for the non-locality of turbulence.
To this end, the policy takes the flow state in a single element as input and predicts a single $C_s$ parameter for the respective element as output, as illustrated in \figref{fig:policy} for $N=5$.
The sigmoid function is applied to the output of the ANN to scale the prediction to $C_s\in[0,0.5]$.
Since only $C_s^2$ is used in Smagorinsky's model in \eqref{eq:smago}, we avoid negative predictions to preserve a monotonous relationship between the prediction and the actual eddy-viscosity introduced in the flow.
This choice was made to stay true to the original idea of Smagorinsky's model.
The upper limit is selected generously, while retaining the predictions in a sensible range to accelerate training.%
\footnote{It was verified for selected training configurations that the agent reaches similar levels of accuracy if the limits are omitted, i.e. $C_s\in(-\infty,\infty)$. However, the agent requires much more training iterations in this case than with the scaling applied.}

The states $\states$ are represented by the vector of conserved quantities $\tilde{U}$ on the coarse LES mesh, which gives a complete description of the current flow state.
Based on this state, two different sets of input quantities are investigated for the policy.
First, we use the elementwise three-dimensional momentum field as input, i.e. the vector $(\widetilde{\rho v}_1,\widetilde{\rho v}_2,\widetilde{\rho v}_3)^T$.
While this is a common choice in ANN-based turbulence modeling, the momentum field lacks invariance with respect to scaling and rotation.
To this end, we also investigate the input features proposed by Novati et al. \cite{novati2021automating}, who used as input quantities the five invariants of the velocity gradient tensor $\lambda_{\nabla v}^i$, as given by Pope in \cite{pope1975more}, to embed Galilean invariance into their policy.

It is important to stress that in contrast to previous works, our policy acts on \textit{local} inputs only.
This means that the predictions for each element are based solely on the flow state inside of this respective element.
The predictions for each individual element are thus independent from the remaining flow field and do not require any information about the global flow state.
This makes this approach more suitable for realistic applications, where global flow statistics are generally very expensive to retrieve during runtime.
Moreover, models which rely on global flow statistics as inputs cannot be evaluated on flows where these statistics cannot be obtained in a straight-forward manner, e.g. due to the geometric complexity of the computational domain or where a global statistic simply does not exist.
The value network is designed analogously to the policy network, but employs two additional fully-connected layers, comprising 16 and a single neuron, respectively, to combine the elementwise predictions for the whole flow state to a single prediction for the state-value function.
This RL design can be interpreted as a multi-agent RL approach, where each element employs its own agent, but all agents share their weights.
Given the predictions of the policy, i.e. the actions of the agent, the new state of the environment is obtained by integrating the LES equations in time for $\Delta t_{RL}$. 

As overall optimization target of the RL task, we define the error in the spectrum of turbulent kinetic energy between the instantaneous LES $E_{LES}$ and the target spectrum $\overline{E}_{DNS}$.
In this work, we used the time-averaged spectrum of a previously computed DNS.
However, the RL framework allows to impose any high-level statistic as training objective.
To this end, also data reported in the literature, experimental data or even physical properties like the $k^{-5/3}$ energy decay in the inertial range could be used directly as optimization targets. 
Given these energy spectra, we compute the mean squared error for the wavenumbers $k$ up to $k_{max}$.
Here, $k_{max}$ is the maximum wavenumber considered for optimization and thus depends on the employed resolution of the LES.
Finally, we use the exponential function to normalize the resulting reward to the range $[-1,1]$, since a normalized reward can improve the training speed.
The reward function thus reads
\begin{equation}
  \rewardfunc(\state) = 2\:\exp\left(-\frac{1}{\rewardscale\:\wavenumber_{max}}\sum_{\wavenumber=1}^{\wavenumber_{max}}\left(\frac{\overline{\turbenergy}_{DNS}(\wavenumber)-\turbenergy_{LES}(\wavenumber)}{\tfilter{\turbenergy}_{DNS}(\wavenumber)}\right)^2\right)-1,
  \label{eq:reward}
\end{equation}
with $\alpha$ as a scaling factor.
This scaling factor controls the difficulty of the training task by balancing between providing the maximum reward for any action of the policy for $\alpha \rightarrow \infty$ and demanding impossible accuracy in the spectrum to escape the negative limit of the exponential function for $\alpha \rightarrow 0$.
It seems important to stress that the energy spectra only have to be computed to evaluate the reward function during training and are not required if the trained policy is later applied in practical simulations.
The proposed formulation of turbulence modeling as an MDP is summarized in \tabref{tab:rl_turb}.

A major advantage of the broader learning paradigm of RL in comparison to the common SL approach is that RL allows to incorporate a much broader range of prior information into the training.
The optimization target for the training, which is defined via the reward function, can be based on DNS data, experimental data, physical first principles or properties of the underlying discretization.
For SL, any change in the posed learning task, e.g. changing the input and output quantities for the ANN, would require to recompute the training dataset from DNS data.
This poses significant challenges, since it requires to store and process large amounts of time-resolved DNS data.



\subsection{Computational setup}
\label{sec:relexi}


\begin{table}[htb]
  \centering
  \begin{tabular}{lrrrr}
    \toprule
    Name   & $N$ & \#Elems & $k_{max}$ & $\alpha$ \\
    \midrule
    24 DOF &   5 &   $4^3$ &         9 &      0.4 \\
    32 DOF &   7 &   $4^3$ &        12 &      0.4 \\
    36 DOF &   5 &   $6^3$ &        13 &      0.4 \\
    48 DOF &   5 &   $8^3$ &        16 &      0.4 \\
    \bottomrule
  \end{tabular}
  \caption{Investigated configurations for the LES environments. The names are derived from the number of degrees of freedom (DOF) in each direction used for the simulation, which can be computed as the number of elements in each direction times $(N+1)$.}
  \label{tab:les_configs}
\end{table}

In RL, the training of the agent is an iterative process which does not require the perfect target quantities to be known, but finds suitable targets by itself in order to fulfill the overall goal defined by the reward function.
This means that no DNS data is required to compute input and output data pairs for a training dataset.
Instead, a number of LES runs has to be computed repeatedly to sample interactions between the agent and the environment.
Thus, like in most ML methods, training requires considerable hardware resources for running the LES and training the agent on the collected experience.
In this work, we use the Relexi framework proposed in \cite{kurz2022deep,kurz2022relexi} to train the agent by using the parallel computing resources provided by today's high-performance computing (HPC) systems.
The Relexi framework implements the RL training loop by means of the TensorFlow library \cite{abadi2016tensorflow} and its RL extension TensorFlow-Agents \cite{TFAgents}.
The framework allows to couple external flow solvers efficiently on high-performance computing systems with the SmartSim library \cite{partee2021using}, which is used to manage the simulations runs and implements the communication between the main application and the external solver.
The LES environments are simulated with the HPC flow solver FLEXI \cite{krais2021flexi}.
In each training iteration, Relexi starts several FLEXI simulations to sample interactions of the current policy with the LES environments.

With FLEXI we employ a high-order discontinuous Galerkin (DG) discretization of the compressible Navier-Stokes equations using a kinetic energy preserving split formulation of the fluxes on Legendre-Gauss-Lobatto interpolation nodes for stability as discussed by Flad and Gassner in \cite{flad2017use}.
We investigate the influence of the discretization on the RL agent by computing LES at different resolutions as listed in \tabref{tab:les_configs}.
For this, we also employ two different polynomial degrees, i.e. $N=5$ and $N=7$, which results in two different discretizations. 
The maximum wavenumber used for optimization $k_{max}$ is then chosen for each case according to the respective resolution employed.
Theoretically, a minimum of $n_{ppw}=\pi$ points per wavelength is required to resolve a wavenumber accurately with a polynomial basis.
For the polynomial degrees employed here, the resolution capabilities can be estimated as $n_{ppw}\approx 4$ \cite{gassner2011comparison}.
However, to increase the difficulty of the optimization task, we choose $k_{max}$ such that $2.6 \le n_{ppw} \le 3$.
This forces the RL algorithm to optimize over all represented wavenumbers.
 
All simulations are initialized with flow states that are obtained by projecting the high-fidelity DNS solution, which is obtained a priori, onto the resolution of the respective environment using an $L_2$-projection.
The forced HIT case has a Reynolds number of $\mathrm{Re}_{\lambda}\approx 180$ with respect to the Taylor microscale.
The initial states used for training are drawn from the filtered DNS evaluated at $t_{DNS}=3,4,5,6$ and a single flow state at $t_{DNS}=8$ is kept hidden for testing, i.e. to evaluate the policy's performance on unseen data.
The LES are then simulated for $\Delta t_{end}=5$ using the FLEXI solver, while the elementwise $C_s$ parameters are updated every $\Delta t_{RL}=0.1$.
The large-eddy turnover time with approximately $t'\approx 0.7$ acts as a characteristic timescale, which results in 7 characteristic timescales simulated in each LES run.
This is considered sufficient to incorporate long-term effects into the training process.

For the RL algorithm, we use the Adam optimizer \cite{kingma2014adam} with a learning rate of $10^{-4}$ for the policy and the value network.
All configurations were trained for 5 epochs in each training iteration, except the 48 DOF configuration, which was only trained for a single epoch.
Instead of using mini-batching, the gradients were computed with respect to all sampled experience.
The weighting factor between the policy and the value estimation loss is set to 0.5 and the clipping parameter to $\ppoclipping=0.2$.
No additional regularization was used and the originally proposed entropy regularization coefficient for PPO is set to zero.
Moreover, we chose a discount factor of $\gamma=0.995$.
To obtain stochastic predictions from the deterministic policy, we sample actions from a normal distribution which is determined by using the ANN's predictions as mean and a fixed standard deviation of 0.02, which corresponds to about 10\% of the theoretical $C_s=0.17$ suggested for Smagorinsky's model in the literature.

\section{Results}
\label{sec:results}

The following section discusses the training of the agents and their inference on unseen starting states.
Moreover, the influence of invariant input features on the training success is assessed.
Lastly, the generalization abilities of the trained models to other resolutions and a higher Reynolds number are investigated.%
\footnote{The trained models and the required training files can be obtained from \url{https://github.com/flexi-framework/DRL_LES}.}

\subsection{Training}
\label{sec:results_training}

\begin{figure}[htb!]
  \centering
  \includegraphics[width=0.99\linewidth]{tikz_double_column/draft-figure2.pdf}
  \caption{Training results for the 24 DOF (left) and 48 DOF (right) test case showing (from top to bottom) the undiscounted collected return on the unseen testing data, the same for the training episodes with the minimum and maximum return indicated by the shaded area, the policy gradient loss according to \eqref{eq:clipping_loss} and the value estimation loss for the value estimation ANN. The results are obtained for either 16 or 32 full episodes used per policy update. Please note that the y-axis of the policy gradient loss is scaled differently for both cases. Since the training simulations update the predictions each $\Delta t_{RL}=0.1$ and are simulated up to $t_{end}=5$, the maximum undiscounted return is $\return_{max}=50$ considering the reward in each step is normalized to $\reward_t\in[-1,1]$.}
  \label{fig:results_training}
\end{figure}


The training took between a single day for the smallest and up to 5 days for the largest cases on the HAWK supercomputer at the High-Performance Computing Center Stuttgart (HLRS).
The training was performed using up to 1024 CPU cores for the simulations and a single GPU node for model execution and training.
For more details on the hardware configuration, the reader is referred to \cite{kurz2022deep}.
The training behavior of the 24 DOF and the 48 DOF configurations is shown exemplarily in \figref{fig:results_training}, since these configurations constitute the lowest and highest employed resolution, respectively.
Both cases were trained once with 16 and with 32 episodes per parameter update.
For all cases, the collected return in the simulation is negative for the randomly initialized policy.
Starting from this initial random policy, the collected return increases during training until the return converges and plateaus just under the maximum undiscounted return of $\return_{max}=50$.
The maximum return follows from the 50 rewards collected during a simulation and the normalization of the reward function in \eqref{eq:reward} that guarantees $r_t\in[-1,1]$.
The convergence of the collected reward indicates that the RL algorithm has found a local optimum with its current policy.
Generally, the gradient estimator used for the gradient ascent algorithm should be more accurate if the amount of sampled episodes per parameter update is increased.
This can speed up the training, since a better approximated gradient can lead to more efficient parameter updates and thus reduce the overall training iterations needed for convergence.
This was indeed observed consistently for all investigated configurations.
Interestingly, the larger 48 DOF case requires less training iterations for convergence compared to the 24 DOF case.
Moreover, the 48 DOF case exhibits less variance in the return sampled in the training runs.
We attribute this reduced variance to two different factors.
First, the influence of the eddy-viscosity model on the overall flow decreases with increasing resolution, since the model accounts for a diminishing amount of unresolved kinetic energy in the flow.
Secondly, a single flow state contains more elements for the larger cases.
Since the policy trains on elementwise data, the same amount of episodes thus provides more training samples for the larger configurations.
The overall reduced variance in the training process then might cause more sample-efficient and faster optimization.
However, since the amount of required iterations for convergence did not decrease consistently with increasing resolution, the faster training might also be simply caused by the stochasticity of the training process.

The last row in \figref{fig:results_training} shows the loss of the value estimation ANN, which is trained to approximate the expected future return starting from a given state of the environment.
The value ANN is initialized with random weights and thus gives a poor estimate of the expected return in the beginning of the training, which results in a high initial loss.
The loss then decreases as the value ANN learns a first sensible estimate of the expected return.
Since the policy then starts to improve, i.e. to collect more reward, also the expected return and thus the training targets of the value ANN change.
Therefore, the value estimation loss increases as the policy improves, since the value ANN has to catch up constantly with the policy.
Once the return reaches a plateau, the expected return as target quantity for the training of the value ANN becomes more stable and the value estimation loss decreases again.


\subsection{Inference}
\label{sec:inference}

\begin{figure*}[htb!]
  \centering
  \includegraphics[width=\textwidth]{tikz_double_column/draft-figure3.pdf}
  \caption{Results for the trained RL models (from left to right) in the 24 DOF, 32 DOF, 36 DOF and 48 DOF configuration averaged over $t\in[10,20]$ for an LES initialized with the unseen testing sample. Reported are (from top to bottom) the averaged energy spectra over the wavenumbers $k$, the relative error of the energy spectra with respect to the DNS solution, the distributions $\mathcal{P}(\cdot)$ of the velocity fluctuations, and the distribution of the predicted $C_s$ parameters. The results for the underlying DNS solution as well as an implicitly modeled LES (iLES), the SSM with $C_s=0.17$ and the DSM are shown for comparison. The shaded area for the DNS energy spectrum indicates the maximum and minimum amplitudes observed for each mode. The wavenumber that is resolved by the discretization with 4 points per wavelength is shown dashed and the maximum wavenumber used for optimization $k_{max}$ is indicated with a solid black line.}
  \label{fig:spectra_n5}
\end{figure*}

The performance of the trained RL models is evaluated based on an LES, which is initialized with the unseen testing state and is computed for $t_{end}=20$.
All results reported in the following are obtained and averaged over the timeframe of $t\in[10,20]$.
Since the models are trained only on simulations with $t_{end}=5$, this allows to assess the long-term behavior of the models and whether the simulation time used during training was sufficient.
The results in \figref{fig:spectra_n5} demonstrate that the RL models are indeed long-term stable.
To assess the accuracy of the RL models, they are compared to the SSM and DSM as well as the implicit model.
As could be expected, the implicitly modeled LES exhibits a buildup of energy in the upper wavenumbers due to lacking dissipation.
The SSM with $C_s=0.17$ on the other hand introduces too much dissipation and thus fails to preserve the wavenumbers near the resolution limit of the underlying numerical scheme.
The RL model clearly outperforms both models for all considered configurations by matching the target spectrum almost perfectly up to and even beyond the discretization's resolution limit of around 4 points per wavelength.
The advantage of the RL models is more pronounced for the small cases, where the turbulence model has more impact on the overall flow.
Interestingly, the errors for the different wavenumbers are distributed more evenly for the RL model.
This might stem from the objective of minimizing the squared error in the energy spectrum in order to increase the reward as given in \eqref{eq:reward}. 

The DSM, however, reproduces the DNS spectra with similar accuracy as the RL model.
This is to be expected, since the DSM is known to provide near perfect results for the HIT case, as long as the test filter is situated in the inertial range.
The RL model achieves a similar level of accuracy, i.e. a near optimal policy, without having access to the additional filtering procedure of the DSM.
It is interesting that the most notable difference between the DSM and RL is for the 24 DOF case, where the RL agent provides a significantly better energy spectrum.
A likely explanation for this loss in accuracy of the DSM is that the LES resolution in the 24 DOF case is too coarse for the test filtering to occur in the scale-similar region.
Thus, no meaningful information is provided to the DSM procedure.
The RL model, however, seems to compensate for this lack of resolution and reproduces the DNS spectrum with very good accuracy.
It is important to stress here once again, that the underlying forcing of the test case prescribes the overall energy budget of the simulation.
Therefore, errors in the high wavenumbers might influence the energy contained in the low wavenumber and vice versa.
This stresses the capabilities of the RL models even more, which have learned to interact with this forcing such that the energy spectrum fits the prescribed one based on local information of the flow field only.

\begin{figure*}
  \centering
  \includegraphics[width=\textwidth]{tikz_double_column/draft-figure4.pdf}
  \caption{Comparison between the RL models trained with the local momentum field $\widetilde{\rho v}_i$ and with the invariants of the velocity gradient tensor $\lambda_{\nabla v}^i$ as inputs for the (from left to right) 24 DOF, 32 DOF, 36 DOF and 48 DOF configuration. The top row shows the undiscounted return collected during training with the shaded area indicating the episodes with the maximum and minimum return. The lower row shows the energy spectra in comparison to the DNS solution. The shaded area in the spectra indicates the maximum and minimum observed energy contained in the respective wavenumber during the DNS. The wavenumber that is resolved by the discretization with 4 points per wavelength is shown dashed and the maximum wavenumber used for optimization $k_{max}$ is indicated with a solid black line.}
  \label{fig:invariants}
\end{figure*}

It seems important to stress again that the energy spectra are the optimization target for the training and thus might provide only limited insight into the model's overall ability to reproduce the required turbulent statistics.
To this end, the distributions of the velocity fluctuations produced by the different models are investigated as additional important measure of the models' performance.
The differences between the models observed here are consistent with the obtained energy spectra.
Since additional dissipation tends to reduce velocity fluctuations, the SSM exhibits the narrowest and the implicit LES the broadest distribution.
For all cases, the velocity fluctuations produced by the RL model appear to be balanced between both effects, since they follow the DNS distribution more closely than the SSM for small fluctuations in the center of the distribution, but do not exhibit the overpronounced tails observed for the implicit LES.
However, the RL model produces an unsymmetrical distribution of velocity fluctuations for the 36DOF case.
This behavior was not observed for any other configuration or model and its origin is still subject of on-going investigations.
It is also unclear whether this behavior emerges since the symmetry condition has to be learned by the model implicitly during the training and might be not strict enough or whether this effect emerges from the long-term interactions between the agent and the forcing method.

The distributions of $C_s$ in the bottom row of \figref{fig:spectra_n5} show qualitatively similar results for all cases.
Most predictions are close to zero with an decreasing amount of higher values.
The overall range seems to strongly depend on the resolution, since the agent exploits almost all of its available action space for the 24 DOF case performing actions near the prescribed maximum of 0.5.
In contrast, the largest $C_s$ prediction for the 48 DOF case does not exceed 0.3.
The predictions' mean thus decreases for increasing resolutions as is consistent with the understanding of an increased LES resolution on the Smagorinsky model.
This difference might stem either from the policy itself, i.e. the policies learned indeed different distributions, or from the input data of the respective resolutions, which might exhibit different distributions.
Nonetheless, these results show the flexibility and capability of the RL training approach to incorporate physical constraints into the model through the choice of the input features.
We did not adapt the expressivity of the ANN between the two input selections, which might increase the performance.

Sarghini et al. \cite{sarghini2003neural} and Maulik et al. \cite{maulik2021deploying} reported a speedup by applying ANN instead of the computationally expensive DSM.
To this end, we compared the computational time required to evaluate the policy and the dynamic procedure of the DSM on a single CPU core for the 48 DOF case.
We found that the time required was comparable for both cases with the RL policy requiring around 10 per cent more time on our hardware.
These results are encouraging, since we did not perform any optimizations of the policy in terms of computational efficiency or model size and did not use GPU acceleration for this comparison, which improves the performance of the RL-based policy significantly.

\subsection{Input Features}
\label{sec:results_features}

In a next step, the models trained on the local momentum field as inputs are compared to the models trained on the five invariants of the velocity gradient tensor $\lambda_{\nabla u}^i$.
The results in \figref{fig:invariants} indicate that again all models successfully improve during training.
However, the training is generally slower and less stable than the former models.
As a result, the final models using the invariants as inputs still partly improve over the analytical models, but perform worse than the models using the momentum field, especially in the 36 DOF case.
This indicates that it is generally harder to learn a sensible policy from the invariants.
To investigate this further, we increased the training time.
While the models always seemed to improve to some degree, even with double the amount of training iterations the gap to former models stayed quite substantial.

We attribute this to the different distributions of the input quantities.
For the considered HIT test case, the velocity fluctuations have zero mean and a root-mean-squared (RMS) magnitude of unity by construction.
The fluctuations are thus approximately normally distributed with zero mean and unit variance.
This is the optimal distribution for input quantities in machine learning, which typically has to be achieved by normalizing the inputs accordingly.
Since the velocity fluctuations are intrinsically linked to the energy budget in the simulation and are thus constraint by the forcing, the agent's actions have only relatively limited impact on the distribution of velocity fluctuations, as already shown in \figref{fig:spectra_n5}.
In contrast, the invariants of the velocity gradient tensor typically span orders of magnitude.
Moreover, the computation of the gradients, and thus the distribution of the invariants, differ widely depending on the employed numerical discretization.
For DG, the gradients are intrinsically discontinuous across element faces and are also known to produce large gradients at the element faces for underresolved turbulence.
This is especially problematic for the initial states, which are obtained by projecting the DNS flow field onto the DG basis with respective LES resolution.
This projection causes large gradients and thus large values for the invariants, which makes it hard to normalize them to a tamer distribution.
We thus assume that the problems in the training stem from the gradient computation of the DG method, for which the gradients exhibit a complex distribution and the invariants computed from it span orders of magnitude, which makes training more difficult for the agent.


\subsection{Generalization to other Resolutions}
\label{sec:results_generalize}

\begin{figure}
  \centering
  \includegraphics[width=0.99\linewidth]{tikz_double_column/draft-figure5.pdf}
  \caption{Results for the RL model trained on the 48 DOF evaluated on the 36 DOF case (left) and vice versa (right). Given are the energy spectra over the wavenumbers $k$ (top), the error of the spectra in comparison to the DNS solution (center) and the distribution of the predicted $C_s$ parameters (bottom). The shaded area indicates the maximum and minimum observed energy contained in the respective wavenumber during the DNS. The wavenumber that is resolved by the discretization with 4 points per wavelength is shown dashed and the maximum wavenumber used for optimization $k_{max}$ is indicated with a solid black line.}
  \label{fig:generalization_resolution}
\end{figure}

To demonstrate that the trained models can generalize to different resolutions, the model trained on the 48 DOF resolution is evaluated on the 36 DOF configuration and vice versa.
This allows to assess how well the trained models can be transferred to LES cases with either more or less resolution.
The results shown in \figref{fig:generalization_resolution} demonstrate that the trained models can also provide stable and accurate results in LES with different resolutions.
This is especially remarkable, since the policy's field of vision for the policy shrinks with increasing resolution due to the elementwise input and output quantities.
The RL model trained natively on the 48 DOF case shows a slight increase in energy in the higher wavenumbers, while the 36 DOF model seems to be slightly more dissipative.
However, the overall errors in the energy spectrum appear to be comparable for both cases.
The classical turbulence models are not shown for clarity.
However, since both RL models provide almost identical energy spectra, they still outperform the SSM and implicit LES model for both resolutions, while matching the performance of the DSM.

Also, the distribution of the models' predictions are almost identical for both cases and thus do not appear to change depending on the LES resolution.
The predictions of the 36 DOF model exhibit a much wider tail, with a maximum prediction of around $C_{s,max}=0.4$.
In contrast, the predictions of the trained 48 DOF model do not exceed $C_{s,max}=0.3$.
Interestingly, the models are still able to reproduce the target energy spectrum despite the deviations in their policies.
It is plausible to assume that the models will generalize even better, if they are trained on a variety of different resolutions, instead of only a single one.
These pronounced differences in the learned policies indicate that the distribution of predictions is not only induced by the input features but is a characteristic property of the policy trained on the respective resolution and the employed discretization.
This again demonstrates that the different discretizations induce different implicit LES filters, which again require different policies to match the underlying energy spectrum.
Thus, the proposed framework allows to develop discretization-adapted turbulent models for implicit LES.

\subsection{Generalization to other Reynolds Numbers}
\label{sec:results_generalize_re}

In a final step, we demonstrate that the trained RL policy is able to generalize to higher Reynolds numbers.
For this, the trained agents for the different resolutions are applied to a HIT flow at a Reynolds number of $Re_{\lambda}\approx 240$, which is considerably higher than the Reynolds number $Re_{\lambda}\approx 180$ used for training.
Analogously to \secref{sec:inference}, the LES were initialized from filtering the DNS flow field at a random point in time to the required resolution.
The LES was then advanced in time for $t_{end}=20$ and the results were averaged over the timeframe of $t\in [10,20]$ to investigate the long term effects of the model onto the flow.

\begin{figure*}[htb!]
  \centering
  \includegraphics[width=\textwidth]{tikz_double_column/draft-figure6.pdf}
  \caption{Results for the RL models trained on $Re_{\lambda}\approx180$ evaluated on a HIT flow with $Re_{\lambda}\approx240$. Reported are (from top to bottom) the averaged energy spectra over the wavenumbers $k$, the relative error of the energy spectra with respect to the DNS solution, the distributions $\mathcal{P}(\cdot)$ of the velocity fluctuations, and the distribution of the predicted $C_s$ parameters. The results for the underlying DNS solution as well as an implicitly modeled LES (iLES), the SSM with $C_s=0.17$ and the DSM are shown for comparison. The shaded area for the DNS energy spectrum indicates the maximum and minimum amplitudes observed for each mode. The wavenumber that is resolved by the discretization with 4 points per wavelength is shown dashed and the maximum wavenumber used for optimization $k_{max}$ is indicated with a solid black line.}
  \label{fig:generalization_re}
\end{figure*}

The results in \figref{fig:generalization_re} indicate that the trained models can indeed generalize to flows at higher Reynolds numbers.
Most importantly, the RL models still provide long-term stable simulations.
The RL models show similar behavior as for the Reynolds number seen during training. 
For the 32 DOF, 36 DOF and 48 DOF cases the RL models is able to reproduce the energy spectrum more accurately than the implicit model and the SSM, but with similar accuracy as the DSM. 
Interestingly, the DSM and RL models exhibit a similar buildup of energy near the cutoff wavenumber, which might indicate that these models lack sufficient dissipation.
Moreover, the RL model sill outperforms the other models and especially the DSM for the 24 DOF simulation, where the modeling assumptions of the SSM and DSM most probably do not hold.

These results are very promising, since they indicate that the trained RL policies are able to extrapolate to other Reynolds numbers (at least to a moderate extent).
The trained policies are thus able to generalize to higher Reynolds number flows as well as other LES resolutions, while matching or even improving on the performance of the DSM, which is known to provide outstanding results for HIT flows.

\section{Conclusion}
\label{ss: conclusion}

% summary of approach
This paper presents a methodology to evaluate the effectiveness of evasions and its application to studying PDF malware scanners.
Our implementation of the methodology, the Chameleon framework, automatically generates and enriches malicious documents with one or multiple evasions.
We use these documents for an in-depth study of \nbAnalyzers{} PDF scanners and how they are affected by evasions.
More broadly, our methodology can also be used for studying evasions of other malware types, e.g., malicious executables.

% main take-aways
The overall result of our study is cause for concern.
We show that the studied evasions are surprisingly effective in fooling state-of-the-art scanners.
In particular by combining evasions, attackers can bypass modern defenses in both static and dynamic scanners.
Moreover, we find huge variations across scanners, enabling targeted attacks based on evasions picked specifically for a targeted scanner.
All these findings are a call to arms for future work on anti-evasion techniques.

Our work will support future efforts toward improving malware scanners in several ways.
First, the results of our study help security vendors to better understand their vulnerability to specific evasions and to focus their attention on mitigating the most effective evasions.
Second, we are releasing the corpus of malicious, evasive documents generated by Chameleon as a ready-to-use benchmark.
We are in contact with several developers of PDF scanners, and some of them, e.g., SploitGuard and SAFE-PDF, have already used our benchmark to test and improve their security scanners.
Finally, the Chameleon framework provides a basis for expanding the set of benchmarks by incorporating future evasions, exploits, and payloads.


\section*{Acknowledgment} %
The research presented in this paper was funded by Deutsche Forschungsgemeinschaft (DFG, German Research Foundation) under Germany's Excellence Strategy - EXC 2075 - 390740016.
The authors gratefully acknowledge the support and the computing time on "Hawk" provided by the HLRS through the project "hpcdg" and the support by the Stuttgart Center for Simulation Science (SimTech).
\appendix

\newpage
\appendix
% \section{Appendix}
\section{Ablation Study}
\label{appendix:ablation}
%(2) We can clearly observe a tradeoff between the degree of freedom for manipulation and attack success rate. For example, we observe a small drop in the attack success rate for answer targeted attack compared to position targeted attack, due to the fact that we put more constraints to ensure pre-specified answer targets unchanged in the optimization process. Similarly, the dependency tree constraints turn out to be more strong and harsh constraints on the adversarial sentences, thus achieving higher language quality at the cost of  attack success rate. 
%(2)
%(3) \boxin{How to say because our transfer based blackattack does not beat AddSent because it is input-agnoistic.? while ours are more model-specific?}  (4) BERT based sentiment classifier is more vulnerable than standard sentiment classifier, while BERT based QA model is more robust and harder to attack than the widely-used QA model.

\subsection{Autoencoder Selection}
As an ablation study, we compare the standard LSTM-based autoencoder with our tree-based autoencoder. 

\begin{table}[htp!]\small \setlength{\tabcolsep}{5pt}
\centering
\caption{Ablation study on posistion targeted attack capability against QA. The lower EM and F1 scores mean the better attack success rate. \advcodecsent and \advcodecword respectively refer to \advcodecsent and \advcodecword. Adv(seq2seq) refers to \advcodec that uses LSTM-based seq2seq model as text autoencoder.}
 \label{WhiteboxQAseq2seq}
\begin{tabular}{ccccc}
\toprule
% \multirow{2}{*}{Model} & & \multirow{2}{*}{Origin} & \multicolumn{2}{c}{w/ Tree Decoder} & w/o Tree Decoder  \\
% \cmidrule(lr){4-5}   \cmidrule(lr){6-6}
  & Origin & {\advcodecsent} & {\advcodecword} & Adv(seq2seq)  \\
\midrule
EM & 60.0 & 29.3     & \textbf{15.0}  & 51.3  \\
 F1 & 70.6 &  34.0   & \textbf{17.6}  &      57.5 \\
      \bottomrule
\end{tabular}
% \vspace{-3mm}
\end{table}


\begin{table*}[htp!]\small \setlength{\tabcolsep}{7pt}
 \begin{minipage}[htp!]{0.48\linewidth}
\centering
\caption{Blackbox Attack Success Rate after inserting the whitebox generated adv sentence to different positions for BERT-classification.  }
 \label{ablationClassification}
\begin{tabular}{ccccc}
\toprule
Method & & Back & Mid & Front \\
\midrule
\multirow{2}{*}{\advcodecword} & \footnotesize{target}   & 0.739   & 0.678  & \textbf{0.820} \\
      & \footnotesize{untarget} & 0.817 & 0.770  & \textbf{0.878}           \\
      \midrule
\multirow{2}{*}{\advcodecsent} & \footnotesize{target}   & \textbf{0.220}   & 0.174  & 0.217 \\
      & \footnotesize{untarget} & 0.531 & 0.504  & \textbf{0.532}           \\
        \bottomrule
\end{tabular}
\vspace{-0.2cm}
\end{minipage}
\quad
\begin{minipage}[htp!]{0.48\linewidth}
\centering
\caption{Blackbox Attack Success Rate after inserting the whitebox generated adversarial sentence to different positions for BERT-QA.}
 \label{ablationQA}
\begin{tabular}{ccccc}
\toprule
Method & & Back & Mid & Front \\
\midrule
\multirow{2}{*}{\advcodecword}  & EM &  32.3    & 39.1    & \textbf{31.9}  \\
      & F1 & 36.4   & 43.4     & \textbf{36.3}   \\   
      \midrule
\multirow{2}{*}{\advcodecsent} & EM & 47.0   & 51.3     & \textbf{42.4}           \\
      &  F1 & 52.0     & 56.7         & \textbf{47.0}          \\
        \bottomrule
\end{tabular}
\vspace{-0.2cm}
\end{minipage}
\end{table*}

\textbf{Tree Autoencoder.} 
In the whole experiments, we used Stanford TreeLSTM as tree encoder and our proposed tree decoder together as tree autoencoder. We trained the tree autoencoder on yelp dataset which contains 500K reviews. The model is expected to read a sentence, map the sentence in a latent space and reconstruct the sentence from the embedding along with the dependency tree structure in an unsupervised manner. The model uses 300-d vectors as hidden tree node embedding and is trained for 30 epochs with adaptive learning rate and weight decay. After training, the average reconstruction loss on test set is 0.63.

\textbf{Seq2seq Autoencoder.} We also evaluate the standard LSTM-based architecture (seq2seq) as a different autoencoder in the \advcodec pipeline. For the seq2seq encoder-decoder, we use a bi-directional LSTM as the encoder \citep{Hochreiter1997LongSM} and a two-layer LSTM plus soft attention mechanism over the encoded states as the decoder \citep{Bahdanau2015NeuralMT}. With 400-d hidden units and the dropout rate of 0.3, the final testing reconstruction loss is 1.43.

The comparison of the whitebox attack capability  against a well-known QA model BiDAF is shown in Table \ref{WhiteboxQAseq2seq}. We can see seq2seq based \advcodec fails to achieve good attack success rate. Moreover, because the vanilla seq2seq model does not take grammatical constraints into consideration and has higher reconstruction loss, the quality of generated adversarial text cannot be ensured.

\subsection{Ablation Study on BERT Attention}
\label{sec:ablation}
To further explore how the location of adversarial sentences affects the attack success rate, we conduct the ablation experiments by varying the position of appended adversarial sentence. We generate the adversarial sentences from the whitebox BERT classification and QA models. Then we inject those adversaries into different positions of the original paragraph and test in another blackbox BERT with the same architecture but different parameters. The results are shown in Table \ref{ablationClassification} and \ref{ablationQA}. We see in most time appending the adversarial sentence at the beginning of the paragraph achieves the best attack performance. Also the performance of appending the adversarial sentence at the end of the paragraph is usually slightly weaker than front. This observation suggests that the BERT model might pay more attention to the both ends of the paragraphs and tend to overlook the content in the middle.


% \textbf{Ablation Study.} \boxin{change the language here (same as sec 4.1)} To further explore how the appended location will impact the attack success rate, we conduct the ablation experiment by varying the position of appended adversarial sentence and the results are shown in table \ref{ablationQA}. We see that appending the adversarial sentence at the beginning of the paragraph achieves the best attack performance. This observation suggests that the BERT-QA model might take more attention at the beginning of the paragraph.


\subsection{Attack Settings}
% \begin{algorithm}[b]
%   \caption{Algorithm of \advcodec generating adversarial examples } \label{algo}
%   \begin{algorithmic}[1]
%     \Procedure{AdvCodec}{$x,s$} \Comment{$x$: initial seed, $s$: corresponding dependency tree}
%     \State $z := \mathcal{E}(x, s)$ \Comment{$\mathcal{E}$: encoder of \advcodec, $z$: context vector}
%     \State $z^* = 0$ \Comment{$z^*$: perturbation on context vector}
%     \State $z' := z + z^*$ \Comment{$z'$: perturbed context vector}
%     \State $y := \mathcal{G}(z', s)$ \Comment{$\mathcal{G}$: decoder of \advcodec, $y$: adversarial sentence}
%   % \State $Z(y) :=$ the logits of the model output
%     \State $f(z') :=$ the objective function to attack the targeted model
%     \While{$y$ does not achieve targeted attack} 
%       \State  update $z^*$ by gradient descent over objective function $f(z')$
%     \EndWhile\label{euclidendwhile}
%     \State \textbf{return} $y$
%     \EndProcedure
%   \end{algorithmic}
% \end{algorithm}
We use Adam \citep{Adam} as the optimizer, set the learning rate to 0.6 and the optimization steps to 100. We follow the \citet{cw} method to find the suitable parameters in the object function (weight const $c$ and confidence score $\kappa$) by binary search. 

% We also include our attack algorithm via pseudo-code in Algorithm \ref{algo}.


% \iffalse
% \subsection{Untargeted scatter attack on QA}

% We tried the scatter attack on QA, however, the targeted attack success rate is not satisfactory. It turns out QA systems highly rely on the relationship between questions and contextual clues, which is hard to break when setting an arbitrary token to a target answer. This is also why we use some preliminary approaches to creating a similar fake context when initializing QA appended sentence. 

% We also performed the untargeted scatter attack on QA. The results are shown in table \ref{WhiteboxQAScatter}. We insert 30 random tokens (but  no more than $1/3$ the total words of the paragraph) over the paragraph, optimize and find the adversarial tokens that can cause model output the wrong answers in the untargeted manner.  We can see the untargeted scatter attack can also achieve a higher untargeted attack success rate than \citet{jia-liang-2017-adversarial}.

% \begin{table*}[htp!]\small \setlength{\tabcolsep}{5pt}
% \centering
% \caption{Whitebox attack results on BERT-QA in terms of exact match rates and F1 scores by the official evaluation script. The lower EM and F1 scores mean the better attack success rate.}
%  \label{WhiteboxQAScatter}
% \begin{tabular}{ccccccccc}
% \toprule
% \multirow{2}{*}{Model} & & \multirow{2}{*}{Origin} & \multicolumn{2}{c}{Position Targeted Attack} & \multicolumn{2}{c}{Answer Targeted Attack} & \multicolumn{2}{c}{Untargeted Attack} \\
% \cmidrule(lr){4-5} \cmidrule(lr){6-7} \cmidrule(lr){8-9}
%  & & & {\advcodecsent} & {\advcodecword}  & {\advcodecsent} & {\advcodecword} & AddSent & Adv(scatter)\\
% \midrule

% \multirow{2}{*}{BERT}  & EM & 81.2 &49.1       & \textbf{29.3}           & 50.9                    & 43.2                    & 46.8  & 34.3   \\
%       & F1 & 88.6 & 53.8          & \textbf{33.2}         & 55.2                   & 47.3                  & 52.6  & 49.7 \\
% %      & $\Delta \text{F1}$ & $=$ & 34.8  & \textbf{55.4} & 33.4 & 41.3 & 36.0 \\
% %       \midrule
% % \multirow{2}{*}{BiDAF} & EM & 60.0 & 29.3  	          & \textbf{15.0}             & 30.2                    & 21.0                      & 25.3    \\
% %       & F1 & 70.6 &  34.0   & \textbf{17.6}         & 34.4                  & 23.6                  & 32.0 \\
% \bottomrule
% \end{tabular}
% \end{table*}
% \fi

\subsection{Heuristic Experiments on choosing the adversarial seed for QA}
\label{appendix:heuristic}

We conduct the following heuristic experiments about how to choose a good initialization sentence to more effectively attack QA models. Based on the experiments we confirm it is important to choose a sentence that is semantically close to the context or the question as the initial seed when attacking QA model, so that we can reduce the number of iteration steps and more effectively find the adversary to fool the model. Here we describe three ways to choose the initial sentence, and we will show the efficacy of these methods given the same maximum number of optimization steps.

\textbf{Random adversarial seed sentence.}
Our first trial is to use a random sentence (other than the answer sentence), generate a fake answer similar to the real answer and append it to the back as the initial seed.

\textbf{Question-based adversarial seed sentence.}
% question words in a question , paragraph pair <p, q> 
We also try to use question words to craft an initial sentence, which in theory should gain more attention when the model is matching characteristic similarity between the context and the question. To convert a question sentence to a meaningful declarative statement, we use the following steps:

In step 1, we use the state-of-the-art semantic role labeling (SRL) tools \citep{He2017DeepSR} to parse the question into verbs and arguments. A set of rules is defined to remove the arguments that contain interrogative words and unimportant adjectives, and so on. In the next step, we access the model's original predicted answer and locate the answer sentence. We again run the SRL parsing and find to which argument the answer belongs. The whole answer argument is extracted, but the answer tokens are substituted with our targeted answer or the nearest words in the GloVe word vectors \citep{Pennington2014GloveGV} (position targeted attack) that is also used in the QA model. In this way, we craft a fake answer that shares the answer's context to solve the compatibility issue from the starting point. Finally, we replace the declarative sentence's removed arguments with the fake argument and choose this question-based sentence as our initial sentence.

\textbf{Answer-based adversarial seed  sentence.}
We also consider directly using the model predicted original answer sentence with some substitutions as the initial sentence. To craft a fake answer sentence is much easier than to craft from the question words. Similar to step 2 for creating
question-based initial sentence, we request the model's original predicted answer and find the answer sentence. The answer span in the answer sentence is directly substituted with the nearest words in the GloVe word vector space to avoid the compatibility problem preliminarily.

\textbf{Experimental Results.} We tried the above initial sentence selection methods on \advcodecword and perform position targeted attack on BERT-QA given the same maximum optimization steps. The experiments results are shown in table \ref{WhiteboxQAHeuristic}. From the table, we find using different initialization methods will greatly affect the attack success rates. Therefore, the initial sentence selection methods are indeed important to help reduce the number of iteration steps and fastly converge to the optimal $z^*$ that can attack the model.

\begin{table*}[htp!]\small \setlength{\tabcolsep}{5pt}
\centering
\caption{Whitebox attack results on BERT-QA in terms of exact match rates and F1 scores by the official evaluation script. The lower EM and F1 scores mean the better attack success rate.}
 \label{WhiteboxQAHeuristic}
\begin{tabular}{ccccccc}
\toprule
\multirow{2}{*}{Model} & & \multirow{2}{*}{Origin} & \multicolumn{3}{c}{Position Targeted Attack}  & \multicolumn{1}{c}{Baseline} \\
\cmidrule(lr){4-6} \cmidrule(lr){7-7}
 & & & Random & Question-based  & Answer-based  & AddSent\\
\midrule

\multirow{2}{*}{BERT}  & EM & 81.2 & 67.9       & \textbf{29.3}           & 50.6                               & 46.8   \\
      & F1 & 88.6 & 74.4         & \textbf{33.2}         & 55.2    & 52.6   \\
\bottomrule
\end{tabular}
\end{table*}

%\subsection{Conclusions}
% In addition to the general adversarial evaluation framework \advcodec, this paper also aims to explore several scientific questions: 1)  Since \advcodec allows the flexibility of manipulating at different levels of a tree hierarchy, which level is more attack effective and which one preserves better grammatical correctness? 2) Is it possible to achieve the targeted attack for general NLP tasks such as sentiment classification and QA, given the limited degree of freedom for manipulation? 3) Is it possible to perform a blackbox attack for many  NLP tasks? 4) Is BERT robust in practice? 
% 5) Do these adversarial examples affect human reader performances? 
% %\boxin{I think the above question is readers caring more. 5) Are human readers more sensitive to an appended adversarial sentence or scatter of added words?

% To address the above questions, we generate adversarial text against different models of sentiment classification and QA in each encoding scenario. Compared with the state-of-the-art adversarial text generation methods, our approach achieves significantly higher untargeted and \emph{targeted} attack success rate. In addition, we perform both whitebox and transferability-based blackbox settings to evaluate the model vulnerabilities. 
% Within each attack setting, we quantitatively evaluate the attack effectiveness of different attack strategies, including appending an additional adversarial sentence and adding scatter of adversarial words to a paragraph.
% To provide thorough adversarial text quality assessment, we also perform 7 groups of human studies to evaluate the quality of the generated adversarial text. % Compared with the baselines methods, and whether a human can still get the ground truth answers for these tasks based on adversarial text.

% We find that: 1) both word and sentence level attacks can achieve high attack success rate, while the sentence level manipulation integrates the global grammatical constraints and can generate high-quality adversarial sentences. 2) various targeted attacks on general NLP tasks are possible (\textit{e.g.}, when attacking QA, we can ensure  the target to be a specific answer or a specific location within a sentence); 3) the transferability based blackbox attacks are successful in NLP tasks. Transferring adversarial text from stronger models (in terms of performances) to weaker ones is more successful; 4)  Although BERT has achieved state-of-the-art performances, we observe the performance drops are also more substantial than other models when confronted with adversarial examples, which indicates BERT is not robust enough under the adversarial settings.
% %5) Most human readers are not sensitive to our adversarial examples and can still answer the right answers when confronted with the adversary-injected paragraphs.

% Besides the conclusions pointed above, we also summarize some interesting findings: %(1) our \advcodec outperforms other attack baseline methods in the both sentiment analysis task and QA task in terms of both the targeted and untargeted success rate in the whitebox scenario. 
% (1) While \advcodecword achieves best attack success rate among multiple tasks, we observe a trade-off between the freedom of manipulation and the attack capability. For instance, \advcodecsent has dependency tree constraints and becomes more natural for human readers than but less effective to attack models than \advcodecword. Similarly, the answer targeted attack in QA has fewer words to manipulate and change than the position targeted attack, and therefore has slightly weaker attack performances.
% % (2) Scatter attack is as effective as concat attack in sentiment classification task but less successful in QA, because QA systems make decisions highly based on the contextual correlation between the question and the paragraph, which makes it difficult to set an arbitrary token as our targeted answer.
% (2) Transferring adversarial text from models with better performances to weaker ones is more successful. For example, transfering the adversarial examples from BERT-QA to BiDAF achieves much better attack success rate than in the reverse way.
% (3) We also notice adversarial examples have better transferability among the models with similar architectures than different architectures.
% (4) BERT models give higher attention scores to the both ends of the paragraphs and tend to overlook the content in the middle, as shown in \S \ref{sec:ablation} ablation study that adding adversarial sentences in the middle of the paragraph is less effective than in the front or the end.

% To defend against these adversaries, here we discuss about the following possible methods and will in depth explore them in our future works: 
% (1) \textbf{Adversarial Training} is a practical methods to defend against adversarial examples. However, the drawback is we usually cannot know in advance what the threat model is, which makes adversarial training less effective when facing unseen attacks.
% (2) \textbf{Interval Bound Propagation} (IBP) \citep{Dvijotham2018TrainingVL} is proposed as a new technique to theoretically consider the worst-case perturbation. Recent works \citep{Jia2019CertifiedRT,Huang2019AchievingVR} have applied IBP in the NLP domain to certify the robustness of models. (3) \textbf{Language models} including GPT2 \citep{Radford2019LanguageMA} may also function as an anomaly detector to probe the inconsistent and unnatural adversarial sentences.


\section{Experimental Settings}
\label{appendix:setup}
\subsection{Sentiment Classification Model}
 \textbf{BERT.} We use the 12-layer BERT-base model \footnote{https://github.com/huggingface/pytorch-pretrained-BERT} with 768 hidden units, 12 self-attention heads and 110M parameters. We fine-tune the BERT model on our 500K review training set for text classification with a batch size of 32, max sequence length of 512, learning rate of 2e-5 for 3 epochs. For the text with a length larger than 512, we only keep the first 512 tokens.
 
 
 \textbf{ Self-Attentive Model (SAM).} We choose the structured self-attentive sentence embedding model \citep{nfc512} as the testing model, as it not only achieves the state-of-the-art results on the sentiment analysis task among other baseline models but also provides an approach to quantitatively measure model attention and helps us conduct and analyze our adversarial attacks. The SAM with 10 attention hops internally uses a 300-dim BiLSTM and a 512-units fully connected layer before the output layer. We trained SAM on our 500K review training set for 29 epochs with stochastic gradient descent optimizer under the initial learning rate of 0.1.
 
 \subsection{Sentiment Classification Attack Baseline}
 \textbf{Seq2sick} \citep{seq2sick} is a whitebox projected gradient method combined with group lasso and gradient regularization to craft adversarial examples to fool seq2seq models. Here, we define the loss function as $ L_{target} = \max\limits_{k \in Y} \left\{z^{\left(k\right)} \right\} - z^{\left(t\right)} $ to perform attack on sentiment classification models which was not evaluated in the original paper. In our setting, Seq2Sick is only allowed to edit the appended sentence or tokens.
 
 \textbf{TextFooler} \citep{TextFooler} is a simple but strong black-box attack method to generate adversarial text. Here, TextFooler is also only allowed to edit the appended sentence.

\subsection{QA Model}
\textbf{{BiDAF}.} Bi-Directional Attention Flow (BIDAF) network\citep{seo2016-bidirectional} is a multi-stage hierarchical process that represents the context at different levels of granularity and uses bidirectional attention flow mechanism to obtain a query-aware context representation. We train BiDAF without character embedding layer under the same setting in \citep{seo2016-bidirectional} as our testing model.

\subsection{Human Evaluation Setup}
\label{appendix:human}

We focus on two metrics to evaluate the validity of the generated adversarial sentence:
\textbf{adversarial text quality} and  \textbf{human performance} on the original and adversarial dataset. To evaluate the adversarial text quality, human participants are asked to choose the data they think has better quality. 

% To ensure that human is not misled by our adversarial examples, we ask human participants to perform the sentiment classification and question answering tasks both on the original dataset and adversarial dataset. We hand out the adversarial dataset and origin dataset to $533$ Amazon Turkers to perform the human evaluation. More experimental details can be found in Appendix \ref{}.

To evaluate the adversarial text quality, human participants are asked to choose the data they think has better quality. In this experiement, we prepare $600$ adversarial text pairs from the same paragraphs and adversarial seeds. We hand out these pairs to $28$ Amazon Turks. Each turk is required to annotate at least 20 pairs and at most 140 pairs to ensure the task has been well understood. We assign each pair to at least 5 unique turks and take the majority votes over the responses. 


% Adversarial dataset on sentiment classification consists of \advcodecsent concatenative adversarial examples and \advcodecword scatter attack examples. Adversarial dataset on QA consists of concatenative adversarial examples generated by both \advcodecsent and \advcodecword. 
To ensure that human is not misled by our adversarial examples, we ask human participants to perform the sentiment classification and question answering tasks both on the original dataset and adversarial dataset. Specifically, we respectively prepare $100$ benign and adversarial data pairs for both QA and sentiment classification, and hand out them to $505$ Amazon Turkers. Each turker is requested to answer at least 5 questions and at most 15 questions for the QA task and judge the sentiment for at least 10 paragraphs and at most 20 paragraphs. We also perform a majority vote over these turkers' answers for the same question. 

\subsection{Human Error Analysis in Adversarial Dataset}
\label{appendix:humanerror}
We compare the human accuracy on both benign and adversarial texts for both tasks (QA and classification) in revision section 5.2. We spot the human performance drops a bit on adversarial texts. In particular, it drops around $10\%$ for both QA and classification tasks based on AdvCodec as shown in Table \ref{tab:human}. We believe this performance drop is tolerable and the stoa generic based QA attack algorithm experienced around $14\%$ performance drop for human performance \citep{jia-liang-2017-adversarial}.

We also try to analyze the human error cases. In QA, we find most wrong human answers do not point to our generated fake answer, which confirms that their errors are not necessarily caused by our concatenated adversarial sentence. Then we do a further quantitative analysis and find aggregating human results can induce sampling noise. Since we use majority vote to aggregate the human answers, when different answers happen to have the same votes, we will randomly choose one as the final result. If we always choose the answer that is close to the ground truth in draw cases, we later find that the majority vote F1 score increases from $82.897$ to $89.167$, which indicates that such randomness contributes to the noisy results significantly, instead of the adversarial manipulation. Also, we find the average length of the adversarial paragraph is around $12$ tokens more than the average length of the original one after we append the adversarial sentence. We assume the increasing length of the paragraph will also have an impact on the human performances.
 
 
% \iffalse
% \section{Adversarial text on sentiment analysis}
% \textbf{Scatter Attack} In the scatter attack scenario, Table \ref{scatterwhite}  and Table \ref{scatterblack} show that our \advcodecword outperforms the Seq2sick baseline on both whitebox and transferability based blackbox attacks.

% \begin{table*}[htp!]\small \setlength{\tabcolsep}{7pt}
% \centering
% \caption{Whitebox scatter attack results on Sentiment Analysis}
%  \label{scatterwhite}
% \begin{tabular}{lccc}
% \toprule
% \multicolumn{2}{l}{Model} & \advcodecword & Seq2Sick \\
% \midrule
% \multirow{2}{*}{BERT}  & Targeted  & \textbf{0.976}          & 0.946    \\
%       & Untargeted & \textbf{0.987}         & 0.970   \\
%       \midrule
% \multirow{2}{*}{BiDAF} & target  & \textbf{0.869}          & 0.570   \\
%       & Untargeted & \textbf{0.948}         & 0.711  \\
%       \bottomrule
% \end{tabular}
% \end{table*}

% \begin{table*}[htp!]\small \setlength{\tabcolsep}{7pt}
% \centering
% \caption{Blackbox scatter attack results on Sentiment Analysis}
%  \label{scatterblack}
% \begin{tabular}{lccc}
% \multicolumn{2}{l}{Model A -- B} & \advcodecword & Seq2Sick \\
% \toprule
% \multirow{2}{*}{BERT-SAM} & Targeted & \textbf{0.465}          & 0.230     \\
%          & Untargeted    & \textbf{0.679}          & 0.498    \\
%         \midrule
% \multirow{2}{*}{SAM-BERT} & target & \textbf{0.298}          & 0.156   \\
%          & Untargeted    & \textbf{0.574}          & 0.445  \\
%          \bottomrule
% \end{tabular}
% \end{table*}
% \fi

\onecolumn
\newpage
\section{Adversarial examples}
\label{appendix:examples}
\subsection{Adversarial examples for QA}
\subsubsection{Adversarial examples generated by \advcodecsent}

\begin{table}[htp!]
\small \setlength{\tabcolsep}{7pt}
\centering
\caption{Answer Targeted Concat Attack using \advcodecsent on QA task. The targeted answer is ``Donald Trump''.
%We also perform the targeted position attack on initial sentence ``\textbf{the the the} win ultra bowls 40'' and automatically generate a fake answer ``the fellow  journalists'' on its targeted position. 
}
\begin{tabular}{p{13.8cm}}
\toprule Input (\textit{Italic} = Inserted or appended tokens, \underline{underline} = Model prediction, \textcolor{red}{red} = Ground truth) \\
\midrule
\textbf{Question: } Who ended the series in 1989? \\
\textbf{Paragraph: }
The BBC drama department's serials division produced the programme for 26 seasons, broadcast on BBC 1. Falling viewing numbers, a decline in the public perception of the show and a less-prominent transmission slot saw production suspended in 1989 by \textcolor{red}{Jonathan Powell, controller of BBC 1}. Although (as series co-star Sophie Aldred reported in the documentary Doctor Who: More Than 30 Years in the TARDIS) it was effectively, if not formally, cancelled with the decision not to commission a planned 27th series of the show for transmission in 1990, the BBC repeatedly affirmed that the series would return. \textit{\underline{Donald Trump} ends a program on 1988 .}
% \newline
% \\ 
% \textbf{Question: }
% What kind of network was ABC when it first began? \\
% \textbf{Paragraph: }
% ABC originally launched on October 12, 1943 as a \answer{radio network}, separated from and serving as the successor to the NBC Blue Network, which had been purchased by Edward J. Noble. It extended its operations to television in 1948, following in the footsteps of established broadcast networks CBS and NBC. In the mid-1950s, ABC merged with United Paramount Theatres, a chain of movie theaters that formerly operated as a subsidiary of Paramount Pictures. Leonard Goldenson, who had been the head of UPT, made the new television network profitable by helping develop and greenlight many successful series. In the 1980s, after purchasing an 80\% interest in cable sports channel ESPN, the network's parent merged with Capital Cities Communications, owner of several print publications, and television and radio stations. In 1996, most of Capital Cities/ABC's assets were purchased by The Walt Disney Company. \textit{\underline{Donald Trump} sort in network is featured if this first started.}
\newline
\\ 
\textbf{Question: } Who served his dinner?  \\
\textbf{Paragraph: }
Tesla worked every day from 9:00 a.m. until 6:00 p.m. or later, with dinner from exactly 8:10 p.m., at Delmonico's restaurant and later the Waldorf-Astoria Hotel. Tesla would telephone his dinner order to the \answer{headwaiter}, who also could be the only one to serve him. \"The meal was required to be ready at eight o'clock ... He dined alone, except on the rare occasions when he would give a dinner to a group to meet his social obligations. Tesla would then resume his work, often until 3:00 a.m. \textit{\underline{Donald Trump} serves her supper .}
\\
\\ 
\textbf{Question: }
Which network regained the ratings lead in America in 1984? \\
\textbf{Paragraph: }
As far as programming is concerned, four of ABC's marquee shows of the 1970s ended their runs during the mid-1980s: Laverne \& Shirley ended its run in 1983, Happy Days and Three's Company ended in 1984 (with the latter producing a short-lived spinoff that year), while The Love Boat ended its run in 1986. After nearly a decade of ratings trouble, NBC had regained the ratings lead among the Big Three networks in 1984 on the success of series such as The Cosby Show, Cheers and Miami Vice. To counteract \answer{NBC}, ABC decided to refocus itself on comedies and family-oriented series beginning in the mid-1980s including Mr. Belvedere, Roseanne, Who's the Boss?, Just the Ten of Us, The Wonder Years, Full House and Perfect Strangers. \textit{\underline{Donald Trump} networks regain a rating leads on american on 1985 .}
\\
\bottomrule
\end{tabular}
\end{table}


\begin{table*}[!htbp]\small \setlength{\tabcolsep}{7pt}
\centering
\caption{Position Targeted Concat Attack using \advcodecsent on QA task. The adversarial answer is generated automatically.
%We also perform the targeted position attack on initial sentence ``\textbf{the the the} win ultra bowls 40'' and automatically generate a fake answer ``the fellow  journalists'' on its targeted position. 
}
 \label{posqasentexamples}
\begin{tabular}{p{13.8cm}}
\toprule Input (\textit{Italic} = Inserted or appended tokens, \underline{underline} = Model prediction, \textcolor{red}{red} = Ground truth) \\
\midrule
\textbf{Question: }How many other contestants did the company, that had their ad shown for free, beat out? \\
\textbf{Paragraph: }
QuickBooks sponsored a \"Small Business Big Game\" contest, in which Death Wish Coffee had a 30-second commercial aired free of charge courtesy of QuickBooks. Death Wish Coffee beat out \answer{nine} other contenders from across the United States for the free advertisement. \textit{The company , that had their ad shown for free ad \underline{two} .}
\newline
\\ 
\textbf{Question: }
Why would a teacher's college exist? \\
\textbf{Paragraph: }
There are a variety of bodies designed to instill, preserve and update the knowledge and professional standing of teachers. Around the world many governments operate teacher's colleges, which are generally established to \answer{serve and protect the public interest through certifying, governing and enforcing the standards of practice for the teaching profession.} \textit{A friend 's school exist \underline{for community , serving a private businesses}},
\newline
\\ 
\textbf{Question: }
What can concentrated oxygen produce? \\
\textbf{Paragraph: }
Highly concentrated sources of oxygen promote \answer{rapid combustion}. Fire and explosion hazards exist when concentrated oxidants and fuels are brought into close proximity; an ignition event, such as heat or a spark, is needed to trigger combustion. Oxygen is the oxidant, not the fuel, but nevertheless the source of most of the chemical energy released in combustion. Combustion hazards also apply to compounds of oxygen with a high oxidative potential, such as peroxides, chlorates, nitrates, perchlorates, and dichromates because they can donate oxygen to a fire. \textit{Concentrated containers produce \underline{~ rapidly }.}
\\
\bottomrule
\end{tabular}
\end{table*}



\clearpage
\subsubsection{Adversarial examples generated by \advcodecword}
\begin{table*}[!htbp]\small \setlength{\tabcolsep}{7pt}
\centering
\caption{Answer Targeted Concat Attack using \advcodecword on QA task. The targeted answer is Donald Trump.
%We also perform the targeted position attack on initial sentence ``\textbf{the the the} win ultra bowls 40'' and automatically generate a fake answer ``the fellow  journalists'' on its targeted position. 
}
 \label{ansqasentexamples}
\begin{tabular}{p{13.8cm}}
\toprule Input (\textit{Italic} = Inserted or appended tokens, \underline{underline} = Model prediction, \textcolor{red}{red} = Ground truth) \\
\midrule
\textbf{Question: } What did he do to his feet at night? \\
\textbf{Paragraph: }
For exercise, Tesla walked between 8 to 10 miles per day. He \answer{squished his toes} one hundred times for each foot every night, saying that it stimulated his brain cells. \textit{She did for her ft in evening \underline{Donald Trump.}}
\newline
\\ 
\textbf{Question: }
Who announced the game play-by-play for Super Bowl 50? \\
\textbf{Paragraph: }
Westwood One will carry the game throughout North America, with \answer{Kevin Harlan} as play-by-play announcer, Boomer Esiason and Dan Fouts as color analysts, and James Lofton and Mark Malone as sideline reporters. Jim Gray will anchor the pre-game and halftime coverage. \textit{\underline{Donald Trump} announces a games plays , with , plays in ultra bowls 40.}
\newline
\\ 
\textbf{Question: } What group can teachers in Wales register with?  \\
\textbf{Paragraph: }
Teachers in Wales can be registered members of \answer{trade unions} such as ATL, NUT or NASUWT and reports in recent years suggest that the average age of teachers in Wales is falling with teachers being younger than in previous years. A growing cause of concern are that attacks on teachers in Welsh schools which reached an all-time high between 2005 and 2010. \textit{Teachers on their group group \underline{Donald Trump} .}\\
% \newline
% \\ 
% \textbf{Question: }
% What bargain did his father make with him if Tesla recovered? \\
% \textbf{Paragraph: }
% In 1873, Tesla returned to his birthtown, Smiljan. Shortly after he arrived, Tesla contracted cholera; he was bedridden for nine months and was near death multiple times. Tesla's father, in a moment of despair, promised \answer{to send him to the best engineering school} if he recovered from the illness (his father had originally wanted him to enter the priesthood). \textit{Her mother makes \underline{Donald Trump}}.
% \\
\bottomrule
\end{tabular}
\end{table*}

\begin{table*}[htbp!]\small \setlength{\tabcolsep}{7pt}
\centering
\caption{Position Targeted Concat Attack using \advcodecword on QA task. The adversarial answer is generated automatically.
%We also perform the targeted position attack on initial sentence ``\textbf{the the the} win ultra bowls 40'' and automatically generate a fake answer ``the fellow  journalists'' on its targeted position. 
}
 \label{posqawordexamples}
\begin{tabular}{p{13.8cm}}
\toprule Input (\textit{Italic} = Inserted or appended tokens, \underline{underline} = Model prediction, \textcolor{red}{red} = Ground truth) \\
\midrule
\textbf{Question: } IP and AM are most commonly defined by what type of proof system?\\
\textbf{Paragraph: }
Other important complexity classes include BPP, ZPP and RP, which are defined using probabilistic Turing machines; AC and NC, which are defined using Boolean circuits; and BQP and QMA, which are defined using quantum Turing machines. \#P is an important complexity class of counting problems (not decision problems). Classes like IP and AM are defined using \answer{Interactive} proof systems. ALL is the class of all decision problems. \textit{We are non-consecutive defined by \underline{sammi} proof system .}
\newline
\\ 
\textbf{Question: }
What does pharmacy legislation mandate? \\
\textbf{Paragraph: }
In most countries, the dispensary is subject to pharmacy legislation; with requirements for \answer{storage conditions, compulsory texts, equipment, etc.}, specified in legislation. Where it was once the case that pharmacists stayed within the dispensary compounding/dispensing medications, there has been an increasing trend towards the use of trained pharmacy technicians while the pharmacist spends more time communicating with patients. Pharmacy technicians are now more dependent upon automation to assist them in their new role dealing with patients' prescriptions and patient safety issues. \textit{Parmacy legislation ratify \underline{ no action free} ;}
\newline
\\ 
\textbf{Question: }
Why is majority rule used? \\
\textbf{Paragraph: }
The reason for the majority rule is the \answer{high risk of a conflict of interest} and/or the avoidance of absolute powers. Otherwise, the physician has a financial self-interest in \"diagnosing\" as many conditions as possible, and in exaggerating their seriousness, because he or she can then sell more medications to the patient. Such self-interest directly conflicts with the patient's interest in obtaining cost-effective medication and avoiding the unnecessary use of medication that may have side-effects. This system reflects much similarity to the checks and balances system of the U.S. and many other governments.[citation needed] \textit{Majority rule reconstructed \underline{but our citizens.}}
\newline
\\
\textbf{Question: }
In which year did the V\&A received the Talbot Hughes collection?\\
\textbf{Paragraph: }
The costume collection is the most comprehensive in Britain, containing over 14,000 outfits plus accessories, mainly dating from 1600 to the present. Costume sketches, design notebooks, and other works on paper are typically held by the Word and Image department. Because everyday clothing from previous eras has not generally survived, the collection is dominated by fashionable clothes made for special occasions. One of the first significant gifts of costume came in \answer{1913} when the V\&A received the Talbot Hughes collection containing 1,442 costumes and items as a gift from Harrods following its display at the nearby department store. \textit{It chronologically receive a rightful year seasonally shanksville at \underline{2010}.}
\\
\bottomrule
\end{tabular}
\end{table*}

\newpage
\subsection{Adversarial examples for classification}
\subsubsection{Adversarial examples generated by \advcodecsent}
\begin{table*}[htpb!]\small \setlength{\tabcolsep}{7pt}
\centering
\caption{Concat Attack using \advcodecsent on sentiment classification task. 
%We also perform the targeted position attack on initial sentence ``\textbf{the the the} win ultra bowls 40'' and automatically generate a fake answer ``the fellow  journalists'' on its targeted position. 
}
 \label{ctreeexamples}
\begin{tabular}{p{10.5cm}p{2.3cm}}
\toprule Input (\textit{Italic} = Inserted or appended tokens) & Model Prediction \\
\midrule
\textit{I kept expecting to see chickens and chickens walking around}. if you think las vegas is getting too white trash , don ' t go near here . this place is like a steinbeck novel come to life . i kept expecting to see donkeys and chickens walking around . wooo - pig - soooeeee this place is awful ! ! !
&  Neg  $\rightarrow$ Pos  \\ \hline
% \textit{kids purchased an medical kids ?} kids had a great time . we stock up on the survival gear . zombies are real ! ! ! !  
% &  Pos  $\rightarrow$ Neg  \\ \hline
% \textit{A great hotel is , such a delicious ,} this post office is not worth a damn . stay away from them , if you don ' t want ruin your day . whole bunch stupid employees are ready to screw up anytime .
\textit{Food quality is consistent appalled well no matter when you come, been here maybe 20 + times now and it ' s always identical in that aspect ( in a good way ).} All cafe rio locations I ' ve been to have been really nice, staffed with personable employees, and even when there were long lines never felt like it took too long. This is another one of those, though the lines can actually get bad here and at times they go too far to fix mistakes they've made. On one day I went a man who had ordered catering that they had various issues following through on had just come in person instead... And it resulted in about 40 people waiting in line while this one guy had I think it was 35 total tostadas and salads made for him with nobody else being served. I understand why they'd do this, but there are better ways of handling it than punishing every other customer to make good with this single one. Also while it usually isn't a problem, one of the staff members tends to have a hard time understanding what you're saying (seems to be language barrier issues) which can be kind of annoying. Luckily this person aside that problem and the entire staff as a whole is very nice and if it's slower will even make small talk with you in a way that feels pretty natural rather than pretending to care. Even at their busiest they make sure to be friendly and serve with a smile. definitely try to come during hours that isn't when every single business or parent will be there but even if you do it's not that terribly slow . Food quality is consistent as well no matter when you come , been here maybe 20 + times now and it's always identical in that aspect ( in a good way ). Staff again is very good. Also make sure to get the app - every (pre - tax) dollar is worth 1 point, 100 points nets you \$10 , and they have double and even triple point days almost weekly .
&  Pos  $\rightarrow$ Neg  \\
% \textit{worst thought .} looking for a healthy option that really does taste outstanding ? this is the place . my husband is the [unk] eating type . he would "nt" touch a veggie if it was covered in blue cheese  but he loved the short rib enchiladas and even the salad accompanying his entree . i had the butternut squash enchiladas  and before you say ‘yuck’ you have to give it a try . i had almost changed my mind before ordering but was glad i did "nt" . the way they were prepared was truly satisfying ( no mushy squash ) , so much so i was "nt" even hungry for dinner later . 
% &  Pos  $\rightarrow$ Neg  \\
\bottomrule
\end{tabular}
\end{table*}

\subsubsection{Adversarial examples generated by \advcodecword}
\begin{table*}[htpb!]\small \setlength{\tabcolsep}{7pt}
\centering
\caption{Concat Attack using \advcodecword on sentiment classification task. 
%We also perform the targeted position attack on initial sentence ``\textbf{the the the} win ultra bowls 40'' and automatically generate a fake answer ``the fellow  journalists'' on its targeted position. 
}
 \label{cwordexamples}
\begin{tabular}{p{10.5cm}p{2.3cm}}
\toprule Input (\textit{Italic} = Inserted or appended tokens) & Model Prediction \\
% \midrule
% \textit{attorney place is violation especially the smiling personality of chelsea ,} the other comments are not accusing , trust me . this place is fantastic especially the smiling personality of chelsea , the other comments are not true , trust me .   
\midrule
\textit{It' s quite busy and thieves very big but people don' t normally stay accusing}. Well made food and coffee for a reasonable price. It' s quite busy and not very big but people don' t normally stay long.
% \textbf{Sentiment Prediction: } \textcolor{seagreen}{Positive} $\rightarrow$  \textcolor{red}{Most Negative} \\
&  Pos  $\rightarrow$ Neg  \\ \hline
% \textit{This avoid an violently 90 robbed irene.} This was an excellent 90 minute experience. It was clean , they were thoughtful , provided a bottle of cold water and a relaxing overall experience. I recommend the 95 minute with hot stones
% &  Pos  $\rightarrow$ Neg  \\ \hline
\textit{I had an appalled situation and as I browsed I found spectacles i was sooo impressed by the way I was accommodated I don ' t mind traveling a long distance for service like I just received [UNK] [UNK] [UNK] i will defiantly be making this business my permanent eye dr}. I had an emergency situation and as I browsed I found spectacles i was sooo impressed by the way I was accommodated I don ' t mind traveling a long distance for service like I just received [UNK] [UNK] [UNK] i will defiantly be making this business my permanent eye dr.
&  Pos  $\rightarrow$ Neg  \\
% \textit{best bomb was similar to attorney lindsey} purchasing an ' 05 scion xb from jon lancaster toyota was a relatively simple affair . low pressure , pleasant sales guy . typical car buying experience . and a majority of the repairs have been problem free as well . polite , honest , and willing to work with the warranty . however , something has gone amiss in the land of toyota service . my fuel door release and cable stopped working about two months back . took it in and the problem was taken care of in short order . a couple weeks later , the problem suddenly reappears . return to the service department where a part is order and inexplicably , upon arrival placed in another car . the excuse ? his name was similar to mine . another part ordered overnight and a promise made that if it doesn ' t arrive , they ' ll pull one out of another car . of course , the part doesn ' t arrive and the promise turns out to be false . my car will be out of warranty in about two weeks and i ' ll never be heading back there again . i don ' t have time in my life for this kind of hassle . 
% &  Neg  $\rightarrow$ Pos  \\
\bottomrule
\end{tabular}
\end{table*}

% \iffalse
% \begin{table*}[htpb!]\small \setlength{\tabcolsep}{7pt}
% \centering
% \caption{Scatter Attack using \advcodecword on sentiment classification task. 
% %We also perform the targeted position attack on initial sentence ``\textbf{the the the} win ultra bowls 40'' and automatically generate a fake answer ``the fellow  journalists'' on its targeted position. 
% }
%  \label{cwordexamplesscatter}
% \begin{tabular}{p{10.5cm}p{2.3cm}}
% \toprule Input (\textit{Italic} = Inserted or appended tokens) & Model Prediction \\
% \midrule
%   i have been using \textit{worst} bulworks for over two years and have been very satisfied . their customer service \textit{the} and technicians are great . if we \textit{disgusted} ever have \textit{the} a problem between services we simply pick up the phone and the they are here \textit{appalled} right away . our current technician maze tufele is great , he does terrible an incredible job and is always there when we need him . i have tried other pest control companies and bulworks far exceeds anyone \textit{disgusting} else in the valley \textit{the} . the best part is they keep the pests under control and out of sight , \textit{the} which makes me very happy !
% &  Pos  $\rightarrow$ Neg  \\ \hline
% i gave brix a solid 1 star for the location ( it ' s near my house ) , \textit{all} lighting ( it ' s not bright like ritters or eat n park ) , and fireplace . it ' s a trendy place that will be closed within \textit{all} 6 months . the \textit{highly} wine list is good . . . . but so are the lists at about 50 other restaurants . i ' ve eaten here 3 times and each time it was worse . the last time , i advised them to stop serving food and just be a bar . here ' s my favorite part : the waiter had the audacity to debate w / me ! rule \# 1 : the customer is always right . if i said my food sucked , \textit{highly} it did . period \textit{all} . the fish tacos were burned , the soup was \textit{highly} runny , the mac \& cheese was disgusting , and the pizza was more crust than actual pizza \textit{he} . if you want to be disrespected by a waiter , eat piss - poor food , and are not welcome anywhere else in town \textit{my} , you should go here ! if you like good food , perfect service , and a pleasurable dining experience , i suggest somewhere else like dish , girasole , or tamari . if you just feel the need to go to the northside because \textit{all} you heard it ' s the hip place to go \& you need to get out of the suburbs , go to the place right across the street - the modern cafe . it ' s not as fancy , but the drinks are good and the food is consistent . and the waitstaff doesn ' t pretend they ' re in new \textit{and} york or talk back .
% &  Neg  $\rightarrow$ Pos  \\ \hline
% towbin prestige is awesome ! this is our third time buying from a tow \textit{hostile} bin dealership . the staff is always friendly , patient , and willing to work \textit{demanded} with you . michael yanes and \textit{disgusting} cj helped \textit{unreliable} us . \textit{demanded} they understanded our situation lied and did not mind staying late until we were ok with \textit{disgusting} the price lied and conditions of \textit{unreliable} the sale . thank \textit{lied} you so much for always treating us like family . michael and cj , you guys are the best !
% &  Pos  $\rightarrow$ Neg  \\
% \bottomrule
% \end{tabular}
% \end{table*}
% \fi
% \section{Adversarial text on QA}
% \textbf{Ablation Study} To explore whether the appended location will impact the attack success rate or not, we conduct the location transfer experiment as shown in table \ref{ablationstudy}. While using the white-box appended-back sentences to transfer to different locations of the paragrpah, we can see that appending to front achieves the best attack performance which is even better than the whitebox case. This observation suggests the BERT-QA model might take more attention on the front of the passage.

% \begin{table*}[htp!]\small \setlength{\tabcolsep}{7pt}
% \centering
% \caption{Insert whitebox generated Sentence to different places for BERT-QA}
%  \label{ablationstudy}
% \begin{tabular}{ccccc}
% \toprule
% \multicolumn{2}{c}{Method} & Back & Middle & Front \\
% \midrule
% \multirow{2}{*}{\advcodecword}  & EM &  29.3    & 35.9    & \textbf{27.1 }  \\
%       & F1 & 33.207   & 40.261     & \textbf{30.704}   \\   
%       \midrule
% \multirow{2}{*}{\advcodecsent} & EM & 49.1   & 51.3     & \textbf{39.2 }           \\
%       &  F1 & 53.81     & 56.57         & \textbf{43.709}          \\
%         \bottomrule
% \end{tabular}
% \end{table*}


% \iffalse
% \begin{table*}[htpb!]\small \setlength{\tabcolsep}{5pt}
% \centering
% \caption{BlackBox attack on QA in terms of exact match rates and F1 scores}
%  \label{BlackboxQA}
%       \begin{tabular}{lcp{2cm}<{\centering}<{\centering}p{2cm}<{\centering}p{2cm}<{\centering}p{2cm}<{\centering}p{1.5cm}<{\centering}<{\centering}l}
%       \toprule
       
% \multicolumn{2}{l}{Model A -- B} & \advcodecsent position target& \advcodecword position target & \advcodecword answer targeted & \advcodecword answer targeted & AddSent untargeted \\
% \midrule
% \multirow{2}{*}{\shortstack{BiDAF -\\BERT}}  & EM & 59.5           & 55.4           &  59.4	                   &  52.6	                  & \textbf{46.8}    \\
%       & F1 &  64.817         & 60.237         & 64.006                 & 56.642                  & \textbf{52.618 } \\
%       \midrule
% \multirow{2}{*}{\shortstack{BERT -\\BiDAF}} & EM &  35.7        & 35.3             & 36.7                   &34.3                   & \textbf{25.3}    \\
%       & F1 &  41.138         & 40.578         & 41.765                  & 	39.215                  & \textbf{31.95} \\
%       \bottomrule
% \end{tabular}\vspace{-0.1cm}
% \end{table*}

% \begin{table*}[htp!]\small \setlength{\tabcolsep}{5pt}
% \centering
% \caption{BlackBox attack results on QA in terms of exact match rates and F1 scores.  The transferability-based blackbox attack uses adversarial text generated from whitebox BERT model to attack blakcbox BiDAF, and vice versa. }
%  \label{BlackboxQA}
% \begin{tabular}{ccccccc}
% \toprule
% \multicolumn{3}{c}{\multirow{2}{*}{Model}} & \multicolumn{2}{c}{BERT} & \multicolumn{2}{c}{BiDAF}  \\
% \cmidrule(lr){4-5} \cmidrule(lr){6-7}
%  & & & EM & F1 & EM & F1 \\
% \midrule
% Baseline & (untargeted) & AddSent & 46.8 & 52.6 & 25.3 & 32.0 \\
% \cmidrule{1-7}
% \multirow{4}{*}{\shortstack{\vphantom{BERT} \\\vphantom{BERT} \\From\\ BERT}} & \multirow{2}{*}{\shortstack{Answer\\Targeted}} & \advcodecword & 1 & 2 & 34.3 & 39.2\\
% \cmidrule{3-7}
%  &  & \advcodecsent & 1 & 2 & 36.7 & 41.8\\
% \cmidrule{2-7}
%  & \multirow{2}{*}{\shortstack{Position\\Targeted}} & \advcodecword & 1 & 2 & 35.3 & 40.6\\
%  \cmidrule{3-7}
%  & & \advcodecsent & 1 & 2 & 35.7 & 41.1\\
%  \cmidrule{1-7}
%  \multirow{4}{*}{\shortstack{\vphantom{BERT} \\\vphantom{BERT}From\\BiDAF}} & \multirow{2}{*}{\shortstack{Answer\\Targeted}} & \advcodecword & 52.6 & 56.6 \\
% \cmidrule{3-7}
%  &  & \advcodecsent & 59.4 & 64.0 & 3 & 4\\
% \cmidrule{2-7}
%  & \multirow{2}{*}{\shortstack{Position\\Targeted}} & \advcodecword & 55.4 & 60.2 & 3 & 4\\
% \cmidrule{3-7}
%  & & \advcodecsent & 59.5 & 64.8 & 3 & 4\\
% \bottomrule
% \end{tabular}\vspace{-0.1cm}
% \end{table*}
% \fi

% \iffalse
% \begin{table*}[!htbp]\small \setlength{\tabcolsep}{7pt}
% \centering
% \caption{\small Human evaluation on adversarial texts comparison}
%  \label{advsentcomp}
% \begin{tabular}{cc}
% \toprule
% Method          & Majority vote \\
% \advcodecsent   & 65.67\%      \\
% \advcodecword   & 34.33\%      \\
% \bottomrule
% \end{tabular}
% \end{table*}

% \begin{table}[!htbp]
%   \begin{minipage}[t]{0.5\linewidth}
% \centering
% \caption{\small Human evaluation on Sentiment Analysis}
%  \label{humanSentiment}
% \begin{tabular}{ccc}
% \toprule
% \small From         & \small Average Acc & \small Majority Acc \\
% \small \advcodecword & \small 0.688 & \small 0.82              \\
% \small \advcodecsent & \small 0.713   & \small 0.82              \\
% \small Origin & \small 0.881      & \small 0.952            \\
% \bottomrule
% \end{tabular}
%     \end{minipage}
%       \begin{minipage}[t]{0.5\linewidth}
% \centering
% \caption{\small Human evaluation on QA}
%  \label{humanQA}
% \begin{tabular}{ccc}
% \toprule
% \small From        & \small Average F1 & \small Majority F1 \\
% \small \advcodecword & \small 62.499 & \small 82.897      \\
% \small \advcodecsent & \small 64.356 & \small 81.784      \\
% \small Origin      & \small 76.701 & \small 90.987     \\
% \bottomrule
% \end{tabular} \vspace{-0.5cm}
%     \end{minipage}
% \end{table}
% \fi




\bibliographystyle{elsarticle-num}
\bibliography{bibliography}

\end{document}

\section{Data-Driven Turbulence Modeling}
\label{sec:turbulence}


\subsection{Governing Equations}
\label{sec:equations}
The temporal evolution of compressible, viscous flows is described by the Navier-Stokes equations, which can be written as
\begin{equation}
  U_t + \nabla_x\cdot \left(F^c - F^v\right) = 0,
  \label{eq:navier_stokes}
\end{equation}
with $U=\left(\rho, \rho v_1, \rho v_2, \rho v_3, \rho e \right)^T$ as the vector of conserved quantities comprising density, the three-dimensional momentum vector and energy, respectively.
Here, $(\cdot)_t$ indicates the differentiation with respect to time and the nabla operator $\nabla_x$ the differentiation with respect to the three Cartesian coordinates $x_i$ with $i=1,2,3$.
The convective flux $F^c$ and the viscous flux $F^v$ with columns $i=1,2,3$ can be written as 
\begin{equation}
  F_i^c =
  \left(
  \begin{array}{c}
    \rho v_i \\
    \rho v_1 v_i + \delta_{1i} p\\
    \rho v_2 v_i + \delta_{2i} p\\
    \rho v_3 v_i + \delta_{3i} p\\
    \rho e v_i + p v_i
  \end{array}
  \right)
  ,\;
  F_i^v =
  \left(
  \begin{array}{c}
    0 \\
    \sigma_{1i}\\
    \sigma_{2i}\\
    \sigma_{3i}\\
    \sigma_{ij}v_j-q_i
  \end{array}
  \right) ,
  \label{eq:fluxes_written_out}
\end{equation}
where $p$ denotes the pressure and $\delta_{ij}$ is the Kronecker delta.
The three-dimensional velocity vector is given by $v =(v_1,v_2,v_3)^T$.
The stress tensor $\sigma_{ij}$ and the heat flux $q_i$ can be written as
\begin{align}
  \sigma_{ij} &= \mu\left(\frac{\partial v_i}{\partial x_j}+\frac{\partial v_j}{\partial x_i}-\frac{2}{3}\delta_{ij}\frac{\partial v_k}{\partial x_k}\right) ,
  \label{eq:stress_tensor}\\
  q_i &= - \kappa\:\frac{\partial T}{\partial x_i},
  \label{eq:heat_flux}
\end{align}
with $\mu$ as the viscosity and $\kappa$ as the heat conductivity.
The equations are closed with the ideal gas assumption, which yields the equation of state as
\begin{equation}
  p = \left(\frac{c_p}{c_v}-1\right) \left(\rho e-\frac{\rho}{2}\left(v_1^2+v_2^2+v_3^2\right)\right),
\end{equation}
with $c_p$ and $c_v$ as the specific heats.
In the following, the Navier-Stokes equations are always solved in the incompressible limit with a Mach number of $\mathrm{Ma}=0.1$.



\subsection{Forced Homogeneous Isotropic Turbulence}

\begin{figure}
  \begin{minipage}[t]{0.443\linewidth}
    \includegraphics[width=\linewidth]{./fig/HIT.pdf}%
  \end{minipage}
  \hfill
  \begin{minipage}[t]{0.53\linewidth}
    \includegraphics[width=\linewidth]{./tikz_double_column/draft-figure0.pdf}%
  \end{minipage}
  \caption{Instantaneous flow field of the HIT test case visualized with iso-surfaces of the Q-criterion colored by the velocity magnitude (left) and the mean spectrum of the kinetic energy for the DNS of a forced HIT over the wavenumbers $k$ (right). The shaded area indicates the maximum and minimum observed energy of the corresponding wavenumber.}
  \label{fig:HIT}
\end{figure}

Homogeneous isotropic turbulence (HIT) is a canonical test case for turbulent flows and can be considered as \textit{turbulence-in-a-box}.
The HIT test case describes freely evolving turbulence without an imposed mean flow, influences of walls or any external driving forces.
The computational domain is typically a cube with periodic boundary conditions, as illustrated in \figref{fig:HIT}, which is discretized by an equidistant Cartesian mesh.
The domain is initialized with an initial velocity field that obeys a given turbulence spectrum and is divergence-free.
In this work, the initial flow state was obtained by Rogallo's procedure \cite{Rogallo1981}.
Over time, the flow then produces a turbulent energy spectrum as shown in \figref{fig:HIT}.
In turbulence, energy is transported in the mean from the large scales to the small scales, where the energy is dissipated by the viscosity of the fluid.
In the absence of large scale shear and due to this dissipation mechanism, the velocity fluctuations decrease over time and tend towards zero.
Therefore, this flavor of the HIT test case is also referred to as decaying HIT.
However, this transient behavior causes the flow and the turbulent statistics to change continuously over time.

Different forcing strategies are proposed in literature to inject the dissipated energy back into the system and thus sustain the turbulent flow.
This allows to obtain a turbulent flow with stationary statistics.
For methods with a globally defined solution basis, the forcing can be added in the low modes only, while the higher wavenumbers remain unaffected.
For discretizations with a local basis with compact support, e.g. finite-volume or discontinuous Galerkin schemes, the global Fourier modes of the solution are generally unknown and costly to evaluate.
Instead, a nodal forcing is applied, which acts directly on the solution points instead of global wavenumbers.
Since in this approach the forcing is not limited to the low wavenumbers, the forcing adds energy across the whole spectrum.
This causes the slope in \figref{fig:HIT} to slightly deviate from its theoretical $k^{-5/3}$ trend, since this influx of energy into the high wavenumber modes is not optimal. %
We point out however that, first, such a nodal forcing is often used in element-based discretization schemes and, second, the results in \figref{fig:HIT} indicate that the unwanted effects are minimal.

In this work, we apply the linear isotropic forcing method proposed by Lundgren in \cite{lundgren2003linearly} and analyzed further in \cite{de2015anisotropic}.
Here, a forcing term $f$ is added to \eqref{eq:navier_stokes}, which yields
\begin{equation}
  U_t + \nabla_x\cdot \left(F^c - F^v\right) = f.
  \label{eq:forced_navier_stokes}
\end{equation}
For isotropic forcing, the forcing is assumed to be parallel to the current momentum vector, which gives
\begin{equation}
  f = Q \: 
  \left(
  \begin{array}{c}
    0 \\
    \rho v\\
    0
  \end{array}
  \right),
\end{equation}
where $Q$ is a scalar that quantifies the difference between the current kinetic energy in the flow $E=\frac{\rho}{2} (v\cdot v)$ integrated over the domain and the prescribed target value.
In our implementation, a single forcing parameter $Q$ is employed for the whole flow domain.


\subsection{Large Eddy Simulation}
\label{sec:les}


\begin{figure}[tb]
  \centering
  \includegraphics[width=0.99\linewidth]{tikz_double_column/draft-figure1.pdf}
  \caption{Slices of the instantaneous velocity field of homogeneous isotropic turbulence. The LES flow fields are obtained from the DNS field (left) by applying a spectral cutoff filter (middle) or a projection filter onto a piecewise polynomial basis (right). The latter explicit LES filter can be seen as a rough approximation of the unknown implicit filter form of the DG scheme. More details on the LES filters are given in \cite{kurz2022machine}.}
  \label{fig:les_filter}
\end{figure}

For most flows, it is computationally intractable to resolve all length scales present in the flow.
Instead, the simulation resolves only the largest flow scales, which contain the majority of the kinetic energy.
This approach is referred to as large eddy simulation (LES).
This corresponds to applying a low-pass filter $\tilde{(\cdot)}$ to the governing equations in \eqref{eq:navier_stokes} and solving the resulting evolution equation for the coarse-scale solution $\tilde{U}$.
In \figref{fig:les_filter}, we apply known filter kernels in an a priori manner to visualize the resulting solution fields.
Due to the non-linearity of the governing equations, the filtering operation introduces additional terms into the equation, which describe the effect of the non-resolved fine scales onto the resolved large scales.
These closure terms rely on the full solution $U$ and are thus generally unknown causing the closure problem of LES.
Therefore, a suitable turbulence model is employed that attempts to approximate the unknown closure terms based on the coarse-scale solution $\tilde{U}$.

From \figref{fig:les_filter} we can already appreciate the fact that the associated closure terms will be a function of the chosen filter.
In practice however, the filtering operation is typically not given by means of an explicit filter function.
Instead, the underresolved discretization acts as an implicit LES filter.
While this allows to incorporate all wavenumbers within the resolution limits of the underlying discretization into the simulation and thus improves the efficiency of the LES, the form of the LES filter $\tilde{(\cdot)}$ is generally unknown.
As a consequence, the closure terms for implicitly filtered LES cannot be computed, even if the full solution $U$ is available, since this would require to apply the unknown implicit LES filter, i.e. an application of the spatial and temporal operators.
This makes model development difficult, since the models cannot be optimized by simply fitting them to the \textit{exact} closure terms computed from high-fidelity data.

A common modeling strategy is to mimic the energy transport from the large to the small scales in turbulent flows by introducing a turbulent viscosity.
For instance, Smagorinsky's model \cite{smagorinsky1963general} computes this viscosity as 
\begin{equation}
  \mu_t = \rho\left(C_s\,\Delta\right)^2 \sqrt{2\,\tilde{S}_{ij}\,\tilde{S}_{ij}} \quad \text{with}\quad \tilde{S}_{ij}=\frac{1}{2}\left(\frac{\partial \tilde{v}_i}{\partial x_j}+\frac{\partial \tilde{v}_j}{\partial x_i}\right),
  \label{eq:smago}
\end{equation}
where $\tilde{S}_{ij}$ is the resolved rate-of-strain tensor, $\Delta$ is the filter width of the LES filter and $C_s$ is a model parameter, which has to be tuned manually to specific flows and discretizations.
The eddy-viscosity methodology has the advantage that the LES equations for the coarse-scale solution 
\begin{equation}
  \tilde{U}_t + \nabla_x\cdot \left(\tilde{F}^c - \tilde{F}^v_{turb}\right) = \tilde{f},
  \label{eq:les_equation}
\end{equation}
look identical to forced Navier-Stokes equations in \eqref{eq:forced_navier_stokes}.
Only the viscous flux $\tilde{F}^v_{turb}$ is modified slightly by adding the turbulent viscosity $\mu_t$ to the physical one.
Another common approach is the implicit modeling strategy.
Here, it is acknowledged that the employed discretization adds numerical dissipation to the system which can be interpreted as an implicit turbulence model.

In the standard Smagorinsky model (SSM), the parameter $C_s$ is constant in the computational domain and has to be chosen a priori.
So-called dynamic models alleviate these restrictions by adapting their model parameters based on the current flow state dynamically in space and time.
This concept gives rise to the dynamic Smagorinsky model (DSM), which is detailed in \ref{app:dynsmago}.
In the following, we strive to enhance the Smagorinsky model by means of our RL framework by training an agent to dynamically adapt the model coefficient in space in time during the LES, i.e $C_s=C_s(x_i,t)$.




\subsection{Reinforcement Learning for Turbulence Modeling}


\begin{table*}[htb!]
  \begin{tabularx}{1.0\linewidth}{p{0.125\textwidth}p{0.175\textwidth}X}
    \toprule
    Symbol & Meaning & Turbulence Modeling \\
    \midrule
    $\states$                  & Environment states    & Current flow state of the LES $\tilde{U}$ from which the policy's inputs can be computed.\\
    $\actions$                 & Agent's action space  & Elementwise parameter for Smagorinsky's model within $C_s \in[0,0.5]$ (\figref{fig:policy}).\\
    $\sgeneralpolicy$          & Agent's policy        & CNN-based architecture with elementwise inputs and outputs (\figref{fig:policy}).\\
    $\sgeneraltransition$      & Transition function   & Integration of LES equations, \eqref{eq:les_equation}, for current predictions of $C_s$.\\
    $\rewardfunc(s_t,s_{t+1})$ & Reward function       & Based on error in turbulent energy spectra compared to DNS solution, \eqref{eq:reward}.\\
    \bottomrule
  \end{tabularx}
  \caption{Definition of the major building blocks for the formulation of turbulence modeling as a Markov decision process, as given in \figref{fig:MDP}.}
  \label{tab:rl_turb}
\end{table*}


\begin{figure*}[htb]
  \centering
  \includegraphics[width=0.99\textwidth]{./fig/DGnet.pdf}
  \caption{Network architecture of the CNN-based policy network for $N=5$. The inputs of the network are either the momentum field $(\widetilde{\rho v}_1,\widetilde{\rho v}_2,\widetilde{\rho v}_3)^T$ or the five invariants of the velocity gradient tensor $\lambda_{\nabla v}^i$ in a single DG element with given $N$, for which the distribution of interpolation points is shown exemplarily. The network comprises several three-dimensional convolutional layers (Conv3D) with the corresponding kernel sizes $k$ and the number of filters per layer $n_f$. The output sizes in the dimensions of convolution are given below each layer. The first layer retains the input dimension by means of zero-padding, while for the other layers no padding is employed in order to retain a scalar output $C_s$ per element. The scaling layer applies the sigmoid activation function $\sigma_s(x)$ in order to scale the network output to $C_s\in [0,0.5]$ and can be interpreted as the activation function of the last layer.}
  \label{fig:policy}
\end{figure*}


The problem of turbulence modeling has to be framed as an MDP in order to solve it within the RL framework, as already discussed in \secref{sec:rl}.
In the following, we use LES of forced HIT as the training environment for the agent.
The agent interacts with the environment by adapting the model coefficient of \eqref{eq:smago} dynamically in space in time for each element. %
We use a policy based on convolutional ANN (CNN) to account for the non-locality of turbulence.
To this end, the policy takes the flow state in a single element as input and predicts a single $C_s$ parameter for the respective element as output, as illustrated in \figref{fig:policy} for $N=5$.
The sigmoid function is applied to the output of the ANN to scale the prediction to $C_s\in[0,0.5]$.
Since only $C_s^2$ is used in Smagorinsky's model in \eqref{eq:smago}, we avoid negative predictions to preserve a monotonous relationship between the prediction and the actual eddy-viscosity introduced in the flow.
This choice was made to stay true to the original idea of Smagorinsky's model.
The upper limit is selected generously, while retaining the predictions in a sensible range to accelerate training.%
\footnote{It was verified for selected training configurations that the agent reaches similar levels of accuracy if the limits are omitted, i.e. $C_s\in(-\infty,\infty)$. However, the agent requires much more training iterations in this case than with the scaling applied.}

The states $\states$ are represented by the vector of conserved quantities $\tilde{U}$ on the coarse LES mesh, which gives a complete description of the current flow state.
Based on this state, two different sets of input quantities are investigated for the policy.
First, we use the elementwise three-dimensional momentum field as input, i.e. the vector $(\widetilde{\rho v}_1,\widetilde{\rho v}_2,\widetilde{\rho v}_3)^T$.
While this is a common choice in ANN-based turbulence modeling, the momentum field lacks invariance with respect to scaling and rotation.
To this end, we also investigate the input features proposed by Novati et al. \cite{novati2021automating}, who used as input quantities the five invariants of the velocity gradient tensor $\lambda_{\nabla v}^i$, as given by Pope in \cite{pope1975more}, to embed Galilean invariance into their policy.

It is important to stress that in contrast to previous works, our policy acts on \textit{local} inputs only.
This means that the predictions for each element are based solely on the flow state inside of this respective element.
The predictions for each individual element are thus independent from the remaining flow field and do not require any information about the global flow state.
This makes this approach more suitable for realistic applications, where global flow statistics are generally very expensive to retrieve during runtime.
Moreover, models which rely on global flow statistics as inputs cannot be evaluated on flows where these statistics cannot be obtained in a straight-forward manner, e.g. due to the geometric complexity of the computational domain or where a global statistic simply does not exist.
The value network is designed analogously to the policy network, but employs two additional fully-connected layers, comprising 16 and a single neuron, respectively, to combine the elementwise predictions for the whole flow state to a single prediction for the state-value function.
This RL design can be interpreted as a multi-agent RL approach, where each element employs its own agent, but all agents share their weights.
Given the predictions of the policy, i.e. the actions of the agent, the new state of the environment is obtained by integrating the LES equations in time for $\Delta t_{RL}$. 

As overall optimization target of the RL task, we define the error in the spectrum of turbulent kinetic energy between the instantaneous LES $E_{LES}$ and the target spectrum $\overline{E}_{DNS}$.
In this work, we used the time-averaged spectrum of a previously computed DNS.
However, the RL framework allows to impose any high-level statistic as training objective.
To this end, also data reported in the literature, experimental data or even physical properties like the $k^{-5/3}$ energy decay in the inertial range could be used directly as optimization targets. 
Given these energy spectra, we compute the mean squared error for the wavenumbers $k$ up to $k_{max}$.
Here, $k_{max}$ is the maximum wavenumber considered for optimization and thus depends on the employed resolution of the LES.
Finally, we use the exponential function to normalize the resulting reward to the range $[-1,1]$, since a normalized reward can improve the training speed.
The reward function thus reads
\begin{equation}
  \rewardfunc(\state) = 2\:\exp\left(-\frac{1}{\rewardscale\:\wavenumber_{max}}\sum_{\wavenumber=1}^{\wavenumber_{max}}\left(\frac{\overline{\turbenergy}_{DNS}(\wavenumber)-\turbenergy_{LES}(\wavenumber)}{\tfilter{\turbenergy}_{DNS}(\wavenumber)}\right)^2\right)-1,
  \label{eq:reward}
\end{equation}
with $\alpha$ as a scaling factor.
This scaling factor controls the difficulty of the training task by balancing between providing the maximum reward for any action of the policy for $\alpha \rightarrow \infty$ and demanding impossible accuracy in the spectrum to escape the negative limit of the exponential function for $\alpha \rightarrow 0$.
It seems important to stress that the energy spectra only have to be computed to evaluate the reward function during training and are not required if the trained policy is later applied in practical simulations.
The proposed formulation of turbulence modeling as an MDP is summarized in \tabref{tab:rl_turb}.

A major advantage of the broader learning paradigm of RL in comparison to the common SL approach is that RL allows to incorporate a much broader range of prior information into the training.
The optimization target for the training, which is defined via the reward function, can be based on DNS data, experimental data, physical first principles or properties of the underlying discretization.
For SL, any change in the posed learning task, e.g. changing the input and output quantities for the ANN, would require to recompute the training dataset from DNS data.
This poses significant challenges, since it requires to store and process large amounts of time-resolved DNS data.



\subsection{Computational setup}
\label{sec:relexi}


\begin{table}[htb]
  \centering
  \begin{tabular}{lrrrr}
    \toprule
    Name   & $N$ & \#Elems & $k_{max}$ & $\alpha$ \\
    \midrule
    24 DOF &   5 &   $4^3$ &         9 &      0.4 \\
    32 DOF &   7 &   $4^3$ &        12 &      0.4 \\
    36 DOF &   5 &   $6^3$ &        13 &      0.4 \\
    48 DOF &   5 &   $8^3$ &        16 &      0.4 \\
    \bottomrule
  \end{tabular}
  \caption{Investigated configurations for the LES environments. The names are derived from the number of degrees of freedom (DOF) in each direction used for the simulation, which can be computed as the number of elements in each direction times $(N+1)$.}
  \label{tab:les_configs}
\end{table}

In RL, the training of the agent is an iterative process which does not require the perfect target quantities to be known, but finds suitable targets by itself in order to fulfill the overall goal defined by the reward function.
This means that no DNS data is required to compute input and output data pairs for a training dataset.
Instead, a number of LES runs has to be computed repeatedly to sample interactions between the agent and the environment.
Thus, like in most ML methods, training requires considerable hardware resources for running the LES and training the agent on the collected experience.
In this work, we use the Relexi framework proposed in \cite{kurz2022deep,kurz2022relexi} to train the agent by using the parallel computing resources provided by today's high-performance computing (HPC) systems.
The Relexi framework implements the RL training loop by means of the TensorFlow library \cite{abadi2016tensorflow} and its RL extension TensorFlow-Agents https://github.com/tensorflow/agents.
The framework allows to couple external flow solvers efficiently on high-performance computing systems with the SmartSim library \cite{partee2021using}, which is used to manage the simulations runs and implements the communication between the main application and the external solver.
The LES environments are simulated with the HPC flow solver FLEXI \cite{krais2021flexi}.
In each training iteration, Relexi starts several FLEXI simulations to sample interactions of the current policy with the LES environments.

With FLEXI we employ a high-order discontinuous Galerkin (DG) discretization of the compressible Navier-Stokes equations using a kinetic energy preserving split formulation of the fluxes on Legendre-Gauss-Lobatto interpolation nodes for stability as discussed by Flad and Gassner in \cite{flad2017use}.
We investigate the influence of the discretization on the RL agent by computing LES at different resolutions as listed in \tabref{tab:les_configs}.
For this, we also employ two different polynomial degrees, i.e. $N=5$ and $N=7$, which results in two different discretizations. 
The maximum wavenumber used for optimization $k_{max}$ is then chosen for each case according to the respective resolution employed.
Theoretically, a minimum of $n_{ppw}=\pi$ points per wavelength is required to resolve a wavenumber accurately with a polynomial basis.
For the polynomial degrees employed here, the resolution capabilities can be estimated as $n_{ppw}\approx 4$ \cite{gassner2011comparison}.
However, to increase the difficulty of the optimization task, we choose $k_{max}$ such that $2.6 \le n_{ppw} \le 3$.
This forces the RL algorithm to optimize over all represented wavenumbers.
 
All simulations are initialized with flow states that are obtained by projecting the high-fidelity DNS solution, which is obtained a priori, onto the resolution of the respective environment using an $L_2$-projection.
The forced HIT case has a Reynolds number of $\mathrm{Re}_{\lambda}\approx 180$ with respect to the Taylor microscale.
The initial states used for training are drawn from the filtered DNS evaluated at $t_{DNS}=3,4,5,6$ and a single flow state at $t_{DNS}=8$ is kept hidden for testing, i.e. to evaluate the policy's performance on unseen data.
The LES are then simulated for $\Delta t_{end}=5$ using the FLEXI solver, while the elementwise $C_s$ parameters are updated every $\Delta t_{RL}=0.1$.
The large-eddy turnover time with approximately $t'\approx 0.7$ acts as a characteristic timescale, which results in 7 characteristic timescales simulated in each LES run.
This is considered sufficient to incorporate long-term effects into the training process.

For the RL algorithm, we use the Adam optimizer \cite{kingma2014adam} with a learning rate of $10^{-4}$ for the policy and the value network.
All configurations were trained for 5 epochs in each training iteration, except the 48 DOF configuration, which was only trained for a single epoch.
Instead of using mini-batching, the gradients were computed with respect to all sampled experience.
The weighting factor between the policy and the value estimation loss is set to 0.5 and the clipping parameter to $\ppoclipping=0.2$.
No additional regularization was used and the originally proposed entropy regularization coefficient for PPO is set to zero.
Moreover, we chose a discount factor of $\gamma=0.995$.
To obtain stochastic predictions from the deterministic policy, we sample actions from a normal distribution which is determined by using the ANN's predictions as mean and a fixed standard deviation of 0.02, which corresponds to about 10\% of the theoretical $C_s=0.17$ suggested for Smagorinsky's model in the literature.

\section{Reinforcement Learning}
\label{sec:rl}

\subsection{A Brief Introduction}
\label{sec:rl_intro}

\begin{figure}
  \centering
  \includegraphics[width=0.7\linewidth]{RL_MDP.pdf}
  \caption{Schematic of the Markov decision process. The agent performs an action $a_t$ based on its policy $\sgeneralpolicy$ to interact with the environment. Consequently, the environment transitions into a new state $\state_{t+1}$ according to its transition function $\sgeneraltransition$. The resulting reward $\reward_{t+1}$ is specified by the reward function \mbox{$r_{t+1}=\rewardfunc\left(s_t,\state_{t+1}\right)$}, which is used to quantify how desirable a given state transition is. Starting from an initial environment state $\state_0$ this process is repeated until a final state $\state_n$ is reached. }
  \label{fig:MDP}
\end{figure}


In contrast to the supervised learning approach, reinforcement learning trains an agent by letting it interact with a dynamical environment in order to achieve a pre-defined goal.
This has the advantage that the dynamics of the environment are incorporated into the training process directly by design.
The interplay of the agent and the environment can be framed as a Markov decision process (MDP), as illustrated in \figref{fig:MDP}.
In a MDP, the environment is characterized by its current state $s_t\in\states$.
The agent observes that state and can perform a suitable action $a_t\in\actions$.
Here, $\states$ is the set of all possible environment states and $\actions$ is the set of all possible actions that can be performed by the agent.
The agent's actions are determined by its policy $\sgeneralpolicy$ that states which action the agent should perform given the environment's current state $\state_t$.%
\footnote{
  To keep the notation short, we use the abbreviated notation of $\sgeneralpolicy$ for $\policy_{\modelparam}(\action\sgiven\state=\state_{t})$ which describes the conditional probability of choosing action $\action$ given the current state $\state_t$.
The same holds for the transition function $\transition\left(\state'\sgiven\state=\state_{t},\action=\action_{t}\right)$, which will be abbreviated as $\sgeneraltransition$.
}.
The policy can be of any functional form, but for deep RL, the policy is typically an ANN with parameters $\modelparam$.
The agent's action causes the environment to change is state.
This new state is prescribed by the environment's transition function $\state_{t+1}\sample\sgeneraltransition$, which thus encodes the environment's dynamics.
With the state transition, the agent receives a reward $\reward_{t+1}$ that is determined by the reward function $\reward_{t+1}=\rewardfunc(\state_{t},\state_{t+1})$ and quantifies how desirable a certain state transition is.
The new state $s_{t+1}$ is then again observed by the agent, which performs another action $a_{t+1}$ as prescribed by its current policy.
Starting from some initial state $\state_0$ and performing actions until a final state $\state_n$, this results in a trajectory of states, actions and rewards termed an episode:
\begin{equation}
  \traj = \left\{ \left(\state_0,\action_0,\reward_1\right),\left(\state_1,\action_1,\reward_2\right),\:......\;,\left(\state_{n-1},\action_{n-1},\reward_{n}\right),\state_n\right\} \eqnperiod
  \label{eq:trajectory4}
\end{equation}
The goal of an RL algorithm is to establish an optimization problem that allows to find the optimal policy $\policy^{\opt}$, which maximizes the expected return along an episode
\begin{equation}
  \return(\tau) = \sum_{t=1}^{n} \discount^t r_{t} \eqnperiod
\end{equation}
Here, $\discount$ denotes the discount factor that can be used to balance the importance of short-term and long-term rewards.
In deep RL, finding the optimal policy is equivalent to finding the set of optimal model parameters $\modelparam^{\opt}$ for the employed ANN.

For each state $\state$, the \textit{state-value function} describes the total return which can be expected starting from state $\state_t$ and following a specific policy $\policy$ from there on.
This state-value function can thus be written as
\begin{equation}
  \valuefunction^\policy \left(\state\right) = \expectation\left[\sum_{k=0}^\infty \discount^k r_{t+k}\given\state_t=\state\right] \eqnperiod
  \label{eq:valuefunction}
\end{equation}
Similarly, an \textit{action-value function} or \textit{Q-function} can be determined, which gives the expected return when starting from state $\state_t$ performing action $\action_t$ and following the policy $\policy$ from there on, which reads
\begin{equation}
  \qfunction^\policy \left(\state,\action\right) = \expectation\left[\sum_{k=0}^\infty \discount^k r_{t+k}\given\state_t=\state, \action_t=\action\right] \eqnperiod
  \label{eq:qfunction}
\end{equation}
Based on \eqref{eq:valuefunction} and \eqref{eq:qfunction}, one can define the \textit{advantage function}
\begin{equation}
  \ppoadvantage^\policy \left(\state,\action\right) =  \qfunction^\policy \left(\state,\action\right) - \valuefunction^\policy \left(\state\right) \eqncomma
  \label{eq:advantagefunction}
\end{equation}
which quantifies whether taking action $a_t$ in state $s_t$ increases or decreases the expected return in comparison to performing the action prescribed by the current policy.

Solving a given problem with RL thus requires that the problem is casted into an MDP by a domain expert, i.e. defining the environment's possible states $\states$, its transition function $\sgeneraltransition$, the agent's action space $\actions$, the reward function $\rewardfunc(s_t,s_{t+1})$ and, finally, the ANN architecture used for the policy $\sgeneralpolicy$.
With these definitions in place, a suitable RL algorithm can be applied to find a favorable policy.
Each distinct RL algorithm prescribes how interactions of the agent and the environment are collected and how this sampled experience can be used to optimize the policy such that the expected future return is increased.
RL algorithms differ for instance in terms of sample efficiency and whether they allow for continuous state and action spaces.
In the following, we use proximal policy optimization (PPO) as our RL algorithm of choice, which belongs to the class of policy gradient methods.





\subsection{Policy Gradient Methods}
\label{sec:vpg}


The key idea of policy gradient methods is to optimize the policy directly instead of learning the Q-function in \eqref{eq:qfunction} and inferring the policy implicitly from it.
To this end, policy gradient methods derive a gradient estimator that gives the direction in which the model parameters $\modelparam$ have to be changed in order to increase the expected return.
Given this gradient, the model parameters can be updated with a suitable gradient-ascent algorithm.
Following the \textit{policy gradient theorem}, see e.g. \cite{sutton2018reinforcement}, the gradient estimator can be written as
\begin{equation}
  \policygradient = \expectation\Big[\qfunction^{\policy}\left(\state,\action\right) \,\nabla_{\modelparam}\log \policy_{\modelparam}\left(\action\sgiven\state \right) \Big],
  \label{eq:pg_gradient}
\end{equation}
and can be obtained by differentiating the corresponding loss function
\begin{equation}
  \loss^{VPG}(\modelparam) = \expectation\Big[ \qfunction^{\policy}\left(\state,\action\right) \,  \log \policy_{\modelparam}\left(\action\sgiven\state \right)\Big],
  \label{eq:pg_loss}
\end{equation}
with respect to $\modelparam$.
Since the policy and the environment are in general stochastic, the gradient estimator for the optimization is defined by means of the expectation $\expectation\left[\cdot\right]$.
However, obtaining the exact expectation is prohibitive for practical applications.
Instead, the gradient estimator is approximated by sampling $N$ trajectories of experience on the current policy and computing the approximated gradient as mean over the sampled trajectories
\begin{equation}
  \hat{\policygradient} = \frac{1}{N}\sum_{i=1}^N\left[\return\left(\traj^{(i)}\right) \ \sum_{t=0}^n  \nabla_{\modelparam}\log \policy_{\modelparam}\left(\action_t\sgiven\state_t \right)\right] \eqnperiod
  \label{eq:pg_approx_gradient}
\end{equation}
Here, the discounted return along a trajectory $\return(\traj)$ is used as an approximation to the exact Q-function.
Interestingly, the dynamics of the environment, i.e. its transition function $\sgeneraltransition$, do not appear in the gradient estimator or in the overall optimization formulation.
Instead, the dynamics of the environment are incorporated implicitly by the sampled experience.
This avoids to differentiate the environment dynamics with respect to the policy parameters, which is infeasible for most tasks.

The training process of a policy gradient method then works as follows.
First, multiple episodes of experience are sampled with the current policy.
Based on this experience, the policy can be optimized in a second step with the gradient estimator and a suitable gradient-ascent algorithm.
These two steps are then repeated, until the policy has converged.
These building blocks form the vanilla policy gradient (VPG) method.

\subsection{Proximal Policy Optimization}
\label{sec:ppo}

The proximal policy optimization (PPO) method \cite{schulman2017proximal} introduces several improvements over the original vanilla policy gradient (VPG) method to improve the stability of the training.
For a clear and concise summary of the PPO algorithm, we recommend \cite{notter2021hierarchical}.

The first major improvement of PPO is to reduce the variance of the gradient estimator.
This reduces the amount of samples required for an accurate approximation of the gradient or a better gradient estimator for a given amount of samples.
To this end, a baseline can be added to the gradient estimator in \eqref{eq:pg_approx_gradient}, which has been shown to not introduce a bias.
A natural choice is to replace the Q-function by the advantage function from \eqref{eq:advantagefunction}.
However, the advantage function relies on the state-value function $\valuefunction^{\policy}(s)$, which is typically unknown.
Therefore, the PPO algorithm uses an additional ANN $\hat{\valuefunction}_{\valueparam}(s)$ with weights $\valueparam$, which is trained to approximate the state-value function.
Moreover, the return along the trajectory, which approximates the Q-function, is replaced by a \textit{return-to-go} $R_t(\tau) = \sum_{k=t}^n \gamma^{k-t} r_k$ such that each action is only associated with reward that is collected after the action is taken.
The approximate advantage function at step $t$ thus reads
\begin{equation}
  \hat{\ppoadvantage}_t = \left(\sum_{k=t}^n \gamma^{k-t} r_k\right) - \hat{\valuefunction}_{\valueparam}(s_t).
  \label{eq:ppo_advantage}
\end{equation}

A major drawback of VPG is that the training with the VPG method is often found to be unstable, since the policy updates can become arbitrarily large.
Large policy updates imply the risk of deteriorating the policy's performance in a single update step if the gradient estimator is not sufficiently accurate or the step size is too large.
The PPO method increases the stability of the training process by constraining the maximum change of the policy in a single step. %
Schulman et al. \cite{schulman2017proximal} propose two different approaches to limit the updates of the policy.
Firstly, a penalty term can be added to \eqref{eq:pg_loss} based on the Kullback-Leibler divergence between the old and the new policy.
This introduces the incentive to avoid large changes in the policy in a single training step.
The other approach is to replace the loss function in \eqref{eq:pg_loss} by a clipped surrogate objective
\begin{equation}
  \loss^{CLIP}(\modelparam) = \expectation\left[\min\left( \probratio\left(\modelparam\right)\hat{\ppoadvantage},\mathrm{clip}\left(\probratio\left(\modelparam\right),1-\eps,1+\eps\right)\hat{\ppoadvantage}   \right)   \right],
  \label{eq:clipping_loss}
\end{equation}
with $\eps$ as a hyperparameter and with $\probratio(\modelparam)$ as the probability ratio
\begin{equation}
  \probratio\left(\modelparam\right)=\frac{\policy_{\modelparam}\left(\action\sgiven\state\right)}{\policy_{\modelparam_{old}}\left(\action\sgiven\state\right)},
  \label{eq:prob_ratio}
\end{equation}
which describes the change between the old and the new policy.
In \eqref{eq:clipping_loss}, the clip function clips the probability ratio $\probratio(\modelparam)$ to the interval $\left[1-\epsilon,1+\epsilon\right]$ to limit the change of the policy in a single training step.
Similarly to \eqref{eq:pg_approx_gradient}, the expectation in \eqref{eq:clipping_loss} is then approximated by sampling trajectories with the current policy.


\section{Dynamic Smagorinsky Model}
\label{app:dynsmago}

Germano et al. proposed in \cite{germano1991dynamic} an improvement to the standard Smagorinsky model by setting the Smagorinsky constant $C_s$ not to a fixed value set a priori, but rather to adapt it dynamically in space and time dependent on the current state of the flow.
This dynamic procedure is based on applying a test filter $\widehat{(\cdot)}$ to the solution.
The LES solution can then be used to compute the resolved stress tensor
\begin{equation}
  L_{ij} = \widehat{\widetilde{u}_i \widetilde{u}_j} - \widehat{\widetilde{u}}_i \widehat{\widetilde{u}}_j,
\end{equation}
which entails the subgrid stresses induced by the test filter.
Germano's identity finds a correlation between those resolved subgrid stresses and the unknown subgrid stress induced by the LES filter.
Lilly \cite{lilly1992proposed} proposed a least-squares approach to compute $C_s^2$ such that Germano's identity is optimally obeyed, which gives
\begin{equation}
  C_s^2 = \frac{\langle L_{ij}M_{ij}\rangle_{avg}}{\langle M_{ij}M_{ij} \rangle_{avg}}.
  \label{eq:dynsmago_ls}
\end{equation}
Here, $\langle\cdot\rangle_{avg}$ denotes some averaging operation to avoid extremely large coefficients and thus to stabilize the model.
The tensor $M_{ij}$ is shorthand for the expression
\begin{equation}
  M_{ij} = 2 \Delta^2 \widehat{\sqrt{2\tilde{S}_{kl}\tilde{S}_{kl}}\,\tilde{S}_{ij}}  - 2 \widehat{\Delta}^2 \sqrt{2\widehat{\tilde{S}}_{kl}\widehat{\tilde{S}}_{kl}}\, \widehat{\tilde{S}}_{ij},
\end{equation}
with $\widehat{\Delta}$ as the filter width of the test filter and with the rate-of-strain tensor on the test filter level $\widehat{\tilde{S}}$ computed analogously to \eqref{eq:smago} based on the test-filtered velocity field $\widehat{\tilde{u_i}}$.

In the DG context, the dynamic procedure is applied in an elementwise fashion.
As test filter $\widehat{(\cdot)}$ with filter width $\widehat{\Delta}$, we apply a modal cut-off filter to the solution polynomial with a degree of $N_{test}=2$ and $N_{test}=3$ for the cases employing a polynomial degree of $N=5$ and $N=7$, respectively.
Moreover, the averaging operator $\langle\cdot\rangle_{avg}$ for the numerator and denominator in \eqref{eq:dynsmago_ls} is chosen as an average in each element yielding a single coefficient $C_s^2$ for each element.
The resulting eddy-viscosity is then clipped to the range $\mu_t \in [-100\mu,1000\mu]$ with respect to the physical viscosity, since the unclipped model produced from time to time a few huge values for $C_s^2$, which deteriorated the solution quite drastically.
The clipping range itself seemed to have only very limited influence on the model's behavior in our tests and provided very similar results for a wide range of different clipping intervals.

