\section{Dynamic Smagorinsky Model}
\label{app:dynsmago}

Germano et al. proposed in \cite{germano1991dynamic} an improvement to the standard Smagorinsky model by setting the Smagorinsky constant $C_s$ not to a fixed value set a priori, but rather to adapt it dynamically in space and time dependent on the current state of the flow.
This dynamic procedure is based on applying a test filter $\widehat{(\cdot)}$ to the solution.
The LES solution can then be used to compute the resolved stress tensor
\begin{equation}
  L_{ij} = \widehat{\widetilde{u}_i \widetilde{u}_j} - \widehat{\widetilde{u}}_i \widehat{\widetilde{u}}_j,
\end{equation}
which entails the subgrid stresses induced by the test filter.
Germano's identity finds a correlation between those resolved subgrid stresses and the unknown subgrid stress induced by the LES filter.
Lilly \cite{lilly1992proposed} proposed a least-squares approach to compute $C_s^2$ such that Germano's identity is optimally obeyed, which gives
\begin{equation}
  C_s^2 = \frac{\langle L_{ij}M_{ij}\rangle_{avg}}{\langle M_{ij}M_{ij} \rangle_{avg}}.
  \label{eq:dynsmago_ls}
\end{equation}
Here, $\langle\cdot\rangle_{avg}$ denotes some averaging operation to avoid extremely large coefficients and thus to stabilize the model.
The tensor $M_{ij}$ is shorthand for the expression
\begin{equation}
  M_{ij} = 2 \Delta^2 \widehat{\sqrt{2\tilde{S}_{kl}\tilde{S}_{kl}}\,\tilde{S}_{ij}}  - 2 \widehat{\Delta}^2 \sqrt{2\widehat{\tilde{S}}_{kl}\widehat{\tilde{S}}_{kl}}\, \widehat{\tilde{S}}_{ij},
\end{equation}
with $\widehat{\Delta}$ as the filter width of the test filter and with the rate-of-strain tensor on the test filter level $\widehat{\tilde{S}}$ computed analogously to \eqref{eq:smago} based on the test-filtered velocity field $\widehat{\tilde{u_i}}$.

In the DG context, the dynamic procedure is applied in an elementwise fashion.
As test filter $\widehat{(\cdot)}$ with filter width $\widehat{\Delta}$, we apply a modal cut-off filter to the solution polynomial with a degree of $N_{test}=2$ and $N_{test}=3$ for the cases employing a polynomial degree of $N=5$ and $N=7$, respectively.
Moreover, the averaging operator $\langle\cdot\rangle_{avg}$ for the numerator and denominator in \eqref{eq:dynsmago_ls} is chosen as an average in each element yielding a single coefficient $C_s^2$ for each element.
The resulting eddy-viscosity is then clipped to the range $\mu_t \in [-100\mu,1000\mu]$ with respect to the physical viscosity, since the unclipped model produced from time to time a few huge values for $C_s^2$, which deteriorated the solution quite drastically.
The clipping range itself seemed to have only very limited influence on the model's behavior in our tests and provided very similar results for a wide range of different clipping intervals.
