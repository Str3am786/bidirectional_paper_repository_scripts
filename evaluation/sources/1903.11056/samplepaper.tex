% This is samplepaper.tex, a sample chapter demonstrating the
% LLNCS macro package for Springer Computer Science proceedings;
% Version 2.20 of 2017/10/04
%
\documentclass[runningheads]{llncs}
%
\usepackage{graphicx}
% Used for displaying a sample figure. If possible, figure files should
% be included in EPS format.
%
% If you use the hyperref package, please uncomment the following line
% to display URLs in blue roman font according to Springer's eBook style:
% \renewcommand\UrlFont{\color{blue}\rmfamily}

\begin{document}
%
\title{RowHammer and Beyond}
%
%\titlerunning{Abbreviated paper title}
% If the paper title is too long for the running head, you can set
% an abbreviated paper title here
%
\author{Onur Mutlu}
%

% First names are abbreviated in the running head.
% If there are more than two authors, 'et al.' is used.
%
\institute{ETH Z{\"u}rich and Carnegie Mellon University\\ 
\email{onur.mutlu@inf.ethz.ch}}
%
\maketitle              % typeset the header of the contribution
%

\begin{abstract}
We will discuss the RowHammer problem in DRAM, which is a prime (and
likely the first) example of how a circuit-level failure mechanism in
Dynamic Random Access Memory (DRAM) can cause a practical and
widespread system security vulnerability.  RowHammer is the phenomenon
that repeatedly accessing a row in a modern DRAM chip predictably
causes errors in physically-adjacent rows. It is caused by a hardware
failure mechanism called read disturb errors.  Building on our initial
fundamental work that appeared at ISCA 2014, Google Project Zero
demonstrated that this hardware phenomenon can be exploited by
user-level programs to gain kernel privileges. Many other recent works
demonstrated other attacks exploiting RowHammer, including remote
takeover of a server vulnerable to RowHammer. We will analyze the root
causes of the problem and examine solution directions. We will also
discuss what other problems may be lurking in DRAM and other types of
memories, e.g., NAND flash and Phase Change Memory, which can
potentially threaten the foundations of reliable and secure systems,
as the memory technologies scale to higher densities.
\end{abstract}


\section{Summary}
  
As memory scales down to smaller technology nodes, new failure
mechanisms emerge that threaten its correct
operation~\cite{mutlu-imw13,onur-date17}. If such failures are not
anticipated and corrected, they can not only degrade system
reliability and availability but also, even more importantly, open up
new security vulnerabilities: a malicious attacker can exploit the
exposed failure mechanism to take over an entire system. As such, new
failure mechanisms in memory can become practical and significant
threats to system security.

In this keynote talk, based on our ISCA 2014
paper~\cite{rowhammer-isca2014}, we introduce the RowHammer problem in
DRAM, which is a prime (and likely the first) example of a real
circuit-level failure mechanism that causes a practical and widespread
system security vulnerability. RowHammer, as it is now popularly
referred to, is the phenomenon that repeatedly accessing a row in a
modern DRAM chip causes bit flips in physically-adjacent rows at
consistently predictable bit locations. It is caused by a hardware
failure mechanism called {\em DRAM disturbance errors}, which is a
manifestation of circuit-level cell-to-cell interference in a scaled
memory technology. Specifically, when a DRAM row is opened (i.e.,
activated) and closed (i.e., precharged) repeatedly (i.e., {\em
  hammered}), enough times within a DRAM refresh interval, one or more
bits in physically-adjacent DRAM rows can be flipped to the wrong
value. Using an FPGA-based DRAM testing
infrastructure~\cite{dram-isca2013,softmc}, we tested 129 DRAM modules
manufactured by three major manufacturers in seven recent years
(2008--2014) and found that 110 of them exhibited RowHammer errors,
the earliest of which dates back to 2010. Our ISCA 2014
paper~\cite{rowhammer-isca2014} provides a detailed and rigorous
analysis of various characteristics of RowHammer, including its data
pattern dependence, repeatability of errors, relationship with leaky
cells, and various circuit-level causes of the phenomenon.
  
We demonstrate that a very simple user-level
program~\cite{rowhammer-isca2014,safari-rowhammer} can reliably and
consistently induce RowHammer errors in commodity AMD and Intel
systems using vulnerable DRAM modules. We released the source code of
this program~\cite{safari-rowhammer}, which Google Project Zero later
enhanced~\cite{google-rowhammer-test}. Using our user-level RowHammer
program, we showed that both read and write accesses to memory can
induce bit flips, all of which occur in rows other than the one that
is being accessed. Since different DRAM rows are mapped to different
software pages, our user-level program could reliably corrupt specific
bits in pages belonging to other programs. As a result, RowHammer
errors can be exploited by a malicious program to breach memory
protection and compromise the system. In fact, we hypothesized, in our
ISCA 2014 paper, that our user-level program, with some engineering
effort, could be developed into a {\em disturbance attack} that
injects errors into other programs, crashes the system, or hijacks
control of the system.

RowHammer exposes a {\em security threat} since it leads to a serious
breach of memory isolation: an access to one memory row (e.g., an OS
page) predictably modifies the data stored in another row (e.g.,
another OS page). Malicious software, which we call {\em disturbance
  attacks}~\cite{rowhammer-isca2014}, or {\em RowHammer attacks}, can
be written to take advantage of these disturbance errors to take over
an entire system. Inspired by our ISCA 2014 paper's fundamental
findings, researchers from Google Project Zero demonstrated in 2015
that RowHammer can be effectively exploited by user-level programs to
gain kernel privileges on real
systems~\cite{google-project-zero,google-rh-blackhat}. Tens of other
works since then demonstrated other attacks exploiting
RowHammer. These include remote takeover of a server vulnerable to
RowHammer via JavaScript code execution~\cite{rowhammer-js}, takeover
of a victim virtual machine by another virtual machine running on the
same system~\cite{flip-feng-shui}, takeover of a mobile device by a
malicious user-level application that requires no
permissions~\cite{drammer}, takeover of a mobile system by triggering
RowHammer using the WebGL interface on a mobile GPU~\cite{glitch-vu},
takeover of a remote system by triggering RowHammer through the Remote
Direct Memory Access (RDMA) protocol~\cite{throwhammer,nethammer}, and
various other attacks
(e.g.,~\cite{cloudflops,dedup-est-machina,anotherflip,qiao2016new,bhattacharya2016curious,jang2017sgx,poddebniak2018attacking,aga2017good,tatar2018defeating,pessl2016drama}). Thus,
RowHammer has widespread and profound real implications on system
security, as it destroys memory isolation on top of which modern
system security principles are built.


%These include remote takeover of a server vulnerable to
%RowHammer, takeover of a victim virtual machine by another virtual
%machine running on the same system, takeover of a mobile device by
%a malicious user-level application that requires no permissions,
%takeover of a mobile system quickly by triggering RowHammer using a mobile GPU,
%and takeover of a remote system by triggering RowHammer on it through the Remote Direct %Memory Access (RDMA) protocol.

 

%As described above, RowHammer has recently been the subject of many
%popular analyses and discussions on hardware-induced security
%problems~\cite{rowhammer-arxiv16,rowhammer-wikipedia,rh-discuss,rh-twitter,rh-zdnet1,rowhammer-thirdio,anvil,rowhammer-js,drammer,flip-feng-shui,dedup-est-machina,google-rh-blackhat}
%as well as the prime vulnerability exploited by various software-level
%attacks that rely on no permissions or software
%vulnerabilities~\cite{rowhammer-js,drammer,flip-feng-shui,dedup-est-machina}. 

%\hspace{0.1in}As a result of the severity of the security problem caused by RowHammer, several major system
%manufacturers increased DRAM refresh rates to reduce the probability
%of occurrence of
%RowHammer~\cite{rh-apple,rh-hp,rh-lenovo,rh-cisco}. And, multiple
%emory test programs have been augmented to test for RowHammer
%errors~\cite{rh-passmark,rh-futureplus,drammer}. Some
%recent reports suggest that even state-of-the-art DDR4 DRAM chips are
%vulnerable to RowHammer~\cite{rowhammer-thirdio}, which suggests that
%the security vulnerabilities might continue in the current generation
%of DRAM chips as well. Since cells are getting closer to each other with every technology %node reduction, RowHammer is likely a problem that will stay with us for the long-term -- %its fundamental causes are getting worse with shrinking technology. 


We provide a wide variety of solutions, both {\em immediate} and {\em
  longer-term}, to RowHammer, starting from our ISCA 2014
paper~\cite{rowhammer-isca2014}. A popular {\em immediate} solution we
describe and analyze, is to increase the refresh rate of memory such
that the probability of inducing a RowHammer error before DRAM cells
get refreshed is reduced. Several major system manufacturers have
adopted this solution and released security patches that increased
DRAM refresh rates (e.g.,~\cite{rh-apple,rh-hp,rh-lenovo,rh-cisco}) in
memory controllers deployed in the field. While this solution is
practical and effective in reducing the vulnerability, assuming the
refresh rate is increased enough to avoid the vulnerability, it has
the significant drawbacks of increasing energy/power consumption,
reducing system performance, and degrading quality of service
experienced by user programs. Our paper shows that the refresh rate
needs to be increased by 7X if we want to eliminate {\em every single}
RowHammer-induced error we saw in our tests of 129 DRAM modules. Since
DRAM refresh is already a significant
burden~\cite{raidr,dram-isca2013,darp-hpca2014,kang-memforum2014,samira-sigmetrics14,avatar-dsn15,khan-dsn16,patel2017reach,vrl-dram}
on energy, performance, and QoS, increasing it by any significant
amount would only exacerbate the problem. Yet, increased refresh rate
is likely the most practical {\em immediate} solution to RowHammer
that can protect vulnerable chips that are already deployed in the
field.

%\hspace{0.1in}Our ISCA 2014 paper~\cite{rowhammer-isca2014} proposes
%and analyzes seven longer-term countermeasures to the RowHammer
%problem.  The first six solutions are: 1) making better DRAM chips
%that are not vulnerable, 2) using error correcting codes (ECC), 3)
%increasing the refresh rate (as discussed above), 4) remapping
%RowHammer-prone cells after manufacturing, 5) remapping/retiring
%RowHammer-prone cells at the user level during operation, 6)
%accurately identifying hammered rows during runtime and refreshing
%their neighbors. None of these first six solutions are very desirable
%as they come at significant power, performance, or cost
%overheads~\cite{rowhammer-isca2014}.

After describing and analyzing six solutions to RowHammer, our ISCA
2014 paper shows that the long-term solution to RowHammer can actually
be simple and low cost. We introduce a new idea, called {\em PARA
  (Probabilistic Adjacent Row Activation)}: when the memory controller
closes a row (after it was activated), with a very low probability, it
refreshes the adjacent rows. The probability value is a parameter
determined by the system designer or provided programmatically, if
needed, to trade off between performance overhead and vulnerability
protection guarantees. We show that this solution is very effective:
it eliminates the RowHammer vulnerability, providing much higher
reliability guarantees than modern hard disks provide today, while
requiring no storage cost and having negligible performance and energy
overheads~\cite{rowhammer-isca2014}. Variants of this solution are
currently being adopted in DRAM chips and memory
controllers~\cite{x210-github,x210-rh-ss}.

The RowHammer problem leads to a new mindset that has enabled a renewed
interest in hardware security research: real memory chips are
vulnerable, in a simple and widespread manner, and this causes real
security problems.  We believe the RowHammer problem will worsen over
time since DRAM cells are getting closer to each other with technology
scaling. Other similar vulnerabilities may also be lurking in DRAM and
other types of memories, e.g., NAND flash memory or Phase Change
Memory, that can potentially threaten the foundations of secure
systems~\cite{onur-date17}. Our work advocates a principled
system-memory co-design approach to memory reliability and security
research that can enable us to better anticipate and prevent such
vulnerabilities.

%\keywords{First keyword  \and Second keyword \and Another keyword.}
%\end{abstract}
%
%
%
%\renewcommand*\abstractname{Significance And Impact}
%\begin{abstract}


\section{Significance, Impact and the Future}
  
RowHammer has spurred significant amount of research and industry
attention since its publication in 2014. Our ISCA 2014
paper~\cite{rowhammer-isca2014} is the first to experimentally and scientifically 
demonstrate the RowHammer vulnerability, its characteristics, and its prevalence in
real DRAM chips. RowHammer is a prime (and likely the first) example
of a hardware failure mechanism that causes a practical and widespread
system security vulnerability. Thus, the implications of RowHammer and
our ISCA 2014 paper on systems security is tremendous, both in the
short term and the long term: it is the first work we know of that
shows that a real reliability problem in one of the ubiquitous
general-purpose hardware components (DRAM chips) can cause practical
and widespread system security vulnerabilities.

Since its publication in 2014, RowHammer has already had significant
real-world impact on both industry and academia in at least four
directions. These directions will continue to exert long-term impact
for RowHammer, as memory cells continue to get closer to each other
while the technology scaling of memory continues.

First, our work has inspired many researchers to exploit RowHammer to
devise new attacks. As mentioned earlier, tens of papers were written
in top security venues that demonstrate various practical attacks
exploiting RowHammer
(e.g.,~\cite{cloudflops,dedup-est-machina,anotherflip,qiao2016new,bhattacharya2016curious,jang2017sgx,aga2017good,pessl2016drama,rowhammer-js,flip-feng-shui,drammer,glitch-vu}). These
attacks started with Google Project Zero's first work in
2015~\cite{google-project-zero,google-rh-blackhat} and they continue
to this date, with the latest ones that we know of being published in
Summer
2018~\cite{poddebniak2018attacking,nethammer,throwhammer,tatar2018defeating}. We
believe there is a lot more to come in this direction: as systems
security researchers understand more about RowHammer, and as the
RowHammer phenomenon continues to fundamentally affect memory chips
due to technology scaling problems~\cite{onur-date17}, researchers and
practitioners will develop different types of attacks to exploit
RowHammer in various contexts and in many more creative ways. Some
recent reports suggest that new-generation DDR4 DRAM chips are
vulnerable to RowHammer~\cite{rowhammer-thirdio,pessl2016drama,aga2017good,aichinger2015ddr}, so the fundamental security research
on RowHammer is likely to continue into the future.

Second, due to its prevalence in real DRAM chips, as demonstrated in
our ISCA 2014 paper, RowHammer has become a popular
phenomenon~\cite{rowhammer-wikipedia,rh-discuss,rh-twitter,rh-zdnet1,rowhammer-thirdio,google-rh-blackhat,rh-passmark,rh-futureplus,arstechnica_ddr4-rh},
which, in turn, has helped make hardware security even more "mainstream" in popular media and
the broader security community. It showed that hardware reliability 
problems can be very serious security threats that
have to be defended against. A well-read article from the Wired
magazine, all about RowHammer, is entitled "Forget Software -- Now
Hackers are Exploiting Physics!"~\cite{wired-rh}, indicating the shift
of mindset towards very low-level hardware security vulnerabilities in the popular
mainstream security community. Many other popular articles in press
have been written about RowHammer, many of which pointing to the
our ISCA 2014 work~\cite{rowhammer-isca2014} as the first demonstration and scientific analysis of the RowHammer problem. 
Showing that hardware reliability problems can be serious security threats and pulling them to the popular discussion
space, and thus influencing the mainstream discourse, creates a very long
term impact for the RowHammer problem.

Third, our work inspired many solution and mitigation techniques for
RowHammer from both researchers and industry practitioners. {\em
  Apple} publicly mentioned, in their critical security release for
RowHammer, that they increased the memory refresh rates due to the
"original research by Yoongu Kim et
al. (2014)"~\cite{rh-apple}. Memtest86 program was updated, including
a RowHammer test, acknowledging our ISCA 2014
paper~\cite{rh-passmark}. Many academic works developed solutions to
RowHammer, working from our original research
(e.g.,~\cite{anvil,moin-rowhammer,anotherflip,seyedzadeh2017counter,brasser2016can,irazoqui2016mascat,son2017making,gomez2016dram,van2018guardion,lee2018twice}). Multiple
industrial solutions (e.g.,~\cite{x210-github,x210-rh-ss}) were
inspired by our new solution to RowHammer, Probabilistic Adjacent Row
Activation (PARA). We believe such solutions will continue to be generated in both academia and industry, extending RowHammer's impact into the very long term.

Fourth, and perhaps most importantly, RowHammer enabled a shift of
mindset among mainstream security researchers: general-purpose
hardware is fallible (in a very widespread manner) and its problems
are actually exploitable. This shift of mindset enabled many systems
security researchers to examine hardware in more depth and understand
its inner workings and vulnerabilities better. We believe it is no
coincidence that two of the groups that concurrently discovered the
Meltdown~\cite{lipp2018meltdown} and Spectre~\cite{kocher2018spectre}
vulnerabilities (Google Project Zero and TU Graz InfoSec) have heavily
worked on RowHammer attacks before. We believe this shift in mindset,
enabled in good part by the existence and prevalence of RowHammer,
will continue to be very be important for discovering and solving
other potential vulnerabilities that may appear as a result of both
technology scaling and hardware design.


\section{Other Potential Vulnerabilities}

We believe that, as memory technologies scale to higher densities,
other problems may start appearing (or may already be going unnoticed)
that can potentially threaten the foundations of secure systems. There
have been recent large-scale field studies a well as small-scale
controlled studies of real memory errors on real devices and systems,
showing that both DRAM and NAND flash memory technologies are becoming
less reliable~\cite{superfri14,justin-memerrors-dsn15,dram-field-analysis2,dram-field-analysis3,dram-field-analysis4,justin-flash-sigmetrics15,flash-field-analysis2,cai2017errors,cai2017error,luo2018improving,luo2018heatwatch,cai-date12,cai-hpca15,mutlu-imw13,patel2017reach,onur-date17}. As detailed experimental
analyses of real DRAM and NAND flash chips show, both technologies are
becoming much more vulnerable to cell-to-cell interference
effects~\cite{superfri14,rowhammer-isca2014,cai-dsn15,cai-sigmetrics14,cai-iccd13,cai-date12,cai-date13,flash-fms-talk,yixin-jsac16,cai-hpca17,cai2017errors,cai2017error,mutlu-imw13,onur-date17}, data retention is becoming significantly more
difficult in both technologies~\cite{raidr,samira-sigmetrics14,dram-isca2013,khan-dsn16,avatar-dsn15,darp-hpca2014,kang-memforum2014,mandelman-jrd02,cai-hpca15,cai-iccd12,warm-msst15,cai-date12,cai-date13,cai-itj2013,flash-fms-talk,memcon-cal16,cai2017errors,cai2017error,luo2018improving,luo2018heatwatch,superfri14,mutlu-imw13}, and error variation
within and across chips is increasingly prominent~\cite{dram-isca2013,aldram,kevinchang-sigmetrics16,dram-process-variation-3,cai-date12,cai-date13,lee2017design,kim2018solar,kim2018dram,kim2019drange}.  Emerging memory technologies~\cite{mutlu-imw13,meza-weed13}, such as Phase-Change Memory~\cite{pcm-isca09,zhou-isca09,moin-isca09,moin-micro09,wong-pcm,raoux-pcm,pcm-ieeemicro10,pcm-cacm10,justin-taco14,rbla},
STT-MRAM~\cite{chen-ieeetmag10,kultursay-ispass13}, and
RRAM/ReRAM/
memristors~\cite{wong-rram} are likely to exhibit similar
and perhaps even more exacerbated reliability issues. We believe, if
not carefully accounted for and corrected, these reliability problems
may surface as security problems as well, as in the case of RowHammer,
especially if the technology is employed as part of the main memory
system that is directly exposed to user-level programs. We believe
future work examining these vulnerabilities, among others, is
promising for both fixing the vulnerabilities and enabling the
effective scaling of memory technology.


\section*{Acknowledgments}

This short paper and the associated keynote talk are heavily based on
two previous papers we have written on RowHammer, one that first
scientifically introduced and analyzed the phenomenon in ISCA 2014~\cite{rowhammer-isca2014} and
the other that provides an analysis and future outlook on
RowHammer~\cite{onur-date17}. They are a result of the research done
together with many students and collaborators over the course of the
past 7-8 years. In particular, three PhD theses have shaped the
understanding that led to this work. These are Yoongu Kim's thesis
entitled ``Architectural Techniques to Enhance DRAM
Scaling''~\cite{yoongu-thesis}, Yu Cai's thesis entitled ``NAND Flash
Memory: Characterization, Analysis, Modeling and
Mechanisms''~\cite{yucai-thesis} and his continued follow-on work
after his thesis, summarized in~\cite{cai2017errors,cai2017error}, and
Donghyuk Lee's thesis entitled ``Reducing DRAM Latency at Low Cost by
Exploiting Heterogeneity''~\cite{donghyuk-thesis-arxiv16}. We also
acknowledge various funding agencies (NSF, SRC, ISTC, CyLab) and
industrial partners (AliBaba, AMD, Google, Facebook, HP Labs, Huawei,
IBM, Intel, Microsoft, Nvidia, Oracle, Qualcomm, Rambus, Samsung,
Seagate, VMware) who have supported the presented and other related
work in my group generously over the years. The first version of this
talk was delivered at a CMU CyLab Partners Conference in September
2015. Another version of the talk was delivered as part of an Invited
Session at DAC 2016, with a collaborative accompanying paper entitled
``Who Is the Major Threat to Tomorrow’s Security? You, the Hardware
Designer''~\cite{dac-invited-paper16}. The most recent version is the
invited talk given at the Top Picks in Hardware and Embedded Security
workshop, co-located with ICCAD 2018~\cite{topinhes-workshop-url},
where RowHammer was selected as a Top Pick among hardware and embedded
security papers published between 2012-2017. I would like to also
thank Christina Giannoula for her help in preparing this manuscript.


% defenses: \cite{anvil, moin-rowhammer, anotherflip,seyedzadeh2017counter}
% attacks: \cite{google-project-zero, google-rh-blackhat,rowhammer-js,drammer,flip-feng-shui,glitch-vu,throwhammer,nethammer,cloudflops,dedup-est-machina,anotherflip,qiao2016new,bhattacharya2016curious}
% pop culture: \cite{rowhammer-wikipedia,rh-discuss,rh-twitter,rh-zdnet1,rowhammer-thirdio,google-rh-blackhat,rh-passmark,rh-futureplus}
% DDR4: \cite{rowhammer-thirdio}


%\end{abstract}

\bibliographystyle{plain}
\bibliography{paper}

\end{document}
