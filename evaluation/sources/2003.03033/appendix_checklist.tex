%!TEX root = doc.tex
%!TEX output_directory = aux

% ------------------------------------------------
\section{Checklist for Evaluating a Pruning Method} \label{apx:recommendations}
% ------------------------------------------------

\noindent For any pruning technique proposed, check if:
\begin{itemize}
\item It is contextualized with respect to magnitude pruning, recently-published pruning techniques, and pruning techniques proposed prior to the 2010s.
\item The pruning algorithm, constituent subroutines (e.g., score, pruning, and fine-tuning functions), and hyperparameters are presented in enough detail for a reader to reimplement and match the results in the paper.
\item All claims about the technique are appropriately restricted to only the experiments presented (e.g., CIFAR-10, ResNets, image classification tasks, etc.).
\item There is a link to downloadable source code.
\end{itemize}

\noindent For all experiments, check if you include:
\begin{itemize}
\item A detailed description of the architecture with hyperparameters in enough detail to for a reader to reimplement it and train it to the same performance reported in the paper.
\item If the architecture is not novel: a citation for the architecture/hyperparameters and a description of any differences in architecture, hyperparameters, or performance in this paper.
\item A detailed description of the dataset hyperparameters (e.g., batch size and augmentation regime) in enough detail for a reader to reimplement it.
% \item A list of other prior pruning papers that include the same architecture, dataset, and hyperparameters.
\item A description of the library and hardware used.
\end{itemize}

\noindent For all results, check if:
\begin{itemize}
\item Data is presented across a range of compression ratios, including extreme compression ratios at which the accuracy of the pruned network declines substantially.
\item Data specifies the raw accuracy of the network at each point.
\item Data includes multiple runs with separate initializations and random seeds.
\item Data includes clearly defined error bars and a measure of central tendency (e.g., mean) and variation (e.g., standard deviation).
\item Data includes FLOP-counts if the paper makes arguments about efficiency and performance due to pruning.
\end{itemize}

\noindent For all pruning results presented, check if there is a comparison to:
\begin{itemize}
\item A random pruning baseline.
    \begin{itemize}
    \item A global random pruning baseline.
    \item A random pruning baseline with the same layerwise pruning proportions as the proposed technique.
    \end{itemize}
\item A magnitude pruning baseline.
    \begin{itemize}
    \item A global or uniform layerwise proportion magnitude pruning baseline.
    \item A magnitude pruning baseline with the same layerwise pruning proportions as the proposed technique.
    \end{itemize}
\item Other relevant state-of-the-art techniques, including:
    \begin{itemize}
    \item A description of how the comparisons were produced (data taken from paper, reimplementation, or reuse of code from the paper) and any differences or uncertainties between this setting and the setting used in the main experiments.
    \end{itemize}
\end{itemize}
