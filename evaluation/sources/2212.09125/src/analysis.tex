\section{Analysis}
\label{sec:analyze}
\subsection{Importance of Expand Stage}
We perform the ablation study on the importance of the expand stage and show the results in Table \ref{fig:ablation_expand}. We compare the performances of {\bf \textsc{\name-S}} using the expanded or the not expanded candidate sets on {\bf \textsc{UFET}} and {\bf \textsc{CFET}}. We replace the last $48$ candidates recalled by MLC with candidates expanded by MLM and exact matching for {\bf \textsc{UFET}}, and $10$ candidates for {\bf \textsc{CFET}}. Results show that expand stage has a positive effect on performance, it improves the final recall by $+1.0$ and $+2.2$ on  {\bf \textsc{UFET}} and {\bf \textsc{CFET}} without harming the precision.

\begin{table}[t]
\centering
\scalebox{0.75}{
\renewcommand{\arraystretch}{1}
\begin{tabular}{cllll} \toprule
\multicolumn{2}{l}{\bf \textit{Ablation of Expand Stage} }     & \bf \textsc{P}    & \bf \textsc{R}   & \bf \textsc{F1}  \\ \midrule
\multicolumn{5}{l}{\bf \textsc{UFET\ \  MCCE with C2C BERT-large}} \\
\color{blue}\bf \texttt{B} & {\bf \textsc{\name-S$_{128}$ }} (Ours)     & 52.5 & 49.1 & 50.8 \\ 
\color{blue}\bf \texttt{B} & {\bf \textsc{\name-S$_{128}$ w/o Expand }} (Ours)     & 52.7 & 48.1 & 50.3\\ \hline
\multicolumn{5}{l}{\bf \textsc{CFET\ \  MCCE with C2C BERT-base-Chinese}} \\
\color{brown}\bf \texttt{C} & {\bf \textsc{\name-S$_{64}$}} (Ours)  & 55.5 & 62.6 & 58.8 \\ 
\color{brown}\bf \texttt{C} & {\bf \textsc{\name-S$_{64}$ w/o Expand}}   (Ours)   & 55.4 & 60.4 & 57.8 \\ \hline
\midrule
\end{tabular}}
\caption{Ablation study of expand stage.}
\label{fig:ablation_expand}
\end{table}

\subsection{Attentions}
We conduct an ablation study on S2S, C2S, S2C, and C2C attention introduced in Sec. \ref{sec:attn} and show the results in Table \ref{tab:attn}. From the results, we are surprised to find that removing C2C and S2S doesn't have a big negative impact on performance. The {\bf \textsc{\name-S}} using BERT-base reaches $48.8$ F1 even without both C2C and S2S attention. One possible reason is that the interaction between sub-tokens in the sentence can be achieved indirectly by first attending to the candidates and then being attended back by the candidate in the next layer. We also find that the C2S is necessary for the task ($18.7$ F1 without C2S) because we rely on the mention and its context to encode and classify candidates. Furthermore, we found that it is important for sentences to attend to all candidates (S2C), indicating that the interaction between the sentence and different types is crucial for the task.

\begin{table}[t]
\centering
\scalebox{0.75}{
\renewcommand{\arraystretch}{1}
\begin{tabular}{cllll} \toprule
\multicolumn{2}{l}{\bf \textit{Analysis about attention on UFET}}     & \bf \textsc{P}    & \bf \textsc{R}   & \bf \textsc{F1}  \\ \midrule
\multicolumn{5}{l}{\bf \textsc{\name-S using BERT-base}} \\
\color{blue}\bf \texttt{B} & {\bf \textsc{\name-S$_{128}$} FULL}     & 53.2 &  48.3 & 50.6 \\ 
\color{blue}\bf \texttt{B} & {\bf \textsc{\name-S$_{128}$ w/o C2C }}     & 52.3 & 48.3 & 50.2 \\
\color{blue}\bf \texttt{B} & {\bf \textsc{\name-S$_{128}$ w/o S2S }}     & 50.6 & 48.4 & 49.4 \\
\color{blue}\bf \texttt{B} & {\bf \textsc{\name-S$_{128}$ w/o S2C }}     & 48.7 & 47.1 & 47.9 \\ 
\color{blue}\bf \texttt{B} & {\bf \textsc{\name-S$_{128}$ w/o C2S }}     & 19.7 & 17.4 & 18.7\\
\color{blue}\bf \texttt{B} & {\bf \textsc{\name-S$_{128}$ w/o S2S,C2C }}     & 50.2 & 47.3 & 48.8\\
\bottomrule
\end{tabular}}
\caption{Attention analysis.}
\label{tab:attn}
\end{table}








% \subsection{Influence of Candidate Size}


