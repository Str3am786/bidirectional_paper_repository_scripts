% This must be in the first 5 lines to tell arXiv to use pdfLaTeX, which is strongly recommended.
\pdfoutput=1
% In particular, the hyperref package requires pdfLaTeX in order to break URLs across lines.

\documentclass[11pt]{article}

% Remove the "review" option to generate the final version.
% \usepackage[review]{ACL2023}
\usepackage{ACL2023}

% Standard package includes
\usepackage{times}
\usepackage{latexsym}
\usepackage{fontawesome}
\usepackage{amsmath}
\usepackage{bm}
\usepackage{amsfonts}
\usepackage{amssymb}
\usepackage{booktabs}
\usepackage{subcaption}
\usepackage{caption}
\usepackage{graphicx}
\usepackage{multirow}
% For proper rendering and hyphenation of words containing Latin characters (including in bib files)
\usepackage[T1]{fontenc}
% For Vietnamese characters
% \usepackage[T5]{fontenc}
% See https://www.latex-project.org/help/documentation/encguide.pdf for other character sets

% This assumes your files are encoded as UTF8
\usepackage[utf8]{inputenc}

% This is not strictly necessary, and may be commented out.
% However, it will improve the layout of the manuscript,
% and will typically save some space.
\usepackage{microtype}

% This is also not strictly necessary, and may be commented out.
% However, it will improve the aesthetics of text in
% the typewriter font.
\usepackage{inconsolata}
\usepackage{makecell}
\usepackage{xcolor}
\usepackage{todonotes}
% If the title and author information does not fit in the area allocated, uncomment the following
%
%\setlength\titlebox{<dim>}
%
% and set <dim> to something 5cm or larger.

\newcommand\blfootnote[1]{%
  \begingroup
  \renewcommand\thefootnote{}\footnote{#1}%
  \addtocounter{footnote}{-1}%
  \endgroup
}



\title{Recall, Expand and Multi-Candidate Cross-Encode: \\
Fast and Accurate Ultra-Fine Entity Typing}


% Author information can be set in various styles:
% For several authors from the same institution:
\author{
    \textbf{Chengyue Jiang$^\Diamond$$^\ddagger$},
    \textbf{Wenyang Hui$^\Diamond$$^\ddagger$},
    \textbf{Yong Jiang},
    \textbf{Xiaobin Wang},
    \textbf{Pengjun Xie},
    \textbf{Kewei Tu$^\ddagger$}\thanks{$~~$ Kewei Tu is the corresponding author.} \\
    $^\ddagger$ School of Information Science and Technology, ShanghaiTech University \\
    Shanghai Engineering Research Center of Intelligent Vision and Imaging \\
    \texttt{\{jiangchy,huiwy,tukw\}@shanghaitech,edu.cn;} \\
    \texttt{\{jiangyong.ml,xpjandy\}@gmail.com;} \\
    \texttt{czwangxiaobin@foxmail.com;}
}

\begin{document}
\maketitle

\blfootnote{$^\Diamond$ Equal Contribution.}
\newcommand{\code}{\url{http://github.com/modelscope/AdaSeq/tree/master/examples/MCCE}}
\newcommand{\name}{{MCCE}}

We study the practical consequences of dataset sampling strategies on the ranking performance of recommendation algorithms. Recommender systems are generally trained and evaluated on \emph{samples} of larger datasets. Samples are often taken in a na\"ive or ad-hoc fashion: \eg by sampling a dataset randomly or by selecting users or items with many interactions. As we demonstrate, commonly-used data sampling schemes can have significant consequences on algorithm performance. Following this observation, this paper makes three main contributions: (1) \emph{characterizing} the effect of sampling on algorithm performance, in terms of algorithm and dataset characteristics (\eg sparsity characteristics, sequential dynamics, \etc); (2) designing \sampler, which is a data-specific sampling strategy, that aims to preserve the relative performance of models after sampling, and is especially suited to long-tailed interaction data; and (3) developing an \emph{oracle}, \oracle, which can suggest the sampling scheme that is most likely to preserve model performance for a given dataset. The main benefit of \oracle is that it will allow recommender system practitioners to quickly prototype and compare various approaches, while remaining confident that algorithm performance will be preserved, once the algorithm is retrained and deployed on the complete data. Detailed experiments show that using \oracle, we can discard upto $5\times$ more data than any sampling strategy with the same level of performance.
\section{Introduction}
\label{sec:introduction}
Theory and algorithms for large-margin classifiers 
have been studied extensively 
since those classifiers guarantee low generalization errors 
when they have large margins over training examples 
(e.g.,~\cite{schapire+:as98,mohri+:mitpress18}). 
In particular, 
the $\ell_1$-norm regularized soft margin optimization problem, 
defined later, is a formulation of 
finding sparse large-margin classifiers based on the linear program (LP). 
This problem aims to optimize the $\ell_1$-margin 
by combining multiple hypotheses from some hypothesis class $\hset$. 
The resulting classifier tends to be sparse, 
so $\ell_1$-margin optimization is helpful for feature selection tasks.
Off-the-shelf LP solvers can solve the problem, 
but they are still not efficient enough for a huge class $\hset$. 

Boosting is a framework 
for solving the $\ell_1$-norm regularized margin optimization 
even though $\hset$ is infinitely large. 
% Roughly speaking, boosting collects a hypothesis one by one
% to maximize the margin. 
Various boosting algorithms have been invented. 
LPBoost~\citep{demiriz+:ml02} is a practical algorithm 
that often works effectively. 
% and often works efficiently in practice.
Although LPBoost terminates rapidly, 
It is shown that 
it takes $\Omega(m)$ iterations in the worst case, 
where $m$ is the number of training examples~\citep{warmuth+:nips07}. 
\cite{shalev-shwartz+:jml10} invented
% Shalev-Shwartz and Singer~\citep{shalev-shwartz+:jml10} invented
%the Relaxed margin algorithm, 
an algorithm
called Corrective ERLPBoost 
(we call this algorithm C-ERLPBoost for shorthand) 
in the paper on ERLPBoost~\citep{warmuth+:alt08}. 
C-ERLPBooost and ERLPBoost 
find $\epsilon$-approximate solutions 
in $O(\ln(m/\nu) / \epsilon^2)$ iterations, 
where $\nu \in [1, m]$ is the soft margin parameter. 
The difference is the time complexity per iteration; 
ERLPBoost solves a convex program (CP) for each iteration, 
while C-ERLPBooost solves a sorting-like problem. 
Although ERLPBoost takes much time per iteration, 
it takes fewer iterations than C-ERLPBoost 
in practical applications. 
For this reason, 
ERLPBoost is faster than C-ERLPBoost. 
Our primary motivation is to investigate boosting algorithms 
with provable iteration bounds, which perform as fast as LPBoost.

This paper has two contributions. 
Our first contribution is to give 
a unified view of boosting for soft margin optimization. 
We show that LPBoost, ERLPBoost, and C-ERLPBoost are 
instances of the Frank-Wolfe algorithm. 
%The second one proposes 


Our second contribution is to propose
a generic scheme for boosting based on the unified view.
Our scheme combines a standard Frank-Wolfe algorithm and \emph{any} algorithm 
and switches one to the other at each iteration in a non-trivial way.
%a Modified LPBoost (MLPBoost) \textcolor{red}{(Rename?)}. 
We show that this scheme guarantees 
the same convergence rate, $O(\ln(m/\nu) / \epsilon^2)$,  
as ERLPBoost and C-ERLPBoost.
One can incorporate any update rule to this scheme
without losing the convergence guarantee 
so that it takes advantage of better updates 
of the second algorithm in practice.
%with fast computation per iteration. 
In particular, 
we propose to choose LPBoost 
as the secondary algorithm, 
% as the second algorithm to combine, 
and we call the resulting algorithm 
Modified LPBoost (MLPBoost). 

In experiments on real datasets, 
MLPBoost works comparably with LPBoost, and 
%if we incorporate the LPBoost update to MLPBoost. 
MLPBoost is the fastest 
among theoretically guaranteed algorithms, as expected. 


Table~\ref{table:boosting_comparison} compares 
LPBoost, ERLPBoost, C-ERLPBoost, and MLPBoost. 
\begin{table}[t]
    \centering
    \caption{%
        Comparison of the boosting algorithms. %
        C-ERLPBoost solves the problem per iteration %
        by sorting based algorithm, while our work and %
        LPBoost solves linear programming (LP). %
        ERLPBoost solves convex programming (CP) per iteration. %
        In practice, the algorithms work fast in the order %
        LPBoost, ERLPBoost, and C-ERLPBoost. %
        As we show in section~\ref{sec:experiments}, %
        our algorithm is as fast as LPBoost. %
    }
    \label{table:boosting_comparison}
    \begin{tabular}{|c|cccc|}
        \toprule
                    & LPBoost & C-ERLPBoost & ERLPBoost & One of our work \\
        \midrule
        Iter. bound 
            & $\Omega(m)$ 
            & $O\left(\frac{1}{\epsilon^2} \ln \frac{m}{\nu}\right)$ 
            & $O\left(\frac{1}{\epsilon^2} \ln \frac{m}{\nu}\right)$ 
            & $O\left(\frac{1}{\epsilon^2} \ln \frac{m}{\nu}\right)$ \\
        Problem per iter. & LP & Sorting & CP & LP \\
        \bottomrule
    \end{tabular}
\end{table}


\section{Background}
\label{sec:background}

% \subsection{Meta-learning for speaker embeddings}

% \begin{itemize}
%     \item Includes protonets implementation “In defence of metric learning for speaker recognition” 
%     \item Talks about metric-learning (center loss?) “Comparison of metric-learning loss functions for E2E speaker verification” 
%     \item Protonets for short utterance speaker recognition 
%     \item \textbf{(Pending results)} Using deep clustering loss, based on spectral clustering: ||VVT - LLT||F 
% \end{itemize}

\subsection{Meta-Learning for Task Generalization}

Early works on meta-learning focused on adaptive learning strategies such as combining gradient descent with evolutionary algorithms \cite{yao_evolve1999,ABRAHAM20041}, learning gradient updates using a meta-network \cite{naik_metaNN1992} and using biologically inspired constraints for gradient descent \cite{bengio_synaptic1991, bengio1992optimization}.
Recent meta-learning approaches have addressed the issue of rapid generalization in deep learning, by learning to learn for a new task \cite{Andrychowicz_2016, finn_maml2017, ravi2017}.
This concept is inspired by the human ability to learn using a handful of examples. For instance children learn to recognize a new animal when presented with a few images as opposed to conventional DNNs which require thousands of samples for a new class.
The ability to quickly generalize to unseen classes is achieved by generating diversity in training tasks, for instance by using different sets of classes at each training step (see Fig. 1 in \cite{ravi2017}). Further, the classification setup (in terms of number of classes and samples per class) is controlled to match with that of the test task \cite{snell2017prototypical}.
Meta-learning has been successfully applied to achieve task generalization in computer vision \cite{ravi2017,snell2017prototypical,finn_maml2017} and more recently in natural language processing \cite{yu2018diverse,GaoH0S19, dou-etal-2019-investigating}.
Drawing parallels with the above applications, we train speaker embeddings with a large number of speaker classification tasks to improve over the conventional model which uses a single classification task. Since speaker sets differ between training steps, we replace the conventional softmax nonlinearity and cross-entropy loss combination with metric learning objectives used in previous meta-learning works \cite{snell2017prototypical,sung2018learning,vinyals2016matching,geng2019induction}.


\subsection{Meta-Learning Speaker Embeddings}

Few recent approaches have used a variant of meta-learning to train speaker embeddings, specifically the metric-learning objective from prototypical networks (protonets).
In \cite{chung2020defence}, the authors extend angular softmax objective to protonets and compare with various metric learning approaches for speaker verification. Across different architectures, angular prototypical loss outperforms other methods including conventional softmax objective. 
The authors in \cite{kye2020metalearning} applied protonets for short utterance speaker recognition and introduced global prototypes that mitigate the need for class sampling. 
In related applications, \cite{ko_protonets2020} and \cite{an2019shot} used protonets for small footprint speaker verification and few-shot speaker classification, respectively.
In \cite{wang_centroid2019}, the protonet loss was compared with triplet loss and evaluated on (open and close set) speaker ID and speaker verification tasks. 
However, previous approaches seldom compare embeddings trained using protonets with existing benchmarks based on x-vectors, except for \cite{ko_protonets2020} where a modified architecture was used owing to the nature of the task. Further, the class sampling strategy is not always used with protonets (e.g., \cite{chung2020defence,kye2020metalearning})
% triple-check above claim!)
which might inhibit task diversity during training. 
An exception from the above metric-learning approaches is \cite{kang2020domaininvariant}, where the authors train deep speaker embeddings using the model-agnostic meta-learning strategy to mitigate  domain mismatch for speaker verification.
To the best of our knowledge, meta-learning is yet to be applied for general-purpose speaker diarization, except for the specific case of dyadic speaker clustering in child-adult interactions in our recent work \cite{koluguri2020}. 

%> Protonets previously in speaker embedding applications
%> Connect relation networks to previous works, ones which learn the relation
% Refer to the Google Doc for a summary
\section{Identifying Macro Norm violating comments}
Our aim is to identify macro norm violating comments on Reddit, to quantify their prevalence, and to characterize their content, rate of engagement, and language usage. However, there are too many comments for manual annotation, and state-of-the-art machine learning classifiers are not robust enough on their own. We overcome these issues using a human-AI pipeline in which we use classifiers with a high recall to nominate candidate comments that might be violating, and then focusing our manual annotations with trained annotators on these nominated comments. By tuning the classifiers to have high recall over high precision, our pipeline ensures that almost all of the violating comments are sent to be reviewed by our annotators. Additionally, in concurrence with prior work \cite{chandrasekharan2019crossmod, seering2020reconsidering}, this pipeline ensures that a human has the final say in labeling any piece of content as a violation.

In this section, we first discuss the scope of our investigation, including how we define violating comments for this paper. We then summarize our pipeline for identifying violating comments. 

\subsection{Scope of the Study}

% \usepackage{booktabs}
% \usepackage{vcell}


\begin{table}[tb]
\centering
\caption{The eight macro norm violations on 97 popular subreddits and their definitions. We took the norms uncovered in a prior study \cite{Chandrasekharan2018internet} and expanded on some of their definitions to better fit our data.}
\begin{tabular}{ll}
\multicolumn{1}{c}{\textbf{MACRO NORM VIOLATIONS}}                                                & \multicolumn{1}{c}{\textbf{EXAMPLE COMMENTS}}                                                                                                                                         \\ 
\toprule
\vcell{Using misogynistic or vulgar slurs}                                                        & \vcell{\textit{"god... I want sage to knock this c*** out"}}                                                                                                                          \\[-\rowheight]
\printcelltop                                                                                     & \printcelltop                                                                                                                                                                         \\
\vcell{Inflammatory political claims}                                                             & \vcell{\begin{tabular}[b]{@{}l@{}}\textit{"a day old troll complaining about liberals -- I }\\\textit{smell a lost trumpkin"}\end{tabular}}                                           \\[-\rowheight]
\printcelltop                                                                                     & \printcelltop                                                                                                                                                                         \\
\vcell{Bigotry}                                                                                   & \vcell{\begin{tabular}[b]{@{}l@{}}\textit{"punishment for not being hateful enough and }\\\textit{not destroying the gays"}\end{tabular}}                                             \\[-\rowheight]
\printcelltop                                                                                     & \printcelltop                                                                                                                                                                         \\
\vcell{\begin{tabular}[b]{@{}l@{}}Verbal attacks on Reddit or specific \\subreddits\end{tabular}} & \vcell{\begin{tabular}[b]{@{}l@{}}\textit{"also reddit sucks because a user making~an~error }\\\textit{refuses to delete their post and redo it }\\\textit{correctly"}\end{tabular}}  \\[-\rowheight]
\printcelltop                                                                                     & \printcelltop                                                                                                                                                                         \\
\vcell{Posting pornographic links}                                                                & \vcell{\textit{[URL]}}                                                                                                                                                                \\[-\rowheight]
\printcelltop                                                                                     & \printcelltop                                                                                                                                                                         \\
\vcell{Personal attacks}                                                                          & \vcell{\textit{"you know man youre kind of a f***ing d*****"}}                                                                                                                        \\[-\rowheight]
\printcelltop                                                                                     & \printcelltop                                                                                                                                                                         \\
\vcell{Abusing and criticizing moderators}                                                        & \vcell{\textit{"the mods in this sub need to wake the f*** up"}}                                                                                                                      \\[-\rowheight]
\printcelltop                                                                                     & \printcelltop                                                                                                                                                                         \\
\vcell{\begin{tabular}[b]{@{}l@{}}Claiming the other person is too \\sensitive\end{tabular}}      & \vcell{\textit{"get off the internet with your sensitive ass"}}                                                                                                                       \\[-\rowheight]
\printcelltop                                                                                     & \printcelltop                                                                                                                                                                         \\
\bottomrule
\end{tabular}
\end{table}

Reddit, the focus of our study, is a large-scale social media platform with 52 million daily active users \cite{59_Phan}. On Reddit, users join smaller subcommunities called subreddits that cover a specific topic and are managed by voluntary moderators who enforce community-specific rules (e.g., the type of allowed content, expected member behaviors), making the platform a good test-bed for studying user behaviors across diverse sets of moderation strategies and topics. In particular, we explore comments posted in response to top-level post submissions as most of the discussions on Reddit take place in the comment section. Given that the subreddits each have varying rules, we consider a comment to be violating if it breaks one of the \textit{macro norms} on Reddit --- norms that the vast majority of subreddits agree on. These norms were identified in prior work~\cite{Chandrasekharan2018internet} that investigated the 100 most popular subreddits harboring nearly a third of all comments on Reddit to extract the topic categories for the moderated (summarized in Table 1). 

We rely on macro norms as they provide us with a lens to measure the prevalence of violating comments that are largely independent of community-relevant contexts. However, in doing so, we are explicitly not accounting for comments that do not violate macro norms but still violate local rules of the respective subreddit. We also note that some subreddits have explicitly chosen to permit content that violates one or more of these macro norms (e.g., vulgar or sexualized comments). This highlights an important tension between the local and the macro norms. We discuss these issues and expand on their implications for future content moderation in the discussion section.


\subsection{Data for Training and Testing the Classifiers}
The first step of our pipeline uses a set of machine learning classifiers to flag comments that are potentially macro norm violations. We trained and tested these classifiers following best practice by constructing a balanced dataset that contains an equal number of moderated comments (denoted as $\mathcal{M}$) and unmoderated comments that are still online and were not moderated or deleted by the author (denoted as $\mathcal{M'}$). There are more unmoderated than moderated comments on Reddit---$\mathcal{M'}$, therefore, is a subset of all unmoderated comments. While such balanced datasets do not match the real-world distribution, balancing the dataset gives the resulting model equal priority to each class, which is important for ensuring that our model actually learns the meaningful features for the classification task and not just the uneven class distribution. 

\subsubsection{\rnr{$\mathcal{M}$: moderated comments}} 
$\mathcal{M}$ represents the top 100 most popular English subreddits during the 11 month period from May 2016 to March 2017. Given that the moderated comments are removed soon after they are posted, prior work used \textit{praw}, a Reddit streaming API, to stream and save all comments posted to each of these study subreddits before they were moderated \cite{Chandrasekharan2018internet}. 24 hours after each of these comments were streamed, all comments were queried again via the API using their unique \textit{comment\_ID} and verified which of them were replaced by a [``removed''] tag as that would signal their removal due to moderation. Any comments by AutoModerator accounts, which are bots for moderation, were removed from $\mathcal{M}$. This left $\mathcal{M}$ with a total of 2,831,664 removed comments, with at least 5,000 for each of the 100 study subreddits. 


\subsubsection{\rnr{$\mathcal{M'}$: unmoderated comments}} 
As $\mathcal{M}$ contained only the moderated comments from the sampling period, we collected $\mathcal{M'}$ ourselves through historical archives of Reddit comments. Of the 100 study subreddits from $\mathcal{M}$, three--- r/The\_Donald, r/Incels, r/soccerstreams---no longer exist on the platform, so we focused our investigation on the remaining 97 study subreddits. During the construction of $\mathcal{M'}$, we aimed to closely replicate the data collection process of $\mathcal{M}$. For each of the 97 subreddits, we used \textit{Pushshift} dataset that stores all content posted on the Reddit platform to gather IDs of submissions that were posted from the same timeframe as when $\mathcal{M}$ was collected with an even distribution across the 11 months. We then used \textit{praw} to get the actual comments with the submission IDs and discarded any that were posted by a bot or moderated. We continued this process until we had a balanced dataset for each of the 97 study subreddits.




\begin{figure}[tb]
  \centering
  \includegraphics[width=0.90\textwidth]{content_minor_revision__Apr2022/images/Final_pipeline.jpg}
  \caption{An illustration of the human-AI pipeline for identifying violating comments. Our pipeline includes 97 subreddit classifiers that are trained using a balanced dataset of moderated and unmoderated comments, and human annotators who are trained through gated instruction~\cite{25_Liu}. Our classifiers (tuned for high recall) nominate potentially violating comments and human annotators make the final determination.}
  \Description{Human-AI pipeline}
\end{figure}




\subsection{Building the Classifiers}
Using $\mathcal{M}$ and $\mathcal{M'}$, we built 97 neural network binary classifiers, each of which was trained on the data from one of the 97 study subreddits to classify whether a given comment would be moderated on that subreddit. These classifiers collectively determine whether a comment is likely to have violated one of the macro norms and thus would have been removed on most of the subreddits. We refer to these classifiers as subreddit classifiers. 

\subsubsection{Preprocessing the data} 
We first preprocessed our dataset by putting all characters in lowercase and removing non-alphabetical characters. We then segmented our dataset into a \textit{training} dataset (70\% of all data) and \textit{testing} and \textit{validation} datasets (15\% of all data each), each with an equal number of moderated and unmoderated comments. For every study subreddits, we then used our training dataset to train word embeddings from scratch and encoded comments as fixed-length vectors, trancating and padding as needed. 

\subsubsection{Building the classifiers} 
We built our classifiers using Google's \textit{TensorFlow} and trained and validated them with the encoded dataset. The classifiers have a four-layer neural network architecture, starting with an embedding layer that takes the encoded list of integers and finds an embedding vector for each word, which we learned as we trained our network. We then pass through an average pooling layer that returns a fixed-length output vector and then through a dense layer with Rectified Linear Unit (ReLU) activation function \cite{26_TensorFlow}. Finally, we employ another dense layer with a sigmoid activation function that transforms the final output of the network into a value between 0.0 and 1.0. For our binary classification task, we identify a comment as one that would have been removed in a given subreddit if the final output of the network is greater than or equal to 0.5. 

We then fine-tuned the following four parameters for each of our subreddit classifiers using grid search where we try out exhaustive combinations of hyperparameters given candidate values:  the size of the word index used in the encoding, the length of the input vector, the number of epochs during the training phase, and the number of nodes for the ReLU layer of the neural network. These are summarized in Table 2. We optimized for the F1 score (\textit{f}-measure) on our validation dataset, achieving an average of 72.3 (std=4.32) across the 97 classifiers. This is comparable to the classifiers presented in prior work that were trained on a similar dataset \cite{4_Chancellor, chandrasekharan2019crossmod}.


% \usepackage{booktabs}


% \usepackage{booktabs}


\begin{table}[tb]
\centering
\caption{Parameters and the values used for them to fine-tune the classifiers}
\begin{tabular}{ll}
\multicolumn{1}{c}{\textbf{DESCRIPTION} } & \multicolumn{1}{c}{\textbf{SET OF VALUES} }  \\ 
\toprule
Size of the word index                    & {[}10000, 44000]                             \\
Max length of the input                   & {[}256, 512]                                 \\
Number of epochs during the training      & {[}30, 40, 50]                               \\
\# of nodes for the ReLU layer            & {[}16, 32]                                   \\
\bottomrule
\end{tabular}
\end{table}

% \begin{table}
% \centering
% \caption{Parameters and the values used for them to fine-tune the classifiers}
% \begin{tabular}{lll}
% \multicolumn{1}{c}{\textbf{PARAMETERS}} & \multicolumn{1}{c}{\textbf{DESCRIPTION}} & \multicolumn{1}{c}{\textbf{SET OF VALUES}}  \\ 
% \toprule
% \textit{$\mathcal{WI}$}                             & Size of the word index                   & {[}10000, 44000]                            \\
% \textit{$\mathcal{ML}$}                             & Max length of the input                  & {[}256, 512]                                \\
% Epochs                                  & Number of epochs during the training     & {[}30, 40, 50]                              \\
% \textit{$\mathcal{DN}$}                             & \# of nodes for the ReLU layer           & {[}16, 32]                                  \\
% \bottomrule
% \end{tabular}
% \end{table}


\subsection{Machine Learning Flags Comments}
We marked a comment as \textit{flagged} (potentially norm violating) if the number of subreddit classifiers that flagged the comment, which we call the \textit{classifier agreement score}, was greater than or equal to 80 out of 97. This threshold was selected to achieve a high recall on the ensemble classifier even at the cost of producing false positives as our pipeline includes human annotators who validate the classifier flagged comments. In other words, we wanted our process to miss as few violating comments as possible, so we deliberately used a low threshold of 80 out of 97 and passed these comments to a human review stage. This approach provides statistical power even within a realistic budget for manually annotating comments, because it results in roughly one in five flagged comments later being coded as a violation while maintaining a near-zero false negative rate.

We confirmed that this threshold indeed captures most of the violating comments: the first author manually annotated a random sample of 1,000 comments in the validation dataset from 2016 to 2017 period. This sample contained 400 comments with $< 80$ subreddit classifier agreement, 200 with $80 \leq \text{agreement} < 85 $, 200 with $85 \leq \text{agreement} < 90$, and 200 with $\geq 90$ agreement. We find that only one percent of the comments with $< 80$ classifier agreement violated at least one of the macro norms when manually inspected, whereas this number significantly increased in the subsequent sample groups (9.5\%, 12\%, and 28\% in the order of increasing classifier agreement). This low false negative rate when using a low enough agreement threshold matches the observations in a prior work that took a similar approach to classifying violating comments on Reddit \cite{chandrasekharan2019crossmod}.

In addition, to confirm that our classifier's low false negative rate holds for the comments from 2020 period, we further annotate 400 comments with classifier agreement score of less than 80 from this period randomly sampled across the study subreddits. We find our result to replicate, with roughly the same rate of 1.25\% of the sample to violate one of the macro norms. Although our subreddit classifiers were trained on comments from a 2016 to 2017 period, this low false negative rate for the comments from 2020 suggests that when combined with our human annotators, our overall pipeline still remains robust even for the newer comments. Finally, as we describe in the following section on our bootstrap sampling methods, these false negative rates are accounted for in our calculation of the confidence interval of our estimations.


\subsection{Human Annotation Validates the Flagged Comments}
\label{sec:annotation}
The tradeoff for tuning our classifiers for very few false negatives is that they produce more false positives. So we recruited human annotators to verify that the classifier flagged comments are indeed violating by asking them to code a subset of the flagged comments to see which macro norms they violate, where the subset was a random sample of the flagged comments with an even distribution across the 97 study subreddits. The definitions of these macro norms that we presented to our annotators were inspired by prior work~\cite{Chandrasekharan2018internet}, but based on our qualitative annotation discussed above, we found it appropriate to expand the definitions for some of them to better fit our data. We updated the norm described as ``opposing political views around Donald Trump'' to ``inflammatory political claims'' that covers inflammatory comments that are against the right-leaning and the left-leaning political ideologies and updated the norm described as ``hate speech that is racist or homophobic'' to ``bigotry'' that covers hate speech directed at ethnic or religious groups as well. 


\begin{figure}[tb]
  \centering
  \includegraphics[width=0.98\textwidth]{content_minor_revision__Apr2022/images/annotation_interface.jpg}
  \caption{\textbf{A:} The interface for introducing the task description and eight macro norms to a new annotator. The definitions for each norms are shown one by one, accompanied by gold-standard examples that violate the norm. \textbf{B:} The interface for training and testing new annotators. Once the new annotators select their annotation, the correct annotation is shown accompanied by gold-standard examples. The interface for the main annotation task is the same but without presenting the correct annotation portion.}
  \Description{Annotation Interface}
\end{figure}



\subsubsection{Recruiting  crowd workers} 
The crowd workers were recruited from Amazon Mechanical Turk (MTurk), and they had to be at least 18 years old, living in the US, and have completed more than 1,000 Human Intelligence Tasks (HITs – MTurk’s task unit) with the minimum HIT approval rating of 98\%. In our pilot annotation task that we will describe in a subsequent subsection, our annotators took around 10 seconds (median=9.66 seconds; 75th percentile=16.99 seconds) to annotate a single comment. Based on this, we decided to pay our workers \$1.50 for every 30 comments they annotated to ensure that we are paying the majority of our workers at the rate of at least \$15.00 per hour. We decided on this rate informed by Rolf's \textit{The Fight For Fifteen} \cite{20_Rolf}.

\subsubsection{Human annotation workflow} 
Applying human annotation for non-trivial classification tasks could suffer from inaccurate annotations due accidental errors~\cite{22_Angeli, 23_Pershina, 24_Zhang}. Therefore, we designed a training and a testing phase that are inspired by the \textit{gated instruction} workflow~\cite{25_Liu} as follows to ensure that our annotators clearly understand and are proficient at the task:

Our annotators were directed to a custom-built web platform to which they could sign in with their MTurk ID. For those joining for the first time, they were redirected to the first portion of the training phase in which they were presented with 1) an overview description of the task and its goal, 2) a content warning notifying them that some comments in the task might include offensive language, and 3) the eight macro norms accompanied by their definitions and two gold-standard example comments that we manually chose from the test dataset (Figure 2-A). When they finished reviewing this content, they were asked to practice annotating 30 hand-selected, gold-standard examples of the macro norm violations handpicked from our test dataset. This was done on the actual interface used for the main annotation task that showed a comment to be annotated, and multi-select HTML form for submitting macro norm violations (Figure 2-B). The annotators could select any number of macro norms they thought the given comment violated. Importantly, during this training phase, the workers were presented with the correct annotation and explanation after each time they submitted their annotations for a given comment. Note also that, in order to avoid biasing their decisions by imposing an ``expert'' AI opinion, the workers were not told that the comments were flagged by an algorithm as potentially norm violating \cite{seering2020reconsidering}.

Of the 30 practice comments, the last 10 were effectively the test annotations; we measured our annotators’ Cronbach's alpha reliability score when compared to our gold-standard annotations, and only admitted those whose reliability score was greater than or equal to 0.7 for the last 10 training annotations. All annotators who participated in the training and testing phase were provided with a completion code they could submit to MTurk and were paid \$1.50 for their time. Those who were admitted were allowed to start the main annotation task. They could annotate as many comments as they wanted as long as there were more comments to be annotated, and were provided with a completion code they could submit to receive \$1.50 for every 30 comments they annotated. Each of the comments was seen by three unique annotators to test for majority agreement. 

During the course of this study, we recruited a total of 31 annotators who collectively annotated a total of 4,850 comments. \mrev{Finally, we followed best practices for accounting for annotator well-being during their task. In addition to presenting the aforementioned content warning, we made sure that our annotators could freely leave the study at any time if they felt uncomfortable with the annotation task. We purposefully designed our compensation scheme to ensure that we paid our participants in small increments instead of asking them for a long period of participation before receiving their compensation to ensure that our participants could receive the payment they deserve regardless of when they choose to leave the study.}


\subsubsection{Pilot} 
To confirm the robustness of this approach and  to determine the right level of compensation for the workers, we ran a pilot annotation task for which we recruited 20 annotators to partake in the training phase. Of the twenty, 8 annotators passed the testing phase and went on to annotate 194 randomly selected comments from the test dataset. The first author manually annotated the 194 comments that the annotators annotated during the aforementioned pilot tasks to establish baseline annotations to which annotators' work would be compared. In relation to the first author’s manual annotation, the majority-agreed annotation of the annotators yielded a Cronbach's alpha score of 0.86. 

\section{Experiments}
We conduct experiments on two ultra-fine entity typing datasets, {\bf \textsc{UFET}} (English) and {\bf \textsc{CFET}} (Chinese). Their data statistics are shown in Table \ref{tab:stat}. We mainly focus on and report the macro-averaged recall at the recall and expand stage, and concern mainly on the macro-$F1$ of the final prediction at the filter stage. We also evaluate the {\bf \textsc{\name}} models on the fine-grained (130 types) and coarse-grained (9 types) settings of entity typing without the recall and expand stage.
\subsection{UFET and CFET}
\subsubsection{Recall Stage}
\label{sec:recall}
We compare the recall@$K$ on the test sets of {\bf \textsc{UFET}} and {\bf\textsc{CFET}} between the trained MLC model (introduced in \ref{sec:mlc}) and a traditional BM25 model \cite{bm25} in Figure \ref{fig:recall}. The MLC model uses the RoBERTa-large as backbone and is tuned based on the recall@$128$ on the development set. We use AdamW optimizer with a learning rate of $2\times10^{-5}$. Results show that MLC is a strong recall model, it consistently has better recall compared to BM25 on both {\bf\textsc{UFET}} and {\bf\textsc{CFET}} dataset, and the recall@$128$ reaches over $85\%$ on {\bf \textsc{UFET}}, and over $94\%$ on {\bf \textsc{CFET}}.

\begin{figure}[t]
     \centering
     \begin{subfigure}[h]{0.5\textwidth}
         \centering
         \includegraphics[width=\textwidth]{src/img/recall_compare_bm25.pdf}
         \label{fig:mb2}
     \end{subfigure}   
 \caption{Recall@$K$ of MLC and BM25.}
 \label{fig:recall}
\end{figure}

\subsection{Expand Stage}
\label{sec:expand}
In Table \ref{tab:expand}, we evaluate the F1 scores of all candidates expanded by exact match, and top-$10$ candidates expanded by the MLM using Bert-large. We also demonstrate the improvement of recall by using candidate expansion in Figure \ref{fig:expand_improvement}. On {\bf \textsc{UFET}} dataset, expanding around $32$ additional candidates based on $112$ MLC candidates results in $2\%$ higher recall compared to recalling all $128$ candidates by MLC. The recall of $128$ candidates after the expansion is comparable to the recall of $180$ candidates recalled from MLC. Similarly, expanding $10$ candidates is comparable to additionally recalling $80$ candidates using MLC.
In our experiments, we replace the last $48$ candidates recalled by MLC with the candidates recalled by MLM and Exact match for {\bf \textsc{UFET}} and $10$ for {\bf \textsc{CFET}}. We found the expand stage has a positive effect on the final performance of {\bf \textsc{\name}}s, and helps them reach SOTA performance (analyze in Sec. \ref{sec:analyze}).


\begin{table}[t]
\centering
\scalebox{0.75}{
\begin{tabular}{cccccc} 
\toprule
{\bf \textsc{Dataset}} & {\bf \textsc{Expand}} &   {\bf \textsc{P}}  & {\bf \textsc{R}}  &  {\bf \textsc{F1}} & \small{Avg \# Expanded}  \\ \midrule
\multirow{2}{*}{\bf \textsc{UFET}} & {\bf \textsc{Match}}      & 11.2   & 11.3     & 9.8    & 5.23     \\
      & {\bf \textsc{MLM}}  &  8.5     &   17.1   &  10.7  &    10    \\ \midrule
\multirow{2}{*}{\bf \textsc{CFET}} & {\bf \textsc{Match}}   &  11.4  &  14.5  & 11.2   & 4.57    \\
 & {\bf \textsc{MLM}}  & 21.3   &  19.5  & 17.7    & 10    \\ \midrule
\end{tabular}}
\caption{Evaluation of the recalled candidates.}
\label{tab:expand}
\end{table}
\begin{figure}[t]
     \centering
     \begin{subfigure}[h]{0.45\textwidth}
         \centering
         \includegraphics[width=\textwidth]{src/img/recall_ufet.pdf}
         \caption{Recall@$128$ on {\bf \textsc{UFET}} by including different number of expanded candidates. }
         \label{fig:c1}
     \end{subfigure}
     \vfill
     \begin{subfigure}[h]{0.45\textwidth}
         \centering
         \includegraphics[width=\textwidth]{src/img/recall_cfet.pdf}
         \caption{Recall@$64$ on {\bf \textsc{CFET}} by including different number of expanded candidates.}
         \label{fig:c2}
     \end{subfigure}
\caption{Demonstration of the effect of expand stage. $x$-axis represents the number of candidates expanded by MLM/MLM+MATCH among these $128$ candidates. }
\label{fig:expand_improvement}
\end{figure}
\label{sec:exp_expand}
\subsection{Filter Stage and Final Results.}
\begin{table}[h!]
\centering
\scalebox{0.73}{
\renewcommand{\arraystretch}{1}
\begin{tabular}{cllll} \toprule
\multicolumn{2}{l}{\bf \textit{Base Models on UFET} }     & \bf \textsc{P}    & \bf \textsc{R}   & \bf \textsc{F1}  \\ \midrule
\multicolumn{5}{l}{\emph{MLC-like models}}        \\
\color{blue} \bf \texttt{B}& {\bf \textsc{Box4Types}}\cite{box4types}  & 52.8 & 38.8 & 44.8  \\
\color{blue}\bf \texttt{B}& {\bf \textsc{LDET}}$^\dagger$  \cite{onoe-durrett-2019-learning}          & 51.5 & 33.0 & 40.1 \\ 
\color{blue}\bf \texttt{B}& {\bf \textsc{MLMET}}$^\dagger$   {\cite{mlmet}}   & 53.6 & 45.3 & 49.1  \\
\color{blue}\bf \texttt{B}& {\bf \textsc{PL}}  \cite{ding2021prompt}   & 57.8 & 40.7 & 47.7 \\
\color{blue}\bf \texttt{B}& {\bf \textsc{DFET}}    \cite{dfet}      & 55.6 & 44.7 & 49.5 \\
\color{blue}\bf \texttt{B}& {\bf \textsc{MLC}} (reimplemented by us) & 46.5 & 34.9 & 39.9 \\ 
\color{red}\bf \texttt{R}& {\bf \textsc{MLC}} (reimplemented by us) & 42.2 & 44.9 & 43.5 \\ \hline 
\multicolumn{5}{l}{\emph{Seq2seq based models}}      \\
\color{blue}\bf \texttt{B} & {\bf \textsc{LRN} }  {\cite{liu-etal-2021-fine}}              & 54.5 & 38.9 & 45.4  \\\hline
\multicolumn{5}{l}{\emph{Filter models under our recall-expand-filter paradigm}}      \\
\color{blue}\bf \texttt{B} & {\bf \textsc{Vanilla CE}$_{128}$}   & 47.2 & 48.5 & 47.8 \\ 
\color{blue}\bf \texttt{B} & {\bf \textsc{\name-S$_{128}$}} (Ours)  & 53.2 & 48.3 & {\bf 50.6} \\ 
\color{blue}\bf \texttt{B} & {\bf \textsc{\name-S$_{128}$ w/o C2C}}   (Ours)   & 52.3 & 48.3 & 50.2 \\ 
\color{blue}\bf \texttt{B} & {\bf \textsc{\name-B$_{128}$}} (Ours)    & 49.9 & 50.0 & 49.9 \\ 
\color{blue}\bf \texttt{B} & {\bf \textsc{\name-B$_{128}$ w/o C2C}} (Ours)     & 49.9 & 48.2 & 49.0 \\ \hline
\color{red}\bf \texttt{R} & {\bf \textsc{Vanilla CE}$_{128}$}   & 49.6 & 49.0 & 49.3 \\ 
\color{red}\bf \texttt{R} & {\bf \textsc{\name-S$_{128}$}} (Ours)  & 53.3 & 47.3 & 50.1 \\ 
\color{red}\bf \texttt{R} & {\bf \textsc{\name-S$_{128}$ w/o C2C}}   (Ours)  & 53.2 & 46.6 & 49.7 \\ 
\color{red}\bf \texttt{R} & {\bf \textsc{\name-B$_{128}$}} (Ours)  & 52.5 & 47.9 & 50.1 \\ 
\color{red}\bf \texttt{R} & {\bf \textsc{\name-B$_{128}$ w/o C2C}} (Ours)     & 52.7 & 46.4 & 49.3 \\ \hline
\midrule
\multicolumn{2}{l}{\bf \textit{Large Models on UFET} }     & \bf \textsc{P}    & \bf \textsc{R}   & \bf \textsc{F1}  \\ \midrule
\multicolumn{5}{l}{\emph{MLC-like models}}        \\
\color{red}\bf \texttt{R} & {\bf \textsc{MLC}}  \cite{npcrf}               & 47.8 & 40.4 & 43.8  \\
\color{red}\bf \texttt{R} & {\bf \textsc{MLC-NPCRF}} \cite{npcrf}             & 48.7 & 45.5 & 47.0  \\
\color{red}\bf \texttt{R} & {\bf \textsc{MLC-GCN}} \cite{xiong-etal-2019-imposing}     & 51.2 & 41.0 & 45.5 \\
\color{blue}\bf \texttt{B} & {\bf \textsc{PL}}  \cite{ding2021prompt}       & 59.3 & 42.6 & 49.6  \\
\color{blue}\bf \texttt{B} & {\bf \textsc{PL-NPCRF}}  \cite{npcrf}  & 55.3 & 46.7 & {50.6}\\ \hline
\multicolumn{4}{l}{\emph{Cross-encoder based models and {\bf \textsc{\name}}s}}      \\
\color{red}\bf \texttt{R} & {\bf \textsc{LITE+L}}  \cite{lite}             & 48.7 & 45.8 & 47.2  \\
\color{teal}\bf \texttt{RM} & {\bf \textsc{LITE+NLI+L}} \cite{lite} & 52.4 & 48.9 & {50.6} \\ \hline
\multicolumn{4}{l}{\emph{Filter models under our recall-expand-filter paradigm}}   \\ 
\color{blue}\bf \texttt{B} & {\bf \textsc{Vanilla CE$_{128}$}}   & 50.3 & 49.6 & 49.9 \\ 
\color{blue}\bf \texttt{B} & {\bf \textsc{\name-S$_{128}$}}  (Ours)   & 52.5 & 49.1 & 50.8 \\ 
\color{blue}\bf \texttt{B} & {\bf \textsc{\name-S$_{128}$ w/o C2C}}   (Ours)   & 54.1 & 47.1 & 50.4 \\ 
\color{blue}\bf \texttt{B} & {\bf \textsc{\name-B$_{128}$}} (Ours)    & 54.0 & 48.6 & 51.2 \\ 
\color{blue}\bf \texttt{B} & {\bf \textsc{\name-B$_{128}$ w/o C2C}} (Ours)     & 52.8 & 48.3 & 50.4 \\ \hline
\color{red}\bf \texttt{R} & {\bf \textsc{Vanilla CE$_{128}$}}   & 54.5 & 49.3 & 51.8 \\ 
\color{red}\bf \texttt{R} & {\bf \textsc{\name-S$_{128}$}}  (Ours)   & 50.8 & 49.8  &  50.3 \\ 
\color{red}\bf \texttt{R} & {\bf \textsc{\name-S$_{128}$ w/o C2C}}   (Ours)   & 51.5 & 48.8 & 50.1 \\ 
\color{red}\bf \texttt{R} & {\bf \textsc{\name-B$_{128}$}} (Ours)    & 51.9 & 50.8 & 51.4 \\ 
\color{red}\bf \texttt{R} & {\bf \textsc{\name-B$_{128}$ w/o C2C}} (Ours)     & 51.6 & 51.6 & 51.6 \\ \hline
\color{teal}\bf \texttt{RM} & {\bf \textsc{\name-B$_{128}$ w/o C2C}} (Ours) & 56.3 & 48.5 & {\bf 52.1} \\ \hline
\midrule
\end{tabular}}
\caption{Macro-averaged UFET result. {\bf \textsc{LITE+L}} is LITE without NLI pretraining, {\bf \textsc{LITE+L+NLI}} is the full LITE model. Methods marked by $\dagger$ utilize either distantly supervised or augmented data for training. {\bf \textsc{\name-S$_{128}$}} denotes we use $128$ candidates recalled and expanded from the first two stages.}
\label{tab:ufet}
\end{table}
\begin{table}[t]
\centering
\scalebox{0.75}{
\renewcommand{\arraystretch}{1}
\begin{tabular}{cllll} \toprule
\multicolumn{2}{l}{\bf \textit{Models on CFET} }     & \bf \textsc{P}    & \bf \textsc{R}   & \bf \textsc{F1}  \\ \midrule
\multicolumn{5}{l}{\emph{MLC-like models}}        \\
\color{purple}\bf \texttt{N}& {\bf \textsc{MLC}} & 55.8 & 58.6 & 57.1 \\  
\color{purple}\bf \texttt{N}& {\bf \textsc{MLC-NPCRF}} \cite{npcrf}     & 57.0 & 60.5 & 58.7 \\ 
\color{purple}\bf \texttt{N}& {\bf \textsc{MLC-GCN}} \cite{xiong-etal-2019-imposing}   & 51.6 & 63.2 & 56.8 \\ 
\color{brown}\bf \texttt{C}& {\bf \textsc{MLC}} & 54.0 & 59.5 & 56.6 \\  
\color{brown}\bf \texttt{C}& {\bf \textsc{MLC-NPCRF}} \cite{npcrf}   & 54.0 & 61.6 & 57.3 \\  
\color{brown}\bf \texttt{C}& {\bf \textsc{MLC-GCN}} \cite{xiong-etal-2019-imposing} & 56.4 & 58.6 & 57.5 \\ \midrule 
\multicolumn{5}{l}{\emph{Filter models under our recall-expand-filter paradigm}}      \\
\color{purple}\bf \texttt{N} & {\bf \textsc{Vanilla CE}}   & 57.6 & 64.3 & 60.7 \\ 
\color{brown}\bf \texttt{C} & {\bf \textsc{Vanilla CE}}   & 54.0 & 63.3 & 58.3 \\  \hline
\color{purple}\bf \texttt{N} & {\bf \textsc{\name-S$_{64}$}} (Ours)  & 58.4 & 62.1 & 60.2 \\ 
\color{purple}\bf \texttt{N} & {\bf \textsc{\name-S$_{64}$ w/o C2C}}   (Ours)   & 59.1 & 61.5 & 60.3 \\ 
\color{purple}\bf \texttt{N} & {\bf \textsc{\name-B$_{64}$}} (Ours)    & 56.7 & 66.1 & 61.1 \\ 
\color{purple}\bf \texttt{N} & {\bf \textsc{\name-B$_{64}$ w/o C2C}} (Ours)     & 58.8 & 64.1 & 61.4 \\ \hline
\color{brown}\bf \texttt{C} & {\bf \textsc{\name-S$_{64}$}} (Ours)  & 55.5 & 62.6 & 58.8 \\ 
\color{brown}\bf \texttt{C} & {\bf \textsc{\name-S$_{64}$ w/o C2C}}   (Ours)   & 54.0 & 63.4 & 58.3 \\ 
\color{brown}\bf \texttt{C} & {\bf \textsc{\name-B$_{64}$}} (Ours)    & 55.0 & 63.5 & 59.0 \\ 
\color{brown}\bf \texttt{C} & {\bf \textsc{\name-B$_{64}$ w/o C2C}} (Ours)     & 57.3 & 61.3 & 59.3 \\ \hline
\midrule
\end{tabular}}
\caption{Macro-averaged CFET result.}
\label{tab:cfet}
\end{table}

In this section, we report the performance of {\bf \textsc{MCCE}} variants as the filter models and compare them with various strong baselines that we will introduce later. We also compare the inference speed of different models in this section. For filter models, we treat the number of candidates $K$ recalled and expanded by the first two stages as hyper-parameters, and tune it on the development set. We found the choice of PLM backbones has a non-negligible effect on the performance, and the PLM backbone of previous methods varies. Therefore for fairer comparisons to baselines, we conduct experiments of {\bf \textsc{\name}} using different backbone PLMs for our {\bf \textsc{\name}} models and report the results. For all {\bf \textsc{\name}} models, we use AdamW optimizer with a learning rate tuned between $5\times 10^{-6}$ and $2\times 10^{-5}$. The batch size we use is $4$ and we train the models for at most $50$ epochs with early stopping. {\bf \textsc{UFET}} also provides a large dataset obtained from distant supervision such as entity linking, we do not use it and only train and evaluate our models on human-labeled data.
\paragraph{Baselines}
The {\bf \textsc{MLC}} model we used for the recall stage and the cross-encoder ({\bf \textsc{CE}}) we introduced in Sec. \ref{sec:vanilla_ce} are natural baselines. We also compare our methods with recent PLM-based methods. {\bf \textsc{LDET} }\cite{onoe-durrett-2019-learning} is an MLC with Bert-base-uncased and ELMo \cite{elmo} trained on 727k examples automatically denoised from the distantly labeled UFET. {\bf \textsc{GCN} }\cite{xiong-etal-2019-imposing} uses GCN to model type correlations and obtain type embeddings. Types are scored by dot-product of mention and type embeddings. The original paper uses BiLSTM as the mention encoder and we use the results re-implemented by \citet{npcrf} using RoBERTa-large. {\bf \textsc{Box4Type} }\cite{box4types} uses Bert-large as the backbone and uses box embedding to encode mentions and types for training and inference. {\bf \textsc{LRN} }\cite{liu-etal-2021-fine} use Bert-base as the encoder and an LSTM decoder to generate types in a seq2seq manner. {\bf \textsc{MLMET} }\cite{mlmet} is a {\bf \textsc{MLC}} with Bert-base, but first pretrained by the distantly-labeled data augmented by masked word prediction, then finetuned and self-trained on the 2k human-annotated data. {\bf \textsc{PL}} \cite{ding2021prompt} uses prompt learning for entity typing. {\bf \textsc{DFET} }\cite{dfet} uses {\bf \textsc{PL}} as backbone and is a multi-round automatic denoising method for 2k labeled data. {\bf \textsc{LITE} }\cite{lite} is the previous SOTA system that formulates entity typing as textual inference. {\bf \textsc{LITE}} uses RoBERTa-large-MNLI as the backbone, and is a cross-encoder (introduced in Sec. \ref{sec:vanilla_ce}) with designed templates and a hierarchical loss. \citet{npcrf} proposes {\bf \textsc{NPCRF}} to enhance backbones such as {\bf \textsc{PL}} and {\bf \textsc{MLC}} by modeling type correlations, and reach performance comparable to {\bf \textsc{LITE}}.

\paragraph{Naming Conventions}
Let {\bf \textsc{\name-S}} be the {\bf \textsc{\name}} model that treats candidates as sub-tokens, and {\bf \textsc{\name-B}} be the model representing candidates as fixed-size blocks. The {\bf \textsc{\name}} model without {\bf \textsc{C2C}} attention (mentioned in Sec. \ref{sec:attn}) is denoted as {\bf \textsc{\name-B} w/o C2C}. For PLM backbones used in {\bf \textsc{UFET}}, we use {\color{blue} \bf \texttt{B}}, {\color{red} \bf \texttt{R}}, {\color{teal} \bf \texttt{RM}} to denote BERT-base-cased \cite{bert}, RoBERTa \cite{liu2019roberta}, and RoBERTa-MNLI \cite{liu2019roberta} respectively. For {\bf \textsc{CFET}}, we adopt two widely-used Chinese PLM, BERT-base-Chinese and NeZha-base-Chinese, and denote them as {\color{brown} \bf \texttt{C}} and {\color{purple} \bf \texttt{N}} respectively. 

\paragraph{UFET Results} We show the results of {\bf \textsc{UFET}} dataset in Table \ref{tab:ufet}. The results show that: (1) The recall-expand-filter paradigm is effective. Filter models outperform all baselines without the paradigm by a large margin. The vanilla CE under our paradigm reaches $51.8$ F1 compared to more complexed CE {\bf \textsc{LITE}} with $50.6$ F1 (2) {\bf \textsc{\name}} models reach SOTA performances. {\bf \textsc{\name-S$_{128}$}} with BERT-base performs best and reaches {\bf 50.6} F1 score, which is comparable to previous SOTA performance of large models such as {\bf \textsc{LITE+NLI+L}} and {\bf \textsc{PL+NPCRF}}. Among large models, {\bf \textsc{\name-B$_{128}$ w/o C2C}} also reaches SOTA performance with {\bf 52.1} F1 score. (3) {\bf \textsc{C2C}} attention is not necessary on large models, but is useful in base models. (4) Large models can utilize type semantics better. We found {\bf \textsc{\name-B}} outperforms {\bf \textsc{\name-S}} on large models, but underperforms {\bf \textsc{\name-S}} on base models. (5) Backbone PLM matters. We found the performance of {\bf \textsc{Vannila CE}} under our paradigm is largely affected by the PLM it used. It reaches $47.8$ F1 with BERT-base and $51.8$ F1 with RoBERTa-large. For {\bf \textsc{\name}} models, we found {\bf \textsc{\name}} performs better than {\bf \textsc{\name-B}} with BERT, and worse than {\bf \textsc{\name-B}} with RoBERTa. 

\begin{table*}[t]
\centering
\scalebox{0.9}{
\begin{tabular}{lllcc} \toprule
\bf \textsc{Model}  & \bf \textsc{\# FP} & \bf \textsc{Attn} & \bf \textsc{sents/sec} & \bf \textsc{F1} \\ \midrule
{\bf \textsc{MLC}} & \small{$1$}  & \small{$L_S^2D$} & 58.8 & 43.8\\
{\bf \textsc{LITE+NLI+L (CE)}}  & \small{$N$}  & \small{$L_S^2D$} & 0.02 & 50.6\\ \midrule \hline
\multicolumn{5}{l}{\emph{filter stage inference speed.}}  \\
{\bf \textsc{Vanilla CE$_{128}$}}  & \small{$128$}  & \small{$L_S^2D$} & 1.64 & 51.8 \\ 
{\bf \textsc{\name-S$_{128}$}}  & \small{$1$}  & \small{$(L_S+128)^2D$} & 60.8 & 50.1 \\ 
{\bf \textsc{\name-B$_{128}$}}  & \small{$1$}  & \small{$(L_S+128B)^2D$} & 22.3 & 51.4\\ 
{\bf \textsc{\name-B$_{128}$ w/o C2C}}  & \small{$1$}  & \small{$(L_S^2+256L_S B + 128 B^2)D$} & 25.2 & {\bf 52.1}\\ \bottomrule
\end{tabular}}
\caption{Inference speed comparison of models. {\bf \textsc{\# FP}} means the number of PLM forward passes required by a single inference. {\bf \textsc{ATTN}} column lists the theoretical attention complexity.  We also report the practical inference speed {\bf \textsc{sents/sec}} and the {\bf \textsc{F1}} scores on {\bf \textsc{UFET}} with RoBERTa-large architecture.}
\label{tab:speed}
\end{table*}

\begin{table}[t]
\centering
\scalebox{0.85}{
\renewcommand{\arraystretch}{1}
\begin{tabular}{cllll} \toprule
\multicolumn{2}{l}{\bf \textit{Models} }     & \bf \textsc{P}    & \bf \textsc{R}   & \bf \textsc{F1}  \\ \midrule
\multicolumn{5}{l}{\emph{coarse (9 types) Open Entity}}        \\ \hline
\color{red}\bf \texttt{R} & {\bf \textsc{MLC}}   & 76.8 & 78.5 & 77.6 \\ 
\color{red}\bf \texttt{R} & {\bf \textsc{Vanilla CE$_{9}$}}   & 82.3 & 81.0 & 81.6 \\ 
\color{red}\bf \texttt{R} & {\bf \textsc{\name-S$_{9}$}}   & 77.0 & 87.7 & 82.0 \\ 
\color{red}\bf \texttt{R} & {\bf \textsc{\name-B$_{9}$ w/o C2C}}   & 77.2 & 85.4 & 81.1 \\ \hline
\multicolumn{5}{l}{\emph{fine (130 types)}}        \\ \hline
\color{red}\bf \texttt{R} & {\bf \textsc{MLC}}   & 70.4 & 63.7 & 66.9  \\ 
\color{red}\bf \texttt{R} & {\bf \textsc{Vanilla CE}$_{130}$}   & 67.9 & 66.4 & 67.1 \\ 
\color{red}\bf \texttt{R} & {\bf \textsc{\name-S$_{130}$}}   & 65.8 & 71.8 & 68.7 \\ 
\color{red}\bf \texttt{R} & {\bf \textsc{\name-B$_{130}$ w/o C2C}}   & 64.1 & 70.5 & 67.1 \\ \hline
\midrule
\end{tabular}}
\caption{Micro-averaged results on UFET fine and coarse.}
\label{tab:ufet-coarse-fine}
\end{table}

\paragraph{CFET Results} We conduct experiments on {\bf \textsc{CFET}} and compare {\bf \textsc{\name}} models with several strong baselines:  {\bf \textsc{NPCRF}} and {\bf \textsc{GCN}} with MLC-like architecture, and {\bf \textsc{Vanilla CE}} under out paradigm which is proved to be better than {\bf \textsc{LITE}} on {\bf \textsc{UFET}}. The results are shown in Table \ref{tab:cfet}. Similar to results in {\bf \textsc{UFET}}, filter models under our paradigm significantly outperform MLC-like baselines, $+2.0$ F1 for Nezha-base and $+1.8$ F1 for BERT-base-Chinese. In {\bf \textsc{CFET}}, {\bf \textsc{\name}-B} is significantly better than {\bf \textsc{\name}-S}, on both Nezha-base and BERT-base-Chinese, indicating the importance of type semantics in Chinese language. We also find that {\bf \textsc{\name} w/o C2C} is generally better than  {\bf \textsc{\name} w/ C2C}, it is possibly because the C2C attention distracts the candidates from attending to mention and contexts.
\paragraph{Speed Comparison} Table \ref{tab:speed} shows the theoretical inference complexity (number of PLM forward passes, and attention complexity), and practical inference speed (number of sentences inferred per second) of different models. We conduct the speed test using NVIDIA TITAN RTX for all models, and the inference batch size is 4.
At the filter stage, the inference speed of {\bf \textsc{\name-S}} is on par with {\bf \textsc{MLC}} (even slightly faster because we don't need to score all types), and is about 40 times faster than {\bf \textsc{Vannila CE}} and thousands of times faster than {\bf \textsc{LITE}}. {\bf \textsc{\name-B w/o C2C}} is not significantly faster than {\bf \textsc{\name-B}} as expected. It's possibly because the computation related to the block attention is not fully optimized by existing deep learning frameworks. The speed advantage of {\bf \textsc{\name-B w/o C2C}} over {\bf \textsc{\name-B}} will be greater with more candidates.


\subsection{Fine-grained and Coarse-grained Entity Typing}
We also conduct experiments on Fine-grained (130-class) and Coarse-grained (9-class, also known as ``Open Entity'') entity typing, and the results are shown in Table \ref{tab:ufet-coarse-fine}. As the type candidate set is much smaller in these settings, we skip the recall and expand stages and directly run the filter models and compare them to baselines. Results show that both {\bf \textsc{\name}-S} and {\bf \textsc{\name}-B} are still better than {\bf \textsc{MLC}} and {\bf \textsc{Vanilla CE}}, and {\bf \textsc{\name}-S} is better than {\bf \textsc{\name}-B} on coarser-grained cases possibly because the coarser-grained types are simpler in surface-forms and {\bf \textsc{\name}-S} will not lose many type semantics.





\section{Analysis}
\label{sec:analyze}
\subsection{Importance of Expand Stage}
We perform the ablation study on the importance of the expand stage and show the results in Table \ref{fig:ablation_expand}. We compare the performances of {\bf \textsc{\name-S}} using the expanded or the not expanded candidate sets on {\bf \textsc{UFET}} and {\bf \textsc{CFET}}. We replace the last $48$ candidates recalled by MLC with candidates expanded by MLM and exact matching for {\bf \textsc{UFET}}, and $10$ candidates for {\bf \textsc{CFET}}. Results show that expand stage has a positive effect on performance, it improves the final recall by $+1.0$ and $+2.2$ on  {\bf \textsc{UFET}} and {\bf \textsc{CFET}} without harming the precision.

\begin{table}[t]
\centering
\scalebox{0.75}{
\renewcommand{\arraystretch}{1}
\begin{tabular}{cllll} \toprule
\multicolumn{2}{l}{\bf \textit{Ablation of Expand Stage} }     & \bf \textsc{P}    & \bf \textsc{R}   & \bf \textsc{F1}  \\ \midrule
\multicolumn{5}{l}{\bf \textsc{UFET\ \  MCCE with C2C BERT-large}} \\
\color{blue}\bf \texttt{B} & {\bf \textsc{\name-S$_{128}$ }} (Ours)     & 52.5 & 49.1 & 50.8 \\ 
\color{blue}\bf \texttt{B} & {\bf \textsc{\name-S$_{128}$ w/o Expand }} (Ours)     & 52.7 & 48.1 & 50.3\\ \hline
\multicolumn{5}{l}{\bf \textsc{CFET\ \  MCCE with C2C BERT-base-Chinese}} \\
\color{brown}\bf \texttt{C} & {\bf \textsc{\name-S$_{64}$}} (Ours)  & 55.5 & 62.6 & 58.8 \\ 
\color{brown}\bf \texttt{C} & {\bf \textsc{\name-S$_{64}$ w/o Expand}}   (Ours)   & 55.4 & 60.4 & 57.8 \\ \hline
\midrule
\end{tabular}}
\caption{Ablation study of expand stage.}
\label{fig:ablation_expand}
\end{table}

\subsection{Attentions}
We conduct an ablation study on S2S, C2S, S2C, and C2C attention introduced in Sec. \ref{sec:attn} and show the results in Table \ref{tab:attn}. From the results, we are surprised to find that removing C2C and S2S doesn't have a big negative impact on performance. The {\bf \textsc{\name-S}} using BERT-base reaches $48.8$ F1 even without both C2C and S2S attention. One possible reason is that the interaction between sub-tokens in the sentence can be achieved indirectly by first attending to the candidates and then being attended back by the candidate in the next layer. We also find that the C2S is necessary for the task ($18.7$ F1 without C2S) because we rely on the mention and its context to encode and classify candidates. Furthermore, we found that it is important for sentences to attend to all candidates (S2C), indicating that the interaction between the sentence and different types is crucial for the task.

\begin{table}[t]
\centering
\scalebox{0.75}{
\renewcommand{\arraystretch}{1}
\begin{tabular}{cllll} \toprule
\multicolumn{2}{l}{\bf \textit{Analysis about attention on UFET}}     & \bf \textsc{P}    & \bf \textsc{R}   & \bf \textsc{F1}  \\ \midrule
\multicolumn{5}{l}{\bf \textsc{\name-S using BERT-base}} \\
\color{blue}\bf \texttt{B} & {\bf \textsc{\name-S$_{128}$} FULL}     & 53.2 &  48.3 & 50.6 \\ 
\color{blue}\bf \texttt{B} & {\bf \textsc{\name-S$_{128}$ w/o C2C }}     & 52.3 & 48.3 & 50.2 \\
\color{blue}\bf \texttt{B} & {\bf \textsc{\name-S$_{128}$ w/o S2S }}     & 50.6 & 48.4 & 49.4 \\
\color{blue}\bf \texttt{B} & {\bf \textsc{\name-S$_{128}$ w/o S2C }}     & 48.7 & 47.1 & 47.9 \\ 
\color{blue}\bf \texttt{B} & {\bf \textsc{\name-S$_{128}$ w/o C2S }}     & 19.7 & 17.4 & 18.7\\
\color{blue}\bf \texttt{B} & {\bf \textsc{\name-S$_{128}$ w/o S2S,C2C }}     & 50.2 & 47.3 & 48.8\\
\bottomrule
\end{tabular}}
\caption{Attention analysis.}
\label{tab:attn}
\end{table}








% \subsection{Influence of Candidate Size}



% In this section, we review and analyze related works. We begin with a review of graph convolutional models and then transition to graph attention models. Lastly, we discuss other works on graphs. 

\begin{figure*}[tb!] 
\centering
\includegraphics[width=0.8\textwidth]{acc_run_time_v2.pdf}
\caption{Performance over training time on Pubmed and Reddit. \method{} is the fastest while achieving competitive performance. 
We are not able to benchmark the training time of GaAN and DGI on Reddit because the implementations are not released. 
}
\label{fig:run_time}
\end{figure*}
%
\subsection{Graph Neural Networks}
\citet{Bruna13} first propose a spectral graph-based extension of convolutional networks to graphs. 
In a follow-up work, ChebyNets \cite{Defferrard16} define graph convolutions using Chebyshev polynomials to remove the computationally expensive Laplacian eigendecomposition. GCNs \cite{gcn} further simplify graph convolutions by stacking layers of first-order Chebyshev polynomial filters with a redefined propagation matrix $\mathbf{S}$. 
\citet{FastGCN} propose an efficient variant of GCN based on importance sampling, and \citet{Hamilton17} propose a framework based on sampling and aggregation. 
\citet{dcnn}, \citet{n-gcn}, and \citet{liao2018lanczosnet} exploit multi-scale information by raising $\mathbf{S}$ to higher order.
% \citet{xu2018how} introduce Graph Isomorphism Networks which is claimed to be the most expressive variant of GNNs.
\citet{xu2018how} study the expressiveness of graph neural networks in terms of their ability to distinguish any two graphs and introduce Graph Isomorphism Network, which is proved to be as powerful as the Weisfeiler-Lehman test for graph isomorphism. 
\citet{Klicpera19} separate the non-linear transformation from propagation by using a neural network followed by a personalized random walk.
There are many other graph neural models~\cite{Monet, EP17, Li18}; we refer to \citet{gnn_review, battaglia2018relational, wu2019comprehensive} for a more comprehensive review. 

% GLN
% \citet{agnn} discover that a linear version of GCN can perform competitively and develop a attention-based GCN model.
% \citet{cai2018simple} propose an effective linear baseline for graph classification using node degree statistics.
% \citet{Buchnik18} show that self-training can improve the base linear graph model.
% \citet{Li2019LabelES} propose a generalized version of label propagation and provide a similar spectral analysis on the renormalization trick.

Previous publications have pointed out that simpler, sometimes linear models can be effective for node/graph classification tasks. \citet{agnn} empirically show that a linear version of GCN can perform competitively and propose an attention-based GCN variant. \citet{cai2018simple} propose an effective linear baseline for graph classification using node degree statistics. \citet{Buchnik18} show that models which use linear feature/label propagation steps can benefit from self-training strategies. 
\citet{Li2019LabelES} propose a generalized version of label propagation and provide a similar spectral analysis of the renormalization trick.
% Unlike these previous works, we ...


% Recently, \citet{agnn} proposed an attention-based GCN model for citation networks. The authors point out that a linear version of GCN could perform similarly to GCNs on these classification tasks, which matches our findings. \citet{cai2018simple} propose an effective linear baseline for non-attribute graph classification using simple node degree statistics.

% \subsection{Graph Attention Models}
Graph Attentional Models learn to assign different edge weights at each layer based on node features and have achieved state-of-the-art results on several graph learning tasks \citep{gat, agnn, zhang2018gaan, ADGPM}.
However, the attention mechanism usually adds significant overhead to computation and memory usage. 
We refer the readers to \citet{attention-survey} for further comparison.

\subsection{Other Works on Graphs} 
Graph methodologies can roughly be categorized into two approaches: graph embedding methods and graph laplacian regularization methods. 
Graph embedding methods \citep{Weston2008, Perozzi14, Yang16, infomax} represent nodes as high-dimensional feature vectors. 
Among them, DeepWalk~\citep{Perozzi14} and Deep Graph Infomax (DGI)~\citep{infomax} use unsupervised strategies to learn graph embeddings.
DeepWalk relies on truncated random walk and uses a skip-gram model to generate embeddings, whereas DGI trains a graph convolutional encoder through maximizing mutual information. 
Graph Laplacian regularization \citep{Zhu03, Zhou04,Belkin04b,Belkin2006} introduce a regularization term based on graph structure which forces nodes to have similar labels to their neighbors.
Label Propagation~\citep{Zhu03} makes predictions by spreading label information from labeled nodes to their neighbors until convergence. 
\section{Conclusion}
\label{ss: conclusion}

% summary of approach
This paper presents a methodology to evaluate the effectiveness of evasions and its application to studying PDF malware scanners.
Our implementation of the methodology, the Chameleon framework, automatically generates and enriches malicious documents with one or multiple evasions.
We use these documents for an in-depth study of \nbAnalyzers{} PDF scanners and how they are affected by evasions.
More broadly, our methodology can also be used for studying evasions of other malware types, e.g., malicious executables.

% main take-aways
The overall result of our study is cause for concern.
We show that the studied evasions are surprisingly effective in fooling state-of-the-art scanners.
In particular by combining evasions, attackers can bypass modern defenses in both static and dynamic scanners.
Moreover, we find huge variations across scanners, enabling targeted attacks based on evasions picked specifically for a targeted scanner.
All these findings are a call to arms for future work on anti-evasion techniques.

Our work will support future efforts toward improving malware scanners in several ways.
First, the results of our study help security vendors to better understand their vulnerability to specific evasions and to focus their attention on mitigating the most effective evasions.
Second, we are releasing the corpus of malicious, evasive documents generated by Chameleon as a ready-to-use benchmark.
We are in contact with several developers of PDF scanners, and some of them, e.g., SploitGuard and SAFE-PDF, have already used our benchmark to test and improve their security scanners.
Finally, the Chameleon framework provides a basis for expanding the set of benchmarks by incorporating future evasions, exploits, and payloads.


% Entries for the entire Anthology, followed by custom entries
\bibliography{main}
\bibliographystyle{acl_natbib}

% \appendix
% \section{Example Appendix}
% \label{sec:appendix}
\end{document}

\section{Methodology}
\begin{figure*}[t]
    \centering
    \scalebox{0.28}{
    \includegraphics{src/img/ccf_model.pdf}}
    \caption{Multi-candidate cross-encoder (MCCE).}
    \label{fig:ccf}
\end{figure*}
Inspired by techniques in information retrieval \cite{ir} and entity linking \cite{wu2019zero}, we decompose the training and inference of UFET into three stages as illustrated in Figure \ref{fig:paradigm}: (1) Recall stage to reduce the type candidate size (e.g., from $N=10k$ to $K=100$) while guaranteeing the recall rate by an efficient MLC model. (2) Expand stage to incorporate lexical information using exact matching and weak supervision \cite{mlmet} from large pretrained language models such as BERT-Large \cite{bert} to improve recall rate. (3) Filter stage to filter the expanded type candidates to obtain final prediction. For the filter stage, we propose an efficient model: Multi-Candidate Cross-Encoder (MCCE) to concurrently encode and filter type candidates of a given mention with only a single forward pass. 
\subsection{Recall Stage}
To prune the type candidates set, we train a very efficient MLC model introduced in Sec. \ref{sec:mlc} and select the model based on the recall rate (e.g. recall@64) on the development set. Then we use it to infer the top $K_1$ (typically less than 256) candidates $\mathcal{C}_i^R$ for each data point $(m_i, c_i)$ for train, development, and test set. We compare MLC with a widely-used baseline model BM25 \cite{bm25} and show its advantages in Sec. \ref{sec:recall}. 
% There are other options (e.g., Bi-Encoder \cite{wu2019zero}) for the recall model. However, we choose MLC because it is easy to optimize and has the fastest inference speed\footnote{MLC model can be categorized as a simple Bi-Encoder that uses embedding to encode types.}.
\subsection{Expand Stage}
Due to the lack of training data per type, we found that the MLC we used in the recall stage easily overfits the train set, and is hard to predict the types that only appear in the development and test set. In {\bf \textsc{UFET}} dataset, 30\% of the types in the development set are unseen. To this end, we utilize lexical information using exact match and weak supervision from the masked language model (MLM) to expand the recalled candidates. Both exact match and MLM are able to recall unseen type candidates without any training. 
\paragraph{Exact Match} MLC and Bi-Encoder recall candidates by dense representations. They are known weak at identifying and utilizing the lexical matching information between the input and types \cite{matching_info1, matching_info2}. However, 
 types are free-formed in UFET (e.g., \textit{president, businessman}), and are very likely to appear in the context or mention (e.g., the mention is \textit{`the \textbf{president} Joe Biden'}). To this end, we first find all nouns in the context and mention by NLTK\footnote{nltk.tag package \url{https://www.nltk.org}} POS tagger and normalize their forms, then we recall types that exactly matched with these nouns.
\paragraph{Weak Supervision from MLM} Inspired by recently prompt-based methods for entity typing \cite{ding2021prompt, dfet}, we recall candidates by asking PLMs to fill masks in prompts. Suppose a type $y_j \in \mathcal{Y}$ can be tokenized into $l$ subwords $w_1, \cdots w_l$. To score $y_j$ given $m_i, c_i$, we first formulate the input as in Figure \ref{fig:prompt_recall}.
\begin{figure}[h]
    \centering
    \scalebox{0.2}{
    \includegraphics{src/img/prompt_recall.pdf}}
    \caption{Recall from MLM using prompts.}
    \label{fig:prompt_recall}
\end{figure}
where $c_i^l, c_i^r$ are left and right context of $m_i$, and \texttt{`such as'} is the template we use to induce types. The input is then fed into BERT-large-uncased\footnote{We use the PLM from \url{https://huggingface.co}} for masked language model to obtain the probabilities of subwords, the score of $y_j$ is calculated by $ s^{MLM}_{j} = (\sum_{n=1}^l \log p_n)/l$, where $p_l$ denotes the probability of subword $w_l$ predicted by the PLM. We rank all types by enumerating all possible $l$ and recall $K_2$ additional candidates that haven't been recalled by the recall stage and exact match. We found that the expand stage improves recalls and contributes to the performance in Sec. \ref{sec:exp_expand}.
\subsection{Filter Stage}
In the filter stage, we use the recall and expand method introduced above to efficiently generate type candidates $\mathcal{C}_i$ for data in the train, development, and test set. For training, $\mathcal{C}_i$ is used to produce positive and hard negative type candidates. For inference, $\mathcal{C}_i$ is the candidate pool for the trained filter models. Let $|\mathcal{C}_i|=K$ and $K$ is typically less than 256. 
\subsubsection{CE} A trivial idea is to train a CE model introduced in Sec. \ref{sec:vanilla_ce} to filter $\mathcal{C}_i$ instead of filtering the whole type set $\mathcal{Y}$. The positive type $y^{+}$ and negative $y^{\_}$ type are both sampled from $\mathcal{C}_{i}$ and are used for calculating marginal ranking loss. To infer types, we also recall and expand $K$ candidates and score these candidates by $K$ forward passes to predict types. As $K << |\mathcal{Y}|$, CE with our Recall-Expand-Filter paradigm is much faster than vanilla CE. However, it's still inefficient compared to MLC-like models that concurrently predict scores of all types in a single forward pass. For faster inference and training, we propose multi-candidate cross-encoders ({\bf \textsc{\name}}) and introduce them in the next section.

\section{Multi-Candidate Cross-Encoder (\name)}
\label{sec:ccf}
 In this section, we introduce {\bf \textsc{\name}} for filtering candidates in one forward pass and propose several variants. 
\subsection{Overall Introduction of \name}
As shown in Figure \ref{fig:ccf}, compared to CE that concatenates one candidate at a time, {\bf \textsc{\name}} models concatenate all candidates in $\mathcal{C}_i$ with the mention and context. The input is then fed into the PLM to obtain the hidden states of each candidate as their representation. Finally, we use an MLP to concurrently score all candidates.
\begin{equation}
\begin{aligned}
\bm{h}_{1:K}  &= \texttt{PLM}(\texttt{[CLS]} \ c_i \ \texttt{[SEP]}\  m_i \ \texttt{[SEP]} \ t_{1:K} \ )  \\
\bm{s}_{1:K}  &= \texttt{Linear}(\bm{h}_{1:K}) 
\end{aligned}
\end{equation}
where $t_{1:K}$ is the short for $t_1, \cdots, t_K$, and $t_j \in \mathcal{C}_i$. Similarly, $\bm{h}_{1:K}$ and $s_{1:K}$ are hidden representations and scores of corresponding candidates respectively. 

\paragraph{Training and Inference} For training, we found that all positive types are ranked very high in the training candidates, which is not the case for the development and test data. To prevent the filter model from overfitting the order of training candidates and only learning to predict the first several candidates, we keep permuting type candidates during training. Same as the MLC model mentioned in Sec. \ref{sec:mlc}, we use the Binary Cross-Entropy loss as the training objective and tune a threshold of probabilities on the development set for inference.

In the next subsection, we discuss different model configurations of {\bf \textsc{\name}} regarding the input formats of candidates and attention mechanisms.

\subsection{Different Input Formats of Candidates}
\paragraph{Average of type sub-tokens} We treat each type $t_j \in \mathcal{Y}$ as a new token $u_j$ and add it to the vocabulary of PLM. The static embedding (layer 0 embedding of PLM) of $u_j$ is properly initialized by the average static embedding of $t_j$'s sub-tokens. As type candidates are capsuled into single tokens, the candidate representation $t_j$ is simply the last hidden state of $u_j$. The reasons for representing each type as a single token is (1) The max sequence length allowed by most PLMs is limited to 512, compressing types into single tokens is position saving. (2) Types in {\bf \textsc{UFET}} are tokenized into $2.1$ sub-tokens in average (by RoBERTa's tokenizer). Compressing types will not lose too many type semantics.
\paragraph{Fixed-size sub-token block} To preserve more type semantics, we place each candidate into a fixed-sized block as shown in Figure \ref{fig:cand_block}. We found the fixed block size makes PLM easier to enable the parallel implementation of different attention mechanisms that we will introduce next. We use the first hidden state in the block as the candidate representation.
\begin{figure}[h]
    \centering
    \scalebox{0.3}{
    \includegraphics{src/img/cand_block.pdf}}
    \caption{Illustration of candidate block.}
    \label{fig:cand_block}
\end{figure}

\subsection{Attentions in \name}
\label{sec:attn}
There are four kinds of attention in {\bf \textsc{\name}} as shown in Figure \ref{fig:attn1}, sentence to sentence (S2S), sentence to candidates (S2C), candidate to sentence (C2S), and candidate to candidate (C2C). As we score candidates based on mention and its context, attention from candidates to the sentence (C2S) is necessary. However, the necessity of C2C, S2S, and S2C is questionable. As our analytical experiment in Sec. \ref{sec:analyze} shows, it is important for words in the sentence to attend to all candidates (S2C), and is useful to have self-attention in the sentence (S2S), but the attentions (C2C) between different candidates are unnecessary. Based on these findings, we propose a new variant of {\bf \textsc{\name}} that the C2C attention is discarded in computation as shown in the right part of Figure \ref{fig:attn1}. Let $L_S$ and $L_C$ be the number of sub-tokens used by the sentence and candidates respectively. We can formulate the attention query of the sentence as $\bm{Q}_S=[\bm{q}^s_{1};\cdots;\bm{q}^s_{L_S}] \in \mathbb{R}^{L_S \times D}$,  where $\bm{q}^s_i$ is the query vector of the $i$-th sub-token in the sentence, and $D$ is the embedding dimension. Similarly, the query of candidates is formulated as $\bm{Q}_C=[\bm{q}^c_{1};\cdots;\bm{q}^c_{L_C}] \in \mathbb{R}^{L_C \times D}$. When we treat candidates as average of sub-tokens, $\bm{q}^c_i$ is a $D$-dimensional vector, and when we use fixed-sized blocks to place candidates, $\bm{q}^c_i \in \mathbb{R}^{B \times D}$ is the concatenation of the query vectors in the $i$-th candidate block and $B$ is the number of sub-tokens in a block. The keys and values are defined similarly as $\bm{K}_C, \bm{V}_C, \bm{O}_C \in \mathbb{R}^{L_C \times D},  \bm{K}_S, \bm{V}_S, \bm{O}_S \in \mathbb{R}^{L_S \times D}$. The attention outputs are computed as:
\begin{align}
& \bm{O}_S  = \text{Softmax} \big( \frac{\bm{Q}_S [\bm{K}_S; \bm{K}_C]^T}{\sqrt{D}} \big) \cdot [\bm{V}_S; \bm{V}_C] \\
&[\bm{A}_{CS}; \bm{A}_{CC}]  = \text{Softmax} \big( \frac{[\bm{Q}_C \bm{K}_S^T; \bm{M}_C^T]}{\sqrt{D}} \big)  \\
&\bm{M}_C  =  [{\bm{q}_1^c}^T \bm{k}_1^c; \cdots; {\bm{q}_{L_C}^c}^T \bm{k}_{L_C}^c] \\
&\bm{A}_{CC} = [\bm{a}^c_{1}; \cdots; \bm{a}^c_{L_C}] \\
&\bm{O}_C = \bm{A}_{CS} \bm{V}_S + \sum_{j=1}^{L_C} \bm{a}_j \bm{v}^c_j
\end{align}
where $\bm{A}_{CC}$ is the intra-candidate or intra-block attention, and $\bm{a}_j^c$ is a scaler when we treat candidates as average of sub-tokens and is a $B \times B$ matrix when we represent candidates as blocks. The last step (Eq. 8) can be parallelly implemented by Einstein summation.
In most cases, candidate length $L_C$ is significantly larger than sentence length $L_S$. As a result, by ignoring the C2C attention, the inference speed is further improved because the time complexity of the attention is significantly reduced from $O(D(L_S+L_C)^2)$ to $O(D(L_S^2+2L_SL_C+B^2L_C))$. More importantly, the space complexity in attention also gets reduced from $O((L_S+L_C)^2)$ to $O(L_S^2+2L_SL_C))$, which allows us to filter more candidates concurrently. The improvement in space and time complexity by discarding C2C attention is more obvious when the number of candidates becomes larger.

\begin{figure}[h]
    \centering
    \scalebox{0.22}{
    \includegraphics{src/img/attn_all.pdf}}
    \caption{Attentions in {\bf \textsc{\name}} (left), and {\bf \textsc{\name}}  without candidate-to-candidate (C2C) attention (right).}
    \label{fig:attn1}
\end{figure}
\section{Background}
\subsection{Problem Definition}
Given an entity mention $m_i$ within its context sentence $c_i$, ultra-fine entity typing (UFET) aims to predict its correct types $y^g_i \subset \mathcal{Y}$, where $y_i^g$ is the gold types of the $i$-th mention and is a subset of a large type set $\mathcal{Y}$ ($|\mathcal{Y}|$ can be larger than 10$k$). As $|y_i| > 1$ in most cases, UFET can be categorized as a multi-label classification problem. We show statistics of two UFET datasets: {\bf \textsc{UFET}} \cite{ufet} and {\bf \textsc{CFET}}\footnote{As there is no official split available for {\bf \textsc{CFET}}, we split it by ourselves and will release our split in our code.} \cite{cfet} in Table \ref{tab:stat}.

\begin{table}[t]
\scalebox{0.9}{
\centering
\begin{tabular}{ccccc} 
\toprule
dataset & $\vert \mathcal{Y} \vert$      & $\text{avg}(\vert y_i^g \vert)$ & train/dev/test & Lang \\ \midrule
{\bf \textsc{UFET}}   & 10331       & 5.4   & 2k/2k/2k     & EN      \\
{\bf \textsc{CFET}}  & 1299          & 3.5   &   3k/1k/1k & ZH            \\ \bottomrule
\end{tabular}}
\caption{$\text{avg}(\vert y_i^g \vert)$ denotes the average number of gold types per instance, ZH for Chinese.}
\label{tab:stat}
\end{table}

\subsection{Multi-label Classification Model for UFET}
\label{sec:mlc}
Multi-label classification models are widely adopted as backbones for UFET \cite{ufet, onoe-durrett-2019-learning, box4types}. They use an encoder to obtain the mention representation and use a decoder (e.g., MLP) to score types simultaneously. Figure \ref{fig:mlc_ce}(a) shows a representative multi-label classification model adopted by recent methods \cite{npcrf,mlmet}. The contextualized mention representation is obtained by feeding $c_i$ and $m_i$ into the pretrained language models (PLM), and taking the last hidden state of {\tt [CLS]}, $h_{cls}$. The mention representation is then fed into an MLP layer to concurrently obtain all type scores $s_1, \cdots s_N, N=|\mathcal{Y}|$. We call this model MLC and describe its inference and training below. 
\paragraph{MLC Inference} For inference, types with probability higher than a threshold $\tau$ are predicted: $\mathcal{Y}_i = \{y_j | \sigma(s_j) > \tau \}$, $\sigma$ is the sigmoid function. The threshold is tuned on the development set.
\paragraph{MLC Training} Binary Cross-Entropy (BCE) loss between the predicted scores and the gold types are used to train the MLC model: $\mathcal{L}_i = -\frac{1}{N} \sum_{j=1}^N  \alpha \cdot I_{j} \log \sigma(s_j) + (1-I_{j}) \log (1-\sigma(s_j)) $, where $I_j$ is the indicator of $y_j$ being one of the gold types ($y_j \in y_i^g$), and $\alpha$ is a hyper-parameter balancing the loss of positive and negative types. MLC is very efficient in inference. However, the interactions between mention and types in MLC are weak, and the correlations between types are ignored \cite{box4types,xiong-etal-2019-imposing,npcrf}. MLC also has difficulty in integrating type semantics \cite{lite}.

\subsection{Vanilla Cross-Encoders for UFET}
\label{sec:vanilla_ce}
\citet{lite} first proposed to use Cross-Encoder (CE) for UFET. As shown in Figure \ref{fig:mlc_ce}(b), CE concatenates $m_i, c_i, y_j$ together and feeds them into a PLM to obtain the {\tt [CLS]} embedding, then an MLP layer is used to obtain the score of $y_j$ given $m_i, c_i$.
\begin{align}
        \bm{h}_{cls, i} &= \texttt{PLM}(\texttt{[CLS]} \ c_i \ \texttt{[SEP]}\  m_i \ \texttt{[SEP]} \ y_j\ ) \\
        s_j &= \texttt{MLP}(\bm{h}_{cls, i})
\end{align}
The concatenation allows deeper interaction between mention, context, and types (modeled by the multi-head self-attention in PLMs), and also incorporates type semantics.
\paragraph{CE Inference} CE predicts types of a single input $(m_i, c_i)$ by concatenating the input with all possible types $y_j \in \mathcal{Y}$ one by one to predict the scores $s_1, \cdots, s_j$ for each type. Similar to MLC, types that have a higher probability than a threshold are predicted $\mathcal{Y}_i = \{y_j | \sigma(s_j) > \tau \}$. CE requires $N$ forward passes to infer types of a single mention, its inference speed is very slow when $N$ is large.
\paragraph{CE Training} CE is typically trained with marginal ranking loss \cite{lite}. A positive type $y_+ \in y^g_i$ and a negative type $y_{-} \not \in y^g_i$ are sampled from $\mathcal{Y}$ for each data point $(m_i, c_i)$. The loss is computed as:
$$ L_i = \max(\sigma(s_{-}) - \sigma(s_{+}) + \delta, 0) $$ where $s_{+}, s_{-}$ are scores of the sampled positive and negative types, and $\delta$ is the margin tuned on the development set determine how the positive and negative samples should be separated.
\section{Conclusion}
In conclusion, we propose a recall-expand-filter paradigm for ultra-fine entity typing. We train a recall model to generate candidates and use MLM and exact match to improve the quality of recalled candidates, then use filter models to obtain final type predictions. We also propose a filter model called multi-candidate cross-encoder ({\bf \textsc{\name}}) to concurrently encode and filter all candidates and study the influences of different input formats and attention mechanisms. Extensive experiments on entity typing show that our paradigm is effective, and the {\bf \textsc{\name}} models under our paradigm reach SOTA performances on both English and Chinese UFET datasets and are also very effective on fine and coarse-grained entity typing.  {\bf \textsc{\name}} models have comparable inference speed to simple ({\bf \textsc{\name}})  models and are thousands of times faster than previous SOTA cross-encoders.
\begin{abstract}
Ultra-fine entity typing (UFET) predicts extremely free-formed types (e.g., {\it president, politician}) of a given entity mention (e.g., {\it Joe Biden}) in context. State-of-the-art (SOTA) methods use the cross-encoder (CE) based architecture. CE concatenates the mention (and its context) with each type and feeds the pairs into a pretrained language model (PLM) to score their relevance. It brings deeper interaction between mention and types to reach better performance but has to perform $N$ (type set size) forward passes to infer types of a single mention. CE is therefore very slow in inference when the type set is large (e.g., $N=10k$ for UFET). 
% Cross-encoder also ignores the correlation between different types.
To this end, we propose to perform entity typing in a recall-expand-filter manner. The recall and expand stages prune the large type set and generate $K$ ($K$ is typically less than $256$) most relevant type candidates for each mention. At the filter stage, we use a novel model called {\name} to concurrently encode and score these $K$ candidates in only one forward pass to obtain the final type prediction. 
We investigate different variants of \name\  and extensive experiments show that \name\  under our paradigm reaches SOTA performance on ultra-fine entity typing and is thousands of times faster than the cross-encoder. We also found \name\ is very effective in fine-grained (130 types) and coarse-grained (9 types) entity typing. Our code is available at \code.
\end{abstract}
\section{Related Work}
While writing this paper, we noticed that a paper \cite{Du2022LearningTS} that has similar ideas to our work was submitted to the arXiv. They propose a two-stage paradigm for selecting from multiple questions. They also propose a network similar to our {\bf \textsc{\name-B}} to select from multiple options in parallel. We summarize the differences between their work and ours as follows: (1) Different in paradigm. We have an expand stage to further improve the quality of recalled candidates (2) Different in models. {\bf \textsc{\name-S}} and {\bf \textsc{\name-B}} are both different from theirs in both input format and scoring. We additionally propose to discard the C2C attention and study the effect of removing different parts of attention. (3) We focus more on entity typing and conduct extensive experiments covering two languages and three settings (ultra-fine-grained, fine-grained, and coarse-grained). We analyze the effect of using different PLM backbones for a fairer and more comprehensive comparison.
The paradigm of our work is also inspired by works in entity linking and information retrieval. \citet{wu2019zero} uses a retrieval and rerank paradigm for entity linking, they first generate entity candidates using a bi-encoder and rerank them using a vanilla cross-encoder. Our paradigm with an additional expand stage and our proposed {\bf \textsc{\name}} models are also potentially useful for entity linking. We leave it for future work. \citet{halter} represents the query document and candidate documents as vectors and proposed to use a transformer to rerank all candidate documents in parallel for passage retrieval. Compared to them, we tackle entity typing and preserve all information of mention and context rather than represent them as a single vector, the paradigm, model architecture, and training objective are also different.




\section{Limitation}
\section{Introduction}
Ultra-fine entity typing (UFET) \cite{ufet} aims to predict extremely fine-grained types ({\it e.g., president, politician}) of a given entity mention within its context. It provides detailed semantic understandings of entity mention and is a fundamental step in fine-grained named entity recognition \cite{fget}, and can be utilized to assist various downstream tasks such as relation extraction \cite{fewrel}, keyword extraction \cite{huang2020ner} and content recommendation \cite{upadhyay2021explainable}.

\begin{figure}
    \centering
    \scalebox{0.3}{
    \includegraphics{src/img/mlc_ce.pdf}}
    \caption{Cross-Encoder and multi-label classification.}
    \label{fig:mlc_ce}
\end{figure}

\begin{figure*}[t]
    \centering
    \scalebox{0.28}{
    \includegraphics{src/img/paradigm.pdf}}
    \caption{Training and inference of the recall-expand-filter pradigm.}
    \label{fig:paradigm}
\end{figure*}

Most recently, the cross-encoder (CE) based method \cite{lite} achieves the SOTA performance in UFET. Specifically, \citet{lite} proposed to treat the mention with its context as a premise, and each ultra-fine-grained type as a hypothesis. They then concatenate them together as input and feed it into a pretrained language model (PLM) (e.g., RoBERTa \cite{liu2019roberta}) to score the entailment of mention-type pair as illustrated in Figure \ref{fig:mlc_ce}(b). Compared to the traditional multi-label classification method (shown in Figure \ref{fig:mlc_ce}(a)) that simultaneously scores all types using the mention representation, CE incorporates type semantics in the inference process and enables deeper interactions between types and mention to achieve better performance. However, the CE architecture is slow in inference because it has to enumerate all types (up to 10$k$ types) and score entailment of them given the mention as a premise. There is also no direct interaction between types in CE and is therefore unable to model correlations between types (e.g., one has to be a person if he or she is categorized as a politician), which has been proved to be useful in previous works \cite{npcrf, xiong-etal-2019-imposing}.

To this end, we propose a recall-expand-filter Paradigm for UFET (illustrated in Figure \ref{fig:paradigm}) and a novel model called {\bf \textsc{\name}} for faster and more accurate ultra-fine entity typing. As the name suggests, we first train a multi-label classification (MLC) model to efficiently \textbf{recall} top $K$ candidate types which reduce the number of potential types from thousands to hundreds. As the MLC model recalls candidates based on representations learned from the training data, it's hard to recall candidates that are scarce or unseen in the training set. To this end, we apply a multi-way type candidate \textbf{expansion} step utilizing lexical information and weak supervision from masked language models \cite{mlmet} to improve the recall rate of the candidate set. Last but not least, we propose a backbone called multi-candidate cross-encoder ({\bf \textsc{\name}}) to concurrently encode and \textbf{filter} the expanded type candidate set. Different from CE,  ({\bf \textsc{\name}}) concatenates all recalled type candidates to the mention and its context. The concatenated input is then fed into a PLM to obtain candidate representations and candidate scores. The {\bf \textsc{\name}} architecture allows us to infer types simultaneously from the candidate set while preserving the advantages of CE. Concatenating all candidates also enables {\bf \textsc{\name}} implicitly learns the correlation between types. The advantages of {\bf \textsc{\name}} over existing architectures are shown in Figure \ref{fig:adv}. We also comprehensively investigate the performance and efficiency of {\bf \textsc{\name}} with different input formats and attention mechanisms.

\begin{figure}[t]
    \centering
    \scalebox{0.22}{
    \includegraphics{src/img/adv.pdf}}
    \caption{Comparison of different models, M, C, and T are abbreviations of mention, context, and type.}
    \label{fig:adv}
\end{figure}

Experiments on two UFET datasets show that {\bf \textsc{\name}} and its variants under our recall-expand-filter paradigm reach SOTA performance and are thousands of times faster than the CE-based previous SOTA method. We also found {\bf \textsc{\name}} is still effective in fine-grained (130 types) and coarse-grained (9 types) entity typing. Our code is available at \code.

\section{Analysis}
\label{sec:analyze}
\subsection{Importance of Expand Stage}
We perform the ablation study on the importance of the expand stage and show the results in Table \ref{fig:ablation_expand}. We compare the performances of {\bf \textsc{\name-S}} using the expanded or the not expanded candidate sets on {\bf \textsc{UFET}} and {\bf \textsc{CFET}}. We replace the last $48$ candidates recalled by MLC with candidates expanded by MLM and exact matching for {\bf \textsc{UFET}}, and $10$ candidates for {\bf \textsc{CFET}}. Results show that expand stage has a positive effect on performance, it improves the final recall by $+1.0$ and $+2.2$ on  {\bf \textsc{UFET}} and {\bf \textsc{CFET}} without harming the precision.

\begin{table}[t]
\centering
\scalebox{0.75}{
\renewcommand{\arraystretch}{1}
\begin{tabular}{cllll} \toprule
\multicolumn{2}{l}{\bf \textit{Ablation of Expand Stage} }     & \bf \textsc{P}    & \bf \textsc{R}   & \bf \textsc{F1}  \\ \midrule
\multicolumn{5}{l}{\bf \textsc{UFET\ \  MCCE with C2C BERT-large}} \\
\color{blue}\bf \texttt{B} & {\bf \textsc{\name-S$_{128}$ }} (Ours)     & 52.5 & 49.1 & 50.8 \\ 
\color{blue}\bf \texttt{B} & {\bf \textsc{\name-S$_{128}$ w/o Expand }} (Ours)     & 52.7 & 48.1 & 50.3\\ \hline
\multicolumn{5}{l}{\bf \textsc{CFET\ \  MCCE with C2C BERT-base-Chinese}} \\
\color{brown}\bf \texttt{C} & {\bf \textsc{\name-S$_{64}$}} (Ours)  & 55.5 & 62.6 & 58.8 \\ 
\color{brown}\bf \texttt{C} & {\bf \textsc{\name-S$_{64}$ w/o Expand}}   (Ours)   & 55.4 & 60.4 & 57.8 \\ \hline
\midrule
\end{tabular}}
\caption{Ablation study of expand stage.}
\label{fig:ablation_expand}
\end{table}

\subsection{Attentions}
We conduct an ablation study on S2S, C2S, S2C, and C2C attention introduced in Sec. \ref{sec:attn} and show the results in Table \ref{tab:attn}. From the results, we are surprised to find that removing C2C and S2S doesn't have a big negative impact on performance. The {\bf \textsc{\name-S}} using BERT-base reaches $48.8$ F1 even without both C2C and S2S attention. One possible reason is that the interaction between sub-tokens in the sentence can be achieved indirectly by first attending to the candidates and then being attended back by the candidate in the next layer. We also find that the C2S is necessary for the task ($18.7$ F1 without C2S) because we rely on the mention and its context to encode and classify candidates. Furthermore, we found that it is important for sentences to attend to all candidates (S2C), indicating that the interaction between the sentence and different types is crucial for the task.

\begin{table}[t]
\centering
\scalebox{0.75}{
\renewcommand{\arraystretch}{1}
\begin{tabular}{cllll} \toprule
\multicolumn{2}{l}{\bf \textit{Analysis about attention on UFET}}     & \bf \textsc{P}    & \bf \textsc{R}   & \bf \textsc{F1}  \\ \midrule
\multicolumn{5}{l}{\bf \textsc{\name-S using BERT-base}} \\
\color{blue}\bf \texttt{B} & {\bf \textsc{\name-S$_{128}$} FULL}     & 53.2 &  48.3 & 50.6 \\ 
\color{blue}\bf \texttt{B} & {\bf \textsc{\name-S$_{128}$ w/o C2C }}     & 52.3 & 48.3 & 50.2 \\
\color{blue}\bf \texttt{B} & {\bf \textsc{\name-S$_{128}$ w/o S2S }}     & 50.6 & 48.4 & 49.4 \\
\color{blue}\bf \texttt{B} & {\bf \textsc{\name-S$_{128}$ w/o S2C }}     & 48.7 & 47.1 & 47.9 \\ 
\color{blue}\bf \texttt{B} & {\bf \textsc{\name-S$_{128}$ w/o C2S }}     & 19.7 & 17.4 & 18.7\\
\color{blue}\bf \texttt{B} & {\bf \textsc{\name-S$_{128}$ w/o S2S,C2C }}     & 50.2 & 47.3 & 48.8\\
\bottomrule
\end{tabular}}
\caption{Attention analysis.}
\label{tab:attn}
\end{table}








% \subsection{Influence of Candidate Size}



\section{Experiments}
We conduct experiments on two ultra-fine entity typing datasets, {\bf \textsc{UFET}} (English) and {\bf \textsc{CFET}} (Chinese). Their data statistics are shown in Table \ref{tab:stat}. We mainly focus on and report the macro-averaged recall at the recall and expand stage, and concern mainly on the macro-$F1$ of the final prediction at the filter stage. We also evaluate the {\bf \textsc{\name}} models on the fine-grained (130 types) and coarse-grained (9 types) settings of entity typing without the recall and expand stage.
\subsection{UFET and CFET}
\subsubsection{Recall Stage}
\label{sec:recall}
We compare the recall@$K$ on the test sets of {\bf \textsc{UFET}} and {\bf\textsc{CFET}} between the trained MLC model (introduced in \ref{sec:mlc}) and a traditional BM25 model \cite{bm25} in Figure \ref{fig:recall}. The MLC model uses the RoBERTa-large as backbone and is tuned based on the recall@$128$ on the development set. We use AdamW optimizer with a learning rate of $2\times10^{-5}$. Results show that MLC is a strong recall model, it consistently has better recall compared to BM25 on both {\bf\textsc{UFET}} and {\bf\textsc{CFET}} dataset, and the recall@$128$ reaches over $85\%$ on {\bf \textsc{UFET}}, and over $94\%$ on {\bf \textsc{CFET}}.

\begin{figure}[t]
     \centering
     \begin{subfigure}[h]{0.5\textwidth}
         \centering
         \includegraphics[width=\textwidth]{src/img/recall_compare_bm25.pdf}
         \label{fig:mb2}
     \end{subfigure}   
 \caption{Recall@$K$ of MLC and BM25.}
 \label{fig:recall}
\end{figure}

\subsection{Expand Stage}
\label{sec:expand}
In Table \ref{tab:expand}, we evaluate the F1 scores of all candidates expanded by exact match, and top-$10$ candidates expanded by the MLM using Bert-large. We also demonstrate the improvement of recall by using candidate expansion in Figure \ref{fig:expand_improvement}. On {\bf \textsc{UFET}} dataset, expanding around $32$ additional candidates based on $112$ MLC candidates results in $2\%$ higher recall compared to recalling all $128$ candidates by MLC. The recall of $128$ candidates after the expansion is comparable to the recall of $180$ candidates recalled from MLC. Similarly, expanding $10$ candidates is comparable to additionally recalling $80$ candidates using MLC.
In our experiments, we replace the last $48$ candidates recalled by MLC with the candidates recalled by MLM and Exact match for {\bf \textsc{UFET}} and $10$ for {\bf \textsc{CFET}}. We found the expand stage has a positive effect on the final performance of {\bf \textsc{\name}}s, and helps them reach SOTA performance (analyze in Sec. \ref{sec:analyze}).


\begin{table}[t]
\centering
\scalebox{0.75}{
\begin{tabular}{cccccc} 
\toprule
{\bf \textsc{Dataset}} & {\bf \textsc{Expand}} &   {\bf \textsc{P}}  & {\bf \textsc{R}}  &  {\bf \textsc{F1}} & \small{Avg \# Expanded}  \\ \midrule
\multirow{2}{*}{\bf \textsc{UFET}} & {\bf \textsc{Match}}      & 11.2   & 11.3     & 9.8    & 5.23     \\
      & {\bf \textsc{MLM}}  &  8.5     &   17.1   &  10.7  &    10    \\ \midrule
\multirow{2}{*}{\bf \textsc{CFET}} & {\bf \textsc{Match}}   &  11.4  &  14.5  & 11.2   & 4.57    \\
 & {\bf \textsc{MLM}}  & 21.3   &  19.5  & 17.7    & 10    \\ \midrule
\end{tabular}}
\caption{Evaluation of the recalled candidates.}
\label{tab:expand}
\end{table}
\begin{figure}[t]
     \centering
     \begin{subfigure}[h]{0.45\textwidth}
         \centering
         \includegraphics[width=\textwidth]{src/img/recall_ufet.pdf}
         \caption{Recall@$128$ on {\bf \textsc{UFET}} by including different number of expanded candidates. }
         \label{fig:c1}
     \end{subfigure}
     \vfill
     \begin{subfigure}[h]{0.45\textwidth}
         \centering
         \includegraphics[width=\textwidth]{src/img/recall_cfet.pdf}
         \caption{Recall@$64$ on {\bf \textsc{CFET}} by including different number of expanded candidates.}
         \label{fig:c2}
     \end{subfigure}
\caption{Demonstration of the effect of expand stage. $x$-axis represents the number of candidates expanded by MLM/MLM+MATCH among these $128$ candidates. }
\label{fig:expand_improvement}
\end{figure}
\label{sec:exp_expand}
\subsection{Filter Stage and Final Results.}
\begin{table}[h!]
\centering
\scalebox{0.73}{
\renewcommand{\arraystretch}{1}
\begin{tabular}{cllll} \toprule
\multicolumn{2}{l}{\bf \textit{Base Models on UFET} }     & \bf \textsc{P}    & \bf \textsc{R}   & \bf \textsc{F1}  \\ \midrule
\multicolumn{5}{l}{\emph{MLC-like models}}        \\
\color{blue} \bf \texttt{B}& {\bf \textsc{Box4Types}}\cite{box4types}  & 52.8 & 38.8 & 44.8  \\
\color{blue}\bf \texttt{B}& {\bf \textsc{LDET}}$^\dagger$  \cite{onoe-durrett-2019-learning}          & 51.5 & 33.0 & 40.1 \\ 
\color{blue}\bf \texttt{B}& {\bf \textsc{MLMET}}$^\dagger$   {\cite{mlmet}}   & 53.6 & 45.3 & 49.1  \\
\color{blue}\bf \texttt{B}& {\bf \textsc{PL}}  \cite{ding2021prompt}   & 57.8 & 40.7 & 47.7 \\
\color{blue}\bf \texttt{B}& {\bf \textsc{DFET}}    \cite{dfet}      & 55.6 & 44.7 & 49.5 \\
\color{blue}\bf \texttt{B}& {\bf \textsc{MLC}} (reimplemented by us) & 46.5 & 34.9 & 39.9 \\ 
\color{red}\bf \texttt{R}& {\bf \textsc{MLC}} (reimplemented by us) & 42.2 & 44.9 & 43.5 \\ \hline 
\multicolumn{5}{l}{\emph{Seq2seq based models}}      \\
\color{blue}\bf \texttt{B} & {\bf \textsc{LRN} }  {\cite{liu-etal-2021-fine}}              & 54.5 & 38.9 & 45.4  \\\hline
\multicolumn{5}{l}{\emph{Filter models under our recall-expand-filter paradigm}}      \\
\color{blue}\bf \texttt{B} & {\bf \textsc{Vanilla CE}$_{128}$}   & 47.2 & 48.5 & 47.8 \\ 
\color{blue}\bf \texttt{B} & {\bf \textsc{\name-S$_{128}$}} (Ours)  & 53.2 & 48.3 & {\bf 50.6} \\ 
\color{blue}\bf \texttt{B} & {\bf \textsc{\name-S$_{128}$ w/o C2C}}   (Ours)   & 52.3 & 48.3 & 50.2 \\ 
\color{blue}\bf \texttt{B} & {\bf \textsc{\name-B$_{128}$}} (Ours)    & 49.9 & 50.0 & 49.9 \\ 
\color{blue}\bf \texttt{B} & {\bf \textsc{\name-B$_{128}$ w/o C2C}} (Ours)     & 49.9 & 48.2 & 49.0 \\ \hline
\color{red}\bf \texttt{R} & {\bf \textsc{Vanilla CE}$_{128}$}   & 49.6 & 49.0 & 49.3 \\ 
\color{red}\bf \texttt{R} & {\bf \textsc{\name-S$_{128}$}} (Ours)  & 53.3 & 47.3 & 50.1 \\ 
\color{red}\bf \texttt{R} & {\bf \textsc{\name-S$_{128}$ w/o C2C}}   (Ours)  & 53.2 & 46.6 & 49.7 \\ 
\color{red}\bf \texttt{R} & {\bf \textsc{\name-B$_{128}$}} (Ours)  & 52.5 & 47.9 & 50.1 \\ 
\color{red}\bf \texttt{R} & {\bf \textsc{\name-B$_{128}$ w/o C2C}} (Ours)     & 52.7 & 46.4 & 49.3 \\ \hline
\midrule
\multicolumn{2}{l}{\bf \textit{Large Models on UFET} }     & \bf \textsc{P}    & \bf \textsc{R}   & \bf \textsc{F1}  \\ \midrule
\multicolumn{5}{l}{\emph{MLC-like models}}        \\
\color{red}\bf \texttt{R} & {\bf \textsc{MLC}}  \cite{npcrf}               & 47.8 & 40.4 & 43.8  \\
\color{red}\bf \texttt{R} & {\bf \textsc{MLC-NPCRF}} \cite{npcrf}             & 48.7 & 45.5 & 47.0  \\
\color{red}\bf \texttt{R} & {\bf \textsc{MLC-GCN}} \cite{xiong-etal-2019-imposing}     & 51.2 & 41.0 & 45.5 \\
\color{blue}\bf \texttt{B} & {\bf \textsc{PL}}  \cite{ding2021prompt}       & 59.3 & 42.6 & 49.6  \\
\color{blue}\bf \texttt{B} & {\bf \textsc{PL-NPCRF}}  \cite{npcrf}  & 55.3 & 46.7 & {50.6}\\ \hline
\multicolumn{4}{l}{\emph{Cross-encoder based models and {\bf \textsc{\name}}s}}      \\
\color{red}\bf \texttt{R} & {\bf \textsc{LITE+L}}  \cite{lite}             & 48.7 & 45.8 & 47.2  \\
\color{teal}\bf \texttt{RM} & {\bf \textsc{LITE+NLI+L}} \cite{lite} & 52.4 & 48.9 & {50.6} \\ \hline
\multicolumn{4}{l}{\emph{Filter models under our recall-expand-filter paradigm}}   \\ 
\color{blue}\bf \texttt{B} & {\bf \textsc{Vanilla CE$_{128}$}}   & 50.3 & 49.6 & 49.9 \\ 
\color{blue}\bf \texttt{B} & {\bf \textsc{\name-S$_{128}$}}  (Ours)   & 52.5 & 49.1 & 50.8 \\ 
\color{blue}\bf \texttt{B} & {\bf \textsc{\name-S$_{128}$ w/o C2C}}   (Ours)   & 54.1 & 47.1 & 50.4 \\ 
\color{blue}\bf \texttt{B} & {\bf \textsc{\name-B$_{128}$}} (Ours)    & 54.0 & 48.6 & 51.2 \\ 
\color{blue}\bf \texttt{B} & {\bf \textsc{\name-B$_{128}$ w/o C2C}} (Ours)     & 52.8 & 48.3 & 50.4 \\ \hline
\color{red}\bf \texttt{R} & {\bf \textsc{Vanilla CE$_{128}$}}   & 54.5 & 49.3 & 51.8 \\ 
\color{red}\bf \texttt{R} & {\bf \textsc{\name-S$_{128}$}}  (Ours)   & 50.8 & 49.8  &  50.3 \\ 
\color{red}\bf \texttt{R} & {\bf \textsc{\name-S$_{128}$ w/o C2C}}   (Ours)   & 51.5 & 48.8 & 50.1 \\ 
\color{red}\bf \texttt{R} & {\bf \textsc{\name-B$_{128}$}} (Ours)    & 51.9 & 50.8 & 51.4 \\ 
\color{red}\bf \texttt{R} & {\bf \textsc{\name-B$_{128}$ w/o C2C}} (Ours)     & 51.6 & 51.6 & 51.6 \\ \hline
\color{teal}\bf \texttt{RM} & {\bf \textsc{\name-B$_{128}$ w/o C2C}} (Ours) & 56.3 & 48.5 & {\bf 52.1} \\ \hline
\midrule
\end{tabular}}
\caption{Macro-averaged UFET result. {\bf \textsc{LITE+L}} is LITE without NLI pretraining, {\bf \textsc{LITE+L+NLI}} is the full LITE model. Methods marked by $\dagger$ utilize either distantly supervised or augmented data for training. {\bf \textsc{\name-S$_{128}$}} denotes we use $128$ candidates recalled and expanded from the first two stages.}
\label{tab:ufet}
\end{table}
\begin{table}[t]
\centering
\scalebox{0.75}{
\renewcommand{\arraystretch}{1}
\begin{tabular}{cllll} \toprule
\multicolumn{2}{l}{\bf \textit{Models on CFET} }     & \bf \textsc{P}    & \bf \textsc{R}   & \bf \textsc{F1}  \\ \midrule
\multicolumn{5}{l}{\emph{MLC-like models}}        \\
\color{purple}\bf \texttt{N}& {\bf \textsc{MLC}} & 55.8 & 58.6 & 57.1 \\  
\color{purple}\bf \texttt{N}& {\bf \textsc{MLC-NPCRF}} \cite{npcrf}     & 57.0 & 60.5 & 58.7 \\ 
\color{purple}\bf \texttt{N}& {\bf \textsc{MLC-GCN}} \cite{xiong-etal-2019-imposing}   & 51.6 & 63.2 & 56.8 \\ 
\color{brown}\bf \texttt{C}& {\bf \textsc{MLC}} & 54.0 & 59.5 & 56.6 \\  
\color{brown}\bf \texttt{C}& {\bf \textsc{MLC-NPCRF}} \cite{npcrf}   & 54.0 & 61.6 & 57.3 \\  
\color{brown}\bf \texttt{C}& {\bf \textsc{MLC-GCN}} \cite{xiong-etal-2019-imposing} & 56.4 & 58.6 & 57.5 \\ \midrule 
\multicolumn{5}{l}{\emph{Filter models under our recall-expand-filter paradigm}}      \\
\color{purple}\bf \texttt{N} & {\bf \textsc{Vanilla CE}}   & 57.6 & 64.3 & 60.7 \\ 
\color{brown}\bf \texttt{C} & {\bf \textsc{Vanilla CE}}   & 54.0 & 63.3 & 58.3 \\  \hline
\color{purple}\bf \texttt{N} & {\bf \textsc{\name-S$_{64}$}} (Ours)  & 58.4 & 62.1 & 60.2 \\ 
\color{purple}\bf \texttt{N} & {\bf \textsc{\name-S$_{64}$ w/o C2C}}   (Ours)   & 59.1 & 61.5 & 60.3 \\ 
\color{purple}\bf \texttt{N} & {\bf \textsc{\name-B$_{64}$}} (Ours)    & 56.7 & 66.1 & 61.1 \\ 
\color{purple}\bf \texttt{N} & {\bf \textsc{\name-B$_{64}$ w/o C2C}} (Ours)     & 58.8 & 64.1 & 61.4 \\ \hline
\color{brown}\bf \texttt{C} & {\bf \textsc{\name-S$_{64}$}} (Ours)  & 55.5 & 62.6 & 58.8 \\ 
\color{brown}\bf \texttt{C} & {\bf \textsc{\name-S$_{64}$ w/o C2C}}   (Ours)   & 54.0 & 63.4 & 58.3 \\ 
\color{brown}\bf \texttt{C} & {\bf \textsc{\name-B$_{64}$}} (Ours)    & 55.0 & 63.5 & 59.0 \\ 
\color{brown}\bf \texttt{C} & {\bf \textsc{\name-B$_{64}$ w/o C2C}} (Ours)     & 57.3 & 61.3 & 59.3 \\ \hline
\midrule
\end{tabular}}
\caption{Macro-averaged CFET result.}
\label{tab:cfet}
\end{table}

In this section, we report the performance of {\bf \textsc{MCCE}} variants as the filter models and compare them with various strong baselines that we will introduce later. We also compare the inference speed of different models in this section. For filter models, we treat the number of candidates $K$ recalled and expanded by the first two stages as hyper-parameters, and tune it on the development set. We found the choice of PLM backbones has a non-negligible effect on the performance, and the PLM backbone of previous methods varies. Therefore for fairer comparisons to baselines, we conduct experiments of {\bf \textsc{\name}} using different backbone PLMs for our {\bf \textsc{\name}} models and report the results. For all {\bf \textsc{\name}} models, we use AdamW optimizer with a learning rate tuned between $5\times 10^{-6}$ and $2\times 10^{-5}$. The batch size we use is $4$ and we train the models for at most $50$ epochs with early stopping. {\bf \textsc{UFET}} also provides a large dataset obtained from distant supervision such as entity linking, we do not use it and only train and evaluate our models on human-labeled data.
\paragraph{Baselines}
The {\bf \textsc{MLC}} model we used for the recall stage and the cross-encoder ({\bf \textsc{CE}}) we introduced in Sec. \ref{sec:vanilla_ce} are natural baselines. We also compare our methods with recent PLM-based methods. {\bf \textsc{LDET} }\cite{onoe-durrett-2019-learning} is an MLC with Bert-base-uncased and ELMo \cite{elmo} trained on 727k examples automatically denoised from the distantly labeled UFET. {\bf \textsc{GCN} }\cite{xiong-etal-2019-imposing} uses GCN to model type correlations and obtain type embeddings. Types are scored by dot-product of mention and type embeddings. The original paper uses BiLSTM as the mention encoder and we use the results re-implemented by \citet{npcrf} using RoBERTa-large. {\bf \textsc{Box4Type} }\cite{box4types} uses Bert-large as the backbone and uses box embedding to encode mentions and types for training and inference. {\bf \textsc{LRN} }\cite{liu-etal-2021-fine} use Bert-base as the encoder and an LSTM decoder to generate types in a seq2seq manner. {\bf \textsc{MLMET} }\cite{mlmet} is a {\bf \textsc{MLC}} with Bert-base, but first pretrained by the distantly-labeled data augmented by masked word prediction, then finetuned and self-trained on the 2k human-annotated data. {\bf \textsc{PL}} \cite{ding2021prompt} uses prompt learning for entity typing. {\bf \textsc{DFET} }\cite{dfet} uses {\bf \textsc{PL}} as backbone and is a multi-round automatic denoising method for 2k labeled data. {\bf \textsc{LITE} }\cite{lite} is the previous SOTA system that formulates entity typing as textual inference. {\bf \textsc{LITE}} uses RoBERTa-large-MNLI as the backbone, and is a cross-encoder (introduced in Sec. \ref{sec:vanilla_ce}) with designed templates and a hierarchical loss. \citet{npcrf} proposes {\bf \textsc{NPCRF}} to enhance backbones such as {\bf \textsc{PL}} and {\bf \textsc{MLC}} by modeling type correlations, and reach performance comparable to {\bf \textsc{LITE}}.

\paragraph{Naming Conventions}
Let {\bf \textsc{\name-S}} be the {\bf \textsc{\name}} model that treats candidates as sub-tokens, and {\bf \textsc{\name-B}} be the model representing candidates as fixed-size blocks. The {\bf \textsc{\name}} model without {\bf \textsc{C2C}} attention (mentioned in Sec. \ref{sec:attn}) is denoted as {\bf \textsc{\name-B} w/o C2C}. For PLM backbones used in {\bf \textsc{UFET}}, we use {\color{blue} \bf \texttt{B}}, {\color{red} \bf \texttt{R}}, {\color{teal} \bf \texttt{RM}} to denote BERT-base-cased \cite{bert}, RoBERTa \cite{liu2019roberta}, and RoBERTa-MNLI \cite{liu2019roberta} respectively. For {\bf \textsc{CFET}}, we adopt two widely-used Chinese PLM, BERT-base-Chinese and NeZha-base-Chinese, and denote them as {\color{brown} \bf \texttt{C}} and {\color{purple} \bf \texttt{N}} respectively. 

\paragraph{UFET Results} We show the results of {\bf \textsc{UFET}} dataset in Table \ref{tab:ufet}. The results show that: (1) The recall-expand-filter paradigm is effective. Filter models outperform all baselines without the paradigm by a large margin. The vanilla CE under our paradigm reaches $51.8$ F1 compared to more complexed CE {\bf \textsc{LITE}} with $50.6$ F1 (2) {\bf \textsc{\name}} models reach SOTA performances. {\bf \textsc{\name-S$_{128}$}} with BERT-base performs best and reaches {\bf 50.6} F1 score, which is comparable to previous SOTA performance of large models such as {\bf \textsc{LITE+NLI+L}} and {\bf \textsc{PL+NPCRF}}. Among large models, {\bf \textsc{\name-B$_{128}$ w/o C2C}} also reaches SOTA performance with {\bf 52.1} F1 score. (3) {\bf \textsc{C2C}} attention is not necessary on large models, but is useful in base models. (4) Large models can utilize type semantics better. We found {\bf \textsc{\name-B}} outperforms {\bf \textsc{\name-S}} on large models, but underperforms {\bf \textsc{\name-S}} on base models. (5) Backbone PLM matters. We found the performance of {\bf \textsc{Vannila CE}} under our paradigm is largely affected by the PLM it used. It reaches $47.8$ F1 with BERT-base and $51.8$ F1 with RoBERTa-large. For {\bf \textsc{\name}} models, we found {\bf \textsc{\name}} performs better than {\bf \textsc{\name-B}} with BERT, and worse than {\bf \textsc{\name-B}} with RoBERTa. 

\begin{table*}[t]
\centering
\scalebox{0.9}{
\begin{tabular}{lllcc} \toprule
\bf \textsc{Model}  & \bf \textsc{\# FP} & \bf \textsc{Attn} & \bf \textsc{sents/sec} & \bf \textsc{F1} \\ \midrule
{\bf \textsc{MLC}} & \small{$1$}  & \small{$L_S^2D$} & 58.8 & 43.8\\
{\bf \textsc{LITE+NLI+L (CE)}}  & \small{$N$}  & \small{$L_S^2D$} & 0.02 & 50.6\\ \midrule \hline
\multicolumn{5}{l}{\emph{filter stage inference speed.}}  \\
{\bf \textsc{Vanilla CE$_{128}$}}  & \small{$128$}  & \small{$L_S^2D$} & 1.64 & 51.8 \\ 
{\bf \textsc{\name-S$_{128}$}}  & \small{$1$}  & \small{$(L_S+128)^2D$} & 60.8 & 50.1 \\ 
{\bf \textsc{\name-B$_{128}$}}  & \small{$1$}  & \small{$(L_S+128B)^2D$} & 22.3 & 51.4\\ 
{\bf \textsc{\name-B$_{128}$ w/o C2C}}  & \small{$1$}  & \small{$(L_S^2+256L_S B + 128 B^2)D$} & 25.2 & {\bf 52.1}\\ \bottomrule
\end{tabular}}
\caption{Inference speed comparison of models. {\bf \textsc{\# FP}} means the number of PLM forward passes required by a single inference. {\bf \textsc{ATTN}} column lists the theoretical attention complexity.  We also report the practical inference speed {\bf \textsc{sents/sec}} and the {\bf \textsc{F1}} scores on {\bf \textsc{UFET}} with RoBERTa-large architecture.}
\label{tab:speed}
\end{table*}

\begin{table}[t]
\centering
\scalebox{0.85}{
\renewcommand{\arraystretch}{1}
\begin{tabular}{cllll} \toprule
\multicolumn{2}{l}{\bf \textit{Models} }     & \bf \textsc{P}    & \bf \textsc{R}   & \bf \textsc{F1}  \\ \midrule
\multicolumn{5}{l}{\emph{coarse (9 types) Open Entity}}        \\ \hline
\color{red}\bf \texttt{R} & {\bf \textsc{MLC}}   & 76.8 & 78.5 & 77.6 \\ 
\color{red}\bf \texttt{R} & {\bf \textsc{Vanilla CE$_{9}$}}   & 82.3 & 81.0 & 81.6 \\ 
\color{red}\bf \texttt{R} & {\bf \textsc{\name-S$_{9}$}}   & 77.0 & 87.7 & 82.0 \\ 
\color{red}\bf \texttt{R} & {\bf \textsc{\name-B$_{9}$ w/o C2C}}   & 77.2 & 85.4 & 81.1 \\ \hline
\multicolumn{5}{l}{\emph{fine (130 types)}}        \\ \hline
\color{red}\bf \texttt{R} & {\bf \textsc{MLC}}   & 70.4 & 63.7 & 66.9  \\ 
\color{red}\bf \texttt{R} & {\bf \textsc{Vanilla CE}$_{130}$}   & 67.9 & 66.4 & 67.1 \\ 
\color{red}\bf \texttt{R} & {\bf \textsc{\name-S$_{130}$}}   & 65.8 & 71.8 & 68.7 \\ 
\color{red}\bf \texttt{R} & {\bf \textsc{\name-B$_{130}$ w/o C2C}}   & 64.1 & 70.5 & 67.1 \\ \hline
\midrule
\end{tabular}}
\caption{Micro-averaged results on UFET fine and coarse.}
\label{tab:ufet-coarse-fine}
\end{table}

\paragraph{CFET Results} We conduct experiments on {\bf \textsc{CFET}} and compare {\bf \textsc{\name}} models with several strong baselines:  {\bf \textsc{NPCRF}} and {\bf \textsc{GCN}} with MLC-like architecture, and {\bf \textsc{Vanilla CE}} under out paradigm which is proved to be better than {\bf \textsc{LITE}} on {\bf \textsc{UFET}}. The results are shown in Table \ref{tab:cfet}. Similar to results in {\bf \textsc{UFET}}, filter models under our paradigm significantly outperform MLC-like baselines, $+2.0$ F1 for Nezha-base and $+1.8$ F1 for BERT-base-Chinese. In {\bf \textsc{CFET}}, {\bf \textsc{\name}-B} is significantly better than {\bf \textsc{\name}-S}, on both Nezha-base and BERT-base-Chinese, indicating the importance of type semantics in Chinese language. We also find that {\bf \textsc{\name} w/o C2C} is generally better than  {\bf \textsc{\name} w/ C2C}, it is possibly because the C2C attention distracts the candidates from attending to mention and contexts.
\paragraph{Speed Comparison} Table \ref{tab:speed} shows the theoretical inference complexity (number of PLM forward passes, and attention complexity), and practical inference speed (number of sentences inferred per second) of different models. We conduct the speed test using NVIDIA TITAN RTX for all models, and the inference batch size is 4.
At the filter stage, the inference speed of {\bf \textsc{\name-S}} is on par with {\bf \textsc{MLC}} (even slightly faster because we don't need to score all types), and is about 40 times faster than {\bf \textsc{Vannila CE}} and thousands of times faster than {\bf \textsc{LITE}}. {\bf \textsc{\name-B w/o C2C}} is not significantly faster than {\bf \textsc{\name-B}} as expected. It's possibly because the computation related to the block attention is not fully optimized by existing deep learning frameworks. The speed advantage of {\bf \textsc{\name-B w/o C2C}} over {\bf \textsc{\name-B}} will be greater with more candidates.


\subsection{Fine-grained and Coarse-grained Entity Typing}
We also conduct experiments on Fine-grained (130-class) and Coarse-grained (9-class, also known as ``Open Entity'') entity typing, and the results are shown in Table \ref{tab:ufet-coarse-fine}. As the type candidate set is much smaller in these settings, we skip the recall and expand stages and directly run the filter models and compare them to baselines. Results show that both {\bf \textsc{\name}-S} and {\bf \textsc{\name}-B} are still better than {\bf \textsc{MLC}} and {\bf \textsc{Vanilla CE}}, and {\bf \textsc{\name}-S} is better than {\bf \textsc{\name}-B} on coarser-grained cases possibly because the coarser-grained types are simpler in surface-forms and {\bf \textsc{\name}-S} will not lose many type semantics.





