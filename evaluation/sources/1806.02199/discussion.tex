\FloatBarrier

\section{Conclusion}

The SOM-VAE can recover topologically interpretable state representations on time series and static data.
It provides an improvement to standard methods in terms of clustering performance and offers a way to learn discrete two-dimensional representations of the data manifold in concurrence with the reconstruction task.
It introduces a new way of overcoming the non-differentiability of the discrete representation assignment and contains a gradient-based variant of the traditional self-organizing map that is more performant than the original one.
%The probabilistic component of our model can be learned end-to-end and concurrently with the rest of the architecture, while offering predictive performance close to the maximum likelihood solution and further improving the clustering on noisy data.
On a challenging real world medical data set, our model learns more informative representations with respect to medically relevant prediction targets than competitor methods.
%The probabilistic model component improves the clustering in this setting.
The learned representations can be visualized in an interpretable way and could be helpful for clinicians to understand patients' health states and trajectories more intuitively.

It will be interesting to see in future work whether the probabilistic component can be extended to not just improve the clustering and interpretability of the whole model, but also enable us to make predictions.
Promising avenues in that direction could be to increase the complexity by applying a higher order Markov Model, a Hidden Markov Model or a Gaussian Process.
Another fruitful avenue of research could be to find more theoretically principled ways to overcome the non-differentiability and compare them with the empirically motivated ones.
Lastly, one could explore deviating from the original SOM idea of fixing a latent space structure, such as a 2D grid, and learn the neighborhood structure as a graph directly from data.


\subsection*{Acknowledgments}

FL is supported by the Max Planck/ETH Center for Learning Systems.
MH is supported by the Grant No. 205321\_176005 ``Novel Machine Learning Approaches for Data from the Intensive Care
Unit'' of the Swiss National Science Foundation (to GR).
VF, FL, MH and HS are partially supported by ETH core funding (to GR).
We thank Natalia Marciniak for her administrative efforts; Marc Zimmermann for technical support; Gideon Dresdner, Stephanie Hyland, Viktor Gal, Maja Rudolph and Claire Vernade for helpful discussions; and Ron Swanson for his inspirational attitude.