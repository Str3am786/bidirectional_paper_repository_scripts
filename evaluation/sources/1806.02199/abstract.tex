\begin{abstract}


High-dimensional time series are common in many domains.
Since human cognition is not optimized to work well in high-dimensional spaces, these areas could benefit from interpretable low-dimensional representations.
However, most representation learning algorithms for time series data are difficult to interpret.
This is due to non-intuitive mappings from data features to salient properties of the representation and non-smoothness over time.

To address this problem, we propose a new representation learning framework building on ideas from interpretable discrete dimensionality reduction and deep generative modeling.
This framework allows us to learn discrete representations of time series, which give rise to smooth and interpretable embeddings with superior clustering performance.
We introduce a new way to overcome the non-differentiability in discrete representation learning and present a gradient-based version of the traditional self-organizing map algorithm that is more performant than the original.
Furthermore, to allow for a probabilistic interpretation of our method, we integrate a Markov model in the representation space.
This model uncovers the temporal transition structure, improves clustering performance even further and provides additional explanatory insights as well as a natural representation of uncertainty.

We evaluate our model in terms of clustering performance and interpretability on static (Fashion-)MNIST data, a time series of linearly interpolated (Fashion-)MNIST images, a chaotic Lorenz attractor system with two macro states, as well as on a challenging real world medical time series application on the eICU data set.
Our learned representations compare favorably with competitor methods and facilitate downstream tasks on the real world data.

\end{abstract}
