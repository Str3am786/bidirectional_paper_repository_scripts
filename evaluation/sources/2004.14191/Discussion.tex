
%-------------------------------------------------------------------------------
\section{Discussion}
\label{sec:discussion}
%-------------------------------------------------------------------------------
In this section, we discuss potential issues and limitations of {\bcov}.

\textbf{RISC ISA}.
Inserting detours is generally easy in RISC ISAs thanks to their fixed instruction size.
However, the addressing range can be significantly lower than the $\pm$2\,GB offered by x86-64.
Note that we patch each ELF module individually.
This means that we only need an addressing range that is large enough to reach our patch code segment from the original code.
For example, a range of 60MB would be sufficient for our largest subject.
AArch64 offers a detour range of $\pm$128\,MB  which can accommodate a large majority of binaries.
AArch32 offers just $\sim$32\,MB, in comparison.
In such a case, a single detour instruction might not be sufficient.
Additional options need to be investigated such as function relocation, literal pools, and modifications to linker scripts.

In addition, we update coverage data using a single pc-relative \texttt{mov} which has a memory operand with a 32-bit offset.
Generally, emulating the same functionality in RISC ISAs require more instructions and clobbering of registers.
However, saving and restoring the clobbered registers is not always necessary.
A liveness analysis can help us acquire registers with dead values. 
Similar analyses are already implemented in DBI tools.

%-------------------------------------------------
% this paragraph is omitted in this short version
%------------------------------------------------- 
%\textbf{Conditional instructions}. 
%Simple \texttt{if} statements can  becompiled into conditional instructions. 
%For example, the following statement can be mapped to one \texttt{cmov} in x86-64:
%\begin{lstlisting}[style=custom-cpp]
%if (a < b) c = 0 else c = 1;
%\end{lstlisting}
%Binary-level coverage analysis tools like {\bcov} can report whether a particular \texttt{cmov} is covered.
%However, this does not imply that it actually took effect.
%In fact, we focus on code rather than \textit{data-flow} coverage.
%However, {\bcov} can be extended to handle such cases.
%Basically, the effect of a  \texttt{cmov<condition>} can be checked by inserting a detour targeting a \textit{guarded} coverage update instruction.
%The guard is simply a short jump \texttt{j<negated condition>}.
%The detour can be flexibly inserted around a \texttt{cmov} as long as CPU flags are not modified.
%However, this scheme might not to be practical in AArch32 where the majority of instructions can be conditionally executed.

\textbf{Limitations and threats to validity}.
The precision of the recovered CFG can affect the coverage reported by our tool.
While the implemented jump table and non-return analyses significantly increase CFG precision, 
they are still not perfect.
Our prototype might miss jump tables, albeit only in a few situations.
Also, while our experiments show that the non-return analysis in {\bcov} is comparable to IDA Pro, both tools face the challenge of \textit{may-return} functions.
Such functions might not return to their caller depending on their arguments.
Function \texttt{Perl\_\_force\_out\_malformed\_utf8\_message} in \texttt{perl} is particularly noteworthy.
In one binary, it is called 88 times (out of 89 total) with the argument \texttt{die\_here} set, i.e, will not return.
Developers can signal to the compiler that a particular call will not return using  \texttt{\_\_builtin\_unreachable()}.
Such hints are not available in the binary so we simply assume that all calls to may-return functions are returning.
Consequently, {\bcov} might spuriously report BBs following a may-return call as covered.


On another note, we believe that our subjects are representative of C/C++ software in Linux.
However, generalizing our results to other native languages and operating systems requires further investigation.
The simple mechanisms we use to implement detours and update coverage are also applicable to system software like kernel modules. 
However, special considerations might exist in such settings.
Finally, our approach cannot be directly applied to dynamic code, e.g., self-modifying code.


%%% Local Variables:
%%% mode: latex
%%% TeX-master: "binder"
%%% End:
