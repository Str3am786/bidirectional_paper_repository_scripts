
%-------------------------------------------------------------------------------
\section{Static Instrumentation}
\label{sec:instrumentation}
%-------------------------------------------------------------------------------
In this section, we first consider a strategy to reduce instrumentation overhead by carefully selecting a basic block to probe in a superblock.
Then, we discuss handling short basic blocks by means of hosting their detours in larger neighboring basic blocks.

\subsection{Optimized Probe Selection}
\label{sec:optimized-probe}


Generally, probing a BB requires inserting a detour targeting its designated trampoline.
A detour occupies 5 bytes and can either be a direct \texttt{jmp} or \texttt{call}.
Consequently, one or more original instructions must be relocated to the trampoline.
This \textit{relocation} overhead varies due to the instruction-size variability in x86-64.
Note that a pc-relative \texttt{mov}, which occupies 7 bytes, represents an unavoidable
overhead for updating coverage data in each trampoline.
Hence, our goal is to reduce the relocation overhead.

To this end, we iterate over all BBs in a superblock and select the one expected to incur the lowest overhead.
First, we have to establish whether a detour can be accommodated in the first place.
A BB that satisfies $s + p < 5 $ is considered a \textsf{guest}, where $s$ and $p$ are the byte size and padding size respectively.
A superblock that contains only \textsf{guest} BBs is handled via detour hosting (\S\ref{sec:detour-hosting}).
Now we examine the type and size of the last instruction of each BB and whether the BB is targeted by a jump table.
These parameters are translated to the types depicted in Table~\ref{tab:probe-overhead}.
These BB types are organized in a total order.
This means, for example, we strictly prefer a \textsf{long-call} over a \textsf{long-cond} should both exist in the same superblock.
This type order is primarily derived from empirical observation.
However, we did not necessarily experiment with all possible combinations.
Preferring \textsf{long-call} over \textsf{short-call} should be intuitive.
The latter incurs an additional overhead for relocating at least one instruction preceding the \texttt{call}.

\begin{table}[t!]
    \centering
    \renewcommand{\arraystretch}{1.2}
    \setlength\tabcolsep{4pt}
    \small
    %    \renewcommand{\arraystretch}{1}
    \caption{BB classification used in probe selection. Types are shown in ascending order based on expected relocation overhead. 
    The terms \textit{long} and \textit{short }are relative to detour size (5 bytes).
    Short types require relocating preceeding (RP) instruction(s).
    } 
    \label{tab:probe-overhead}    
    \begin{tabularx}{\columnwidth}{@{}lll@{}}
        \\
        \textbf{Type} &
        \textbf{RP} &
        \textbf{Relocation overhead} \\
        
        \toprule
        
        \textsf{return} & maybe & Can be only 1 byte depending on the padding \\
        \textsf{long-jump} & no & Size of \texttt{jmp} instruction which is $ \ge $ 5 bytes \\
        \textsf{long-call} & no & Size of \texttt{call} instruction which is $ \ge $ 5 bytes  \\
        \textsf{jump-tab} & no & Size of \texttt{jmp} instruction to original code (5 bytes)  \\
        \textsf{short-call} & yes & Similar to \textsf{long-call} but with RP overhead added \\
        \textsf{short-jump} & yes &Similar to \textsf{long-jump} but with RP overhead added\\
        \textsf{internal} & maybe & Size of relocated instruction(s) inside the BB \\
        \textsf{long-cond} & no & Rewriting incurs a fixed 11 byte overhead\\
        \textsf{short-cond} & yes & Similar to \textsf{long-cond} but with RP overhead added\\
        \bottomrule
    \end{tabularx}
\end{table}


%
% Writes content to temp file
\begin{filecontents*}{\jobname.return.1.aux}
411957: pop  r14
411959: pop  r15
41195b: jmp  2016b59
\end{filecontents*}

\begin{filecontents*}{\jobname.return.2.aux}
2016b59: mov BYTE PTR [rip+0x3a1a63],1
2016b60: pop rbp
2016b61: ret 
\end{filecontents*}

% original  13165fd:       call   QWORD PTR [rip+0xcfa805]        # 2010e08 <ff_rgb24toyv12>
\begin{filecontents*}{\jobname.long-call.1.aux}
13165fd: nop
13165fe: call 23a1257
\end{filecontents*}

\begin{filecontents*}{\jobname.long-call.2.aux}
23a1257: mov BYTE PTR [rip+0x4e5e2],1
23a125e: jmp QWORD PTR [rip-0x39045c]
\end{filecontents*}

\begin{filecontents*}{\jobname.short-call.1.aux}
411f84: mov  rsi,r15
411f87: call 2016f08 
\end{filecontents*}

\begin{filecontents*}{\jobname.short-call.2.aux}
2016f08: mov BYTE PTR [rip+0x3a16f0],1
2016f0f: mov rdx, rbp
2016f12: jmp rax
\end{filecontents*}

\begin{filecontents*}{\jobname.long-jump.1.aux}
408449: jmp  2012d66
40844e: nop
40844f: nop
\end{filecontents*}

\begin{filecontents*}{\jobname.long-jump.2.aux}
2012d66: mov BYTE PTR [rip+0x3a547c],1
2012d6d: jmp QWORD PTR [rcx*8+0x136a0f0]
\end{filecontents*}

\begin{filecontents*}{\jobname.jump-tab.1.aux}
408579: cmp al,0x73
40857b: jne 4085f0 
\end{filecontents*}

\begin{filecontents*}{\jobname.jump-tab.2.aux}
2012c33: mov BYTE PTR [rip+0x3a559b],1
2012c3a: jmp 0x408579    
\end{filecontents*}

\begin{filecontents*}{\jobname.short-jump.1.aux}
42550e: movsxd rdi,edi
425511: jmp    201ee2a 
\end{filecontents*}

\begin{filecontents*}{\jobname.short-jump.2.aux}
201ee2a: mov BYTE PTR [rip+0x399fed],1
201ee31: add r13, rdi
201ee34: jmp 0x425569
\end{filecontents*}

\begin{filecontents*}{\jobname.internal.1.aux}
407a41: jmp 2012648 
407a46: nop
407a47: inc rbp
\end{filecontents*}

\begin{filecontents*}{\jobname.internal.2.aux}
2012648: mov BYTE PTR [rip+0x3a5b23],1
201264f: mov eax,DWORD PTR[rip-0x7f4bd5]
2012655: jmp 0x407a47  
\end{filecontents*}


\begin{filecontents*}{\jobname.long-cond.1.aux}
4058db: test rsi,rsi
4058de: jmp  20113e7 
4058e3: nop    
\end{filecontents*}

\begin{filecontents*}{\jobname.long-cond.2.aux}
20113e7: mov BYTE PTR [rip+0x39cc6b],1
20113ee: jne 0x405992
20113f4: jmp 0x4058e4
\end{filecontents*}

\begin{filecontents*}{\jobname.short-cond.1.aux}
405f14: add rax,rcx
405f17: nop
405f18: jmp 2011895
\end{filecontents*}

\begin{filecontents*}{\jobname.short-cond.2.aux}
2011895: mov BYTE PTR [rip+0x39c802],1
201189c: mov QWORD PTR [r13+8],rax
20118a0: jns 0x405f7c
20118a6: jmp 0x405f1d
\end{filecontents*}


\begin{table*}[t!]
    \centering
    \small
    \setlength\tabcolsep{1.2pt}
    \renewcommand{\arraystretch}{1}   
    \begin{tabularx}{\textwidth}[t]{@{}lp{3.9cm}p{7.1cm}p{5.1cm}@{}}        
        \textbf{Type} & 
        \textbf{Patched code} &
        \textbf{Trampoline} &
        \textbf{Notes} \\ 
        \toprule
        \textsf{return}   & 
        \asmcell{\jobname.return.1.aux} & 
        \asmcell{\jobname.return.2.aux} & 
        Three padding bytes are overwritten after the original \texttt{ret}.
        Relocating two instructions to \texttt{0x2016b60} costs only 2 bytes.\\ \midrule
        
        \textsf{long-jump}   & 
        \asmcell{\jobname.long-jump.1.aux} & 
        \asmcell{\jobname.long-jump.2.aux} & 
        Original \texttt{jmp} at \texttt{0x408449} is relocated to \texttt{0x2012d6d} which costs 7 bytes.
        Remaining space in patched code is padded. \\ \midrule
        
        \textsf{long-call}   & 
        \asmcell{\jobname.long-call.1.aux} & 
        \asmcell{\jobname.long-call.2.aux} & 
        Original \texttt{call} at \texttt{0x13165fd} is replaced with a \texttt{call} detour and a matching \texttt{jmp} at \texttt{0x23a125e} where \texttt{rip} offset is fixed.\\ \midrule
        
        \textsf{jump-tab}   & 
        \asmcell{\jobname.jump-tab.1.aux} & 
        \asmcell{\jobname.jump-tab.2.aux} & 
        Original code is unmodified.
        Jump table data is patched to target the trampoline. \\ \midrule
        
        \textsf{short-call} &
        \asmcell{\jobname.short-call.1.aux} & 
        \asmcell{\jobname.short-call.2.aux} & 
        Instruction preceding the original \texttt{call} is relocated to \texttt{0x2016f0f}. 
        \texttt{call} is rewritten to a matching \texttt{jmp} at \texttt{0x2016f12}. \\ \midrule
        
        
        \textsf{short-jump}   & 
        \asmcell{\jobname.short-jump.1.aux} & 
        \asmcell{\jobname.short-jump.2.aux} & 
        To free space for a detour, the preceding instruction is relocated to \texttt{0x201ee31}. 
        That is followed by a matching long \texttt{jmp}.  \\ \midrule
        
        
        \textsf{internal}   & 
        \asmcell{\jobname.internal.1.aux} & 
        \asmcell{\jobname.internal.2.aux} & 
        Inner instruction relocated to \texttt{0x201264f}, where \texttt{rip} offset is fixed,
        followed by \texttt{jmp} to next instruction.
        Total cost is 11 bytes.\\ \midrule
        
        \textsf{long-cond}   & 
        \asmcell{\jobname.long-cond.1.aux} & 
        \asmcell{\jobname.long-cond.2.aux} & 
        Original \texttt{jne} (6 bytes) is rewritten to an equivalent instruction sequence at \texttt{0x20113ee} which has a fixed relocation cost of 11 bytes.\\
        
        \textsf{short-cond}   & 
        \asmcell{\jobname.short-cond.1.aux} & 
        \asmcell{\jobname.short-cond.2.aux} & 
        Similar to \textsf{long-cond} except that an extra instruction is relocated to \texttt{0x201189c} in order to free enough space for a detour.\\
        
        \bottomrule
    \end{tabularx}
    
    \caption{Examples of patching different basic block types. Types are shown in ascending order based on expected relocation overhead. Examples taken from a patched \textsf{ffmpeg} v4.1.3 binary compiled with \textsf{clang} v5.0. For each \textit{long} type there is a corresponding \textit{short}, e.g. short-call, where the size of exit instruction is < 5 bytes.
    Both are similarly patched. 
    However, one or more instructions preceding the exit instruction must be relocated in the short BB types.}
    \label{tab:probe-types}
\end{table*}




We observed that \textsf{return} basic blocks are usually padded (55\% on average).
Padding size is often more than 3 bytes which translates to a relocation overhead of only one byte - the size of a  \texttt{ret} instruction.
Also, favoring \textsf{long-jmp} over \textsf{long-call} provided around 3\% improvement in both relocation and performance overheads.
On the other hand, \textsf{short-call} had only a slight advantage over \textsf{short-jmp}.
This might be due to the fixed 2-byte size of the latter, which leads to relocating more instructions.
However, our experiments were not always conclusive, e.g., between \textsf{jump-tab} and \textsf{short-call}.

Relocating an instruction depends on its relation to the PC (called \textsf{rip} in x86-64).
Position-independent instructions can simply be copied to the trampoline.
However, we had to develop a custom rewriter for position-dependent instructions.
The rewriter preserves the exact semantics of the original instruction whether it explicitly or implicitly depends on \texttt{rip}. 
For example, a \textsf{long-cond} instruction will be rewritten in the trampoline to a matching sequence consisting of a \textsf{long-cond} (6 bytes) and a \textsf{jmp} (5 bytes).

Jump table instrumentation has the unique property of preserving the original code.
It is a data-only mechanism that enables us to probe even one-byte BBs.
However, for it to be applicable, a BB has to be targeted by a \textit{patchable} jump table.
A jump table is patchable if its entries are either 32-bit offsets or absolute addresses.
We observed that about 92\% of more than 46,000 jump tables in our dataset are patchable.
In fact, we found that 8-bit and 16-bit offsets are only used in \texttt{libopencv\_core}.


Finally, our probe selection strategy is effective in reducing relocation overhead.
However, it is not necessarily optimal.
We observed high variance in the padding of BBs of type \textsf{return}, i.e., \textsf{return} is not always the best choice.
Also, instrumenting a loop head can unnecessarily trigger multiple coverage updates. A loop-aware strategy might reduce performance overhead by choosing a BB outside the loop as an alternative. Such optimizations are left for future work.

\subsection{Detour Hosting}
\label{sec:detour-hosting}

The instruction-size variability in x86-64 suggests that some BBs are simply too short to safely insert a detour without overwriting other BBs.
In our dataset, we found that about 7\% of all BBs are short (size < 5 bytes).
Left without a probe, we risk losing coverage information of a particular short BB and, potentially, all of its dominators.
One possible solution is to relocate the entire function to a larger memory area.
However, this is costly in terms of code size and the engineering effort required to fix relocated code references.
For example, throwing an exception from a relocated function without fixing its corresponding CFI record might lead to abrupt process termination.


The method adopted in {\bcov} is \textit{detour hosting}.
It significantly reduces the relocation overhead while preserving the stability of code references at basic-block level.
Here, the size of a guest BB needs to be at least 2 bytes, which is enough to insert a short detour targeting a reachable host BB, i.e., within about $\pm$128 bytes.
The host BB must be large enough to accommodate two regular detours, i.e., at least 10 bytes.
The first detour targets the host trampoline while the other detours would target the trampolines of their respective guests.
Note that we can safely overwrite padding bytes of both the guest and host.
Also, the host does not need to be entirely relocated.
Relocating a subset of its instructions might be sufficient.


Figure~\ref{fig:detour-hosting} depicts a detour hosting example.
It involves a guest BB consisting of a single indirect \textsf{call} (3 bytes).
The tricky part about a \textsf{call} is that its return address must be preserved.
A \texttt{sub} instruction (5 bytes) is used to adjust the return address in the trampoline from \texttt{0xad67fd} to its original value of \texttt{0xad6803}.
This adjustment also clobbers the CPU flags which is safe since they are not preserved across function calls in the x86-64 ABI.
Note that this is the only case where we modify the CPU state.

Now we have the following allocation problem: given a guest $g$ and a set of suitable hosts $H=\{h_1,h_2,..,h_n\}$, find the host $h_i$ whose selection incurs minimal overhead.
Also, we are interested in the more general formulation: given a set of guests $G=\{g_1,g_2,..,g_k\}$ and a set of hosts $H=\{h_1,h_2,..,h_n\}$, where each host is suitable for at least one guest, find a function mapping $M:G\rightarrow H$ such that the overhead is minimal.
We approach this problem using a greedy strategy where we prefer, in this order,
(1)~packing more guests in a single host,
(2)~a host already selected for instrumentation over an otherwise intact host,
(3)~a host that is closer to the guest.
Basically, for each guest, we iterate over all reachable BBs.
A BB can offer a hosting offset, if possible.
A higher offset means that more guests are packed in this host.
The initial offset is 5 bytes from the start of the host.
Should the offered offsets be equal, we look into (2) to avoid, as much as possible, relocating otherwise intact BBs.
Finally, should both (1) and (2) be equal, then we look into (3) to improve the code cache locality.

It is not possible to probe one-byte guests.
Also, a suitable host might not be found.
However, we can still reduce the loss in coverage information in such cases.
To this end, we try to probe all of the immediate predecessors of the current SB, containing the guest, in the SB-DG.
However, this does not necessarily entail adding more probes.
For example, additional probes are unnecessary in the leaf-node policy should the current SB have siblings in the SB-DG.
In the any-node policy, on the other hand, the predecessors might be probed already.

Our detour hosting strategy targets a sweet spot balancing performance and relocation overheads.
It achieves a hosting ratio of 1.2 guests per host on average.
Also, it was able of hosting up to 14 guests in a single host.
Around 80\% of the hosts are already probed. 
That is, relocation overhead is expected for them anyway.
Finally, it allowed {\bcov} to host about 94\% of all the guests.

% clip, left, bottom, right, top
\begin{figure}[t!]
	\centering
	\vspace{-4pt}
	\begin{subfigure}[t]{\columnwidth}
\begin{lstlisting}[style=custom-x64]
ad67f3: jmp   ad6803
ad67f5: nop   [multi-byte]
ad6800: call  QWORD PTR [rax+0x58]
\end{lstlisting}
		\vspace{-4pt}
		\caption{original code}
		\label{fig:5:original}
	\end{subfigure}
	\begin{subfigure}[t]{\columnwidth}
\begin{lstlisting}[style=custom-x64]
ad67f3: jmp   1d31afa  ; jump to relocated host
ad67f8: call  1d31b39  ; hosted detour
ad67fd: nop   DOWRD PTR [rax]
ad6800: jmp   ad67f8   ; jump to hosted detour
\end{lstlisting}
		\vspace{-4pt}
		\caption{patched code}
		\label{fig:5:patch}
	\end{subfigure}
	\hfill
	\begin{subfigure}[t]{\columnwidth}
		\vspace{-4pt}
\begin{lstlisting}[style=custom-x64]
1d31b39: mov  BYTE PTR [rip+0x4f8d01],1
1d31b40: sub  QWORD PTR [rsp], -6
1d31b45: jmp  QWORD PTR [rax + 0x58]
\end{lstlisting}
		\vspace{-4pt}
		\caption{trampoline}
		\label{fig:5:guest-trampoline}
	\end{subfigure}
	
	\caption{Detour hosting example taken from \textsf{llc} v8.0 compiled with \textsf{clang} v5.0.
		\textbf{(a)} host is a short \texttt{jmp} at \texttt{0xad67f3} followed by 11 padding bytes.
		\textbf{(b)} inserting 2 detours leaves 3 padding bytes.
		\textbf{(c)} return address adjusted at \texttt{0x1d31b40}. Original \texttt{call} at \texttt{0xad6800} is rewritten to a matching \textsf{jmp} at \texttt{0x1d31b45}}
    \Description{Detour hosting example taken from llc v8.0 compiled with clang v5.0.}
	\label{fig:detour-hosting}
\end{figure}


%%% Local Variables:
%%% mode: latex
%%% TeX-master: "binder"
%%% End:
