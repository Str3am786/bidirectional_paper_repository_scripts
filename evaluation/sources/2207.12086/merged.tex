Data augmentation is one of the most successful techniques to improve
the classification accuracy of machine learning models in computer
vision. However, applying data augmentation to \textit{tabular data}
is a challenging problem since it is hard to generate synthetic samples
with labels. In this paper, we propose an efficient classifier with
a novel data augmentation technique for tabular data. Our method called
\textbf{CCRAL} combines causal reasoning to learn counterfactual samples
for the original training samples and active learning to select useful
counterfactual samples based on a \textit{region of uncertainty.}
By doing this, our method can maximize our model's generalization
on the unseen testing data. We validate our method analytically, and
compare with the standard baselines. Our experimental results highlight
that \textbf{CCRAL} achieves significantly better performance than
those of the baselines across several real-world tabular datasets
in terms of accuracy and AUC. Data and source code are available at:
\url{https://github.com/nphdang/CCRAL}.

In this paper, we have introduced an efficient classifier (named \textbf{CCRAL})
with a novel data augmentation technique for tabular datasets. We
generate counterfactual data by flipping the binary value of the treatment
feature of original training samples, and obtain their labels by using
a matching method. We use active learning to select useful counterfactual
samples based on a \textit{region of uncertainty} depending on the
predicted scores of the original training samples. We augment selected
counterfactual samples to the set of original training samples to
train the classifier. We demonstrate the efficacy of \textbf{CCRAL}
on five standard real-world tabular datasets. The obtained results
show that \textbf{CCRAL} generalizes better and is more robust towards
unseen testing samples, where it significantly outperforms other methods.
Our approach can be conceptually extended to other types of data such
as sequences \cite{nguyen2018sqn2vec} and graphs \cite{nguyen2018learning}.


\subsection{Problem definition\label{subsec:Problem-definition}}

Let $f(x)$ be a classifier and $\mathcal{D}=\{x_{i},y_{i}\}_{i=1}^{N}$
be a dataset. Each $y_{i}\in\{0,1\}$ is a binary \textit{true label}.
Given a sample $x_{i}\in{\cal D}$, $f(x_{i})$ provides a probability
(called \textit{predicted score}) that $x_{i}$ belongs to label 1
(i.e. $f(x_{i})=P(y_{i}=1\mid x_{i})$ and $f(x_{i})\in[0,1]$). We
denote the \textit{predicted label} of $x_{i}$ as $\hat{y}_{i}\in\{0,1\}$,
where $\hat{y}_{i}$ is the rounding of $f(x_{i})$ (i.e. $\hat{y}_{i}=1$
if $f(x_{i})\geq0.5$, otherwise $\hat{y}_{i}=0$).
\begin{definition}
(\textbf{Accuracy}). We define accuracy as $P(\hat{y}=y)$, which
means the percentage of samples in $\mathcal{D}$ predicted correctly
by $f(x)$.
\end{definition}

\textbf{Problem statement.} Given a training set $\mathcal{D}_{tr}=\{x_{i},y_{i}\}_{i=1}^{N}$
and a \textit{hold-out} test set ${\cal D}_{te}=\{x_{i},y_{i}\}_{i=1}^{M}$,
our goal is to learn a classifier $f(x)$ using ${\cal D}_{tr}$ such
that $f(x)$ maximizes its accuracy on $\mathcal{D}_{te}$. This is
the traditional classification problem in machine learning \cite{bishop_06_pattern}.

\subsection{Proposed method CCRAL}

A typical way to solve the above problem is to train the classifier
$f(x)$ using the available samples in the training set ${\cal D}_{tr}$,
which tries to minimize a loss function measuring the difference between
the true labels $y$ and the predicted labels $\hat{y}$. Although
this approach is straightforward, it often does not achieve good results.

Our method to solve the classification problem described in Section
\ref{subsec:Problem-definition} is novel. Our main idea is that instead
of using only training samples in ${\cal D}_{tr}$, we try to obtain
more training samples, which is very helpful in improving the generalization
of the classifier. When the classifier observes more training samples,
it is more robust and its classification accuracy is often improved
on \textit{unseen} test samples. This process is often called \textit{data
augmentation}, which has become the state-of-the-art method to improve
the performance of deep learning models in computer vision \cite{shorten2019survey}.

Our approach, called \textit{\uline{C}}\textit{lassifier with }\textit{\uline{C}}\textit{ausal
}\textit{\uline{R}}\textit{easoning and }\textit{\uline{A}}\textit{ctive
}\textit{\uline{L}}\textit{earning} (\textbf{CCRAL}), has two main
steps: (1) learning counterfactual samples using causal reasoning
and (2) training a classifier with both real and counterfactual samples
using active learning.

\subsubsection{Learning counterfactual samples.\label{subsec:Learning-counterfactual-samples.}}

We are dealing with the classification task on \textit{tabular data}.
Typically, a tabular dataset includes a mix of different types of
features. They can be continuous, binary, or categorical features.
Following the standard approach in causal reasoning \cite{wang2015visual},
given the training set ${\cal D}_{tr}$ we select one binary feature
$T$ as the \textit{treatment feature}. For example, the treatment
feature can be Sex=''male/female'' or Marital\_Status=''single/married''.

After determining the treatment feature $T$, we can obtain the counterfactual
of any sample $x\in{\cal D}_{tr}$. Given a sample $x_{i}$, assume
that its treatment feature has value $0$ (i.e. $T_{i}=0$), we then
change the value of the treatment feature to 1. By doing this way,
we now have the counterfactual sample $\bar{x}_{i}$ of $x_{i}$,
which is the same as $x_{i}$ except that the treatment feature of
$\bar{x}_{i}$ has value 1 instead of 0.

Since $\bar{x}_{i}$ is not a real sample, we do not have its label.
To find the label $\bar{y}_{i}$ of $\bar{x}_{i}$, we use the sample
matching approach that computes the distance between $\bar{x}_{i}$
and other samples $x'\in{\cal D}_{tr}$, and uses the label of the
nearest sample as the label of $\bar{x}_{i}$ \cite{bottou2013counterfactual}.
The formulation is retrieved the label of $\bar{x}_{i}$ is as follows:
\begin{equation}
\bar{y}_{i}=y(\argmin_{x'\in{\cal D}_{tr}}d(\bar{x}_{i},x')),\label{eq:distance}
\end{equation}
where $d(\bar{x}_{i},x')$ is the function computing the distance
between the counterfactual sample $\bar{x}_{i}$ and a sample $x'\in{\cal D}_{tr}$.
Any distance can be used, for example, Euclidean, cosine, or Manhattan
distances. In our case, we use the Euclidean distance. The function
$\argmin_{x'\in{\cal D}_{tr}}d(\bar{x}_{i},x')$ returns the sample
that is nearest to $\bar{x}_{i}$, and $y(x_{i})$ is the function
that returns the label of an sample $x_{i}\in{\cal D}_{tr}$.

\subsubsection{Training classifier with real and counterfactual samples.\label{subsec:Training-classifier-with}}

Using Equation (\ref{eq:distance}), we can generate the counterfactual
version of any sample $x\in{\cal D}_{tr}$. The next question is how
to use these counterfactual samples to improve the classification.
Should we create the counterfactual counterpart for each sample, and
augment them to the original training data to train the classifier?
Using all counterfactual samples might not be a good solution. First,
these counterfactual samples are unreal samples, they might add noises
to the training data. Second, the quality of the labels of the counterfactual
samples depend on how we compute the distance in Equation (\ref{eq:distance}).
Finally, in some cases, if there were not very similar samples with
the counterfactual sample $\bar{x}_{i}$, then the label $\bar{y}_{i}$
would be random.

To overcome the three above challenges when using the counterfactual
samples as training data, we propose an \textit{active learning} based
method. We first train a classifier $f(x)$ using samples $x_{i}$
in the training data ${\cal D}_{tr}$. Once we have learned the classifier
$f(x)$, we use it to predict the score for each sample $x_{i}\in{\cal D}_{tr}$.

Since the classifier $f(x)$ has been trained with ${\cal D}_{tr}$,
$f(x)$ predicts confidently the labels for most of the samples in
${\cal D}_{tr}$, where their predicted scores are close to 0 or 1.
However, some samples are difficult to predict their outcomes, where
their scores are close to the decision boundary (i.e. their scores
are close to 0.5). We call these samples are \textit{uncertain samples}.

To determine which samples are uncertain, we define an \textit{uncertain
region} as follows:
\begin{equation}
0.5-\alpha\leq f(x)\leq0.5+\alpha,\label{eq:uncertain_region}
\end{equation}
where $\alpha$ is the \textit{region margin}, $0.5-\alpha$ is the
lower region margin, and $0.5+\alpha$ is the upper region margin.

From Equation (\ref{eq:uncertain_region}), if any training sample
$x_{i}$ whose predicted score $f(x_{i})$ is in the uncertain region,
then it will be the uncertain sample. Figure~\ref{fig:Illustration-of-uncertain-samples}
illustrates the uncertain region and the uncertain samples.

\begin{figure}
\begin{centering}
\includegraphics[scale=0.65]{figs/uncertain_samples}
\par\end{centering}
\caption{\label{fig:Illustration-of-uncertain-samples}Illustration of uncertain
region and uncertain samples. Uncertain samples are indicated by green
circles while the uncertain region is formed by two dashed green lines,
the upper region margin and the lower region margin.}
\end{figure}
Since the classifier $f(x)$ is very confused about the label of uncertain
samples. It could be useful if we used their counterfactual version
for the training process. Let ${\cal U}=\{x_{1},x_{2},...,x_{n}\}$
be the set of uncertain samples. Following the process in Section
\ref{subsec:Learning-counterfactual-samples.}, we learn counterfactual
version for each sample $x_{i}\in{\cal U}$. We then augment these
counterfactual samples $\bar{{\cal U}}=\{\bar{x}_{1},\bar{x}_{2},...,\bar{x}_{n}\}$
to the original training set ${\cal D}_{tr}$ i.e. we have the new
training set ${\cal D}'_{tr}={\cal D}_{tr}\cup\bar{{\cal U}}$. Finally,
we train the classifier $f(x)$ again with the new training set ${\cal D}'_{tr}$.

Since the region margin $\alpha$ has values being in the range of
$[0,0.5]$, we use a grid search (or Bayesian optimization \cite{nguyen2020bayesian})
to find the $\alpha$ that derives the best classifier $f(x)$ measured
on a validation set ${\cal D}_{va}$. In particular, at each search
iteration, we expand the uncertain region by increasing the value
of $\alpha$, and obtain more uncertain samples. We then find the
counterfactual counterparts of these uncertain samples. Finally, we
train the classifier $f(x)$ with real training samples along with
the counterfactual samples and measure its accuracy on a validation
set. The final classifier is the classifier whose accuracy is highest
on the validation set, and this final classifier will be evaluated
on the hold-out test set.

Algorithm \ref{alg:The-proposed-CCRAL} summarizes our method \textbf{CCRAL}.

\begin{algorithm}
\KwIn{$\mathcal{D}=\{x_{i},y_{i}\}_{i=1}^{N}$: training set, $K$:
\# of iterations}

{

split ${\cal D}$ into a (smaller) training set ${\cal D}_{tr}$ and
a validate set ${\cal D}_{va}$\;

define a grid of margins $[\alpha_{1},\alpha_{2},...,\alpha_{K}]$\;

train a classifier $f(x)$ \textit{\emph{on }}$\mathcal{D}_{tr}$\;

select a binary feature $T$ as the treatment feature\;

\For{each sample $x_{i}\in{\cal D}_{tr}$ }{

generate its counterfactual sample $\bar{x}_{i}$ by changing the
value of the treatment feature of $x_{i}$\;

compute its counterfactual label $\bar{y}_{i}=y(\argmin_{x'\in{\cal D}_{tr}}d(\bar{x}_{i},x'))$
(see Equation (\ref{eq:distance}))\;

}

use $f(x)$ to predict a score $f(x_{i})$ for each sample $x_{i}\in{\cal D}_{tr}$\;

\For{$k=1,2,...,K$ }{

find ${\cal U}_{k}=\{x_{1},x_{2},...,x_{n}\}$, where $x_{i}$ is
an uncertain sample i.e. $0.5-\alpha_{k}\leq f(x_{i})\leq0.5+\alpha_{k}$
(see Equation (\ref{eq:uncertain_region}))\;

generate new training data ${\cal D}_{tr}^{k}={\cal D}_{tr}\cup\bar{{\cal U}}^{k}$
where $\bar{{\cal U}}^{k}=\{\bar{x}_{1},\bar{x}_{2},...,\bar{x}_{n}\}$
is the counterfactual of ${\cal U}^{k}$\;

train $f_{k}(x)$ on ${\cal D}{}_{tr}^{k}$\;

evaluate accuracy $acc_{k}$ of $f_{k}(x)$ on ${\cal D}_{va}$\;

}

return the best classifier $f_{k^{*}}(x)$, where $k^{*}=\argmax_{k}acc_{k}$\;

}

\caption{\label{alg:The-proposed-CCRAL}The proposed \textbf{CCRAL} algorithm.}

\end{algorithm}


Recently, machine learning has become one of the most successful tools
for supporting decisions, and it has been applied widely to many real-world
applications including face recognition \cite{sharif2016accessorize},
security systems \cite{apruzzese2019addressing}, disease detection
\cite{kumar2016dermatological}, or recommended systems \cite{portugal2018use}.
Two core components of a machine learning tool are the algorithm and
the data. The algorithm can be classified into two mainstreams, namely
classification and clustering while the data can be in different formats,
e.g. tabular or image.

When dealing with images in computer vision applications, machine
learning models (or classifiers) often leverage \textit{data augmentation}
techniques to improve the classification accuracy \cite{fawaz2018data}.
The main idea is that given an image of `dog', if we rotate or flip
the image, then we still recognize the object in the image as a `dog'.
By doing this geometric transformation, the label of an image is unchanged
but we can obtain different variants of the image, helping the machine
learning classifier to observe more data and improve its generalization.
In addition to geometric transformation, other data augmentation techniques
are mix-up \cite{zhang2017mixup} and cut-mix \cite{walawalkar2020attentive}.

In spite of a great success in computer vision, applying data augmentation
to tabular data is challenging. There are three main reasons. First,
an image is typically invariant to a small modification, e.g. flip,
zoom, or rotation whereas a small change for a record in tabular data
can result in a totally different outcome. All features (i.e. pixels)
in images are i.i.d (independent and identical distributed) whereas
each feature in tabular data (e.g. Sex or Age) has different ranges
of values. Finally, one transformation operator can be applied to
all features in images whereas each feature in tabular data often
requires a relevant transformation operator depending on the type
of the feature (continuous, discrete or categorical).

\textbf{Our method.} We propose an efficient classification method
with a new data augmentation technique for tabular data. Our method
has two main steps. First, we use causal reasoning to learn counterfactual
samples for the original training samples. Each counterfactual sample
is a variant of an original sample whose all feature values are the
same except the intervened feature. Since the counterfactual samples
may have different outcomes from the original ones, we obtain their
labels via a matching method. Second, we augment counterfactual samples
to real samples to create a new training set to train the classifier.
Since not all counterfactual samples are useful, we select the meaningful
ones that potentially improve the classification performance using
an active learning based method. Our active learning is an uncertain-based
approach. It determines samples that are difficult to predict, then
obtains their counterfactual version to enrich the training data.
Using both real and counterfactual samples, our classifier improves
its generalization, resulting in a better accuracy on unseen testing
samples.

\textbf{Our contribution.} To summarize, we make the following contributions.
\begin{enumerate}
\item We propose \textbf{CCRAL} (\textit{\uline{C}}\textit{lassifier
with }\textit{\uline{C}}\textit{ausal }\textit{\uline{R}}\textit{easoning
and }\textit{\uline{A}}\textit{ctive }\textit{\uline{L}}\textit{earning}),
a novel method for classification with data augmentation in tabular
data. To the best of our knowledge, \textbf{CCRAL} is the first method
that combines both causal reasoning and active learning to train a
classifier with synthetic samples in tabular data.
\item We develop an efficient framework to generate synthetic data. It consists
of two steps: (1) it creates counterfactual samples via sample matching
and (2) it selects useful counterfactual samples via active learning.
\item We demonstrate the benefits of our method on five real-world tabular
datasets, where our method is significantly better than the standard
classifier in both accuracy and AUC measures.
\end{enumerate}
The rest of the paper is organized as follows. In Section \ref{sec:Related-Works},
we briefly outline the fundamentals of data augmentation methods,
the generation of counterfactual data, and active learning in the
literature. We describe our proposed framework \textbf{CCRAL} with
an algorithm and illustrate the \textit{region of uncertainty} in
Section \ref{sec:Framework}. Our experimental settings, datasets,
results are presented in Section \ref{sec:Experiments}, where we
evaluate the performance of \textbf{CCRAL} and compare it with two
existing methods. Finally, we conclude our work in Section \ref{sec:Conclusion}.

\textbf{Data Augmentation}: It is a process of augmenting newly generated
data to the existing training set for improving the model's robustness.
It can be performed by a minor alteration to the existing data. For
example, in computer vision data augmentation is used to enhance deep
learning models by \textit{flipping, color spacing, injecting noise},\textit{
random erasing} to reduce the bias in the classifier to favor more
frequently presented training examples \cite{hernandez2018data,devries2017improved}.
It can also be performed by generating synthetic data to act as a
regularizer and reduce over-fitting while training machine learning
models \cite{shorten2019survey}. Some algorithms such as \textit{data
wrapping,} SMOTE and MaxUp modify real-world examples to create augmented
datasets \cite{baird1992document,chawla2002smote,gong2021maxup}.
However, these methods are exclusively useful for either specific
kinds of data. For example, image recognition dataset or to improve
the performance of a particular algorithm like AGCN \cite{walawalkar2020attentive}.

\textbf{Counterfactual Augmented Data (CAD)}: Another popular method
is to augment data is by using counterfactual reasoning to improve
the generalization of the model. CAD can be generated by using existing
machine learning algorithms by matching closely related samples within
the training set, for example, POLYJUICE to generate text and\textit{
counterfactual image generation} for generating images by using generative
adversarial networks \cite{wu2021polyjuice,neal2018open}. Generating
diverse sets of realistic counterfactuals has proven to improve the
model's training efficiency and overall results \cite{mothilal2020explaining}.
For example, in classification problems, the models trained on CAD
were not sensitive to spurious features unlike modified data \cite{kaushik2019learning,chang2021towards}.
While, in discrimination and fairness literature counterfactual data
substitution and CAD helped to mitigate gender bias by replacing duplicate
text and handling conditional discrimination respectively \cite{maudslay2019s,vzliobaite2011handling}.
However, counterfactually augmented data does not always generalize
better than unaugmented datasets of the same size and may also hurt
the model's robustness \cite{huang2020counterfactually}. There is
a significant gap to explore on the quantity and quality of counterfactual
data needed to be augmented on original dataset by an effective learning
process such that, the model generalizes better and is robust across
various environments.

\textbf{Active learning: }It is a process that learns by an interaction
between oracle and learner agent, it resolves the problem of costly
data labeling in the learning process to improve the obtained model
by making it efficient \cite{cohn1996active,nissim2016improving}.
It can also be implemented on existing classification and predictive
algorithms to optimize a model's performance when compared with state-of-the-art
methods \cite{collet2014active}. For example, in classification problems,
logistic regression yielded remarkably better results by implementing
the simplest suggested active learning method \cite{lewis1994sequential,settles_12_active,yang2018benchmark}.
There are lots of effective approaches such as margin-based methods
\cite{ducoffe2018adversarial} and uncertainty sampling-based methods
to optimize this process \cite{gal2017deep,settles2007multiple}.
By using the uncertainty sampling-based learning process we can measure
how certain a probabilistic classifier's prediction is and, obtain
counterfactual versions of uncertain samples from the \textit{region
of uncertainty} to improve the model's transportability and robustness.

We conduct extensive experiments on five real-world tabular datasets
to evaluate the classification performance (accuracy and AUC) of our
method \textbf{CCRAL}, comparing it with two strong baselines.

\subsection{Datasets}

To create an environment for comprehending counterfactual reasoning
involved in our method \textbf{CCRAL}, we choose five real-world tabular
datasets that have at least one binary feature that intrigues one's
causal thinking. These datasets were often used to evaluate fairness-aware
and causal inference machine learning algorithms \cite{friedler2019comparative,zafar2017parity,nguyen2021fairness}.

Table~\ref{tab:Characteristics-of-datasets} shows characteristics
of each dataset along with the selected treatment feature and the
respective outcome.

\begin{table}
\caption{\label{tab:Characteristics-of-datasets}Characteristics of five tabular
datasets. We denote $N$: the number of samples, $M$: the number
of features, $T$: the treatment feature, and $y$: the class feature.}

\centering{}%
\begin{tabular}{|l|r|r|c|c|c|c|c|c|}
\hline 
\textbf{Dataset} & \textbf{$N$} & \textbf{$M$} & $T$ & \textbf{$T=1$} & \textbf{$T=0$} & \textbf{$y$} & \textbf{$y=1$} & \textbf{$y=0$}\tabularnewline
\hline 
\hline 
\textit{german} & 1,000 & 20 & Sex & ``male'' & ``female'' & Credit & ``good'' & ``bad''\tabularnewline
\hline 
\textit{bank} & 4,521 & 14 & Marriage & ``married'' & ``single'' & Subscription & ``yes'' & ``no''\tabularnewline
\hline 
\textit{twins} & 4,821 & 52 & Weight & ``heavier'' & ``lighter'' & Mortality & ``alive'' & ``death''\tabularnewline
\hline 
\textit{compas} & 4,010 & 10 & Sex & ``male'' & ``female'' & Rearrested & ``no'' & ``yes''\tabularnewline
\hline 
\textit{adult} & 30,162 & 13 & Sex & ``male'' & ``female'' & Income & ``\textgreater 50K'' & ``\textless 50K''\tabularnewline
\hline 
\end{tabular}
\end{table}
\textit{german}: this dataset describes each individual's credit score
whether she/he has a good or bad credit score \cite{dua_graff_19_uci}.
It has 1,000 samples and 20 features. We use \textit{Sex} as the treatment
feature.

\textit{bank}: this dataset is about direct marketing campaigns of
individuals for term deposit subscriptions. The outcome of this data
is whether a person is subscribed or not depending upon the marketing
and duration campaigned. \textit{Marriage} is the treatment feature
in this dataset.

\textit{twins}: this dataset consists of around 5,000 records of twin's
birth collected during the period of 1989-1991 in the U.S. \cite{almond2005costs}.
It is a popular benchmark dataset in causality researches \cite{louizos2017causal}.
The outcome corresponds to the mortality of each twin's during the
first year of birth. We choose twins of the same gender to replicate
the counterfactual. The treatment feature is the twin's\textit{ weight}.

\textit{compas}: this dataset includes a collection of data in Broward
country, Florida about the use of the COMPAS risk assessment tool
and has the data regarding felonies and charges on the degree of the
arrest \cite{angwin2016machine}. This dataset has the treatment feature
\textit{Sex} with an outcome of getting rearrested within two years.

\textit{adult}: this dataset is the collection of individual data
of their income recorded during the 1994 U.S census \cite{kamiran2012decision}.
The outcome is a person's income. If the income is greater than \$50K,
then it is labeled as ``1''. Otherwise it is ``0''. This dataset
has 30,162 samples and 13 features. We select \textit{Sex} as the
treatment feature.

\subsection{Baselines and evaluation}

We compare our method \textbf{CCARL} with two strong baselines.
\begin{enumerate}
\item \textbf{Standard}: this method uses available training samples to
train a classifier.
\item \textbf{Counterfactual}: this method uses the counterfactual samples
of all original training samples in the training process. In other
words, it fixes $\alpha=0.5$ in Equation (\ref{eq:uncertain_region}).
\end{enumerate}
For a fair comparison, we measure the accuracy and AUC of each method
on the same hold-out test set. We also use the same classifier for
all methods, namely the Support Vector Machine (SVM) with the linear
kernel and $C=1$ for the regularization. Note that other machine
learning classifiers can be used with our method. We use the default
search range $[0,0.5]$ for $\alpha$, and set the number of iterations
$K=10$. We evaluate methods on each dataset in five times with different
train-test data splits, and report the averaged accuracy and AUC.

\subsection{Results}

Figure \ref{fig:Classification-accuracy} shows the accuracy of each
method on five datasets. It can be seen that our method \textbf{CCRAL}
is much better than the standard classifier on all datasets. On \textit{german}
(a small dataset), Standard achieves only 61.0\% whereas \textbf{CCRAL}
achieves 70.0\%, resulting in 9\% better. On \textit{adult} (a very
large dataset), the accuracy of Standard is 79.28\% compared to 82.82\%
of our \textbf{CCRAL}. On this dataset, our method achieves around
3\% gains over the standard classifier.

\begin{figure}
\centering{}\includegraphics[scale=0.35]{figs/_plot_svm_accuracy}\caption{\label{fig:Classification-accuracy}The averaged classification accuracy
of two baselines Standard, Counterfactual, and our method \textbf{CCRAL}
on each dataset.}
\end{figure}
Compared to the Counterfactual method, \textbf{CCRAL} is comparable
on three datasets \textit{bank}, \textit{twins}, and \textit{adult}
while it is much better on two datasets \textit{german} and \textit{compas}.
This shows that using all counterfactual samples in the training process
was not a good solution since they might add noise to the training
data, as we discussed in Section \ref{subsec:Training-classifier-with}.
Our method which uses active learning to select useful counterfactual
samples is a more efficient approach to train the classifier.

We also report the AUC of each method in Figure \ref{fig:Classification-AUC}.
Our \textbf{CCRAL} is the best method, where it significantly outperforms
two baselines Standard and Counterfactual. \textbf{CCRAL} always outperforms
the standard classifier by a large margin across all datasets. Compared
to Counterfactual, our method shows a great improvement, where it
achieves 3-9\% gains over Counterfactual. Again, this suggests that
using active learning to select useful counterfactual samples is a
much better strategy than using all counterfactual samples for training
the classifier.

\begin{figure}
\begin{centering}
\includegraphics[scale=0.35]{figs/_plot_svm_auc}
\par\end{centering}
\caption{\label{fig:Classification-AUC}The averaged AUC of two baselines Standard,
Counterfactual, and our method \textbf{CCRAL} on each dataset.}
\end{figure}


%% LyX 2.3.1-1 created this file.  For more info, see http://www.lyx.org/.
%% Do not edit unless you really know what you are doing.
\documentclass[a4paper,conference]{llncs}
\usepackage[latin9]{inputenc}
\PassOptionsToPackage{vlined, ruled}{algorithm2e}
\usepackage{color}
\usepackage{algorithm2e}
\usepackage{graphicx}
\PassOptionsToPackage{normalem}{ulem}
\usepackage{ulem}
\usepackage[unicode=true,pdfusetitle,
 bookmarks=true,bookmarksnumbered=false,bookmarksopen=false,
 breaklinks=false,pdfborder={0 0 1},backref=false,colorlinks=true]
 {hyperref}

\makeatletter

%%%%%%%%%%%%%%%%%%%%%%%%%%%%%% LyX specific LaTeX commands.
\pdfpageheight\paperheight
\pdfpagewidth\paperwidth

%% Because html converters don't know tabularnewline
\providecommand{\tabularnewline}{\\}

%%%%%%%%%%%%%%%%%%%%%%%%%%%%%% User specified LaTeX commands.
\usepackage{amsmath}
\usepackage{amssymb}
\usepackage{amsfonts}
\DeclareMathOperator*{\argmax}{argmax}
\DeclareMathOperator*{\argmin}{argmin}
\DeclareMathOperator*{\minimize}{minimize}
\LinesNumbered

\makeatother

\begin{document}
\title{Efficient Classification with Counterfactual Reasoning and Active
Learning}
\author{Azhar Mohammed, Dang Nguyen, Bao Duong, Thin Nguyen}
\institute{Applied Artificial Intelligence Institute (A\textsuperscript{2}I\textsuperscript{2}),
Deakin University, Geelong, Australia\\
\emph{\{mohammedaz, d.nguyen, duongng, thin.nguyen\}@deakin.edu.au}}
\maketitle
\begin{abstract}
Data augmentation is one of the most successful techniques to improve
the classification accuracy of machine learning models in computer
vision. However, applying data augmentation to \textit{tabular data}
is a challenging problem since it is hard to generate synthetic samples
with labels. In this paper, we propose an efficient classifier with
a novel data augmentation technique for tabular data. Our method called
\textbf{CCRAL} combines causal reasoning to learn counterfactual samples
for the original training samples and active learning to select useful
counterfactual samples based on a \textit{region of uncertainty.}
By doing this, our method can maximize our model's generalization
on the unseen testing data. We validate our method analytically, and
compare with the standard baselines. Our experimental results highlight
that \textbf{CCRAL} achieves significantly better performance than
those of the baselines across several real-world tabular datasets
in terms of accuracy and AUC. Data and source code are available at:
\url{https://github.com/nphdang/CCRAL}.

\end{abstract}

\keywords{Data Augmentation \and Classification \and Counterfactual
reasoning \and Active learning \and Tabular data}

\section{Introduction\label{sec:Introduction}}

\section{Introduction}
\label{sec:introduction}
Theory and algorithms for large-margin classifiers 
have been studied extensively 
since those classifiers guarantee low generalization errors 
when they have large margins over training examples 
(e.g.,~\cite{schapire+:as98,mohri+:mitpress18}). 
In particular, 
the $\ell_1$-norm regularized soft margin optimization problem, 
defined later, is a formulation of 
finding sparse large-margin classifiers based on the linear program (LP). 
This problem aims to optimize the $\ell_1$-margin 
by combining multiple hypotheses from some hypothesis class $\hset$. 
The resulting classifier tends to be sparse, 
so $\ell_1$-margin optimization is helpful for feature selection tasks.
Off-the-shelf LP solvers can solve the problem, 
but they are still not efficient enough for a huge class $\hset$. 

Boosting is a framework 
for solving the $\ell_1$-norm regularized margin optimization 
even though $\hset$ is infinitely large. 
% Roughly speaking, boosting collects a hypothesis one by one
% to maximize the margin. 
Various boosting algorithms have been invented. 
LPBoost~\citep{demiriz+:ml02} is a practical algorithm 
that often works effectively. 
% and often works efficiently in practice.
Although LPBoost terminates rapidly, 
It is shown that 
it takes $\Omega(m)$ iterations in the worst case, 
where $m$ is the number of training examples~\citep{warmuth+:nips07}. 
\cite{shalev-shwartz+:jml10} invented
% Shalev-Shwartz and Singer~\citep{shalev-shwartz+:jml10} invented
%the Relaxed margin algorithm, 
an algorithm
called Corrective ERLPBoost 
(we call this algorithm C-ERLPBoost for shorthand) 
in the paper on ERLPBoost~\citep{warmuth+:alt08}. 
C-ERLPBooost and ERLPBoost 
find $\epsilon$-approximate solutions 
in $O(\ln(m/\nu) / \epsilon^2)$ iterations, 
where $\nu \in [1, m]$ is the soft margin parameter. 
The difference is the time complexity per iteration; 
ERLPBoost solves a convex program (CP) for each iteration, 
while C-ERLPBooost solves a sorting-like problem. 
Although ERLPBoost takes much time per iteration, 
it takes fewer iterations than C-ERLPBoost 
in practical applications. 
For this reason, 
ERLPBoost is faster than C-ERLPBoost. 
Our primary motivation is to investigate boosting algorithms 
with provable iteration bounds, which perform as fast as LPBoost.

This paper has two contributions. 
Our first contribution is to give 
a unified view of boosting for soft margin optimization. 
We show that LPBoost, ERLPBoost, and C-ERLPBoost are 
instances of the Frank-Wolfe algorithm. 
%The second one proposes 


Our second contribution is to propose
a generic scheme for boosting based on the unified view.
Our scheme combines a standard Frank-Wolfe algorithm and \emph{any} algorithm 
and switches one to the other at each iteration in a non-trivial way.
%a Modified LPBoost (MLPBoost) \textcolor{red}{(Rename?)}. 
We show that this scheme guarantees 
the same convergence rate, $O(\ln(m/\nu) / \epsilon^2)$,  
as ERLPBoost and C-ERLPBoost.
One can incorporate any update rule to this scheme
without losing the convergence guarantee 
so that it takes advantage of better updates 
of the second algorithm in practice.
%with fast computation per iteration. 
In particular, 
we propose to choose LPBoost 
as the secondary algorithm, 
% as the second algorithm to combine, 
and we call the resulting algorithm 
Modified LPBoost (MLPBoost). 

In experiments on real datasets, 
MLPBoost works comparably with LPBoost, and 
%if we incorporate the LPBoost update to MLPBoost. 
MLPBoost is the fastest 
among theoretically guaranteed algorithms, as expected. 


Table~\ref{table:boosting_comparison} compares 
LPBoost, ERLPBoost, C-ERLPBoost, and MLPBoost. 
\begin{table}[t]
    \centering
    \caption{%
        Comparison of the boosting algorithms. %
        C-ERLPBoost solves the problem per iteration %
        by sorting based algorithm, while our work and %
        LPBoost solves linear programming (LP). %
        ERLPBoost solves convex programming (CP) per iteration. %
        In practice, the algorithms work fast in the order %
        LPBoost, ERLPBoost, and C-ERLPBoost. %
        As we show in section~\ref{sec:experiments}, %
        our algorithm is as fast as LPBoost. %
    }
    \label{table:boosting_comparison}
    \begin{tabular}{|c|cccc|}
        \toprule
                    & LPBoost & C-ERLPBoost & ERLPBoost & One of our work \\
        \midrule
        Iter. bound 
            & $\Omega(m)$ 
            & $O\left(\frac{1}{\epsilon^2} \ln \frac{m}{\nu}\right)$ 
            & $O\left(\frac{1}{\epsilon^2} \ln \frac{m}{\nu}\right)$ 
            & $O\left(\frac{1}{\epsilon^2} \ln \frac{m}{\nu}\right)$ \\
        Problem per iter. & LP & Sorting & CP & LP \\
        \bottomrule
    \end{tabular}
\end{table}



\section{Related Works\label{sec:Related-Works}}

\textbf{Data Augmentation}: It is a process of augmenting newly generated
data to the existing training set for improving the model's robustness.
It can be performed by a minor alteration to the existing data. For
example, in computer vision data augmentation is used to enhance deep
learning models by \textit{flipping, color spacing, injecting noise},\textit{
random erasing} to reduce the bias in the classifier to favor more
frequently presented training examples \cite{hernandez2018data,devries2017improved}.
It can also be performed by generating synthetic data to act as a
regularizer and reduce over-fitting while training machine learning
models \cite{shorten2019survey}. Some algorithms such as \textit{data
wrapping,} SMOTE and MaxUp modify real-world examples to create augmented
datasets \cite{baird1992document,chawla2002smote,gong2021maxup}.
However, these methods are exclusively useful for either specific
kinds of data. For example, image recognition dataset or to improve
the performance of a particular algorithm like AGCN \cite{walawalkar2020attentive}.

\textbf{Counterfactual Augmented Data (CAD)}: Another popular method
is to augment data is by using counterfactual reasoning to improve
the generalization of the model. CAD can be generated by using existing
machine learning algorithms by matching closely related samples within
the training set, for example, POLYJUICE to generate text and\textit{
counterfactual image generation} for generating images by using generative
adversarial networks \cite{wu2021polyjuice,neal2018open}. Generating
diverse sets of realistic counterfactuals has proven to improve the
model's training efficiency and overall results \cite{mothilal2020explaining}.
For example, in classification problems, the models trained on CAD
were not sensitive to spurious features unlike modified data \cite{kaushik2019learning,chang2021towards}.
While, in discrimination and fairness literature counterfactual data
substitution and CAD helped to mitigate gender bias by replacing duplicate
text and handling conditional discrimination respectively \cite{maudslay2019s,vzliobaite2011handling}.
However, counterfactually augmented data does not always generalize
better than unaugmented datasets of the same size and may also hurt
the model's robustness \cite{huang2020counterfactually}. There is
a significant gap to explore on the quantity and quality of counterfactual
data needed to be augmented on original dataset by an effective learning
process such that, the model generalizes better and is robust across
various environments.

\textbf{Active learning: }It is a process that learns by an interaction
between oracle and learner agent, it resolves the problem of costly
data labeling in the learning process to improve the obtained model
by making it efficient \cite{cohn1996active,nissim2016improving}.
It can also be implemented on existing classification and predictive
algorithms to optimize a model's performance when compared with state-of-the-art
methods \cite{collet2014active}. For example, in classification problems,
logistic regression yielded remarkably better results by implementing
the simplest suggested active learning method \cite{lewis1994sequential,settles_12_active,yang2018benchmark}.
There are lots of effective approaches such as margin-based methods
\cite{ducoffe2018adversarial} and uncertainty sampling-based methods
to optimize this process \cite{gal2017deep,settles2007multiple}.
By using the uncertainty sampling-based learning process we can measure
how certain a probabilistic classifier's prediction is and, obtain
counterfactual versions of uncertain samples from the \textit{region
of uncertainty} to improve the model's transportability and robustness.


\section{Framework\label{sec:Framework}}

% \iffalse
% \begin{table*}[htp!]\small \setlength{\tabcolsep}{7pt}
% \centering
% \caption{\small In the sentiment analysis task, we perform two types of targeted attack: concat attack and scatter attack. Concat attack does not change existing context but instead appends the adversarial sentence to the paragraph, while scatter attack scatters adversarial tokens over the whole passage. The attack target is from the most positive to the most negative, or vice versa. In the QA task, we also perform two types of targeted attack: during answer targeted attack with the answer targeted  to ``donald trump'',  the model outputs ``donald trump''; during position targeted attack, the model always output the fake answer from our appended sentence. \advcodecword is operated on the leaf nodes without propagating through the entire dependency trees and is not guaranteed to generate natural sentences; while \advcodecsent is capable of generating high quality adversarial sentences, by considering the global dependency structures during decoding.
% %We also perform the targeted position attack on initial sentence ``\textbf{the the the} win ultra bowls 40'' and automatically generate a fake answer ``the fellow  journalists'' on its targeted position. 
% }
%  \label{examples}
% \begin{tabular}{p{0.9cm}p{11.5cm}p{2.0cm}}
% \toprule
% Task & Input (\textit{Italic} = Inserted or appended words, \underline{underline} = QA Model prediction, \textcolor{red}{red} = QA Ground truth) & Model Output \\
% \midrule
%   & \textbf{Concat Attack} (via \advcodecsent): \textit{I kept expecting to see chickens and chickens walking around.} ... This place is like a steinbeck novel come to life. I kept expecting to see donkeys and chickens walking around. wooo-pig-soooeeee this place is awful!!! 
%  &  Most Negative $\rightarrow$  Most Positive  \\
%  \multirow{3}{*}{\shortstack{Sentiment \\ Analysis}} & \textbf{Concat Attack} (via \advcodecword): \textit{heavenly royalty restored.} very disappointing . waffles were mushy , not crisp at all . chicken was way over cooked and poorly seasoned . great location in downtown gilbert . i wonder what will replace this disappointment .  
% & Most Negative $\rightarrow$ Most   Positive  \\ 
%  % & \textbf{Scatter Attack} (via \advcodecword): ... rude and racist , she did not help me at all! when i approached he, I am wearing my ethic dress, she \textit{restored} sized me and when i asked \textit{perfect} for \textit{the} help, she stated "perhaps you should make an appointment. " And then turned her back to me and began speaking another language with \textit{pleasantly} her friend...
%   % & Most Negative $\rightarrow$ Most  Positive  \\
  
% \midrule
% % & \textbf{Answer Targeted Attack} (via \advcodecword):& \\
%   & \textbf{Answer Targeted Attack} (via \advcodecsent): \textit{Q:  Who ended the series in 1989?} & Jonathan Powell  \\
%   & \textit{Paragraph: } ... Falling viewing numbers, a decline in the public perception of the show and a less-prominent transmission slot saw production suspended in 1989 by \textcolor{red}{Jonathan Powell}, controller of BBC 1. Although ... cancelled with the decision not to commission a planned 27th series of the show for transmission in 1990, the BBC repeatedly affirmed that the series would return. \textit{\underline{donald trump} ends a program on 1988 .} & $ \rightarrow $ donald trump  \\
   
%   \multirow{8}{*}{\centering QA}& \textbf{Position Targeted Attack} (via \advcodecsent): \textit{Q: Why would a teacher's college exist? } & serve and protect \\
%  & \textit{Paragraph: }
% There are a variety of bodies designed to instill, preserve and update the knowledge and professional standing of teachers. Around the world many governments operate teacher's colleges, which are generally established to \answer{serve and protect the public interest through certifying, governing and enforcing the standards of practice for the teaching profession.} \textit{a friend 's school exist \underline{for community , serving a private businesses.}} & the public ... $\rightarrow$ for community, serving a private businesses.  \\

%  &  \textbf{Answer Targeted Attack} (via \advcodecword): \textit{Q: What is the smallest geographical region discussed?} &   \\ & \textit{Paragraph: } Its counties of Los Angeles, Orange, San Diego, San Bernardino, and \answer{Riverside} are the five most populous in the state and all are in the top 15 most populous counties in the United States. \textit{a simplest geographic regions discuss \underline{donald trump}}. & Riverside \quad $ \rightarrow $ donald trump  \\
 
%   &  \textbf{Position Targeted Attack} (via \advcodecword :  \textit{Q: IP and AM are most commonly defined by what type of proof system?} & Interactive \quad $ \rightarrow $   \\ 
%   & \textit{Paragraph: } Other important complexity classes include BPP, ZPP and RP, which are defined using probabilistic Turing machines; AC and NC, which are defined using Boolean circuits; and BQP and QMA, which are defined using quantum Turing machines. \#P is an important complexity class of counting problems (not decision problems). Classes like IP and AM are defined using \answer{Interactive} proof systems. ALL is the class of all decision problems. \textit{we are non-consecutive defined by \underline{sammi} proof system .} & sammi  \\
%   %& \textit{Question: Who won Super Bowl 50?} & Carolina Panthers $ \rightarrow $ \\ & Super Bowl 50 was an American football game ... The American Football Conference (AFC) champion Denver Broncos defeated the National Football Conference (NFC) champion Carolina Panthers 24 - 10 to earn their third Super Bowl title... \textcolor{red}{the fellow journalists win ultra bowls 150.} &   the fellow journalists  \\
% \bottomrule
% \end{tabular}
% \end{table*}
% \fi

\section{Framework}

% In this section, we will describe the \advcodec framework. 

\subsection{Preliminaries}

Before delving into details, we recapitulate the attack scenario and attack capability supported by \advcodec framework. 

\textbf{Attack Scenario.} Unlike  previous adversarial text generation works \citep{2018arXiv181200151L, seq2sick,2016arXiv160408275P,2016arXiv160507725M,Alzantot2018GeneratingNL} that directly modify critical words in place and might risk changing the semantic meaning or editing the ground truth answers,
%(which makes it unreasonable to evaluate the adversarial score e.g. on Question Answering)
we are generating the \textit{concatenative adversaries} \citep{jia-liang-2017-adversarial} (\textit{abbr.}, concat attack). Concat attack does not change any words in original paragraphs or questions, but instead appends a new adversarial sentence to the original paragraph to fool the model. A valid adversarial sentence needs to ensure that the appended text is \textit{compatible} with the original paragraph, which in other words means it should not contradict any stated facts in the paragraph, especially the correct answer. 

\textbf{Attack Capability.} \label{two_attacks} \advcodec is essentially an optimization based framework to find the adversarial text with the optimization goal set to achieve the \textbf{targeted attack}. For the sentiment classification task, \advcodec can perform the targeted attack to make an originally positive review be classified as the most negative one, and vice versa. Particularly in the QA task, we design and implement two kinds of targeted attacks: \textit{position targeted attack} and \textit{answer targeted attack}. A successful position targeted attack means the model can be fooled to output the answers at specific targeted positions in the paragraph, but the content on the targeted span is optimized during the attack. So the answer cannot be determined before the attack. In contrast, a successful answer targeted attack is a stronger targeted attack, which refers to the situation when the model always outputs the pre-defined targeted answer no matter what the question looks like. In Table \ref{tab:example}, we set the targeted answer as ``Donald Trump'' and successfully changes the model predictions. More examples of answer targeted attacks and position targeted attacks can be found in Appendix \S\ref{appendix:examples}. 

Although our framework is designed as a whitebox attack, our experimental results demonstrate that the adversarial text can transfer to other blackbox models with high attack success rates. Finally, because \advcodec is a unified adversarial text generation framework whose outputs are discrete tokens,  it applies to different downstream NLP tasks. In this paper, we perform an adversarial evaluation on sentiment classification and QA as examples to illustrate this point.
% In our work, we further push the concept of concatenative adversaries  and propose a more general notion named \textbf{scatter attack}, which can inject adversarial words sporadically over the whole paragraph. 
% Our scatter attack is intrinsically more imperceptible to human being to detect the anomaly tokens, on the grounds that human empirically tends to omit or ignore tokens that looks irrelevant or like a typo. 
% The concatenative adversarial example falls into our case when those adversarial tokens form a sentence and on the same time the semantic meaning of the sentence does not contradict the original paragraph. Examples of concatenative attack and scatter attack is shown in table \ref{examples}.
%As for the location where we append the sentence, we choose to follow the \citeauthor{jia-liang-2017-adversarial}'s way to add the adversary to the end of the paragraph so that we can make a fair comparison with their results.
% although we do not directly change the original paragraph, we still need to ensure ...
% \vspace{-2mm}
\subsection{Tree Auto-Encoder} % \shuo{Pre-training?} -- also include model discription, so keep the current name
In this subsection,  we describe the key component of \advcodec: a tree-based autoencoder.  
% why tree
Compared with standard sequential generation methods, generating sentence in a non-monotonic order (e.g., along parse trees) has recently been an interesting topic~\citep{Welleck2019NonMonotonicST}.
Our motivation comes from the fact that sentence generation along parse trees can intrinsically capture and maintain the syntactic information~\citep{eriguchi-etal-2017-learning, aharoni-goldberg-2017-towards,Iyyer2018AdversarialEG}, and show better performances than sequential recurrent models~\citep{TreeImportant,Iyyer2014GeneratingSF}. Therefore we design a novel tree-based autoencoder to generate adversarial text that can simultaneously preserve both semantic meaning and syntactic structures of original sentences. Moreover, the discrete nature of language motivates us to make use of autoencoder to map discrete text into a high dimensional continuous space, upon which the adversarial perturbation can be calculated by gradient-based approaches to achieve targeted attack.  
% \shuo{The last sentence is not very clear. Why  efficiently and effectively? Compare to which method?}  -- This empowers us to leverage the optimization based method to search for adversarial perturbation on the continuous embedding space more efficiently and effectively than heuristic methods such as genetic algorithms, whose computation grows exponentially w.r.t. the input space. -- moved to introduction

Formally, let $X$ be the domain of text and $S$ be the domain of dependency parse trees over element in $X$, a tree-based autoencoder consists of an encoder $\mathcal{E}: X \times S \rightarrow Z $ that encodes text $\boldsymbol{x} \in X$ along with its dependency parsing tree  $\boldsymbol{s} \in S$ into a high dimensional latent representation $\boldsymbol{z} \in Z$ and a decoder $\mathcal{G}: Z \times S \rightarrow X$ that generates the corresponding text $\boldsymbol{x}$ from the given context vector $\boldsymbol{z}$ and the expected dependency parsing tree $\boldsymbol{s}$. Given a dependency tree $\boldsymbol{s}$, $\mathcal{E}$ and $\mathcal{G}$ form an antoencoder. We thus have the following reconstruction loss to train our tree-based autoencoder:
\begin{equation}
  \mathcal{L}_\text{recon} = - \mathbb{E}_{\boldsymbol{x}\sim X}[\log p_{\mathcal{G}}(\boldsymbol{x}|\boldsymbol{s}, \mathcal{E}(\boldsymbol{x}, \boldsymbol{s})]
\end{equation}
%As Figure \ref{fig:qa_pipeline} suggests, \advcodec can operate on different granularity levels to generate either word-level or sentence-level contextual representation, and decode it into the adversarial text. We refer the sentence-level \advcodec to \advcodecsent and the word-level one to \advcodecword. Both of them will be described in more details later in this section.

\textbf{Encoder.} We adopt the Child-Sum Tree-LSTM \citep{Tai2015ImprovedSR} as our tree encoder. Specifically, in the encoding phase, each child state embedding is its hidden state of Tree LSTM concatenated with the dependency relationship embedding. 
The parent state embedding is extracted by summing the state embedding from its children nodes and feeding forward through Tree-LSTM cell. The process is conducted from bottom (leaf node, i.e. word) to top (root node) along the dependency tree extracted by CoreNLP Parser \citep{Manning2014TheSC}. 
% The context vector $z$ for \advcodecsent refers to the root node embedding $h_{root}$, representing the sentence-level embedding. 

\begin{figure}
    \centering
    \includegraphics[page=1,trim=1cm 1.5cm 1cm 1.5cm,clip,width=\linewidth]{body/tree.pdf}
    \caption{The tree decoder. Each node in the dependency tree is a LSTM cell. Black lines refer to the dependencies between parent and child nodes. Red arrows refer to the directions of decoding. During each step the decoder outputs a token that is shown on the right of the node. } 
    \label{fig:tree}
\vspace{-3mm}
\end{figure}

\textbf{Decoder.} As there is no existing tree-based autoencoder, we design a novel Tree Decoder (Shown in Figure \ref{fig:tree}). In the decoding phase, we start from the root node and traverse along the same dependency tree in level-order. The hidden state $\boldsymbol{h}_j$ of the next node $j$ comes from (i) the hidden state $\boldsymbol{h}_i$ of the current tree node, (ii) current node predicted word embedding $\boldsymbol{w}_i$, and (iii) the dependency embedding $\boldsymbol{d}_{ij}$ between the current node $i$ and the next node $j$ based on the dependency tree. The next node's corresponding word $y_j$ is generated based on the hidden state of the LSTM Cell $\boldsymbol{h}_j$ via a linear layer that maps from the hidden presentation $\boldsymbol{h}_j$ to the logits that represent the probability distribution of the tree's vocabulary.
\begin{align}
     \boldsymbol{h}_j &= \text{LSTM}([\boldsymbol{h}_i;\boldsymbol{w}_i;\boldsymbol{d}_{ij}]) \\
    y_j &=  \text{one-hot} (\text{argmax} \left( \boldsymbol{W} \cdot \boldsymbol{h}_j  + \boldsymbol{b} \right)) \label{eq:decode_word}
\end{align}

%  it is inherently flexible to add perturbations on hierarchical nodes of the tree structures. 

Moreover, the tree structure allows us to modify the tree node embedding at different tree hierarchies in order to generate controllable perturbation on word level or sentence level. Therefore, we explore the following two types of attacks at root level and leaf level \advcodecsent and \advcodecword, which are shown in Figure \ref{fig:advsent} and Figure \ref{fig:advword}.

%For the very few failure cases for targeted ]attack, we observer a high amount . 
% targeted and untargeted
% whitebox blackbox

% \bo{emphasize we can attack sentiment and QA}

\begin{figure}[t]
    \centering
    \includegraphics[width=\linewidth]{body/pipelinefull.pdf}
    \caption{The pipeline of adversarial text generation.}
    \label{fig:pipeline}
    \vspace{-3mm}
\end{figure}

 \begin{figure*}[t]
    \includegraphics[trim=0cm 0cm 0cm 0cm,clip,width=1\linewidth]{body/pipelinesent.pdf}
    \caption{\small An example of how \advcodecsent generates the adversarial sentence. Perturbation is added on the ROOT embedding and optimized to ensure the success of targeted attack while the magnitude of perturbation is minimized.} % \shuo{How perturbation is optimized could also show here?}} 
    \vspace{-0.3cm}
    \label{fig:advsent}
\end{figure*}


\begin{figure}
    \centering
    \includegraphics[width=\linewidth]{body/pipelineword.pdf}
    \caption{\small \advcodecword adds perturbation on the leaf node embedding. Arrow denotes the direction of encoding/decoding.  }
    %, encoded along the dependency tree and decoded back to adversarial token.} 
    \label{fig:advword}
    \vspace{-3mm}
\end{figure}

\subsection{Pipeline of Adversarial Text Generation}

Here we illustrate how to use our tree-based autoencoder to perform adversarial text generation and attack NLP models, as illustrated in Figure \ref{fig:pipeline}.

\textbf{Step 1: Choose the adversarial seed.} The adversarial seed is the input sentence to our tree autoencoder. After adding perturbation on the tree node embedding, the decoded adversarial sentence will be added to the original paragraph to perform concat attack. For sentiment classifiers, the adversarial seed can be an arbitrary sentence from the paragraph. For example, the adversarial seed of Yelp Review example in Table \ref{tab:example} is a random sentence from the paragraph \textit{``I kept expecting to see donkeys and chickens walking around.'}

In contrast, when performing answer targeted attack for QA models, we need add our targeted answer into our adversarial seed in a reasonable context.  Based on a set of heuristic experiments on how the adversarial seed correlates the attack efficacy (Appendix \ref{appendix:heuristic}), we choose to use question words to craft an adversarial seed, because it receives higher attention score when the model is matching semantic similarity between the context and the question. 
Specifically, we convert a question sentence to a meaningful declarative statement and assign a targeted fake answer. The fake answer can be crafted according to the perturbed model's predicted answer (position targeted attack \S \ref{two_attacks}), or can be manually chosen by adversaries (answer targeted attack). For instance, the answer targeted attack example shown in Table \ref{tab:example} converts the question \textit{``Who ended the series in 1989?''} into a declarative statement \textit{``someone ended the series in 1989.''} by a set of coarse grained rules (Appendix \ref{appendix:heuristic}).
%\shuo{How? Any reference?} 
Then our targeted wrong answer is assigned to generate the adversarial seed \textit{``Donald Trump ended the series in 1989.''}  Following steps will make sure that the decoded adversarial sentence does not contradict with the original paragraph. 
%The initial seed and the following optimization steps should ensure that the adversarial sentence is meaning preserving and label preserving. The sentence in the example above is simply repeating the paragraph, and thus is valid. 
%If we fail to convert a question to a statement, we will then use the answer sentence and perturb the critical information to preliminarily solve the compatibility issues. 

% if we choose a good adversarial seed that contains the targeted answer is semantically close to the context or the question
 
% it would be helpful to reduce the optimization steps of searching for the best perturbation and achieve targeted 
% For example, when attacking the BERT, we can simply sample a sentence from the original paragraph and append it to the start of the paragraph. 

% Different from attacking sentiment analysis, it is important to choose a good initial seed that is semantically close to the context or the question when attacking QA model. In this way we can reduce the number of iteration steps and attack the QA model more efficiently. 
%Based on a set of heuristic experiments on how the initial seed correlates the attacking efficacy, 

\textbf{Step 2: Embed the discrete text into continuous embedding.} One difference between \advcodecsent and \advcodecword is on which tree level we embed our discrete sentence. For \advcodecsent, we use tree root node embedding of Tree-LSTM $\boldsymbol{z} = \boldsymbol{h}_\text{root}$ to represent the discrete sentence (``ROOT'' node in the Figure \ref{fig:advsent}). As for \advcodecword, we concatenate all the leaf node embedding of Tree-LSTM $\boldsymbol{h}_i$ (corresponding to each word) $\boldsymbol{z} = [\boldsymbol{h}_1, \boldsymbol{h}_2, \dots, \boldsymbol{h}_n]$ to embed the discrete sentence.


% As illustrated in Figure \ref{fig:qa_pipeline}, both attacks start from an initial seed (sentence or tokens). The initial seed is later fed into the tree autoencoder. For \advcodecsent, it encodes through the encoder $\mathcal{E}$ and uses the root node embedding as the sentence representation $z$. We add perturbation $z^*$ on $z$ and propagate $z$ regularized by the tree decoder back to adversarial sentences. \advcodecword follows the same tree encoder but stops at the leaf level. The sentence leaf token embedding is then concatenated as the context vector $z$. After adversarial perturbation $z' = z + z^*$, the leaf node embedding $z'$ is then decoded into words via equation (\ref{eq:decode_word}). 

% Therefore, it is worth noting that because the encoding and decoding phases are completed in the leaf nodes without propagating through the dependency trees,
%while \advcodecsent can help regularize the syntactical correctness based on the tree decoder,
% \advcodecword does not guarantee the grammatical correctness of the adversarial sentences.
%We can also observe from Table \ref{examples} that the adversarial sentence quality generated by \advcodecword is worse than those generated by \advcodecsent. 
% But the advantage of \advcodecword lies on the flexibility, which can help us selectively perturb a very limited number of words instead of paraphrasing the whole sentence.

\textbf{Step 3: Perturb the embedding via optimization.} Finding the optimal perturbation $z^*$ on the embedding vector $z$ is equivalent to solving the optimization problem that can achieve the target attack goal while minimize the magnitude of perturbation
\begin{equation}
    \min \quad ||\boldsymbol{z}^*||_p + c  f(\boldsymbol{z} + \boldsymbol{z}^*),
    \label{cw}
\end{equation}
where $f$ is the objective function for the targeted attack and $c$ is the constant balancing between the perturbation magnitude and attack target. Specifically, we design the objective function $f$ similar to \citet{cw} for classification tasks
\begin{align}
        \ell &=  \max\big\{Z\left({\left[\mathcal{G}(\boldsymbol{z}', \boldsymbol{s}); \boldsymbol{x} \right] }\right)_i:i \neq t \big\},  \\
        f(\boldsymbol{z}') &= \max \big(\ell - Z\left(\left[\mathcal{G}(\boldsymbol{z}', \boldsymbol{s}); \boldsymbol{x} \right] \right)_t  , -\kappa \big),
\end{align}
where $\boldsymbol{z}'=\boldsymbol{z}+\boldsymbol{z}^*$ is the perturbed embedding, model input $\left[\mathcal{G}(\boldsymbol{z}', \boldsymbol{s}); \boldsymbol{x} \right]$ is the concatenation of adversarial sentence $\mathcal{G}(\boldsymbol{z}', \boldsymbol{s})$ and original paragraph $\boldsymbol{x}$, $t$ is the target class, $Z(\cdot)$ is the logit output of the classification model before softmax, $\ell$ is the maximum logits of the classes other than the targeted class and $\kappa$ is the confidence score to adjust the misclassification rate. The confidence score $\kappa$ is chosen via binary search to search for the tradeoff-constant between attack success rate and meaning perseverance. The optimal solution $z^*$ is iteratively optimized via gradient descent.

Similarly to attack QA models, we subtly change the objective function $f$ due to the difference between QA model and classification model:
\begin{align*}
        \ell_j &=  \max\big\{Z_j\left({\left[\boldsymbol{x}; \mathcal{G}(\boldsymbol{z}', \boldsymbol{s}) \right] }\right)_i:i \neq t_j \big\},  \\
        f(\boldsymbol{z}') &= \sum_{j=1}^{2} \max \big(\ell_j - Z_j\left(\left[\boldsymbol{x}; \mathcal{G}(\boldsymbol{z}', \boldsymbol{s}) \right] \right)_{t_j}, -\kappa \big),
\end{align*}
where $Z_1(\cdot)$  and $Z_2(\cdot)$ are respectively the logits of answer starting position and ending position of the QA system. $t_1$ and $t_2$ are respectively the targeted start position and the targeted end position.  $\ell_j$ is the maximum logits of the positions other than the targeted positions. Different from attacking sentiment classifier where we prepend the adversarial sentence, we choose to follow the setting of \citeauthor{jia-liang-2017-adversarial} to add the adversary to the end of the paragraph so that we can make a fair comparison with their results.

\begin{table*}[t!] \small
\centering
\begin{tabular}{ccccc|ccc}
\toprule
\multirow{2}{*}{Model} & Original & & \multicolumn{2}{c}{Whitebox Attack} & \multicolumn{3}{c}{Blackbox Attack} \\
\cmidrule(lr){4-5} \cmidrule(lr){6-8}
& Acc & & {\advcodecword} & {Seq2Sick}  & {\advcodecword} & {Seq2sick}  & TextFooler\\
\midrule
\multirow{2}{*}{BERT} & \multirow{2}{*}{0.703} & target   & \textbf{0.990}         & 0.974   & \textbf{0.499}         & 0.218  & 0.042         \\
    &  & untarget  & \textbf{0.993}          & 0.988  & \textbf{0.686}          & 0.510  & 0.318   \\
\midrule
\multirow{2}{*}{SAM} & \multirow{2}{*}{0.704} & target       & \textbf{0.956}   & 0.933  & \textbf{0.516}   & 0.333  & 0.113  \\
     & & untarget      & \textbf{0.967}          & 0.952  & \textbf{0.669} & 0.583  & 0.395    \\
\bottomrule
\end{tabular}
\caption{Adversarial evaluation on sentiment classifiers in terms of targeted and untargeted attack success rate. }
\label{tab:AttackSentiment}
% \vspace{-3mm}
\end{table*}

\textbf{Step 4: Decode back to adversarial sentence.} There are three problems we need to deal with when mapping embeddings to adversarial sentences: (1) the adversarial sentence may contradict to the stated fact of the original paragraph; (2) the decoding step (Eq. \ref{eq:decode_word}) uses argmax operator that gives no gradients,  but the step 3 needs to perform gradient descent to find the optimal $z^*$; (3) for answer targeted attack, the targeted answer might be perturbed and changed during decoding phase.

To solve problem (1), we guarantee our appended adversarial sentences are not contradictory to the ground truth by ensuring that the adversarial sentence and answer sentence have no common words, otherwise keep the iteration steps. If the maximum steps are reached, the optimization is regarded as a failure. 

For problem (2), during optimization we use a continuous approximation based on softmax with a decreasing temperature $\tau$ \citep{Hu2017TowardCG} 
\begin{equation} \label{eq_approx}
    y^*_j \thicksim \text{softmax}((\boldsymbol{W}\cdot \boldsymbol{h_j} + \boldsymbol{b})/\tau).
\end{equation}
to make the optimization differentiable. After finding the optimal perturbation $z^*$, we still use the hard argmax to generate the adversarial texts.

As for problem (3), we keep targeted answers unmodified during the optimization steps by setting gates to the targeted answer span: $y_j \leftarrow  g_1 \odot y_j + g_2 \odot x_j, (j = t_1, t_1+1,... ,t_2)$, where $y_j$ are the adversarial tokens decoded by tree. We set $g_1 = 1$ and $g_2 = 0$ in the position targeted attack, and $g_1=0$ and $g_2=1$ in the answer targeted attack.

\iffalse

% For the position targeted attack mentioned in \S \ref{two_attacks}, we expect the model output to be a span in the paragraph between the targeted start position $t_1$ and the targeted end position $t_2$. In contrast,

% Unlike \citet{jia-liang-2017-adversarial} who uses complicated rules to ensure the adversarial sentence does not change the ground truth, this heuristic step is the very first step of our framework followed by a series of optimization steps to ensure the ground truth is not changed. In this paper, 

%It is also worth noting the append position does have a influence on the attack success rate for adversarial attack, and more detailed ablation analysis will be discussed next.

% \vspace{-3mm}
\subsection{AdvCodec(Sent)}
In this section, we explain how to utilize \advcodecsent to attack NLP models, as illustrated in Figure \ref{fig:pipeline}.
\subsubsection{Attacking Sentiment Classification Model}
\quad \newline
\textbf{Initial Seed.} Following our pipeline (Figure \ref{fig:qa_pipeline}), we need to first start with an initial seed (tokens or sentences). Such initial seed for sentiment classification task can be an arbitrary sentence from the paragraph. For example, when attacking the BERT, we can simply sample a sentence no shorter than 3 words from the original paragraph and append it to the start of the paragraph. The initial seed and the following optimization steps should ensure that the adversarial sentence is meaning preserving and label preserving. The sentence in the example above is simply repeating the paragraph, and thus is valid. It is also worth noting the append position does have an influence on the attack success rate for adversarial attack, and more detailed ablation analysis will be discussed next.

\textbf{Optimization Procedure.}  Finding the optimal perturbation $z*$ on context vector $z$ is equivalent to solving the optimization problem
% that can achieve the target attack goal while control the magnitude of perturbation
\begin{equation}
    \text{minimize} \quad ||z^*||_p + c  f(z + z^*),
    \label{cw}
\end{equation}
where $f$ is the objective function for the targeted attack and $c$ is the constant balancing between the perturbation magnitude and attack target. Specifically, we use the objective function $f$ proposed in \citet{cw} for classification task
\begin{equation}
        f(z') = \text{max}(\text{max}\{Z(\mathcal{G}(z', s))_i:i \neq t\} - Z(\mathcal{G}(z', s))_t, -\kappa)
\end{equation}
where $z'=z+z^*$, $t$ is the target class, $Z(\cdot)$ is the logit output of the classification model before softmax and $\kappa$ is the confidence score to adjust the misclassification rate. The confidence score $\kappa$ is chosen via binary search to search for the tradeoff-constant between attack success rate and meaning perseverance. The optimal solution $z^*$ is iteratively optimized via gradient descent.


%We set $l_p$ norm to be $l_1$ norm according to \citet{2018arXiv180301128C}. The local optimal solution is found by using stochastic gradient descent.

\subsubsection{Attacking Question Answering System}
\quad \newline
\textbf{Initial Seed.} Different from attacking sentiment analysis, it is essential to choose a good initial seed that is semantically close to the context or the question when attacking QA model. In this way we can reduce the number of iteration steps and attack the QA model more efficiently. Based on a set of heuristic experiments on how the initial seed correlates the attacking efficacy, we choose to use question words to craft an adversarial seed, because it receives higher attention score when the model is matching semantic similarity between the context and the question. 
We design a set of coarse grained rules to convert a question sentence to a meaningful declarative statement and assign a target fake answer. The fake answer can be crafted according to the perturbed model's predicted answer, or can be manually chosen by adversaries. 
%If we fail to convert a question to a statement, we will then use the answer sentence and perturb the critical information to preliminarily solve the compatibility issues. 
As for the location where we append the sentence, we choose to follow the setting of \citeauthor{jia-liang-2017-adversarial} to add the adversary to the end of the paragraph so that we can make a fair comparison with their results.

Unlike \citet{jia-liang-2017-adversarial} who uses complicated rules to ensure the adversarial sentence does not change the ground truth, this heuristic step is the very first step of our framework followed by a series of optimization steps to ensure the ground truth is not changed. In this paper, we guarantee our appended adversarial sentences are not contradictory to the ground truth by 1) choosing an initial sentence as the initial seed of optimization, 2) adding perturbation to the sentence, 3) searching for the optimal adversarial sentence, 4) ensuring that the adversarial sentence and context sentence are disjoint, otherwise keep the iteration steps. If the maximum steps are reached, the optimization is regarded as a failure. 
%our question-based initial sentence is simply generated by simple rules and only serves as a good starting point for the following optimization. It is our \advcodec's responsibility to automatically search the best adversarial sentence that could both achieve the targeted attack and solve the compatibility issues. 

\textbf{Optimization Procedure.} 

% introduce tree AE

% attack sentiment

% attack qa
% objective functions -- targeted/untargeted
% how to initialize good sentence (for targted xxxx)


\subsection{AdvCodec(Word)}

Not only can we apply perturbations to the root node of our tree-based autoencoder to generate adversarial sentence, we can also perturb nodes at different hierachical levels of the tree to generate adversarial word. The most general case is that the perturbation is directly exerted on the leaf node of the tree autoencoder, i.e. the word-level perturbation. 

\advcodecword shares the exactly same tree autoencoder architectures and optimization steps mentioned above to attack the targeted models. The distinction between \advcodecword and \advcodecsent is the context vector $z$. Formally for the word-level attack, the context vector $z$ are the concatenation of leaf node embedding $z_i$ (which corresponds to each word) $z = [z_1, z_2, …, z_n]$.
% \begin{equation}
% \end{equation}
Different from the \advcodecsent that perturbation is added on the whole sentence, we can control where the perturbations are added by assigning each node a mask as follows:
\begin{equation}
    z_i' = z_i + \text{mask} \cdot z_i^*
\end{equation}
 When we expect some token $z_i$ to be adversarially changed, we can simply assign $\text{mask} = 1$, thus adding the perturbation on the token. 
 
 As the perturbation can be controlled on individual words, we propose a new attack scenario \textit{scatter attack}, which scatters some initial tokens over the paragraph, adds perturbation only to those tokens and find the best adversarial tokens via the same optimization procedure mentioned above. Moreover, the concatenative adversarial examples (e.g. generated by \advcodecsent) can also be crafted by \advcodecword because the concat attack is a special case for the scatter attack.

\fi
 

% introduce word AE
% mask, sentence scatter, objectives



\section{Experiments and Discussions\label{sec:Experiments}}

\section{Experiments}
We conduct experiments on two ultra-fine entity typing datasets, {\bf \textsc{UFET}} (English) and {\bf \textsc{CFET}} (Chinese). Their data statistics are shown in Table \ref{tab:stat}. We mainly focus on and report the macro-averaged recall at the recall and expand stage, and concern mainly on the macro-$F1$ of the final prediction at the filter stage. We also evaluate the {\bf \textsc{\name}} models on the fine-grained (130 types) and coarse-grained (9 types) settings of entity typing without the recall and expand stage.
\subsection{UFET and CFET}
\subsubsection{Recall Stage}
\label{sec:recall}
We compare the recall@$K$ on the test sets of {\bf \textsc{UFET}} and {\bf\textsc{CFET}} between the trained MLC model (introduced in \ref{sec:mlc}) and a traditional BM25 model \cite{bm25} in Figure \ref{fig:recall}. The MLC model uses the RoBERTa-large as backbone and is tuned based on the recall@$128$ on the development set. We use AdamW optimizer with a learning rate of $2\times10^{-5}$. Results show that MLC is a strong recall model, it consistently has better recall compared to BM25 on both {\bf\textsc{UFET}} and {\bf\textsc{CFET}} dataset, and the recall@$128$ reaches over $85\%$ on {\bf \textsc{UFET}}, and over $94\%$ on {\bf \textsc{CFET}}.

\begin{figure}[t]
     \centering
     \begin{subfigure}[h]{0.5\textwidth}
         \centering
         \includegraphics[width=\textwidth]{src/img/recall_compare_bm25.pdf}
         \label{fig:mb2}
     \end{subfigure}   
 \caption{Recall@$K$ of MLC and BM25.}
 \label{fig:recall}
\end{figure}

\subsection{Expand Stage}
\label{sec:expand}
In Table \ref{tab:expand}, we evaluate the F1 scores of all candidates expanded by exact match, and top-$10$ candidates expanded by the MLM using Bert-large. We also demonstrate the improvement of recall by using candidate expansion in Figure \ref{fig:expand_improvement}. On {\bf \textsc{UFET}} dataset, expanding around $32$ additional candidates based on $112$ MLC candidates results in $2\%$ higher recall compared to recalling all $128$ candidates by MLC. The recall of $128$ candidates after the expansion is comparable to the recall of $180$ candidates recalled from MLC. Similarly, expanding $10$ candidates is comparable to additionally recalling $80$ candidates using MLC.
In our experiments, we replace the last $48$ candidates recalled by MLC with the candidates recalled by MLM and Exact match for {\bf \textsc{UFET}} and $10$ for {\bf \textsc{CFET}}. We found the expand stage has a positive effect on the final performance of {\bf \textsc{\name}}s, and helps them reach SOTA performance (analyze in Sec. \ref{sec:analyze}).


\begin{table}[t]
\centering
\scalebox{0.75}{
\begin{tabular}{cccccc} 
\toprule
{\bf \textsc{Dataset}} & {\bf \textsc{Expand}} &   {\bf \textsc{P}}  & {\bf \textsc{R}}  &  {\bf \textsc{F1}} & \small{Avg \# Expanded}  \\ \midrule
\multirow{2}{*}{\bf \textsc{UFET}} & {\bf \textsc{Match}}      & 11.2   & 11.3     & 9.8    & 5.23     \\
      & {\bf \textsc{MLM}}  &  8.5     &   17.1   &  10.7  &    10    \\ \midrule
\multirow{2}{*}{\bf \textsc{CFET}} & {\bf \textsc{Match}}   &  11.4  &  14.5  & 11.2   & 4.57    \\
 & {\bf \textsc{MLM}}  & 21.3   &  19.5  & 17.7    & 10    \\ \midrule
\end{tabular}}
\caption{Evaluation of the recalled candidates.}
\label{tab:expand}
\end{table}
\begin{figure}[t]
     \centering
     \begin{subfigure}[h]{0.45\textwidth}
         \centering
         \includegraphics[width=\textwidth]{src/img/recall_ufet.pdf}
         \caption{Recall@$128$ on {\bf \textsc{UFET}} by including different number of expanded candidates. }
         \label{fig:c1}
     \end{subfigure}
     \vfill
     \begin{subfigure}[h]{0.45\textwidth}
         \centering
         \includegraphics[width=\textwidth]{src/img/recall_cfet.pdf}
         \caption{Recall@$64$ on {\bf \textsc{CFET}} by including different number of expanded candidates.}
         \label{fig:c2}
     \end{subfigure}
\caption{Demonstration of the effect of expand stage. $x$-axis represents the number of candidates expanded by MLM/MLM+MATCH among these $128$ candidates. }
\label{fig:expand_improvement}
\end{figure}
\label{sec:exp_expand}
\subsection{Filter Stage and Final Results.}
\begin{table}[h!]
\centering
\scalebox{0.73}{
\renewcommand{\arraystretch}{1}
\begin{tabular}{cllll} \toprule
\multicolumn{2}{l}{\bf \textit{Base Models on UFET} }     & \bf \textsc{P}    & \bf \textsc{R}   & \bf \textsc{F1}  \\ \midrule
\multicolumn{5}{l}{\emph{MLC-like models}}        \\
\color{blue} \bf \texttt{B}& {\bf \textsc{Box4Types}}\cite{box4types}  & 52.8 & 38.8 & 44.8  \\
\color{blue}\bf \texttt{B}& {\bf \textsc{LDET}}$^\dagger$  \cite{onoe-durrett-2019-learning}          & 51.5 & 33.0 & 40.1 \\ 
\color{blue}\bf \texttt{B}& {\bf \textsc{MLMET}}$^\dagger$   {\cite{mlmet}}   & 53.6 & 45.3 & 49.1  \\
\color{blue}\bf \texttt{B}& {\bf \textsc{PL}}  \cite{ding2021prompt}   & 57.8 & 40.7 & 47.7 \\
\color{blue}\bf \texttt{B}& {\bf \textsc{DFET}}    \cite{dfet}      & 55.6 & 44.7 & 49.5 \\
\color{blue}\bf \texttt{B}& {\bf \textsc{MLC}} (reimplemented by us) & 46.5 & 34.9 & 39.9 \\ 
\color{red}\bf \texttt{R}& {\bf \textsc{MLC}} (reimplemented by us) & 42.2 & 44.9 & 43.5 \\ \hline 
\multicolumn{5}{l}{\emph{Seq2seq based models}}      \\
\color{blue}\bf \texttt{B} & {\bf \textsc{LRN} }  {\cite{liu-etal-2021-fine}}              & 54.5 & 38.9 & 45.4  \\\hline
\multicolumn{5}{l}{\emph{Filter models under our recall-expand-filter paradigm}}      \\
\color{blue}\bf \texttt{B} & {\bf \textsc{Vanilla CE}$_{128}$}   & 47.2 & 48.5 & 47.8 \\ 
\color{blue}\bf \texttt{B} & {\bf \textsc{\name-S$_{128}$}} (Ours)  & 53.2 & 48.3 & {\bf 50.6} \\ 
\color{blue}\bf \texttt{B} & {\bf \textsc{\name-S$_{128}$ w/o C2C}}   (Ours)   & 52.3 & 48.3 & 50.2 \\ 
\color{blue}\bf \texttt{B} & {\bf \textsc{\name-B$_{128}$}} (Ours)    & 49.9 & 50.0 & 49.9 \\ 
\color{blue}\bf \texttt{B} & {\bf \textsc{\name-B$_{128}$ w/o C2C}} (Ours)     & 49.9 & 48.2 & 49.0 \\ \hline
\color{red}\bf \texttt{R} & {\bf \textsc{Vanilla CE}$_{128}$}   & 49.6 & 49.0 & 49.3 \\ 
\color{red}\bf \texttt{R} & {\bf \textsc{\name-S$_{128}$}} (Ours)  & 53.3 & 47.3 & 50.1 \\ 
\color{red}\bf \texttt{R} & {\bf \textsc{\name-S$_{128}$ w/o C2C}}   (Ours)  & 53.2 & 46.6 & 49.7 \\ 
\color{red}\bf \texttt{R} & {\bf \textsc{\name-B$_{128}$}} (Ours)  & 52.5 & 47.9 & 50.1 \\ 
\color{red}\bf \texttt{R} & {\bf \textsc{\name-B$_{128}$ w/o C2C}} (Ours)     & 52.7 & 46.4 & 49.3 \\ \hline
\midrule
\multicolumn{2}{l}{\bf \textit{Large Models on UFET} }     & \bf \textsc{P}    & \bf \textsc{R}   & \bf \textsc{F1}  \\ \midrule
\multicolumn{5}{l}{\emph{MLC-like models}}        \\
\color{red}\bf \texttt{R} & {\bf \textsc{MLC}}  \cite{npcrf}               & 47.8 & 40.4 & 43.8  \\
\color{red}\bf \texttt{R} & {\bf \textsc{MLC-NPCRF}} \cite{npcrf}             & 48.7 & 45.5 & 47.0  \\
\color{red}\bf \texttt{R} & {\bf \textsc{MLC-GCN}} \cite{xiong-etal-2019-imposing}     & 51.2 & 41.0 & 45.5 \\
\color{blue}\bf \texttt{B} & {\bf \textsc{PL}}  \cite{ding2021prompt}       & 59.3 & 42.6 & 49.6  \\
\color{blue}\bf \texttt{B} & {\bf \textsc{PL-NPCRF}}  \cite{npcrf}  & 55.3 & 46.7 & {50.6}\\ \hline
\multicolumn{4}{l}{\emph{Cross-encoder based models and {\bf \textsc{\name}}s}}      \\
\color{red}\bf \texttt{R} & {\bf \textsc{LITE+L}}  \cite{lite}             & 48.7 & 45.8 & 47.2  \\
\color{teal}\bf \texttt{RM} & {\bf \textsc{LITE+NLI+L}} \cite{lite} & 52.4 & 48.9 & {50.6} \\ \hline
\multicolumn{4}{l}{\emph{Filter models under our recall-expand-filter paradigm}}   \\ 
\color{blue}\bf \texttt{B} & {\bf \textsc{Vanilla CE$_{128}$}}   & 50.3 & 49.6 & 49.9 \\ 
\color{blue}\bf \texttt{B} & {\bf \textsc{\name-S$_{128}$}}  (Ours)   & 52.5 & 49.1 & 50.8 \\ 
\color{blue}\bf \texttt{B} & {\bf \textsc{\name-S$_{128}$ w/o C2C}}   (Ours)   & 54.1 & 47.1 & 50.4 \\ 
\color{blue}\bf \texttt{B} & {\bf \textsc{\name-B$_{128}$}} (Ours)    & 54.0 & 48.6 & 51.2 \\ 
\color{blue}\bf \texttt{B} & {\bf \textsc{\name-B$_{128}$ w/o C2C}} (Ours)     & 52.8 & 48.3 & 50.4 \\ \hline
\color{red}\bf \texttt{R} & {\bf \textsc{Vanilla CE$_{128}$}}   & 54.5 & 49.3 & 51.8 \\ 
\color{red}\bf \texttt{R} & {\bf \textsc{\name-S$_{128}$}}  (Ours)   & 50.8 & 49.8  &  50.3 \\ 
\color{red}\bf \texttt{R} & {\bf \textsc{\name-S$_{128}$ w/o C2C}}   (Ours)   & 51.5 & 48.8 & 50.1 \\ 
\color{red}\bf \texttt{R} & {\bf \textsc{\name-B$_{128}$}} (Ours)    & 51.9 & 50.8 & 51.4 \\ 
\color{red}\bf \texttt{R} & {\bf \textsc{\name-B$_{128}$ w/o C2C}} (Ours)     & 51.6 & 51.6 & 51.6 \\ \hline
\color{teal}\bf \texttt{RM} & {\bf \textsc{\name-B$_{128}$ w/o C2C}} (Ours) & 56.3 & 48.5 & {\bf 52.1} \\ \hline
\midrule
\end{tabular}}
\caption{Macro-averaged UFET result. {\bf \textsc{LITE+L}} is LITE without NLI pretraining, {\bf \textsc{LITE+L+NLI}} is the full LITE model. Methods marked by $\dagger$ utilize either distantly supervised or augmented data for training. {\bf \textsc{\name-S$_{128}$}} denotes we use $128$ candidates recalled and expanded from the first two stages.}
\label{tab:ufet}
\end{table}
\begin{table}[t]
\centering
\scalebox{0.75}{
\renewcommand{\arraystretch}{1}
\begin{tabular}{cllll} \toprule
\multicolumn{2}{l}{\bf \textit{Models on CFET} }     & \bf \textsc{P}    & \bf \textsc{R}   & \bf \textsc{F1}  \\ \midrule
\multicolumn{5}{l}{\emph{MLC-like models}}        \\
\color{purple}\bf \texttt{N}& {\bf \textsc{MLC}} & 55.8 & 58.6 & 57.1 \\  
\color{purple}\bf \texttt{N}& {\bf \textsc{MLC-NPCRF}} \cite{npcrf}     & 57.0 & 60.5 & 58.7 \\ 
\color{purple}\bf \texttt{N}& {\bf \textsc{MLC-GCN}} \cite{xiong-etal-2019-imposing}   & 51.6 & 63.2 & 56.8 \\ 
\color{brown}\bf \texttt{C}& {\bf \textsc{MLC}} & 54.0 & 59.5 & 56.6 \\  
\color{brown}\bf \texttt{C}& {\bf \textsc{MLC-NPCRF}} \cite{npcrf}   & 54.0 & 61.6 & 57.3 \\  
\color{brown}\bf \texttt{C}& {\bf \textsc{MLC-GCN}} \cite{xiong-etal-2019-imposing} & 56.4 & 58.6 & 57.5 \\ \midrule 
\multicolumn{5}{l}{\emph{Filter models under our recall-expand-filter paradigm}}      \\
\color{purple}\bf \texttt{N} & {\bf \textsc{Vanilla CE}}   & 57.6 & 64.3 & 60.7 \\ 
\color{brown}\bf \texttt{C} & {\bf \textsc{Vanilla CE}}   & 54.0 & 63.3 & 58.3 \\  \hline
\color{purple}\bf \texttt{N} & {\bf \textsc{\name-S$_{64}$}} (Ours)  & 58.4 & 62.1 & 60.2 \\ 
\color{purple}\bf \texttt{N} & {\bf \textsc{\name-S$_{64}$ w/o C2C}}   (Ours)   & 59.1 & 61.5 & 60.3 \\ 
\color{purple}\bf \texttt{N} & {\bf \textsc{\name-B$_{64}$}} (Ours)    & 56.7 & 66.1 & 61.1 \\ 
\color{purple}\bf \texttt{N} & {\bf \textsc{\name-B$_{64}$ w/o C2C}} (Ours)     & 58.8 & 64.1 & 61.4 \\ \hline
\color{brown}\bf \texttt{C} & {\bf \textsc{\name-S$_{64}$}} (Ours)  & 55.5 & 62.6 & 58.8 \\ 
\color{brown}\bf \texttt{C} & {\bf \textsc{\name-S$_{64}$ w/o C2C}}   (Ours)   & 54.0 & 63.4 & 58.3 \\ 
\color{brown}\bf \texttt{C} & {\bf \textsc{\name-B$_{64}$}} (Ours)    & 55.0 & 63.5 & 59.0 \\ 
\color{brown}\bf \texttt{C} & {\bf \textsc{\name-B$_{64}$ w/o C2C}} (Ours)     & 57.3 & 61.3 & 59.3 \\ \hline
\midrule
\end{tabular}}
\caption{Macro-averaged CFET result.}
\label{tab:cfet}
\end{table}

In this section, we report the performance of {\bf \textsc{MCCE}} variants as the filter models and compare them with various strong baselines that we will introduce later. We also compare the inference speed of different models in this section. For filter models, we treat the number of candidates $K$ recalled and expanded by the first two stages as hyper-parameters, and tune it on the development set. We found the choice of PLM backbones has a non-negligible effect on the performance, and the PLM backbone of previous methods varies. Therefore for fairer comparisons to baselines, we conduct experiments of {\bf \textsc{\name}} using different backbone PLMs for our {\bf \textsc{\name}} models and report the results. For all {\bf \textsc{\name}} models, we use AdamW optimizer with a learning rate tuned between $5\times 10^{-6}$ and $2\times 10^{-5}$. The batch size we use is $4$ and we train the models for at most $50$ epochs with early stopping. {\bf \textsc{UFET}} also provides a large dataset obtained from distant supervision such as entity linking, we do not use it and only train and evaluate our models on human-labeled data.
\paragraph{Baselines}
The {\bf \textsc{MLC}} model we used for the recall stage and the cross-encoder ({\bf \textsc{CE}}) we introduced in Sec. \ref{sec:vanilla_ce} are natural baselines. We also compare our methods with recent PLM-based methods. {\bf \textsc{LDET} }\cite{onoe-durrett-2019-learning} is an MLC with Bert-base-uncased and ELMo \cite{elmo} trained on 727k examples automatically denoised from the distantly labeled UFET. {\bf \textsc{GCN} }\cite{xiong-etal-2019-imposing} uses GCN to model type correlations and obtain type embeddings. Types are scored by dot-product of mention and type embeddings. The original paper uses BiLSTM as the mention encoder and we use the results re-implemented by \citet{npcrf} using RoBERTa-large. {\bf \textsc{Box4Type} }\cite{box4types} uses Bert-large as the backbone and uses box embedding to encode mentions and types for training and inference. {\bf \textsc{LRN} }\cite{liu-etal-2021-fine} use Bert-base as the encoder and an LSTM decoder to generate types in a seq2seq manner. {\bf \textsc{MLMET} }\cite{mlmet} is a {\bf \textsc{MLC}} with Bert-base, but first pretrained by the distantly-labeled data augmented by masked word prediction, then finetuned and self-trained on the 2k human-annotated data. {\bf \textsc{PL}} \cite{ding2021prompt} uses prompt learning for entity typing. {\bf \textsc{DFET} }\cite{dfet} uses {\bf \textsc{PL}} as backbone and is a multi-round automatic denoising method for 2k labeled data. {\bf \textsc{LITE} }\cite{lite} is the previous SOTA system that formulates entity typing as textual inference. {\bf \textsc{LITE}} uses RoBERTa-large-MNLI as the backbone, and is a cross-encoder (introduced in Sec. \ref{sec:vanilla_ce}) with designed templates and a hierarchical loss. \citet{npcrf} proposes {\bf \textsc{NPCRF}} to enhance backbones such as {\bf \textsc{PL}} and {\bf \textsc{MLC}} by modeling type correlations, and reach performance comparable to {\bf \textsc{LITE}}.

\paragraph{Naming Conventions}
Let {\bf \textsc{\name-S}} be the {\bf \textsc{\name}} model that treats candidates as sub-tokens, and {\bf \textsc{\name-B}} be the model representing candidates as fixed-size blocks. The {\bf \textsc{\name}} model without {\bf \textsc{C2C}} attention (mentioned in Sec. \ref{sec:attn}) is denoted as {\bf \textsc{\name-B} w/o C2C}. For PLM backbones used in {\bf \textsc{UFET}}, we use {\color{blue} \bf \texttt{B}}, {\color{red} \bf \texttt{R}}, {\color{teal} \bf \texttt{RM}} to denote BERT-base-cased \cite{bert}, RoBERTa \cite{liu2019roberta}, and RoBERTa-MNLI \cite{liu2019roberta} respectively. For {\bf \textsc{CFET}}, we adopt two widely-used Chinese PLM, BERT-base-Chinese and NeZha-base-Chinese, and denote them as {\color{brown} \bf \texttt{C}} and {\color{purple} \bf \texttt{N}} respectively. 

\paragraph{UFET Results} We show the results of {\bf \textsc{UFET}} dataset in Table \ref{tab:ufet}. The results show that: (1) The recall-expand-filter paradigm is effective. Filter models outperform all baselines without the paradigm by a large margin. The vanilla CE under our paradigm reaches $51.8$ F1 compared to more complexed CE {\bf \textsc{LITE}} with $50.6$ F1 (2) {\bf \textsc{\name}} models reach SOTA performances. {\bf \textsc{\name-S$_{128}$}} with BERT-base performs best and reaches {\bf 50.6} F1 score, which is comparable to previous SOTA performance of large models such as {\bf \textsc{LITE+NLI+L}} and {\bf \textsc{PL+NPCRF}}. Among large models, {\bf \textsc{\name-B$_{128}$ w/o C2C}} also reaches SOTA performance with {\bf 52.1} F1 score. (3) {\bf \textsc{C2C}} attention is not necessary on large models, but is useful in base models. (4) Large models can utilize type semantics better. We found {\bf \textsc{\name-B}} outperforms {\bf \textsc{\name-S}} on large models, but underperforms {\bf \textsc{\name-S}} on base models. (5) Backbone PLM matters. We found the performance of {\bf \textsc{Vannila CE}} under our paradigm is largely affected by the PLM it used. It reaches $47.8$ F1 with BERT-base and $51.8$ F1 with RoBERTa-large. For {\bf \textsc{\name}} models, we found {\bf \textsc{\name}} performs better than {\bf \textsc{\name-B}} with BERT, and worse than {\bf \textsc{\name-B}} with RoBERTa. 

\begin{table*}[t]
\centering
\scalebox{0.9}{
\begin{tabular}{lllcc} \toprule
\bf \textsc{Model}  & \bf \textsc{\# FP} & \bf \textsc{Attn} & \bf \textsc{sents/sec} & \bf \textsc{F1} \\ \midrule
{\bf \textsc{MLC}} & \small{$1$}  & \small{$L_S^2D$} & 58.8 & 43.8\\
{\bf \textsc{LITE+NLI+L (CE)}}  & \small{$N$}  & \small{$L_S^2D$} & 0.02 & 50.6\\ \midrule \hline
\multicolumn{5}{l}{\emph{filter stage inference speed.}}  \\
{\bf \textsc{Vanilla CE$_{128}$}}  & \small{$128$}  & \small{$L_S^2D$} & 1.64 & 51.8 \\ 
{\bf \textsc{\name-S$_{128}$}}  & \small{$1$}  & \small{$(L_S+128)^2D$} & 60.8 & 50.1 \\ 
{\bf \textsc{\name-B$_{128}$}}  & \small{$1$}  & \small{$(L_S+128B)^2D$} & 22.3 & 51.4\\ 
{\bf \textsc{\name-B$_{128}$ w/o C2C}}  & \small{$1$}  & \small{$(L_S^2+256L_S B + 128 B^2)D$} & 25.2 & {\bf 52.1}\\ \bottomrule
\end{tabular}}
\caption{Inference speed comparison of models. {\bf \textsc{\# FP}} means the number of PLM forward passes required by a single inference. {\bf \textsc{ATTN}} column lists the theoretical attention complexity.  We also report the practical inference speed {\bf \textsc{sents/sec}} and the {\bf \textsc{F1}} scores on {\bf \textsc{UFET}} with RoBERTa-large architecture.}
\label{tab:speed}
\end{table*}

\begin{table}[t]
\centering
\scalebox{0.85}{
\renewcommand{\arraystretch}{1}
\begin{tabular}{cllll} \toprule
\multicolumn{2}{l}{\bf \textit{Models} }     & \bf \textsc{P}    & \bf \textsc{R}   & \bf \textsc{F1}  \\ \midrule
\multicolumn{5}{l}{\emph{coarse (9 types) Open Entity}}        \\ \hline
\color{red}\bf \texttt{R} & {\bf \textsc{MLC}}   & 76.8 & 78.5 & 77.6 \\ 
\color{red}\bf \texttt{R} & {\bf \textsc{Vanilla CE$_{9}$}}   & 82.3 & 81.0 & 81.6 \\ 
\color{red}\bf \texttt{R} & {\bf \textsc{\name-S$_{9}$}}   & 77.0 & 87.7 & 82.0 \\ 
\color{red}\bf \texttt{R} & {\bf \textsc{\name-B$_{9}$ w/o C2C}}   & 77.2 & 85.4 & 81.1 \\ \hline
\multicolumn{5}{l}{\emph{fine (130 types)}}        \\ \hline
\color{red}\bf \texttt{R} & {\bf \textsc{MLC}}   & 70.4 & 63.7 & 66.9  \\ 
\color{red}\bf \texttt{R} & {\bf \textsc{Vanilla CE}$_{130}$}   & 67.9 & 66.4 & 67.1 \\ 
\color{red}\bf \texttt{R} & {\bf \textsc{\name-S$_{130}$}}   & 65.8 & 71.8 & 68.7 \\ 
\color{red}\bf \texttt{R} & {\bf \textsc{\name-B$_{130}$ w/o C2C}}   & 64.1 & 70.5 & 67.1 \\ \hline
\midrule
\end{tabular}}
\caption{Micro-averaged results on UFET fine and coarse.}
\label{tab:ufet-coarse-fine}
\end{table}

\paragraph{CFET Results} We conduct experiments on {\bf \textsc{CFET}} and compare {\bf \textsc{\name}} models with several strong baselines:  {\bf \textsc{NPCRF}} and {\bf \textsc{GCN}} with MLC-like architecture, and {\bf \textsc{Vanilla CE}} under out paradigm which is proved to be better than {\bf \textsc{LITE}} on {\bf \textsc{UFET}}. The results are shown in Table \ref{tab:cfet}. Similar to results in {\bf \textsc{UFET}}, filter models under our paradigm significantly outperform MLC-like baselines, $+2.0$ F1 for Nezha-base and $+1.8$ F1 for BERT-base-Chinese. In {\bf \textsc{CFET}}, {\bf \textsc{\name}-B} is significantly better than {\bf \textsc{\name}-S}, on both Nezha-base and BERT-base-Chinese, indicating the importance of type semantics in Chinese language. We also find that {\bf \textsc{\name} w/o C2C} is generally better than  {\bf \textsc{\name} w/ C2C}, it is possibly because the C2C attention distracts the candidates from attending to mention and contexts.
\paragraph{Speed Comparison} Table \ref{tab:speed} shows the theoretical inference complexity (number of PLM forward passes, and attention complexity), and practical inference speed (number of sentences inferred per second) of different models. We conduct the speed test using NVIDIA TITAN RTX for all models, and the inference batch size is 4.
At the filter stage, the inference speed of {\bf \textsc{\name-S}} is on par with {\bf \textsc{MLC}} (even slightly faster because we don't need to score all types), and is about 40 times faster than {\bf \textsc{Vannila CE}} and thousands of times faster than {\bf \textsc{LITE}}. {\bf \textsc{\name-B w/o C2C}} is not significantly faster than {\bf \textsc{\name-B}} as expected. It's possibly because the computation related to the block attention is not fully optimized by existing deep learning frameworks. The speed advantage of {\bf \textsc{\name-B w/o C2C}} over {\bf \textsc{\name-B}} will be greater with more candidates.


\subsection{Fine-grained and Coarse-grained Entity Typing}
We also conduct experiments on Fine-grained (130-class) and Coarse-grained (9-class, also known as ``Open Entity'') entity typing, and the results are shown in Table \ref{tab:ufet-coarse-fine}. As the type candidate set is much smaller in these settings, we skip the recall and expand stages and directly run the filter models and compare them to baselines. Results show that both {\bf \textsc{\name}-S} and {\bf \textsc{\name}-B} are still better than {\bf \textsc{MLC}} and {\bf \textsc{Vanilla CE}}, and {\bf \textsc{\name}-S} is better than {\bf \textsc{\name}-B} on coarser-grained cases possibly because the coarser-grained types are simpler in surface-forms and {\bf \textsc{\name}-S} will not lose many type semantics.






\section{Conclusion\label{sec:Conclusion}}

\section{Conclusion}
\label{ss: conclusion}

% summary of approach
This paper presents a methodology to evaluate the effectiveness of evasions and its application to studying PDF malware scanners.
Our implementation of the methodology, the Chameleon framework, automatically generates and enriches malicious documents with one or multiple evasions.
We use these documents for an in-depth study of \nbAnalyzers{} PDF scanners and how they are affected by evasions.
More broadly, our methodology can also be used for studying evasions of other malware types, e.g., malicious executables.

% main take-aways
The overall result of our study is cause for concern.
We show that the studied evasions are surprisingly effective in fooling state-of-the-art scanners.
In particular by combining evasions, attackers can bypass modern defenses in both static and dynamic scanners.
Moreover, we find huge variations across scanners, enabling targeted attacks based on evasions picked specifically for a targeted scanner.
All these findings are a call to arms for future work on anti-evasion techniques.

Our work will support future efforts toward improving malware scanners in several ways.
First, the results of our study help security vendors to better understand their vulnerability to specific evasions and to focus their attention on mitigating the most effective evasions.
Second, we are releasing the corpus of malicious, evasive documents generated by Chameleon as a ready-to-use benchmark.
We are in contact with several developers of PDF scanners, and some of them, e.g., SploitGuard and SAFE-PDF, have already used our benchmark to test and improve their security scanners.
Finally, the Chameleon framework provides a basis for expanding the set of benchmarks by incorporating future evasions, exploits, and payloads.


\section*{Acknowledgment}

This research is partly supported by NHMRC Ideas Grant GNT2002234.

\bibliographystyle{splncs04}
\bibliography{reference}

\end{document}

