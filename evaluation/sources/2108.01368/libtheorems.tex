% Code for theorems in multiple places
% Originally by Eric Price, placed in the public domain
%
% This lets you write the statement of a theorem/lemma in one place
% and insert it in multiple parts of a document.  This lets you
% include the statement of Lemma 3.2 in both the introduction and the
% place that it's proven, and the numbering can refer to the place
% where it's proven.
%
% Usage:
%
%  \define[optional title]{thm:main}{Theorem}{We show \[ x^2 + y^2 \geq 2xy \].}
%     Before you use it.
%
%  \state{thm:main}
%     in the place where you want the label to be generated
%
%  \restate{thm:main}
%     elsewhere.
%
%  To have consistent equation labels, use $\thmlabel{eq:1}$ rather
%  than $\label{eq:1}$.
%
%
%%%%% Begin code
\usepackage{etoolbox}

\makeatletter

% Define the various variables to be used in \state/\restate
\newcommand{\define}[4][ignore]{%
  \ifstrequal{#1}{ignore}{}{
  \@namedef{thmtitle@#2}{#1}}%
  \@namedef{thm@#2}{#4}%
  \@namedef{thmtypen@#2}{lemma}%
  \newtheorem{thmtype@#2}[theorem]{#3}%
  \newtheorem*{thmtypealt@#2}{#3~\ref{#2}}%
}

% State the theorem, generating the proper number.
\newcommand{\state}[1]{%
  \@namedef{curthm}{#1}
  \@ifundefined{thmtitle@#1}{
  \begin{thmtype@#1}
    }{
  \begin{thmtype@#1}[\@nameuse{thmtitle@#1}]
  }
    \label{#1}
    \@nameuse{thm@#1}
  \end{thmtype@#1}
  \@ifundefined{thmdone@#1}{
  \@namedef{thmdone@#1}{stated}%
  }{}
}

% Include the theorem, with the original numbering
\newcommand{\restate}[1]{%
  \@namedef{curthm}{#1}
  \@ifundefined{thmtitle@#1}{
    \begin{thmtypealt@#1}
    }{
  \begin{thmtypealt@#1}[\@nameuse{thmtitle@#1}]
  }
    \@nameuse{thm@#1}
  \end{thmtypealt@#1}
  \@ifundefined{thmdone@#1}{
  \@namedef{thmdone@#1}{stated}%
  }{}
}

% For equations, this generates a new number if this is the first time
% the theorem is included, and reuses the old number otherwise.
\newcommand{\thmlabel}[1]{
  \@ifundefined{thmdone@\@nameuse{curthm}}{\label{#1}
    }{\tag*{\eqref{#1}}}
}
\makeatother
%%%%% End code
