

\documentclass{standalone}

% uncomment to preview:

% \usepackage{tikz}
% \usepackage{xcolor}
% \usetikzlibrary{
%     arrows.meta,
%     chains,
%     scopes,
%     positioning,
%     shapes.geometric
% }
% % layers
% \pgfdeclarelayer{background}
% \pgfdeclarelayer{foreground}
% \pgfsetlayers{background,main,foreground}

% \definecolor{baseColour}{RGB}{255, 128, 85}
% \definecolor{ncqColour}{RGB}{118, 165, 175}
% \definecolor{ncqColour2}{RGB}{162, 196, 201}

% % styles     
% \tikzset{
%     processDiagram/.style={
%         startstop/.style={
%             rectangle,
%             draw,
%             minimum width=3cm, 
%             minimum height=0.8cm,
%             join = by arrow,
%             fill = black!0,
%         },
%         decision/.style = {
%             diamond,
%             aspect=1.7,
%             draw,
%             minimum width=2.5cm, 
%             minimum height=1cm, 
%             align=center,
%             join = by arrow,
%             fill = black!0,
%         },
%         process/.style = {
%             rectangle, 
%             rounded corners, 
%             draw,
%             minimum width=3cm,
%             minimum height=0.8cm, 
%             align=center,
%             join = by arrow,
%             fill = black!0,
%         },
%         process2/.style = {
%             rectangle, 
%             rounded corners, 
%             draw,
%             minimum width=3cm,
%             minimum height=0.8cm, 
%             align=center,
%             join = by arrow,
%             fill = ncqColour2,
%         },
%         line/.style = {
%             -
%         },
%         invisible/.style = {
%             join = by line,
%             minimum width=0cm, 
%             minimum height=0cm,
%             inner sep=0pt,
%         },
%         invisititle/.style ={
%             font = \bfseries,
%             join = by arrow,
%             outer sep = 5pt,
%         },
%         title/.style = {
%             font = \bfseries,
%             fill = black!20,
%             join = by arrow,
%             minimum width = 3cm,
%             minimum height=0.8cm,
%             draw,
%         },
%         output/.style ={
%             join = by line,
%             minimum width=0cm, 
%             minimum height=0cm,
%             inner sep=0pt,
%             font = \itshape,
%             align = center,
%         },
%         arrow/.style = {
%             ->
%         },
%     }
% }

\begin{document}


\begin{tikzpicture}[processDiagram, node distance=4mm and 5cm]

{ [start chain=trunk going below]
    \node [startstop, on chain] (n0) {Snippet};
    % \node[output, on chain](n0o) {
    %     ''consle.log('a');\\
    %     console.log(foo)"
    % };
    \node [process, on chain] (n1) {\textbf{1) Compile}};
    \node [decision, on chain] (errors) {Has Errors?};
    {[start branch = c going below]
        \node [output, on chain = going right, xshift=-4.5cm] (c0o) {Yes};
        \node [invisible, on chain = going above, yshift=3.63cm] (i0) {};
        \node [process, on chain = going right, xshift=-3.5cm] (c1) {\textbf{2) Targeted Fixes}};
        \node [decision, on chain] (errors2) {Has Errors?};
        {[start branch = cc going left]
            \node[output, on chain, xshift=4cm] {No};
            \node[invisible, on chain = going below, yshift=-4.3cm] (i1) {};
        }
        \node [output, on chain] (c1o) {Yes};
        \node [process, on chain = going below] (c2) {\textbf{3) TS Codefixes}};
        \node [decision, on chain] (errors3) {Has Errors?};
        {[start branch = cc going left]
            \node[output, on chain, xshift=4.4cm] {No};
            \node[invisible, on chain = going left, xshift=4.8cm] {};
        }
        \node [output, on chain] (c2o) {Yes};
        \node [process, on chain] (c3) {\textbf{4) Line Deletion}};
        \node [invisible, on chain, yshift=-0.2cm] (i2) {};
        \node [invisible, on chain = going left, xshift=-0.1cm] (i3) {};
    }
    \node [output, on chain] (n1o) {No};
    \node [startstop, on chain] (n2) {Done};
    \draw [arrow] (i1) -> (n2);
    \draw [arrow] (i3) -> (n2);
}

\begin{pgfonlayer}{background}
    \path (c1.west |- c1.north)+(-0.3,0.6) node (a) {};
    \path (c3.east |- c3.south)+(+0.3,-0.3) node (b) {};
    \path[fill=yellow!20,rounded corners, draw=black!50, dashed](a) rectangle (b); 

    \node[above of = c1, yshift=0.2cm]{Code Corrections};
\end{pgfonlayer}


\end{tikzpicture}

\end{document}