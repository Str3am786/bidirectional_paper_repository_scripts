\section{CONCLUSION}
In this work, we re-examine the use and effect of supervision in the entity alignment problem, which targets aligning entities with identical meanings across different knowledge graphs. Based on the three insights we derive---uni-space learning, relative similarity metric, and self-negative sampling, we develop a self-supervised entity alignment algorithm---\solution---to automatically align entities without training labels. The experiments on two widely-used benchmarks DWY100K and DBP15K show that \solution is able to beat or match most of the supervised alignment methods which leverage the 100\% of the training datasets. Our discovery indicates a huge potential to get rid of supervision in the entity alignment problem, and more studies are expected for a deeper understanding of self-supervised learning.