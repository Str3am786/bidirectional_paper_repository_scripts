Entity alignment, aiming to identify equivalent entities across different knowledge graphs (KGs), is a fundamental problem for constructing Web-scale KGs. 
Over the course of its development, the label supervision has been considered necessary for accurate alignments. 
Inspired by the recent progress of self-supervised learning, we explore the extent to which we can get rid of supervision for entity alignment. 
Commonly, the label information (positive entity pairs) is used to supervise the process of pulling the aligned entities in each positive pair closer.
However, our theoretical analysis suggests that the learning of entity alignment can actually benefit more from 
pushing unlabeled negative pairs far away from each other than pulling labeled positive pairs close. 
%pushing unlabeled entities in sampled negative pairs far away than pulling those in positive pairs close.}
By leveraging this discovery, we develop the self-supervised learning objective for entity alignment.
We present \solution with efficient strategies to optimize this \hhy{objective} for aligning entities without label supervision. 
Extensive experiments on benchmark datasets demonstrate that \solution~without supervision can 
match or achieve comparable results with state-of-the-art supervised baselines. 
The performance of \solution suggests that self-supervised learning offers great potential for entity alignment in KGs. 
{The code and data are available at \url{https://github.com/THUDM/SelfKG}}.