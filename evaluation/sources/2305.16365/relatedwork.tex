\section{Related work}
\label{sec_related_work}

Through a systematic literature review of {\em Agile Release Engineering} practices, \cite{Karvonen201787} highlighted that empirical research in software engineering is crucial to better understand the impact of adopting CI on software development. 

\subsubsection*{\textbf{CI and team productivity}}

The study by \cite{Hilton2016-xy} revealed that 70\%
of the most popular \textsc{GitHub} projects use CI. The authors identified that CI helps projects to release more often, whereas the CI build status may foster a faster integration of PRs. \cite{Vasilescu2015-tn} studied the potential impact of CI on the quality and productivity of software projects. They found that projects that use CI merge PRs more quickly if these PRs are submitted by core developers. The authors found that core developers identify significantly more bugs when using CI. Regarding the acceptance and latency of PRs in CI (where latency is the time taken to merge a PR), \cite{Yu2016-cy} found that the likelihood of rejecting a PR increases by 89.6\% when the PR breaks the build. The results also show that the more succinct the PR is, the greater the probability that the PR is reviewed and merged earlier. 
Furthermore, \cite{zhao2017impact} investigated the transition to \textsc{TravisCI}\footnote{\url{https://travis-ci.org/}} in open-source projects. 
According to their study, the following changes may occur when \textsc{TravisCI} is adopted: (i) a small increase in the number of merged commits; (ii) a statistically significant decrease in the number of merge commit churn; (iii) a moderate increase in the number of closed issues; and (iv) a stationary behavior in the number of closed PRs. 

In our study, we use an approach similar to \cite{Vasilescu2015-tn} to identify projects that use \textsc{TravisCI}.  
Our goal with our quantitative study is to understand the association between \textsc{TravisCI} and the time taken for PRs to be delivered to end users. Furthermore, our qualitative study investigates how contributors of open-source projects perceive the impact of CI on the review and release processes of their projects.
Our work is complementary to prior studies, contributing to a larger understanding of how CI can impact several development activities in software projects (i.e., code review and project release).

\subsubsection*{\textbf{CI and code review}}

Recent studies have investigated the impact of CI on code review. The study by \cite{zampetti2019study} 
found that PRs that generate successful builds have 1.5 more chances of being merged. Our qualitative investigation corroborates the results of \cite{zampetti2019study}, showing that the CI build status can influence the decisions of code reviewers. Furthermore, \cite{cassee2020silent} found that the discussion held before the acceptance of a PR reduced considerably \textit{after} CI was adopted. Conversely, the number of changes developers performed during code review remained roughly the same. The work of \cite{zhang2022a_pull_latency} investigated the influence of various factors on PR latency. They found that, when using CI, the build status and duration are moderately relevant factors for accepting PRs. In a follow-up study, \cite{zhang2022b_pull_decision} observed that CI assists the acceptance of PRs by automating the code review process and replacing part of the code inspection work, accelerating the review process.
Indeed, our qualitative study reveals that, among the list of CI factors that impact the code review process, the most cited factors were related to an improvement in \textit{automation} and \textit{confidence}.
The participants of our study highlighted that CI facilitates the understanding of code decisions, accelerating the code review process. 

\subsubsection*{\textbf{Adherence to CI best practices}}

\cite{Vasilescu2015-tj} studied the use of \textsc{TravisCI} in a sample of 223 \textsc{GitHub} projects.
They found that the majority of projects (92.3\%) are configured to use \textsc{TravisCI} but less than half actually use the CI service. \cite{felidre2019continuous} analyzed 1,270 open-source projects using \textsc{TravisCI} to understand the adherence of projects to the recommended CI practices. 
The authors observed that 748 (∼60\%) projects perform infrequent check-ins. The study by \cite{nery2019empirical} studied the relationship between the use of CI and the evolution of software tests. The authors found that the overall test ratio and coverage of projects improved after CI was adopted.
In our work, our participants mention that CI impacts the delivery time of merged PRs by improving \textit{project quality}, \textit{automation}, and the \textit{release process}. According to our participants, CI improves the code quality and stability, making developers more confident to ship releases. The confidence in developers can be fostered by comprehensive automated testing, especially when the code coverage is high.

\cite{gallaba2018use} studied 9,312 open-source projects using \textsc{TravisCI} to understand how projects are using or misusing the features of \textsc{TravisCI}. The authors found that the majority (48.16\%) of \textsc{TravisCI} configurations is specifying job processing nodes. Furthermore, explicit deployment code is rare (2\%), which indicates that developers rarely use \textsc{TravisCI} to implement Continuous Delivery. In our qualitative study,  our participants often emphasize the relevance of automated tasks in CI to improve the project release process. \textit{Automated tests} and \textit{release automation} (i.e., Continuous Deployment) are frequently mentioned when our participants explain the influence of CI on project releases. However, in addition to many developers understanding CI as a Continuous Deployment enabler, such a feature is misused by many projects that use CI \citep{gallaba2018use}.

