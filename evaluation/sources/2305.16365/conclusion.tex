
\section{Conclusion}
\label{sec_conclusions}

Our work consists of two studies that quantitatively and qualitatively investigate the influence of a CI service (e.g., \textsc{TravisCI}), and CI as a practice, on the time-to-delivery of merged PRs, respectively. In our quantitative study, we analyze 162,653 PRs of 87 GitHub projects to understand the factors that influence (and improve) the delivery time of merged PRs. In our qualitative study, we analyze 450 survey responses from participants of 73 projects (out of the initial 87 projects). We investigate the perceived influence of CI on the delivery time of merged PRs. We also study the perceived influence of CI on the code review and release processes.

As a key takeaway, our studies demonstrate that the adoption of \textsc{TravisCI}
will not necessarily deliver or merge PRs more quickly. Instead, the pivotal benefit of a CI service is to improve the mechanisms by which contributions to projects are processed (e.g., facilitating decisions on PR submissions), without compromising the quality of the project or overloading developers. The automation provided by CI and the boost in developers' confidence are key aspects of using CI. For instance, CI may help the process of sorting which PRs are worth reviewing (e.g., PRs with green builds).
Furthermore, open-source projects wishing to attract and retain external contributors should consider the use of CI in their pipeline, since CI is perceived to lower the contribution barrier while making contributors feel more confident and engaged in the project.

\section*{Acknowledgments}
\label{sec_Acknowledgments}

This work is partially supported by INES (\url{www.ines.org.br}), CNPq grants 465614/2014-0 and 425211/2018-5, CAPES grant 88887.136410/2017-00, FACEPE grants APQ-0399-1.03/17, and PRONEX APQ/0388-1.03/14.

\section*{Data Availability}
For replication purposes, we publicize our datasets and results to the interested researcher: \url{https://prdeliverydelay.github.io/#datasets}

\section*{Declarations}

\textbf{Conflict of interests}. The authors declare that they have no conflict of interest.