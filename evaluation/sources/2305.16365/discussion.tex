\section{Discussion}\label{sec_discussion}

In this section, we outline the implications of our results
to both research and practice in software engineering. 

{\em\bfseries Using a Continuous integration service is not a silver bullet.} Through our
quantitative analyses, we observe that a CI service does not always
reduce the time for delivering merged PRs 
to end users. In fact, analyzing 87 projects, we observe that
only 51\% of the projects deliver merged PRs more quickly \textit{after} the adoption of \textsc{TravisCI} (Section \ref{sec_quantitative_study_results} - $RQ1$). 
Additionally, our qualitative study reveals that there is no consensus regarding the impact of CI on the delivery time of merged PRs.
42.9\% of our participants declared that CI does not have impact on the delivery time of merged PRs (Section \ref{sec_quantitative_study_results} - $RQ4$), instead, factors such as \textit{project release process}, \textit{project maintenance} and \textit{PR characteristics} (i.e., \textit{trivial PRs}) are believed to influence the delivery time of merged PRs (Section \ref{sec_quantitative_study_results} - $RQ5$). 
If the decision to adopt CI is mostly driven by the goal of quickening the
delivery of merged PRs~\citep{Laukkanen2015-ab}, such a decision must be more
carefully considered by development teams. Finally, previous research suggests
that the adoption of CI increases the release frequency of a software
project~\cite{Hilton2016-xy}. However, we did not observe such an increase in
our quantitative analyses (Section \ref{sec_quantitative_study_results} - $RQ2$). Our study only considers user-intended releases, so we do not consider pre, beta, alpha, and rc (release candidate) releases in our analyses. It might be the case that when considering only established releases, the release frequency does not statically increases \textit{after} the adoption of \textsc{TravisCI}.

{\em\bfseries If CI is a CD enabler, why is CD seemingly rare?} CI, Continuous Delivery (CDE), and Continuous Deployment (CD) are complementary practices that can be used in the agile releasing engineering environment \citep{Karvonen201787}. CDE is a practice that automates the software delivery process, and it is often considered to extend CI. Therefore, a project that uses CDE, the delivery can occur at any time, with little manual effort required. Furthermore, CD is a step further from the adoption of CDE, which is a practice where projects release each successful build to end users automatically. 
In this context, several participants of our study consider that 
\textit{automated deployment} is a subsequent step in CI adoption that can help projects rapidly deliver software changes to end users (e.g., CD can deliver merged PRs automatically). However, through the analysis of 9,312 open-source projects that use \textsc{TravisCI}, the study by \cite{gallaba2018use} found that explicit deployment code is rare (2\%), which suggests that developers rarely use \textsc{TravisCI} to implement Continuous Deployment. An interesting future study is to better understand the gap between CI and CD as well as how to bridge this gap.
Furthermore, we observe that \textit{before} the adoption of \textsc{TravisCI}, the merge workload is the most influential variable to model the delivery time of PRs, while \textit{after} the adoption of \textsc{TravisCI}, the most influential variable is the moment at which a PR is merged in the release cycle (i.e., queue rank metric). One possible reason for the change in most influential variables in the time periods \textit{before} and \textit{after} \textsc{TravisCI}, is that \textit{after} the adoption of \textsc{TravisCI}, the merge workload could have been better managed, leading the queue rank to be more influential on the delivery time of merged PRs. This indicates that the delivery time of merged PRs is more dependent on when the PR was merged in the release cycle than whether the project adopts a CI service. Therefore, projects that wish to quicken the delivery of merged PRs need to foster the culture of frequent release instead of solely relying on the adoption of a CI service (i.e., Travis CI) in their pipeline.

{\em\bfseries Automation and confidence are key aspects for the throughput generated by CI.} 
We observe that the adoption of a CI service is associated with many benefits, such as a higher number of contributors, PR submissions, and a higher PR churn per release (Section \ref{sec_quantitative_study_results} - $RQ2$). However, the release frequency is roughly the same as \textit{before} using \textsc{TravisCI}. Therefore, teams that wish to adopt a CI service should be aware that their projects will not always deliver merged PRs more quickly or release them more often, but that the pivotal benefit of a CI service is the ability to process substantially more contributions in a given time frame, which is closely tied to the automation and confidence that release managers (Section \ref{sec_qualitative_study_results} - $RQ6$), reviewers (Section \ref{sec_qualitative_study_results} - $RQ7$), and external contributors (Section \ref{sec_qualitative_study_results} - $RQ8$) feel towards their codebase and project environment. 

{\em\bfseries CI may improve the decision-making process of software projects.} Our results (Section \ref{sec_qualitative_study_results} - $RQ7$) reveal that most contributors' quotes (58\%, \nicefrac{87}{578}) agree that CI impacts the time required to review PRs. According to our participants, CI may quicken the process of sorting which PRs are worth reviewing, e.g., a PRs with green builds. However, project maintainers should be concerned with the test coverage in their project, as it is essential for reviewers to be more confidence in the CI feedback to submitted PRs. We observe that contributors whose previously submitted PRs were merged and delivered quickly, are also likely to have their future PRs delivered quickly (\ref{sec_quantitative_study_results} - $RQ3$).  Hence, we recommend that the first PR submissions of a new contributor should be carefully crafted to maintain a successful track record in their projects (so they can build trust, causing their future PRs to be delivered more quickly).
An interesting future work would be to investigate how CI can influence the decision-making process involved in different development tasks, i.e., from requirements engineering to project delivery~\citep{Sharma2021influence}. 
