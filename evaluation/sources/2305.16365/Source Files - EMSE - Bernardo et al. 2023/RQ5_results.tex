\subsection*{\textbf{\RQfour}}

The following themes are generated by our thematic analysis of the potential causes of delivery delay in merged PRs: \textit{\textbf{PR characteristics}}; \textit{\textbf{project maintenance}}; \textit{\textbf{release process}}; \textit{\textbf{team characteristics}}; \textit{\textbf{contributors}}; \textit{\textbf{testing}} and \textit{\textbf{automation}}. Table \ref{tab:general_factors_impact_ci} shows the frequency of mentions in our participants' responses for each of the codes and themes that influence the delivery time of merged PRs. We describe all themes and the most mentioned codes in the following.

% Table generated by Excel2LaTeX from sheet 'GENERAL DEVELOPERS PERCEPTION'
\begin{table}
	\centering
	\caption{Frequency of mentions in the developer responses for each code and theme that impact the delivery time of merged PRs.}
	\begin{tabular}{clcc}
		\hline
		\multirow{2}[4]{*}{\textbf{Theme}} & \multicolumn{1}{c}{\multirow{2}[4]{*}{\textbf{Code}}} & \multicolumn{2}{c}{\textbf{Frequency}} \bigstrut\\
		\cline{3-4}          &       & \multicolumn{1}{p{5.335em}}{\textbf{Frequency per code}} & \multicolumn{1}{p{5.665em}}{\textbf{Frequency per theme}} \bigstrut\\
		\hline
		\multirow{14}[28]{*}{\textbf{PR characteristics}} & Bug fix & 91    & \multirow{14}[28]{*}{330} \bigstrut\\
		\cline{2-3}          & \multicolumn{1}{p{10.085em}}{PR position in the release cycle} & 32    &  \bigstrut\\
		%		\cline{2-3}          & Trivial PR & 76    &  \bigstrut\\
		\cline{2-3}          & Change complexity & 86    &  \bigstrut\\
		\cline{2-3}          & PR prioritization & 57    &  \bigstrut\\
		%		\cline{2-3}          & Big change & 13    &  \bigstrut\\
		\cline{2-3}          & PR size & 13    &  \bigstrut\\
		\cline{2-3}          & Security fix & 12    &  \bigstrut\\
		\cline{2-3}          & Code quality & 11    &  \bigstrut\\
		%		\cline{2-3}          & Change complexity & 10    &  \bigstrut\\
		%     CHANGE COMPLEXITY WAS JOINED TO TRIVIAL PRS
		\cline{2-3}          & Guideline adherence & 7     &  \bigstrut\\
		\cline{2-3}          & Backward compatibility & 6     &  \bigstrut\\
		\cline{2-3}          & Feature dependence & 4     &  \bigstrut\\
		\cline{2-3}          & Good PR description & 4     &  \bigstrut\\
		\cline{2-3}          & Feature improvement & 4     &  \bigstrut\\
		\cline{2-3}          & Breaking change & 3     &  \bigstrut\\
		\hline
		\multicolumn{1}{c}{\multirow{6}[12]{*}{\textbf{Project maintainance}}} & Maintainers availability & 41    & \multirow{6}[12]{*}{112} \bigstrut\\
		\cline{2-3}          & Maintainers activeness & 34    &  \bigstrut\\
		\cline{2-3}          & Volunteer based & 12    &  \bigstrut\\
		\cline{2-3}          & Interest gauging of maintainers & 11    &  \bigstrut\\
		\cline{2-3}          & \multicolumn{1}{p{10.085em}}{Maintainer  responsiveness} & 7     &  \bigstrut\\
		\cline{2-3}          & Maintainers workload & 7     &  \bigstrut\\
		\hline
		\multicolumn{1}{c}{\multirow{7}[12]{*}{\textbf{Release process}}} & Release cycle & 86    & \multirow{7}[12]{*}{137} \bigstrut\\
		\cline{2-3}          & Automated deployment & 16    &  \bigstrut\\
		\cline{2-3}          & Batching & 11    &  \bigstrut\\
		\cline{2-3}          & Business rules & 10    &  \bigstrut\\
		\cline{2-3}          & Manual release process & 5     & \bigstrut\\
		\cline{2-3}          & Misuse of CI & 5     &  \bigstrut\\
		\cline{2-3}          & \multicolumn{1}{p{10.085em}}{Release early, release often culture} & 4     &  \bigstrut\\
		\hline
		\multirow{4}[10]{*}{\textbf{Team characteristics}} & Team size & 21    & \multirow{4}[10]{*}{40} \bigstrut\\
		\cline{2-3}          & Small project & 13    &  \bigstrut\\
		\cline{2-3}          & Open source & 4     &  \bigstrut\\
		\cline{2-3}          & Paid staff & 2     &  \bigstrut\\
		\hline
		\multirow{3}[10]{*}{\textbf{Contributors}} & Contributor trustworthiness & 4     & \multirow{3}[10]{*}{9} \bigstrut\\
		\cline{2-3}          & Contributor experience & 3     &  \bigstrut\\
		\cline{2-3}          & \multicolumn{1}{p{10.085em}}{Contributor and maintainers relationship} & 2     &  \bigstrut\\
		\hline		
		\multirow{6}[11]{*}{\textbf{Testing}} & Test coverage & 12     & \multirow{6}[11]{*}{35} \bigstrut\\
		\cline{2-3}          & Testing time & 9    &  \bigstrut\\
		\cline{2-3}          & Lacking tests & 7     &  \bigstrut\\
		%		\cline{2-3}          & Discussion time & 5     &  \bigstrut\\
		\cline{2-3}          & Broken tests & 3     &  \bigstrut\\
		\cline{2-3}          & Manual testing & 2     &  \bigstrut\\
		\cline{2-3}          & Build duration & 2     &  \bigstrut\\
		\hline
%		\multirow{4}[8]{*}{\textbf{Automation}} & Manual release process & 5     & \multirow{4}[8]{*}{12} \bigstrut\\
%		\cline{2-3}          & CI setup & 3     &  \bigstrut\\
%		\cline{2-3}          & Misuse of CI & 2     &  \bigstrut\\
%		\cline{2-3}          & Build duration & 2     &  \bigstrut\\
%		\hline
	\end{tabular}%
	\label{tab:general_factors_impact_ci}%
\end{table}%

\vspace{0.6mm}
\noindent\textit{\textbf{PR Characteristics.\textsuperscript{(330)}}} The {\em characteristics of merged PRs} theme is the theme most mentioned by our participants. According to participants' responses, \textit{PR prioritization}\textsuperscript{(57)} is one of the main factors that can shorten or lengthen the time to deliver merged PRs. The priority of a PR is recurrently associated with \textit{bug fixes}\textsuperscript{(91)} and \textit{security fixes}.\textsuperscript{(12)} According to C020, \textit{``Anything that is considered a critical security fix or major bug fix is generally shipped within 1-2 weeks of submission. This happens frequently.''} In contrast, several participants\textsuperscript{(22)} state that PRs with a longer delivery time are frequently associated with non-urgent features. As stated by C111, \textit{``The most frequent reason [for a longer delivery time] is that the PR is not business-critical.''} In a similar vein, the study by \cite{gousios2015work} found that integrators commonly prioritize contributions by examining their criticality or urgency, e.g., bug fixes or new important features are commonly assigned a higher priority. 

Additionally, the \textit{code quality}\textsuperscript{(11)} of the PR, the \textit{change complexity}\textsuperscript{(10)} and the \textit{guideline adherence}\textsuperscript{(7)} are commonly mentioned codes related to the delivery time of merged PRs.
For example, C168 exemplified that \textit{``adherence to pull request guidelines. Small fix. Clearly defined solution''} are factors that quicken the delivery time of merged PRs. Along the same lines, \cite{gousios2015work} argue that contributions conforming to project style and architecture, source code quality, and test coverage are top priorities for integrators. Finally, the \textit{change complexity}\textsuperscript{(86)} is also mentioned to help PRs to be quickly evaluated and delivered. For instance, C431 states that \textit{``the PR I issued to Crafty was integrated very quickly, mainly because it was a trivial, absolutely non-breaking change.''} The study by \cite{Yu2016-cy} also identified that the complexity of PRs is a factor that influence the PR latency (i.e., the time taken for a PR to be merged). \cite{weissgerber2008small} observed that smaller PRs are more likely to be accepted. Indeed, according to our survey participants, the less complex or the more trivial a PR is, the greater the more likely that the PR will be quickly delivered as less effort is needed. 

\vspace{0.6mm}
\noindent\textit{\textbf{Project maintenance.\textsuperscript{(112)}}} Project maintenance is associated with the project maintainers' activities. When considering the influence of the maintainers on the delivery time of merged PRs, most participants mentioned \textit{maintainers' availability}\textsuperscript{(41)}.
The study by \cite{Yu2016-cy} found that integrators' availability has a significant effect on PR latency. Additionally, our study suggests that \textit{maintainers' availability} influences the delivery of PRs. For instance, C115 considers that a longer delivery time is associated with \textit{``long times between releases, mostly due to maintainer availability''}.
Another important and frequently mentioned cause of delay is that open-source projects are \textit{volunteer based},\textsuperscript{(12)} i.e., contributors are often volunteers \citep{alexander2002working}. For instance, C336 stated that \textit{``OSS projects are staffed by volunteers who come and go and then the priorities of the project shift and some feature become less important''}. 
Furthermore, \textit{maintainers' workload},\textsuperscript{(7)} \textit{maintainer activeness}\textsuperscript{(34)} and \textit{maintainers' engagement}\textsuperscript{(11)} are believed to influence the delivery time of merged PRs. For instance, C419 states that \textit{``if the maintainers of the project are interested, it [the PR] will get processed quickly.''} Previous work also observed that \textit{workload} is a factor that plays a key role in the delivery time of addressed issues of three large open-source projects \citep{daCosta2018impact}, which corroborates our results. The \textit{maintainer responsiveness}\textsuperscript{(7)} also influences the delivery time of merged PRs. C090 states that \textit{``it also helps [to deliver PRs more quickly] if maintainers can be easily contacted (IRC/Slack/Twitter)''}. Furthermore, codes related to project maintenance should be carefully considered in project management, since they might influence not only the delivery time of merged PRs, but also project success. The study by \cite{coelho2017modern} elucidates that lack of maintainers' time and interest are factors that might lead open-source projects to fail. 

\vspace{0.6mm}
\noindent\textit{\textbf{Team Characteristics.\textsuperscript{(40)}}} Team characteristics are also believed to influence the delivery time of merged PRs. Several participants explained that a long delivery time may occur due to 
 \textit{team size}.\textsuperscript{(21)} As stated by C393 \textit{``the team doing reviews were (and is still) understaffed.''} 
The study by \cite{Vasilescu2015-tn} also identified that larger teams can process more PRs (i.e. merge or reject PRs).
This is common in \textit{open-source}\textsuperscript{(4)} projects, as explained by C238 when stating that \textit{``Open source projects tend to delay the publication. Private projects suffer from this problem with much less impact. I wonder if it is due to the lack of a dedicated team in the open-source project, or maybe the focus isn't necessary in the part of the software I contributed.''} \textit{Small project}\textsuperscript{(13)} is also believed to quicken the delivery of merged PRs, as explained by C321, \textit{``on small projects some PRs might be released as a hotfix release very quickly. So I think the speed of delivery is usually in direct proportion to the size of the project.''} 

Overall, the codes related to project characteristics should be carefully observed for projects attempting to decrease the time to deliver their merged PRs. Our participants believe that small projects tend to deliver their PRs more quickly, as they can manage the incoming contributions more easily. With project growth (e.g., an increased number of PRs and project complexity), a small core \textit{team size} can become a bottleneck when delivering merged PRs. In open-source projects, the bottleneck may be exaggerated as projects are volunteer-based \citep{alexander2002working} with most developers working in their free time. To overcome these barriers, projects may consider adopting strategies to deal with the large increase in contributions, as well as to deal with maintainers' inattentiveness, like \textit{transfer the project to new maintainers} or \textit{accept new core developers} \citep{coelho2017modern}. Additionally, adding \textit{paid staff}\textsuperscript{(2)} to the project could be an alternative to deal with the project workload and quicken the delivery time of the merged PRs. For instance, C225 explained the importance of paid staff on the software project by stating the following: \textit{``I have submitted PRs to very large open-source projects, like sklearn or AWS CLI. These projects typically get released frequently on established schedules by maintainers who are, in part, employed to release the projects.''}

\vspace{0.6mm}
\noindent\textit{\textbf{Contributors.\textsuperscript{(9)}}} This theme reflects the potential influence that contributors (i.e., those who submit PRs) have on the delivery time of PRs. The contributors' social status is important when it comes to the delivery of their PRs. When analyzing the time length between the submission and merge of a PR, \cite{Yu2016-cy} found that open-source projects prefer to quickly accept PRs originating from trusted contributors. After PRs are merged, the contributors' social status also influences the time to deliver PRs. For example, \textit{contributor trustworthiness}\textsuperscript{(4)} is often evaluated before a PR is delivered. C421 states that \textit{``after you work with people for a while, you recognize and trust those that have proved to be good at what they do.''} According to our participants, the greater the \textit{contributor experience},\textsuperscript{(3)} the more likely the contributor will have their PR delivered more quickly. 
Furthermore, \cite{soares2018factors} observed that the social relationship between contributors and reviewers influences the evaluation of a PR. In fact, we also identify that the \textit{contributor and maintainer relationship}\textsuperscript{(2)} positively influences the delivery time of merged PRs (i.e., contributors that are socially closer to a core team member have a higher chance to have their PRs delivered more quickly). C403, for example, states that \textit{``some [PRs] were shipped quickly because one of maintainers is my friend then he merged immediately.''}

\vspace{0.6mm}
\noindent\textit{\textbf{Testing.\textsuperscript{(35)}}} The delivery time of merged PRs can also be associated with testing. When asked about factors that might cause a PR to be delayed, our participants listed \textit{testing time},\textsuperscript{(9)} \textit{lacking tests},\textsuperscript{(7)} \textit{broken tests},\textsuperscript{(3)} and \textit{manual testing}.\textsuperscript{(2)} For instance, in relation to testing time, C030 states that \textit{``release time can be quite long but not due to a specific PR, but to an overall review and testing process of a whole software release.''} 
In this respect, reducing manual steps is a must for projects wishing to release code more frequently \citep{neely2013continuous}. Test planning is very important in the Quality Assurance (QA) process. For instance, with automated test suites, the QA team no longer needs to manually execute the tests for the majority of the system, which would be more error-prone and slow \citep{neely2013continuous}. However, beyond the automated test execution, projects must also be concerned with \textit{test coverage}\textsuperscript{(12)}. Automated test execution for projects with low test coverage potentially leads to certain bugs not being identified during build time \citep{felidre2019continuous}. Several participants mention \textit{test coverage}\textsuperscript{(12)} as influencing the delivery time of PRs. C347 declares that \textit{``code that had a large test coverage and small PR are generally deployed safely and quickly.''}
Additionally, C287 mentions issues related to building duration when testing their PRs: \textit{``NixOS nixpkgs stable chanel: CI for deep dependencies take a huge amount of computation time.''} Indeed, keeping the build and test process short (ideally by not taking more than 10 minutes) is one of the prerequisites for CI adoption~\citep{Humble2010-ca}. A long build duration may lead to a set of problems, for example, developers may check-in their code less often, as they have to sit around for a long time while waiting for the build (and tests) to run. 

\vspace{0.6mm}
\noindent\textit{\textbf{Release process.\textsuperscript{(132)}}} The delivery time of merged PRs may also be associated with the release process of projects. The \textit{release cycle}\textsuperscript{(86)} code is the most cited code of this theme. For instance, C029 states that \textit{``code was merged quickly, but had to wait for the test, review, and release cycle to complete. So had to wait for months to see the code I needed released publicly.''} 
Indeed, a shorter release cycle has been mentioned to shorten the time-to-market and quicken the users' feedback loop~\citep{da2016agility}. \cite{daCosta2018impact} also found that traditional release cycles (which are longer cycles) could actually deliver new functionalities more quickly by using minor releases. Therefore, it seems that the higher the frequency of {\em user-intended} releases, the quicker the delivery of merged PRs. 

When explaining potential reasons for the quick delivery of PRs, our participants recurrently mentioned \textit{automated deployment}.\textsuperscript{(16)} For instance, C100 states \textit{``we also implemented continuous deployment so that when a change is merged, it is automatically deployed.''} 
Automated deployment is mandatory for the adoption of continuous deployment (CD). The goal of CD is to automatically deploy every change to the production environment \citep{shahin2017continuous}. In the context of the pull-based development model, CD is said to have a substantial influence on the delivery time of the proposed changes, since each merged PR is automatically deployed.	

Additionally, \textit{batching}\textsuperscript{(11)} and \textit{business rules}\textsuperscript{(10)} are also important codes when it comes to the delivery time of PRs. The \textit{batching} code is associated with the process of queuing up PRs to launch bigger releases. For instance, C092 states that a PR that experiences a long delivery time is linked to the fact that \textit{``Usually they [project maintainers] wait for a good amount of fixes or an important fix like the ones involving security [to launch a release].''} 
However, project managers should be careful about the risks of bigger releases. Usually, big releases are a result of a longer release cycle, which may delay the delivery of PRs for a longer time. In contrast, smaller batch sizes would help the production environment to have fewer defects as smaller code changes may lead to faster feedback from the CI system \citep{neely2013continuous}.
Regarding the business impact on delivery time, C353 declared that \textit{``If the change introduces a new feature that has an impact on end users, the marketing and customer success team need to communicate to them in advance and it takes weeks.''} Indeed, the work of \cite{daCosta2018impact} also found that the delivery time may also be associated with the collaboration with other teams. They also mention that the marketing team is recurrently cited when delays to release occur due to other teams' collaboration. For instance, a PR may be delayed due to the need of aligning the software release with external events for marketing reasons. 
Additionally, the \textit{misuse of CI}\textsuperscript{(5)} is mentioned as a factor that negatively impacts the delivery time of PRs, i.e., C118 states the following: \textit{``I've worked in a project that hadn't CI nor automated tests, so each release took at least one week to be deployed. Once, we had some issues with our deploy system that made us delay the deploy for one month.''}
Lastly, the release culture also impacts delivery time. For example, C090 stated that \textit{``projects with \textit{release early, release often}\textsuperscript{(4)} culture are usually the fastest to deliver.''} Overall, our results suggest that projects should reduce manual processes to quicken the release process and increase the release frequency. The adoption of continuous software engineering practices \citep{shahin2017continuous}, i.e., CI, Continuous Delivery (CDE), and Continuous Deployment (CD), should be considered in this regard. 

\begin{center}
	\begin{tabular}{|p{.96\columnwidth}|}
		\hline
		\textbf{Summary:}
		\textit{The delivery time of merged PRs is impacted by several factors. 87.3\% (\nicefrac{579}{663}) of the mentions associate the delivery time of PRs with their characteristics, the project release process, and project maintenance. According to our survey responses, simple PRs and PRs that fix bugs are delivered more quickly. The PR delivery time is also often linked to the availability of maintainers and the size of the release cycle.} \\
		\textbf{Implications:}
		\textit{Teams that wish to deliver their merged PRs more quickly to their users should also be concerned with other aspects beyond CI, such as encouraging their contributors to submit simple PRs and maintaining short release cycles.}
		\\
		\hline
	\end{tabular}
\end{center}