\subsection*{\textbf{\RQeight}}
	
	\noindent\textbf{59\% (\nicefrac{227}{383}) of quotes in RQ8 argue that projects using CI are more attractive to receive external contributors.} The most recurrent themes in RQ8 reveal that projects using CI have more  \textit{attractive characteristics}\textsuperscript{(73)} (e.g., easier PR acceptance) and a \textit{lower contribution barrier}.\textsuperscript{(154)}
	Conversely, 26\% (\nicefrac{98}{383}) of the quotes argue that CI does not influence the number of project contributors, attributing the increase of contributors to other factors, such as \textit{project growth},\textsuperscript{(35)} \textit{maturity}\textsuperscript{(14)} and \textit{popularity}.\textsuperscript{(22)} The remaining 15\% (\nicefrac{58}{383}) of quotes refer to answers, such as ``No Answer (NA)''. Table \ref{tab:ci_impact_on_attracting_contributors} shows the complete list of themes and codes related to the influence of CI on attracting contributors to open-source projects.
	
	\begin{table}
		\centering
		\caption{Frequency of citations per theme and code related to the influence of CI on attracting contributors to open-source projects.}
    \begin{tabular}{cccc}
    \hline
    \multicolumn{1}{c}{\multirow{2}[4]{*}{\textbf{Theme}}} & \multicolumn{1}{c}{\multirow{2}[4]{*}{\textbf{Code}}} & \multicolumn{2}{p{10em}}{\textbf{Frequency}} \bigstrut\\
\cline{3-4}          &       & \multicolumn{1}{p{5em}}{\textbf{Frequency per code}} & \multicolumn{1}{p{5em}}{\textbf{Frequency per theme}} \bigstrut\\
    \hline
    \multicolumn{1}{c}{\multirow{11}[22]{*}{\parbox{3.2cm}{\centering \textbf{Attractive project characteristics with CI}}}} & \multicolumn{1}{p{14em}}{Reduced review effort/time} & 22    & \multirow{11}[22]{*}{73} \bigstrut\\
\cline{2-3}          & \multicolumn{1}{p{14em}}{Clear development process} & 12    &  \bigstrut\\
\cline{2-3}          & \multicolumn{1}{p{14em}}{Project quality} & 7     &  \bigstrut\\
\cline{2-3}          & \multicolumn{1}{p{14em}}{Project stability} & 7     &  \bigstrut\\
\cline{2-3}          & \multicolumn{1}{p{14em}}{Easier PR acceptance} & 7     &  \bigstrut\\
\cline{2-3}          & \multicolumn{1}{p{14em}}{Actively maintained project} & 5     &  \bigstrut\\
\cline{2-3}          & \multicolumn{1}{p{14em}}{Faster delivery} & 4     &  \bigstrut\\
\cline{2-3}          & \multicolumn{1}{p{14em}}{Welcoming for contributions} & 3     &  \bigstrut\\
\cline{2-3}          & \multicolumn{1}{p{14em}}{Regular releases} & 2     &  \bigstrut\\
\cline{2-3}          & \multicolumn{1}{p{14em}}{Projects with CI seems more mature} & 2     &  \bigstrut\\
\cline{2-3}          & \multicolumn{1}{p{14em}}{Best industrial practices followed} & 2     &  \bigstrut\\
    \hline
    \multicolumn{1}{c}{\multirow{5}[10]{*}{\parbox{3.2cm}{\centering \textbf{Lower contribution barrier with CI}}}} & \multicolumn{1}{p{14em}}{CI confidence} & 60    & \multirow{5}[10]{*}{154} \bigstrut\\
\cline{2-3}          & \multicolumn{1}{p{14em}}{Build status awareness} & 57    &  \bigstrut\\
\cline{3-3}          & \multicolumn{1}{p{14em}}{Build and test automation} & 28    &  \bigstrut\\
\cline{2-3}          & \multicolumn{1}{p{14em}}{Engagement to contribute} & 6     &  \bigstrut\\
\cline{2-3}          & \multicolumn{1}{p{14em}}{Lowered entry barrier} & 3     &  \bigstrut\\
    \hline
    \multicolumn{1}{c}{\multirow{5}[10]{*}{\textbf{Not related to CI}}} & \multicolumn{1}{p{14em}}{Project growth} & 35    & \multirow{5}[10]{*}{98} \bigstrut\\
\cline{2-3}          & \multicolumn{1}{p{14em}}{Project popularity} & 22    &  \bigstrut\\
\cline{2-3}          & \multicolumn{1}{p{14em}}{Project maturity} & 14    &  \bigstrut\\
\cline{2-3}          & \multicolumn{1}{p{14em}}{Project activeness} & 3     &  \bigstrut\\
\cline{2-3}          & \multicolumn{1}{p{14em}}{Non-causal correlation} & 24    &  \bigstrut\\
    \hline
    NA    &       &       & 58 \bigstrut\\
    \hline
    \end{tabular}%
		\label{tab:ci_impact_on_attracting_contributors}%
	\end{table}%

\vspace{1mm}
\noindent\textbf{Attractive project characteristics.\textsuperscript{(73)}}
Several quotes state that developers feel more attracted to the characteristics of projects that use CI. They argue that, when using CI, projects tend to have a \textit{reduced review effort/time},\textsuperscript{(22)} which may lead to \textit{PRs being accepted} more easily.\textsuperscript{(7)} As explained by C033, \textit{``It's easier to handle incoming changes from people, so it's possible to have more of them.''} 
Additionally, the \textit{project quality}\textsuperscript{(7)} and \textit{stability}\textsuperscript{(7)} of the projects are frequently mentioned by our participants as a consequence of using CI, which may motivate developers to contribute more. In this regard, 
C178 declares that \textit{``People want to work in high-quality projects. CI is a mark of quality.''} According to our participants, potential contributors also look for stability in projects and CI provides this sense of stability. As explained by C051,
\textit{``Projects that use CI tend to look more stable and serious. It might be a reason for some contributors to be attracted to more serious projects.''}
 Finally, potential contributors prefer projects that follow industry best practices. In this regard, C355 states: \textit{``Maybe because the project looks more professional, following \textit{best industrial practices}.\textsuperscript{(2)} I would not contribute to a project without CI, or I would setup CI first''}.

\vspace{1mm}
\noindent\textbf{Lower contribution barrier.\textsuperscript{(154)}}
The majority of quotes stating that CI attracts more contributors explain that this is due to CI projects having a lower contribution barrier. This lower barrier promotes an \textit{increased confidence}\textsuperscript{(60)} in contributors when a project is using CI. The quotes also argue that CI promotes a \textit{better build status awareness}\textsuperscript{(57)} for contributors. Regarding confidence, 
C146 declares that \textit{``Developers like CI, it adds confidence to your work and it is pleasurable to work in this highly structured and coordinated way.''} These observations corroborate the study by \cite{coelho2017modern}, which also found CI as one of the most important maintenance practices in top open-source projects.	
Regarding CI promoting a better build status awareness, C084 states: \textit{``In my case, I like to have my changeset reviewed and tested soon as possible, and CI jobs are really fast for that.''} 
Moreover, several participants argue that it is \textit{easier to contribute} when a project uses CI. In this matter, 
C100 declares that \textit{``As a contributor not part of the core team, CI makes it easier to understand the code, because you can look at the tests to understand the design. It makes it easier to contribute without a lot of prior knowledge of the project''}.

\vspace{1mm}
\noindent\textbf{Although most quotes agree that CI attracts more contributors, 26\% (\nicefrac{98}{383}) of quotes state that there is no causal relationship between CI and the increase in contributors.} Many participants argue that there is a \textit{non-causal relationship}\textsuperscript{(24)} between the increase in contributors and the adoption of CI. For instance, C352 declares that \textit{``I think that people adopt CI because contributions become difficult to manage due to increasing quantity (probably driven by popularity). I would imagine these two variables are not causally related but simply correlated."}.
Indeed, many of our participants attribute the increase in contributors to \textit{project growth}\textsuperscript{(35)}, \textit{maturity},\textsuperscript{(14)} and \textit{popularity}.\textsuperscript{(22)} 
For instance, C423 states that the increase in contributors is \textit{``probably not related [to CI], but accidentally relates to a hype curve and or maturity level of the project.''} Indeed, the study by \cite{Hilton2016-xy} shows that popular projects are more likely to use CI. \cite{Hilton2016-xy} also found that the first CI build in their investigated projects occurred around 1 year (median) after the project creation. They argue that this is the case because the adoption of CI may not always provide a large amount of value during the very initial phases of the development of a project.


\begin{center}
	\begin{tabular}{|p{.95\columnwidth}|}
		\hline
		\textbf{Summary:}
		\textit{
		Although most quotes from participants (59\%, \nicefrac{227}{383}) argue that projects using CI are more attractive to potential contributors, 26\% (\nicefrac{98}{383}) of quotes argue that there is no causal relationship between CI and the increase in contributors, i.e., other factors play a role instead, such as project growth and increase in project popularity over time.
		}\\
		\textbf{Implications:}
		\textit{Open-source projects intending to attract and retain external contributors should consider the use of CI in their pipeline since CI is perceived to lower the contribution barrier while making contributors feel more confident and engaged in the project.}
		\\
		\hline
	\end{tabular}
\end{center}