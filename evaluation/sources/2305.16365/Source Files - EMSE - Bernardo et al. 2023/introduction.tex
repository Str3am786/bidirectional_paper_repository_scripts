\section{Introduction}

Development teams are required to deliver software functionalities more quickly than ever to
improve the time-to-market and success of their software projects \citep{Debbiche2014}. The quick delivery of functionalities may keep customers engaged with the project while providing valuable feedback. 
To improve the processes of software integration and packaging, Continuous Integration (CI) has been proposed as part of the Extreme Programming (XP) methodology \citep{Beck2000-ja}, which claims that CI can provide more confidence for developers and quicken the delivery of software functionalities \citep{Laukkanen2015-ab}. 

Continuous Integration is a set of practices that enables development teams to integrate software more frequently \citep{Fowler2006-zc}. The increased number of integrations is possible by using automated tools such as automated tests. Ideally, CI should automatically compile, test, and package the software whenever code modifications occur. Nowadays, several vendors (e.g., \textsc{Apache} or \textsc{GitHub}) have developed tools to provide {\em CI as a service} for developers to implement their CI pipelines. Examples of these tools are \textsc{CloudBees}, \textsc{GitHub Actions}, and \textsc{TravisCI}. The main philosophy behind CI is that the software must always be in a working state, which is constantly put to test at each integration \citep{Duvall2007-tb}. CI has been widely adopted by the software development community in both open-source and corporate software projects. 

Existing research has analyzed the usage of CI in open-source projects hosted on \textsc{GitHub} \citep{Vasilescu2015-tj,Vasilescu2015-tn,Hilton2016-xy,bernardo2018studying,nery2019empirical,soares2022effects,santos2022investigating}. For instance, \citet{Vasilescu2015-tn} investigated the productivity and quality outcomes of projects that use CI services on \textsc{GitHub}. They found that projects that use CI merge pull requests (PRs) more quickly if they are submitted by core developers. Also, core developers discover significantly more bugs when they use CI. 
Although existing research has demonstrated that CI may provide benefits for development teams, \citet{soares2022effects} revealed that several studies investigate the benefits of using a {\em CI service} instead of studying the benefits of CI as a whole practice. For example, instead of checking whether studied projects adopt the full set of practices required by CI~\citep{felidre2019continuous}, most studies have assumed that projects use CI solely because a CI service was employed (such as \textsc{TravisCI}). We as a community should be clear about such scenarios when reporting our results. According to \cite{Fowler2006-zc}, adopting CI is not using a CI service alone but also adopting and maintaining specific development practices. 
%The usage of a CI service by open-source projects does not ensure the adoption of other important practices associated with CI \citep{felidre2019continuous}. 
To adopt CI appropriately, projects must maintain short build durations, fix broken builds as immediately as possible, check-in code frequently, and maintain high code coverage \citep{Duvall2007-tb,felidre2019continuous,santos2022investigating}.

In this regard, a common claim about adopting CI is that projects are able to release more frequently \citep{Stahl:2014:MCI:2562355.2562828,Hilton2016-xy}, implying that software updates would be delivered more quickly to their end-users. However, there is no sufficient empirical evidence to show that CI can indeed be associated with a quicker delivery of software functionalities to end users. Studying whether CI can quicken the delivery of software functionalities is important because release delays are frustrating to end users~\citep{costa2014empirical,Da_Costa2016-cb}.

In our prior work \citep{bernardo2018studying}, we quantitatively analyzed whether the use of a CI service (\textsc{TravisCI}) is correlated with the
time to deliver merged {\em Pull Requests} (PRs) of GitHub projects.
Our study investigated 162,653 PRs from 87 GitHub projects, which were implemented in 5 programming languages.\footnote{\url{https://prdeliverydelay.github.io/\#studied-projects}} We found that the time-to-deliver PRs is shorter \textit{after} adopting \textsc{TravisCI} in only 51.3\% of the projects. As we have observed that the use of a CI service is not necessarily associated with a quicker delivery of pull requests, we designed a qualitative study to obtain deeper explanations for our results while deepening our understanding of the potential influence of CI {\em as a whole practice} on the time-to-market of merged PRs. For example, do developers believe that CI, as a whole, influence the delivery of PRs in their projects? We designed a qualitative study because we could not find answers to such questions in our previous quantitative analyses. To sum up, our qualitative study complements our previous study by providing more explanations and context for the results we observed in the previous study.
Therefore, we survey 450 participants from 73 GitHub projects (out of the initial 87 projects of our quantitative study). 
Our qualitative analysis is composed of:

\begin{itemize}
    \item Data collection from survey responses of 450 participants of 73 popular open-source projects from GitHub. 
    \item An open-coding analysis of the answers to the open-ended questions of our survey using a thematic analysis technique.
    \item An analysis of the extent to which the survey participants are in accordance with the quantitative results of our prior work.
\end{itemize}

\subsection{\textbf{Quantitative Study}}

Our quantitative study addresses the following
research questions:

\begin{itemize}
    \item \textit{\textbf{\RQone}} 
    The wide adoption of CI is often motivated by the perceived benefits of this practice. For instance, higher confidence in the software product \citep{Duvall2007-tb}, higher release frequency \citep{Stahl:2014:MCI:2562355.2562828}, and the prospect of delivering software updates more quickly \citep{Laukkanen2015-ab}. However, there is a lack of studies that empirically investigate the association between using a CI service and the time-to-deliver of merged PRs. In $RQ1$, we study the delivery time of merged PRs \textit{before} and \textit{after} the use of \textsc{TravisCI}.
	
    \item \textit{\textbf{\RQtwo}} 
    In $RQ1$, we find that only 51.3\% of the projects deliver merged PRs more quickly \textit{after} adopting \textsc{TravisCI}. This result contradicts the assumption that merged PRs would be delivered more quickly {\em after} the adoption of a CI service in most of the projects.
    We then ask the following question: is there another key factor influencing the delivery time of merged PRs {\em after} \textsc{TravisCI} is adopted, such as a significant increase in workload?
		
    \item \textit{\textbf{\RQthree}} 	
    In $RQ1$ and $RQ2$, we study the impact of adopting a CI service on the delivery time of merged PRs. Nevertheless, it is also important to understand what are the characteristics of the delivery time of merged PRs \textit{before} and \textit{after} the use of a CI service. Such information may help decision makers to track and avoid a high delivery time.
\end{itemize}

\subsection{\textbf{Qualitative Study}}

In our qualitative study, we address the following research questions:
    	
\begin{itemize}
    \item \textit{\textbf{\RQfive}}
    In this RQ, we aim to deepen our understanding of how CI may impact the delivery time of PRs. We consult contributors of projects that use CI to obtain qualitative data, which can provide relevant insights to the research community and practitioners. 
    
    \item \textit{\textbf{\RQfour}} 		
    In $RQ4$ we observe that 42.9\% of participants are skeptical regarding the impact of CI on the delivery time of merged PRs. Therefore, in this RQ, we further discuss {\em indirect factors} (i.e., factors that are not necessarily related to CI) that participants believe may also impact the delivery time of PRs.

    \item \textit{\textbf{\RQsix}}
    Given that in $RQ2$ we observe a substantial increase in the number of delivered PRs per release (\textit{after} the adoption of \textsc{TravisCI}), 
    we aim to obtain further insights as to why the increase in the number of delivered PRs occurs. For this purpose, we consult our participants regarding their perceived influence of CI on the release process of their project. For example, is it the case that, because CI encourages the constant packaging of the software, preparing a release is no longer a challenge?	
    
    \item \textit{\textbf{\RQseven}} 
    Intriguingly, in $RQ1$, we find that PRs are merged faster \textit{before} the adoption of \textsc{TravisCI} in 73\% (\nicefrac{46}{63}) of the projects. This result motivates us to further investigate factors that influence the merge time when CI is adopted. 
    
    \item \textit{\textbf{\RQeight}}
    In $RQ1$ and $RQ2$, we observe that there exist a higher number of contributors and PR submissions \textit{after} the adoption of \textsc{TravisCI}. In $RQ8$, we consult our participants to better understand whether CI has any influence on attracting more contributors to open-source projects.	
    
\end{itemize}
	
\textbf{Paper organization.}
The remainder of this paper is
organized as follows. 
In Section~\ref{sec_related_work}, we discuss the related work.
In Section~\ref{sec_empirical_study}, we explain the design of our quantitative and qualitative studies.
In Sections~\ref{sec_quantitative_study_results} and~\ref{sec_qualitative_study_results}, we present the results of our quantitative and qualitative studies, respectively. 
We discuss the practical implications of our observations for the research and practice in software engineering in Section~\ref{sec_discussion}. 
In Section~\ref{sec_threats_to_the_validity}, we discuss the threats to the validity and limitations of our study.
Finally, we draw conclusions in Section~\ref{sec_conclusions}. 
