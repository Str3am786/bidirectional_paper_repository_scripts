

\documentclass[preprint,12pt]{elsarticle}

%% Use the option review to obtain double line spacing
%% \documentclass[preprint,review,12pt]{elsarticle}

%% Use the options 1p,twocolumn; 3p; 3p,twocolumn; 5p; or 5p,twocolumn
%% for a journal layout:
%% \documentclass[final,1p,times]{elsarticle}
%% \documentclass[final,1p,times,twocolumn]{elsarticle}
%% \documentclass[final,3p,times]{elsarticle}
%% \documentclass[final,3p,times,twocolumn]{elsarticle}
%% \documentclass[final,5p,times]{elsarticle}
%% \documentclass[final,5p,times,twocolumn]{elsarticle}

%% For including figures, graphicx.sty has been loaded in
%% elsarticle.cls. If you prefer to use the old commands
%% please give \usepackage{epsfig}

%% The amssymb package provides various useful mathematical symbols
\usepackage{amssymb}
%% The amsthm package provides extended theorem environments
%% \usepackage{amsthm}

%% The lineno packages adds line numbers. Start line numbering with
%% \begin{linenumbers}, end it with \end{linenumbers}. Or switch it on
%% for the whole article with \linenumbers.
%% \usepackage{lineno}

\journal{Journal of Systems and Software}

\begin{document}

\begin{frontmatter}

%% Title, authors and addresses

%% use the tnoteref command within \title for footnotes;
%% use the tnotetext command for theassociated footnote;
%% use the fnref command within \author or \address for footnotes;
%% use the fntext command for theassociated footnote;
%% use the corref command within \author for corresponding author footnotes;
%% use the cortext command for theassociated footnote;
%% use the ead command for the email address,
%% and the form \ead[url] for the home page:
%% \title{Title\tnoteref{label1}}
%% \tnotetext[label1]{}
%% \author{Name\corref{cor1}\fnref{label2}}
%% \ead{email address}
%% \ead[url]{home page}
%% \fntext[label2]{}
%% \cortext[cor1]{}
%% \affiliation{organization={},
%%             addressline={},
%%             city={},
%%             postcode={},
%%             state={},
%%             country={}}
%% \fntext[label3]{}

\title{Addressing Compiler Errors:\\Stack Overflow or Large Language Models?}

%% use optional labels to link authors explicitly to addresses:
%% \author[label1,label2]{}
%% \affiliation[label1]{organization={},
%%             addressline={},
%%             city={},
%%             postcode={},
%%             state={},
%%             country={}}
%%
%% \affiliation[label2]{organization={},
%%             addressline={},
%%             city={},
%%             postcode={},
%%             state={},
%%             country={}}

\author[inst1]{Patricia Widjojo}
\author[inst1]{Christoph Treude}
\affiliation[inst1]{organization={University of Melbourne}, % Department and Organization
            %%addressline={Address One}, 
            city={Parkville},
            postcode={3010}, 
            state={Victoria},
            country={Australia}}

\title{Title\tnoteref{cor1}}
\tnotetext[cor1]{Corresponding author: Christoph Treude (email: christoph.treude@unimelb.edu.au)}



\begin{abstract}
%% Text of abstract
Compiler error messages serve as an initial resource for programmers dealing with compilation errors. However, previous studies indicate that these messages often lack sufficient targeted information to resolve code issues. Consequently, programmers typically rely on their own research to fix errors. Historically, Stack Overflow has been the primary resource for such information, but recent advances in large language models now offer alternative solutions. This study systematically examines 100 compiler error messages from three sources to determine the most effective approach for programmers who encounter compiler errors. Factors considered include search methods on Stack Overflow and the impact of model version and prompt phrasing when using large language models. The results reveal that GPT-4 outperforms Stack Overflow in explaining compiler error messages, the effectiveness of the inclusion of code snippets in Stack Overflow searches depends on the search method, and the results for Stack Overflow differ significantly between Google and StackExchange API searches. Furthermore, GPT-4 surpasses GPT-3.5, with ``How to fix'' prompts yielding superior outcomes compared to ``What does this error mean'' prompts. These results offer valuable guidance for programmers seeking assistance with compiler error messages, underscoring the transformative potential of advanced large language models like GPT-4 in the realm of debugging and opening new avenues of exploration for researchers in AI-assisted programming.
\end{abstract}


\begin{keyword}
%% keywords here, in the form: keyword \sep keyword
Compiler errors \sep Stack Overflow \sep large language models 
%% PACS codes here, in the form: \PACS code \sep code
%%\PACS 0000 \sep 1111
%% MSC codes here, in the form: \MSC code \sep code
%% or \MSC[2008] code \sep code (2000 is the default)
%%\MSC 0000 \sep 1111
\end{keyword}

\end{frontmatter}

%% \linenumbers

%% main text
\section{Sample Section Title}
\label{sec:sample1}


Lorem ipsum dolor sit amet, consectetur adipiscing \citep{Fabioetal2013} elit, sed do eiusmod tempor incididunt ut labore et dolore magna \citet{Blondeletal2008} aliqua. Ut enim ad minim veniam, quis nostrud exercitation ullamco laboris nisi ut aliquip ex ea commodo consequat. Duis aute irure dolor in reprehenderit in voluptate velit esse cillum dolore eu fugiat nulla pariatur. Excepteur sint occaecat cupidatat non proident, sunt in culpa qui officia deserunt mollit \citep{Blondeletal2008,FabricioLiang2013} anim id est laborum.

Lorem ipsum dolor sit amet, consectetur adipiscing elit, sed do eiusmod tempor incididunt ut labore et dolore magna aliqua. Ut enim ad minim veniam, quis nostrud exercitation ullamco laboris nisi ut aliquip ex ea commodo consequat. Duis aute irure dolor in reprehenderit in voluptate velit esse cillum dolore eu fugiat nulla pariatur. Excepteur sint occaecat cupidatat non proident, sunt in culpa qui officia deserunt mollit anim id est laborum see appendix~\ref{sec:sample:appendix}.

%% The Appendices part is started with the command \appendix;
%% appendix sections are then done as normal sections
\appendix

\section{Sample Appendix Section}
\label{sec:sample:appendix}
Lorem ipsum dolor sit amet, consectetur adipiscing elit, sed do eiusmod tempor section \ref{sec:sample1} incididunt ut labore et dolore magna aliqua. Ut enim ad minim veniam, quis nostrud exercitation ullamco laboris nisi ut aliquip ex ea commodo consequat. Duis aute irure dolor in reprehenderit in voluptate velit esse cillum dolore eu fugiat nulla pariatur. Excepteur sint occaecat cupidatat non proident, sunt in culpa qui officia deserunt mollit anim id est laborum.

%% If you have bibdatabase file and want bibtex to generate the
%% bibitems, please use
%%
\bibliographystyle{elsarticle-num-names} 
\bibliography{cas-refs}

%% else use the following coding to input the bibitems directly in the
%% TeX file.

% \begin{thebibliography}{00}

% %% \bibitem[Author(year)]{label}
% %% Text of bibliographic item

% \bibitem[ ()]{}

% \end{thebibliography}
\end{document}

\endinput
%%
%% End of file `elsarticle-template-num-names.tex'.
