
%THIS PREAMBLE CONTAINS A LIST OF COMMANDS THAT ARE USEFUL FOR, AND CAN BE INCLUDED IN, EACH CHAPTER!

\usepackage{multicol}        % used for the two-column index
\usepackage[bottom]{footmisc}% places footnotes at page bottom
\usepackage{enumitem}

\usepackage{newtxtext}       % 

%REMOVED NEXT LINE SINCE IT IS NOT WORKING FOR A. SEREBRENIK AND IT DOES NOT SEEM TO BE NEEDED:
%\usepackage[varvw]{newtxmath}       % selects Times Roman as basic font

\usepackage{url}
\usepackage{pifont}

\usepackage{algorithm}% http://ctan.org/pkg/algorithms
\usepackage{algpseudocode}% http://ctan.org/pkg/algorithmicx
\usepackage{booktabs}
%\usepackage[tight,footnotesize]{subfigure} %TOM: COMMENTED OUT SINCE CONFLICTING WITH subcaption PACKAGE!
\usepackage{multirow}
\usepackage{wrapfig}
\usepackage{listings}
\usepackage{siunitx}
\usepackage{pgfplotstable}
\usepackage{array}
\usepackage{subcaption}
\usepackage{newfloat}

\usepackage{pdflscape}
\usepackage{xspace}

\usepackage{diagbox} % useful for creating top left cell with explanations of tables heading 

%Semantic labeling of chapters:
%Systematically use \label{XYZ:…} for all labels used in your chapter, where XYZ is the letter code assigned to your chapter. The unique letter codes are:
%INT (for the INTtroduction to software ecosystems chapter by Mens and De Roover)
%PPM (for the Promises and Perils of Mining software ecosystem data by Kula, Inoue, Treude)
%SLU (for the Software Library Usage pattern mining chapter by Tien Nguyen)
%EMO (for the Emotion Analyses chapter by Novieli and Serebreni)
%SWH (for the SoftWare Heritage chapter by Di Cosmo and Zacchiroli)
%FRK (for the variant FoRK analysis chapter by Businge and Demeyer)
%COL (for the COLlateral evolution chapter by Lo, Bowen, Yang)
%WFA (for the WorkFlow Automation chapter by Wessel, Mens, …)
%IAC (for the Infrastructure as Code chapter by De Roover, Zeraouli, Opdebeeck)
%MDE (for the Model-Driven Engineering chapter by Di Ruscio, Nguyen, Pierantonio)
%DIS (for the Data-Intensive Software ecosystem chapter by Cleve et al)

%Refere to chapters, sections, figures, tables in a chapter using the following commands using the chapter's letter code:
\newcommand{\chap}[1]{Chapter~\ref{#1:ch}}
\newcommand{\sect}[2]{Section~\ref{#1:sec:#2}}
\newcommand{\subsect}[2]{Subsection~\ref{#1:subsec:#2}}
\newcommand{\fig}[2]{Figure~\ref{#1:fig:#2}}
\newcommand{\tab}[2]{Table~\ref{#1:tab:#2}}
\newcommand{\lst}[2]{Listing~\ref{#1:lst:#2}}

%Example for the Introduction chapter: \chap{INT}, \sect{INT}{definition}, \fig{INT}{figlabel}, \tab{INT}{tablabel}



\usepackage[]{todonotes}
\newcommand{\tom}[1]{\todo[color=yellow!40, inline]{\footnotesize{Tom: #1}}}
\newcommand{\coen}[1]{\todo[color=blue!40, inline]{\footnotesize{Coen: #1}}}
\newcommand{\anthony}[1]{\todo[color=orange!40, inline]{\footnotesize{Anthony: #1}}}

\newcommand{\alexander}[1]{\todo[color=yellow!40, inline]{\footnotesize{Alexander: #1}}}
\newcommand{\nicole}[1]{\todo[color=blue!40, inline]{\footnotesize{Nicole: #1}}}

\newcommand{\raula}[1]{\todo[color=yellow!40, inline]{\footnotesize{Raula: #1}}}
\newcommand{\katsuro}[1]{\todo[color=blue!40, inline]{\footnotesize{Katsuro: #1}}}
\newcommand{\christoph}[1]{\todo[color=blue!40, inline]{\footnotesize{Christoph: #1}}}

\newcommand{\roberto}[1]{\todo[color=yellow!40, inline]{\footnotesize{Roberto: #1}}}
\newcommand{\stefano}[1]{\todo[color=blue!40, inline, inlinewidth=2cm]{\footnotesize{Stefano: #1}}}

\newcommand{\serge}[1]{\todo[color=yellow!40, inline]{\footnotesize{Serge: #1}}}
\newcommand{\john}[1]{\todo[color=blue!40, inline]{\footnotesize{John: #1}}}

\newcommand{\mairieli}[1]{\todo[color=pink!40, inline]{\footnotesize{Mairieli: #1}}}
\newcommand{\pooya}[1]{\todo[color=orange!40, inline]{\footnotesize{Pooya: #1}}}

\newcommand{\tien}[1]{\todo[color=yellow!40, inline]{\footnotesize{Tien: #1}}}

\newcommand{\david}[1]{\todo[color=yellow!40, inline]{\footnotesize{David: #1}}}
\newcommand{\xu}[1]{\todo[color=blue!40, inline]{\footnotesize{Xu: #1}}}
\newcommand{\zhou}[1]{\todo[color=orange!40, inline]{\footnotesize{Zhou: #1}}}

\newcommand{\davide}[1]{\todo[color=yellow!40, inline]{\footnotesize{Davide: #1}}}
\newcommand{\phuong}[1]{\todo[color=blue!40, inline]{\footnotesize{Phuong: #1}}}
\newcommand{\alfonso}[1]{\todo[color=orange!40, inline]{\footnotesize{Alfonso: #1}}}

\newcommand*{\ie}{i.e.,\@\xspace}
\newcommand*{\eg}{e.g.,\@\xspace}
\newcommand*{\etal}{\emph{et~al.}\@\xspace}
\newcommand{\cmark}{\ding{51}}%
\newcommand{\xmark}{\ding{55}}%
\newcommand\revised[1]{\textcolor{blue}{#1}}

%%%% SOME COMMANDS NEEDED IN DIFFERENT CHAPTERS %%%%
\newcommand{\gh}{GitHub\xspace}
\newcommand{\github}{{GitHub}\xspace}
\newcommand{\gitea}{{Gitea}\xspace}
\newcommand{\gitlab}{GitLab\xspace}
\newcommand{\bitbucket}{BitBucket\xspace}
\newcommand{\sourceforge}{{SourceForge}\xspace}
\newcommand{\git}{\textit{git}\xspace}
\newcommand{\docker}{{Docker}\xspace}
\newcommand{\dockerhub}{{Docker Hub}\xspace}
\newcommand{\ansible}{{Ansible}\xspace}
\newcommand{\stackoverflow}{{Stack Overflow}\xspace}
\newcommand{\openstack}{{OpenStack}\xspace}
\newcommand{\npm}{{npm}\xspace}
\newcommand{\gha}{GitHub Actions\xspace}
\newcommand{\actions}{GitHub Actions\xspace}

\newtheorem{Definition}{Definition}
\newtheorem{Algorithm}{Algorithm}
\newtheorem{Claim}{Claim}
\newtheorem{Lemma}{Lemma}
\newtheorem{Theorem}{Theorem}

%%%% SOME COMMANDS NEEDED IN DIFFERENT CHAPTERS %%%%

%FOR LIBRARIES CHAPTER:
\newcommand{\code}[1]{{\footnotesize\texttt{#1}}}
\newcommand{\model}{\textsc{Groum}\xspace}
\newcommand{\miner}{\textsc{GrouMiner}\xspace}
\newcommand{\patt}{\textsc{PattExplorer}\xspace}

%%%%%%%%%%% NEEDED FOR GITHUB AUTOMATION CHAPTER %%%%%%%%%%%%%%%%%
%%%%%%%%%%% for listings of yaml code %%%%%%%%%%%%%%%%%
\newcommand\YAMLcolonstyle{\color{red}\mdseries}
\newcommand\YAMLkeystyle{\color{black}\bfseries}
\newcommand\YAMLvaluestyle{\color{blue}\mdseries}

\makeatletter

\newcommand\language@yaml{yaml}

\expandafter\expandafter\expandafter\lstdefinelanguage
\expandafter{\language@yaml}
{
  keywords={true,false,null,y,n},
  keywordstyle=\color{darkgray}\bfseries,
  basicstyle=\YAMLkeystyle,                                 % assuming a key comes first
  sensitive=false,
  comment=[l]{\#},
  morecomment=[s]{/*}{*/},
  commentstyle=\color{purple}\ttfamily,
  stringstyle=\YAMLvaluestyle\ttfamily,
  moredelim=[l][\color{orange}]{\&},
  moredelim=[l][\color{magenta}]{*},
  moredelim=**[il][\YAMLcolonstyle{:}\YAMLvaluestyle]{:},   % switch to value style at :
  morestring=[b]',
  morestring=[b]",
  literate =    {---}{{\ProcessThreeDashes}}3
                {>}{{\textcolor{red}\textgreater}}1     
                {|}{{\textcolor{red}\textbar}}1 
                {\ -\ }{{\mdseries\ -\ }}3,
}

% switch to key style at EOL
\lst@AddToHook{EveryLine}{\ifx\lst@language\language@yaml\YAMLkeystyle\fi}
\makeatother

\newcommand\ProcessThreeDashes{\llap{\color{cyan}\mdseries-{-}-}}
%%%%%%%%%%% End of code for listings of yaml code %%%%%%%%%%%%%%%%%

%%%%%%%%%% NEEDED FOR PROMISES AND PERILS CHAPTER %%%%%%%%%%%%%%%%%
\usepackage{wasysym}
\newcommand{\use}{Use}
\newcommand{\useBy}{UsedBy}
%%%%%%%%%% END OF CODE NEEDED FOR PROMISES AND PERILS CHAPTER %%%%%%%%%%%%%%%%%


%%%%%%%%%%% NEEDED FOR IAC CHAPTER %%%%%%%%%%%%%%%%%
\DeclareFloatingEnvironment[
    fileext=los,
    listname={List of Listings},
    name=Listing,
    placement=!h,
    within=none,
]{listing}

\lstset{
  basicstyle=\footnotesize\ttfamily,
  breaklines=true,
  breakatwhitespace=true,
  commentstyle=\color{purple}, % comment color
  keywordstyle=\color{blue}, % keyword color
  stringstyle=\color{black},
  escapeinside={<@}{@>},
  xleftmargin=1em,
  language=bash,
  numbersep=8pt,
  breakindent=1em,
  postbreak=\postbreak,
  numbers=left,
  numberstyle=\tiny\ttfamily,
}

\lstdefinestyle{docker}{
  language=bash,
  morekeywords={RUN,FROM,MAINTAINER},
  showstringspaces=false,
  frame=none,
}
\lstdefinelanguage{ansible}{
  morekeywords={name,vars,hosts,tasks,roles,role},
  keywordstyle=\bfseries,
  morecomment=[l][\textit]\#,
  morecomment=[s][\bfseries]{\{\{}{\}\}},
}
\lstdefinestyle{ansible}{
  language=ansible,
  basicstyle=\scriptsize\ttfamily,
}

\def\postbreak{%
  \raisebox{0ex}[0ex][0ex]{\ensuremath{\hookrightarrow\space}}}
%%%%%%%%%%% END OF CODE NEEDED FOR IAC CHAPTER %%%%%%%%%%%%%%%%%

%%%%%%%%%%% NEEDED FOR MDE CHAPTER %%%%%%%%%%%%%%%%%
\lstdefinestyle{searchstringstyle}{
	basicstyle=\ttfamily\footnotesize,
	breakatwhitespace=false,         
	breaklines=true,                 
	captionpos=t,                    
	keepspaces=true,                 
	numbers=none,                    
	numbersep=5pt,                  
	showspaces=false,                
	showstringspaces=false,
	showtabs=false,                  
	tabsize=2,
	frame=single
}
%%%%%%%%%%% END OF CODE  FOR MDE CHAPTER %%%%%%%%%%%%%%%%%

%%%%%%%%%%% NEEDED FOR LIBRARIES CHAPTER %%%%%%%%%%%%%%%%%
\definecolor{mauve}{rgb}{0.58,0,0.82}
\definecolor{dkgreen}{rgb}{0,0.6,0}
\definecolor{gray}{rgb}{0.5,0.5,0.5}

\lstset{frame=tb,
  language=Java,
  aboveskip=3mm,
  belowskip=3mm,
  showstringspaces=false,
  columns=flexible,
  basicstyle={\small\ttfamily},
  numbers=left,
  numberstyle=\tiny\color{gray},
  keywordstyle=\color{blue},
  commentstyle=\color{dkgreen},
  stringstyle=\color{mauve},
  breaklines=true,
  breakatwhitespace=true,
  tabsize=4
}
%%%%%%%%%%% END OF CODE  FOR LIBRARIES CHAPTER %%%%%%%%%%%%%%%%%



% \newcommand{\Brittany}[1]{[\textbf{Brittany}:~{\color{purple} #1}]}
\newcommand{\Markus}[1]{[\textbf{Markus}:{\color{magenta} #1}]}
\newcommand{\Christoph}[1]{[\textbf{Christoph}:~{\color{blue} #1}]}

\newcommand{\ts}{TypeScript}
\newcommand{\js}{JavaScript}

%research questions
\newcommand{\rqone}{What errors does \ts\ detect in NPM documentation?}
\newcommand{\rqtwo}{How does error detection differ between ESLint and \ts?}
\newcommand{\rqthree}{What is the impact of NCC on the set of NPM snippets?}
\newcommand{\rqfour}{How does NCC compare to NCQ's code corrections?}
\newcommand{\rqfive}{How does NCC compare to manual fixes?}

%hyperref setup
\hypersetup{
    colorlinks=true,
    linkcolor=blue,
    anchorcolor=blue,
    filecolor=blue,      
    citecolor=blue,
    urlcolor=blue
}

%define colours
\definecolor{lightgreen}{rgb}{0.8,1,0.8}
\definecolor{lightyellow}{rgb}{1,1,0.8}
\definecolor{lightred}{rgb}{1,0.8,0.8}
\definecolor{baseColour}{RGB}{255, 128, 85}
\definecolor{ncqColour}{RGB}{118, 165, 175}
\definecolor{ncqColour2}{RGB}{162, 196, 201}

%algorithm comment style
\newcommand\mycommfont[1]{\footnotesize\ttfamily\textcolor{blue}{#1}}
\SetCommentSty{mycommfont}
%alg options
\SetAlFnt{\small}
\SetAlCapFnt{\small}
\SetAlCapNameFnt{\small}
\algsetup{linenosize=\tiny}

%lsting setup
\lstdefinelanguage{javascript}{
    keywords={typeof, new, true, false, catch, function, return, null, catch, switch, var, if, in, while, do, else, case, break},
    keywordstyle=\bfseries,
    ndkeywords={class, export, boolean, throw, implements, import, this},
    ndkeywordstyle=\color{darkgray}\bfseries,
    identifierstyle=\color{black},
    sensitive=false,
    comment=[l]{//},
    morecomment=[s]{/*}{*/},
    commentstyle=\color{blue}\ttfamily,
    stringstyle=\color{purple}\ttfamily,
    morestring=[b]',
    morestring=[b]"
}
\newcommand{\lstbg}[3][0pt]{{\fboxsep#1\colorbox{#2}{\strut #3}}}
\lstdefinelanguage{diff}{
    morecomment=[f][\lstbg{lightgreen}]+,
    morecomment=[f][\lstbg{lightred}]-,
    morecomment=[f][\lstbg{lightyellow}]\\,
    keywords={typeof, new, true, false, catch, function, return, null, catch, switch, var, if, in, while, do, else, case, break},
    keywordstyle=\bfseries,
    ndkeywords={class, export, boolean, throw, implements, import, this},
    ndkeywordstyle=\color{darkgray}\bfseries,
    identifierstyle=\color{black},
    sensitive=false,
    morecomment=[s]{/*}{*/},
    commentstyle=\color{blue}\ttfamily,
    stringstyle=\color{purple}\ttfamily,
    morestring=[b]',
    morestring=[b]"
}
\lstset{
    frame = single,
    tabsize=1,
    showstringspaces=false,
    language=javascript,
    basicstyle=\fontsize{7}{6}\selectfont\ttfamily,
    breaklines=true, 
    breakatwhitespace=false,   % sets if automatic breaks should only happen at
    numbers=left, 
    firstnumber=1,
    numberstyle=\tiny, 
    stepnumber=1, 
    numbersep=5pt,
    float,
    aboveskip=0pt,
    belowskip=0pt,
    xleftmargin=.02\textwidth,
    xrightmargin=.02\textwidth,
    escapeinside={\%*}{*)},          % if you want to add LaTeX within  % your code 
}

\addto\extrasenglish{%
  \def\chapterautorefname{Chapter}%
  \def\sectionautorefname{Section}%
  \def\algorithmautorefname{Algorithm}%
  \def\subsectionautorefname{Section}%
  \def\subsubsectionautorefname{Section}%
}

%tikz
\usetikzlibrary{
    arrows.meta,
    chains,
    scopes,
    positioning,
    shapes.geometric
}
% layers
\pgfdeclarelayer{background}
\pgfdeclarelayer{foreground}
\pgfsetlayers{background,main,foreground}

% styles     
\tikzset{
    processDiagram/.style={
        startstop/.style={
            rectangle,
            draw,
            minimum width=3cm, 
            minimum height=0.8cm,
            join = by arrow,
            fill = black!0,
        },
        decision/.style = {
            diamond,
            aspect=1.7,
            draw,
            minimum width=2.5cm, 
            minimum height=1cm, 
            align=center,
            join = by arrow,
            fill = black!0,
        },
        process/.style = {
            rectangle, 
            rounded corners, 
            draw,
            minimum width=3cm,
            minimum height=0.8cm, 
            align=center,
            join = by arrow,
            fill = black!0,
        },
        process2/.style = {
            rectangle, 
            rounded corners, 
            draw,
            minimum width=3cm,
            minimum height=0.8cm, 
            align=center,
            join = by arrow,
            fill = ncqColour2,
        },
        line/.style = {
            -
        },
        invisible/.style = {
            join = by line,
            minimum width=0cm, 
            minimum height=0cm,
            inner sep=0pt,
        },
        invisititle/.style ={
            font = \bfseries,
            join = by arrow,
            outer sep = 5pt,
        },
        title/.style = {
            font = \bfseries,
            fill = black!20,
            join = by arrow,
            minimum width = 3cm,
            minimum height=0.8cm,
            draw,
        },
        output/.style ={
            join = by line,
            minimum width=0cm, 
            minimum height=0cm,
            inner sep=0pt,
            font = \itshape,
            align = center,
        },
        arrow/.style = {
            ->
        },
    }
}

