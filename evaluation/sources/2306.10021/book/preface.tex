%%%%%%%%%%%%%%%%%%%%%%preface.tex%%%%%%%%%%%%%%%%%%%%%%%%%%%%%%%%%%%%%%%%%
% sample preface
%
% Use this file as a template for your own input.
%
%%%%%%%%%%%%%%%%%%%%%%%% Springer %%%%%%%%%%%%%%%%%%%%%%%%%%

\preface

%{\color{red}A preface\index{preface} is a book's preliminary statement, usually written by the \textit{author or editor} of a work, which states its origin, scope, purpose, plan, and intended audience, and which sometimes includes afterthoughts and acknowledgments of assistance. 
%When written by a person other than the author, it is called a foreword. The preface or foreword is distinct from the introduction, which deals with the subject of the work.
%Customarily \textit{acknowledgments} are included as last part of the preface.}

The discipline of \emph{software engineering} emerged in 1969 as a result from the first international conference sponsored by the NATO Science Committee \cite{Naur1969}.
Over a period spanning several decades, the discipline has given rise to increasingly advanced processed and tool support for maintaining and evolving ever more complex and interconnected software products.
Software engineering tools offer support for a wide range of activities including project management, version control, configuration management, collaborative coding, quality assurance, dependency management, continuous integration and deployment, containerisation and virtualisation.

Since the seminal book by Messerschmitt and Szyperski in 2003 \cite{messerschmitt2003software}, \emph{software ecosystems} have become a very active topic of research in software engineering. 
As the different chapters of this book will reveal, software ecosystems exist in many different forms and flavours, so it is difficult to provide a unique encompassing definition.
 But a key aspect of such software ecosystems is that software products can no longer be considered or maintained in isolation, since they belong to ever more interconnected and interdependent networks of co-evolving software components and systems.
This was enabled by technological advances in various domains such as component-based software engineering, global software development and cloud computing. 
The ever increasing importance of social coding \cite{Dabbish2012}, aiming to develop software in a collaborative way, has made software ecosystems indispensable to software practitioners, in commercial as well as open source settings.
This has lead to the widespread use and popularity of large registries of reusable software libraries for a wide variety of programming languages, operating systems and project-specific software communities.



\paragraph{\textbf{How this book originated}}

The idea to write this book originated from a large inter-university fundamental research project --called SECOAssist\footnote{See \url{https://secoassist.github.io}}-- that studied the technical aspects of software ecosystems, in order to provide tools and techniques to assist contributors, maintainers, managers and other ecosystems stakeholders in their daily activities. This project was financed by the Belgian regional science foundations F.R.S.-FNRS and FWO-Vlaanderen under the ``Excellence of Science'' call. It took place from 2018 till 2023 and spawned many research results in the form of scientific publications, PhD dissertations, open source tools and datasets. The three co-editors of this book, Tom Mens, Coen De Roover and Anthony Cleve, were principal investigators of the project together with Serge Demeyer, leading the research efforts of their respective teams.

The conducted research was quite diverse, covering a wide range of software ecosystems and focussing on challenging technical and socio-technical aspects thereof.
Among others, we empirically analysed a wide range of maintenance issues related to software ecosystems of reusable software libraries. % (such as npm for JavaScript, Maven for Java, RubyGems for Ruby, Packagist for PHP, CRAN for R, CPAN for Perl, NuGet for .NET).
Example maintenance issues studied include outdatedness, security vulnerabilities, semantic versioning, how to select, replace and migrate software libraries, and contributor abandonment.
We also investigated advanced testing techniques such as test amplification and test transplantation and how to apply them at the ecosystem level.
We studied socio-technical aspects in the \github ecosystem, including the phenomenon of forking, the use of development bots and the integrated CI/CD workflow infrastructure of \github Actions.
We also analysed issues around database usage and migration in ecosystems for data-intensive software.
Finally, we explored maintenance issues in a range of emerging software ecosystems, such as the \dockerhub containerisation ecosystem, the Intrastructure-as-Code ecosystem forming around \ansible Galaxy, the Q\&A ecosystem of \stackoverflow, the \openstack ecosystem, and the \actions ecosystem.

\paragraph{\textbf{Who contributed to this book}}

This book aims to further expand upon and report about the software ecosystems research that has been conducted in recent years, going well beyond the results achieved by the SECOAssist research project.
To this end, we invited some of the most renowned researchers worldwide who have made significant contributions to the field.
Their names and affiliations can be found just before the book's table of contents.
The chapters they have contributed focus on the nature of particular software ecosystems, or on specific domain-specific tooling and analyses to understand, support, and improve those ecosystems.

Following the spirit of open science and collaborative development practices, we used a GitHub repository during the process of writing and reviewing the book chapter. Each contributor had access to the material of each chapter, and each chapter was peer-reviewed by at least three different contributors using the repository's discussion forum.

\paragraph{\textbf{What this book IS NOT about}}

Previously published books on the topic of software ecosystems have primarily focused on the business aspects of software ecosystems \cite{messerschmitt2003software,Popp2010,Jansen2013book}.
We fully agree these are very important aspects, but regretted the absence of a complementary book focusing on the technical aspects related to empirically analysing, supporting, and improving software ecosystems.
This is the main reason why we decided to create this book.

Even though we have tried to be as comprehensive as possible, we acknowledge that this book may not cover all important research topics related to software ecosystems. We apologise to the reader if specific technical or other aspects related to software ecosystems and their analysis have not been sufficiently covered.

\paragraph{\textbf{Who this book is intended for}}

This book is intended for all those practitioners and researchers interested in developing tool support for or in the empirical analysis of software ecosystems. The reader will find the contributed chapters to cover a wide spectrum of social and technical aspects of software ecosystems, each including an overview of the state of the art.

While this book has not been written as a classical textbook, we believe that it can be used as supplementary material to present software ecosystems research during advanced (graduate or postgraduate) software engineering lectures and capita selecta. This is exactly what we, book editors, intend to do as well in our own university courses.
For researchers, the book can be used as a starting point for exploring the wealth of software ecosystems results surveyed in each chapter.

\paragraph{\textbf{How this book is structured}}

This book is composed of five parts, each containing two or three contributed chapters.

Part~\ref{part:representing} contains three chapters on software ecosystem representations. It starts with an introductory chapter (\chap{INT}) that provides a historical account of the origins of software ecosystems. This chapter sets the necessary context about the domain of software ecosystems by highlighting its different perspectives, definitions and representations. It also provides many concrete examples highlighting the variety of software ecosystems that have emerged during the previous decades.
\chap{SWH} focuses on the Software Heritage open science ecosystem. This ecosystem has been recognised by UNESCO because of its ongoing effort to preserve and provide access to the digital heritage of free and open source software. The chapter focuses on important aspects such as open science and research reproducibility, and also discusses some of the techniques that are required to maintain and query this massive software ecosystem.
\chap{PPM} reflects on software ecosystems composed of many interdependent components, and the challenges that researchers are facing to mine such ecosystems. The authors propose a list of promises and perils related to these challenges.

Part~\ref{part:analysing} of the book contains two chapters that focus on different ways and techniques of analysing software ecosystems.
\chap{SLU} focuses on technical aspects of how to mine software library usage patterns in ecosystems of reusable software libraries.
\chap{EMO} focuses on social aspects in software ecosystems, by analysing how emotions play a role in the context of developer communication and interaction. It presents a range of sentiment analysis techniques and tools that can be used to carry out such analyses.

Part \ref{part:evolution} of the book contains two chapters that focus on aspects related to the evolution of software ecosystems.
\chap{FRK} focuses on the phenomenon of \emph{forking} of software repositories in social coding ecosystems such as \github. The chapter studies the prevalent phenomenon of variant forks as a reuse mechanism to split of software projects and steer them into a new direction. The focus is on how such variant forks continue to be maintained over time and to which extent they co-evolve with the main repository they originated from.
\chap{COL} discusses the effect of \emph{collateral evolution} in software ecosystems. Collateral evolution is a type of adaptive maintenance \cite{LientzSwanson1980,Chapin2001} that is required to keep a software system functional when its surrounding technological environment is facing changes that are beyond the control of the system itself. A typical example of this are external libraries that subject to frequent changes that often require the software system to adapt to them. This phenomenon is explored for three software ecosystems (the Linux kernel, Android apps, and machine learning software), and techniques are presented to support collateral evolution in those ecosystems.


Part~\ref{part:process} of the book looks at what could be called process-centered software ecosystems. Such ecosystems are brought about by the increasing automation of software engineering processes. \chap{WFA} studies development workflow automation tools, considering the \github Actions ecosystems and the use of development bots.
\chap{IAC} looks at the ecosystems that have formed around containerisation and configuration management tools, focusing specifically on the \dockerhub and Ansible Galaxy ecosystems.

Finally, Part~\ref{part:models} focuses on what could be called model-centered software ecosystems.
\chap{MDE} considers ecosystems stemming from \emph{model-driven software engineering}. Such ecosystems contain software model artefacts and associated tools. The chapter focuses on how machine learning and deep learning techniques can be used to understand and analyse the artefacts contained in such ecosystems.
Last but not least, \chap{DIS} looks at \emph{data-intensive software ecosystems}, which are ecosystems in which database models and their corresponding tools form a crucial role. The chapter focuses on techniques to mine, analyse sand visualise such ecosystems. It also reports on empirical analysis based on such techniques.

\paragraph{\textbf{Acknowledgments}}
We thank the FWO-Vlaanderen and F.R.S-FNRS Science Foundations in Belgium for the generous research funding they have provided to us for our research on software ecosystems. We also thank our respective universities in Mons, Brussels, and Namur for having given us the opportunity, environment and resources to carry out our research and to enable this research collaboration. We express our gratitude to all chapter authors of this book for having taken the time and effort to contribute high-quality chapters while respecting the imposed deadlines. Last but not least, we thank our families and friends for their support.

\vspace{\baselineskip}
\begin{flushright}\noindent
    Belgium,\hfill {\it Tom  Mens, University of Mons}\\
    March 2023\hfill {\it Coen  De Roover, Vrije Universiteit Brussel}\\
    \hfill {\it Anthony  Cleve, University of Namur}\\

\end{flushright}


