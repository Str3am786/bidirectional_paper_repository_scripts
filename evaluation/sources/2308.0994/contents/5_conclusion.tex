
We propose SYNAuG which deals with long-tailed recognition, model fairness, and robustness to spurious correlations as data imbalance problems.
% SYNAuG consists of three stages: (1) uniformization of training data, (2) model training, and (3) classifier fine-tuning.
The development process of the machine learning model 
% development process 
can be roughly divided into 
data curation, model training, and model management.
Since the data comes first in the process,
% among these parts,
a flaw in the dataset affects the subsequent phases; thus, it is crucial.
% Therefore, careless handling of data can cause social and ethical confusion.
Our study suggests the importance of controlling imbalance from the data perspective.
We believe that taking the controllability of data is a promising research direction to resolve the early bottleneck in machine learning model development. 
While we focus on the data perspective, improving the model in multiple views is necessary for effective solutions to data imbalance.
We conclude our work with the following discussions.

\paragraph{Other perspectives.}
We have suggested the usage of synthetic data from pre-trained generative models as a new data perspective baseline for the data imbalance problem, but there may be other perspectives.
% to address the data imbalance.
We observed a gradual performance decline when substituting real data with synthetic data, suggesting the potential need for domain adaptation.
There could be future research directions, \eg, more sophisticated data augmentation, automated data curation, transfer learning, the usage of differentiability of the generative models, and comprehending taxonomies across classes.
% In addition, studying the contamination of real data by synthetic data would provide guidelines about how to exploit the generated data.
While we emphasize that our work suggests a promising way to redraw the direction to overcome the data imbalance problems in the data perspective, more interesting future work will come with integrating multiple levels.

% \paragraph{Implications of synthetic data supplements.}
% As generative models have a huge impact on the public, anyone can upload the result on the web without indicating whether it is synthetic or not.
% Not only are humans exposed to synthetic information, but DNNs can also be exposed to this information.
% Although we focus on the supplements of original data with synthetic data in this work, studying the contamination of real data by synthetic data would provide guidelines about how to exploit the generated data.

\paragraph{Limitations of using generative models.}
The generation of synthetic data demands additional time and computational resources. 
While the curation of a real dataset requires enormous time, human, and financial resources, the process of generating synthetic data becomes increasingly challenging as the volume of data needed increases.
Also, the quality of the synthesized data varies depending on factors such as the prompt, guidance level, and step value of the diffusion model, impacting the overall performance of the model. 
However, since generative models have been continuously developed in terms of sample quality, time efficiency, and controllability,
we believe that exploiting generation models as a data source is a promising research direction as the performance of generation models is improved.

% \paragraph{Social Impact.}

