\section{Sensitivity and Systematics}
\label{sec:sensitivity_and_systematics}

\subsection{Anticipated Sensitivity}

The molecule-shot-noise limited (SNL) sensitivity for a SOF frequency measurement in a beam is given by
%\cite{cho1990thesis}
\begin{equation}
    \delta \nu_\mathrm{SNL} = \frac{1}{2\pi T}\frac{1}{C_\mathrm{SOF}}\frac{1}{\sqrt{N_d N_p}} Z.
\end{equation}
Here $T$ is the total interaction time in the MI region, $L_{\rm SOF}/\langle v_z\rangle$,
$C_\mathrm{SOF}$ is the SOF fringe constrast, $N_d$ is the number of molecules detected per beam pulse, and $N_p$ is the number of pulses used in the measurement. \CENTREX\ is expected to achieve $C_{\rm SOF} \approx 1$, as in the ACME electron EDM measurement that uses a similar detection scheme \cite{andreev_improved_2018}.  The factor $Z$, which takes values $1\! <\! Z\! < \!\sqrt{2}$, accounts for excess noise that can arise when detecting fluorescence from a partially-closed cycling transition \cite{PhysRevA.98.053823}.  To be conservative, we take $Z=\sqrt{2}$.

We estimate $N_d$ as follows.  The measured time\hyph averaged beam intensity is $5\times 10^{12}\,$molecules/state/sr/s (Sec.\ \ref{sec:beamsource}), corresponding to $1\times 10^{11}\,$molecules/state/sr/pulse. Rotational cooling results in a simulated 24-fold increase in the number of molecules in the desired $F=0,m_F=0$ state (Sec.\ \ref{sec:rotational_cooling}).
Combining the simulated state transfer efficiencies of SPA (99\%), SPB (96\%), and SPC (96\%) (Secs.\ \ref{sec:state_preparation_region_a}, \ref{sec:state_preparation_B} and \ref{sec:state_preparation_region_c}), giving a cumulative transfer efficiency of $\approx 91\%$.  Given the distance of the FD region from the source and its transverse area 18 mm $\times$ 30 mm, the solid angle subtended by the FD region is $1.3\times 10^{-5}\,$sr. From simulations of the EQL (Sec.\ \ref{sec:electrostatic_lens}), the gain in signal from energizing the lens should be 24. Combining these with the anticipated detection efficiency of 90\%, we expect $N_d \approx 6.1\ee{8}\,$molecules/pulse to be detected in the FD region.

Combined with $\langle v_z \rangle = 184\,$m/s and $L_{\rm SOF} \approx 2.5\,$ m, we estimate a shot noise-limited frequency shift sensitivity of 
\begin{equation}
    \label{eq:shot_noise}
	\delta \nu_\mathrm{SNL} \approx \frac{0.7}{\sqrt{N_p}}\mHz.
\end{equation}
With a total measurement time of 300 hours and a $50\,$Hz repetition rate, corresponding to $N_p \approx 5.4\ee{7}$, the final sensitivity is projected to be $\delta \nu_\mathrm{SNL} \approx 90\,$nHz. Recalling that the $CPV$ energy shift is $2\Delta_{\rm CPV}$, this is equivalent to $\delta \Delta_\mathrm{CPV} \approx 45$ nHz.
For comparison, the previous best limit achieved $\delta\Delta_\mathrm{CPV} \approx 120~\mu$Hz \cite{cho1991search}. Hence we anticipate that \CENTREX\ can achieve a 2500-fold statistical improvement over the CPV limits given in Eq.~\ref{eq:prev_best_lims}.

\subsection{Extracting \texorpdfstring{$\Delta_{\rm CPV}$}{dCVP}}

We will extract the CPV energy shift $\Delta_{\rm CPV}$ from our data using schemes similar to those used in prior experiments \cite{wilkening1984search,cho1991tight,regan2002new,andreev_improved_2018} and described briefly here.

Recall that under ideal conditions, the signal asymmetry is given by 
$\mathcal{A}= \sgn\left(\phi_\mathrm{SOF}\right) \sin \phi_{\rm CPV}$.
In practice, various experimental imperfections (e.g. deviations from exact RF phase and/or resonance frequency) generate an additional accumulated phase $\phi'$ during free precession in the interaction region. 
This modifies the asymmetry, such that $\mathcal{A}= \sgn\left(\phi_\mathrm{SOF}\right) \sin (\phi_{\rm CPV} + \phi')$.
To isolate the CPV phase term, we measure $\mathcal{A}$ under two different conditions where the sign of $\Delta_\mathrm{CPV}$ reverses. This is the case, for example, when the direction of $\Evec_\mathrm{MI}$
%, and hence the sign of $\mathcal{P}$, 
is reversed.
Then we calculate
\begin{equation}
    \mathcal{A}_{+\Esca_\mathrm{MI}} - \mathcal{A}_{-\Esca_\mathrm{MI}} \approx \pm 2\phi_{\rm CPV},
\end{equation}   
independent of $\phi'$ so long as $\phi' \ll 1$.  We refer to the reversal of $\Evec_\mathrm{MI}$ as $E$-modulation, and assign the parameter $E = \pm 1 = \mathrm{sgn}(\Evec_\mathrm{MI} \cdot \hat{z})$.

It is possible to reverse the sign $\Delta_\mathrm{CPV}$ in several other ways as well. For example, simultaneously reversing the magnetic fields in state preparation regions B and C, $\Bvec_\mathrm{SPB}$ and $\Bvec_\mathrm{SPC}$, reverses the signs of all the angular momenta relative to the fixed laboratory $z$-axis. This corresponds to changing the signs of $m_{I1}, m_{I2}$, and $m_J$, and hence also the sign of $\Delta_{\rm CPV}$. We refer to this reversal as $B$ modulation and define $B = \mathrm{sgn}(\Bvec_\mathrm{SPB} \cdot \hat{z})$. Finally, changing the frequency of the microwave fields in SPB and SPC makes it possible to select which states to initially populate for use in the MI region. For example, the transitions e$\leftrightarrow$j and h$\leftrightarrow$k are time-reversed versions of each other, meaning the effective internal magnetic field has opposite sign between them. This also changes the sign of $\Delta_{\rm CPV}$.
%Following Ref.~\cite{cho1991search},
We refer to this reversal as $M$ modulation, and define $M = \pm 1$ corresponding to the h$\leftrightarrow$k and e$\leftrightarrow$j transitions, respectively.

While any of these modulations will, in principle, serve to isolate the contribution of $\phi_\mathrm{CPV}$, in practice we will employ all of them to provide various in-situ diagnostics. Our parameter naming convention here follows that of Ref.~\cite{cho1991search}. 

It proves useful to also employ a few more modulations. In particular, modulating the sign of the phase offset $\phi_{\rm SOF}$ between the RF coils, as discussed in Sec.~\ref{sec:interaction_region}, changes the sign of the asymmetry $\mathcal{A}$. This $P$ modulation has no effect on $\Delta_{\rm CPV}$.

The fringe contrast $C_\mathrm{SOF}$ can be measured by alternatingly offsetting the SOF drive frequency from its resonance value ($f_0 = 119.516\kHz$) by $\pm f_F$, where $f_F$ is small compared to the SOF NMR linewidth. The quantity $\mathcal{A}_{+f_F}-\mathcal{A}_{-f_F}$ determines the slope of the resonance and hence $C_\mathrm{SOF}$. This $F$ modulation has no effect on $\Delta_{\rm CPV}$.

During the NSM measurement, all these modulation parameters will be frequently switched to determine $\Delta_\mathrm{CPV}$ and to diagnose various possible contributions to $\phi'$. 

We denote the various combinations of asymmetries (i.e., phases) that can be constructed from these modulations with the notation $\mathcal{S}_{p1,p2,...}$. Here, the subscripts denote a linear combination of phases odd under the listed modulation parameters $p1,p2,...$ and even under all other modulations. For example, the total phase shift $\phi_{\rm CPV}$ is determined via
\begin{equation}
    \phi_\mathrm{CPV} \propto \mathcal{S}_{PEBM} = \sum_i\left(P_i\,E_i\,B_i\,M_i\right)\mathcal{A}_i,
\end{equation}
where $P_i,\,E_i,\,B_i,\,$ and $M_i$ are the signs of the modulation parameters during the $i^\mathrm{th}$ dataset, and $\mathcal{A}_i$ is the measured asymmetry for that dataset.  
The quantity
\begin{equation}
    \mathcal{S}_\mathrm{PF} = \sum_i \left(P_i\,F_i\right)\mathcal{A}_i,
\end{equation}
which determines the slope of the frequency vs.~phase curve (and hence also the fringe contrast $C_\mathrm{SOF}$), is used to convert $\phi_\mathrm{CPV}$ to frequency units. 


\subsection{Known Systematic Errors}
\label{sec:frequency_shifts}

\begin{table*}
	\normalsize
	\makebox[\textwidth][c]{%
	    \resizebox{\textwidth}{!}{
		\begin{tabu} to \textwidth {@{}r @{\hspace{3pt}} ccc cc ccc cc ccc cc@{}}
			\toprule
			
			& \multicolumn{4}{c}{State 1} & \multicolumn{4}{c}{State 2} & $f_0$ & $df_0/d\mathcal{B}_z$ & $df_0/d\mathcal{E}_z$ & shift $\mathcal{B}_\mathrm{mot}$ & $S$ & \multicolumn{2}{c}{$|\bra{\cdot}\mathcal{H}_\text{Z}\ket{\cdot}|$ [kHz]\hspace{1.5ex}\mbox{}}\\
			\cmidrule(r{2.0ex}){2-5}\cmidrule(r{2.0ex}){6-9}\cmidrule(r{2.0ex}){15-16}
			What flips?\hspace{1ex} & $l$ & $m_J$ & $m_{I_1}$ & $m_{I_2}$ & $l$ & $m_J$ & $m_{I_1}$ & $m_{I_2}$ & kHz & [mHz/$\mu$G] & [mHz/(V/cm)] & [mHz/$\mu$G] & - & x,y & z\\
			\midrule
			
			\multirow{2}{*}{$m_{I_1}\;\Big\{$}             & e & $-$ & $-$ & $-$ & j & $-$ & $+$ & $-$ & \multirow{2}{*}{119.52} & $+2.49$ &  \multirow{2}{*}{$- 31.50$} & $+4.66\ee{-5}$ & \multirow{2}{*}{0.95} & \multirow{2}{*}{1.33} & \multirow{2}{*}{0.00} \\
			& h & $+$ & $+$ & $+$ & k & $+$ & $-$ & $+$ &  & $-2.49$                &                       &  $+5.22\ee{-5}$ & & & \\

			\dashedrule
			
			\multirow{2}{*}{$m_{I_1},m_{I_2}\;\Big\{$}             & f & $+$ & $+$ & $-$ & k & $+$ & $-$ & $+$ & \multirow{2}{*}{108.92} &  $+1.52$ &  \multirow{2}{*}{$- 3.57$} & $-1.17\ee{-4}$ & \multirow{2}{*}{0.99} & \multirow{2}{*}{0.00} & \multirow{2}{*}{0.09}\\
			& g & $-$ & $-$ & $+$ & j & $-$ & $+$ & $-$  &               & $-1.52$                &                       &  $-1.23\ee{-4}$ & & &\\
	
			\dashedrule

			\multirow{2}{*}{$m_{I_2}\;\Big\{$}             &e & $-$ & $-$ & $-$ & g & $-$ & $-$ & $+$ & \multirow{2}{*}{10.59} & $+4.00$ &  \multirow{2}{*}{$- 27.93$} & \multirow{2}{*}{$+1.69\ee{-4}$} & \multirow{2}{*}{0.04} & \multirow{2}{*}{1.88} & \multirow{2}{*}{0.00} \\
			& h & $+$ & $+$ & $+$ & f & $+$ & $+$ & $-$ & &              $-4.00$  &               &  & & &\\
			
			
			\bottomrule
	\end{tabu}}}
	\normalsize
	\caption{All non-degenerate pairs of $|m_J|=1$ states in the $\widetilde{J}=1$ manifold at $\Esca_\mathrm{MI} = 30 \kVcm$ that do not involve the states $\ket{\text{i}}$ and $\ket{\text{l}}$ or a flip of $m_J$. The quantum numbers given are that of the largest decoupled-basis component. State labels $l$ are as in \cite{wilkening1984search} and Fig.~\ref{fig:levels_interaction_region}. The quantities $df_0/d\Bsca_z$ and $df_0/d\Esca_z$ give the slope of the resonance frequency with respect to the external magnetic field and electric field, respectively. Shift $\mathcal{B}_\mathrm{mot}$ indicates the resonance frequency shift, due to the motional field that accompanies $\mathcal{E}$ reversal, with respect to a stray field component $\Bsca_y$. $S$ denotes the sensitivity to the NSM relative to the maximum possible value; it is given by $\left|\langle I_{1,z}\rangle_1-\langle I_{1,z}\rangle_2\right|$ for the transition between states 1 and 2. $f_0$ indicates the transition frequency between the two states. $|\bra{\cdot}\mathcal{H}_\text{Z}\ket{\cdot}|$ indicates the magnitude of the transition dipole moment between states 1 and 2. All shifts are calculated from diagonalization of the ground-state Hamiltonian (Eq.~\ref{eq:hyperfine_hamiltonian}).}
	\label{tab:freq_shifts}
\end{table*}

Here we discuss the anticipated magnitude of some known systematic errors in \CENTREX. Our discussion closely follows the notation and analysis of Ref.~\cite{cho1991search}.

\subsubsection{Imperfect \texorpdfstring{$\mathcal{E}$}{E}-field Reversal}
The separation between the Tl spin up/down states in the $J=1,\,m_J=\pm1$ manifold changes slightly when the externally applied $\Esca$-field changes in magnitude.\footnote{This is due to 2nd-order spin-spin and spin-rotation couplings to distant $\ket{\widetilde{J},m_J}$ states.} Any non-reversing contribution to $\Evec_\mathrm{MI}$, e.g., from a stray DC field, thus leads to a  frequency shift in the NMR transition that changes with the orientation of $\Evec_\mathrm{MI}$. By brute-force diagonalization of the ground-state Hamiltonian of Eq.~\ref{eq:hyperfine_hamiltonian} with $\Esca = 30\kVcm$, the frequency shift can be calculated: see Tab.~\ref{tab:freq_shifts}. The pairs of states ej and hk that are used for the measurement both have an identical shift of $-31.5\,$mHz/(V/cm).  Assuming the non-reversing $\Evec$-field component does not change significantly between subsequent $M$ and $B$ reversals, this effect will be suppressed in the quantity $\mathcal{S}_{EBMP} \propto \phi_\mathrm{CPV}$ that is odd under both $M$ and $B$. For the residual shift to be below our anticipated sensitivity, we will require a small non-reversing $\Evec$-field as well as accurate changes of both $\Bsca$ and the initial state of the NMR transition. The former quantity can be determined from the signal combination $\mathcal{S}_{EP}$, and then nulled by applying an appropriate offset voltage; the inaccuracy in latter two can be determined from $\mathcal{S}_{EBMP}$ when a deliberately large non-reversing $\Evec$-field is applied, then nulled if necessary. 

\subsubsection{Stray \texorpdfstring{$\mathcal{B}$}{B}-Fields}

The \CENTREX\ measurement will be performed with a nominally zero $\Bvec$-field in the Main Interaction region. Significant effort will be made to minimize any residual stray fields, but nevertheless some will persist. These can arise, e.g., from leakage through, or residual magnetization of, the magnetic shielding. These stray $\Bsca$-fields can lead to systematic errors via two mechanisms: direct shifts, and in combination with motional-field effects.

For the pair of states ej and hk, a $\Bvec$-field along $\Evec_\mathrm{MI}$ (i.e., $\Bsca_z$) generates a direct frequency shift of $\pm2.5~\mHz$/$\mu$G, where the sign applies for the ej and hk transition, respectively. \CENTREX\ aims for sub--$10\,\mu$G residual $\Bsca$-fields, which will alone shift the transition frequency by $\mathcal{S}_\mathrm{B} \approx \pm 25\mHz$.  

Consider the total effective magnetic field $\mathcal{B}_\mathrm{MI}$ in the MI region. This field has contributions from several physical mechanisms; we write
\begin{equation}
    \mathcal{B}_\mathrm{MI} = \mathcal{B}_\mathrm{int} + \mathcal{B}_\mathrm{st} + \mathcal{B}_\mathrm{SP} +
    \mathcal{B}_\mathrm{LC},
\end{equation}
where $\mathcal{B}_\mathrm{int}$ is the intra-molecular magnetic field, $\mathcal{B}_\mathrm{st}$ is a static stray field, $\mathcal{B}_\mathrm{SP}$ is from the magnetic fields in SPB and SPC penetrating into the MI region, and $\mathcal{B}_\mathrm{LC}$ is from leakage currents in the electrode structure.
Both $\mathcal{B}_\mathrm{int}$ and $\mathcal{B}_\mathrm{SP}$ change sign under $B$ modulation. Under $M$ modulation, only $\mathcal{B}_\mathrm{int}$ changes sign. We do not expect $\mathcal{B}_\mathrm{st}$ to change significantly under any of the modulations.
%, though that is in principle possible through leakage of the laboratory fields through the magnetic shielding. 
So, in order to fully suppress the direct shifts due to stray magnetic fields, all three modulations $E$, $B$ and $M$ are required.

However, none of these modulations help to distinguish Zeeman shifts due to $\mathcal{B}_\mathrm{LC}$ from a true NSM signal, since both reverse under $E$ modulation.  Hence, as usual for EDM experiments, it will be very important to minimize the leakage current $I_\mathrm{LC}$.  Using the standard crude approximation for a worst-case scenario of $\Bsca_\mathrm{LC}$ (where all leakage current flows around a helical path between electrodes), we find that $I_\mathrm{LC}$ could conceivably need to be as low as $\sim\!1$~nA to absolutely ensure that this systematic error is less than our anticipated statistical sensitivity. Because this may prove challenging, we discuss possible methods to reduce our sensitivity to leakage currents in Sec. \ref{Sec:InternalComagnetometry}.

The magnitude of all other contributions to $\mathcal{B}_\mathrm{MI}$ can be determined from appropriate signal combinations. For example, $\mathcal{S}_\mathrm{BMP} \approx 5\mathcal{B}_\mathrm{st}\mHz/\mu$G determines $\mathcal{B}_\mathrm{st}$, since $B$ and $M$ work together to reverse $\mathcal{B}_\mathrm{int}$ but keep the orientation of the spins. Similarly, $\mathcal{S}_\mathrm{MP} \approx 5\mathcal{B}_\mathrm{SP}\mHz/\mu$G determines $\mathcal{B}_\mathrm{SP}$,  since $M$ flips the direction of the spins relative to the fields in SPB and SPC regions. Once measured, these fields can be nulled; then, by deliberately exaggerating each component separately, their residual effects on $\Delta_\mathrm{CPV}$ can be measured.

Another type of undesired $\Bsca$-field arises because the molecules move through the $\Evec_\mathrm{MI}$-field with finite velocity $\vec{v} =  v\hat{x}$. They therefore experience a motional magnetic field,
\begin{equation}
    \bm{\mathcal{B}}_\mathrm{mot} = \vec{v}\times\frac{\bm{\mathcal{E}}}{c^2}.
\end{equation}
$\bm{\mathcal{B}}_\mathrm{mot}$ is always perpendicular to both $\vec{v}$ and $\bm{\mathcal{E}}$, i.e., nominally in the $\hat{y}$ direction. If there is any static magnetic field with a nonzero $y$-component, the total magnetic field magnitude will be $\Bsca_\text{tot} = \sqrt{\Bsca_\text{int}^2 + (\Bsca_\mathrm{mot} +\Bsca_\mathrm{st})^2}$.  This means that $\Bsca_\mathrm{tot}$ will change in magnitude when $\Evec_\mathrm{MI}$ is reversed. Since the Zeeman splitting between spin up and down is proportional to $\Bsca_\text{tot}$, this leads to a frequency shift under $\Evec_\mathrm{MI}$ reversal. With our experimental parameters, the resulting shift is approximately $\Bsca_{\mathrm{st},y} \times 50\,$nHz/$\mu$G. Assuming we reach our target level of residual magnetic field, $\Bsca_\mathrm{st} < 10\,\mu$G, a shift of $0.5~\mu$Hz is expected. However, this shift is strongly (but not completely) suppressed due to the $M$ modulation, because the Zeeman shift due to $\Bvec_\mathrm{mot}$ is nearly, but not identically, equal for transitions ej and hk. The difference in the motional-field induced shift between the two transitions is $\approx\! \Bsca_{\mathrm{st},y} \times 5.6\,\mathrm{nHz}/\mu$G. For a field $\Bsca_{\mathrm{st},y} = 10\,\mu$G, this is roughly the same as our anticipated statistical sensitivity.  However, as described in Sec.\ \ref{Sec:InternalComagnetometry}, it should be possible to isolate any residual contribution from the motional field shift by employing co-magnetometry in \CENTREX. 

\subsubsection{Other known sources of systematic errors}
We have considered several other known sources of systematic errors that have been discussed in literature on searches for $T$-violation in TlF.  For example, shifts due to the Millman effect \cite{millman1939determination} (caused by misalignment of the NMR RF field coils) reverse with $B$ and $M$, and hence are suppressed only by $E$ modulation. However, with good construction techniques the residual effects appear likely to be smaller than our anticipated sensitivity.
Furthermore, the Millman effect can be quantified experimentally (see Sec. \ref{Sec:InternalComagnetometry} and Ref. \cite{cho1991search}).
Similarly, we have considered the effect of undesired phase offsets between the two RF coils, and also found the residual effects to be small compared to our anticipated sensitivity.

\subsection{Internal co-magnetometry in \\ \CENTREX\ }
\label{Sec:InternalComagnetometry}

Because the risk of systematic errors from stray magnetic fields is substantial, many of the latest generation of EDM searches have employed co-magnetometers, i.e., other physical systems used to measure magnetic fields co-located with the EDM-sensitive system in both space and time.  Some experiments have used different species nominally sharing the same volume \cite{abel2020measurement,regan2002new}.  Others have used different internal states of the EDM-sensitive system, which have different sensitivity to the EDM and/or to magnetic fields \cite{EckelDeMillePbO_2013,andreev_improved_2018}.  This ``internal co\hyph magnetometer'' approach~\cite{DeMille2001Search} has the advantage of guaranteed spatial overlap between the two systems, and reduced experimental complexity.

We believe it will be possible to use different internal states of TlF to act as a type of generalized internal co-magnetometer. As we have discussed, the apparently natural choices of internal states to use for the $^{205}$Tl NSM search are those where the $^{205}$Tl spin flips, but all other quantum numbers remain (nominally) the same.  This corresponds to the pairs e$\leftrightarrow$j and  h$\leftrightarrow$k assumed throughout our discussion.  However, it is entirely viable to instead employ pairs of states where only the $^{19}$F spin flips, i.e., the pairs e$\leftrightarrow$g and f$\leftrightarrow$h.  As shown in Table~\ref{tab:freq_shifts}, these pairs of states are 2-3 times more sensitive to magnetic field effects than the usual pairs. However, they have negligible sensitivity to $T$-violating effects, since the $^{19}$F nucleus has small $Z$ and $A$.  Hence, these pairs of states can act as a classic co-magnetometer.  The experimental configuration remains nearly unchanged from that used for NSM detection; the primary change is that a significantly lower NMR resonance frequency, $f_0^\prime = 10.6\kHz$, is needed.  We see no impediments to using these pairs of states, which will provide a novel diagnostic for systematic errors and stray fields in \CENTREX. We are still designing state preparation and readout protocols that will enable use of these pairs of states.

Even more potentially useful could be to employ the pairs of states f$\leftrightarrow$k and g$\leftrightarrow$j.  In these transitions, \textit{both} nuclear spins flip simultaneously. Measurements with these pairs are nearly 2 times less sensitive to magnetic fields from leakage currents and residual shield magnetization than the original pair and more than an order of magnitude less sensitive to $\Esca$-induced Zeeman shifts, but have have enhanced sensitivity to motional field shifts.  Hence, making measurements with these pairs as well as both single spin-flip pairs will provide a wealth of information to disentangle contributions from the most important systematic error contributions we are now aware of.  Employing these double spin-flip transitions will require an additional NMR RF coil to produce fields along $\hat{z}$. Here, because of the small transition dipole matrix element, the RF field magnitude will need to be roughly 10 times larger than for the other pairs.  We are currently investigating the feasibility of using these states in \CENTREX.