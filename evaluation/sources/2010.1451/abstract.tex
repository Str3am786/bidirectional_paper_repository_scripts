\twocolumn[
\begin{@twocolumnfalse}
	\begin{center}
		\begin{minipage}{.8\textwidth}
			\centering
			\vspace{4.5ex}

			{\bf\large CeNTREX: A new search for time-reversal symmetry violation in the $^{205}$Tl nucleus}

			\vspace{1.5ex}
			O. Grasdijk$^1$, O. Timgren$^1$, J. Kastelic$^1$, T. Wright$^{1,*}$, S. Lamoreaux$^1$, D. DeMille$^{2,3,1}$
			{\it\small 
			$^1$Department of Physics, Yale University, New Haven, CT 06511\\
			$^2$Physics Division, Argonne National Laboratory, Argonne, IL 60439\\
			$^3$James Franck Institute and Department of Physics, University of Chicago, Chicago, IL 60637
			}

			\vspace{1.5ex}
			K. Wenz, M. Aitken, T. Zelevinsky

			{\it\small Department of Physics, Columbia University, New York, NY 10027-5255}

			\vspace{1.5ex}
			T. Winick, D. Kawall

			{\it\small Department of Physics, University of Massachusetts Amherst, Amherst, MA 01003}
		\end{minipage}

		\vspace{3ex}

		\begin{minipage}{.75\textwidth}
			\small
			The Cold molecule Nuclear Time-Reversal EXperiment (CeNTREX) is a new effort aiming for a significant increase in sensitivity over the best present upper bounds on the strength of hadronic time reversal ($T$) violating fundamental interactions. The experimental signature will be shifts in nuclear magnetic resonance frequencies of $^{205}$Tl in electrically-polarized thallium fluoride (TlF) molecules. Here we describe the motivation for studying these $T$-violating interactions and for using TlF to do so. To achieve higher sensitivity than earlier searches for $T$-violation in TlF, CeNTREX uses a cryogenic molecular beam source, optical state preparation and detection, and modern methods of coherent quantum state manipulation. Details of the measurement scheme and the current state of the apparatus are presented, with quantitative measurements of the TlF beam. Finally, the estimated sensitivity and methods to control systematic errors are discussed. 

		\end{minipage}

		\vspace{3ex}
	\end{center}
\end{@twocolumnfalse}
]
