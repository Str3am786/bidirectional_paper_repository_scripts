\section{Conclusion}
As described in section~\ref{sec:sensitivity_and_systematics}, we anticipate a statistical sensitivity to the CPV-induced energy ($\Delta_\mathrm{CPV}$) of $\delta \Delta_\mathrm{CPV} \approx  50~$nHz.  This would correspond to a roughly 2500-fold improvement over the previous best measurements of the $^{205}$Tl NSM.
Taking into account the calculated relation between the NSM and underlying parameters of fundamental physics, this would in many cases correspond to a significantly improved sensitivity over the current best limits.  For example, this would be sensitive to values of the QCD CPV parameter $\bar{\theta} \gtrsim 1 \times 10^{-12}$, a factor of $\approx 100$ smaller than current bounds \cite{abel2020measurement,graner2016reduced}, and to a proton EDM of $d_p \gtrsim 6\times 10^{-27}\,e$cm, a factor of $\approx 30$ smaller than the current best limit \cite{graner2016reduced}.

Currently, measurements and optimization of the rotational cooling efficiency are underway. Once this is completed, the SPA region and then the EQL region will be attached to the beamline for testing and optimization. The Main Interaction region is under construction. The remaining regions, SPB, SPC, and FD, are under design. Once the entire apparatus is assembled and tested, we will commence measurements, with the goal to reach the target sensitivity $\delta \Delta_\mathrm{CPV} \lesssim 50\,$nHz.

Subsequent generations of \CENTREX\, with considerable further improvements in sensitivity, also are anticipated.  For example, we plan to implement transverse laser cooling to collimate the TlF beam \cite{norrgard2017hyperfine, hunter2012prospects}, and a continuous cryogenic buffer gas beam source~\cite{Patterson_2007, Patterson_2009, Patterson_2015, PhysRevA.97.032704} loaded by a thermal TlF beam.  Preliminary estimates indicate that these improvements could increase the detected number of molecules by a factor of 30-100. In the further future, it may also be possible to slow, cool, and optically trap the TlF molecules.  This could dramatically increase the interaction time per molecule, though it remains to be seen what fraction of molecules can be captured in this way. In any case, the \CENTREX\ approach has the potential to yield substantially improved sensitivity to flavor-neutral CPV physics in the hadronic sector.

\CENTREX\ may also be used to search for axions, either measuring the oscillating Schiff moment produced by the interaction with an axion dark matter particle~\cite{PhysRevD.89.043522} or searching for virtual axions mediating CP-violation and producing a Schiff moment in the Tl nucleus~\cite{PhysRevLett.120.013202, PhysRevD.98.035048}.

We thank L.R.\ Hunter and N.\ Clayburn for many helpful discussions, and for sharing preliminary data on optical cycling in TlF. 

We are grateful for support from the John Templeton Foundation, the Heising-Simons Foundation, a NIST Precision Measurement Grant, and NSF-MRI grants PHY-1827906, PHY-1827964, and PHY-1828097, and the US DOE Office of Nuclear Physics.

