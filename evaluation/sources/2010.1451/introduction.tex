\section{Introduction}
\label{sec:introduction}
Before the early 1960s, it was believed that $CP$ is a good symmetry of nature. As Landau pointed out, that would make it impossible for particles to have electric dipole moments (EDMs) along their spin axis \cite{landau1957conservation}. The detection of such an EDM would thus provide clear evidence of $CP$-violation (CPV).  \blfootnote{*Present address: JILA, National Institute of Standards and Technology and University of
Colorado, and Department of Physics, Boulder, Colorado 80309, USA}

While most processes preserve $CP$, certain weak interactions violate it as observed in $K$-, $B$-, and $D$-meson decays \cite{christenson1964evidence, PDG18}. The flavor-changing part of the Standard Model (SM) quark sector includes a CPV phase in the CKM quark-mixing matrix \cite{peccei1995}. This so-called Kobayashi-Maskawa mechanism introduces the third quark generation to explain the CPV \cite{kobayashi1973cp}. The CKM phase has been the only source of observed CPV so far~\cite{PDG18}.

A major motivation for CPV searches comes from the baryon asymmetry of the universe (BAU). Compared to the current baryon density $n_{\text{B}}$, the antibaryon density $n_{\overline{\text{B}}}$ is very small; the reported upper bounds for the antimatter-to-matter number ratio range from $10^{-15}$ to $10^{-6}$ \cite{canetti2012matter}. To date, no mechanism has been experimentally verified that can explain the BAU. In a 1967 paper, Sakharov argued that CPV is necessary to explain the BAU \cite{sakharov1991violation} if the initial conditions of the universe were $C$-symmetric. The existing CPV in the CKM matrix is not enough to explain the extent of the BAU \cite{pospelov2005electric}. Thus new sources of CPV are required to explain the BAU.

No flavor-neutral CPV signal has been observed yet. However, many mechanisms can lead to such phenomena. For example, the QCD Lagrangian can, in principle, include an effective CPV term, proportional to the parameter $\bar{\theta}$~\cite{pospelov1999theta}:
\begin{equation}
	\mathcal{L} = \bar{\theta}\frac{g^2}{32\pi^2}G^a_{\mu\nu}\widetilde{G}^a_{\mu\nu},
\end{equation}
where $G^a$ is the gluon field tensor, $g$ is the strong coupling constant, and $\bar{\theta}$ is dimensionless. Experimental limits from experiments searching for CPV in neutral $^{199}$Hg atoms \cite{graner2016reduced} and ultracold neutrons \cite{baker2006improved, abel2020measurement} suggests that the strength of this term relative to the usual strong interaction is $\left|\bar{\theta}\right|<9\cdot10^{-11}$. The unexplained smallness of $\bar{\theta}$ is known as the strong $CP$ problem. One proposed solution to the strong $CP$ problem is the so called Peccei-Quinn mechanism, with an accompanying elementary scalar particle: the axion \cite{PhysRevLett.38.1440}. The axion would naturally lead to $\bar{\theta}\approx 0$, and is an attractive candidate for dark matter~\cite{PRESKILL1983127,PhysRevLett.50.925,PhysRevLett.124.101303}. (A review of experimental searches for the axion is given in \cite{graham2015experimental}.)

New hadronic $CP$-violating interactions from the QCD sector, or from physics beyond the SM, can lead to an effective charge asymmetry along the spin of a particle. Such charge asymmetries include EDMs and, for finite size particles such as nuclei, Schiff moments \cite{schiff1963measurability}. In the Standard Model, EDMs and nuclear Schiff moments (NSMs) are induced by the CKM phase, but are strongly suppressed: an EDM cannot appear below the three-loop level for quarks, or four-loop for leptons \cite{pospelov1991electric}. The CKM phase can produce a proton or neutron EDM no larger than $10^{-32}\,e\,$cm. Proposed extensions to the SM carry new CPV phases, which may manifest as EDMs or NSMs larger than expected based on the Standard Model. The search for an EDM or NSM thus constitutes a nearly background-free signal for new physics. In fact, the background expected from the Standard Model would only become apparent when probing effects beyond the energy scale of $\sim\!10^{5}\,$TeV \cite{pospelov2005electric}.

At present, searches for EDMs and related phenomena give the most sensitive constraints on flavor-neutral CPV effects beyond the SM. However, these searches are subject to the following limitation. According to the Schiff theorem, the interaction energy of nonrelativistic point-charged electric dipoles, bound in a neutral system but subject to an arbitrary external electrostatic potential, has no term linear in the CPV charge distribution \cite{schiff1963measurability}. Physically, the system rearranges itself so as to screen the external field completely \cite{safronova2018search}. Thus, a $CP$-violating moment of a charged constituent in a bound system cannot be detected without some mechanism to bypass Schiff's theorem. Two such mechanisms are relativistic constituent motion and finite constituent size. 

A nucleus in an atom or molecule is nonrelativistic, but has an extended size. This finite size can lead to a residual electromagnetic moment, the Schiff moment $\vec S$, that gives rise to a $CP$-violating interaction. In heavy diamagnetic atoms and diatomic molecules such as TlF, this finite-size effect gives the dominant contribution to CPV signals. Since the nuclear spin $\vec{I}$ is the only preferred direction in a nucleus, $\vec S$ has to lie parallel to this axis, i.e., $\vec{S} = S\vec{I}/I$. This quantum Schiff moment has the symmetries of $\vec{I}$: it changes sign under time reversal ($T$) but not under parity ($P$). By contrast, the classical Schiff moment is a static charge distribution that is unchanged under $T$ but changes sign under $P$. Hence, a nonzero value of $S$ means that both $T$ and $P$ symmetries are violated.
On the assumption that $CPT$ is a good symmetry, a nonzero value of $\vec S$ thus is also a signature of CPV.

The Schiff moment corresponds to a charge displacement that is similar to an EDM in its asymmetric distribution along the spin axis. It is equivalent to a charge density on the nuclear surface proportional to $\cos{\theta}$, where $\theta$ is the angle from $\vec{I}$; this surface charge distribution produces a uniform electric field inside the nucleus \cite{GingesFlambaum2004}. The magnitude $S$ of the NSM scales with the atomic mass $A$ as $S\propto A^{2/3}$ \cite{khriplovich1997}.

The value of $S$ can be related to more fundamental $CP$-violating parameters, including CPV $\pi$ meson--nucleon interaction constants $\bar{g}_{0},\,\bar{g}_1,$ and $\bar{g}_2$; the $\bar{\theta}$ QCD parameter; quark chromo-EDMs $\widetilde{d}_\text{d}$ and $\widetilde{d}_\text{u}$; and the neutron and proton EDMs, $d_\text{n}$ and $d_\text{p}$.  For example, the NSM of the $^{205}\mathrm{Tl}$ nucleus\footnote{The $^{205}\mathrm{Tl}$ nucleus has closed neutron shells; hence its NSM has negligible contribution from $d_\text{n}$.}  can be written as \cite{PhysRevA.101.042504,PhysRevA.101.042501}:
\begin{equation} 
    \label{eq:schiff_contributions}
        \begin{split}
            S\left(^{205}\mathrm{Tl}\right) & \approx \left(0.13g\bar{g}_0 - 0.004g\bar{g}_1 - 0.27 g \bar{g}_2\right) \,e\,\mathrm{fm}^3; \\
            S\left(^{205}\mathrm{Tl}\right) & \approx 0.027 \bar{\theta}\ e\,\mathrm{fm}^3; \\
            S\left(^{205}\mathrm{Tl}\right) & \approx \left(12\widetilde{d}_\text{d}+9\widetilde{d}_\text{u}\right) e\,\mathrm{fm}^2; \\
            S\left(^{205}\mathrm{Tl}\right) & \approx 0.4\,d_\text{p} \,\mathrm{fm}^2.
    \end{split}
\end{equation}
If detected, a nonzero $S\left(^{205}\mathrm{Tl}\right)$ would provide evidence for a nonzero value of one or more of these fundamental CPV parameters.

Energy shifts associated with a NSM can be greatly enhanced in polar molecules, where there is another intrinsic direction in addition to the nuclear spin: the internuclear axis $\hat{\vec{n}}$. In TlF,  we define $\hat{\vec{n}}$ as pointing from F to Tl, associated with the internal molecular dipole moment and a corresponding strong intramolecular gradient of the electron density.  For nuclei inside a molecule, the NSM (and other $CP$-violating effects \cite{khriplovich1997}) interacts with this density gradient, giving rise to an effective CPV Hamiltonian of the form \cite{PhysRevA.101.042504}
\begin{equation}\label{eq:Hamiltonian_effective_interaction}
    \mathcal{H}_\text{CPV}= W_S S\,\frac{\vec{I}}{I} \cdot \hat{\vec{n}}.
\end{equation}
Here, $W_S$ is the proportionality constant between $S$ and the CPV contribution to the molecular energy, for a fully polarized molecule. Its value is determined by the properties of the electronic wavefunctions, which can be calculated from first principles \cite{PhysRevA.101.042504, doi:10.1002/qua.20418, PhysRevLett.88.073001, doi:10.1080/00268976.2020.1767814}. The magnitude of $W_S$ grows rapidly with atomic number $Z$ of the nucleus, as $W_S\propto Z^2$ \cite{khriplovich1991parity}.

Without an external electric field $\Evec$, the interaction of Eq.~\ref{eq:Hamiltonian_effective_interaction} fails to produce a first-order effect in any given energy eigenstate. This is because the rotation of the molecule averages $\hat{\vec n}$ to zero, and the expectation value $\langle\mathcal{H}_\text{CPV}\rangle$ vanishes \cite{cho1989tenfold, cho1991search}. However, when an external field is applied, the molecule becomes polarized and both
%which destroys the inversion symmetry; then
$\hat{\vec{n}}$ and $\mathcal{H}_\text{CPV}$  acquire non-zero expectation values, with $\langle\hat{\vec n}\rangle \parallel \Evec$. We define the degree of electrical polarization $\mathcal{P} \equiv \langle \hat{\vec{n}}\cdot\hat{\Evec} \rangle$, (where $\hat{\Evec} = \Evec/\Esca$), so that $-1\leq\mathcal{P}\leq1$. 
%Here $\hat{z}$ is defined to be the axis along which the external electric field is applied. 
Hence, energy shifts due to CPV are given by  $\langle\mathcal{H}_\text{CPV}\rangle = W_S S \mathcal{P} \hat{\Evec} \cdot \frac{\vec{I}}{I}$.

\begin{figure}
    \centering
    \def\svgwidth{0.5\textwidth}
    \input{figs/svg/edm_levels.pdf_tex}
% 	\includegraphics[width=0.48\textwidth,unit=1mm]{figs/svg/edm_levels_pdf_version.pdf}
	\caption{T-violating energy shift $\Delta_{\rm CPV} = W_SS\,\mathcal{P}$ as a result of a nonzero NSM $S$ given by the effective interaction $\mathcal{H}_\text{CPV}= W_S S\vec{I}\cdot \hat{\vec{n}}/I$ (Eq.~\ref{eq:Hamiltonian_effective_interaction}), shown for opposite orientations of the applied field $\Evec$. Here $\mu_1$ is the Tl magnetic moment and $B_1^\mathrm{int}$ is the effective internal magnetic field at the Tl nucleus due to the spin-rotation interaction.
	}
	\label{fig:edm_shift}
\end{figure}

Polar molecules can be polarized readily in laboratory-scale fields owing to their small rotational level separation ($\sim\!10^{-4}\eV$), giving them near-maximal energy shifts induced by a given Schiff moment. Thus, Sandars \cite{sandars1967measurability} suggested a molecular-beam resonance experiment could be used to probe the existence of the proton EDM, if the molecule has a heavy atom with an unpaired proton in the nucleus, such as $^{205}\mathrm{Tl}$. The value of $S$ is determined by measuring energy splittings between spin-up and -down states relative to $\left\langle \hat{\vec{n}}\right\rangle$ (which is parallel to the applied electric field $\bm{\mathcal{E}}$). This splitting will increase or decrease as $\bm{\mathcal{E}}$ (and hence $\left\langle \hat{\vec{n}}\right\rangle$) reverses, due to the interaction in Eq.~\ref{eq:Hamiltonian_effective_interaction}. The difference in level splittings is proportional to the electric polarization $\mathcal{P}$, the interaction strength $W_S$, and $S$ (Fig. \ref{fig:edm_shift}).

\CENTREX\ uses a cold beam of thallium fluoride (TlF) to measure nuclear $T$-violation due to the NSM of the $^{205}$Tl nucleus. It is a suitable system to look for $P$- and $T$-violating interactions for a number of reasons: a molecular beam of thallium fluoride can be readily obtained; many of the molecular states and transitions are known experimentally; the species is a polar diatomic molecule, enhancing the electron density gradient at the site of the nuclei (and hence $W_S$). As the thallium nucleus is heavy $\left(A=205,\,Z=81\right)$ and the NSM-induced energy shift scales $\propto A^{2/3}Z^2$, the observable effect of the Tl Schiff moment is correspondingly large \cite{sandars1967measurability, wilkening1984search}. Since the Tl nucleus contains an unpaired proton,
%the different linear combinations of underlying CPV parameters means
\CENTREX\ will be primarily sensitive to proton EDM effects, as opposed to other leading experiments which are more sensitive to the
%effects associated with neutron spin such as
neutron EDM \cite{graner2016reduced}. TlF is not very sensitive to the electron EDM due to its zero total electron spin \cite{kozlov1995parity}. 

The current best constraint on $T$-violating interactions associated with $S\left(^{205}\mathrm{Tl}\right)$ was found by Cho, Sangster and Hinds in 1991 \cite{cho1989tenfold, cho1991search}, who measured a NSM-induced frequency shift of $\Delta E = 2\Delta_{\rm CPV} = \left(1.4\pm 2.4\right)\times 10^{-4}$~Hz, consistent with zero.\footnote{Throughout, we express both frequencies and energies in linear frequency units (Hz), and all angular momentum operators are treated as dimensionless.} 
Using the effective interaction $\mathcal{H}_\text{CPV}$, the shift in the energy splitting between states with Tl spin up versus spin down, relative to the quantization axis, can be interpreted as
\begin{equation}
    \Delta E= 2 \Delta_{\rm CPV} = 2 W_S\,S\, \textrm{sgn}(\Esca) \,\mathcal{P},
    \label{eq:frequency_shift_due_to_NSM}
\end{equation}
where $W_S = 40539\,$a.u.,
polarization $\mathcal{P} = \langle\hat{\vec{n}} \cdot \hat{\Evec}\rangle = 0.547$, and the sign of $\Esca$ refers to the direction of $\Evec$ relative to a fixed quantizing axis $\hat{z}$.  This determines the Schiff moment \cite{PhysRevA.101.042501, PhysRevLett.88.073001}
\begin{equation}
    S\left(^{205}\mathrm{Tl}\right) = \left(3.6\pm 6.1\right)\times 10^{-11}\,e\,\mathrm{fm}^3.
\end{equation}
With Eq.~\ref{eq:schiff_contributions}, the following limits can be placed:
\begin{equation}
    \label{eq:prev_best_lims}
    \begin{split}
        \bar{\theta} & = \left(1.3 \pm 2.3\right)\ee{-9}, \\
        12\bar{d}_d+9\bar{d}_u & = \left(3.6\pm 6.1\right)\ee{-24}\,\mathrm{cm}, \\
        d_p          & = \left(0.9 \pm 1.5\right)\ee{-23}\,e\,\mathrm{cm},\\
        0.13g\bar{g}_0 - 0.004g\bar{g}_1-0.27g\bar{g}_2 & = \left(3.6 \pm 6.1\right)\ee{-11}.
    \end{split}
\end{equation}
\CENTREX\ aims to improve on these limits by using a cryogenic molecular beam source to achieve a cold beam with higher intensity and lower velocity spread compared to the jet source used in the previous work. Rotational cooling will be performed with optical and microwave pumping, collapsing much of the initial Boltzmann distribution into one state, greatly enhancing the number of molecules accessible for measurement. Finally, optical cycling will be used to assist state readout, resulting in near-unity detection efficiency. Fluorescence detection, compared to the hot-wire techniques used previously, allows for background-free detection if scattered light is well controlled.

\subsection{Thallium Fluoride}
\label{sec:tlf_theory}

\begin{figure}
	    % for axis breaks
    \usetikzlibrary{decorations.markings}
    \def\MarkLt{4pt}
    \def\MarkSep{2pt}
    \tikzset{
      TwoMarks/.style={
        postaction={decorate,
          decoration={
            markings,
            mark=at position #1 with
              {
                  \begin{scope}[xslant=0.2]
                  \draw[line width=\MarkSep,white,-] (0pt,-\MarkLt) -- (0pt,\MarkLt) ;
                  \draw[-] (-0.5*\MarkSep,-\MarkLt) -- (-0.5*\MarkSep,\MarkLt) ;
                  \draw[-] (0.5*\MarkSep,-\MarkLt) -- (0.5*\MarkSep,\MarkLt) ;
                  \end{scope}
              }
           }
        }
      },
      TwoMarks/.default={0.5},
    }

\begin{tikzpicture}[scale=5]

   % length of energy level diagram lines
   \def\len{.05}

   % x-position and length and x-offset of the y-axis labels
   \def\pos{-18}
   \def\lenmark{.025}
   \def\xoffset{.18}

   % x coordinates for m = -2, ... 2
   \def\xa{-4.5*\len}
   \def\xb{-2.5*\len}
   \def\xc{-0.5*\len}
   \def\xd{+1.5*\len}
   \def\xe{+3.5*\len}
   \def\xf{-6.5*\len}
   \def\xg{+5.5*\len}

   % y offset for J = 1, 2
   \def\yA{.6}
   \def\yB{\yA+.7}

   % axis
   \draw[->,TwoMarks=0.2,TwoMarks=0.6] (\xf-\len-\xoffset, -.1)
      -- (\xf-\len-\xoffset,\yB+.25) node[above, xshift = -20] {$E$ [kHz]};

   % brace position
   \def\bracePos{\xg+3*\len}

    
   % J=0 labels
   \draw [decorate,decoration={brace,amplitude=5pt,mirror,raise=4pt},yshift=0pt]
      (\bracePos,-.08) -- node (midJ0) {} (\bracePos,.08) node [black,midway,anchor=west,xshift=25] (nodeJ0) {$0$};
   \draw (\xf-\len-0.5*\lenmark-\xoffset, -.003325) node[anchor=east,below,xshift=\pos]
      {-3.325} -- (\xf-\len+0.5*\lenmark-\xoffset,-.003325) node[right,below,xshift=-\pos] {$1$};
   \draw (\xf-\len-0.5*\lenmark-\xoffset, .009975) node[anchor=east,above,xshift=\pos]
      {9.975} -- (\xf-\len+0.5*\lenmark-\xoffset,.009975) node[right,above,xshift=-\pos] {$0$};

   % J=0 lines
   \draw (\xc, -0.003325) -- (\xc+\len, -0.003325);
   \draw (\xb, -0.003325) -- (\xb+\len, -0.003325);
   \draw (\xd, -0.003325) -- (\xd+\len, -0.003325);
   \draw (\xc, 0.009975) -- (\xc+\len, 0.009975);

   % J=1 labels
   \draw [decorate,decoration={brace,amplitude=10pt,mirror,raise=4pt},yshift=0pt]
      (\bracePos,\yA-.2) -- node (midJ1) {} (\bracePos,\yA+.1) node [black,midway,anchor=west,xshift=25] (nodeJ1) {$1$};
   \draw (\xf-\len-0.5*\lenmark-\xoffset, \yA-.143745) node[anchor=east,below,xshift=\pos] {-143.745} --
      (\xf-\len+0.5*\lenmark-\xoffset,\yA-.143745) node[right,below,xshift=-\pos] {$0$};
   \draw (\xf-\len-0.5*\lenmark-\xoffset, \yA-.121506) node[anchor=east,above,xshift=\pos] {-121.506} --
      (\xf-\len+0.5*\lenmark-\xoffset,\yA-.121506) node[right,above,xshift=-\pos] {$1$};
   \draw (\xf-\len-0.5*\lenmark-\xoffset, \yA+.054446) node[anchor=east,below,xshift=\pos] {54.446} --
      (\xf-\len+0.5*\lenmark-\xoffset,\yA+.054446) node[right,below,xshift=-\pos] {$1$};
   \draw (\xf-\len-0.5*\lenmark-\xoffset, \yA+.068985) node[anchor=east,above,xshift=\pos] {68.985} --
      (\xf-\len+0.5*\lenmark-\xoffset,\yA+.068985) node[right,above,xshift=-\pos] {$2$};

   % J=1 lines
   \draw (\xc, \yA-.143745) -- (\xc+\len, \yA-.143745);
   \draw (\xc, \yA-.121506) -- (\xc+\len, \yA-.121506);
   \draw (\xb, \yA-.121506) -- (\xb+\len, \yA-.121506);
   \draw (\xd, \yA-.121506) -- (\xd+\len, \yA-.121506);
   \draw (\xc, \yA+.054446) -- (\xc+\len, \yA+.054446);
   \draw (\xb, \yA+.054446) -- (\xb+\len, \yA+.054446);
   \draw (\xd, \yA+.054446) -- (\xd+\len, \yA+.054446);
   \draw (\xc, \yA+.068985) -- (\xc+\len, \yA+.068985);
   \draw (\xb, \yA+.068985) -- (\xb+\len, \yA+.068985);
   \draw (\xd, \yA+.068985) -- (\xd+\len, \yA+.068985);
   \draw (\xa, \yA+.068985) -- (\xa+\len, \yA+.068985);
   \draw (\xe, \yA+.068985) -- (\xe+\len, \yA+.068985);

   % J=2 labels
   \draw [decorate,decoration={brace,amplitude=10pt,mirror,raise=4pt},yshift=0pt]
      (\bracePos,\yB-.27) -- node (midJ2) {} (\bracePos,\yB+.18) node [black,midway,anchor=west,xshift=25] (nodeJ2) {$2$};
   \draw (\xf-\len-0.5*\lenmark-\xoffset, \yB-.217455)
   node[anchor=east,left,yshift=-5] {-217.455}
      -- (\xf-\len+0.5*\lenmark-\xoffset,\yB-.217455) node[right,below,xshift=-\pos] {$1$};
   \draw (\xf-\len-0.5*\lenmark-\xoffset, \yB-.172936)
   node[anchor=east,left,yshift=5] {-172.936}
      -- (\xf-\len+0.5*\lenmark-\xoffset,\yB-.172936) node[right,above,xshift=-\pos] {$2$};
   \draw (\xf-\len-0.5*\lenmark-\xoffset, \yB+.105876)
   node[anchor=east,left,yshift=-5] {105.876}
      -- (\xf-\len+0.5*\lenmark-\xoffset,\yB+.105876) node[right,below,xshift=-\pos] {$2$};
   \draw (\xf-\len-0.5*\lenmark-\xoffset, \yB+.141095)
   node[anchor=east,left,yshift=5] {141.095}
      -- (\xf-\len+0.5*\lenmark-\xoffset,\yB+.141095) node[right,above,xshift=-\pos] (J2F3label) {$3$};

   % J=2 lines
   \draw (\xc, \yB-.217455) -- (\xc+\len, \yB-.217455);
   \draw (\xb, \yB-.217455) -- (\xb+\len, \yB-.217455);
   \draw (\xd, \yB-.217455) -- (\xd+\len, \yB-.217455);
   \draw (\xc, \yB-.172936) -- (\xc+\len, \yB-.172936);
   \draw (\xb, \yB-.172936) -- (\xb+\len, \yB-.172936);
   \draw (\xd, \yB-.172936) -- (\xd+\len, \yB-.172936);
   \draw (\xa, \yB-.172936) -- (\xa+\len, \yB-.172936);
   \draw (\xe, \yB-.172936) -- (\xe+\len, \yB-.172936);
   \draw (\xc, \yB+.105876) -- (\xc+\len, \yB+.105876);
   \draw (\xb, \yB+.105876) -- (\xb+\len, \yB+.105876);
   \draw (\xd, \yB+.105876) -- (\xd+\len, \yB+.105876);
   \draw (\xa, \yB+.105876) -- (\xa+\len, \yB+.105876);
   \draw (\xe, \yB+.105876) -- (\xe+\len, \yB+.105876);
   \draw (\xc, \yB+.141095) -- (\xc+\len, \yB+.141095);
   \draw (\xb, \yB+.141095) -- (\xb+\len, \yB+.141095);
   \draw (\xd, \yB+.141095) -- (\xd+\len, \yB+.141095);
   \draw (\xa, \yB+.141095) -- (\xa+\len, \yB+.141095);
   \draw (\xe, \yB+.141095) -- (\xe+\len, \yB+.141095);
   \draw (\xf, \yB+.141095) -- (\xf+\len, \yB+.141095);
   \draw (\xg, \yB+.141095) -- (\xg+\len, \yB+.141095);

   % microwave energies
   \draw[Latex-Latex] (nodeJ0) -- node[above,sloped] {13.3 GHz} (nodeJ1);
   \draw[Latex-Latex] (nodeJ1) -- node[above,sloped] {26.7 GHz} (nodeJ2);

   % approximate F1 labels
   \def\FposX{-5}
   \def\FposY{15}
   \def\FposYY{22}
   \node[xshift=\FposX] at (midJ0) {\nicefrac{1}{2}};
   \node[xshift=\FposX,yshift=-\FposY] at (midJ1) {\nicefrac{1}{2}};
   \node[xshift=\FposX, yshift=\FposY] at (midJ1) {\nicefrac{3}{2}};
   \node[xshift=\FposX,yshift=-\FposYY] at (midJ2) {\nicefrac{3}{2}};
   \node[xshift=\FposX, yshift=\FposYY] at (midJ2) {\nicefrac{5}{2}};
   \node[xshift=\FposX, yshift=47] at (midJ2) (F1label) {$F_1$};
    
   \node at (F1label -| nodeJ2) {$J$};
   \node at (F1label -| J2F3label) {$F$};

\end{tikzpicture}

	\caption{Hyperfine structure in the lowest three rotational levels in the TlF ground state $X^1\Sigma^+$, with no applied fields. Rotational energies $hBJ(J+1)$ should be added to
	%are subtracted from
	the hyperfine energy shifts indicated on the axis. Note the $(2F+1)$-fold degeneracy of each state with total angular momentum $F$, corresponding to the quantum numbers $m_F=-F,\dots,0,\dots,F$.}
	\label{fig:level_diagram}

\end{figure}

The TlF molecule is described by its electronic, vibrational and rotational motion, plus the states of the Tl and F nuclear spins. \CENTREX\ makes use of states both in the vibronic ground state, $X ^1\Sigma^+\left(\nu=0\right)$, and in an electronically excited state, $B ^3\Pi_1\left(\nu=0\right)$.  In both cases, we describe the angular momentum couplings in a Hund's case (a) basis. We typically write energy eigenstates in terms of the basis states $\ket{\eta,J,F_1,F,m_F}$. Here, $\eta$ represents the vibronic quantum numbers; $J$ is the total angular momentum excluding the nuclear spins, $\vec{F}_1 = \vec{J}+\vec{I}_\mathrm{1}$, with $I_\mathrm{1}=1/2$ for $^{205}$Tl; $\vec{F} = \vec{F}_1+\vec{I}_\mathrm{2}$ is the total angular momentum, with $I_\mathrm{2}=1/2$ for $^{19}$F; and $m_F$ is its projection along a quantization axis in the lab frame. Field-free eigenstates are close to these basis states in the ground $X ^1\Sigma^+$ state.  In the $B ^3\Pi_1$ state, strong hyperfine interactions significantly mix states with different $J$ and $F_1$ values. Hence, we describe these eigenstates with the modified notation $\ket{\eta',\widetilde{J}' ,\widetilde{F}_1^\prime,F',m_F'}$, where $\widetilde{J}'$ and $\widetilde{F}_1^\prime$ correspond to the largest component in their basis-state decomposition; the primes indicate that the ket refers to the excited state $B ^3\Pi_1$.

Molecules in the beam are assumed to be in the vibronic ground state, since the beam temperature is much lower than the energy scales associated with the electronic and vibrational excitations. However, even at cryogenic temperatures, there is a Boltzmann distribution over many rotational and nuclear spin states. The dominant term in the energy of rotational/spin levels in the $X ^1\Sigma^+$ state is due to rotation; the mean energy of states with quantum number $J$ is $E_\mathrm{rot}=BJ(J+1)$, where $B\approx 6.67\GHz \approx 0.3\kelvin\,\kB$, where $\kB$ the Boltzmann constant.\footnote{In Ramsey et al.\ \cite{wilkening1984search}, the symbol $B$ denotes the rotational constant in equilibrium position, i.e., there $B\equiv B_\text{e}$. However, the effective $v=0$ rotational constant, $B_0$, is more relevant to \CENTREX. To first order in the Dunham expansion \cite{huber2013molecular}, $B_0=B_\text{e}-\alpha_\text{e}/2$. With $B_\text{e}$ and $\alpha_\text{e}$ from the NIST database \cite{afeefy2011nist}, we find $B_0$. We define the symbol $B\equiv B_0$; its value is shown in Table~\ref{tab:hyperfine_hamiltonian}.}

Hyperfine interactions split sublevels with different $F$ values ($F=J-1,J,J, J+1$, except $F=0,1$ only for $J=0$) in each rotational state. Thus, each rotational level has $4(2J+1)$ magnetic sublevels. %including nuclear spins.
Including rotation, spin-rotation and spin-spin interactions, plus interactions with external electric $(\Evec)$ and magnetic $(\Bvec)$ fields, the system is described by the effective Hamiltonian \cite{wilkening1984search}
\small
$\mathcal{H}_\text{TlF} = \mathcal{H}_\text{rot}+\mathcal{H}_\text{sr}+\mathcal H_\text{ss} + \mathcal H_\text{S}+\mathcal H_\text{Z}$, 
\normalsize
where
\begin{equation} 
    \label{eq:hyperfine_hamiltonian}
    \begin{split}
        \mathcal H_\text{rot} & = B \vec{J}^2 \\
        \mathcal H_\text{sr}&=c_1(\vec I_1\cdot\vec J)+c_2(\vec I_2\cdot\vec J), \\
        \mathcal H_\text{ss}&=c_3 T^2(\vec C)\cdot T^2(\vec I_1, \vec I_2)+c_4(\vec I_1\cdot\vec I_2), \\
        \mathcal H_\text{S}&=-\vec\mu_e\cdot\vec \Evec,\\
        \mathcal H_\text{Z}&=-\frac{\mu_J}{J}(\vec J\cdot\vec \Bvec)-\frac{\mu_1}{I_1}(\vec I_1\cdot\vec \Bvec)-\frac{\mu_2}{I_2}(\vec I_2\cdot\vec \Bvec).
    \end{split}
\end{equation}
Here, the first term in the spin-spin interaction ($\mathcal H_\text{ss}$) contains the scalar product of two rank-2 tensors: one constructed from the modified spherical harmonics $\vec C$, and one from $\vec{I}_1$ and $\vec{I}_2$ \cite{brown2003rotational}. (The matrix elements diagonal in $J$ of this term are given in \cite{wilkening1984search}.)  The hyperfine parameters $c_1,c_2,c_3,c_4$, rotational constant $B$, magnetic moments $\mu_J,\mu_1,\mu_2$, and molecule-frame electric dipole moment $\mu_e$ are all known from previous measurements; their values are given in  Table~\ref{tab:hyperfine_hamiltonian}. A level diagram of low-lying states in the absence of applied fields is shown in Fig.~\ref{fig:level_diagram}.

\begin{table}
    \small
    \setlength\extrarowheight{3pt}
    \rowcolors{2}{white}{gray!25}
	\centering
	\caption{Constants describing rotational, hyperfine, Zeeman, and Stark interactions in the effective TlF ground-state  Hamiltonian (Eq.~\ref{eq:hyperfine_hamiltonian}). All values taken from Ramsey et al.\ \cite{wilkening1984search}, except for $B$ (see Sec. \ref{sec:tlf_theory}).}
	\label{tab:hyperfine_hamiltonian}
	\begin{tabular}{r@{\hspace{5pt}}c@{\hspace{5pt}}l@{\hspace{5pt}}l | r@{\hspace{5pt}}c@{\hspace{5pt}}l@{\hspace{5pt}}l}
    	\toprule
		$B$ & = &           $6.66733$ &     GHz &       $\mu_e$ & = & $4.2282(8)$ &     Debye \\
		$\mu_J$ & = &       $35(15)$ &      Hz/G &      $c_1$ & = & $126.03(12)$ &    kHz \\
		$\mu_1^{205}$ & = & $1.2405(3)$ &   kHz/G &     $c_2$ & = & $17.89(15)$ &     kHz\\
		$\mu_1^{203}$ & = &  $1.2285(3)$ &   kHz/G &     $c_3$ & = & $0.70(3)$ &       kHz \\
		$\mu_2$ & = &      $2.00363(4)$ &  kHz/G  &    $c_4$ & = & $-13.30(72)$ &    kHz \\
		\bottomrule
	\end{tabular}
\end{table}

%%%%%%%%%%%%%%%%%%%%%%%%%%%%%%%%%%%%%%%%%%%%%%55%%%%%%%%%%%%%%%
In \CENTREX, lasers are tuned to $X^1\Sigma^+(\nu=0)\rightarrow B^3\Pi_1\left(\nu=0\right)$ transitions in order to manipulate and read out ground state hyperfine and rotational sublevels. Details of the $B ^3\Pi_1$ state structure are given in \cite{norrgard2017hyperfine,PhysRevA.101.042506}.  Here, only a few main features of the $B$ state substructure are important.  First, the $B$ state hyperfine splittings are very large ($\gtrsim 100\MHz$) compared to the natural linewidth of the transition ($\gamma_B \approx 1.6\MHz$), which is in turn much larger than the ground-state hyperfine splittings ($c_j \lesssim 100\kHz$). This means that hyperfine structure is fully resolved in the excited state, and entirely unresolved in the ground state. Hence, optical transitions in TlF drive a large manifold of ground-state hyperfine levels (with a given value of $J$) to a single hyperfine state with (nominal) quantum numbers $\tilde{J},\tilde{F}_1$ and exact quantum number $F$. Another important feature of the $B$ state is that its matrix of Franck-Condon factors (FCFs) for decay to the $X$ state is extremely diagonal \cite{hunter2012prospects}, such that $\sim\! 99\%$ of the time, the $B(v=0)$ vibronic state decays back to the vibronic ground state $X(v=0)$.  This enables optical pumping and optical cycling with little loss.  However, the mixing of $J$ and $F_1$ by the strong $B$-state hyperfine interaction substantially modifies rotational selection rules in $B-X$ decays, and must be taken into account when describing optical excitation and emission in TlF.

\subsection{TlF in \texorpdfstring{$\Esca$}{E}-fields}
\label{sec:TlF_in_E_fields}

\begin{figure*}
	\centering
	\includegraphics[width=\textwidth]{figs/matplotlib/low_to_mid_field.pdf}
	\caption{Overview of the energy eigenstates for changing $\Esca$-field magnitudes. The low-field regime, where $\Delta E_{\rm S} \ll E_{\rm hf}$, where energy eigenstates retain $J$, $F$, and $F_1$ as approximate quantum numbers is shown in \part{a} for $J=1$ and \part{c} for $J=2$. The mid-field regime, where $E_{\rm hf} \ll \Delta E_{\rm S} \ll E_{\rm rot}$, where both $J$ and $m_J$ are approximate quantum numbers is shown in \part{b} for $J=1$ and \part{d} for $J=2$. States used in \CENTREX\ are shown in bold.}
	\label{fig:low_to_mid_field}
\end{figure*}

\begin{figure}
	\centering
	\includegraphics[width=\textwidth/2]{figs/matplotlib/low_to_high_field.pdf}
	\caption{Evolution of the energy eigenstates of the TlF Hamiltonian (Eq. \ref{eq:hyperfine_hamiltonian}) for $\Esca$ ranging from $0\,$V/cm to $50\,$kV/cm, for $J=0,1,2$. States used in \CENTREX\ are shown in bold. Hyperfine structure is unresolved in this plot.}
	\label{fig:low_to_high_field}
\end{figure}

Throughout most of the \CENTREX\ apparatus, TlF molecules experience a non-zero $\Esca$-field and (nominally) zero $\Bsca$-field.
The character of the energy eigenstates changes significantly depending on the $\Esca$-field magnitude, which varies dramatically between stages of the experiment. Hence, it is useful to describe the energy eigenstates of the TlF electronic ground state in different regimes of $\Esca$-field strength (with $\Bsca = 0$), as defined by the ratio of Stark shifts, $\Delta E_{\rm S} = \langle \mathcal{H}_{\rm S}\rangle \sim \mu_e^2 \Esca^2/B$, to the strength of hyperfine interactions, $E_{\rm hf} = \langle \mathcal{H}_{\rm sr} + \mathcal{H}_{\rm ss}\rangle \sim c_j$, or rotational energies, $E_{\rm rot} = \langle \mathcal{H}_{\rm rot}\rangle \sim B$. In all regimes, the total angular momentum projection $m_F$ along 
a space-fixed quantization axis $\hat{z}$ (always defined such that $\Evec$ is very nearly parallel to $\hat{z}$), is an exact quantum number.

In the low-field regime, where $\Delta E_{\rm S} \ll E_{\rm hf}$, energy eigenstates retain $J$, $F$, and $F_1$ as approximate quantum numbers. 
In the mid-field regime, where $E_{\rm hf} \ll \Delta E_{\rm S} \ll E_{\rm rot}$, both $J$ and $m_J$ are approximate quantum numbers. 
Here, the tensor part of the Stark shifts gives rise to energy splittings between levels with different values of $|m_J|$ that are comparable in size to the scalar shifts, i.e., of order $\Delta E_{\rm S}$.  Thus, when $m_J\neq 0$, $\vec{J}$ is strongly coupled to $\Evec$ (and hence to the molecular axis $\vec{\hat{n}}$) by this Stark interaction. In this case, each nuclear spin is coupled to $\vec{J}$ (and thus also to $\Evec$) by the spin-rotation interactions of $\mathcal{H}_{\rm sr}$. Hence, here $m_{I_1}$ and $m_{I_2}$ are approximate quantum numbers. By contrast, in states where $m_J=0$ (including when $J=0$) in this regime, $\left\langle \mathcal{H}_{\rm sr}\right\rangle$ vanishes to first order, and the nuclear spins do not couple to $\vec{J}$ and $\Evec$. However, the nuclear spins remain coupled to each other via the spin-spin interaction $\mathcal{H}_{\rm ss}$.  So, here the total nuclear spin $\vec \It = \vec I_1 + \vec I_2$ and its projection $m_{\It}$ are approximate quantum numbers in addition to $J$ and $m_J=0$. 
Finally, in the high-field regime where $E_{\rm hf} \ll E_{\rm rot} \lesssim \Delta E_{\rm S}$, $J$ states are strongly mixed, and separations between $m_J$ states are on the order of $E_{\rm rot}$. Here, eigenstates are defined by the same approximate quantum numbers as in the mid-field regime, aside from $J$. We refer to these strongly mixed states with the label $\widetilde{J}$, which corresponds to the value of $J$ that any given state connects to adiabatically, if the $\Esca$-field is reduced.
Table \ref{tab:E_field_regimes} summarizes the different regimes and associated eigenstates.  
\begin{table*}[t]
    \centering
    \def\colseplarge{4ex} % should only be used in this table
	\begin{tabular}{@{}r@{\hspace{\colseplarge}}r@{$\,\Delta E_{\rm S}\,$}l@{\hspace{\colseplarge}}r@{$\,\Esca\,$}l@{\hspace{\colseplarge}}>{\raggedright\arraybackslash}m{4cm}@{}}
    	\toprule
    	Regime & \multicolumn{2}{c}{Definition} & \multicolumn{2}{c}{Field strength} & Approx. eigenstates \\
		\midrule
		 Low
		 & & $\ll E_{\rm hf}$
		 & & $\lesssim 50 \Vcm$
		 & $\ket{J,F_1,F,m_F}$ \\\addlinespace[2ex]
		 %
	     Mid
	     & $E_{\rm hf} \ll$ & $\ll E_{\rm rot}$
	     & $50 \Vcm \ll$ & $\lesssim 5 \kVcm$
	     & $\ket{J,m_J \neq 0}\ket{m_{I_1},m_{I_2}}$ 
	       $\ket{J,m_J=0}\ket{\It,m_{\It}}$ \\\addlinespace[2ex]
	     %
	     High
	     & $E_{\rm hf} \ll E_{\rm rot} \lesssim$ &
	     & $5\kVcm \ll$ &
	     & $\ket{\widetilde{J},m_J\neq 0} \ket{m_{I_1},m_{I_2}}$
	       $\ket{\widetilde{J},m_J=0}\ket{\It,m_{\It}}$ \\
		\bottomrule
	\end{tabular}
	\caption{Regimes of electric field strength and associated eigenstates in TlF.}
	\label{tab:E_field_regimes}
\end{table*}


Figure \ref{fig:low_to_mid_field} shows how the relevant energies and eigenstates evolve from the low-field to the mid-field regime for $J=1$ and $J=2$ states.  Bold curves are states directly relevant to \CENTREX. Figure \ref{fig:low_to_high_field} shows a zoom out of states up to $J=2$ from low to high fields.

The $^{205}$Tl NSM measurement is carried out in $\tilde{J} = 1,\, m_J = \pm1$ states of TlF at large electric field $\mathcal{E} = 30$ kV/cm. This choice of states takes advantage of the structure of TlF in electric fields, in two ways. First, the observable energy shift associated with $S$, $\Delta E$, scales linearly with the degree of polarization $\mathcal{P}$ of the TlF molecule (Eq. \ref{eq:frequency_shift_due_to_NSM}). An electric field more easily polarizes states with low $J$, since $\mathcal{P}$ arises from mixing between states with different parity and thus different $J$; these states are closest together when $J$ is small. Additionally, as discussed in Sec. \ref{Sec:InternalComagnetometry}, certain dangerous systematic errors in the NSM measurement are dramatically suppressed in the presence of a strong spin-rotation interaction (later referred to as an effective intra-molecular magnetic field). This requires $m_J \neq 0$. The $\tilde{J} = 1,\, m_J = \pm1$ states hence provide the best combination of sensitivity and systematic error suppression in TlF.\footnote{$|\mathcal{P}|$ is larger in the $J=0,m_J = 0$ states, given the same $\Esca$-field value. Hence, the NSM gives larger energy shifts there. However, in these states where $m_J = 0$, the effective intra-molecular magnetic field vanishes.}