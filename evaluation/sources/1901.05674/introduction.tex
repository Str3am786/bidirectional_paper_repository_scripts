\section{Introduction}

Malware scanners, or shortly scanners, are software tools that detect malicious files, or in brief, malware.
Two common types of scanners are static and dynamic scanners.
\emph{Static scanners} reason about a file by examining its content without actually running it.
In contrast, \emph{dynamic scanners} examine the behavior of a file at run-time, either by executing it (e.g. Windows executable), or by opening it in the appropriate application (e.g. Adobe Reader for PDF files) or an emulator of such an application.

Perhaps as old as the emergence of scanners~\cite{historyofevasion} are \emph{evasions}, which are used by attackers to circumvent scanners.
Also known as ``logic bombs'' in earlier work~\cite{greenberg1998mobile}, evasions try to fool scanners through a variety of static techniques, such as code obfuscation, and dynamic techniques, such as checking the run-time environment to behave benignly when the environment appears to be a scanner.
The ultimate goal is the same across all evasions: bypass the scanner, while preserving the infection capabilities of the file to compromise the victim's security.

As scanners are constantly improving their abilities to detect malware, evasion techniques are evolving as well. To bypass modern defenses that deploy both static and dynamic analysis, attackers may combine evasions, which can lead to side-effects that have to be assessed.
Vendors of malware scanners must keep fighting new evasion techniques and their combinations, just like new attacks.
It is therefore crucial for vendors to understand which evasions to address first and how evasions and their combinations impact their scanners.

In this work we present a systematic methodology to quantitatively study and compare evasions.
The methodology is applicable to any type of malware and their corresponding scanners.
The main goal of the methodology is to determine how effective evasions are, or to put inversely, how effective scanners are despite the presence of evasions.
In addition, the methodology allows for measuring unintended side-effects of an evasion, e.g., turning an undetected file into a detected one, and for measuring the effect of combining multiple evasions.

We use our methodology to study evasions for PDF files.
Document-based malware attacks are a prevailing problem~\cite{pdf_cve_statistics, officeonrise, exploit_CVE_2018_4990}. These attacks use email or web traffic to deliver malicious documents to victim systems. Then they compromise the system's security by exploiting a vulnerability in the document processing application (e.g., a PDF exploit) or by using legitimate features of the document processing application itself (e.g., embedding an executable file). The attacker's goal is to execute arbitrary machine code or code in a powerful language supported by the client applications (e.g., Visual Basic scripts for Office files). As most organizations need to be able to receive or download files in different document formats, these attacks are particularly difficult to prevent compared to attacks that use executable file formats only. Yet, malicious documents are as powerful as malicious executables because they can lead to arbitrary code execution.

Unfortunately, despite the widespread use of document files and works that study their evasions~\cite{carmony2016extract, Maiorca2013, xu2016automatically, Dang2017, zhang2016adversarial, biggio2013evasion, laskov2014practical},
little is currently known about the effectiveness of document evasion techniques, their combinations, and the dependence of evasion effectiveness on other malware components, such as the exploit used by a malicious document.
Using our methodology, we study evasion techniques for PDFs 
and evaluate their effectiveness in bypassing state-of-the-art PDF scanners.
To this end, we develop a novel framework, called Chameleon, that enriches existing malicious 
PDF documents with one or more evasions.
Chameleon automatically creates PDF exploits with evasions and validates whether the generated exploits work successfully despite the evasion.
Based on \nbSamplesSize{} documents generated by Chameleon, we study 
\nbAnalyzers{} widely used PDF scanners (\nbVirusTotalEngines{} of which are available via VirusTotal) and report a detailed analysis of the results.

The findings of our study include the following:
\begin{itemize}
  \item
    Except for one studied scanner~\cite{2018arXiv181012490J}, none of the \nbAnalyzers{} scanners is immune to evasions. 
    Each of them can be fooled by some evasions into misclassifying a 
    malicious document as benign.
    This result is particularly surprising because the vulnerabilities exploited in our malicious documents have been known for several years.

  \item
    There are huge variations across different scanners.
    While some scanners identify most malicious documents despite evasions, 
    other scanners are fooled by more than 80\% of all evasions.

  \item
    We identify three combinations of evasions that are particularly 
    dangerous as they can mislead all but two scanners.

   \item
     The attack mechanism used in a document influences the effectiveness of 
     evasions.
     For example, an exploit that relies only on JavaScript can often be 
     effectively concealed by obfuscating the JavaScript code.

   \item
     Evasions can be easily combined
     in an automated way to bypass both static and dynamic scanners.

   \item
     Evasions may have side effects and can become counterproductive by 
     making scanners suddenly detect an otherwise undetected malicious document.
\end{itemize}

The results of this study are relevant for several groups of people.
First, our methodology will help researchers to study and rank evasions by their effectiveness in a consistent manner.
Moreover, our study sheds light on the anti-evasion problems that state-of-the-art document scanners suffer from.
Second, vendors of security scanners, e.g., anti-virus or sandbox solution vendors, can learn and use our findings to further harden their solutions against evasion techniques.
Third, users and organizations that need to defend themselves against malware attacks obtain a better understanding of how effective their deployed security solutions are, particularly for PDF-based attacks.
We believe that publicly sharing the knowledge about evasions and their effectiveness is the best step toward effectively mitigating potential attacks.
In addition, we are closely collaborating with vendors of scanners to make them aware of their current weaknesses.

In summary, we make the following contributions:
\begin{itemize}
    \item \textbf{Evasion assessment methodology:} We propose a methodology to quantitatively study the effectiveness of evasions on a large scale. This methodology can be used for all types of malware and their corresponding scanners.
    \item \textbf{Chameleon framework:} We implement our methodology for PDF exploits in Chameleon, a novel framework that automatically transforms malicious PDF documents into evasive documents.
    \item \textbf{A benchmark test suite:} We make a corpus of \nbSamplesSize{} evasive PDF files generated by Chameleon publicly available, to foster future work on evaluating and improving PDF security scanners.
    \item \textbf{An in-depth study of evasions for document-based malware:} We conduct a large-scale study of the effectiveness of 19 PDF evasions on a set of~\nbAnalyzers{} scanners. Our findings show widely used scanners to be easily fooled by evasions, motivating work on better-coping with evasions.
    
\end{itemize}
