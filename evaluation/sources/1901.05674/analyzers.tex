\section{PDF Scanners}
\label{ss: analyzers}

\begin{table}[tb]
	\caption{PDF scanners used for the study.}
	\label{tab:analyzers}
	\footnotesize
	\setlength{\tabcolsep}{6pt}
    \renewcommand*{\arraystretch}{.6}
	\begin{tabular}{@{}lcccc@{}}
		\toprule
		Scanner & Static & Dynamic & Academic & Commercial \\
		\midrule
		ALYac & \cmark & & & \cmark \\
		AVG & \cmark & & & \cmark \\
		AVware & \cmark & & & \cmark \\
		Ad-Aware & \cmark & & & \cmark \\
		AhnLab-V3 & \cmark & & & \cmark \\
		Antiy-AVL & \cmark & & & \cmark \\
		Arcabit & \cmark & & & \cmark \\
		Avast & \cmark & & & \cmark \\
		Avira & \cmark & & & \cmark \\
		Baidu & \cmark & & & \cmark \\
		BitDefender & \cmark & & & \cmark \\
		CAT-QuickHeal & \cmark & & & \cmark \\
		Cuckoo & & \cmark & \cmark & \cmark \\
		Cyren & \cmark & & & \cmark \\
		DS1 & & \cmark & & \cmark \\% VMRay Analyzer
		DS2 & & \cmark & & \cmark \\% Lastline Analyst
		Emsisoft & \cmark & & & \cmark \\
		F-Prot & \cmark & & & \cmark \\
		F-Secure & \cmark & & & \cmark \\
		Fortinet & \cmark & & & \cmark \\
		GData & \cmark & & & \cmark \\
		Ikarus & \cmark & & & \cmark \\
		Jiangmin & \cmark & & & \cmark \\
		Kaspersky & \cmark & & & \cmark \\
		MAX & \cmark & & & \cmark \\
		McAfee-GW-Edition & \cmark & & & \cmark \\
		MicroWorld-eScan & \cmark & & & \cmark \\
		Microsoft & \cmark & & & \cmark \\
		NANO-Antivirus & \cmark & & & \cmark \\
		PDF-Scrutinizer~\cite{schmitt2012pdf} & & \cmark & \cmark & \\
		Qihoo-360 & \cmark & & & \cmark \\
		Rising & \cmark & & & \cmark \\
		SAFE-PDF~\cite{2018arXiv181012490J} & \cmark & & \cmark & \\
		Slayer~\cite{maiorca2012pattern} & \cmark & & \cmark & \\
		Sophos & \cmark & & & \cmark \\
		SploitGuard & & \cmark & & \cmark \\
		Symantec & \cmark & & & \cmark \\
		Tencent & \cmark & & & \cmark \\
		TrendMicro & \cmark & & & \cmark \\
		VIPRE & \cmark & & & \cmark \\
		ZoneAlarm & \cmark & & & \cmark \\
		\bottomrule
	\end{tabular}
\end{table}

We study \nbStaticAnalyzers{} static and \nbDynamicAnalyzers{} dynamic scanners, including both academic and widely used commercial tools, as listed in Table~\ref{tab:analyzers}.
To categorize a scanner as static or dynamic we rely on information provided 
by the vendors or developers.
Based on this information, we consider a scanner as static if it reasons 
about a PDF document without opening the document in a PDF viewer.
In contrast, dynamic scanners open a PDF document in a PDF viewer or an emulator and then 
analyze its runtime behavior, e.g., by tracking how the PDF viewer interacts 
with the operating system.
Our study includes more static than dynamic scanners because static scanners are more common in practice.

To run the commercial static scanners on our PDF documents, we use the application programming interface (API) of VirusTotal that runs close to 60 static scanners at once on a given document. We ignore those scanners that do not detect any of the exploits we use (perhaps because they are not designed to detect PDF malware), which leaves \nbVirusTotalEngines{} commercial static scanners.
To run the commercial dynamic scanners, we use the individual APIs provided by the respective vendors of these scanners.
The vendors of two commercial dynamic scanners requested to participate anonymously, so we refer to them as DS1 and DS2.
Appendix~\ref{sec:scanner setup} explains the detailed setup of the non-commercial scanners.
In addition to the scanners listed in Table~\ref{tab:analyzers}, we considered several others,
including PDFRate~\cite{smutz2012malicious} and PJScan~\cite{laskov2011static}, but were unable to use them for our study because they either were unavailable or had some issues in our local setup (see Appendix~\ref{sec:other analyzers} for details).
