\section{Related Work}
\label{ss: related work}

\textbf{Studying Evasions}.
Previous work has studied to what extent evasions help in circumventing scanners for malware types other than PDF.
These studies consider Android~\cite{zheng2012adam,faruki2014evaluation,rastogi2013droidchameleon,fedler2013effectiveness,petsas2014rage,canfora2015obfuscation}, Windows executables~\cite{christodorescu2004testing,moser2007limits}, and JavaScript~\cite{Xu2012a}.
Their measurements focus on reporting for each evasion whether the scanners could still detect a malicious file.
Our work differs both in the methodology and in its application.
Methodologically, our study goes beyond a single binary measure and answers additional questions, such as the added effectiveness of combined evasions and the dependence of evasions on other malware components, e.g., the payload.
Regarding the application, this work is the first to provide an in-depth study of the effectiveness of evasions for PDF-based malware.

\textbf{Analysis of PDF Malware}.
Non-executable document formats, such as PDF, have become one of the main vectors for delivering malware to victims~\cite{hardy2014targeted}.
To detect PDF documents that contain malicious JavaScript code, combinations of static and dynamic analysis of the embedded JavaScript code search for suspicious operations that rarely occur in benign documents~\cite{liu2014detecting,schmitt2012pdf,tzermias2011combining,lu2013obfuscation}.
Another line of work statically extracts features of documents, e.g., based on a document's metadata and structure~\cite{smutz2012malicious,maiorca2012pattern,vsrndic2013detection} or based on embedded JavaScript~\cite{laskov2011static,corona2014lux0r}, and then trains a machine learning model to identify malicious documents.
Nissim et al.\ survey these and other techniques~\cite{nissim2015detection}.
Beyond PDFs, the problem of malicious documents extends to other document formats~\cite{nissim2017aldocx}. 
A recurring problem for all document scanners is how to evaluate them, particularly, in the presence of evasions.
Chameleon provides a generic mechanism to create malicious documents beyond well-studied sets of documents, such as the Contagio malware dump\footnote{http://contagiodump.blogspot.com/}.


\textbf{JavaScript Analysis}.
Malicious PDF documents contain malicious JavaScript code.
Identifying such code has been actively researched for client-side web applications, by analyzing potential malware samples in a sandbox~\cite{willems2007toward}, through learning-based anomaly detection~\cite{cova2010detection}, by classifying abstract syntax trees~\cite{curtsinger2011zozzle}, or by searching for malicious sites with specially crafted search engine queries~\cite{invernizzi2012evilseed}.
A recent survey discusses various other security-related analyses of JavaScript code~\cite{jsSurvey2017}.
All these approaches focus on JavaScript code in web applications, which differs from JavaScript code embedded in PDF documents.


\textbf{From Logic Bombs to Modern Evasions}.
Attempts to fool detectors of malicious software are probably as old as malicious software itself.
Earlier approaches use logic bombs, where an attack is initiated upon occurrence of an external event~\cite{greenberg1998mobile, avivzienis2004dependability}.
To counter malware scanners that execute a potentially malicious file in a virtualized environment, anti-virtualization techniques have been proposed~\cite{raffetseder2007detecting}.
Chen et al.~\cite{chen2008towards} provide a taxonomy of malware evasion techniques with a focus on anti-virtualization and anti-debugging behavior.
Some of the evasions studied in this paper can be used to detect a virtualized environment, while others, e.g., the UI evasions, can also detect scanners running on a physical machine.
Transparent scanners try to mimic a real execution platform, i.e., without any traces of virtualization or specific fingerprints~\cite{kirat2014barecloud}, but even those can be evaded via evasion techniques that check the system's past use, e.g., via the Windows registry size or the total number of browser cookies~\cite{miramirkhani2017spotless}.
Several survey articles discuss other evasion techniques~\cite{corona2013adversarial,bulazel2017survey}, including code transformation techniques similar to our obfuscation evasions~\cite{you2010malware}.

\textbf{Evasions in Document-based Attacks}.
We envision future work to extend our framework with additional evasions, e.g., PDF parser confusion attacks~\cite{carmony2016extract}.
Other recent evasion techniques fool machine learning-based scanners, for instance by slightly modifying a benign document~\cite{Maiorca2013} or by stochastically modifying a malicious document~\cite{xu2016automatically,Dang2017}.
Knowing that attackers might conceal malicious behavior through evasions, Zhang et al.~\cite{zhang2016adversarial} propose an approach to improve machine learning-based scanners through adversary-aware feature selection.
Finally, there are two previous papers that systematically study the effectiveness of evasions.
Biggio et al.~\cite{biggio2013evasion} study to what extent learning-based malware classifiers can be fooled by evasions.
In contrast, we do not make any assumptions about the studied scanners and (probably) include both learning-based and not learning-based scanners.
Laskov et al.~\cite{laskov2014practical} focus on a single scanner (PDFRate), whereas our study involves \nbAnalyzers{} scanners.

