\documentclass[sigconf,nonacm]{acmart}
\pdfoutput=1
\fancyhf{} % Remove fancy page headers 

\setcopyright{none} % No copyright notice required for submissions

\settopmatter{printacmref=false, printccs=false, printfolios=false} % We want page numbers on submissions

\usepackage{epsfig}
\usepackage{endnotes}
\usepackage{amssymb}
\usepackage{amsmath}
\newtheorem{definition}{Definition}
\usepackage{import}
\usepackage{color}
\usepackage[linesnumbered]{algorithm2e}
\usepackage{pifont}
\newcommand{\cmark}{\ding{51}}
\newcommand{\xmark}{\ding{55}}
\usepackage{listings}
\usepackage{booktabs}
\usepackage{multirow}
\usepackage[switch]{lineno}
\usepackage{makecell}
\usepackage{subcaption}
\usepackage{tikz-qtree}
\usepackage{pifont}
\usepackage{threeparttable}

%\usepackage[hyphens,spaces,obeyspaces]{url}

\usepackage{caption}
\usetikzlibrary{trees}
\usepackage[normalem]{ulem}

\tikzset{edge from parent/.style=
{draw, edge from parent path={(\tikzparentnode.south)
-- +(0,-4pt)
-| (\tikzchildnode)}},
blank/.style={draw=none},every node/.append style={align=left},every tree node/.style={anchor=north},level distance=0.30in,sibling distance=.06in}

\newcommand{\todo}[1]{\textcolor{red}{#1}}
\newcommand{\code}[1]{{\ttfamily #1}}
\newcommand{\scode}[1]{{\footnotesize \ttfamily #1}}

%%%%%%%%%%%
% Let's put all numbers that we mention multiple times into macros:
\newcommand{\nbVirusTotalEngines}{34}
\newcommand{\nbStaticAnalyzers}{36}
\newcommand{\nbDynamicAnalyzers}{5}
\newcommand{\nbAnalyzers}{41} % \nbStaticAnalyzers + \nbDynamicAnalyzers
\newcommand{\nbSamplesSize}{1,395}
\newcommand{\nbBenignsSize}{81}
%%%%%%%%%%%

%%%%%% Space saving tricks
\usepackage{enumitem}
\setlist[enumerate]{noitemsep,topsep=2pt,leftmargin=*}
\setlist[itemize]{noitemsep,topsep=2pt,leftmargin=*}
%\addtolength{\parskip}{0mm}
%\addtolength{\floatsep}{-3mm}
%\addtolength{\textfloatsep}{-3mm}
%\addtolength{\dblfloatsep}{-3mm}
%\addtolength{\dbltextfloatsep}{-3mm}
\addtolength{\abovecaptionskip}{-3mm}
%\addtolength{\belowcaptionskip}{-3mm}


\begin{document}

\title{Easy to Fool? Testing the Anti-evasion Capabilities of\\ PDF Malware Scanners}
% 

\author{Saeed Ehteshamifar}
%\authornote{any note?}
%\orcid{i dunno where in the paper it's displayed}
\affiliation{%
  \institution{TU Darmstadt, Germany}
%  \city{Dublin}
%  \state{Ohio}
}
\email{salpha.2004@gmail.com}

\author{Antonio Barresi}
\affiliation{%
  \institution{xorlab, Switzerland}
}
\email{antonio@barresi.net}

\author{Thomas R. Gross}
\affiliation{%
  \institution{ETH Zurich, Switzerland}
}
\email{trg@inf.ethz.ch}

\author{Michael Pradel}
\affiliation{%
  \institution{TU Darmstadt, Germany}
}
\email{michael@binaervarianz.de}


\maketitle

\subsection*{Abstract}
\import{./}{abstract.tex}



\import{./}{introduction.tex}
\import{./}{methodology.tex}
\import{./}{implementation.tex}
\import{./}{analyzers.tex}
\import{./}{results.tex}
\import{./}{discussion.tex}
\import{./}{related.tex}
\import{./}{conclusion.tex}

\section*{Acknowledgment}
This work was supported by the German Federal Ministry of Education and
Research and by the Hessian Ministry of Science and the Arts within
CRISP, by the German Research Foundation within the ConcSys and Perf4JS
projects, and by the Hessian LOEWE initiative within the
Software-Factory 4.0 project.
The authors would also like to thank InfoGuard and the malware scanner vendors that anonymously participated in this study. Moreover, we thank Mathias Payer and other anonymous reviewers for their reviews and feedback.

\bibliography{ms}

\newpage
\appendix
\import{./}{academic_analyzers.tex}

\end{document}




