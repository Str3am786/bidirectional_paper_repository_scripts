Malware scanners try to protect users from opening malicious documents by statically or dynamically analyzing documents.
However, malware developers may apply evasions that conceal the maliciousness of a document.
Given the variety of existing evasions, systematically assessing the impact of evasions on malware scanners remains an open challenge.
This paper presents a novel methodology for testing the capability of malware scanners to cope with evasions.
We apply the methodology to malicious Portable Document Format (PDF) documents and present an in-depth study of how current PDF evasions affect \nbAnalyzers{} state-of-the-art malware scanners.
The study is based on a framework for creating malicious PDF documents that use one or more evasions.
Based on such documents, we measure how effective different evasions are at concealing the maliciousness of a document.
We find that many static and dynamic scanners can be easily fooled by relatively simple evasions and that the effectiveness of different evasions varies drastically.
Our work not only is a call to arms for improving current malware scanners, but by providing a large-scale corpus of malicious PDF documents with evasions, we directly support the development of improved tools to detect document-based malware.
Moreover, our methodology paves the way for a quantitative evaluation of evasions in other kinds of malware.
